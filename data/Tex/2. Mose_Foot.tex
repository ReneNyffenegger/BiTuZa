
\bigskip				%%gro�er Abstand

\newpage
\hphantom{x}
\bigskip\bigskip\bigskip\bigskip\bigskip\bigskip
\begin{center}{ \huge {\bf Anmerkungen}}\end{center}

\medskip
Die folgenden Anmerkungen beziehen sich auf W"orter im Buch 2. Mose.\\
Das jeweilige Wort befindet sich unter \textcolor{red}{x.y.z.}\\
Dabei ist \textcolor{red}{x} die Kapitelnummer, \textcolor{red}{y} die Versnummer und \textcolor{red}{z} die Wortnummer.\\

\medskip
\textcolor{red}{3.14.5-7}: 'Ich bin der ich bin' hat den Totalwert 543. Mose hat den Totalwert 345. Die Summe aus beiden ergibt 888!\\
 \\
\textcolor{red}{7.19.26}: Blut (TW 44) ist das charakteristische Wort der 1. Plage (k"urzeste hebr"aische Form). \\
\textcolor{red}{8.6.9}: Fr"osche (TW 449) ist das charakteristische Wort der 2. Plage. \\
\textcolor{red}{8.17.13}: Stechm"ucken (TW 115) ist das charakteristische Wort der 3. Plage. \\
\textcolor{red}{8.22.14}: Ungeziefer (TW 272) ist das charakteristische Wort der 4. Plage. \\
\textcolor{red}{9.3.13}: Seuche (TW 206) ist das charakteristische Wort der 5. Plage. \\
\textcolor{red}{9.10.13}: Geschw"ur (TW 368) ist das charakteristische Wort der 6. Plage. \\
\textcolor{red}{9.18.5}: Hagel (TW 206) ist das charakteristische Wort der 7. Plage. \\
\textcolor{red}{10.4.11}: Heuschrecken (TW 208) ist das charakteristische Wort der 8. Plage. \\
\textcolor{red}{10.21.10}: Finsternis (TW 328) ist das charakteristische Wort der 9. Plage. \\
\textcolor{red}{12.12.6}: Schlagen (TW 451) ist das charakteristische Wort der 10. Plage. Die Totalwerte der 10 charakteristischen W"orter der 10 Plagen ergeben als Summe 2647 (das Geburtsjahr Moses). \\
 \\
\textcolor{red}{12.37.4}: Raemses (200\_70\_40\_60\_60) hat den Totalwert 430. Diese Zahl gibt die Jahre der Wohnzeit der Kinder Israel in "Agypten an. \\
\textcolor{red}{12.37.5}: Sukkoth (60\_20\_400) hat den Totalwert 480. Diese Zahl gibt die Jahre vom Exodus bis zum Tempelbaubeginn durch Salomo an. \\

