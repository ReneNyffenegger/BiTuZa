\documentclass[a4paper,10pt,landscape]{article}
\usepackage[landscape]{geometry}	%%Querformat
\usepackage{fancyhdr}			%%Erweiterte Kopfzeilen
\usepackage{color}			%%Farben
\usepackage[german]{babel}		%%Deutsche Namen
\usepackage{cjhebrew} %%Hebr�isches Packet
\usepackage{upgreek}                    %%nicht kursive griechische Buchstaben
\usepackage{amsbsy}                     %%fette griechische Buchstaben
\setlength{\parindent}{0pt}		%%Damit bei neuen Abs"atzen kein Einzug
\frenchspacing

\pagestyle{fancy}			%%Kopf-/Fu�zeilenstyle

\renewcommand{\headrulewidth}{0.5pt}	%%Strich in Kopfzeile
\renewcommand{\footrulewidth}{0pt}	%%kein Strich in Fu�zeile
\renewcommand{\sectionmark}[1]{\markright{{#1}}}
\lhead{\rightmark}
\chead{}
\rhead{Bibel in Text und Zahl}
\lfoot{pgz}
\cfoot{\thepage}			%%Seitenzahl
\rfoot{}

\renewcommand{\section}[3]{\begin{center}{ \huge {\bf \textsl{{#1}}\\ \textcolor{red}{\textsl{{#2}}}}}\end{center}
\sectionmark{{#3}}}

\renewcommand{\baselinestretch}{0.9}	%%Zeilenabstand

\begin{document}			%%Dokumentbeginn

\section{\bigskip\bigskip\bigskip\bigskip\bigskip\bigskip
\\Die Spr"uche\\ (Sprichw"orter)}
{}
{Spr"uche}	%%�berschrift (�bergibt {schwarzen
										%%Text}{roten Text}{Text in Kopfzeile}


\bigskip				%%gro�er Abstand

\newpage
\hphantom{x}
\bigskip\bigskip\bigskip\bigskip\bigskip\bigskip
\begin{center}{ \huge {\bf Erl"auterungen}}\end{center}

\medskip
In diesem Buch werden folgende Abk"urzungen verwendet:\\
WV = Nummer des Wortes im Vers\\
WK = Nummer des Wortes im Kapitel\\
WB = Nummer des Wortes im Buch\\
ABV = Nummer des Anfangsbuchstabens des Wortes im Vers\\
ABK = Nummer des Anfangsbuchstabens des Wortes im Kapitel\\
ABB = Nummer des Anfangsbuchstabens des Wortes im Buch\\
AnzB = Anzahl der Buchstaben des Wortes\\
TW = Totalwert des Wortes\\

\medskip
Am Ende eines Verses finden sich sieben Zahlen,\\
die folgende Bedeutung haben (von links nach rechts):\\
1. Nummer des Verses im Buch\\
2. Gesamtzahl der Buchstaben im Vers\\
3. Gesamtzahl der Buchstaben (bis einschlie"slich dieses Verses) im Kapitel\\
4. Gesamtzahl der Buchstaben (bis einschlie"slich dieses Verses) im Buch\\
5. Summe der Totalwerte des Verses\\
6. Summe der Totalwerte (bis einschlie"slich dieses Verses) im Kapitel\\
7. Summe der Totalwerte (bis einschlie"slich dieses Verses) im Buch\\



\newpage 
{\bf -- 1.1}\\
\medskip \\
\begin{tabular}{rrrrrrrrp{120mm}}
WV&WK&WB&ABK&ABB&ABV&AnzB&TW&Zahlencode \textcolor{red}{$\boldsymbol{Grundtext}$} Umschrift $|$"Ubersetzung(en)\\
1.&1.&1.&1.&1.&1.&4&380&40\_300\_30\_10 \textcolor{red}{\textcjheb{yl+sm}} MSLJ $|$Spr"uche\\
2.&2.&2.&5.&5.&5.&4&375&300\_30\_40\_5 \textcolor{red}{\textcjheb{hml+s}} SLMH $|$Salomos/von Salomo (Schelomo)\\
3.&3.&3.&9.&9.&9.&2&52&2\_50 \textcolor{red}{\textcjheb{nb}} BN $|$(des) Sohn(es)\\
4.&4.&4.&11.&11.&11.&3&14&4\_6\_4 \textcolor{red}{\textcjheb{dwd}} DWD $|$David(s)\\
5.&5.&5.&14.&14.&14.&3&90&40\_30\_20 \textcolor{red}{\textcjheb{klm}} MLK $|$des K"onigs\\
6.&6.&6.&17.&17.&17.&5&541&10\_300\_200\_1\_30 \textcolor{red}{\textcjheb{l'r+sy}} JSRAL $|$(von) Israel\\
\end{tabular}\medskip \\
Ende des Verses 1.1\\
Verse: 1, Buchstaben: 21, 21, 21, Totalwerte: 1452, 1452, 1452\\
\\
Spr"uche Salomos, des Sohnes Davids, des K"onigs von Israel:\\
\newpage 
{\bf -- 1.2}\\
\medskip \\
\begin{tabular}{rrrrrrrrp{120mm}}
WV&WK&WB&ABK&ABB&ABV&AnzB&TW&Zahlencode \textcolor{red}{$\boldsymbol{Grundtext}$} Umschrift $|$"Ubersetzung(en)\\
1.&7.&7.&22.&22.&1.&4&504&30\_4\_70\_400 \textcolor{red}{\textcjheb{t`dl}} LDaT $|$(um) zu (er)kennen\\
2.&8.&8.&26.&26.&5.&4&73&8\_20\_40\_5 \textcolor{red}{\textcjheb{hmk.h}} CKMH $|$Weisheit\\
3.&9.&9.&30.&30.&9.&5&312&6\_40\_6\_60\_200 \textcolor{red}{\textcjheb{rswmw}} WMWsR $|$und Unterweisung/und Zucht\\
4.&10.&10.&35.&35.&14.&5&97&30\_5\_2\_10\_50 \textcolor{red}{\textcjheb{nybhl}} LHBJN $|$(um) zu verstehen\\
5.&11.&11.&40.&40.&19.&4&251&1\_40\_200\_10 \textcolor{red}{\textcjheb{yrm'}} AMRJ $|$Worte\\
6.&12.&12.&44.&44.&23.&4&67&2\_10\_50\_5 \textcolor{red}{\textcjheb{hnyb}} BJNH $|$des Verstandes/(von) Einsicht\\
\end{tabular}\medskip \\
Ende des Verses 1.2\\
Verse: 2, Buchstaben: 26, 47, 47, Totalwerte: 1304, 2756, 2756\\
\\
um Weisheit und Unterweisung zu kennen, um Worte des Verstandes zu verstehen,\\
\newpage 
{\bf -- 1.3}\\
\medskip \\
\begin{tabular}{rrrrrrrrp{120mm}}
WV&WK&WB&ABK&ABB&ABV&AnzB&TW&Zahlencode \textcolor{red}{$\boldsymbol{Grundtext}$} Umschrift $|$"Ubersetzung(en)\\
1.&13.&13.&48.&48.&1.&4&538&30\_100\_8\_400 \textcolor{red}{\textcjheb{t.hql}} LQCT $|$um zu empfangen/um zu nehmen\\
2.&14.&14.&52.&52.&5.&4&306&40\_6\_60\_200 \textcolor{red}{\textcjheb{rswm}} MWsR $|$Unterweisung/Zucht\\
3.&15.&15.&56.&56.&9.&4&355&5\_300\_20\_30 \textcolor{red}{\textcjheb{lk+sh}} HSKL $|$einsichtsvolle/(ein) Klugsein\\
4.&16.&16.&60.&60.&13.&3&194&90\_4\_100 \textcolor{red}{\textcjheb{qd.s}} "sDQ $|$Gerechtigkeit\\
5.&17.&17.&63.&63.&16.&5&435&6\_40\_300\_80\_9 \textcolor{red}{\textcjheb{.tp+smw}} WMSPt $|$und Recht\\
6.&18.&18.&68.&68.&21.&7&606&6\_40\_10\_300\_200\_10\_40 \textcolor{red}{\textcjheb{myr+symw}} WMJSRJM $|$und Geradheit/und Redlichkeit\\
\end{tabular}\medskip \\
Ende des Verses 1.3\\
Verse: 3, Buchstaben: 27, 74, 74, Totalwerte: 2434, 5190, 5190\\
\\
um zu empfangen einsichtsvolle Unterweisung, Gerechtigkeit und Recht und Geradheit;\\
\newpage 
{\bf -- 1.4}\\
\medskip \\
\begin{tabular}{rrrrrrrrp{120mm}}
WV&WK&WB&ABK&ABB&ABV&AnzB&TW&Zahlencode \textcolor{red}{$\boldsymbol{Grundtext}$} Umschrift $|$"Ubersetzung(en)\\
1.&19.&19.&75.&75.&1.&3&830&30\_400\_400 \textcolor{red}{\textcjheb{ttl}} LTT $|$(um) zu geben\\
2.&20.&20.&78.&78.&4.&6&561&30\_80\_400\_1\_10\_40 \textcolor{red}{\textcjheb{my'tpl}} LPTAJM $|$Einf"altigen/den Unerfahrenen\\
3.&21.&21.&84.&84.&10.&4&315&70\_200\_40\_5 \textcolor{red}{\textcjheb{hmr`}} aRMH $|$Klugheit\\
4.&22.&22.&88.&88.&14.&4&350&30\_50\_70\_200 \textcolor{red}{\textcjheb{r`nl}} LNaR $|$dem J"ungling/der Jugend\\
5.&23.&23.&92.&92.&18.&3&474&4\_70\_400 \textcolor{red}{\textcjheb{t`d}} DaT $|$(Er)Kenntnis\\
6.&24.&24.&95.&95.&21.&5&98&6\_40\_7\_40\_5 \textcolor{red}{\textcjheb{hmzmw}} WMZMH $|$und Besonnenheit\\
\end{tabular}\medskip \\
Ende des Verses 1.4\\
Verse: 4, Buchstaben: 25, 99, 99, Totalwerte: 2628, 7818, 7818\\
\\
um Einf"altigen Klugheit zu geben, dem J"ungling Erkenntnis und Besonnenheit.\\
\newpage 
{\bf -- 1.5}\\
\medskip \\
\begin{tabular}{rrrrrrrrp{120mm}}
WV&WK&WB&ABK&ABB&ABV&AnzB&TW&Zahlencode \textcolor{red}{$\boldsymbol{Grundtext}$} Umschrift $|$"Ubersetzung(en)\\
1.&25.&25.&100.&100.&1.&4&420&10\_300\_40\_70 \textcolor{red}{\textcjheb{`m+sy}} JSMa $|$(es) wird h"oren/er (=es) h"ort\\
2.&26.&26.&104.&104.&5.&3&68&8\_20\_40 \textcolor{red}{\textcjheb{mk.h}} CKM $|$(der) Weise\\
3.&27.&27.&107.&107.&8.&5&162&6\_10\_6\_60\_80 \textcolor{red}{\textcjheb{pswyw}} WJWsP $|$und zunehmen/und er macht hinzuf"ugen\\
4.&28.&28.&112.&112.&13.&3&138&30\_100\_8 \textcolor{red}{\textcjheb{.hql}} LQC $|$an Kenntnis/Belehrung\\
5.&29.&29.&115.&115.&16.&5&114&6\_50\_2\_6\_50 \textcolor{red}{\textcjheb{nwbnw}} WNBWN $|$und der Verst"andige/und der Einsichtige\\
6.&30.&30.&120.&120.&21.&6&846&400\_8\_2\_30\_6\_400 \textcolor{red}{\textcjheb{twlb.ht}} TCBLWT $|$weisen Rat/"Uberlegungen\\
7.&31.&31.&126.&126.&27.&4&165&10\_100\_50\_5 \textcolor{red}{\textcjheb{hnqy}} JQNH $|$wird sich erwerben/er erwirbt\\
\end{tabular}\medskip \\
Ende des Verses 1.5\\
Verse: 5, Buchstaben: 30, 129, 129, Totalwerte: 1913, 9731, 9731\\
\\
Der Weise wird h"oren und an Kenntnis zunehmen, und der Verst"andige wird sich weisen Rat erwerben;\\
\newpage 
{\bf -- 1.6}\\
\medskip \\
\begin{tabular}{rrrrrrrrp{120mm}}
WV&WK&WB&ABK&ABB&ABV&AnzB&TW&Zahlencode \textcolor{red}{$\boldsymbol{Grundtext}$} Umschrift $|$"Ubersetzung(en)\\
1.&32.&32.&130.&130.&1.&5&97&30\_5\_2\_10\_50 \textcolor{red}{\textcjheb{nybhl}} LHBJN $|$(um) zu verstehen (machen)\\
2.&33.&33.&135.&135.&6.&3&370&40\_300\_30 \textcolor{red}{\textcjheb{l+sm}} MSL $|$(einen) Spruch\\
3.&34.&34.&138.&138.&9.&6&181&6\_40\_30\_10\_90\_5 \textcolor{red}{\textcjheb{h.sylmw}} WMLJ"sH $|$und verschlungene Rede/und Sinnspruch\\
4.&35.&35.&144.&144.&15.&4&216&4\_2\_200\_10 \textcolor{red}{\textcjheb{yrbd}} DBRJ $|$Worte\\
5.&36.&36.&148.&148.&19.&5&118&8\_20\_40\_10\_40 \textcolor{red}{\textcjheb{mymk.h}} CKMJM $|$der Weisen/(von) Weisen\\
6.&37.&37.&153.&153.&24.&6&468&6\_8\_10\_4\_400\_40 \textcolor{red}{\textcjheb{mtdy.hw}} WCJDTM $|$und ihre R"atsel(reden)\\
\end{tabular}\medskip \\
Ende des Verses 1.6\\
Verse: 6, Buchstaben: 29, 158, 158, Totalwerte: 1450, 11181, 11181\\
\\
um einen Spruch zu verstehen und verschlungene Rede, Worte der Weisen und ihre R"atsel. -\\
\newpage 
{\bf -- 1.7}\\
\medskip \\
\begin{tabular}{rrrrrrrrp{120mm}}
WV&WK&WB&ABK&ABB&ABV&AnzB&TW&Zahlencode \textcolor{red}{$\boldsymbol{Grundtext}$} Umschrift $|$"Ubersetzung(en)\\
1.&38.&38.&159.&159.&1.&4&611&10\_200\_1\_400 \textcolor{red}{\textcjheb{t'ry}} JRAT $|$(die) Furcht\\
2.&39.&39.&163.&163.&5.&4&26&10\_5\_6\_5 \textcolor{red}{\textcjheb{hwhy}} JHWH $|$Jahwes/(vor) Jahwe\\
3.&40.&40.&167.&167.&9.&5&911&200\_1\_300\_10\_400 \textcolor{red}{\textcjheb{ty+s'r}} RASJT $|$(ist) Anfang\\
4.&41.&41.&172.&172.&14.&3&474&4\_70\_400 \textcolor{red}{\textcjheb{t`d}} DaT $|$(der) (Er)Kenntnis\\
5.&42.&42.&175.&175.&17.&4&73&8\_20\_40\_5 \textcolor{red}{\textcjheb{hmk.h}} CKMH $|$Weisheit\\
6.&43.&43.&179.&179.&21.&5&312&6\_40\_6\_60\_200 \textcolor{red}{\textcjheb{rswmw}} WMWsR $|$und Unterweisung/und Zucht\\
7.&44.&44.&184.&184.&26.&6&97&1\_6\_10\_30\_10\_40 \textcolor{red}{\textcjheb{mylyw'}} AWJLJM $|$die Narren/(die) Toren\\
8.&45.&45.&190.&190.&32.&3&15&2\_7\_6 \textcolor{red}{\textcjheb{wzb}} BZW $|$(sie) verachte(te)n\\
\end{tabular}\medskip \\
Ende des Verses 1.7\\
Verse: 7, Buchstaben: 34, 192, 192, Totalwerte: 2519, 13700, 13700\\
\\
Die Furcht Jahwes ist der Erkenntnis Anfang; die Narren verachten Weisheit und Unterweisung.\\
\newpage 
{\bf -- 1.8}\\
\medskip \\
\begin{tabular}{rrrrrrrrp{120mm}}
WV&WK&WB&ABK&ABB&ABV&AnzB&TW&Zahlencode \textcolor{red}{$\boldsymbol{Grundtext}$} Umschrift $|$"Ubersetzung(en)\\
1.&46.&46.&193.&193.&1.&3&410&300\_40\_70 \textcolor{red}{\textcjheb{`m+s}} SMa $|$h"ore\\
2.&47.&47.&196.&196.&4.&3&62&2\_50\_10 \textcolor{red}{\textcjheb{ynb}} BNJ $|$mein Sohn\\
3.&48.&48.&199.&199.&7.&4&306&40\_6\_60\_200 \textcolor{red}{\textcjheb{rswm}} MWsR $|$die Unterweisung/die Zurechtweisung\\
4.&49.&49.&203.&203.&11.&4&33&1\_2\_10\_20 \textcolor{red}{\textcjheb{kyb'}} ABJK $|$deines Vaters\\
5.&50.&50.&207.&207.&15.&3&37&6\_1\_30 \textcolor{red}{\textcjheb{l'w}} WAL $|$und nicht\\
6.&51.&51.&210.&210.&18.&3&709&400\_9\_300 \textcolor{red}{\textcjheb{+s.tt}} TtS $|$verlass/du sollst verwerfen\\
7.&52.&52.&213.&213.&21.&4&1006&400\_6\_200\_400 \textcolor{red}{\textcjheb{trwt}} TWRT $|$die Belehrung/die Weisung\\
8.&53.&53.&217.&217.&25.&3&61&1\_40\_20 \textcolor{red}{\textcjheb{km'}} AMK $|$deiner Mutter\\
\end{tabular}\medskip \\
Ende des Verses 1.8\\
Verse: 8, Buchstaben: 27, 219, 219, Totalwerte: 2624, 16324, 16324\\
\\
H"ore, mein Sohn, die Unterweisung deines Vaters, und verla"s nicht die Belehrung deiner Mutter!\\
\newpage 
{\bf -- 1.9}\\
\medskip \\
\begin{tabular}{rrrrrrrrp{120mm}}
WV&WK&WB&ABK&ABB&ABV&AnzB&TW&Zahlencode \textcolor{red}{$\boldsymbol{Grundtext}$} Umschrift $|$"Ubersetzung(en)\\
1.&54.&54.&220.&220.&1.&2&30&20\_10 \textcolor{red}{\textcjheb{yk}} KJ $|$denn\\
2.&55.&55.&222.&222.&3.&4&446&30\_6\_10\_400 \textcolor{red}{\textcjheb{tywl}} LWJT $|$(ein) Kranz\\
3.&56.&56.&226.&226.&7.&2&58&8\_50 \textcolor{red}{\textcjheb{n.h}} CN $|$anmutiger/(der) Gnade\\
4.&57.&57.&228.&228.&9.&2&45&5\_40 \textcolor{red}{\textcjheb{mh}} HM $|$werden sie sein/(sind) sie\\
5.&58.&58.&230.&230.&11.&5&551&30\_200\_1\_300\_20 \textcolor{red}{\textcjheb{k+s'rl}} LRASK $|$deinem Haupt/f"ur dein Haupt\\
6.&59.&59.&235.&235.&16.&6&276&6\_70\_50\_100\_10\_40 \textcolor{red}{\textcjheb{myqn`w}} WaNQJM $|$und (ein) Geschmeide(n)\\
7.&60.&60.&241.&241.&22.&8&866&30\_3\_200\_3\_200\_400\_10\_20 \textcolor{red}{\textcjheb{kytrgrgl}} LGRGRTJK $|$deinem Hals/f"ur deinen Hals\\
\end{tabular}\medskip \\
Ende des Verses 1.9\\
Verse: 9, Buchstaben: 29, 248, 248, Totalwerte: 2272, 18596, 18596\\
\\
Denn sie werden ein anmutiger Kranz sein deinem Haupte und ein Geschmeide deinem Halse. -\\
\newpage 
{\bf -- 1.10}\\
\medskip \\
\begin{tabular}{rrrrrrrrp{120mm}}
WV&WK&WB&ABK&ABB&ABV&AnzB&TW&Zahlencode \textcolor{red}{$\boldsymbol{Grundtext}$} Umschrift $|$"Ubersetzung(en)\\
1.&61.&61.&249.&249.&1.&3&62&2\_50\_10 \textcolor{red}{\textcjheb{ynb}} BNJ $|$mein Sohn\\
2.&62.&62.&252.&252.&4.&2&41&1\_40 \textcolor{red}{\textcjheb{m'}} AM $|$wenn\\
3.&63.&63.&254.&254.&6.&5&516&10\_80\_400\_6\_20 \textcolor{red}{\textcjheb{kwtpy}} JPTWK $|$dich locken/sie wollen dich bet"oren\\
4.&64.&64.&259.&259.&11.&5&68&8\_9\_1\_10\_40 \textcolor{red}{\textcjheb{my'.t.h}} CtAJM $|$S"under\\
5.&65.&65.&264.&264.&16.&2&31&1\_30 \textcolor{red}{\textcjheb{l'}} AL $|$(so) nicht\\
6.&66.&66.&266.&266.&18.&3&403&400\_2\_1 \textcolor{red}{\textcjheb{'bt}} TBA $|$willige ein/du sollst willig sein\\
\end{tabular}\medskip \\
Ende des Verses 1.10\\
Verse: 10, Buchstaben: 20, 268, 268, Totalwerte: 1121, 19717, 19717\\
\\
Mein Sohn, wenn S"under dich locken, so willige nicht ein.\\
\newpage 
{\bf -- 1.11}\\
\medskip \\
\begin{tabular}{rrrrrrrrp{120mm}}
WV&WK&WB&ABK&ABB&ABV&AnzB&TW&Zahlencode \textcolor{red}{$\boldsymbol{Grundtext}$} Umschrift $|$"Ubersetzung(en)\\
1.&67.&67.&269.&269.&1.&2&41&1\_40 \textcolor{red}{\textcjheb{m'}} AM $|$wenn\\
2.&68.&68.&271.&271.&3.&5&257&10\_1\_40\_200\_6 \textcolor{red}{\textcjheb{wrm'y}} JAMRW $|$sie sagen\\
3.&69.&69.&276.&276.&8.&3&55&30\_20\_5 \textcolor{red}{\textcjheb{hkl}} LKH $|$gehe\\
4.&70.&70.&279.&279.&11.&4&457&1\_400\_50\_6 \textcolor{red}{\textcjheb{wnt'}} ATNW $|$mit uns\\
5.&71.&71.&283.&283.&15.&5&258&50\_1\_200\_2\_5 \textcolor{red}{\textcjheb{hbr'n}} NARBH $|$wir wollen lauern\\
6.&72.&72.&288.&288.&20.&3&74&30\_4\_40 \textcolor{red}{\textcjheb{mdl}} LDM $|$auf Blut\\
7.&73.&73.&291.&291.&23.&5&275&50\_90\_80\_50\_5 \textcolor{red}{\textcjheb{hnp.sn}} N"sPNH $|$(wir) wollen nachstellen\\
8.&74.&74.&296.&296.&28.&4&190&30\_50\_100\_10 \textcolor{red}{\textcjheb{yqnl}} LNQJ $|$dem Unschuldigen\\
9.&75.&75.&300.&300.&32.&3&98&8\_50\_40 \textcolor{red}{\textcjheb{mn.h}} CNM $|$ohne Ursache/grundlos\\
\end{tabular}\medskip \\
Ende des Verses 1.11\\
Verse: 11, Buchstaben: 34, 302, 302, Totalwerte: 1705, 21422, 21422\\
\\
Wenn sie sagen: Geh mit uns, wir wollen auf Blut lauern, wollen den Unschuldigen nachstellen ohne Ursache;\\
\newpage 
{\bf -- 1.12}\\
\medskip \\
\begin{tabular}{rrrrrrrrp{120mm}}
WV&WK&WB&ABK&ABB&ABV&AnzB&TW&Zahlencode \textcolor{red}{$\boldsymbol{Grundtext}$} Umschrift $|$"Ubersetzung(en)\\
1.&76.&76.&303.&303.&1.&5&192&50\_2\_30\_70\_40 \textcolor{red}{\textcjheb{m`lbn}} NBLaM $|$wir (wollen) verschlingen sie\\
2.&77.&77.&308.&308.&6.&5&357&20\_300\_1\_6\_30 \textcolor{red}{\textcjheb{lw'+sk}} KSAWL $|$wie (der) Scheol\\
3.&78.&78.&313.&313.&11.&4&68&8\_10\_10\_40 \textcolor{red}{\textcjheb{myy.h}} CJJM $|$lebendig(e)\\
4.&79.&79.&317.&317.&15.&7&546&6\_400\_40\_10\_40\_10\_40 \textcolor{red}{\textcjheb{mymymtw}} WTMJMJM $|$und unverletzt/und Gesunde\\
5.&80.&80.&324.&324.&22.&6&250&20\_10\_6\_200\_4\_10 \textcolor{red}{\textcjheb{ydrwyk}} KJWRDJ $|$gleich denen welche pl"otzlich hinabfahren/wie Hinabsteigende\\
6.&81.&81.&330.&330.&28.&3&208&2\_6\_200 \textcolor{red}{\textcjheb{rwb}} BWR $|$(in die) Grube\\
\end{tabular}\medskip \\
Ende des Verses 1.12\\
Verse: 12, Buchstaben: 30, 332, 332, Totalwerte: 1621, 23043, 23043\\
\\
wir wollen sie lebendig verschlingen wie der Scheol, und unverletzt, gleich denen, welche pl"otzlich in die Grube hinabfahren;\\
\newpage 
{\bf -- 1.13}\\
\medskip \\
\begin{tabular}{rrrrrrrrp{120mm}}
WV&WK&WB&ABK&ABB&ABV&AnzB&TW&Zahlencode \textcolor{red}{$\boldsymbol{Grundtext}$} Umschrift $|$"Ubersetzung(en)\\
1.&82.&82.&333.&333.&1.&2&50&20\_30 \textcolor{red}{\textcjheb{lk}} KL $|$allerlei\\
2.&83.&83.&335.&335.&3.&3&61&5\_6\_50 \textcolor{red}{\textcjheb{nwh}} HWN $|$Gut/Reichtum\\
3.&84.&84.&338.&338.&6.&3&310&10\_100\_200 \textcolor{red}{\textcjheb{rqy}} JQR $|$kostbares/wertvoll(e)\\
4.&85.&85.&341.&341.&9.&4&181&50\_40\_90\_1 \textcolor{red}{\textcjheb{'.smn}} NM"sA $|$werden wir erlangen/wir werden finden\\
5.&86.&86.&345.&345.&13.&4&121&50\_40\_30\_1 \textcolor{red}{\textcjheb{'lmn}} NMLA $|$(wir) werden f"ullen\\
6.&87.&87.&349.&349.&17.&5&468&2\_400\_10\_50\_6 \textcolor{red}{\textcjheb{wnytb}} BTJNW $|$unsere H"auser\\
7.&88.&88.&354.&354.&22.&3&360&300\_30\_30 \textcolor{red}{\textcjheb{ll+s}} SLL $|$(mit) Beute\\
\end{tabular}\medskip \\
Ende des Verses 1.13\\
Verse: 13, Buchstaben: 24, 356, 356, Totalwerte: 1551, 24594, 24594\\
\\
wir werden allerlei kostbares Gut erlangen, werden unsere H"auser mit Beute f"ullen;\\
\newpage 
{\bf -- 1.14}\\
\medskip \\
\begin{tabular}{rrrrrrrrp{120mm}}
WV&WK&WB&ABK&ABB&ABV&AnzB&TW&Zahlencode \textcolor{red}{$\boldsymbol{Grundtext}$} Umschrift $|$"Ubersetzung(en)\\
1.&89.&89.&357.&357.&1.&5&259&3\_6\_200\_30\_20 \textcolor{red}{\textcjheb{klrwg}} GWRLK $|$dein Los\\
2.&90.&90.&362.&362.&6.&4&520&400\_80\_10\_30 \textcolor{red}{\textcjheb{lypt}} TPJL $|$sollst du werfen/du wirst werfen\\
3.&91.&91.&366.&366.&10.&6&484&2\_400\_6\_20\_50\_6 \textcolor{red}{\textcjheb{wnkwtb}} BTWKNW $|$mitten unter uns/in unsere Mitte\\
4.&92.&92.&372.&372.&16.&3&90&20\_10\_60 \textcolor{red}{\textcjheb{syk}} KJs $|$Beutel\\
5.&93.&93.&375.&375.&19.&3&13&1\_8\_4 \textcolor{red}{\textcjheb{d.h'}} ACD $|$einen/einer\\
6.&94.&94.&378.&378.&22.&4&30&10\_5\_10\_5 \textcolor{red}{\textcjheb{hyhy}} JHJH $|$wir werden haben/er soll geh"oren\\
7.&95.&95.&382.&382.&26.&5&136&30\_20\_30\_50\_6 \textcolor{red}{\textcjheb{wnlkl}} LKLNW $|$(uns) alle(n)\\
\end{tabular}\medskip \\
Ende des Verses 1.14\\
Verse: 14, Buchstaben: 30, 386, 386, Totalwerte: 1532, 26126, 26126\\
\\
du sollst dein Los mitten unter uns werfen, wir alle werden einen Beutel haben:\\
\newpage 
{\bf -- 1.15}\\
\medskip \\
\begin{tabular}{rrrrrrrrp{120mm}}
WV&WK&WB&ABK&ABB&ABV&AnzB&TW&Zahlencode \textcolor{red}{$\boldsymbol{Grundtext}$} Umschrift $|$"Ubersetzung(en)\\
1.&96.&96.&387.&387.&1.&3&62&2\_50\_10 \textcolor{red}{\textcjheb{ynb}} BNJ $|$mein Sohn\\
2.&97.&97.&390.&390.&4.&2&31&1\_30 \textcolor{red}{\textcjheb{l'}} AL $|$nicht\\
3.&98.&98.&392.&392.&6.&3&450&400\_30\_20 \textcolor{red}{\textcjheb{klt}} TLK $|$wandle/du sollst gehen\\
4.&99.&99.&395.&395.&9.&4&226&2\_4\_200\_20 \textcolor{red}{\textcjheb{krdb}} BDRK $|$auf dem Weg/auf einem Weg\\
5.&100.&100.&399.&399.&13.&3&441&1\_400\_40 \textcolor{red}{\textcjheb{mt'}} ATM $|$mit ihnen\\
6.&101.&101.&402.&402.&16.&3&160&40\_50\_70 \textcolor{red}{\textcjheb{`nm}} MNa $|$halte zur"uck/halte ferne\\
7.&102.&102.&405.&405.&19.&4&253&200\_3\_30\_20 \textcolor{red}{\textcjheb{klgr}} RGLK $|$deinen Fu"s\\
8.&103.&103.&409.&409.&23.&7&942&40\_50\_400\_10\_2\_400\_40 \textcolor{red}{\textcjheb{mtbytnm}} MNTJBTM $|$von ihrem Pfad\\
\end{tabular}\medskip \\
Ende des Verses 1.15\\
Verse: 15, Buchstaben: 29, 415, 415, Totalwerte: 2565, 28691, 28691\\
\\
Mein Sohn, wandle nicht mit ihnen auf dem Wege, halte deinen Fu"s zur"uck von ihrem Pfade;\\
\newpage 
{\bf -- 1.16}\\
\medskip \\
\begin{tabular}{rrrrrrrrp{120mm}}
WV&WK&WB&ABK&ABB&ABV&AnzB&TW&Zahlencode \textcolor{red}{$\boldsymbol{Grundtext}$} Umschrift $|$"Ubersetzung(en)\\
1.&104.&104.&416.&416.&1.&2&30&20\_10 \textcolor{red}{\textcjheb{yk}} KJ $|$denn\\
2.&105.&105.&418.&418.&3.&6&288&200\_3\_30\_10\_5\_40 \textcolor{red}{\textcjheb{mhylgr}} RGLJHM $|$ihre F"u"se\\
3.&106.&106.&424.&424.&9.&3&300&30\_200\_70 \textcolor{red}{\textcjheb{`rl}} LRa $|$dem B"osem zu/(hin) zum B"osem\\
4.&107.&107.&427.&427.&12.&5&312&10\_200\_6\_90\_6 \textcolor{red}{\textcjheb{w.swry}} JRW"sW $|$(sie) laufen\\
5.&108.&108.&432.&432.&17.&6&267&6\_10\_40\_5\_200\_6 \textcolor{red}{\textcjheb{wrhmyw}} WJMHRW $|$und sie eilen\\
6.&109.&109.&438.&438.&23.&4&430&30\_300\_80\_20 \textcolor{red}{\textcjheb{kp+sl}} LSPK $|$(um) zu vergie"sen\\
7.&110.&110.&442.&442.&27.&2&44&4\_40 \textcolor{red}{\textcjheb{md}} DM $|$Blut\\
\end{tabular}\medskip \\
Ende des Verses 1.16\\
Verse: 16, Buchstaben: 28, 443, 443, Totalwerte: 1671, 30362, 30362\\
\\
denn ihre F"u"se laufen dem B"osen zu, und sie eilen, Blut zu vergie"sen.\\
\newpage 
{\bf -- 1.17}\\
\medskip \\
\begin{tabular}{rrrrrrrrp{120mm}}
WV&WK&WB&ABK&ABB&ABV&AnzB&TW&Zahlencode \textcolor{red}{$\boldsymbol{Grundtext}$} Umschrift $|$"Ubersetzung(en)\\
1.&111.&111.&444.&444.&1.&2&30&20\_10 \textcolor{red}{\textcjheb{yk}} KJ $|$denn\\
2.&112.&112.&446.&446.&3.&3&98&8\_50\_40 \textcolor{red}{\textcjheb{mn.h}} CNM $|$vergeblich/grundlos\\
3.&113.&113.&449.&449.&6.&4&252&40\_7\_200\_5 \textcolor{red}{\textcjheb{hrzm}} MZRH $|$wird ausgespannt/sie (=es) ist gespannt\\
4.&114.&114.&453.&453.&10.&4&905&5\_200\_300\_400 \textcolor{red}{\textcjheb{t+srh}} HRST $|$das Netz\\
5.&115.&115.&457.&457.&14.&5&142&2\_70\_10\_50\_10 \textcolor{red}{\textcjheb{yny`b}} BaJNJ $|$vor den Augen\\
6.&116.&116.&462.&462.&19.&2&50&20\_30 \textcolor{red}{\textcjheb{lk}} KL $|$alles/aller\\
7.&117.&117.&464.&464.&21.&3&102&2\_70\_30 \textcolor{red}{\textcjheb{l`b}} BaL $|$/Besitzer\\
8.&118.&118.&467.&467.&24.&3&150&20\_50\_80 \textcolor{red}{\textcjheb{pnk}} KNP $|$Gefl"ugelten/Fl"ugel\\
\end{tabular}\medskip \\
Ende des Verses 1.17\\
Verse: 17, Buchstaben: 26, 469, 469, Totalwerte: 1729, 32091, 32091\\
\\
Denn vergeblich wird das Netz ausgespannt vor den Augen alles Gefl"ugelten;\\
\newpage 
{\bf -- 1.18}\\
\medskip \\
\begin{tabular}{rrrrrrrrp{120mm}}
WV&WK&WB&ABK&ABB&ABV&AnzB&TW&Zahlencode \textcolor{red}{$\boldsymbol{Grundtext}$} Umschrift $|$"Ubersetzung(en)\\
1.&119.&119.&470.&470.&1.&3&51&6\_5\_40 \textcolor{red}{\textcjheb{mhw}} WHM $|$sie aber/doch sie//und sie\\
2.&120.&120.&473.&473.&4.&4&114&30\_4\_40\_40 \textcolor{red}{\textcjheb{mmdl}} LDMM $|$auf ihr (eigenes) Blut\\
3.&121.&121.&477.&477.&8.&5&219&10\_1\_200\_2\_6 \textcolor{red}{\textcjheb{wbr'y}} JARBW $|$(sie) lauern\\
4.&122.&122.&482.&482.&13.&5&236&10\_90\_80\_50\_6 \textcolor{red}{\textcjheb{wnp.sy}} J"sPNW $|$(sie) stellen nach\\
5.&123.&123.&487.&487.&18.&6&900&30\_50\_80\_300\_400\_40 \textcolor{red}{\textcjheb{mt+spnl}} LNPSTM $|$ihren eigenen Seelen/ihrem Leben\\
\end{tabular}\medskip \\
Ende des Verses 1.18\\
Verse: 18, Buchstaben: 23, 492, 492, Totalwerte: 1520, 33611, 33611\\
\\
sie aber lauern auf ihr eigenes Blut, stellen ihren eigenen Seelen nach.\\
\newpage 
{\bf -- 1.19}\\
\medskip \\
\begin{tabular}{rrrrrrrrp{120mm}}
WV&WK&WB&ABK&ABB&ABV&AnzB&TW&Zahlencode \textcolor{red}{$\boldsymbol{Grundtext}$} Umschrift $|$"Ubersetzung(en)\\
1.&124.&124.&493.&493.&1.&2&70&20\_50 \textcolor{red}{\textcjheb{nk}} KN $|$so (sind)\\
2.&125.&125.&495.&495.&3.&5&615&1\_200\_8\_6\_400 \textcolor{red}{\textcjheb{tw.hr'}} ARCWT $|$(die) Pfade\\
3.&126.&126.&500.&500.&8.&2&50&20\_30 \textcolor{red}{\textcjheb{lk}} KL $|$aller derer/jedes\\
4.&127.&127.&502.&502.&10.&3&162&2\_90\_70 \textcolor{red}{\textcjheb{`.sb}} B"sa $|$welche fr"onen/Gierenden\\
5.&128.&128.&505.&505.&13.&3&162&2\_90\_70 \textcolor{red}{\textcjheb{`.sb}} B"sa $|$der Habsucht/nach Gewinn\\
6.&129.&129.&508.&508.&16.&2&401&1\_400 \textcolor{red}{\textcjheb{t'}} AT $|$**\\
7.&130.&130.&510.&510.&18.&3&430&50\_80\_300 \textcolor{red}{\textcjheb{+spn}} NPS $|$das Leben/die Seele\\
8.&131.&131.&513.&513.&21.&5&118&2\_70\_30\_10\_6 \textcolor{red}{\textcjheb{wyl`b}} BaLJW $|$ihrem eigenen Herrn/seiner Besitzer\\
9.&132.&132.&518.&518.&26.&3&118&10\_100\_8 \textcolor{red}{\textcjheb{.hqy}} JQC $|$sie nimmt/er nimmt\\
\end{tabular}\medskip \\
Ende des Verses 1.19\\
Verse: 19, Buchstaben: 28, 520, 520, Totalwerte: 2126, 35737, 35737\\
\\
So sind die Pfade aller derer, welche der Habsucht fr"onen: sie nimmt ihrem eigenen Herrn das Leben.\\
\newpage 
{\bf -- 1.20}\\
\medskip \\
\begin{tabular}{rrrrrrrrp{120mm}}
WV&WK&WB&ABK&ABB&ABV&AnzB&TW&Zahlencode \textcolor{red}{$\boldsymbol{Grundtext}$} Umschrift $|$"Ubersetzung(en)\\
1.&133.&133.&521.&521.&1.&5&474&8\_20\_40\_6\_400 \textcolor{red}{\textcjheb{twmk.h}} CKMWT $|$(die) Weisheit\\
2.&134.&134.&526.&526.&6.&4&106&2\_8\_6\_90 \textcolor{red}{\textcjheb{.sw.hb}} BCW"s $|$drau"sen/im Drau"sen\\
3.&135.&135.&530.&530.&10.&4&655&400\_200\_50\_5 \textcolor{red}{\textcjheb{hnrt}} TRNH $|$(sie) schreit\\
4.&136.&136.&534.&534.&14.&6&618&2\_200\_8\_2\_6\_400 \textcolor{red}{\textcjheb{twb.hrb}} BRCBWT $|$auf den Stra"sen/auf den Pl"atzen\\
5.&137.&137.&540.&540.&20.&3&850&400\_400\_50 \textcolor{red}{\textcjheb{ntt}} TTN $|$sie l"asst erschallen/sie erhebt\\
6.&138.&138.&543.&543.&23.&4&141&100\_6\_30\_5 \textcolor{red}{\textcjheb{hlwq}} QWLH $|$ihre Stimme\\
\end{tabular}\medskip \\
Ende des Verses 1.20\\
Verse: 20, Buchstaben: 26, 546, 546, Totalwerte: 2844, 38581, 38581\\
\\
Die Weisheit schreit drau"sen, sie l"a"st auf den Stra"sen ihre Stimme erschallen.\\
\newpage 
{\bf -- 1.21}\\
\medskip \\
\begin{tabular}{rrrrrrrrp{120mm}}
WV&WK&WB&ABK&ABB&ABV&AnzB&TW&Zahlencode \textcolor{red}{$\boldsymbol{Grundtext}$} Umschrift $|$"Ubersetzung(en)\\
1.&139.&139.&547.&547.&1.&4&503&2\_200\_1\_300 \textcolor{red}{\textcjheb{+s'rb}} BRAS $|$an der Ecke/hoch oben\\
2.&140.&140.&551.&551.&5.&5&461&5\_40\_10\_6\_400 \textcolor{red}{\textcjheb{twymh}} HMJWT $|$l"armender Pl"atze/"uber l"armenden Pl"atzen\\
3.&141.&141.&556.&556.&10.&4&701&400\_100\_200\_1 \textcolor{red}{\textcjheb{'rqt}} TQRA $|$sie ruft\\
4.&142.&142.&560.&560.&14.&5&500&2\_80\_400\_8\_10 \textcolor{red}{\textcjheb{y.htpb}} BPTCJ $|$an den Eing"angen/an den "Offnungen\\
5.&143.&143.&565.&565.&19.&5&620&300\_70\_200\_10\_40 \textcolor{red}{\textcjheb{myr`+s}} SaRJM $|$(der) Tore\\
6.&144.&144.&570.&570.&24.&4&282&2\_70\_10\_200 \textcolor{red}{\textcjheb{ry`b}} BaJR $|$in der Stadt\\
7.&145.&145.&574.&574.&28.&5&256&1\_40\_200\_10\_5 \textcolor{red}{\textcjheb{hyrm'}} AMRJH $|$ihre Worte\\
8.&146.&146.&579.&579.&33.&4&641&400\_1\_40\_200 \textcolor{red}{\textcjheb{rm't}} TAMR $|$redet sie/sie spricht\\
\end{tabular}\medskip \\
Ende des Verses 1.21\\
Verse: 21, Buchstaben: 36, 582, 582, Totalwerte: 3964, 42545, 42545\\
\\
Sie ruft an der Ecke l"armender Pl"atze; an den Eing"angen der Tore, in der Stadt redet sie ihre Worte:\\
\newpage 
{\bf -- 1.22}\\
\medskip \\
\begin{tabular}{rrrrrrrrp{120mm}}
WV&WK&WB&ABK&ABB&ABV&AnzB&TW&Zahlencode \textcolor{red}{$\boldsymbol{Grundtext}$} Umschrift $|$"Ubersetzung(en)\\
1.&147.&147.&583.&583.&1.&2&74&70\_4 \textcolor{red}{\textcjheb{d`}} aD $|$bis\\
2.&148.&148.&585.&585.&3.&3&450&40\_400\_10 \textcolor{red}{\textcjheb{ytm}} MTJ $|$wann\\
3.&149.&149.&588.&588.&6.&4&530&80\_400\_10\_40 \textcolor{red}{\textcjheb{mytp}} PTJM $|$ihr Einf"altigen/Einf"altige\\
4.&150.&150.&592.&592.&10.&5&414&400\_1\_5\_2\_6 \textcolor{red}{\textcjheb{wbh't}} TAHBW $|$wollt ihr lieben/ihr liebt\\
5.&151.&151.&597.&597.&15.&3&490&80\_400\_10 \textcolor{red}{\textcjheb{ytp}} PTJ $|$Einf"altigkeit\\
6.&152.&152.&600.&600.&18.&5&176&6\_30\_90\_10\_40 \textcolor{red}{\textcjheb{my.slw}} WL"sJM $|$und Sp"otter\\
7.&153.&153.&605.&605.&23.&4&176&30\_90\_6\_50 \textcolor{red}{\textcjheb{nw.sl}} L"sWN $|$(an) Spott\\
8.&154.&154.&609.&609.&27.&4&58&8\_40\_4\_6 \textcolor{red}{\textcjheb{wdm.h}} CMDW $|$werden haben Lust/sie begehr(t)en\\
9.&155.&155.&613.&613.&31.&3&75&30\_5\_40 \textcolor{red}{\textcjheb{mhl}} LHM $|$ihre/f"ur sich\\
10.&156.&156.&616.&616.&34.&7&176&6\_20\_60\_10\_30\_10\_40 \textcolor{red}{\textcjheb{mylyskw}} WKsJLJM $|$und Toren/und T"orichte\\
11.&157.&157.&623.&623.&41.&5&367&10\_300\_50\_1\_6 \textcolor{red}{\textcjheb{w'n+sy}} JSNAW $|$(sie) hassen\\
12.&158.&158.&628.&628.&46.&3&474&4\_70\_400 \textcolor{red}{\textcjheb{t`d}} DaT $|$(Er)Kenntnis\\
\end{tabular}\medskip \\
Ende des Verses 1.22\\
Verse: 22, Buchstaben: 48, 630, 630, Totalwerte: 3460, 46005, 46005\\
\\
Bis wann, ihr Einf"altigen, wollt ihr Einf"altigkeit lieben, und werden Sp"otter ihre Lust haben an Spott, und Toren Erkenntnis hassen?\\
\newpage 
{\bf -- 1.23}\\
\medskip \\
\begin{tabular}{rrrrrrrrp{120mm}}
WV&WK&WB&ABK&ABB&ABV&AnzB&TW&Zahlencode \textcolor{red}{$\boldsymbol{Grundtext}$} Umschrift $|$"Ubersetzung(en)\\
1.&159.&159.&631.&631.&1.&5&714&400\_300\_6\_2\_6 \textcolor{red}{\textcjheb{wbw+st}} TSWBW $|$wendet euch um/ihr sollt umkehren\\
2.&160.&160.&636.&636.&6.&7&874&30\_400\_6\_20\_8\_400\_10 \textcolor{red}{\textcjheb{yt.hkwtl}} LTWKCTJ $|$zu meiner Zucht/zu meiner Zurechtweisung\\
3.&161.&161.&643.&643.&13.&3&60&5\_50\_5 \textcolor{red}{\textcjheb{hnh}} HNH $|$siehe\\
4.&162.&162.&646.&646.&16.&5&88&1\_2\_10\_70\_5 \textcolor{red}{\textcjheb{h`yb'}} ABJaH $|$ich will hervorstr"omen lassen/ich will sprudeln lassen\\
5.&163.&163.&651.&651.&21.&3&90&30\_20\_40 \textcolor{red}{\textcjheb{mkl}} LKM $|$euch/"uber euch\\
6.&164.&164.&654.&654.&24.&4&224&200\_6\_8\_10 \textcolor{red}{\textcjheb{y.hwr}} RWCJ $|$meinen Geist\\
7.&165.&165.&658.&658.&28.&6&96&1\_6\_4\_10\_70\_5 \textcolor{red}{\textcjheb{h`ydw'}} AWDJaH $|$(ich) will kundtun\\
8.&166.&166.&664.&664.&34.&4&216&4\_2\_200\_10 \textcolor{red}{\textcjheb{yrbd}} DBRJ $|$meine Reden/meine Worte\\
9.&167.&167.&668.&668.&38.&4&461&1\_400\_20\_40 \textcolor{red}{\textcjheb{mkt'}} ATKM $|$euch\\
\end{tabular}\medskip \\
Ende des Verses 1.23\\
Verse: 23, Buchstaben: 41, 671, 671, Totalwerte: 2823, 48828, 48828\\
\\
Wendet euch um zu meiner Zucht! Siehe, ich will euch meinen Geist hervorstr"omen lassen, will euch kundtun meine Reden. -\\
\newpage 
{\bf -- 1.24}\\
\medskip \\
\begin{tabular}{rrrrrrrrp{120mm}}
WV&WK&WB&ABK&ABB&ABV&AnzB&TW&Zahlencode \textcolor{red}{$\boldsymbol{Grundtext}$} Umschrift $|$"Ubersetzung(en)\\
1.&168.&168.&672.&672.&1.&3&130&10\_70\_50 \textcolor{red}{\textcjheb{n`y}} JaN $|$weil\\
2.&169.&169.&675.&675.&4.&5&711&100\_200\_1\_400\_10 \textcolor{red}{\textcjheb{yt'rq}} QRATJ $|$ich gerufen/ich rief\\
3.&170.&170.&680.&680.&9.&6&503&6\_400\_40\_1\_50\_6 \textcolor{red}{\textcjheb{wn'mtw}} WTMANW $|$und ihr euch geweigert habt/und ihr wolltet nicht\\
4.&171.&171.&686.&686.&15.&5&479&50\_9\_10\_400\_10 \textcolor{red}{\textcjheb{yty.tn}} NtJTJ $|$ausgestreckt/ich streckte aus\\
5.&172.&172.&691.&691.&20.&3&24&10\_4\_10 \textcolor{red}{\textcjheb{ydy}} JDJ $|$meine Hand\\
6.&173.&173.&694.&694.&23.&4&67&6\_1\_10\_50 \textcolor{red}{\textcjheb{ny'w}} WAJN $|$und niemand/und nicht war\\
7.&174.&174.&698.&698.&27.&5&452&40\_100\_300\_10\_2 \textcolor{red}{\textcjheb{by+sqm}} MQSJB $|$hat aufgemerkt/ein Aufmerkender\\
\end{tabular}\medskip \\
Ende des Verses 1.24\\
Verse: 24, Buchstaben: 31, 702, 702, Totalwerte: 2366, 51194, 51194\\
\\
Weil ich gerufen, und ihr euch geweigert habt, meine Hand ausgestreckt, und niemand aufgemerkt hat,\\
\newpage 
{\bf -- 1.25}\\
\medskip \\
\begin{tabular}{rrrrrrrrp{120mm}}
WV&WK&WB&ABK&ABB&ABV&AnzB&TW&Zahlencode \textcolor{red}{$\boldsymbol{Grundtext}$} Umschrift $|$"Ubersetzung(en)\\
1.&175.&175.&703.&703.&1.&6&762&6\_400\_80\_200\_70\_6 \textcolor{red}{\textcjheb{w`rptw}} WTPRaW $|$und ihr verworfen/und ihr lehntet ab\\
2.&176.&176.&709.&709.&7.&2&50&20\_30 \textcolor{red}{\textcjheb{lk}} KL $|$all\\
3.&177.&177.&711.&711.&9.&4&570&70\_90\_400\_10 \textcolor{red}{\textcjheb{yt.s`}} a"sTJ $|$meinen Rat\\
4.&178.&178.&715.&715.&13.&7&850&6\_400\_6\_20\_8\_400\_10 \textcolor{red}{\textcjheb{yt.hkwtw}} WTWKCTJ $|$und meine Zucht/und meine Zurechtweisung\\
5.&179.&179.&722.&722.&20.&2&31&30\_1 \textcolor{red}{\textcjheb{'l}} LA $|$nicht\\
6.&180.&180.&724.&724.&22.&5&453&1\_2\_10\_400\_40 \textcolor{red}{\textcjheb{mtyb'}} ABJTM $|$(ihr) habt gewollt\\
\end{tabular}\medskip \\
Ende des Verses 1.25\\
Verse: 25, Buchstaben: 26, 728, 728, Totalwerte: 2716, 53910, 53910\\
\\
und ihr all meinen Rat verworfen, und meine Zucht nicht gewollt habt:\\
\newpage 
{\bf -- 1.26}\\
\medskip \\
\begin{tabular}{rrrrrrrrp{120mm}}
WV&WK&WB&ABK&ABB&ABV&AnzB&TW&Zahlencode \textcolor{red}{$\boldsymbol{Grundtext}$} Umschrift $|$"Ubersetzung(en)\\
1.&181.&181.&729.&729.&1.&2&43&3\_40 \textcolor{red}{\textcjheb{mg}} GM $|$(so) auch\\
2.&182.&182.&731.&731.&3.&3&61&1\_50\_10 \textcolor{red}{\textcjheb{yn'}} ANJ $|$ich\\
3.&183.&183.&734.&734.&6.&6&77&2\_1\_10\_4\_20\_40 \textcolor{red}{\textcjheb{mkdy'b}} BAJDKM $|$bei eurem Ungl"uck\\
4.&184.&184.&740.&740.&12.&4&409&1\_300\_8\_100 \textcolor{red}{\textcjheb{q.h+s'}} ASCQ $|$(ich) werde lachen\\
5.&185.&185.&744.&744.&16.&4&104&1\_30\_70\_3 \textcolor{red}{\textcjheb{g`l'}} ALaG $|$(ich) werde spotten\\
6.&186.&186.&748.&748.&20.&3&5&2\_2\_1 \textcolor{red}{\textcjheb{'bb}} BBA $|$wenn kommt/beim Kommen\\
7.&187.&187.&751.&751.&23.&5&152&80\_8\_4\_20\_40 \textcolor{red}{\textcjheb{mkd.hp}} PCDKM $|$euer Schrecken/Schrecken vor euch\\
\end{tabular}\medskip \\
Ende des Verses 1.26\\
Verse: 26, Buchstaben: 27, 755, 755, Totalwerte: 851, 54761, 54761\\
\\
so werde auch ich bei eurem Ungl"uck lachen, werde spotten, wenn euer Schrecken kommt;\\
\newpage 
{\bf -- 1.27}\\
\medskip \\
\begin{tabular}{rrrrrrrrp{120mm}}
WV&WK&WB&ABK&ABB&ABV&AnzB&TW&Zahlencode \textcolor{red}{$\boldsymbol{Grundtext}$} Umschrift $|$"Ubersetzung(en)\\
1.&188.&188.&756.&756.&1.&3&5&2\_2\_1 \textcolor{red}{\textcjheb{'bb}} BBA $|$wenn kommt/beim Kommen\\
2.&189.&189.&759.&759.&4.&5&332&20\_300\_1\_6\_5 \textcolor{red}{\textcjheb{hw'+sk}} KSAWH $|$wie ein Unwetter\\
3.&190.&190.&764.&764.&9.&5&152&80\_8\_4\_20\_40 \textcolor{red}{\textcjheb{mkd.hp}} PCDKM $|$euer Schrecken/Schrecken vor euch\\
4.&191.&191.&769.&769.&14.&6&81&6\_1\_10\_4\_20\_40 \textcolor{red}{\textcjheb{mkdy'w}} WAJDKM $|$und euer Ungl"uck\\
5.&192.&192.&775.&775.&20.&5&171&20\_60\_6\_80\_5 \textcolor{red}{\textcjheb{hpwsk}} KsWPH $|$wie ein Sturm(wind)\\
6.&193.&193.&780.&780.&25.&4&416&10\_1\_400\_5 \textcolor{red}{\textcjheb{ht'y}} JATH $|$hereinbricht/er (=es) kommt\\
7.&194.&194.&784.&784.&29.&3&5&2\_2\_1 \textcolor{red}{\textcjheb{'bb}} BBA $|$wenn kommen/beim Kommen\\
8.&195.&195.&787.&787.&32.&5&170&70\_30\_10\_20\_40 \textcolor{red}{\textcjheb{mkyl`}} aLJKM $|$"uber euch\\
9.&196.&196.&792.&792.&37.&3&295&90\_200\_5 \textcolor{red}{\textcjheb{hr.s}} "sRH $|$Bedr"angnis/Not\\
10.&197.&197.&795.&795.&40.&5&207&6\_90\_6\_100\_5 \textcolor{red}{\textcjheb{hqw.sw}} W"sWQH $|$und Angst/und Bedr"angnis\\
\end{tabular}\medskip \\
Ende des Verses 1.27\\
Verse: 27, Buchstaben: 44, 799, 799, Totalwerte: 1834, 56595, 56595\\
\\
wenn euer Schrecken kommt wie ein Unwetter, und euer Ungl"uck hereinbricht wie ein Sturm, wenn Bedr"angnis und Angst "uber euch kommen.\\
\newpage 
{\bf -- 1.28}\\
\medskip \\
\begin{tabular}{rrrrrrrrp{120mm}}
WV&WK&WB&ABK&ABB&ABV&AnzB&TW&Zahlencode \textcolor{red}{$\boldsymbol{Grundtext}$} Umschrift $|$"Ubersetzung(en)\\
1.&198.&198.&800.&800.&1.&2&8&1\_7 \textcolor{red}{\textcjheb{z'}} AZ $|$dann\\
2.&199.&199.&802.&802.&3.&7&421&10\_100\_200\_1\_50\_50\_10 \textcolor{red}{\textcjheb{ynn'rqy}} JQRANNJ $|$werden sie zu mir rufen/sie werden rufen mich\\
3.&200.&200.&809.&809.&10.&3&37&6\_30\_1 \textcolor{red}{\textcjheb{'lw}} WLA $|$und nicht\\
4.&201.&201.&812.&812.&13.&4&126&1\_70\_50\_5 \textcolor{red}{\textcjheb{hn`'}} AaNH $|$ich werde antworten\\
5.&202.&202.&816.&816.&17.&7&628&10\_300\_8\_200\_50\_50\_10 \textcolor{red}{\textcjheb{ynnr.h+sy}} JSCRNNJ $|$sie werden suchen mich (eifrig)\\
6.&203.&203.&823.&823.&24.&3&37&6\_30\_1 \textcolor{red}{\textcjheb{'lw}} WLA $|$und nicht\\
7.&204.&204.&826.&826.&27.&7&251&10\_40\_90\_1\_50\_50\_10 \textcolor{red}{\textcjheb{ynn'.smy}} JM"sANNJ $|$(sie werden) finden mich\\
\end{tabular}\medskip \\
Ende des Verses 1.28\\
Verse: 28, Buchstaben: 33, 832, 832, Totalwerte: 1508, 58103, 58103\\
\\
Dann werden sie zu mir rufen, und ich werde nicht antworten; sie werden mich eifrig suchen, und mich nicht finden:\\
\newpage 
{\bf -- 1.29}\\
\medskip \\
\begin{tabular}{rrrrrrrrp{120mm}}
WV&WK&WB&ABK&ABB&ABV&AnzB&TW&Zahlencode \textcolor{red}{$\boldsymbol{Grundtext}$} Umschrift $|$"Ubersetzung(en)\\
1.&205.&205.&833.&833.&1.&3&808&400\_8\_400 \textcolor{red}{\textcjheb{t.ht}} TCT $|$darum/daf"ur\\
2.&206.&206.&836.&836.&4.&2&30&20\_10 \textcolor{red}{\textcjheb{yk}} KJ $|$dass\\
3.&207.&207.&838.&838.&6.&4&357&300\_50\_1\_6 \textcolor{red}{\textcjheb{w'n+s}} SNAW $|$sie gehasst/sie hass(t)en\\
4.&208.&208.&842.&842.&10.&3&474&4\_70\_400 \textcolor{red}{\textcjheb{t`d}} DaT $|$(Er)Kenntnis\\
5.&209.&209.&845.&845.&13.&5&617&6\_10\_200\_1\_400 \textcolor{red}{\textcjheb{t'ryw}} WJRAT $|$und (die) Furcht\\
6.&210.&210.&850.&850.&18.&4&26&10\_5\_6\_5 \textcolor{red}{\textcjheb{hwhy}} JHWH $|$(vor) Jahwe(s)\\
7.&211.&211.&854.&854.&22.&2&31&30\_1 \textcolor{red}{\textcjheb{'l}} LA $|$nicht\\
8.&212.&212.&856.&856.&24.&4&216&2\_8\_200\_6 \textcolor{red}{\textcjheb{wr.hb}} BCRW $|$erw"ahlt/sie (er)w"ahlten\\
\end{tabular}\medskip \\
Ende des Verses 1.29\\
Verse: 29, Buchstaben: 27, 859, 859, Totalwerte: 2559, 60662, 60662\\
\\
darum, da"s sie Erkenntnis geha"st und die Furcht Jahwes nicht erw"ahlt,\\
\newpage 
{\bf -- 1.30}\\
\medskip \\
\begin{tabular}{rrrrrrrrp{120mm}}
WV&WK&WB&ABK&ABB&ABV&AnzB&TW&Zahlencode \textcolor{red}{$\boldsymbol{Grundtext}$} Umschrift $|$"Ubersetzung(en)\\
1.&213.&213.&860.&860.&1.&2&31&30\_1 \textcolor{red}{\textcjheb{'l}} LA $|$nicht\\
2.&214.&214.&862.&862.&3.&3&9&1\_2\_6 \textcolor{red}{\textcjheb{wb'}} ABW $|$eingewilligt haben in/sie wollten\\
3.&215.&215.&865.&865.&6.&5&600&30\_70\_90\_400\_10 \textcolor{red}{\textcjheb{yt.s`l}} La"sTJ $|$meinen Rat\\
4.&216.&216.&870.&870.&11.&4&147&50\_1\_90\_6 \textcolor{red}{\textcjheb{w.s'n}} NA"sW $|$verschm"aht/sie verachteten\\
5.&217.&217.&874.&874.&15.&2&50&20\_30 \textcolor{red}{\textcjheb{lk}} KL $|$all(e)\\
6.&218.&218.&876.&876.&17.&6&844&400\_6\_20\_8\_400\_10 \textcolor{red}{\textcjheb{yt.hkwt}} TWKCTJ $|$meine Zucht/meine Zurechtweisung\\
\end{tabular}\medskip \\
Ende des Verses 1.30\\
Verse: 30, Buchstaben: 22, 881, 881, Totalwerte: 1681, 62343, 62343\\
\\
nicht eingewilligt haben in meinen Rat, verschm"aht alle meine Zucht.\\
\newpage 
{\bf -- 1.31}\\
\medskip \\
\begin{tabular}{rrrrrrrrp{120mm}}
WV&WK&WB&ABK&ABB&ABV&AnzB&TW&Zahlencode \textcolor{red}{$\boldsymbol{Grundtext}$} Umschrift $|$"Ubersetzung(en)\\
1.&219.&219.&882.&882.&1.&6&73&6\_10\_1\_20\_30\_6 \textcolor{red}{\textcjheb{wlk'yw}} WJAKLW $|$und sie werden essen/und sie sollen essen\\
2.&220.&220.&888.&888.&7.&4&330&40\_80\_200\_10 \textcolor{red}{\textcjheb{yrpm}} MPRJ $|$von der Frucht\\
3.&221.&221.&892.&892.&11.&4&264&4\_200\_20\_40 \textcolor{red}{\textcjheb{mkrd}} DRKM $|$ihres Weges\\
4.&222.&222.&896.&896.&15.&9&701&6\_40\_40\_70\_90\_400\_10\_5\_40 \textcolor{red}{\textcjheb{mhyt.s`mmw}} WMMa"sTJHM $|$und von ihren Ratschl"ussen\\
5.&223.&223.&905.&905.&24.&5&388&10\_300\_2\_70\_6 \textcolor{red}{\textcjheb{w`b+sy}} JSBaW $|$sich s"attigen/sie (=es) sollen satt sein\\
\end{tabular}\medskip \\
Ende des Verses 1.31\\
Verse: 31, Buchstaben: 28, 909, 909, Totalwerte: 1756, 64099, 64099\\
\\
Und sie werden essen von der Frucht ihres Weges, und von ihren Ratschl"agen sich s"attigen.\\
\newpage 
{\bf -- 1.32}\\
\medskip \\
\begin{tabular}{rrrrrrrrp{120mm}}
WV&WK&WB&ABK&ABB&ABV&AnzB&TW&Zahlencode \textcolor{red}{$\boldsymbol{Grundtext}$} Umschrift $|$"Ubersetzung(en)\\
1.&224.&224.&910.&910.&1.&2&30&20\_10 \textcolor{red}{\textcjheb{yk}} KJ $|$denn\\
2.&225.&225.&912.&912.&3.&5&748&40\_300\_6\_2\_400 \textcolor{red}{\textcjheb{tbw+sm}} MSWBT $|$die Abtr"unnigkeit\\
3.&226.&226.&917.&917.&8.&4&530&80\_400\_10\_40 \textcolor{red}{\textcjheb{mytp}} PTJM $|$der Einf"altigen\\
4.&227.&227.&921.&921.&12.&5&648&400\_5\_200\_3\_40 \textcolor{red}{\textcjheb{mgrht}} THRGM $|$wird sie t"oten/sie t"otet sie\\
5.&228.&228.&926.&926.&17.&5&742&6\_300\_30\_6\_400 \textcolor{red}{\textcjheb{twl+sw}} WSLWT $|$und die Sorglosigkeit\\
6.&229.&229.&931.&931.&22.&6&170&20\_60\_10\_30\_10\_40 \textcolor{red}{\textcjheb{mylysk}} KsJLJM $|$der Toren\\
7.&230.&230.&937.&937.&28.&5&447&400\_1\_2\_4\_40 \textcolor{red}{\textcjheb{mdb't}} TABDM $|$sie umbringen/sie vernichtet sie\\
\end{tabular}\medskip \\
Ende des Verses 1.32\\
Verse: 32, Buchstaben: 32, 941, 941, Totalwerte: 3315, 67414, 67414\\
\\
Denn die Abtr"unnigkeit der Einf"altigen wird sie t"oten, und die Sorglosigkeit der Toren sie umbringen;\\
\newpage 
{\bf -- 1.33}\\
\medskip \\
\begin{tabular}{rrrrrrrrp{120mm}}
WV&WK&WB&ABK&ABB&ABV&AnzB&TW&Zahlencode \textcolor{red}{$\boldsymbol{Grundtext}$} Umschrift $|$"Ubersetzung(en)\\
1.&231.&231.&942.&942.&1.&4&416&6\_300\_40\_70 \textcolor{red}{\textcjheb{`m+sw}} WSMa $|$wer aber h"ort/und (der) H"orende\\
2.&232.&232.&946.&946.&5.&2&40&30\_10 \textcolor{red}{\textcjheb{yl}} LJ $|$auf mich\\
3.&233.&233.&948.&948.&7.&4&380&10\_300\_20\_50 \textcolor{red}{\textcjheb{nk+sy}} JSKN $|$wird wohnen/er wohnt\\
4.&234.&234.&952.&952.&11.&3&19&2\_9\_8 \textcolor{red}{\textcjheb{.h.tb}} BtC $|$sicher/in Sicherheit\\
5.&235.&235.&955.&955.&14.&5&407&6\_300\_1\_50\_50 \textcolor{red}{\textcjheb{nn'+sw}} WSANN $|$und wird ruhig sein/und sorglos\\
6.&236.&236.&960.&960.&19.&4&132&40\_80\_8\_4 \textcolor{red}{\textcjheb{d.hpm}} MPCD $|$vor dem Schrecken\\
7.&237.&237.&964.&964.&23.&3&275&200\_70\_5 \textcolor{red}{\textcjheb{h`r}} RaH $|$des "Ubels/des Unheils\\
\end{tabular}\medskip \\
Ende des Verses 1.33\\
Verse: 33, Buchstaben: 25, 966, 966, Totalwerte: 1669, 69083, 69083\\
\\
wer aber auf mich h"ort, wird sicher wohnen, und wird ruhig sein vor des "Ubels Schrecken.\\
\\
{\bf Ende des Kapitels 1}\\
\newpage 
{\bf -- 2.1}\\
\medskip \\
\begin{tabular}{rrrrrrrrp{120mm}}
WV&WK&WB&ABK&ABB&ABV&AnzB&TW&Zahlencode \textcolor{red}{$\boldsymbol{Grundtext}$} Umschrift $|$"Ubersetzung(en)\\
1.&1.&238.&1.&967.&1.&3&62&2\_50\_10 \textcolor{red}{\textcjheb{ynb}} BNJ $|$mein Sohn\\
2.&2.&239.&4.&970.&4.&2&41&1\_40 \textcolor{red}{\textcjheb{m'}} AM $|$wenn\\
3.&3.&240.&6.&972.&6.&3&508&400\_100\_8 \textcolor{red}{\textcjheb{.hqt}} TQC $|$du (an)nimmst\\
4.&4.&241.&9.&975.&9.&4&251&1\_40\_200\_10 \textcolor{red}{\textcjheb{yrm'}} AMRJ $|$meine Reden/(meine) Worte\\
5.&5.&242.&13.&979.&13.&6&552&6\_40\_90\_6\_400\_10 \textcolor{red}{\textcjheb{ytw.smw}} WM"sWTJ $|$und meine Gebote\\
6.&6.&243.&19.&985.&19.&4&620&400\_90\_80\_50 \textcolor{red}{\textcjheb{np.st}} T"sPN $|$du verwahrst/du bewahrst\\
7.&7.&244.&23.&989.&23.&3&421&1\_400\_20 \textcolor{red}{\textcjheb{kt'}} ATK $|$bei dir\\
\end{tabular}\medskip \\
Ende des Verses 2.1\\
Verse: 34, Buchstaben: 25, 25, 991, Totalwerte: 2455, 2455, 71538\\
\\
Mein Sohn, wenn du meine Reden annimmst und meine Gebote bei dir verwahrst,\\
\newpage 
{\bf -- 2.2}\\
\medskip \\
\begin{tabular}{rrrrrrrrp{120mm}}
WV&WK&WB&ABK&ABB&ABV&AnzB&TW&Zahlencode \textcolor{red}{$\boldsymbol{Grundtext}$} Umschrift $|$"Ubersetzung(en)\\
1.&8.&245.&26.&992.&1.&6&447&30\_5\_100\_300\_10\_2 \textcolor{red}{\textcjheb{by+sqhl}} LHQSJB $|$so dass du merken l"asst/indem hinh"ort\\
2.&9.&246.&32.&998.&7.&5&103&30\_8\_20\_40\_5 \textcolor{red}{\textcjheb{hmk.hl}} LCKMH $|$auf Weisheit/zur Weisheit\\
3.&10.&247.&37.&1003.&12.&4&78&1\_7\_50\_20 \textcolor{red}{\textcjheb{knz'}} AZNK $|$dein Ohr\\
4.&11.&248.&41.&1007.&16.&3&414&400\_9\_5 \textcolor{red}{\textcjheb{h.tt}} TtH $|$(du) neigst\\
5.&12.&249.&44.&1010.&19.&3&52&30\_2\_20 \textcolor{red}{\textcjheb{kbl}} LBK $|$dein Herz\\
6.&13.&250.&47.&1013.&22.&6&493&30\_400\_2\_6\_50\_5 \textcolor{red}{\textcjheb{hnwbtl}} LTBWNH $|$zum Verst"andnis/zur Einsicht\\
\end{tabular}\medskip \\
Ende des Verses 2.2\\
Verse: 35, Buchstaben: 27, 52, 1018, Totalwerte: 1587, 4042, 73125\\
\\
so da"s du dein Ohr auf Weisheit merken l"a"st, dein Herz neigst zum Verst"andnis;\\
\newpage 
{\bf -- 2.3}\\
\medskip \\
\begin{tabular}{rrrrrrrrp{120mm}}
WV&WK&WB&ABK&ABB&ABV&AnzB&TW&Zahlencode \textcolor{red}{$\boldsymbol{Grundtext}$} Umschrift $|$"Ubersetzung(en)\\
1.&14.&251.&53.&1019.&1.&2&30&20\_10 \textcolor{red}{\textcjheb{yk}} KJ $|$ja/denn\\
2.&15.&252.&55.&1021.&3.&2&41&1\_40 \textcolor{red}{\textcjheb{m'}} AM $|$wenn\\
3.&16.&253.&57.&1023.&5.&5&97&30\_2\_10\_50\_5 \textcolor{red}{\textcjheb{hnybl}} LBJNH $|$dem Verstand/nach Verst"andnis\\
4.&17.&254.&62.&1028.&10.&4&701&400\_100\_200\_1 \textcolor{red}{\textcjheb{'rqt}} TQRA $|$du rufst\\
5.&18.&255.&66.&1032.&14.&6&493&30\_400\_2\_6\_50\_5 \textcolor{red}{\textcjheb{hnwbtl}} LTBWNH $|$zum Verst"andnis/zur Einsicht\\
6.&19.&256.&72.&1038.&20.&3&850&400\_400\_50 \textcolor{red}{\textcjheb{ntt}} TTN $|$(du) erhebst\\
7.&20.&257.&75.&1041.&23.&4&156&100\_6\_30\_20 \textcolor{red}{\textcjheb{klwq}} QWLK $|$deine Stimme\\
\end{tabular}\medskip \\
Ende des Verses 2.3\\
Verse: 36, Buchstaben: 26, 78, 1044, Totalwerte: 2368, 6410, 75493\\
\\
ja, wenn du dem Verstande rufst, deine Stimme erhebst zum Verst"andnis;\\
\newpage 
{\bf -- 2.4}\\
\medskip \\
\begin{tabular}{rrrrrrrrp{120mm}}
WV&WK&WB&ABK&ABB&ABV&AnzB&TW&Zahlencode \textcolor{red}{$\boldsymbol{Grundtext}$} Umschrift $|$"Ubersetzung(en)\\
1.&21.&258.&79.&1045.&1.&2&41&1\_40 \textcolor{red}{\textcjheb{m'}} AM $|$wenn\\
2.&22.&259.&81.&1047.&3.&6&857&400\_2\_100\_300\_50\_5 \textcolor{red}{\textcjheb{hn+sqbt}} TBQSNH $|$du ihn suchst/du suchst sie\\
3.&23.&260.&87.&1053.&9.&4&180&20\_20\_60\_80 \textcolor{red}{\textcjheb{pskk}} KKsP $|$wie (das) Silber\\
4.&24.&261.&91.&1057.&13.&9&221&6\_20\_40\_9\_40\_6\_50\_10\_40 \textcolor{red}{\textcjheb{mynwm.tmkw}} WKMtMWNJM $|$und wie nach verborgenen Sch"atzen/und wie die Sch"atze\\
5.&25.&262.&100.&1066.&22.&6&843&400\_8\_80\_300\_50\_5 \textcolor{red}{\textcjheb{hn+sp.ht}} TCPSNH $|$ihm nachsp"urst/du erforschst sie\\
\end{tabular}\medskip \\
Ende des Verses 2.4\\
Verse: 37, Buchstaben: 27, 105, 1071, Totalwerte: 2142, 8552, 77635\\
\\
wenn du ihn suchst wie Silber, und wie nach verborgenen Sch"atzen ihm nachsp"urst:\\
\newpage 
{\bf -- 2.5}\\
\medskip \\
\begin{tabular}{rrrrrrrrp{120mm}}
WV&WK&WB&ABK&ABB&ABV&AnzB&TW&Zahlencode \textcolor{red}{$\boldsymbol{Grundtext}$} Umschrift $|$"Ubersetzung(en)\\
1.&26.&263.&106.&1072.&1.&2&8&1\_7 \textcolor{red}{\textcjheb{z'}} AZ $|$dann\\
2.&27.&264.&108.&1074.&3.&4&462&400\_2\_10\_50 \textcolor{red}{\textcjheb{nybt}} TBJN $|$du wirst verstehen\\
3.&28.&265.&112.&1078.&7.&4&611&10\_200\_1\_400 \textcolor{red}{\textcjheb{t'ry}} JRAT $|$(die) Furcht\\
4.&29.&266.&116.&1082.&11.&4&26&10\_5\_6\_5 \textcolor{red}{\textcjheb{hwhy}} JHWH $|$(vor) Jahwe(s)\\
5.&30.&267.&120.&1086.&15.&4&480&6\_4\_70\_400 \textcolor{red}{\textcjheb{t`dw}} WDaT $|$und (die) Erkenntnis\\
6.&31.&268.&124.&1090.&19.&5&86&1\_30\_5\_10\_40 \textcolor{red}{\textcjheb{myhl'}} ALHJM $|$Gottes\\
7.&32.&269.&129.&1095.&24.&4&531&400\_40\_90\_1 \textcolor{red}{\textcjheb{'.smt}} TM"sA $|$(du wirst) finden\\
\end{tabular}\medskip \\
Ende des Verses 2.5\\
Verse: 38, Buchstaben: 27, 132, 1098, Totalwerte: 2204, 10756, 79839\\
\\
dann wirst du die Furcht Jahwes verstehen und die Erkenntnis Gottes finden.\\
\newpage 
{\bf -- 2.6}\\
\medskip \\
\begin{tabular}{rrrrrrrrp{120mm}}
WV&WK&WB&ABK&ABB&ABV&AnzB&TW&Zahlencode \textcolor{red}{$\boldsymbol{Grundtext}$} Umschrift $|$"Ubersetzung(en)\\
1.&33.&270.&133.&1099.&1.&2&30&20\_10 \textcolor{red}{\textcjheb{yk}} KJ $|$denn\\
2.&34.&271.&135.&1101.&3.&4&26&10\_5\_6\_5 \textcolor{red}{\textcjheb{hwhy}} JHWH $|$Jahwe\\
3.&35.&272.&139.&1105.&7.&3&460&10\_400\_50 \textcolor{red}{\textcjheb{nty}} JTN $|$(er) gibt\\
4.&36.&273.&142.&1108.&10.&4&73&8\_20\_40\_5 \textcolor{red}{\textcjheb{hmk.h}} CKMH $|$Weisheit\\
5.&37.&274.&146.&1112.&14.&4&136&40\_80\_10\_6 \textcolor{red}{\textcjheb{wypm}} MPJW $|$aus seinem Mund\\
6.&38.&275.&150.&1116.&18.&3&474&4\_70\_400 \textcolor{red}{\textcjheb{t`d}} DaT $|$(kommen) Erkenntnis\\
7.&39.&276.&153.&1119.&21.&6&469&6\_400\_2\_6\_50\_5 \textcolor{red}{\textcjheb{hnwbtw}} WTBWNH $|$und Verst"andnis/und Einsicht\\
\end{tabular}\medskip \\
Ende des Verses 2.6\\
Verse: 39, Buchstaben: 26, 158, 1124, Totalwerte: 1668, 12424, 81507\\
\\
Denn Jahwe gibt Weisheit; aus seinem Munde kommen Erkenntnis und Verst"andnis.\\
\newpage 
{\bf -- 2.7}\\
\medskip \\
\begin{tabular}{rrrrrrrrp{120mm}}
WV&WK&WB&ABK&ABB&ABV&AnzB&TW&Zahlencode \textcolor{red}{$\boldsymbol{Grundtext}$} Umschrift $|$"Ubersetzung(en)\\
1.&40.&277.&159.&1125.&1.&4&226&6\_90\_80\_50 \textcolor{red}{\textcjheb{np.sw}} W"sPN $|$er bewahrt/und er verwahrt\\
2.&41.&278.&163.&1129.&5.&6&590&30\_10\_300\_200\_10\_40 \textcolor{red}{\textcjheb{myr+syl}} LJSRJM $|$f"ur die Aufrichtigen/den Geraden\\
3.&42.&279.&169.&1135.&11.&5&721&400\_6\_300\_10\_5 \textcolor{red}{\textcjheb{hy+swt}} TWSJH $|$klugen Rat/Gelingen\\
4.&43.&280.&174.&1140.&16.&3&93&40\_3\_50 \textcolor{red}{\textcjheb{ngm}} MGN $|$(ein)(en) Schild\\
5.&44.&281.&177.&1143.&19.&5&95&30\_5\_30\_20\_10 \textcolor{red}{\textcjheb{yklhl}} LHLKJ $|$ist er denen die wandeln/den Gehenden\\
6.&45.&282.&182.&1148.&24.&2&440&400\_40 \textcolor{red}{\textcjheb{mt}} TM $|$in Vollkommenheit/(in) Lauterkeit\\
\end{tabular}\medskip \\
Ende des Verses 2.7\\
Verse: 40, Buchstaben: 25, 183, 1149, Totalwerte: 2165, 14589, 83672\\
\\
Er bewahrt klugen Rat auf f"ur die Aufrichtigen, er ist ein Schild denen, die in Vollkommenheit wandeln;\\
\newpage 
{\bf -- 2.8}\\
\medskip \\
\begin{tabular}{rrrrrrrrp{120mm}}
WV&WK&WB&ABK&ABB&ABV&AnzB&TW&Zahlencode \textcolor{red}{$\boldsymbol{Grundtext}$} Umschrift $|$"Ubersetzung(en)\\
1.&46.&283.&184.&1150.&1.&4&370&30\_50\_90\_200 \textcolor{red}{\textcjheb{r.snl}} LN"sR $|$indem er beh"utet/zu beh"uten\\
2.&47.&284.&188.&1154.&5.&5&615&1\_200\_8\_6\_400 \textcolor{red}{\textcjheb{tw.hr'}} ARCWT $|$(die) Pfade\\
3.&48.&285.&193.&1159.&10.&4&429&40\_300\_80\_9 \textcolor{red}{\textcjheb{.tp+sm}} MSPt $|$des Rechts\\
4.&49.&286.&197.&1163.&14.&4&230&6\_4\_200\_20 \textcolor{red}{\textcjheb{krdw}} WDRK $|$und den Weg\\
5.&50.&287.&201.&1167.&18.&5&88&8\_60\_10\_4\_6 \textcolor{red}{\textcjheb{wdys.h}} CsJDW $|$seiner Frommen\\
6.&51.&288.&206.&1172.&23.&4&550&10\_300\_40\_200 \textcolor{red}{\textcjheb{rm+sy}} JSMR $|$bewahrt/er bewacht\\
\end{tabular}\medskip \\
Ende des Verses 2.8\\
Verse: 41, Buchstaben: 26, 209, 1175, Totalwerte: 2282, 16871, 85954\\
\\
indem er die Pfade des Rechts beh"utet und den Weg seiner Frommen bewahrt.\\
\newpage 
{\bf -- 2.9}\\
\medskip \\
\begin{tabular}{rrrrrrrrp{120mm}}
WV&WK&WB&ABK&ABB&ABV&AnzB&TW&Zahlencode \textcolor{red}{$\boldsymbol{Grundtext}$} Umschrift $|$"Ubersetzung(en)\\
1.&52.&289.&210.&1176.&1.&2&8&1\_7 \textcolor{red}{\textcjheb{z'}} AZ $|$dann\\
2.&53.&290.&212.&1178.&3.&4&462&400\_2\_10\_50 \textcolor{red}{\textcjheb{nybt}} TBJN $|$du wirst verstehen\\
3.&54.&291.&216.&1182.&7.&3&194&90\_4\_100 \textcolor{red}{\textcjheb{qd.s}} "sDQ $|$Gerechtigkeit\\
4.&55.&292.&219.&1185.&10.&5&435&6\_40\_300\_80\_9 \textcolor{red}{\textcjheb{.tp+smw}} WMSPt $|$und Recht\\
5.&56.&293.&224.&1190.&15.&7&606&6\_40\_10\_300\_200\_10\_40 \textcolor{red}{\textcjheb{myr+symw}} WMJSRJM $|$und Geradheit(en)\\
6.&57.&294.&231.&1197.&22.&2&50&20\_30 \textcolor{red}{\textcjheb{lk}} KL $|$jede(n)\\
7.&58.&295.&233.&1199.&24.&4&143&40\_70\_3\_30 \textcolor{red}{\textcjheb{lg`m}} MaGL $|$Bahn/Pfad\\
8.&59.&296.&237.&1203.&28.&3&17&9\_6\_2 \textcolor{red}{\textcjheb{bw.t}} tWB $|$(des) guten\\
\end{tabular}\medskip \\
Ende des Verses 2.9\\
Verse: 42, Buchstaben: 30, 239, 1205, Totalwerte: 1915, 18786, 87869\\
\\
Dann wirst du Gerechtigkeit verstehen und Recht und Geradheit, jede Bahn des Guten.\\
\newpage 
{\bf -- 2.10}\\
\medskip \\
\begin{tabular}{rrrrrrrrp{120mm}}
WV&WK&WB&ABK&ABB&ABV&AnzB&TW&Zahlencode \textcolor{red}{$\boldsymbol{Grundtext}$} Umschrift $|$"Ubersetzung(en)\\
1.&60.&297.&240.&1206.&1.&2&30&20\_10 \textcolor{red}{\textcjheb{yk}} KJ $|$denn\\
2.&61.&298.&242.&1208.&3.&4&409&400\_2\_6\_1 \textcolor{red}{\textcjheb{'wbt}} TBWA $|$(sie (=es)) wird kommen\\
3.&62.&299.&246.&1212.&7.&4&73&8\_20\_40\_5 \textcolor{red}{\textcjheb{hmk.h}} CKMH $|$Weisheit\\
4.&63.&300.&250.&1216.&11.&4&54&2\_30\_2\_20 \textcolor{red}{\textcjheb{kblb}} BLBK $|$in dein Herz\\
5.&64.&301.&254.&1220.&15.&4&480&6\_4\_70\_400 \textcolor{red}{\textcjheb{t`dw}} WDaT $|$und Erkenntnis\\
6.&65.&302.&258.&1224.&19.&5&480&30\_50\_80\_300\_20 \textcolor{red}{\textcjheb{k+spnl}} LNPSK $|$deiner Seele/f"ur deine Seele\\
7.&66.&303.&263.&1229.&24.&4&170&10\_50\_70\_40 \textcolor{red}{\textcjheb{m`ny}} JNaM $|$wird lieblich sein/(sie) wird angenehm sein\\
\end{tabular}\medskip \\
Ende des Verses 2.10\\
Verse: 43, Buchstaben: 27, 266, 1232, Totalwerte: 1696, 20482, 89565\\
\\
Denn Weisheit wird in dein Herz kommen, und Erkenntnis wird deiner Seele lieblich sein;\\
\newpage 
{\bf -- 2.11}\\
\medskip \\
\begin{tabular}{rrrrrrrrp{120mm}}
WV&WK&WB&ABK&ABB&ABV&AnzB&TW&Zahlencode \textcolor{red}{$\boldsymbol{Grundtext}$} Umschrift $|$"Ubersetzung(en)\\
1.&67.&304.&267.&1233.&1.&4&92&40\_7\_40\_5 \textcolor{red}{\textcjheb{hmzm}} MZMH $|$Besonnenheit\\
2.&68.&305.&271.&1237.&5.&4&940&400\_300\_40\_200 \textcolor{red}{\textcjheb{rm+st}} TSMR $|$(sie) wird wachen\\
3.&69.&306.&275.&1241.&9.&4&130&70\_30\_10\_20 \textcolor{red}{\textcjheb{kyl`}} aLJK $|$"uber dich\\
4.&70.&307.&279.&1245.&13.&5&463&400\_2\_6\_50\_5 \textcolor{red}{\textcjheb{hnwbt}} TBWNH $|$Verst"andnis/Einsicht\\
5.&71.&308.&284.&1250.&18.&6&765&400\_50\_90\_200\_20\_5 \textcolor{red}{\textcjheb{hkr.snt}} TN"sRKH $|$(sie) (wird) beh"uten dich\\
\end{tabular}\medskip \\
Ende des Verses 2.11\\
Verse: 44, Buchstaben: 23, 289, 1255, Totalwerte: 2390, 22872, 91955\\
\\
Besonnenheit wird "uber dich wachen, Verst"andnis dich beh"uten:\\
\newpage 
{\bf -- 2.12}\\
\medskip \\
\begin{tabular}{rrrrrrrrp{120mm}}
WV&WK&WB&ABK&ABB&ABV&AnzB&TW&Zahlencode \textcolor{red}{$\boldsymbol{Grundtext}$} Umschrift $|$"Ubersetzung(en)\\
1.&72.&309.&290.&1256.&1.&6&185&30\_5\_90\_10\_30\_20 \textcolor{red}{\textcjheb{kly.shl}} LH"sJLK $|$(um) dich zu erretten\\
2.&73.&310.&296.&1262.&7.&4&264&40\_4\_200\_20 \textcolor{red}{\textcjheb{krdm}} MDRK $|$vor vom Weg\\
3.&74.&311.&300.&1266.&11.&2&270&200\_70 \textcolor{red}{\textcjheb{`r}} Ra $|$(des) B"osen\\
4.&75.&312.&302.&1268.&13.&4&351&40\_1\_10\_300 \textcolor{red}{\textcjheb{+sy'm}} MAJS $|$vor dem Mann\\
5.&76.&313.&306.&1272.&17.&4&246&40\_4\_2\_200 \textcolor{red}{\textcjheb{rbdm}} MDBR $|$der redet/(der) sprechend(er) (ist)\\
6.&77.&314.&310.&1276.&21.&6&911&400\_5\_80\_20\_6\_400 \textcolor{red}{\textcjheb{twkpht}} THPKWT $|$Verkehrtes/Verkehrtheiten\\
\end{tabular}\medskip \\
Ende des Verses 2.12\\
Verse: 45, Buchstaben: 26, 315, 1281, Totalwerte: 2227, 25099, 94182\\
\\
um dich zu erretten von dem b"osen Wege, von dem Manne, der Verkehrtes redet;\\
\newpage 
{\bf -- 2.13}\\
\medskip \\
\begin{tabular}{rrrrrrrrp{120mm}}
WV&WK&WB&ABK&ABB&ABV&AnzB&TW&Zahlencode \textcolor{red}{$\boldsymbol{Grundtext}$} Umschrift $|$"Ubersetzung(en)\\
1.&78.&315.&316.&1282.&1.&6&134&5\_70\_7\_2\_10\_40 \textcolor{red}{\textcjheb{mybz`h}} HaZBJM $|$die da verlassen/den Verlassenden\\
2.&79.&316.&322.&1288.&7.&5&615&1\_200\_8\_6\_400 \textcolor{red}{\textcjheb{tw.hr'}} ARCWT $|$(die) Pfade\\
3.&80.&317.&327.&1293.&12.&3&510&10\_300\_200 \textcolor{red}{\textcjheb{r+sy}} JSR $|$der Geradheit/(von) Geradheit\\
4.&81.&318.&330.&1296.&15.&4&480&30\_30\_20\_400 \textcolor{red}{\textcjheb{tkll}} LLKT $|$um zu wandeln/um zu gehen\\
5.&82.&319.&334.&1300.&19.&5&236&2\_4\_200\_20\_10 \textcolor{red}{\textcjheb{ykrdb}} BDRKJ $|$auf (den) Wegen\\
6.&83.&320.&339.&1305.&24.&3&328&8\_300\_20 \textcolor{red}{\textcjheb{k+s.h}} CSK $|$(der) Finsternis\\
\end{tabular}\medskip \\
Ende des Verses 2.13\\
Verse: 46, Buchstaben: 26, 341, 1307, Totalwerte: 2303, 27402, 96485\\
\\
die da verlassen die Pfade der Geradheit, um auf den Wegen der Finsternis zu wandeln;\\
\newpage 
{\bf -- 2.14}\\
\medskip \\
\begin{tabular}{rrrrrrrrp{120mm}}
WV&WK&WB&ABK&ABB&ABV&AnzB&TW&Zahlencode \textcolor{red}{$\boldsymbol{Grundtext}$} Umschrift $|$"Ubersetzung(en)\\
1.&84.&321.&342.&1308.&1.&6&403&5\_300\_40\_8\_10\_40 \textcolor{red}{\textcjheb{my.hm+sh}} HSMCJM $|$die sich freuen/die sich Freuenden\\
2.&85.&322.&348.&1314.&7.&5&806&30\_70\_300\_6\_400 \textcolor{red}{\textcjheb{tw+s`l}} LaSWT $|$zu tun\\
3.&86.&323.&353.&1319.&12.&2&270&200\_70 \textcolor{red}{\textcjheb{`r}} Ra $|$B"oses\\
4.&87.&324.&355.&1321.&14.&5&59&10\_3\_10\_30\_6 \textcolor{red}{\textcjheb{wlygy}} JGJLW $|$(sie) frohlocken\\
5.&88.&325.&360.&1326.&19.&7&913&2\_400\_5\_80\_20\_6\_400 \textcolor{red}{\textcjheb{twkphtb}} BTHPKWT $|$"uber (die) Verkehrtheit(en)\\
6.&89.&326.&367.&1333.&26.&2&270&200\_70 \textcolor{red}{\textcjheb{`r}} Ra $|$boshafte/(eines) B"osen\\
\end{tabular}\medskip \\
Ende des Verses 2.14\\
Verse: 47, Buchstaben: 27, 368, 1334, Totalwerte: 2721, 30123, 99206\\
\\
die sich freuen, B"oses zu tun, "uber boshafte Verkehrtheit frohlocken;\\
\newpage 
{\bf -- 2.15}\\
\medskip \\
\begin{tabular}{rrrrrrrrp{120mm}}
WV&WK&WB&ABK&ABB&ABV&AnzB&TW&Zahlencode \textcolor{red}{$\boldsymbol{Grundtext}$} Umschrift $|$"Ubersetzung(en)\\
1.&90.&327.&369.&1335.&1.&3&501&1\_300\_200 \textcolor{red}{\textcjheb{r+s'}} ASR $|$/weil\\
2.&91.&328.&372.&1338.&4.&7&664&1\_200\_8\_400\_10\_5\_40 \textcolor{red}{\textcjheb{mhyt.hr'}} ARCTJHM $|$deren Pfade/ihre Wege\\
3.&92.&329.&379.&1345.&11.&5&520&70\_100\_300\_10\_40 \textcolor{red}{\textcjheb{my+sq`}} aQSJM $|$krumm sind/sind verdreht(e)\\
4.&93.&330.&384.&1350.&16.&7&149&6\_50\_30\_6\_7\_10\_40 \textcolor{red}{\textcjheb{myzwlnw}} WNLWZJM $|$und die abbiegen/und verkehrt(e)\\
5.&94.&331.&391.&1357.&23.&8&591&2\_40\_70\_3\_30\_6\_400\_40 \textcolor{red}{\textcjheb{mtwlg`mb}} BMaGLWTM $|$in ihren Bahnen/in ihren Routen\\
\end{tabular}\medskip \\
Ende des Verses 2.15\\
Verse: 48, Buchstaben: 30, 398, 1364, Totalwerte: 2425, 32548, 101631\\
\\
deren Pfade krumm sind, und die abbiegen in ihren Bahnen:\\
\newpage 
{\bf -- 2.16}\\
\medskip \\
\begin{tabular}{rrrrrrrrp{120mm}}
WV&WK&WB&ABK&ABB&ABV&AnzB&TW&Zahlencode \textcolor{red}{$\boldsymbol{Grundtext}$} Umschrift $|$"Ubersetzung(en)\\
1.&95.&332.&399.&1365.&1.&6&185&30\_5\_90\_10\_30\_20 \textcolor{red}{\textcjheb{kly.shl}} LH"sJLK $|$(um) dich zu (er)retten\\
2.&96.&333.&405.&1371.&7.&4&346&40\_1\_300\_5 \textcolor{red}{\textcjheb{h+s'm}} MASH $|$von (der) Frau/von (einer) Frau\\
3.&97.&334.&409.&1375.&11.&3&212&7\_200\_5 \textcolor{red}{\textcjheb{hrz}} ZRH $|$fremden\\
4.&98.&335.&412.&1378.&14.&6&325&40\_50\_20\_200\_10\_5 \textcolor{red}{\textcjheb{hyrknm}} MNKRJH $|$von der Fremden/vor einer Fremden\\
5.&99.&336.&418.&1384.&20.&5&256&1\_40\_200\_10\_5 \textcolor{red}{\textcjheb{hyrm'}} AMRJH $|$(die) ihre Worte\\
6.&100.&337.&423.&1389.&25.&6&158&5\_8\_30\_10\_100\_5 \textcolor{red}{\textcjheb{hqyl.hh}} HCLJQH $|$(sie) gl"attet\\
\end{tabular}\medskip \\
Ende des Verses 2.16\\
Verse: 49, Buchstaben: 30, 428, 1394, Totalwerte: 1482, 34030, 103113\\
\\
um dich zu erretten von dem fremden Weibe, von der Fremden, die ihre Worte gl"attet;\\
\newpage 
{\bf -- 2.17}\\
\medskip \\
\begin{tabular}{rrrrrrrrp{120mm}}
WV&WK&WB&ABK&ABB&ABV&AnzB&TW&Zahlencode \textcolor{red}{$\boldsymbol{Grundtext}$} Umschrift $|$"Ubersetzung(en)\\
1.&101.&338.&429.&1395.&1.&5&484&5\_70\_7\_2\_400 \textcolor{red}{\textcjheb{tbz`h}} HaZBT $|$welche verl"asst/die Verlassende\\
2.&102.&339.&434.&1400.&6.&4&117&1\_30\_6\_80 \textcolor{red}{\textcjheb{pwl'}} ALWP $|$den Freund\\
3.&103.&340.&438.&1404.&10.&6&341&50\_70\_6\_200\_10\_5 \textcolor{red}{\textcjheb{hyrw`n}} NaWRJH $|$ihrer Jugend(zeit)\\
4.&104.&341.&444.&1410.&16.&3&407&6\_1\_400 \textcolor{red}{\textcjheb{t'w}} WAT $|$und **\\
5.&105.&342.&447.&1413.&19.&4&612&2\_200\_10\_400 \textcolor{red}{\textcjheb{tyrb}} BRJT $|$den Bund\\
6.&106.&343.&451.&1417.&23.&5&51&1\_30\_5\_10\_5 \textcolor{red}{\textcjheb{hyhl'}} ALHJH $|$ihres Gottes\\
7.&107.&344.&456.&1422.&28.&4&333&300\_20\_8\_5 \textcolor{red}{\textcjheb{h.hk+s}} SKCH $|$(sie) vergisst\\
\end{tabular}\medskip \\
Ende des Verses 2.17\\
Verse: 50, Buchstaben: 31, 459, 1425, Totalwerte: 2345, 36375, 105458\\
\\
welche den Vertrauten ihrer Jugend verl"a"st und den Bund ihres Gottes vergi"st.\\
\newpage 
{\bf -- 2.18}\\
\medskip \\
\begin{tabular}{rrrrrrrrp{120mm}}
WV&WK&WB&ABK&ABB&ABV&AnzB&TW&Zahlencode \textcolor{red}{$\boldsymbol{Grundtext}$} Umschrift $|$"Ubersetzung(en)\\
1.&108.&345.&460.&1426.&1.&2&30&20\_10 \textcolor{red}{\textcjheb{yk}} KJ $|$denn\\
2.&109.&346.&462.&1428.&3.&3&313&300\_8\_5 \textcolor{red}{\textcjheb{h.h+s}} SCH $|$(sie (=es)) sinkt hinab\\
3.&110.&347.&465.&1431.&6.&2&31&1\_30 \textcolor{red}{\textcjheb{l'}} AL $|$zum\\
4.&111.&348.&467.&1433.&8.&3&446&40\_6\_400 \textcolor{red}{\textcjheb{twm}} MWT $|$Tod\\
5.&112.&349.&470.&1436.&11.&4&417&2\_10\_400\_5 \textcolor{red}{\textcjheb{htyb}} BJTH $|$ihr Haus\\
6.&113.&350.&474.&1440.&15.&3&37&6\_1\_30 \textcolor{red}{\textcjheb{l'w}} WAL $|$und zu\\
7.&114.&351.&477.&1443.&18.&5&331&200\_80\_1\_10\_40 \textcolor{red}{\textcjheb{my'pr}} RPAJM $|$den Schatten/den Verstorbenen\\
8.&115.&352.&482.&1448.&23.&7&558&40\_70\_3\_30\_400\_10\_5 \textcolor{red}{\textcjheb{hytlg`m}} MaGLTJH $|$ihre Bahnen/(f"uhren) ihre Pfade\\
\end{tabular}\medskip \\
Ende des Verses 2.18\\
Verse: 51, Buchstaben: 29, 488, 1454, Totalwerte: 2163, 38538, 107621\\
\\
Denn zum Tode sinkt ihr Haus hinab, und ihre Bahnen zu den Schatten;\\
\newpage 
{\bf -- 2.19}\\
\medskip \\
\begin{tabular}{rrrrrrrrp{120mm}}
WV&WK&WB&ABK&ABB&ABV&AnzB&TW&Zahlencode \textcolor{red}{$\boldsymbol{Grundtext}$} Umschrift $|$"Ubersetzung(en)\\
1.&116.&353.&489.&1455.&1.&2&50&20\_30 \textcolor{red}{\textcjheb{lk}} KL $|$alle\\
2.&117.&354.&491.&1457.&3.&4&18&2\_1\_10\_5 \textcolor{red}{\textcjheb{hy'b}} BAJH $|$die zu ihr eingehen/ihre Kommenden\\
3.&118.&355.&495.&1461.&7.&2&31&30\_1 \textcolor{red}{\textcjheb{'l}} LA $|$nicht\\
4.&119.&356.&497.&1463.&9.&6&374&10\_300\_6\_2\_6\_50 \textcolor{red}{\textcjheb{nwbw+sy}} JSWBWN $|$kehren wieder/sie werden kehren zur"uck\\
5.&120.&357.&503.&1469.&15.&3&37&6\_30\_1 \textcolor{red}{\textcjheb{'lw}} WLA $|$und nicht\\
6.&121.&358.&506.&1472.&18.&5&329&10\_300\_10\_3\_6 \textcolor{red}{\textcjheb{wgy+sy}} JSJGW $|$(sie werden) erreichen\\
7.&122.&359.&511.&1477.&23.&5&615&1\_200\_8\_6\_400 \textcolor{red}{\textcjheb{tw.hr'}} ARCWT $|$(die) Pfade/Wege\\
8.&123.&360.&516.&1482.&28.&4&68&8\_10\_10\_40 \textcolor{red}{\textcjheb{myy.h}} CJJM $|$des Lebens/(der) Lebenden\\
\end{tabular}\medskip \\
Ende des Verses 2.19\\
Verse: 52, Buchstaben: 31, 519, 1485, Totalwerte: 1522, 40060, 109143\\
\\
alle, die zu ihr eingehen, kehren nicht wieder und erreichen nicht die Pfade des Lebens:\\
\newpage 
{\bf -- 2.20}\\
\medskip \\
\begin{tabular}{rrrrrrrrp{120mm}}
WV&WK&WB&ABK&ABB&ABV&AnzB&TW&Zahlencode \textcolor{red}{$\boldsymbol{Grundtext}$} Umschrift $|$"Ubersetzung(en)\\
1.&124.&361.&520.&1486.&1.&4&190&30\_40\_70\_50 \textcolor{red}{\textcjheb{n`ml}} LMaN $|$damit\\
2.&125.&362.&524.&1490.&5.&3&450&400\_30\_20 \textcolor{red}{\textcjheb{klt}} TLK $|$du wandelst/du gehst\\
3.&126.&363.&527.&1493.&8.&4&226&2\_4\_200\_20 \textcolor{red}{\textcjheb{krdb}} BDRK $|$auf dem Weg\\
4.&127.&364.&531.&1497.&12.&5&67&9\_6\_2\_10\_40 \textcolor{red}{\textcjheb{mybw.t}} tWBJM $|$der Guten\\
5.&128.&365.&536.&1502.&17.&6&621&6\_1\_200\_8\_6\_400 \textcolor{red}{\textcjheb{tw.hr'w}} WARCWT $|$und die Pfade\\
6.&129.&366.&542.&1508.&23.&6&254&90\_4\_10\_100\_10\_40 \textcolor{red}{\textcjheb{myqyd.s}} "sDJQJM $|$der Gerechten\\
7.&130.&367.&548.&1514.&29.&4&940&400\_300\_40\_200 \textcolor{red}{\textcjheb{rm+st}} TSMR $|$einh"altst/du bewahrst\\
\end{tabular}\medskip \\
Ende des Verses 2.20\\
Verse: 53, Buchstaben: 32, 551, 1517, Totalwerte: 2748, 42808, 111891\\
\\
Damit du wandelst auf dem Wege der Guten und die Pfade der Gerechten einh"altst.\\
\newpage 
{\bf -- 2.21}\\
\medskip \\
\begin{tabular}{rrrrrrrrp{120mm}}
WV&WK&WB&ABK&ABB&ABV&AnzB&TW&Zahlencode \textcolor{red}{$\boldsymbol{Grundtext}$} Umschrift $|$"Ubersetzung(en)\\
1.&131.&368.&552.&1518.&1.&2&30&20\_10 \textcolor{red}{\textcjheb{yk}} KJ $|$denn\\
2.&132.&369.&554.&1520.&3.&5&560&10\_300\_200\_10\_40 \textcolor{red}{\textcjheb{myr+sy}} JSRJM $|$die Aufrichtigen/die Geraden\\
3.&133.&370.&559.&1525.&8.&5&386&10\_300\_20\_50\_6 \textcolor{red}{\textcjheb{wnk+sy}} JSKNW $|$(sie) werden (be)wohnen\\
4.&134.&371.&564.&1530.&13.&3&291&1\_200\_90 \textcolor{red}{\textcjheb{.sr'}} AR"s $|$(das) Land\\
5.&135.&372.&567.&1533.&16.&7&546&6\_400\_40\_10\_40\_10\_40 \textcolor{red}{\textcjheb{mymymtw}} WTMJMJM $|$und die Vollkommenen\\
6.&136.&373.&574.&1540.&23.&5&622&10\_6\_400\_200\_6 \textcolor{red}{\textcjheb{wrtwy}} JWTRW $|$(sie) werden ("ubrig)bleiben\\
7.&137.&374.&579.&1545.&28.&2&7&2\_5 \textcolor{red}{\textcjheb{hb}} BH $|$darin/in ihm\\
\end{tabular}\medskip \\
Ende des Verses 2.21\\
Verse: 54, Buchstaben: 29, 580, 1546, Totalwerte: 2442, 45250, 114333\\
\\
Denn die Aufrichtigen werden das Land bewohnen, und die Vollkommenen darin "ubrigbleiben;\\
\newpage 
{\bf -- 2.22}\\
\medskip \\
\begin{tabular}{rrrrrrrrp{120mm}}
WV&WK&WB&ABK&ABB&ABV&AnzB&TW&Zahlencode \textcolor{red}{$\boldsymbol{Grundtext}$} Umschrift $|$"Ubersetzung(en)\\
1.&138.&375.&581.&1547.&1.&6&626&6\_200\_300\_70\_10\_40 \textcolor{red}{\textcjheb{my`+srw}} WRSaJM $|$aber die Gesetzlosen/und Frevler\\
2.&139.&376.&587.&1553.&7.&4&331&40\_1\_200\_90 \textcolor{red}{\textcjheb{.sr'm}} MAR"s $|$aus dem Lande\\
3.&140.&377.&591.&1557.&11.&5&636&10\_20\_200\_400\_6 \textcolor{red}{\textcjheb{wtrky}} JKRTW $|$werden ausgerottet/sie werden vergetilgt\\
4.&141.&378.&596.&1562.&16.&7&71&6\_2\_6\_3\_4\_10\_40 \textcolor{red}{\textcjheb{mydgwbw}} WBWGDJM $|$und (die) Treulose(n)\\
5.&142.&379.&603.&1569.&23.&4&84&10\_60\_8\_6 \textcolor{red}{\textcjheb{w.hsy}} JsCW $|$werden weggerissen/(sie) werden herausgerissen\\
6.&143.&380.&607.&1573.&27.&4&135&40\_40\_50\_5 \textcolor{red}{\textcjheb{hnmm}} MMNH $|$daraus/aus ihm\\
\end{tabular}\medskip \\
Ende des Verses 2.22\\
Verse: 55, Buchstaben: 30, 610, 1576, Totalwerte: 1883, 47133, 116216\\
\\
aber die Gesetzlosen werden aus dem Lande ausgerottet, und die Treulosen daraus weggerissen werden.\\
\\
{\bf Ende des Kapitels 2}\\
\newpage 
{\bf -- 3.1}\\
\medskip \\
\begin{tabular}{rrrrrrrrp{120mm}}
WV&WK&WB&ABK&ABB&ABV&AnzB&TW&Zahlencode \textcolor{red}{$\boldsymbol{Grundtext}$} Umschrift $|$"Ubersetzung(en)\\
1.&1.&381.&1.&1577.&1.&3&62&2\_50\_10 \textcolor{red}{\textcjheb{ynb}} BNJ $|$mein Sohn\\
2.&2.&382.&4.&1580.&4.&5&1016&400\_6\_200\_400\_10 \textcolor{red}{\textcjheb{ytrwt}} TWRTJ $|$meine Belehrung/meine Weisung\\
3.&3.&383.&9.&1585.&9.&2&31&1\_30 \textcolor{red}{\textcjheb{l'}} AL $|$nicht\\
4.&4.&384.&11.&1587.&11.&4&728&400\_300\_20\_8 \textcolor{red}{\textcjheb{.hk+st}} TSKC $|$vergiss/sollst du vergessen\\
5.&5.&385.&15.&1591.&15.&6&552&6\_40\_90\_6\_400\_10 \textcolor{red}{\textcjheb{ytw.smw}} WM"sWTJ $|$und meine Gebote\\
6.&6.&386.&21.&1597.&21.&3&300&10\_90\_200 \textcolor{red}{\textcjheb{r.sy}} J"sR $|$bewahre/er (=es) soll bewahren\\
7.&7.&387.&24.&1600.&24.&3&52&30\_2\_20 \textcolor{red}{\textcjheb{kbl}} LBK $|$dein Herz\\
\end{tabular}\medskip \\
Ende des Verses 3.1\\
Verse: 56, Buchstaben: 26, 26, 1602, Totalwerte: 2741, 2741, 118957\\
\\
Mein Sohn, vergi"s nicht meine Belehrung, und dein Herz bewahre meine Gebote.\\
\newpage 
{\bf -- 3.2}\\
\medskip \\
\begin{tabular}{rrrrrrrrp{120mm}}
WV&WK&WB&ABK&ABB&ABV&AnzB&TW&Zahlencode \textcolor{red}{$\boldsymbol{Grundtext}$} Umschrift $|$"Ubersetzung(en)\\
1.&8.&388.&27.&1603.&1.&2&30&20\_10 \textcolor{red}{\textcjheb{yk}} KJ $|$denn\\
2.&9.&389.&29.&1605.&3.&3&221&1\_200\_20 \textcolor{red}{\textcjheb{kr'}} ARK $|$(eine) L"ange\\
3.&10.&390.&32.&1608.&6.&4&100&10\_40\_10\_40 \textcolor{red}{\textcjheb{mymy}} JMJM $|$der Tage/(an) Tagen\\
4.&11.&391.&36.&1612.&10.&5&762&6\_300\_50\_6\_400 \textcolor{red}{\textcjheb{twn+sw}} WSNWT $|$und Jahre(n)\\
5.&12.&392.&41.&1617.&15.&4&68&8\_10\_10\_40 \textcolor{red}{\textcjheb{myy.h}} CJJM $|$des Lebens\\
6.&13.&393.&45.&1621.&19.&5&382&6\_300\_30\_6\_40 \textcolor{red}{\textcjheb{mwl+sw}} WSLWM $|$und Frieden\\
7.&14.&394.&50.&1626.&24.&6&172&10\_6\_60\_10\_80\_6 \textcolor{red}{\textcjheb{wpyswy}} JWsJPW $|$sie (werden) mehren\\
8.&15.&395.&56.&1632.&30.&2&50&30\_20 \textcolor{red}{\textcjheb{kl}} LK $|$(zu) dir\\
\end{tabular}\medskip \\
Ende des Verses 3.2\\
Verse: 57, Buchstaben: 31, 57, 1633, Totalwerte: 1785, 4526, 120742\\
\\
Denn L"ange der Tage und Jahre des Lebens und Frieden werden sie dir mehren. -\\
\newpage 
{\bf -- 3.3}\\
\medskip \\
\begin{tabular}{rrrrrrrrp{120mm}}
WV&WK&WB&ABK&ABB&ABV&AnzB&TW&Zahlencode \textcolor{red}{$\boldsymbol{Grundtext}$} Umschrift $|$"Ubersetzung(en)\\
1.&16.&396.&58.&1634.&1.&3&72&8\_60\_4 \textcolor{red}{\textcjheb{ds.h}} CsD $|$G"ute/(die) Gnade\\
2.&17.&397.&61.&1637.&4.&4&447&6\_1\_40\_400 \textcolor{red}{\textcjheb{tm'w}} WAMT $|$und Wahrheit/und Treue\\
3.&18.&398.&65.&1641.&8.&2&31&1\_30 \textcolor{red}{\textcjheb{l'}} AL $|$nicht\\
4.&19.&399.&67.&1643.&10.&5&109&10\_70\_7\_2\_20 \textcolor{red}{\textcjheb{kbz`y}} JaZBK $|$(sie) m"ogen verlassen dich\\
5.&20.&400.&72.&1648.&15.&4&640&100\_300\_200\_40 \textcolor{red}{\textcjheb{mr+sq}} QSRM $|$binde sie\\
6.&21.&401.&76.&1652.&19.&2&100&70\_30 \textcolor{red}{\textcjheb{l`}} aL $|$um\\
7.&22.&402.&78.&1654.&21.&8&842&3\_200\_3\_200\_6\_400\_10\_20 \textcolor{red}{\textcjheb{kytwrgrg}} GRGRWTJK $|$deinen Hals\\
8.&23.&403.&86.&1662.&29.&4&462&20\_400\_2\_40 \textcolor{red}{\textcjheb{mbtk}} KTBM $|$schreibe sie\\
9.&24.&404.&90.&1666.&33.&2&100&70\_30 \textcolor{red}{\textcjheb{l`}} aL $|$auf\\
10.&25.&405.&92.&1668.&35.&3&44&30\_6\_8 \textcolor{red}{\textcjheb{.hwl}} LWC $|$die Tafel\\
11.&26.&406.&95.&1671.&38.&3&52&30\_2\_20 \textcolor{red}{\textcjheb{kbl}} LBK $|$deines Herzens\\
\end{tabular}\medskip \\
Ende des Verses 3.3\\
Verse: 58, Buchstaben: 40, 97, 1673, Totalwerte: 2899, 7425, 123641\\
\\
G"ute und Wahrheit m"ogen dich nicht verlassen; binde sie um deinen Hals, schreibe sie auf die Tafel deines Herzens;\\
\newpage 
{\bf -- 3.4}\\
\medskip \\
\begin{tabular}{rrrrrrrrp{120mm}}
WV&WK&WB&ABK&ABB&ABV&AnzB&TW&Zahlencode \textcolor{red}{$\boldsymbol{Grundtext}$} Umschrift $|$"Ubersetzung(en)\\
1.&27.&407.&98.&1674.&1.&4&137&6\_40\_90\_1 \textcolor{red}{\textcjheb{'.smw}} WM"sA $|$so wirst du finden/und finde\\
2.&28.&408.&102.&1678.&5.&2&58&8\_50 \textcolor{red}{\textcjheb{n.h}} CN $|$Gunst\\
3.&29.&409.&104.&1680.&7.&4&356&6\_300\_20\_30 \textcolor{red}{\textcjheb{lk+sw}} WSKL $|$und Einsicht/und Verstand\\
4.&30.&410.&108.&1684.&11.&3&17&9\_6\_2 \textcolor{red}{\textcjheb{bw.t}} tWB $|$gute(n)\\
5.&31.&411.&111.&1687.&14.&5&142&2\_70\_10\_50\_10 \textcolor{red}{\textcjheb{yny`b}} BaJNJ $|$in den Augen\\
6.&32.&412.&116.&1692.&19.&5&86&1\_30\_5\_10\_40 \textcolor{red}{\textcjheb{myhl'}} ALHJM $|$Gottes\\
7.&33.&413.&121.&1697.&24.&4&51&6\_1\_4\_40 \textcolor{red}{\textcjheb{md'w}} WADM $|$und der Menschen\\
\end{tabular}\medskip \\
Ende des Verses 3.4\\
Verse: 59, Buchstaben: 27, 124, 1700, Totalwerte: 847, 8272, 124488\\
\\
so wirst du Gunst finden und gute Einsicht in den Augen Gottes und der Menschen. -\\
\newpage 
{\bf -- 3.5}\\
\medskip \\
\begin{tabular}{rrrrrrrrp{120mm}}
WV&WK&WB&ABK&ABB&ABV&AnzB&TW&Zahlencode \textcolor{red}{$\boldsymbol{Grundtext}$} Umschrift $|$"Ubersetzung(en)\\
1.&34.&414.&125.&1701.&1.&3&19&2\_9\_8 \textcolor{red}{\textcjheb{.h.tb}} BtC $|$vertraue\\
2.&35.&415.&128.&1704.&4.&2&31&1\_30 \textcolor{red}{\textcjheb{l'}} AL $|$auf\\
3.&36.&416.&130.&1706.&6.&4&26&10\_5\_6\_5 \textcolor{red}{\textcjheb{hwhy}} JHWH $|$Jahwe\\
4.&37.&417.&134.&1710.&10.&3&52&2\_20\_30 \textcolor{red}{\textcjheb{lkb}} BKL $|$mit deinem ganzen/mit all\\
5.&38.&418.&137.&1713.&13.&3&52&30\_2\_20 \textcolor{red}{\textcjheb{kbl}} LBK $|$(deinem) Herzen\\
6.&39.&419.&140.&1716.&16.&3&37&6\_1\_30 \textcolor{red}{\textcjheb{l'w}} WAL $|$und auf\\
7.&40.&420.&143.&1719.&19.&5&482&2\_10\_50\_400\_20 \textcolor{red}{\textcjheb{ktnyb}} BJNTK $|$deinen Verstand/deine Einsicht\\
8.&41.&421.&148.&1724.&24.&2&31&1\_30 \textcolor{red}{\textcjheb{l'}} AL $|$nicht\\
9.&42.&422.&150.&1726.&26.&4&820&400\_300\_70\_50 \textcolor{red}{\textcjheb{n`+st}} TSaN $|$(du sollst) st"utze(n) dich\\
\end{tabular}\medskip \\
Ende des Verses 3.5\\
Verse: 60, Buchstaben: 29, 153, 1729, Totalwerte: 1550, 9822, 126038\\
\\
Vertraue auf Jahwe mit deinem ganzen Herzen, und st"utze dich nicht auf deinen Verstand.\\
\newpage 
{\bf -- 3.6}\\
\medskip \\
\begin{tabular}{rrrrrrrrp{120mm}}
WV&WK&WB&ABK&ABB&ABV&AnzB&TW&Zahlencode \textcolor{red}{$\boldsymbol{Grundtext}$} Umschrift $|$"Ubersetzung(en)\\
1.&43.&423.&154.&1730.&1.&3&52&2\_20\_30 \textcolor{red}{\textcjheb{lkb}} BKL $|$auf all(en)\\
2.&44.&424.&157.&1733.&4.&5&254&4\_200\_20\_10\_20 \textcolor{red}{\textcjheb{kykrd}} DRKJK $|$deinen Wegen\\
3.&45.&425.&162.&1738.&9.&4&85&4\_70\_5\_6 \textcolor{red}{\textcjheb{wh`d}} DaHW $|$erkenne ihn\\
4.&46.&426.&166.&1742.&13.&4&18&6\_5\_6\_1 \textcolor{red}{\textcjheb{'whw}} WHWA $|$und er\\
5.&47.&427.&170.&1746.&17.&4&520&10\_10\_300\_200 \textcolor{red}{\textcjheb{r+syy}} JJSR $|$wird gerade machen/er wird ebnen\\
6.&48.&428.&174.&1750.&21.&6&639&1\_200\_8\_400\_10\_20 \textcolor{red}{\textcjheb{kyt.hr'}} ARCTJK $|$deine Pfade\\
\end{tabular}\medskip \\
Ende des Verses 3.6\\
Verse: 61, Buchstaben: 26, 179, 1755, Totalwerte: 1568, 11390, 127606\\
\\
Erkenne ihn auf allen deinen Wegen, und er wird gerade machen deine Pfade. -\\
\newpage 
{\bf -- 3.7}\\
\medskip \\
\begin{tabular}{rrrrrrrrp{120mm}}
WV&WK&WB&ABK&ABB&ABV&AnzB&TW&Zahlencode \textcolor{red}{$\boldsymbol{Grundtext}$} Umschrift $|$"Ubersetzung(en)\\
1.&49.&429.&180.&1756.&1.&2&31&1\_30 \textcolor{red}{\textcjheb{l'}} AL $|$nicht\\
2.&50.&430.&182.&1758.&3.&3&415&400\_5\_10 \textcolor{red}{\textcjheb{yht}} THJ $|$sei/du sollst sein\\
3.&51.&431.&185.&1761.&6.&3&68&8\_20\_40 \textcolor{red}{\textcjheb{mk.h}} CKM $|$weise\\
4.&52.&432.&188.&1764.&9.&6&162&2\_70\_10\_50\_10\_20 \textcolor{red}{\textcjheb{kyny`b}} BaJNJK $|$in deinen Augen\\
5.&53.&433.&194.&1770.&15.&3&211&10\_200\_1 \textcolor{red}{\textcjheb{'ry}} JRA $|$f"urchte\\
6.&54.&434.&197.&1773.&18.&2&401&1\_400 \textcolor{red}{\textcjheb{t'}} AT $|$**\\
7.&55.&435.&199.&1775.&20.&4&26&10\_5\_6\_5 \textcolor{red}{\textcjheb{hwhy}} JHWH $|$Jahwe\\
8.&56.&436.&203.&1779.&24.&4&272&6\_60\_6\_200 \textcolor{red}{\textcjheb{rwsw}} WsWR $|$und weiche\\
9.&57.&437.&207.&1783.&28.&3&310&40\_200\_70 \textcolor{red}{\textcjheb{`rm}} MRa $|$vom B"osen\\
\end{tabular}\medskip \\
Ende des Verses 3.7\\
Verse: 62, Buchstaben: 30, 209, 1785, Totalwerte: 1896, 13286, 129502\\
\\
Sei nicht weise in deinen Augen, f"urchte Jahwe und weiche vom B"osen:\\
\newpage 
{\bf -- 3.8}\\
\medskip \\
\begin{tabular}{rrrrrrrrp{120mm}}
WV&WK&WB&ABK&ABB&ABV&AnzB&TW&Zahlencode \textcolor{red}{$\boldsymbol{Grundtext}$} Umschrift $|$"Ubersetzung(en)\\
1.&58.&438.&210.&1786.&1.&5&687&200\_80\_1\_6\_400 \textcolor{red}{\textcjheb{tw'pr}} RPAWT $|$Heilung\\
2.&59.&439.&215.&1791.&6.&3&415&400\_5\_10 \textcolor{red}{\textcjheb{yht}} THJ $|$es wird sein/(sie) sei\\
3.&60.&440.&218.&1794.&9.&4&550&30\_300\_200\_20 \textcolor{red}{\textcjheb{kr+sl}} LSRK $|$f"ur deinen Nabel\\
4.&61.&441.&222.&1798.&13.&5&422&6\_300\_100\_6\_10 \textcolor{red}{\textcjheb{ywq+sw}} WSQWJ $|$und Saft/und ein Trank\\
5.&62.&442.&227.&1803.&18.&8&666&30\_70\_90\_40\_6\_400\_10\_20 \textcolor{red}{\textcjheb{kytwm.s`l}} La"sMWTJK $|$f"ur deine Gebeine\\
\end{tabular}\medskip \\
Ende des Verses 3.8\\
Verse: 63, Buchstaben: 25, 234, 1810, Totalwerte: 2740, 16026, 132242\\
\\
es wird Heilung sein f"ur deinen Nabel und Saft f"ur deine Gebeine. -\\
\newpage 
{\bf -- 3.9}\\
\medskip \\
\begin{tabular}{rrrrrrrrp{120mm}}
WV&WK&WB&ABK&ABB&ABV&AnzB&TW&Zahlencode \textcolor{red}{$\boldsymbol{Grundtext}$} Umschrift $|$"Ubersetzung(en)\\
1.&63.&443.&235.&1811.&1.&3&26&20\_2\_4 \textcolor{red}{\textcjheb{dbk}} KBD $|$ehre\\
2.&64.&444.&238.&1814.&4.&2&401&1\_400 \textcolor{red}{\textcjheb{t'}} AT $|$**\\
3.&65.&445.&240.&1816.&6.&4&26&10\_5\_6\_5 \textcolor{red}{\textcjheb{hwhy}} JHWH $|$Jahwe\\
4.&66.&446.&244.&1820.&10.&5&121&40\_5\_6\_50\_20 \textcolor{red}{\textcjheb{knwhm}} MHWNK $|$von deinem Verm"ogen/mit deiner Habe\\
5.&67.&447.&249.&1825.&15.&7&957&6\_40\_200\_1\_300\_10\_400 \textcolor{red}{\textcjheb{ty+s'rmw}} WMRASJT $|$und von den Erstlingen/und mit den Erstlingen\\
6.&68.&448.&256.&1832.&22.&2&50&20\_30 \textcolor{red}{\textcjheb{lk}} KL $|$all\\
7.&69.&449.&258.&1834.&24.&6&829&400\_2\_6\_1\_400\_20 \textcolor{red}{\textcjheb{kt'wbt}} TBWATK $|$deines Ertrags\\
\end{tabular}\medskip \\
Ende des Verses 3.9\\
Verse: 64, Buchstaben: 29, 263, 1839, Totalwerte: 2410, 18436, 134652\\
\\
Ehre Jahwe von deinem Verm"ogen und von den Erstlingen all deines Ertrages;\\
\newpage 
{\bf -- 3.10}\\
\medskip \\
\begin{tabular}{rrrrrrrrp{120mm}}
WV&WK&WB&ABK&ABB&ABV&AnzB&TW&Zahlencode \textcolor{red}{$\boldsymbol{Grundtext}$} Umschrift $|$"Ubersetzung(en)\\
1.&70.&450.&264.&1840.&1.&6&93&6\_10\_40\_30\_1\_6 \textcolor{red}{\textcjheb{w'lmyw}} WJMLAW $|$so werden sich f"ullen/und sie (=es) werden gef"ullt\\
2.&71.&451.&270.&1846.&7.&5&131&1\_60\_40\_10\_20 \textcolor{red}{\textcjheb{kyms'}} AsMJK $|$deine Speicher\\
3.&72.&452.&275.&1851.&12.&3&372&300\_2\_70 \textcolor{red}{\textcjheb{`b+s}} SBa $|$mit "Uberfluss/(in) S"attigung\\
4.&73.&453.&278.&1854.&15.&6&922&6\_400\_10\_200\_6\_300 \textcolor{red}{\textcjheb{+swrytw}} WTJRWS $|$und (von) Most\\
5.&74.&454.&284.&1860.&21.&5&142&10\_100\_2\_10\_20 \textcolor{red}{\textcjheb{kybqy}} JQBJK $|$deine Kufen\\
6.&75.&455.&289.&1865.&26.&5&386&10\_80\_200\_90\_6 \textcolor{red}{\textcjheb{w.srpy}} JPR"sW $|$"uberflie"sen/(sie) laufen "uber\\
\end{tabular}\medskip \\
Ende des Verses 3.10\\
Verse: 65, Buchstaben: 30, 293, 1869, Totalwerte: 2046, 20482, 136698\\
\\
so werden deine Speicher sich f"ullen mit "Uberflu"s, und deine Kufen von Most "uberflie"sen.\\
\newpage 
{\bf -- 3.11}\\
\medskip \\
\begin{tabular}{rrrrrrrrp{120mm}}
WV&WK&WB&ABK&ABB&ABV&AnzB&TW&Zahlencode \textcolor{red}{$\boldsymbol{Grundtext}$} Umschrift $|$"Ubersetzung(en)\\
1.&76.&456.&294.&1870.&1.&4&306&40\_6\_60\_200 \textcolor{red}{\textcjheb{rswm}} MWsR $|$die Unterweisung/(eine) Z"uchtigung\\
2.&77.&457.&298.&1874.&5.&4&26&10\_5\_6\_5 \textcolor{red}{\textcjheb{hwhy}} JHWH $|$Jahwe(s)\\
3.&78.&458.&302.&1878.&9.&3&62&2\_50\_10 \textcolor{red}{\textcjheb{ynb}} BNJ $|$mein Sohn\\
4.&79.&459.&305.&1881.&12.&2&31&1\_30 \textcolor{red}{\textcjheb{l'}} AL $|$nicht\\
5.&80.&460.&307.&1883.&14.&4&501&400\_40\_1\_60 \textcolor{red}{\textcjheb{s'mt}} TMAs $|$verwirf/sollst du verachten\\
6.&81.&461.&311.&1887.&18.&3&37&6\_1\_30 \textcolor{red}{\textcjheb{l'w}} WAL $|$und nicht\\
7.&82.&462.&314.&1890.&21.&3&590&400\_100\_90 \textcolor{red}{\textcjheb{.sqt}} TQ"s $|$lass verdrie"sen dich/sie (=es) soll ekeln dich\\
8.&83.&463.&317.&1893.&24.&7&842&2\_400\_6\_20\_8\_400\_6 \textcolor{red}{\textcjheb{wt.hkwtb}} BTWKCTW $|$seine Zucht/bei seiner Zurechtweisung\\
\end{tabular}\medskip \\
Ende des Verses 3.11\\
Verse: 66, Buchstaben: 30, 323, 1899, Totalwerte: 2395, 22877, 139093\\
\\
Mein Sohn, verwirf nicht die Unterweisung Jahwes, und la"s dich seine Zucht nicht verdrie"sen.\\
\newpage 
{\bf -- 3.12}\\
\medskip \\
\begin{tabular}{rrrrrrrrp{120mm}}
WV&WK&WB&ABK&ABB&ABV&AnzB&TW&Zahlencode \textcolor{red}{$\boldsymbol{Grundtext}$} Umschrift $|$"Ubersetzung(en)\\
1.&84.&464.&324.&1900.&1.&2&30&20\_10 \textcolor{red}{\textcjheb{yk}} KJ $|$denn\\
2.&85.&465.&326.&1902.&3.&2&401&1\_400 \textcolor{red}{\textcjheb{t'}} AT $|$**\\
3.&86.&466.&328.&1904.&5.&3&501&1\_300\_200 \textcolor{red}{\textcjheb{r+s'}} ASR $|$wen\\
4.&87.&467.&331.&1907.&8.&4&18&10\_1\_5\_2 \textcolor{red}{\textcjheb{bh'y}} JAHB $|$(er (=es)) liebt\\
5.&88.&468.&335.&1911.&12.&4&26&10\_5\_6\_5 \textcolor{red}{\textcjheb{hwhy}} JHWH $|$Jahwe\\
6.&89.&469.&339.&1915.&16.&5&54&10\_6\_20\_10\_8 \textcolor{red}{\textcjheb{.hykwy}} JWKJC $|$den z"uchtigt er/er weist zurecht\\
7.&90.&470.&344.&1920.&21.&4&29&6\_20\_1\_2 \textcolor{red}{\textcjheb{b'kw}} WKAB $|$und (zwar) wie ein Vater\\
8.&91.&471.&348.&1924.&25.&2&401&1\_400 \textcolor{red}{\textcjheb{t'}} AT $|$**\\
9.&92.&472.&350.&1926.&27.&2&52&2\_50 \textcolor{red}{\textcjheb{nb}} BN $|$den Sohn\\
10.&93.&473.&352.&1928.&29.&4&305&10\_200\_90\_5 \textcolor{red}{\textcjheb{h.sry}} JR"sH $|$(an dem) er (Wohl)Gefallen hat\\
\end{tabular}\medskip \\
Ende des Verses 3.12\\
Verse: 67, Buchstaben: 32, 355, 1931, Totalwerte: 1817, 24694, 140910\\
\\
Denn wen Jahwe liebt, den z"uchtigt er, und zwar wie ein Vater den Sohn, an dem er Wohlgefallen hat. -\\
\newpage 
{\bf -- 3.13}\\
\medskip \\
\begin{tabular}{rrrrrrrrp{120mm}}
WV&WK&WB&ABK&ABB&ABV&AnzB&TW&Zahlencode \textcolor{red}{$\boldsymbol{Grundtext}$} Umschrift $|$"Ubersetzung(en)\\
1.&94.&474.&356.&1932.&1.&4&511&1\_300\_200\_10 \textcolor{red}{\textcjheb{yr+s'}} ASRJ $|$gl"uckselig/Seligkeiten\\
2.&95.&475.&360.&1936.&5.&3&45&1\_4\_40 \textcolor{red}{\textcjheb{md'}} ADM $|$der Mensch/(dem) Menschen\\
3.&96.&476.&363.&1939.&8.&3&131&40\_90\_1 \textcolor{red}{\textcjheb{'.sm}} M"sA $|$(d)er hat gefunden\\
4.&97.&477.&366.&1942.&11.&4&73&8\_20\_40\_5 \textcolor{red}{\textcjheb{hmk.h}} CKMH $|$Weisheit\\
5.&98.&478.&370.&1946.&15.&4&51&6\_1\_4\_40 \textcolor{red}{\textcjheb{md'w}} WADM $|$und der Mensch/und dem Menschen\\
6.&99.&479.&374.&1950.&19.&4&200&10\_80\_10\_100 \textcolor{red}{\textcjheb{qypy}} JPJQ $|$(d)er erlangt\\
7.&100.&480.&378.&1954.&23.&5&463&400\_2\_6\_50\_5 \textcolor{red}{\textcjheb{hnwbt}} TBWNH $|$Verst"andnis/Einsicht\\
\end{tabular}\medskip \\
Ende des Verses 3.13\\
Verse: 68, Buchstaben: 27, 382, 1958, Totalwerte: 1474, 26168, 142384\\
\\
Gl"uckselig der Mensch, der Weisheit gefunden hat, und der Mensch, der Verst"andnis erlangt!\\
\newpage 
{\bf -- 3.14}\\
\medskip \\
\begin{tabular}{rrrrrrrrp{120mm}}
WV&WK&WB&ABK&ABB&ABV&AnzB&TW&Zahlencode \textcolor{red}{$\boldsymbol{Grundtext}$} Umschrift $|$"Ubersetzung(en)\\
1.&101.&481.&383.&1959.&1.&2&30&20\_10 \textcolor{red}{\textcjheb{yk}} KJ $|$denn\\
2.&102.&482.&385.&1961.&3.&3&17&9\_6\_2 \textcolor{red}{\textcjheb{bw.t}} tWB $|$besser ist/gut (ist)\\
3.&103.&483.&388.&1964.&6.&4&273&60\_8\_200\_5 \textcolor{red}{\textcjheb{hr.hs}} sCRH $|$ihr Erwerb\\
4.&104.&484.&392.&1968.&10.&4&308&40\_60\_8\_200 \textcolor{red}{\textcjheb{r.hsm}} MsCR $|$als der Erwerb/mehr als (ein) Erwerb\\
5.&105.&485.&396.&1972.&14.&3&160&20\_60\_80 \textcolor{red}{\textcjheb{psk}} KsP $|$(von) Silber\\
6.&106.&486.&399.&1975.&17.&6&350&6\_40\_8\_200\_6\_90 \textcolor{red}{\textcjheb{.swr.hmw}} WMCRW"s $|$und besser als feines Gold/und mehr als Gold\\
7.&107.&487.&405.&1981.&23.&6&814&400\_2\_6\_1\_400\_5 \textcolor{red}{\textcjheb{ht'wbt}} TBWATH $|$ihr Gewinn/ihr Ertrag\\
\end{tabular}\medskip \\
Ende des Verses 3.14\\
Verse: 69, Buchstaben: 28, 410, 1986, Totalwerte: 1952, 28120, 144336\\
\\
Denn ihr Erwerb ist besser als der Erwerb von Silber, und ihr Gewinn besser als feines Gold;\\
\newpage 
{\bf -- 3.15}\\
\medskip \\
\begin{tabular}{rrrrrrrrp{120mm}}
WV&WK&WB&ABK&ABB&ABV&AnzB&TW&Zahlencode \textcolor{red}{$\boldsymbol{Grundtext}$} Umschrift $|$"Ubersetzung(en)\\
1.&108.&488.&411.&1987.&1.&4&315&10\_100\_200\_5 \textcolor{red}{\textcjheb{hrqy}} JQRH $|$kostbar(er) (ist)\\
2.&109.&489.&415.&1991.&5.&3&16&5\_10\_1 \textcolor{red}{\textcjheb{'yh}} HJA $|$sie\\
3.&110.&490.&418.&1994.&8.&6&230&40\_80\_50\_10\_10\_40 \textcolor{red}{\textcjheb{myynpm}} MPNJJM $|$(mehr) als Korallen\\
4.&111.&491.&424.&2000.&14.&3&56&6\_20\_30 \textcolor{red}{\textcjheb{lkw}} WKL $|$und alle(s) (was du begehren magst)\\
5.&112.&492.&427.&2003.&17.&5&208&8\_80\_90\_10\_20 \textcolor{red}{\textcjheb{ky.sp.h}} CP"sJK $|$an Wert/deine Kostbarkeiten\\
6.&113.&493.&432.&2008.&22.&2&31&30\_1 \textcolor{red}{\textcjheb{'l}} LA $|$nicht\\
7.&114.&494.&434.&2010.&24.&4&322&10\_300\_6\_6 \textcolor{red}{\textcjheb{ww+sy}} JSWW $|$kommt gleich/(sie) gleichen\\
8.&115.&495.&438.&2014.&28.&2&7&2\_5 \textcolor{red}{\textcjheb{hb}} BH $|$ihr\\
\end{tabular}\medskip \\
Ende des Verses 3.15\\
Verse: 70, Buchstaben: 29, 439, 2015, Totalwerte: 1185, 29305, 145521\\
\\
kostbarer ist sie als Korallen, und alles, was du begehren magst, kommt ihr an Wert nicht gleich.\\
\newpage 
{\bf -- 3.16}\\
\medskip \\
\begin{tabular}{rrrrrrrrp{120mm}}
WV&WK&WB&ABK&ABB&ABV&AnzB&TW&Zahlencode \textcolor{red}{$\boldsymbol{Grundtext}$} Umschrift $|$"Ubersetzung(en)\\
1.&116.&496.&440.&2016.&1.&3&221&1\_200\_20 \textcolor{red}{\textcjheb{kr'}} ARK $|$(die) L"ange\\
2.&117.&497.&443.&2019.&4.&4&100&10\_40\_10\_40 \textcolor{red}{\textcjheb{mymy}} JMJM $|$des Lebens/der Tage\\
3.&118.&498.&447.&2023.&8.&6&117&2\_10\_40\_10\_50\_5 \textcolor{red}{\textcjheb{hnymyb}} BJMJNH $|$ist in ihrer Rechten\\
4.&119.&499.&453.&2029.&14.&7&384&2\_300\_40\_1\_6\_30\_5 \textcolor{red}{\textcjheb{hlw'm+sb}} BSMAWLH $|$(und) in ihrer Linken\\
5.&120.&500.&460.&2036.&21.&3&570&70\_300\_200 \textcolor{red}{\textcjheb{r+s`}} aSR $|$Reichtum\\
6.&121.&501.&463.&2039.&24.&5&38&6\_20\_2\_6\_4 \textcolor{red}{\textcjheb{dwbkw}} WKBWD $|$und Ehre\\
\end{tabular}\medskip \\
Ende des Verses 3.16\\
Verse: 71, Buchstaben: 28, 467, 2043, Totalwerte: 1430, 30735, 146951\\
\\
L"ange des Lebens ist in ihrer Rechten, in ihrer Linken Reichtum und Ehre.\\
\newpage 
{\bf -- 3.17}\\
\medskip \\
\begin{tabular}{rrrrrrrrp{120mm}}
WV&WK&WB&ABK&ABB&ABV&AnzB&TW&Zahlencode \textcolor{red}{$\boldsymbol{Grundtext}$} Umschrift $|$"Ubersetzung(en)\\
1.&122.&502.&468.&2044.&1.&5&239&4\_200\_20\_10\_5 \textcolor{red}{\textcjheb{hykrd}} DRKJH $|$ihre Wege\\
2.&123.&503.&473.&2049.&6.&4&234&4\_200\_20\_10 \textcolor{red}{\textcjheb{ykrd}} DRKJ $|$(sind) (die) Wege\\
3.&124.&504.&477.&2053.&10.&3&160&50\_70\_40 \textcolor{red}{\textcjheb{m`n}} NaM $|$liebliche/der Annehmlichkeit\\
4.&125.&505.&480.&2056.&13.&3&56&6\_20\_30 \textcolor{red}{\textcjheb{lkw}} WKL $|$und alle\\
5.&126.&506.&483.&2059.&16.&8&883&50\_400\_10\_2\_6\_400\_10\_5 \textcolor{red}{\textcjheb{hytwbytn}} NTJBWTJH $|$ihre Pfade\\
6.&127.&507.&491.&2067.&24.&4&376&300\_30\_6\_40 \textcolor{red}{\textcjheb{mwl+s}} SLWM $|$(sind) Friede(n)\\
\end{tabular}\medskip \\
Ende des Verses 3.17\\
Verse: 72, Buchstaben: 27, 494, 2070, Totalwerte: 1948, 32683, 148899\\
\\
Ihre Wege sind liebliche Wege, und alle ihre Pfade sind Frieden.\\
\newpage 
{\bf -- 3.18}\\
\medskip \\
\begin{tabular}{rrrrrrrrp{120mm}}
WV&WK&WB&ABK&ABB&ABV&AnzB&TW&Zahlencode \textcolor{red}{$\boldsymbol{Grundtext}$} Umschrift $|$"Ubersetzung(en)\\
1.&128.&508.&495.&2071.&1.&2&160&70\_90 \textcolor{red}{\textcjheb{.s`}} a"s $|$(ein) Baum\\
2.&129.&509.&497.&2073.&3.&4&68&8\_10\_10\_40 \textcolor{red}{\textcjheb{myy.h}} CJJM $|$des Lebens\\
3.&130.&510.&501.&2077.&7.&3&16&5\_10\_1 \textcolor{red}{\textcjheb{'yh}} HJA $|$(ist) sie\\
4.&131.&511.&504.&2080.&10.&8&245&30\_40\_8\_7\_10\_100\_10\_40 \textcolor{red}{\textcjheb{myqyz.hml}} LMCZJQJM $|$denen die ergreifen/den Festhaltenden\\
5.&132.&512.&512.&2088.&18.&2&7&2\_5 \textcolor{red}{\textcjheb{hb}} BH $|$sie/an ihr\\
6.&133.&513.&514.&2090.&20.&6&481&6\_400\_40\_20\_10\_5 \textcolor{red}{\textcjheb{hykmtw}} WTMKJH $|$und wer sie festh"alt/und ihre Ergreifenden\\
7.&134.&514.&520.&2096.&26.&4&541&40\_1\_300\_200 \textcolor{red}{\textcjheb{r+s'm}} MASR $|$ist gl"uckselig/werden gl"ucklich gepriesen\\
\end{tabular}\medskip \\
Ende des Verses 3.18\\
Verse: 73, Buchstaben: 29, 523, 2099, Totalwerte: 1518, 34201, 150417\\
\\
Ein Baum des Lebens ist sie denen, die sie ergreifen, und wer sie festh"alt, ist gl"uckselig.\\
\newpage 
{\bf -- 3.19}\\
\medskip \\
\begin{tabular}{rrrrrrrrp{120mm}}
WV&WK&WB&ABK&ABB&ABV&AnzB&TW&Zahlencode \textcolor{red}{$\boldsymbol{Grundtext}$} Umschrift $|$"Ubersetzung(en)\\
1.&135.&515.&524.&2100.&1.&4&26&10\_5\_6\_5 \textcolor{red}{\textcjheb{hwhy}} JHWH $|$Jahwe\\
2.&136.&516.&528.&2104.&5.&5&75&2\_8\_20\_40\_5 \textcolor{red}{\textcjheb{hmk.hb}} BCKMH $|$durch Weisheit/mit Weisheit\\
3.&137.&517.&533.&2109.&10.&3&74&10\_60\_4 \textcolor{red}{\textcjheb{dsy}} JsD $|$(er) hat gegr"undet\\
4.&138.&518.&536.&2112.&13.&3&291&1\_200\_90 \textcolor{red}{\textcjheb{.sr'}} AR"s $|$die Erde\\
5.&139.&519.&539.&2115.&16.&4&126&20\_6\_50\_50 \textcolor{red}{\textcjheb{nnwk}} KWNN $|$(und) festgestellt/er hat befestigt\\
6.&140.&520.&543.&2119.&20.&4&390&300\_40\_10\_40 \textcolor{red}{\textcjheb{mym+s}} SMJM $|$die Himmel\\
7.&141.&521.&547.&2123.&24.&6&465&2\_400\_2\_6\_50\_5 \textcolor{red}{\textcjheb{hnwbtb}} BTBWNH $|$durch Einsicht/in Einsicht\\
\end{tabular}\medskip \\
Ende des Verses 3.19\\
Verse: 74, Buchstaben: 29, 552, 2128, Totalwerte: 1447, 35648, 151864\\
\\
Jahwe hat durch Weisheit die Erde gegr"undet, und durch Einsicht die Himmel festgestellt.\\
\newpage 
{\bf -- 3.20}\\
\medskip \\
\begin{tabular}{rrrrrrrrp{120mm}}
WV&WK&WB&ABK&ABB&ABV&AnzB&TW&Zahlencode \textcolor{red}{$\boldsymbol{Grundtext}$} Umschrift $|$"Ubersetzung(en)\\
1.&142.&522.&553.&2129.&1.&5&482&2\_4\_70\_400\_6 \textcolor{red}{\textcjheb{wt`db}} BDaTW $|$durch seine Erkenntnis/durch sein Wissen\\
2.&143.&523.&558.&2134.&6.&6&857&400\_5\_6\_40\_6\_400 \textcolor{red}{\textcjheb{twmwht}} THWMWT $|$die Tiefen/(die) Urfluten\\
3.&144.&524.&564.&2140.&12.&5&228&50\_2\_100\_70\_6 \textcolor{red}{\textcjheb{w`qbn}} NBQaW $|$sind hervorgebrochen/sie spalteten sich\\
4.&145.&525.&569.&2145.&17.&6&464&6\_300\_8\_100\_10\_40 \textcolor{red}{\textcjheb{myq.h+sw}} WSCQJM $|$und die Wolken\\
5.&146.&526.&575.&2151.&23.&5&366&10\_200\_70\_80\_6 \textcolor{red}{\textcjheb{wp`ry}} JRaPW $|$tr"aufelten herab/(sie) tr"aufeln\\
6.&147.&527.&580.&2156.&28.&2&39&9\_30 \textcolor{red}{\textcjheb{l.t}} tL $|$Tau\\
\end{tabular}\medskip \\
Ende des Verses 3.20\\
Verse: 75, Buchstaben: 29, 581, 2157, Totalwerte: 2436, 38084, 154300\\
\\
Durch seine Erkenntnis sind hervorgebrochen die Tiefen, und die Wolken tr"aufelten Tau herab. -\\
\newpage 
{\bf -- 3.21}\\
\medskip \\
\begin{tabular}{rrrrrrrrp{120mm}}
WV&WK&WB&ABK&ABB&ABV&AnzB&TW&Zahlencode \textcolor{red}{$\boldsymbol{Grundtext}$} Umschrift $|$"Ubersetzung(en)\\
1.&148.&528.&582.&2158.&1.&3&62&2\_50\_10 \textcolor{red}{\textcjheb{ynb}} BNJ $|$mein Sohn\\
2.&149.&529.&585.&2161.&4.&2&31&1\_30 \textcolor{red}{\textcjheb{l'}} AL $|$nicht\\
3.&150.&530.&587.&2163.&6.&4&53&10\_30\_7\_6 \textcolor{red}{\textcjheb{wzly}} JLZW $|$lass weichen sie/sie m"ogen weichen\\
4.&151.&531.&591.&2167.&10.&6&200&40\_70\_10\_50\_10\_20 \textcolor{red}{\textcjheb{kyny`m}} MaJNJK $|$von deinen Augen/aus deinen Augen\\
5.&152.&532.&597.&2173.&16.&3&340&50\_90\_200 \textcolor{red}{\textcjheb{r.sn}} N"sR $|$bewahre\\
6.&153.&533.&600.&2176.&19.&4&715&400\_300\_10\_5 \textcolor{red}{\textcjheb{hy+st}} TSJH $|$klugen Rat/Umsicht\\
7.&154.&534.&604.&2180.&23.&5&98&6\_40\_7\_40\_5 \textcolor{red}{\textcjheb{hmzmw}} WMZMH $|$und Besonnenheit\\
\end{tabular}\medskip \\
Ende des Verses 3.21\\
Verse: 76, Buchstaben: 27, 608, 2184, Totalwerte: 1499, 39583, 155799\\
\\
Mein Sohn, la"s sie nicht von deinen Augen weichen, bewahre klugen Rat und Besonnenheit;\\
\newpage 
{\bf -- 3.22}\\
\medskip \\
\begin{tabular}{rrrrrrrrp{120mm}}
WV&WK&WB&ABK&ABB&ABV&AnzB&TW&Zahlencode \textcolor{red}{$\boldsymbol{Grundtext}$} Umschrift $|$"Ubersetzung(en)\\
1.&155.&535.&609.&2185.&1.&5&37&6\_10\_5\_10\_6 \textcolor{red}{\textcjheb{wyhyw}} WJHJW $|$so werden sie sein/und sie werden sein\\
2.&156.&536.&614.&2190.&6.&4&68&8\_10\_10\_40 \textcolor{red}{\textcjheb{myy.h}} CJJM $|$Leben(de)\\
3.&157.&537.&618.&2194.&10.&5&480&30\_50\_80\_300\_20 \textcolor{red}{\textcjheb{k+spnl}} LNPSK $|$f"ur deine Seele\\
4.&158.&538.&623.&2199.&15.&3&64&6\_8\_50 \textcolor{red}{\textcjheb{n.hw}} WCN $|$und Anmut\\
5.&159.&539.&626.&2202.&18.&8&866&30\_3\_200\_3\_200\_400\_10\_20 \textcolor{red}{\textcjheb{kytrgrgl}} LGRGRTJK $|$deinem Hals/f"ur deinen Hals\\
\end{tabular}\medskip \\
Ende des Verses 3.22\\
Verse: 77, Buchstaben: 25, 633, 2209, Totalwerte: 1515, 41098, 157314\\
\\
so werden sie Leben sein f"ur deine Seele und Anmut deinem Halse.\\
\newpage 
{\bf -- 3.23}\\
\medskip \\
\begin{tabular}{rrrrrrrrp{120mm}}
WV&WK&WB&ABK&ABB&ABV&AnzB&TW&Zahlencode \textcolor{red}{$\boldsymbol{Grundtext}$} Umschrift $|$"Ubersetzung(en)\\
1.&160.&540.&634.&2210.&1.&2&8&1\_7 \textcolor{red}{\textcjheb{z'}} AZ $|$dann\\
2.&161.&541.&636.&2212.&3.&3&450&400\_30\_20 \textcolor{red}{\textcjheb{klt}} TLK $|$wirst du gehen/du gehst\\
3.&162.&542.&639.&2215.&6.&4&49&30\_2\_9\_8 \textcolor{red}{\textcjheb{.h.tbl}} LBtC $|$in Sicherheit\\
4.&163.&543.&643.&2219.&10.&4&244&4\_200\_20\_20 \textcolor{red}{\textcjheb{kkrd}} DRKK $|$deinen Weg\\
5.&164.&544.&647.&2223.&14.&5&259&6\_200\_3\_30\_20 \textcolor{red}{\textcjheb{klgrw}} WRGLK $|$und dein Fu"s\\
6.&165.&545.&652.&2228.&19.&2&31&30\_1 \textcolor{red}{\textcjheb{'l}} LA $|$nicht\\
7.&166.&546.&654.&2230.&21.&4&489&400\_3\_6\_80 \textcolor{red}{\textcjheb{pwgt}} TGWP $|$wird ansto"sen/er (=es) st"o"st an\\
\end{tabular}\medskip \\
Ende des Verses 3.23\\
Verse: 78, Buchstaben: 24, 657, 2233, Totalwerte: 1530, 42628, 158844\\
\\
Dann wirst du in Sicherheit deinen Weg gehen, und dein Fu"s wird nicht ansto"sen.\\
\newpage 
{\bf -- 3.24}\\
\medskip \\
\begin{tabular}{rrrrrrrrp{120mm}}
WV&WK&WB&ABK&ABB&ABV&AnzB&TW&Zahlencode \textcolor{red}{$\boldsymbol{Grundtext}$} Umschrift $|$"Ubersetzung(en)\\
1.&167.&547.&658.&2234.&1.&2&41&1\_40 \textcolor{red}{\textcjheb{m'}} AM $|$wenn\\
2.&168.&548.&660.&2236.&3.&4&722&400\_300\_20\_2 \textcolor{red}{\textcjheb{bk+st}} TSKB $|$du dich niederlegst\\
3.&169.&549.&664.&2240.&7.&2&31&30\_1 \textcolor{red}{\textcjheb{'l}} LA $|$nicht\\
4.&170.&550.&666.&2242.&9.&4&492&400\_80\_8\_4 \textcolor{red}{\textcjheb{d.hpt}} TPCD $|$wirst du erschrecken/du brauchst zu f"urchten dich\\
5.&171.&551.&670.&2246.&13.&5&728&6\_300\_20\_2\_400 \textcolor{red}{\textcjheb{tbk+sw}} WSKBT $|$und (wenn) du liegst\\
6.&172.&552.&675.&2251.&18.&5&283&6\_70\_200\_2\_5 \textcolor{red}{\textcjheb{hbr`w}} WaRBH $|$so wird s"u"s sein/und sie (=es) ist angenehm\\
7.&173.&553.&680.&2256.&23.&4&770&300\_50\_400\_20 \textcolor{red}{\textcjheb{ktn+s}} SNTK $|$dein Schlaf\\
\end{tabular}\medskip \\
Ende des Verses 3.24\\
Verse: 79, Buchstaben: 26, 683, 2259, Totalwerte: 3067, 45695, 161911\\
\\
Wenn du dich niederlegst, wirst du nicht erschrecken; und liegst du, so wird dein Schlaf s"u"s sein.\\
\newpage 
{\bf -- 3.25}\\
\medskip \\
\begin{tabular}{rrrrrrrrp{120mm}}
WV&WK&WB&ABK&ABB&ABV&AnzB&TW&Zahlencode \textcolor{red}{$\boldsymbol{Grundtext}$} Umschrift $|$"Ubersetzung(en)\\
1.&174.&554.&684.&2260.&1.&2&31&1\_30 \textcolor{red}{\textcjheb{l'}} AL $|$nicht\\
2.&175.&555.&686.&2262.&3.&4&611&400\_10\_200\_1 \textcolor{red}{\textcjheb{'ryt}} TJRA $|$f"urchte dich/du musst f"urchten dich\\
3.&176.&556.&690.&2266.&7.&4&132&40\_80\_8\_4 \textcolor{red}{\textcjheb{d.hpm}} MPCD $|$vor Schrecken/vor Schrecknis\\
4.&177.&557.&694.&2270.&11.&4&521&80\_400\_1\_40 \textcolor{red}{\textcjheb{m'tp}} PTAM $|$pl"otzlichem\\
5.&178.&558.&698.&2274.&15.&5&747&6\_40\_300\_1\_400 \textcolor{red}{\textcjheb{t'+smw}} WMSAT $|$noch vor der Verw"ustung/und vor dem Unheil\\
6.&179.&559.&703.&2279.&20.&5&620&200\_300\_70\_10\_40 \textcolor{red}{\textcjheb{my`+sr}} RSaJM $|$der Gesetzlosen/("uber die) Frevler\\
7.&180.&560.&708.&2284.&25.&2&30&20\_10 \textcolor{red}{\textcjheb{yk}} KJ $|$wenn\\
8.&181.&561.&710.&2286.&27.&3&403&400\_2\_1 \textcolor{red}{\textcjheb{'bt}} TBA $|$sie (=es) kommt\\
\end{tabular}\medskip \\
Ende des Verses 3.25\\
Verse: 80, Buchstaben: 29, 712, 2288, Totalwerte: 3095, 48790, 165006\\
\\
F"urchte dich nicht vor pl"otzlichem Schrecken, noch vor der Verw"ustung der Gesetzlosen, wenn sie kommt;\\
\newpage 
{\bf -- 3.26}\\
\medskip \\
\begin{tabular}{rrrrrrrrp{120mm}}
WV&WK&WB&ABK&ABB&ABV&AnzB&TW&Zahlencode \textcolor{red}{$\boldsymbol{Grundtext}$} Umschrift $|$"Ubersetzung(en)\\
1.&182.&562.&713.&2289.&1.&2&30&20\_10 \textcolor{red}{\textcjheb{yk}} KJ $|$denn\\
2.&183.&563.&715.&2291.&3.&4&26&10\_5\_6\_5 \textcolor{red}{\textcjheb{hwhy}} JHWH $|$Jahwe\\
3.&184.&564.&719.&2295.&7.&4&30&10\_5\_10\_5 \textcolor{red}{\textcjheb{hyhy}} JHJH $|$(er) wird sein\\
4.&185.&565.&723.&2299.&11.&5&132&2\_20\_60\_30\_20 \textcolor{red}{\textcjheb{klskb}} BKsLK $|$(als) deine Zuversicht\\
5.&186.&566.&728.&2304.&16.&4&546&6\_300\_40\_200 \textcolor{red}{\textcjheb{rm+sw}} WSMR $|$und (er) wird bewahren\\
6.&187.&567.&732.&2308.&20.&4&253&200\_3\_30\_20 \textcolor{red}{\textcjheb{klgr}} RGLK $|$deinen Fu"s\\
7.&188.&568.&736.&2312.&24.&4&94&40\_30\_20\_4 \textcolor{red}{\textcjheb{dklm}} MLKD $|$vor dem Fang/vor der Schlinge\\
\end{tabular}\medskip \\
Ende des Verses 3.26\\
Verse: 81, Buchstaben: 27, 739, 2315, Totalwerte: 1111, 49901, 166117\\
\\
denn Jahwe wird deine Zuversicht sein, und wird deinen Fu"s vor dem Fange bewahren.\\
\newpage 
{\bf -- 3.27}\\
\medskip \\
\begin{tabular}{rrrrrrrrp{120mm}}
WV&WK&WB&ABK&ABB&ABV&AnzB&TW&Zahlencode \textcolor{red}{$\boldsymbol{Grundtext}$} Umschrift $|$"Ubersetzung(en)\\
1.&189.&569.&740.&2316.&1.&2&31&1\_30 \textcolor{red}{\textcjheb{l'}} AL $|$nicht\\
2.&190.&570.&742.&2318.&3.&4&560&400\_40\_50\_70 \textcolor{red}{\textcjheb{`nmt}} TMNa $|$enthalte vor/du sollst versagen\\
3.&191.&571.&746.&2322.&7.&3&17&9\_6\_2 \textcolor{red}{\textcjheb{bw.t}} tWB $|$Gutes\\
4.&192.&572.&749.&2325.&10.&6&158&40\_2\_70\_30\_10\_6 \textcolor{red}{\textcjheb{wyl`bm}} MBaLJW $|$welchem es zukommt/vor dem sie geb"uhrt\\
5.&193.&573.&755.&2331.&16.&5&423&2\_5\_10\_6\_400 \textcolor{red}{\textcjheb{twyhb}} BHJWT $|$wenn es steht/beim Sein\\
6.&194.&574.&760.&2336.&21.&3&61&30\_1\_30 \textcolor{red}{\textcjheb{l'l}} LAL $|$in (der) Macht\\
7.&195.&575.&763.&2339.&24.&4&44&10\_4\_10\_20 \textcolor{red}{\textcjheb{kydy}} JDJK $|$deiner Hand/deiner H"ande\\
8.&196.&576.&767.&2343.&28.&5&806&30\_70\_300\_6\_400 \textcolor{red}{\textcjheb{tw+s`l}} LaSWT $|$sie (=es) zu tun\\
\end{tabular}\medskip \\
Ende des Verses 3.27\\
Verse: 82, Buchstaben: 32, 771, 2347, Totalwerte: 2100, 52001, 168217\\
\\
Enthalte kein Gutes dem vor, welchem es zukommt, wenn es in der Macht deiner Hand steht, es zu tun. -\\
\newpage 
{\bf -- 3.28}\\
\medskip \\
\begin{tabular}{rrrrrrrrp{120mm}}
WV&WK&WB&ABK&ABB&ABV&AnzB&TW&Zahlencode \textcolor{red}{$\boldsymbol{Grundtext}$} Umschrift $|$"Ubersetzung(en)\\
1.&197.&577.&772.&2348.&1.&2&31&1\_30 \textcolor{red}{\textcjheb{l'}} AL $|$nicht\\
2.&198.&578.&774.&2350.&3.&4&641&400\_1\_40\_200 \textcolor{red}{\textcjheb{rm't}} TAMR $|$sage/sollst du sagen\\
3.&199.&579.&778.&2354.&7.&5&330&30\_200\_70\_10\_20 \textcolor{red}{\textcjheb{ky`rl}} LRaJK $|$zu deinem N"achsten/deinem Gef"ahrten\\
4.&200.&580.&783.&2359.&12.&2&50&30\_20 \textcolor{red}{\textcjheb{kl}} LK $|$geh (hin)\\
5.&201.&581.&785.&2361.&14.&4&314&6\_300\_6\_2 \textcolor{red}{\textcjheb{bw+sw}} WSWB $|$und komm wieder\\
6.&202.&582.&789.&2365.&18.&4&254&6\_40\_8\_200 \textcolor{red}{\textcjheb{r.hmw}} WMCR $|$und morgen\\
7.&203.&583.&793.&2369.&22.&3&451&1\_400\_50 \textcolor{red}{\textcjheb{nt'}} ATN $|$will ich geben\\
8.&204.&584.&796.&2372.&25.&3&316&6\_10\_300 \textcolor{red}{\textcjheb{+syw}} WJS $|$da es doch ist/und es ist\\
9.&205.&585.&799.&2375.&28.&3&421&1\_400\_20 \textcolor{red}{\textcjheb{kt'}} ATK $|$bei dir\\
\end{tabular}\medskip \\
Ende des Verses 3.28\\
Verse: 83, Buchstaben: 30, 801, 2377, Totalwerte: 2808, 54809, 171025\\
\\
Sage nicht zu deinem N"achsten: Geh hin und komm wieder, und morgen will ich geben! -da es doch bei dir ist. -\\
\newpage 
{\bf -- 3.29}\\
\medskip \\
\begin{tabular}{rrrrrrrrp{120mm}}
WV&WK&WB&ABK&ABB&ABV&AnzB&TW&Zahlencode \textcolor{red}{$\boldsymbol{Grundtext}$} Umschrift $|$"Ubersetzung(en)\\
1.&206.&586.&802.&2378.&1.&2&31&1\_30 \textcolor{red}{\textcjheb{l'}} AL $|$nicht\\
2.&207.&587.&804.&2380.&3.&4&908&400\_8\_200\_300 \textcolor{red}{\textcjheb{+sr.ht}} TCRS $|$schmiede/sollst du bereiten\\
3.&208.&588.&808.&2384.&7.&2&100&70\_30 \textcolor{red}{\textcjheb{l`}} aL $|$gegen\\
4.&209.&589.&810.&2386.&9.&3&290&200\_70\_20 \textcolor{red}{\textcjheb{k`r}} RaK $|$deinen N"achsten/deinen Gef"ahrten\\
5.&210.&590.&813.&2389.&12.&3&275&200\_70\_5 \textcolor{red}{\textcjheb{h`r}} RaH $|$B"oses\\
6.&211.&591.&816.&2392.&15.&4&18&6\_5\_6\_1 \textcolor{red}{\textcjheb{'whw}} WHWA $|$w"ahrend er/und er\\
7.&212.&592.&820.&2396.&19.&4&318&10\_6\_300\_2 \textcolor{red}{\textcjheb{b+swy}} JWSB $|$wohnt/(ist) wohnend(er)\\
8.&213.&593.&824.&2400.&23.&4&49&30\_2\_9\_8 \textcolor{red}{\textcjheb{.h.tbl}} LBtC $|$vertrauensvoll/in Sorglosigkeit\\
9.&214.&594.&828.&2404.&27.&3&421&1\_400\_20 \textcolor{red}{\textcjheb{kt'}} ATK $|$bei dir\\
\end{tabular}\medskip \\
Ende des Verses 3.29\\
Verse: 84, Buchstaben: 29, 830, 2406, Totalwerte: 2410, 57219, 173435\\
\\
Schmiede nichts B"oses wider deinen N"achsten, w"ahrend er vertrauensvoll bei dir wohnt. -\\
\newpage 
{\bf -- 3.30}\\
\medskip \\
\begin{tabular}{rrrrrrrrp{120mm}}
WV&WK&WB&ABK&ABB&ABV&AnzB&TW&Zahlencode \textcolor{red}{$\boldsymbol{Grundtext}$} Umschrift $|$"Ubersetzung(en)\\
1.&215.&595.&831.&2407.&1.&2&31&1\_30 \textcolor{red}{\textcjheb{l'}} AL $|$nicht\\
2.&216.&596.&833.&2409.&3.&4&608&400\_200\_6\_2 \textcolor{red}{\textcjheb{bwrt}} TRWB $|$hadere/du sollst zanken\\
3.&217.&597.&837.&2413.&7.&2&110&70\_40 \textcolor{red}{\textcjheb{m`}} aM $|$mit\\
4.&218.&598.&839.&2415.&9.&3&45&1\_4\_40 \textcolor{red}{\textcjheb{md'}} ADM $|$(einem) Menschen\\
5.&219.&599.&842.&2418.&12.&3&98&8\_50\_40 \textcolor{red}{\textcjheb{mn.h}} CNM $|$ohne Ursache/grundlos\\
6.&220.&600.&845.&2421.&15.&2&41&1\_40 \textcolor{red}{\textcjheb{m'}} AM $|$wenn\\
7.&221.&601.&847.&2423.&17.&2&31&30\_1 \textcolor{red}{\textcjheb{'l}} LA $|$nicht(s)\\
8.&222.&602.&849.&2425.&19.&4&93&3\_40\_30\_20 \textcolor{red}{\textcjheb{klmg}} GMLK $|$er angetan hat dir/er zugef"ugt hat dir\\
9.&223.&603.&853.&2429.&23.&3&275&200\_70\_5 \textcolor{red}{\textcjheb{h`r}} RaH $|$B"oses\\
\end{tabular}\medskip \\
Ende des Verses 3.30\\
Verse: 85, Buchstaben: 25, 855, 2431, Totalwerte: 1332, 58551, 174767\\
\\
Hadere nicht mit einem Menschen ohne Ursache, wenn er dir nichts B"oses angetan hat. -\\
\newpage 
{\bf -- 3.31}\\
\medskip \\
\begin{tabular}{rrrrrrrrp{120mm}}
WV&WK&WB&ABK&ABB&ABV&AnzB&TW&Zahlencode \textcolor{red}{$\boldsymbol{Grundtext}$} Umschrift $|$"Ubersetzung(en)\\
1.&224.&604.&856.&2432.&1.&2&31&1\_30 \textcolor{red}{\textcjheb{l'}} AL $|$nicht\\
2.&225.&605.&858.&2434.&3.&4&551&400\_100\_50\_1 \textcolor{red}{\textcjheb{'nqt}} TQNA $|$beneide/du sollst beneiden\\
3.&226.&606.&862.&2438.&7.&4&313&2\_1\_10\_300 \textcolor{red}{\textcjheb{+sy'b}} BAJS $|$den Mann\\
4.&227.&607.&866.&2442.&11.&3&108&8\_40\_60 \textcolor{red}{\textcjheb{sm.h}} CMs $|$(der) Gewalttat\\
5.&228.&608.&869.&2445.&14.&3&37&6\_1\_30 \textcolor{red}{\textcjheb{l'w}} WAL $|$und nicht\\
6.&229.&609.&872.&2448.&17.&4&610&400\_2\_8\_200 \textcolor{red}{\textcjheb{r.hbt}} TBCR $|$erw"ahle/du sollst Gefallen haben\\
7.&230.&610.&876.&2452.&21.&3&52&2\_20\_30 \textcolor{red}{\textcjheb{lkb}} BKL $|$einen/an all\\
8.&231.&611.&879.&2455.&24.&5&240&4\_200\_20\_10\_6 \textcolor{red}{\textcjheb{wykrd}} DRKJW $|$(von) seinen Wegen\\
\end{tabular}\medskip \\
Ende des Verses 3.31\\
Verse: 86, Buchstaben: 28, 883, 2459, Totalwerte: 1942, 60493, 176709\\
\\
Beneide nicht den Mann der Gewalttat, und erw"ahle keinen von seinen Wegen. -\\
\newpage 
{\bf -- 3.32}\\
\medskip \\
\begin{tabular}{rrrrrrrrp{120mm}}
WV&WK&WB&ABK&ABB&ABV&AnzB&TW&Zahlencode \textcolor{red}{$\boldsymbol{Grundtext}$} Umschrift $|$"Ubersetzung(en)\\
1.&232.&612.&884.&2460.&1.&2&30&20\_10 \textcolor{red}{\textcjheb{yk}} KJ $|$denn\\
2.&233.&613.&886.&2462.&3.&5&878&400\_6\_70\_2\_400 \textcolor{red}{\textcjheb{tb`wt}} TWaBT $|$(ein) Gr"auel\\
3.&234.&614.&891.&2467.&8.&4&26&10\_5\_6\_5 \textcolor{red}{\textcjheb{hwhy}} JHWH $|$(f"ur) Jahwe\\
4.&235.&615.&895.&2471.&12.&4&93&50\_30\_6\_7 \textcolor{red}{\textcjheb{zwln}} NLWZ $|$ist der Verkehrte/ist ein Verkehrter\\
5.&236.&616.&899.&2475.&16.&3&407&6\_1\_400 \textcolor{red}{\textcjheb{t'w}} WAT $|$aber bei/und mit\\
6.&237.&617.&902.&2478.&19.&5&560&10\_300\_200\_10\_40 \textcolor{red}{\textcjheb{myr+sy}} JSRJM $|$den Aufrichtigen/Geraden\\
7.&238.&618.&907.&2483.&24.&4&76&60\_6\_4\_6 \textcolor{red}{\textcjheb{wdws}} sWDW $|$ist sein Geheimnis/(ist) sein Einvernehmen\\
\end{tabular}\medskip \\
Ende des Verses 3.32\\
Verse: 87, Buchstaben: 27, 910, 2486, Totalwerte: 2070, 62563, 178779\\
\\
Denn der Verkehrte ist Jahwe ein Greuel, aber sein Geheimnis ist bei den Aufrichtigen.\\
\newpage 
{\bf -- 3.33}\\
\medskip \\
\begin{tabular}{rrrrrrrrp{120mm}}
WV&WK&WB&ABK&ABB&ABV&AnzB&TW&Zahlencode \textcolor{red}{$\boldsymbol{Grundtext}$} Umschrift $|$"Ubersetzung(en)\\
1.&239.&619.&911.&2487.&1.&4&641&40\_1\_200\_400 \textcolor{red}{\textcjheb{tr'm}} MART $|$der Fluch\\
2.&240.&620.&915.&2491.&5.&4&26&10\_5\_6\_5 \textcolor{red}{\textcjheb{hwhy}} JHWH $|$Jahwe(s)\\
3.&241.&621.&919.&2495.&9.&4&414&2\_2\_10\_400 \textcolor{red}{\textcjheb{tybb}} BBJT $|$ist im Haus/(liegt) auf dem Haus\\
4.&242.&622.&923.&2499.&13.&3&570&200\_300\_70 \textcolor{red}{\textcjheb{`+sr}} RSa $|$des Gesetzlosen/(des) Frevlers\\
5.&243.&623.&926.&2502.&16.&4&67&6\_50\_6\_5 \textcolor{red}{\textcjheb{hwnw}} WNWH $|$aber die Wohnung/und die Wohnung\\
6.&244.&624.&930.&2506.&20.&6&254&90\_4\_10\_100\_10\_40 \textcolor{red}{\textcjheb{myqyd.s}} "sDJQJM $|$der Gerechten\\
7.&245.&625.&936.&2512.&26.&4&232&10\_2\_200\_20 \textcolor{red}{\textcjheb{krby}} JBRK $|$er segnet\\
\end{tabular}\medskip \\
Ende des Verses 3.33\\
Verse: 88, Buchstaben: 29, 939, 2515, Totalwerte: 2204, 64767, 180983\\
\\
Der Fluch Jahwes ist im Hause des Gesetzlosen, aber er segnet die Wohnung der Gerechten.\\
\newpage 
{\bf -- 3.34}\\
\medskip \\
\begin{tabular}{rrrrrrrrp{120mm}}
WV&WK&WB&ABK&ABB&ABV&AnzB&TW&Zahlencode \textcolor{red}{$\boldsymbol{Grundtext}$} Umschrift $|$"Ubersetzung(en)\\
1.&246.&626.&940.&2516.&1.&2&41&1\_40 \textcolor{red}{\textcjheb{m'}} AM $|$f"urwahr/wenn\\
2.&247.&627.&942.&2518.&3.&5&200&30\_30\_90\_10\_40 \textcolor{red}{\textcjheb{my.sll}} LL"sJM $|$der Sp"otter\\
3.&248.&628.&947.&2523.&8.&3&12&5\_6\_1 \textcolor{red}{\textcjheb{'wh}} HWA $|$er\\
4.&249.&629.&950.&2526.&11.&4&140&10\_30\_10\_90 \textcolor{red}{\textcjheb{.syly}} JLJ"s $|$(er) spottet\\
5.&250.&630.&954.&2530.&15.&7&216&6\_30\_70\_50\_10\_10\_40 \textcolor{red}{\textcjheb{myyn`lw}} WLaNJJM $|$aber den Dem"utigen/und den Gebeugten\\
6.&251.&631.&961.&2537.&22.&3&460&10\_400\_50 \textcolor{red}{\textcjheb{nty}} JTN $|$er gibt\\
7.&252.&632.&964.&2540.&25.&2&58&8\_50 \textcolor{red}{\textcjheb{n.h}} CN $|$Gnade\\
\end{tabular}\medskip \\
Ende des Verses 3.34\\
Verse: 89, Buchstaben: 26, 965, 2541, Totalwerte: 1127, 65894, 182110\\
\\
F"urwahr, der Sp"otter spottet er, den Dem"utigen aber gibt er Gnade.\\
\newpage 
{\bf -- 3.35}\\
\medskip \\
\begin{tabular}{rrrrrrrrp{120mm}}
WV&WK&WB&ABK&ABB&ABV&AnzB&TW&Zahlencode \textcolor{red}{$\boldsymbol{Grundtext}$} Umschrift $|$"Ubersetzung(en)\\
1.&253.&633.&966.&2542.&1.&4&32&20\_2\_6\_4 \textcolor{red}{\textcjheb{dwbk}} KBWD $|$Ehre\\
2.&254.&634.&970.&2546.&5.&5&118&8\_20\_40\_10\_40 \textcolor{red}{\textcjheb{mymk.h}} CKMJM $|$die Weisen\\
3.&255.&635.&975.&2551.&10.&5&104&10\_50\_8\_30\_6 \textcolor{red}{\textcjheb{wl.hny}} JNCLW $|$(sie werden) erben\\
4.&256.&636.&980.&2556.&15.&7&176&6\_20\_60\_10\_30\_10\_40 \textcolor{red}{\textcjheb{mylyskw}} WKsJLJM $|$aber die Toren/und die Toren\\
5.&257.&637.&987.&2563.&22.&4&290&40\_200\_10\_40 \textcolor{red}{\textcjheb{myrm}} MRJM $|$erh"oht/sind Aufrichtende\\
6.&258.&638.&991.&2567.&26.&4&186&100\_30\_6\_50 \textcolor{red}{\textcjheb{nwlq}} QLWN $|$(die) Schande\\
\end{tabular}\medskip \\
Ende des Verses 3.35\\
Verse: 90, Buchstaben: 29, 994, 2570, Totalwerte: 906, 66800, 183016\\
\\
Die Weisen erben Ehre, aber die Toren erh"oht die Schande.\\
\\
{\bf Ende des Kapitels 3}\\
\newpage 
{\bf -- 4.1}\\
\medskip \\
\begin{tabular}{rrrrrrrrp{120mm}}
WV&WK&WB&ABK&ABB&ABV&AnzB&TW&Zahlencode \textcolor{red}{$\boldsymbol{Grundtext}$} Umschrift $|$"Ubersetzung(en)\\
1.&1.&639.&1.&2571.&1.&4&416&300\_40\_70\_6 \textcolor{red}{\textcjheb{w`m+s}} SMaW $|$h"oret\\
2.&2.&640.&5.&2575.&5.&4&102&2\_50\_10\_40 \textcolor{red}{\textcjheb{mynb}} BNJM $|$S"ohne\\
3.&3.&641.&9.&2579.&9.&4&306&40\_6\_60\_200 \textcolor{red}{\textcjheb{rswm}} MWsR $|$die Unterweisung/auf die Zurechtweisung\\
4.&4.&642.&13.&2583.&13.&2&3&1\_2 \textcolor{red}{\textcjheb{b'}} AB $|$des Vaters\\
5.&5.&643.&15.&2585.&15.&7&429&6\_5\_100\_300\_10\_2\_6 \textcolor{red}{\textcjheb{wby+sqhw}} WHQSJBW $|$und merket auf\\
6.&6.&644.&22.&2592.&22.&4&504&30\_4\_70\_400 \textcolor{red}{\textcjheb{t`dl}} LDaT $|$(um) zu kennen\\
7.&7.&645.&26.&2596.&26.&4&67&2\_10\_50\_5 \textcolor{red}{\textcjheb{hnyb}} BJNH $|$Verstand/Einsicht\\
\end{tabular}\medskip \\
Ende des Verses 4.1\\
Verse: 91, Buchstaben: 29, 29, 2599, Totalwerte: 1827, 1827, 184843\\
\\
H"oret, S"ohne, die Unterweisung des Vaters, und merket auf, um Verstand zu kennen!\\
\newpage 
{\bf -- 4.2}\\
\medskip \\
\begin{tabular}{rrrrrrrrp{120mm}}
WV&WK&WB&ABK&ABB&ABV&AnzB&TW&Zahlencode \textcolor{red}{$\boldsymbol{Grundtext}$} Umschrift $|$"Ubersetzung(en)\\
1.&8.&646.&30.&2600.&1.&2&30&20\_10 \textcolor{red}{\textcjheb{yk}} KJ $|$denn\\
2.&9.&647.&32.&2602.&3.&3&138&30\_100\_8 \textcolor{red}{\textcjheb{.hql}} LQC $|$Lehre\\
3.&10.&648.&35.&2605.&6.&3&17&9\_6\_2 \textcolor{red}{\textcjheb{bw.t}} tWB $|$gute\\
4.&11.&649.&38.&2608.&9.&4&860&50\_400\_400\_10 \textcolor{red}{\textcjheb{yttn}} NTTJ $|$ich gebe\\
5.&12.&650.&42.&2612.&13.&3&90&30\_20\_40 \textcolor{red}{\textcjheb{mkl}} LKM $|$(zu) euch\\
6.&13.&651.&45.&2615.&16.&5&1016&400\_6\_200\_400\_10 \textcolor{red}{\textcjheb{ytrwt}} TWRTJ $|$meine Belehrung/meine Weisung\\
7.&14.&652.&50.&2620.&21.&2&31&1\_30 \textcolor{red}{\textcjheb{l'}} AL $|$nicht\\
8.&15.&653.&52.&2622.&23.&5&485&400\_70\_7\_2\_6 \textcolor{red}{\textcjheb{wbz`t}} TaZBW $|$verlasst/ihr sollt verlassen\\
\end{tabular}\medskip \\
Ende des Verses 4.2\\
Verse: 92, Buchstaben: 27, 56, 2626, Totalwerte: 2667, 4494, 187510\\
\\
Denn gute Lehre gebe ich euch: verlasset meine Belehrung nicht.\\
\newpage 
{\bf -- 4.3}\\
\medskip \\
\begin{tabular}{rrrrrrrrp{120mm}}
WV&WK&WB&ABK&ABB&ABV&AnzB&TW&Zahlencode \textcolor{red}{$\boldsymbol{Grundtext}$} Umschrift $|$"Ubersetzung(en)\\
1.&16.&654.&57.&2627.&1.&2&30&20\_10 \textcolor{red}{\textcjheb{yk}} KJ $|$denn\\
2.&17.&655.&59.&2629.&3.&2&52&2\_50 \textcolor{red}{\textcjheb{nb}} BN $|$(ein) Sohn\\
3.&18.&656.&61.&2631.&5.&5&435&5\_10\_10\_400\_10 \textcolor{red}{\textcjheb{ytyyh}} HJJTJ $|$bin ich gewesen/war ich\\
4.&19.&657.&66.&2636.&10.&4&43&30\_1\_2\_10 \textcolor{red}{\textcjheb{yb'l}} LABJ $|$meinem Vater/f"ur meinen Vater\\
5.&20.&658.&70.&2640.&14.&2&220&200\_20 \textcolor{red}{\textcjheb{kr}} RK $|$(ein) zart(er)\\
6.&21.&659.&72.&2642.&16.&5&38&6\_10\_8\_10\_4 \textcolor{red}{\textcjheb{dy.hyw}} WJCJD $|$und einzig(er)\\
7.&22.&660.&77.&2647.&21.&4&170&30\_80\_50\_10 \textcolor{red}{\textcjheb{ynpl}} LPNJ $|$vor\\
8.&23.&661.&81.&2651.&25.&3&51&1\_40\_10 \textcolor{red}{\textcjheb{ym'}} AMJ $|$meiner Mutter\\
\end{tabular}\medskip \\
Ende des Verses 4.3\\
Verse: 93, Buchstaben: 27, 83, 2653, Totalwerte: 1039, 5533, 188549\\
\\
Denn ein Sohn bin ich meinem Vater gewesen, ein zarter und einziger vor meiner Mutter.\\
\newpage 
{\bf -- 4.4}\\
\medskip \\
\begin{tabular}{rrrrrrrrp{120mm}}
WV&WK&WB&ABK&ABB&ABV&AnzB&TW&Zahlencode \textcolor{red}{$\boldsymbol{Grundtext}$} Umschrift $|$"Ubersetzung(en)\\
1.&24.&662.&84.&2654.&1.&5&276&6\_10\_200\_50\_10 \textcolor{red}{\textcjheb{ynryw}} WJRNJ $|$und er lehrte mich/und er unterwies mich\\
2.&25.&663.&89.&2659.&6.&5&257&6\_10\_1\_40\_200 \textcolor{red}{\textcjheb{rm'yw}} WJAMR $|$und (er) sprach\\
3.&26.&664.&94.&2664.&11.&2&40&30\_10 \textcolor{red}{\textcjheb{yl}} LJ $|$zu mir\\
4.&27.&665.&96.&2666.&13.&4&470&10\_400\_40\_20 \textcolor{red}{\textcjheb{kmty}} JTMK $|$halte fest/er (=es) m"oge festhalten\\
5.&28.&666.&100.&2670.&17.&4&216&4\_2\_200\_10 \textcolor{red}{\textcjheb{yrbd}} DBRJ $|$meine Worte\\
6.&29.&667.&104.&2674.&21.&3&52&30\_2\_20 \textcolor{red}{\textcjheb{kbl}} LBK $|$dein Herz\\
7.&30.&668.&107.&2677.&24.&3&540&300\_40\_200 \textcolor{red}{\textcjheb{rm+s}} SMR $|$beobachte/bewahre\\
8.&31.&669.&110.&2680.&27.&5&546&40\_90\_6\_400\_10 \textcolor{red}{\textcjheb{ytw.sm}} M"sWTJ $|$meine Gebote\\
9.&32.&670.&115.&2685.&32.&4&29&6\_8\_10\_5 \textcolor{red}{\textcjheb{hy.hw}} WCJH $|$und lebe\\
\end{tabular}\medskip \\
Ende des Verses 4.4\\
Verse: 94, Buchstaben: 35, 118, 2688, Totalwerte: 2426, 7959, 190975\\
\\
Und er lehrte mich und sprach zu mir: Dein Herz halte meine Worte fest; beobachte meine Gebote und lebe.\\
\newpage 
{\bf -- 4.5}\\
\medskip \\
\begin{tabular}{rrrrrrrrp{120mm}}
WV&WK&WB&ABK&ABB&ABV&AnzB&TW&Zahlencode \textcolor{red}{$\boldsymbol{Grundtext}$} Umschrift $|$"Ubersetzung(en)\\
1.&33.&671.&119.&2689.&1.&3&155&100\_50\_5 \textcolor{red}{\textcjheb{hnq}} QNH $|$erwirb (dir)\\
2.&34.&672.&122.&2692.&4.&4&73&8\_20\_40\_5 \textcolor{red}{\textcjheb{hmk.h}} CKMH $|$Weisheit\\
3.&35.&673.&126.&2696.&8.&3&155&100\_50\_5 \textcolor{red}{\textcjheb{hnq}} QNH $|$erwirb (dir)\\
4.&36.&674.&129.&2699.&11.&4&67&2\_10\_50\_5 \textcolor{red}{\textcjheb{hnyb}} BJNH $|$Verstand/Einsicht\\
5.&37.&675.&133.&2703.&15.&2&31&1\_30 \textcolor{red}{\textcjheb{l'}} AL $|$nicht\\
6.&38.&676.&135.&2705.&17.&4&728&400\_300\_20\_8 \textcolor{red}{\textcjheb{.hk+st}} TSKC $|$vergiss/sollst du vergessen\\
7.&39.&677.&139.&2709.&21.&3&37&6\_1\_30 \textcolor{red}{\textcjheb{l'w}} WAL $|$und nicht\\
8.&40.&678.&142.&2712.&24.&2&409&400\_9 \textcolor{red}{\textcjheb{.tt}} Tt $|$weiche ab/sollst zu weichen\\
9.&41.&679.&144.&2714.&26.&5&291&40\_1\_40\_200\_10 \textcolor{red}{\textcjheb{yrm'm}} MAMRJ $|$von den Reden/von den Worten\\
10.&42.&680.&149.&2719.&31.&2&90&80\_10 \textcolor{red}{\textcjheb{yp}} PJ $|$meines Mundes\\
\end{tabular}\medskip \\
Ende des Verses 4.5\\
Verse: 95, Buchstaben: 32, 150, 2720, Totalwerte: 2036, 9995, 193011\\
\\
Erwirb Weisheit, erwirb Verstand; vergi"s nicht und weiche nicht ab von den Reden meines Mundes.\\
\newpage 
{\bf -- 4.6}\\
\medskip \\
\begin{tabular}{rrrrrrrrp{120mm}}
WV&WK&WB&ABK&ABB&ABV&AnzB&TW&Zahlencode \textcolor{red}{$\boldsymbol{Grundtext}$} Umschrift $|$"Ubersetzung(en)\\
1.&43.&681.&151.&2721.&1.&2&31&1\_30 \textcolor{red}{\textcjheb{l'}} AL $|$nicht\\
2.&44.&682.&153.&2723.&3.&5&484&400\_70\_7\_2\_5 \textcolor{red}{\textcjheb{hbz`t}} TaZBH $|$verlass sie/du sollst verlassen sie\\
3.&45.&683.&158.&2728.&8.&6&966&6\_400\_300\_40\_200\_20 \textcolor{red}{\textcjheb{krm+stw}} WTSMRK $|$und sie wird dich beh"uten/und sie wird bewahren dich\\
4.&46.&684.&164.&2734.&14.&4&13&1\_5\_2\_5 \textcolor{red}{\textcjheb{hbh'}} AHBH $|$liebe sie\\
5.&47.&685.&168.&2738.&18.&5&716&6\_400\_90\_200\_20 \textcolor{red}{\textcjheb{kr.stw}} WT"sRK $|$und sie wird dich bewahren/und sie wird besch"utzen dich\\
\end{tabular}\medskip \\
Ende des Verses 4.6\\
Verse: 96, Buchstaben: 22, 172, 2742, Totalwerte: 2210, 12205, 195221\\
\\
Verla"s sie nicht, und sie wird dich beh"uten; liebe sie, und sie wird dich bewahren.\\
\newpage 
{\bf -- 4.7}\\
\medskip \\
\begin{tabular}{rrrrrrrrp{120mm}}
WV&WK&WB&ABK&ABB&ABV&AnzB&TW&Zahlencode \textcolor{red}{$\boldsymbol{Grundtext}$} Umschrift $|$"Ubersetzung(en)\\
1.&48.&686.&173.&2743.&1.&5&911&200\_1\_300\_10\_400 \textcolor{red}{\textcjheb{ty+s'r}} RASJT $|$der Anfang\\
2.&49.&687.&178.&2748.&6.&4&73&8\_20\_40\_5 \textcolor{red}{\textcjheb{hmk.h}} CKMH $|$(von) Weisheit (ist)\\
3.&50.&688.&182.&2752.&10.&3&155&100\_50\_5 \textcolor{red}{\textcjheb{hnq}} QNH $|$erwirb (dir)\\
4.&51.&689.&185.&2755.&13.&4&73&8\_20\_40\_5 \textcolor{red}{\textcjheb{hmk.h}} CKMH $|$Weisheit\\
5.&52.&690.&189.&2759.&17.&4&58&6\_2\_20\_30 \textcolor{red}{\textcjheb{lkbw}} WBKL $|$und um alles/und mit all\\
6.&53.&691.&193.&2763.&21.&5&230&100\_50\_10\_50\_20 \textcolor{red}{\textcjheb{knynq}} QNJNK $|$was du erworben hast/deinem Verm"ogen\\
7.&54.&692.&198.&2768.&26.&3&155&100\_50\_5 \textcolor{red}{\textcjheb{hnq}} QNH $|$erwirb (dir)\\
8.&55.&693.&201.&2771.&29.&4&67&2\_10\_50\_5 \textcolor{red}{\textcjheb{hnyb}} BJNH $|$Verstand/Einsicht\\
\end{tabular}\medskip \\
Ende des Verses 4.7\\
Verse: 97, Buchstaben: 32, 204, 2774, Totalwerte: 1722, 13927, 196943\\
\\
Der Weisheit Anfang ist: Erwirb Weisheit; und um alles, was du erworben hast, erwirb Verstand.\\
\newpage 
{\bf -- 4.8}\\
\medskip \\
\begin{tabular}{rrrrrrrrp{120mm}}
WV&WK&WB&ABK&ABB&ABV&AnzB&TW&Zahlencode \textcolor{red}{$\boldsymbol{Grundtext}$} Umschrift $|$"Ubersetzung(en)\\
1.&56.&694.&205.&2775.&1.&5&185&60\_30\_60\_30\_5 \textcolor{red}{\textcjheb{hlsls}} sLsLH $|$halte sie hoch\\
2.&57.&695.&210.&2780.&6.&7&712&6\_400\_200\_6\_40\_40\_20 \textcolor{red}{\textcjheb{kmmwrtw}} WTRWMMK $|$und sie wird dich erh"ohen/und sie wird erheben dich\\
3.&58.&696.&217.&2787.&13.&5&446&400\_20\_2\_4\_20 \textcolor{red}{\textcjheb{kdbkt}} TKBDK $|$sie wird dich zu Ehren bringen/sie wird ehren dich\\
4.&59.&697.&222.&2792.&18.&2&30&20\_10 \textcolor{red}{\textcjheb{yk}} KJ $|$wenn\\
5.&60.&698.&224.&2794.&20.&6&565&400\_8\_2\_100\_50\_5 \textcolor{red}{\textcjheb{hnqb.ht}} TCBQNH $|$du umarmst sie\\
\end{tabular}\medskip \\
Ende des Verses 4.8\\
Verse: 98, Buchstaben: 25, 229, 2799, Totalwerte: 1938, 15865, 198881\\
\\
Halte sie hoch, und sie wird dich erh"ohen; sie wird dich zu Ehren bringen, wenn du sie umarmst.\\
\newpage 
{\bf -- 4.9}\\
\medskip \\
\begin{tabular}{rrrrrrrrp{120mm}}
WV&WK&WB&ABK&ABB&ABV&AnzB&TW&Zahlencode \textcolor{red}{$\boldsymbol{Grundtext}$} Umschrift $|$"Ubersetzung(en)\\
1.&61.&699.&230.&2800.&1.&3&850&400\_400\_50 \textcolor{red}{\textcjheb{ntt}} TTN $|$sie wird verleihen/sie wird geben\\
2.&62.&700.&233.&2803.&4.&5&551&30\_200\_1\_300\_20 \textcolor{red}{\textcjheb{k+s'rl}} LRASK $|$deinem Haupt/auf dein Haupt\\
3.&63.&701.&238.&2808.&9.&4&446&30\_6\_10\_400 \textcolor{red}{\textcjheb{tywl}} LWJT $|$(einen) Kranz\\
4.&64.&702.&242.&2812.&13.&2&58&8\_50 \textcolor{red}{\textcjheb{n.h}} CN $|$anmutigen/(der) Gnade\\
5.&65.&703.&244.&2814.&15.&4&679&70\_9\_200\_400 \textcolor{red}{\textcjheb{tr.t`}} atRT $|$(mit) eine(r) Krone\\
6.&66.&704.&248.&2818.&19.&5&1081&400\_80\_1\_200\_400 \textcolor{red}{\textcjheb{tr'pt}} TPART $|$pr"achtige/(von) Ruhm\\
7.&67.&705.&253.&2823.&24.&5&513&400\_40\_3\_50\_20 \textcolor{red}{\textcjheb{kngmt}} TMGNK $|$dir wird darreichen/sie wird dich beschenken\\
\end{tabular}\medskip \\
Ende des Verses 4.9\\
Verse: 99, Buchstaben: 28, 257, 2827, Totalwerte: 4178, 20043, 203059\\
\\
Sie wird deinem Haupte einen anmutigen Kranz verleihen, wird dir darreichen eine pr"achtige Krone.\\
\newpage 
{\bf -- 4.10}\\
\medskip \\
\begin{tabular}{rrrrrrrrp{120mm}}
WV&WK&WB&ABK&ABB&ABV&AnzB&TW&Zahlencode \textcolor{red}{$\boldsymbol{Grundtext}$} Umschrift $|$"Ubersetzung(en)\\
1.&68.&706.&258.&2828.&1.&3&410&300\_40\_70 \textcolor{red}{\textcjheb{`m+s}} SMa $|$h"ore\\
2.&69.&707.&261.&2831.&4.&3&62&2\_50\_10 \textcolor{red}{\textcjheb{ynb}} BNJ $|$mein Sohn\\
3.&70.&708.&264.&2834.&7.&3&114&6\_100\_8 \textcolor{red}{\textcjheb{.hqw}} WQC $|$und nimm an\\
4.&71.&709.&267.&2837.&10.&4&251&1\_40\_200\_10 \textcolor{red}{\textcjheb{yrm'}} AMRJ $|$meine Reden/(meine) Worte\\
5.&72.&710.&271.&2841.&14.&5&224&6\_10\_200\_2\_6 \textcolor{red}{\textcjheb{wbryw}} WJRBW $|$und mehren werden sich/und sie (=es) mehren (sich)\\
6.&73.&711.&276.&2846.&19.&2&50&30\_20 \textcolor{red}{\textcjheb{kl}} LK $|$dir\\
7.&74.&712.&278.&2848.&21.&4&756&300\_50\_6\_400 \textcolor{red}{\textcjheb{twn+s}} SNWT $|$Jahre\\
8.&75.&713.&282.&2852.&25.&4&68&8\_10\_10\_40 \textcolor{red}{\textcjheb{myy.h}} CJJM $|$des Lebens/(der) Lebenden\\
\end{tabular}\medskip \\
Ende des Verses 4.10\\
Verse: 100, Buchstaben: 28, 285, 2855, Totalwerte: 1935, 21978, 204994\\
\\
H"ore, mein Sohn, und nimm meine Reden an! Und des Lebens Jahre werden sich dir mehren.\\
\newpage 
{\bf -- 4.11}\\
\medskip \\
\begin{tabular}{rrrrrrrrp{120mm}}
WV&WK&WB&ABK&ABB&ABV&AnzB&TW&Zahlencode \textcolor{red}{$\boldsymbol{Grundtext}$} Umschrift $|$"Ubersetzung(en)\\
1.&76.&714.&286.&2856.&1.&4&226&2\_4\_200\_20 \textcolor{red}{\textcjheb{krdb}} BDRK $|$in dem Weg/auf dem Weg\\
2.&77.&715.&290.&2860.&5.&4&73&8\_20\_40\_5 \textcolor{red}{\textcjheb{hmk.h}} CKMH $|$der Weisheit\\
3.&78.&716.&294.&2864.&9.&5&635&5\_200\_400\_10\_20 \textcolor{red}{\textcjheb{kytrh}} HRTJK $|$ich unterweise dich\\
4.&79.&717.&299.&2869.&14.&7&659&5\_4\_200\_20\_400\_10\_20 \textcolor{red}{\textcjheb{kytkrdh}} HDRKTJK $|$(ich) leite dich\\
5.&80.&718.&306.&2876.&21.&6&155&2\_40\_70\_3\_30\_10 \textcolor{red}{\textcjheb{ylg`mb}} BMaGLJ $|$auf Bahnen/auf Pfaden\\
6.&81.&719.&312.&2882.&27.&3&510&10\_300\_200 \textcolor{red}{\textcjheb{r+sy}} JSR $|$(der) Geradheit\\
\end{tabular}\medskip \\
Ende des Verses 4.11\\
Verse: 101, Buchstaben: 29, 314, 2884, Totalwerte: 2258, 24236, 207252\\
\\
Ich unterweise dich in dem Wege der Weisheit, leite dich auf Bahnen der Geradheit.\\
\newpage 
{\bf -- 4.12}\\
\medskip \\
\begin{tabular}{rrrrrrrrp{120mm}}
WV&WK&WB&ABK&ABB&ABV&AnzB&TW&Zahlencode \textcolor{red}{$\boldsymbol{Grundtext}$} Umschrift $|$"Ubersetzung(en)\\
1.&82.&720.&315.&2885.&1.&5&472&2\_30\_20\_400\_20 \textcolor{red}{\textcjheb{ktklb}} BLKTK $|$wenn du gehst/bei deinem Gehen\\
2.&83.&721.&320.&2890.&6.&2&31&30\_1 \textcolor{red}{\textcjheb{'l}} LA $|$nicht\\
3.&84.&722.&322.&2892.&8.&3&300&10\_90\_200 \textcolor{red}{\textcjheb{r.sy}} J"sR $|$wird beengt werden/er (=es) wird gehemmt\\
4.&85.&723.&325.&2895.&11.&4&184&90\_70\_4\_20 \textcolor{red}{\textcjheb{kd`.s}} "saDK $|$dein Schritt\\
5.&86.&724.&329.&2899.&15.&3&47&6\_1\_40 \textcolor{red}{\textcjheb{m'w}} WAM $|$und wenn\\
6.&87.&725.&332.&2902.&18.&4&696&400\_200\_6\_90 \textcolor{red}{\textcjheb{.swrt}} TRW"s $|$du l"aufst\\
7.&88.&726.&336.&2906.&22.&2&31&30\_1 \textcolor{red}{\textcjheb{'l}} LA $|$nicht\\
8.&89.&727.&338.&2908.&24.&4&750&400\_20\_300\_30 \textcolor{red}{\textcjheb{l+skt}} TKSL $|$du wirst straucheln\\
\end{tabular}\medskip \\
Ende des Verses 4.12\\
Verse: 102, Buchstaben: 27, 341, 2911, Totalwerte: 2511, 26747, 209763\\
\\
Wenn du gehst, wird dein Schritt nicht beengt werden, und wenn du l"aufst, wirst du nicht straucheln.\\
\newpage 
{\bf -- 4.13}\\
\medskip \\
\begin{tabular}{rrrrrrrrp{120mm}}
WV&WK&WB&ABK&ABB&ABV&AnzB&TW&Zahlencode \textcolor{red}{$\boldsymbol{Grundtext}$} Umschrift $|$"Ubersetzung(en)\\
1.&90.&728.&342.&2912.&1.&4&120&5\_8\_7\_100 \textcolor{red}{\textcjheb{qz.hh}} HCZQ $|$halte fest\\
2.&91.&729.&346.&2916.&5.&5&308&2\_40\_6\_60\_200 \textcolor{red}{\textcjheb{rswmb}} BMWsR $|$an der Unterweisung/an der Zucht\\
3.&92.&730.&351.&2921.&10.&2&31&1\_30 \textcolor{red}{\textcjheb{l'}} AL $|$nicht\\
4.&93.&731.&353.&2923.&12.&3&680&400\_200\_80 \textcolor{red}{\textcjheb{prt}} TRP $|$lass sie los/du sollst lassen sie\\
5.&94.&732.&356.&2926.&15.&4&345&50\_90\_200\_5 \textcolor{red}{\textcjheb{hr.sn}} N"sRH $|$bewahre sie/beh"ute sie\\
6.&95.&733.&360.&2930.&19.&2&30&20\_10 \textcolor{red}{\textcjheb{yk}} KJ $|$denn\\
7.&96.&734.&362.&2932.&21.&3&16&5\_10\_1 \textcolor{red}{\textcjheb{'yh}} HJA $|$sie (ist)\\
8.&97.&735.&365.&2935.&24.&4&48&8\_10\_10\_20 \textcolor{red}{\textcjheb{kyy.h}} CJJK $|$dein Leben\\
\end{tabular}\medskip \\
Ende des Verses 4.13\\
Verse: 103, Buchstaben: 27, 368, 2938, Totalwerte: 1578, 28325, 211341\\
\\
Halte fest an der Unterweisung, la"s sie nicht los; bewahre sie, denn sie ist dein Leben. -\\
\newpage 
{\bf -- 4.14}\\
\medskip \\
\begin{tabular}{rrrrrrrrp{120mm}}
WV&WK&WB&ABK&ABB&ABV&AnzB&TW&Zahlencode \textcolor{red}{$\boldsymbol{Grundtext}$} Umschrift $|$"Ubersetzung(en)\\
1.&98.&736.&369.&2939.&1.&4&211&2\_1\_200\_8 \textcolor{red}{\textcjheb{.hr'b}} BARC $|$auf den Pfad/auf dem Pfad\\
2.&99.&737.&373.&2943.&5.&5&620&200\_300\_70\_10\_40 \textcolor{red}{\textcjheb{my`+sr}} RSaJM $|$der Gesetzlosen/(der) Frevler\\
3.&100.&738.&378.&2948.&10.&2&31&1\_30 \textcolor{red}{\textcjheb{l'}} AL $|$nicht\\
4.&101.&739.&380.&2950.&12.&3&403&400\_2\_1 \textcolor{red}{\textcjheb{'bt}} TBA $|$(sollst du) komm(en)\\
5.&102.&740.&383.&2953.&15.&3&37&6\_1\_30 \textcolor{red}{\textcjheb{l'w}} WAL $|$und nicht\\
6.&103.&741.&386.&2956.&18.&4&901&400\_1\_300\_200 \textcolor{red}{\textcjheb{r+s't}} TASR $|$schreite einher/du sollst einhergehen\\
7.&104.&742.&390.&2960.&22.&4&226&2\_4\_200\_20 \textcolor{red}{\textcjheb{krdb}} BDRK $|$auf dem Weg\\
8.&105.&743.&394.&2964.&26.&4&320&200\_70\_10\_40 \textcolor{red}{\textcjheb{my`r}} RaJM $|$der B"osen\\
\end{tabular}\medskip \\
Ende des Verses 4.14\\
Verse: 104, Buchstaben: 29, 397, 2967, Totalwerte: 2749, 31074, 214090\\
\\
Komm nicht auf den Pfad der Gesetzlosen, und schreite nicht einher auf dem Wege der B"osen.\\
\newpage 
{\bf -- 4.15}\\
\medskip \\
\begin{tabular}{rrrrrrrrp{120mm}}
WV&WK&WB&ABK&ABB&ABV&AnzB&TW&Zahlencode \textcolor{red}{$\boldsymbol{Grundtext}$} Umschrift $|$"Ubersetzung(en)\\
1.&106.&744.&398.&2968.&1.&5&361&80\_200\_70\_5\_6 \textcolor{red}{\textcjheb{wh`rp}} PRaHW $|$lass ihn fahren/meide ihn\\
2.&107.&745.&403.&2973.&6.&2&31&1\_30 \textcolor{red}{\textcjheb{l'}} AL $|$nicht\\
3.&108.&746.&405.&2975.&8.&4&672&400\_70\_2\_200 \textcolor{red}{\textcjheb{rb`t}} TaBR $|$geh/sollst du hin"uberschreiten\\
4.&109.&747.&409.&2979.&12.&2&8&2\_6 \textcolor{red}{\textcjheb{wb}} BW $|$darauf/zu ihm\\
5.&110.&748.&411.&2981.&14.&3&314&300\_9\_5 \textcolor{red}{\textcjheb{h.t+s}} StH $|$wende dich ab\\
6.&111.&749.&414.&2984.&17.&5&156&40\_70\_30\_10\_6 \textcolor{red}{\textcjheb{wyl`m}} MaLJW $|$von ihm\\
7.&112.&750.&419.&2989.&22.&5&284&6\_70\_2\_6\_200 \textcolor{red}{\textcjheb{rwb`w}} WaBWR $|$und geh vorbei/und geh vor"uber\\
\end{tabular}\medskip \\
Ende des Verses 4.15\\
Verse: 105, Buchstaben: 26, 423, 2993, Totalwerte: 1826, 32900, 215916\\
\\
La"s ihn fahren, geh nicht darauf; wende dich von ihm ab und geh vorbei.\\
\newpage 
{\bf -- 4.16}\\
\medskip \\
\begin{tabular}{rrrrrrrrp{120mm}}
WV&WK&WB&ABK&ABB&ABV&AnzB&TW&Zahlencode \textcolor{red}{$\boldsymbol{Grundtext}$} Umschrift $|$"Ubersetzung(en)\\
1.&113.&751.&424.&2994.&1.&2&30&20\_10 \textcolor{red}{\textcjheb{yk}} KJ $|$denn\\
2.&114.&752.&426.&2996.&3.&2&31&30\_1 \textcolor{red}{\textcjheb{'l}} LA $|$nicht\\
3.&115.&753.&428.&2998.&5.&4&366&10\_300\_50\_6 \textcolor{red}{\textcjheb{wn+sy}} JSNW $|$sie (k"onnen) schlafen\\
4.&116.&754.&432.&3002.&9.&2&41&1\_40 \textcolor{red}{\textcjheb{m'}} AM $|$wenn\\
5.&117.&755.&434.&3004.&11.&2&31&30\_1 \textcolor{red}{\textcjheb{'l}} LA $|$nicht\\
6.&118.&756.&436.&3006.&13.&4&286&10\_200\_70\_6 \textcolor{red}{\textcjheb{w`ry}} JRaW $|$sie B"oses getan/sie tun B"oses\\
7.&119.&757.&440.&3010.&17.&6&101&6\_50\_3\_7\_30\_5 \textcolor{red}{\textcjheb{hlzgnw}} WNGZLH $|$und (sie (=es)) wird geraubt (ihnen)\\
8.&120.&758.&446.&3016.&23.&4&790&300\_50\_400\_40 \textcolor{red}{\textcjheb{mtn+s}} SNTM $|$ihr(en) Schlaf\\
9.&121.&759.&450.&3020.&27.&2&41&1\_40 \textcolor{red}{\textcjheb{m'}} AM $|$wenn\\
10.&122.&760.&452.&3022.&29.&2&31&30\_1 \textcolor{red}{\textcjheb{'l}} LA $|$nicht\\
11.&123.&761.&454.&3024.&31.&6&372&10\_20\_300\_6\_30\_6 \textcolor{red}{\textcjheb{wlw+sky}} JKSWLW $|$sie zu Fall gebracht haben/sie bringen zu Fall\\
\end{tabular}\medskip \\
Ende des Verses 4.16\\
Verse: 106, Buchstaben: 36, 459, 3029, Totalwerte: 2120, 35020, 218036\\
\\
Denn sie schlafen nicht, wenn sie nichts B"oses getan, und ihr Schlaf wird ihnen geraubt, wenn sie nicht zu Fall gebracht haben.\\
\newpage 
{\bf -- 4.17}\\
\medskip \\
\begin{tabular}{rrrrrrrrp{120mm}}
WV&WK&WB&ABK&ABB&ABV&AnzB&TW&Zahlencode \textcolor{red}{$\boldsymbol{Grundtext}$} Umschrift $|$"Ubersetzung(en)\\
1.&124.&762.&460.&3030.&1.&2&30&20\_10 \textcolor{red}{\textcjheb{yk}} KJ $|$denn\\
2.&125.&763.&462.&3032.&3.&4&84&30\_8\_40\_6 \textcolor{red}{\textcjheb{wm.hl}} LCMW $|$sie essen\\
3.&126.&764.&466.&3036.&7.&3&78&30\_8\_40 \textcolor{red}{\textcjheb{m.hl}} LCM $|$Brot\\
4.&127.&765.&469.&3039.&10.&3&570&200\_300\_70 \textcolor{red}{\textcjheb{`+sr}} RSa $|$der Gesetzlosigkeit/(des) Frevels\\
5.&128.&766.&472.&3042.&13.&4&76&6\_10\_10\_50 \textcolor{red}{\textcjheb{nyyw}} WJJN $|$und Wein\\
6.&129.&767.&476.&3046.&17.&5&158&8\_40\_60\_10\_40 \textcolor{red}{\textcjheb{mysm.h}} CMsJM $|$der Gewalttaten/(von) Gewalttat(en)\\
7.&130.&768.&481.&3051.&22.&4&716&10\_300\_400\_6 \textcolor{red}{\textcjheb{wt+sy}} JSTW $|$(sie) trinken\\
\end{tabular}\medskip \\
Ende des Verses 4.17\\
Verse: 107, Buchstaben: 25, 484, 3054, Totalwerte: 1712, 36732, 219748\\
\\
Denn sie essen Brot der Gesetzlosigkeit, und trinken Wein der Gewalttaten.\\
\newpage 
{\bf -- 4.18}\\
\medskip \\
\begin{tabular}{rrrrrrrrp{120mm}}
WV&WK&WB&ABK&ABB&ABV&AnzB&TW&Zahlencode \textcolor{red}{$\boldsymbol{Grundtext}$} Umschrift $|$"Ubersetzung(en)\\
1.&131.&769.&485.&3055.&1.&4&215&6\_1\_200\_8 \textcolor{red}{\textcjheb{.hr'w}} WARC $|$aber der Pfad/und der Pfad\\
2.&132.&770.&489.&3059.&5.&6&254&90\_4\_10\_100\_10\_40 \textcolor{red}{\textcjheb{myqyd.s}} "sDJQJM $|$der Gerechten\\
3.&133.&771.&495.&3065.&11.&4&227&20\_1\_6\_200 \textcolor{red}{\textcjheb{rw'k}} KAWR $|$(ist) wie das Licht\\
4.&134.&772.&499.&3069.&15.&3&58&50\_3\_5 \textcolor{red}{\textcjheb{hgn}} NGH $|$gl"anzende (des) Morgen(s)/Glanz\\
5.&135.&773.&502.&3072.&18.&4&61&5\_6\_30\_20 \textcolor{red}{\textcjheb{klwh}} HWLK $|$/gehend(er)\\
6.&136.&774.&506.&3076.&22.&4&213&6\_1\_6\_200 \textcolor{red}{\textcjheb{rw'w}} WAWR $|$das stets heller leuchtet/und ein Leuchten\\
7.&137.&775.&510.&3080.&26.&2&74&70\_4 \textcolor{red}{\textcjheb{d`}} aD $|$bis\\
8.&138.&776.&512.&3082.&28.&4&126&50\_20\_6\_50 \textcolor{red}{\textcjheb{nwkn}} NKWN $|$zur H"ohe/bereitet werdend(er)\\
9.&139.&777.&516.&3086.&32.&4&61&5\_10\_6\_40 \textcolor{red}{\textcjheb{mwyh}} HJWM $|$(des) Tages/der Tag\\
\end{tabular}\medskip \\
Ende des Verses 4.18\\
Verse: 108, Buchstaben: 35, 519, 3089, Totalwerte: 1289, 38021, 221037\\
\\
Aber der Pfad der Gerechten ist wie das gl"anzende Morgenlicht, das stets heller leuchtet bis zur Tagesh"ohe.\\
\newpage 
{\bf -- 4.19}\\
\medskip \\
\begin{tabular}{rrrrrrrrp{120mm}}
WV&WK&WB&ABK&ABB&ABV&AnzB&TW&Zahlencode \textcolor{red}{$\boldsymbol{Grundtext}$} Umschrift $|$"Ubersetzung(en)\\
1.&140.&778.&520.&3090.&1.&3&224&4\_200\_20 \textcolor{red}{\textcjheb{krd}} DRK $|$der Weg\\
2.&141.&779.&523.&3093.&4.&5&620&200\_300\_70\_10\_40 \textcolor{red}{\textcjheb{my`+sr}} RSaJM $|$der Gesetzlosen/(der) Frevler\\
3.&142.&780.&528.&3098.&9.&5&136&20\_1\_80\_30\_5 \textcolor{red}{\textcjheb{hlp'k}} KAPLH $|$ist dem Dunkel gleich/(ist) wie das Dunkel\\
4.&143.&781.&533.&3103.&14.&2&31&30\_1 \textcolor{red}{\textcjheb{'l}} LA $|$nicht\\
5.&144.&782.&535.&3105.&16.&4&90&10\_4\_70\_6 \textcolor{red}{\textcjheb{w`dy}} JDaW $|$sie erkennen\\
6.&145.&783.&539.&3109.&20.&3&47&2\_40\_5 \textcolor{red}{\textcjheb{hmb}} BMH $|$wor"uber/wodurch\\
7.&146.&784.&542.&3112.&23.&5&366&10\_20\_300\_30\_6 \textcolor{red}{\textcjheb{wl+sky}} JKSLW $|$sie straucheln\\
\end{tabular}\medskip \\
Ende des Verses 4.19\\
Verse: 109, Buchstaben: 27, 546, 3116, Totalwerte: 1514, 39535, 222551\\
\\
Der Weg der Gesetzlosen ist dem Dunkel gleich; sie erkennen nicht, wor"uber sie straucheln.\\
\newpage 
{\bf -- 4.20}\\
\medskip \\
\begin{tabular}{rrrrrrrrp{120mm}}
WV&WK&WB&ABK&ABB&ABV&AnzB&TW&Zahlencode \textcolor{red}{$\boldsymbol{Grundtext}$} Umschrift $|$"Ubersetzung(en)\\
1.&147.&785.&547.&3117.&1.&3&62&2\_50\_10 \textcolor{red}{\textcjheb{ynb}} BNJ $|$mein Sohn\\
2.&148.&786.&550.&3120.&4.&5&246&30\_4\_2\_200\_10 \textcolor{red}{\textcjheb{yrbdl}} LDBRJ $|$auf meine Worte\\
3.&149.&787.&555.&3125.&9.&6&422&5\_100\_300\_10\_2\_5 \textcolor{red}{\textcjheb{hby+sqh}} HQSJBH $|$merke\\
4.&150.&788.&561.&3131.&15.&5&281&30\_1\_40\_200\_10 \textcolor{red}{\textcjheb{yrm'l}} LAMRJ $|$zu meinen Reden\\
5.&151.&789.&566.&3136.&20.&2&14&5\_9 \textcolor{red}{\textcjheb{.th}} Ht $|$neige\\
6.&152.&790.&568.&3138.&22.&4&78&1\_7\_50\_20 \textcolor{red}{\textcjheb{knz'}} AZNK $|$dein Ohr\\
\end{tabular}\medskip \\
Ende des Verses 4.20\\
Verse: 110, Buchstaben: 25, 571, 3141, Totalwerte: 1103, 40638, 223654\\
\\
Mein Sohn, merke auf meine Worte, neige dein Ohr zu meinen Reden.\\
\newpage 
{\bf -- 4.21}\\
\medskip \\
\begin{tabular}{rrrrrrrrp{120mm}}
WV&WK&WB&ABK&ABB&ABV&AnzB&TW&Zahlencode \textcolor{red}{$\boldsymbol{Grundtext}$} Umschrift $|$"Ubersetzung(en)\\
1.&153.&791.&572.&3142.&1.&2&31&1\_30 \textcolor{red}{\textcjheb{l'}} AL $|$nicht\\
2.&154.&792.&574.&3144.&3.&5&63&10\_30\_10\_7\_6 \textcolor{red}{\textcjheb{wzyly}} JLJZW $|$lass sie weichen/sie sollen weichen\\
3.&155.&793.&579.&3149.&8.&6&200&40\_70\_10\_50\_10\_20 \textcolor{red}{\textcjheb{kyny`m}} MaJNJK $|$von deinen Augen/aus deinen Augen\\
4.&156.&794.&585.&3155.&14.&4&580&300\_40\_200\_40 \textcolor{red}{\textcjheb{mrm+s}} SMRM $|$bewahre sie\\
5.&157.&795.&589.&3159.&18.&4&428&2\_400\_6\_20 \textcolor{red}{\textcjheb{kwtb}} BTWK $|$im Innern/inmitten\\
6.&158.&796.&593.&3163.&22.&4&54&30\_2\_2\_20 \textcolor{red}{\textcjheb{kbbl}} LBBK $|$deines Herzens\\
\end{tabular}\medskip \\
Ende des Verses 4.21\\
Verse: 111, Buchstaben: 25, 596, 3166, Totalwerte: 1356, 41994, 225010\\
\\
La"s sie nicht von deinen Augen weichen, bewahre sie im Innern deines Herzens.\\
\newpage 
{\bf -- 4.22}\\
\medskip \\
\begin{tabular}{rrrrrrrrp{120mm}}
WV&WK&WB&ABK&ABB&ABV&AnzB&TW&Zahlencode \textcolor{red}{$\boldsymbol{Grundtext}$} Umschrift $|$"Ubersetzung(en)\\
1.&159.&797.&597.&3167.&1.&2&30&20\_10 \textcolor{red}{\textcjheb{yk}} KJ $|$denn\\
2.&160.&798.&599.&3169.&3.&4&68&8\_10\_10\_40 \textcolor{red}{\textcjheb{myy.h}} CJJM $|$Leben(de)\\
3.&161.&799.&603.&3173.&7.&2&45&5\_40 \textcolor{red}{\textcjheb{mh}} HM $|$sie (sind)\\
4.&162.&800.&605.&3175.&9.&7&216&30\_40\_90\_1\_10\_5\_40 \textcolor{red}{\textcjheb{mhy'.sml}} LM"sAJHM $|$denen die sie finden/den Findenden sie\\
5.&163.&801.&612.&3182.&16.&4&86&6\_30\_20\_30 \textcolor{red}{\textcjheb{lklw}} WLKL $|$und (dem) ganzen/und f"ur all\\
6.&164.&802.&616.&3186.&20.&4&508&2\_300\_200\_6 \textcolor{red}{\textcjheb{wr+sb}} BSRW $|$ihrem Fleisch/sein Fleisch\\
7.&165.&803.&620.&3190.&24.&4&321&40\_200\_80\_1 \textcolor{red}{\textcjheb{'prm}} MRPA $|$Gesundheit/Heilung\\
\end{tabular}\medskip \\
Ende des Verses 4.22\\
Verse: 112, Buchstaben: 27, 623, 3193, Totalwerte: 1274, 43268, 226284\\
\\
Denn Leben sind sie denen, die sie finden, und Gesundheit ihrem ganzen Fleische. -\\
\newpage 
{\bf -- 4.23}\\
\medskip \\
\begin{tabular}{rrrrrrrrp{120mm}}
WV&WK&WB&ABK&ABB&ABV&AnzB&TW&Zahlencode \textcolor{red}{$\boldsymbol{Grundtext}$} Umschrift $|$"Ubersetzung(en)\\
1.&166.&804.&624.&3194.&1.&3&90&40\_20\_30 \textcolor{red}{\textcjheb{lkm}} MKL $|$mehr als alles/vor aller\\
2.&167.&805.&627.&3197.&4.&4&580&40\_300\_40\_200 \textcolor{red}{\textcjheb{rm+sm}} MSMR $|$was zu bewahren ist/Wache\\
3.&168.&806.&631.&3201.&8.&3&340&50\_90\_200 \textcolor{red}{\textcjheb{r.sn}} N"sR $|$beh"ute/bewahre\\
4.&169.&807.&634.&3204.&11.&3&52&30\_2\_20 \textcolor{red}{\textcjheb{kbl}} LBK $|$dein Herz\\
5.&170.&808.&637.&3207.&14.&2&30&20\_10 \textcolor{red}{\textcjheb{yk}} KJ $|$denn\\
6.&171.&809.&639.&3209.&16.&4&136&40\_40\_50\_6 \textcolor{red}{\textcjheb{wnmm}} MMNW $|$von ihm aus (gehen)\\
7.&172.&810.&643.&3213.&20.&6&903&400\_6\_90\_1\_6\_400 \textcolor{red}{\textcjheb{tw'.swt}} TW"sAWT $|$(sind) die Ausg"ange\\
8.&173.&811.&649.&3219.&26.&4&68&8\_10\_10\_40 \textcolor{red}{\textcjheb{myy.h}} CJJM $|$des Lebens\\
\end{tabular}\medskip \\
Ende des Verses 4.23\\
Verse: 113, Buchstaben: 29, 652, 3222, Totalwerte: 2199, 45467, 228483\\
\\
Beh"ute dein Herz mehr als alles, was zu bewahren ist; denn von ihm aus sind die Ausg"ange des Lebens. -\\
\newpage 
{\bf -- 4.24}\\
\medskip \\
\begin{tabular}{rrrrrrrrp{120mm}}
WV&WK&WB&ABK&ABB&ABV&AnzB&TW&Zahlencode \textcolor{red}{$\boldsymbol{Grundtext}$} Umschrift $|$"Ubersetzung(en)\\
1.&174.&812.&653.&3223.&1.&3&265&5\_60\_200 \textcolor{red}{\textcjheb{rsh}} HsR $|$tue/mache weichen\\
2.&175.&813.&656.&3226.&4.&3&100&40\_40\_20 \textcolor{red}{\textcjheb{kmm}} MMK $|$von dir\\
3.&176.&814.&659.&3229.&7.&5&876&70\_100\_300\_6\_400 \textcolor{red}{\textcjheb{tw+sq`}} aQSWT $|$die Verkehrtheit/Falschheit\\
4.&177.&815.&664.&3234.&12.&2&85&80\_5 \textcolor{red}{\textcjheb{hp}} PH $|$(des) Mundes\\
5.&178.&816.&666.&3236.&14.&5&449&6\_30\_7\_6\_400 \textcolor{red}{\textcjheb{twzlw}} WLZWT $|$und die Verdrehtheit/und Verkehrtheit\\
6.&179.&817.&671.&3241.&19.&5&830&300\_80\_400\_10\_40 \textcolor{red}{\textcjheb{mytp+s}} SPTJM $|$der Lippen/(zweier) Lippen\\
7.&180.&818.&676.&3246.&24.&4&313&5\_200\_8\_100 \textcolor{red}{\textcjheb{q.hrh}} HRCQ $|$entferne/halte fern\\
8.&181.&819.&680.&3250.&28.&3&100&40\_40\_20 \textcolor{red}{\textcjheb{kmm}} MMK $|$von dir\\
\end{tabular}\medskip \\
Ende des Verses 4.24\\
Verse: 114, Buchstaben: 30, 682, 3252, Totalwerte: 3018, 48485, 231501\\
\\
Tue von dir die Verkehrtheit des Mundes, und die Verdrehtheit der Lippen entferne von dir. -\\
\newpage 
{\bf -- 4.25}\\
\medskip \\
\begin{tabular}{rrrrrrrrp{120mm}}
WV&WK&WB&ABK&ABB&ABV&AnzB&TW&Zahlencode \textcolor{red}{$\boldsymbol{Grundtext}$} Umschrift $|$"Ubersetzung(en)\\
1.&182.&820.&683.&3253.&1.&5&160&70\_10\_50\_10\_20 \textcolor{red}{\textcjheb{kyny`}} aJNJK $|$deine Augen\\
2.&183.&821.&688.&3258.&6.&4&108&30\_50\_20\_8 \textcolor{red}{\textcjheb{.hknl}} LNKC $|$geradeaus\\
3.&184.&822.&692.&3262.&10.&5&37&10\_2\_10\_9\_6 \textcolor{red}{\textcjheb{w.tyby}} JBJtW $|$lass blicken/sie sollen schauen\\
4.&185.&823.&697.&3267.&15.&7&336&6\_70\_80\_70\_80\_10\_20 \textcolor{red}{\textcjheb{kyp`p`w}} WaPaPJK $|$und deine Wimpern\\
5.&186.&824.&704.&3274.&22.&5&526&10\_10\_300\_200\_6 \textcolor{red}{\textcjheb{wr+syy}} JJSRW $|$stracks hin schauen/sie sollen geradeaus blicken\\
6.&187.&825.&709.&3279.&27.&4&77&50\_3\_4\_20 \textcolor{red}{\textcjheb{kdgn}} NGDK $|$vor dich/vor dir\\
\end{tabular}\medskip \\
Ende des Verses 4.25\\
Verse: 115, Buchstaben: 30, 712, 3282, Totalwerte: 1244, 49729, 232745\\
\\
La"s deine Augen geradeaus blicken, und deine Wimpern stracks vor dich hin schauen. -\\
\newpage 
{\bf -- 4.26}\\
\medskip \\
\begin{tabular}{rrrrrrrrp{120mm}}
WV&WK&WB&ABK&ABB&ABV&AnzB&TW&Zahlencode \textcolor{red}{$\boldsymbol{Grundtext}$} Umschrift $|$"Ubersetzung(en)\\
1.&188.&826.&713.&3283.&1.&3&170&80\_30\_60 \textcolor{red}{\textcjheb{slp}} PLs $|$ebne\\
2.&189.&827.&716.&3286.&4.&4&143&40\_70\_3\_30 \textcolor{red}{\textcjheb{lg`m}} MaGL $|$die Bahn\\
3.&190.&828.&720.&3290.&8.&4&253&200\_3\_30\_20 \textcolor{red}{\textcjheb{klgr}} RGLK $|$deines Fu"ses\\
4.&191.&829.&724.&3294.&12.&3&56&6\_20\_30 \textcolor{red}{\textcjheb{lkw}} WKL $|$und all(e)\\
5.&192.&830.&727.&3297.&15.&5&254&4\_200\_20\_10\_20 \textcolor{red}{\textcjheb{kykrd}} DRKJK $|$deine Wege\\
6.&193.&831.&732.&3302.&20.&4&86&10\_20\_50\_6 \textcolor{red}{\textcjheb{wnky}} JKNW $|$seien gerade/sie seien fest\\
\end{tabular}\medskip \\
Ende des Verses 4.26\\
Verse: 116, Buchstaben: 23, 735, 3305, Totalwerte: 962, 50691, 233707\\
\\
Ebne die Bahn deines Fu"ses, und alle deine Wege seien gerade;\\
\newpage 
{\bf -- 4.27}\\
\medskip \\
\begin{tabular}{rrrrrrrrp{120mm}}
WV&WK&WB&ABK&ABB&ABV&AnzB&TW&Zahlencode \textcolor{red}{$\boldsymbol{Grundtext}$} Umschrift $|$"Ubersetzung(en)\\
1.&194.&832.&736.&3306.&1.&2&31&1\_30 \textcolor{red}{\textcjheb{l'}} AL $|$nicht\\
2.&195.&833.&738.&3308.&3.&2&409&400\_9 \textcolor{red}{\textcjheb{.tt}} Tt $|$biege aus/biege ab\\
3.&196.&834.&740.&3310.&5.&4&110&10\_40\_10\_50 \textcolor{red}{\textcjheb{nymy}} JMJN $|$(zur) Rechten\\
4.&197.&835.&744.&3314.&9.&6&383&6\_300\_40\_1\_6\_30 \textcolor{red}{\textcjheb{lw'm+sw}} WSMAWL $|$noch zur Linken/und (zur) Linken\\
5.&198.&836.&750.&3320.&15.&3&265&5\_60\_200 \textcolor{red}{\textcjheb{rsh}} HsR $|$wende ab/halte fern\\
6.&199.&837.&753.&3323.&18.&4&253&200\_3\_30\_20 \textcolor{red}{\textcjheb{klgr}} RGLK $|$deinen Fu"s\\
7.&200.&838.&757.&3327.&22.&3&310&40\_200\_70 \textcolor{red}{\textcjheb{`rm}} MRa $|$vom B"osen\\
\end{tabular}\medskip \\
Ende des Verses 4.27\\
Verse: 117, Buchstaben: 24, 759, 3329, Totalwerte: 1761, 52452, 235468\\
\\
biege nicht aus zur Rechten noch zur Linken, wende deinen Fu"s ab vom B"osen.\\
\\
{\bf Ende des Kapitels 4}\\
\newpage 
{\bf -- 5.1}\\
\medskip \\
\begin{tabular}{rrrrrrrrp{120mm}}
WV&WK&WB&ABK&ABB&ABV&AnzB&TW&Zahlencode \textcolor{red}{$\boldsymbol{Grundtext}$} Umschrift $|$"Ubersetzung(en)\\
1.&1.&839.&1.&3330.&1.&3&62&2\_50\_10 \textcolor{red}{\textcjheb{ynb}} BNJ $|$mein Sohn\\
2.&2.&840.&4.&3333.&4.&6&508&30\_8\_20\_40\_400\_10 \textcolor{red}{\textcjheb{ytmk.hl}} LCKMTJ $|$auf meine Weisheit\\
3.&3.&841.&10.&3339.&10.&6&422&5\_100\_300\_10\_2\_5 \textcolor{red}{\textcjheb{hby+sqh}} HQSJBH $|$merke/achte\\
4.&4.&842.&16.&3345.&16.&7&898&30\_400\_2\_6\_50\_400\_10 \textcolor{red}{\textcjheb{ytnwbtl}} LTBWNTJ $|$zu meiner Einsicht\\
5.&5.&843.&23.&3352.&23.&2&14&5\_9 \textcolor{red}{\textcjheb{.th}} Ht $|$neige\\
6.&6.&844.&25.&3354.&25.&4&78&1\_7\_50\_20 \textcolor{red}{\textcjheb{knz'}} AZNK $|$dein Ohr\\
\end{tabular}\medskip \\
Ende des Verses 5.1\\
Verse: 118, Buchstaben: 28, 28, 3357, Totalwerte: 1982, 1982, 237450\\
\\
Mein Sohn, merke auf meine Weisheit, neige dein Ohr zu meiner Einsicht,\\
\newpage 
{\bf -- 5.2}\\
\medskip \\
\begin{tabular}{rrrrrrrrp{120mm}}
WV&WK&WB&ABK&ABB&ABV&AnzB&TW&Zahlencode \textcolor{red}{$\boldsymbol{Grundtext}$} Umschrift $|$"Ubersetzung(en)\\
1.&7.&845.&29.&3358.&1.&4&570&30\_300\_40\_200 \textcolor{red}{\textcjheb{rm+sl}} LSMR $|$(um) zu be(ob)achten\\
2.&8.&846.&33.&3362.&5.&5&493&40\_7\_40\_6\_400 \textcolor{red}{\textcjheb{twmzm}} MZMWT $|$Besonnenheit\\
3.&9.&847.&38.&3367.&10.&4&480&6\_4\_70\_400 \textcolor{red}{\textcjheb{t`dw}} WDaT $|$und (damit) Erkenntnis\\
4.&10.&848.&42.&3371.&14.&5&810&300\_80\_400\_10\_20 \textcolor{red}{\textcjheb{kytp+s}} SPTJK $|$deine Lippen\\
5.&11.&849.&47.&3376.&19.&5&356&10\_50\_90\_200\_6 \textcolor{red}{\textcjheb{wr.sny}} JN"sRW $|$(sie m"ogen) bewahren\\
\end{tabular}\medskip \\
Ende des Verses 5.2\\
Verse: 119, Buchstaben: 23, 51, 3380, Totalwerte: 2709, 4691, 240159\\
\\
um Besonnenheit zu beobachten, und damit deine Lippen Erkenntnis bewahren.\\
\newpage 
{\bf -- 5.3}\\
\medskip \\
\begin{tabular}{rrrrrrrrp{120mm}}
WV&WK&WB&ABK&ABB&ABV&AnzB&TW&Zahlencode \textcolor{red}{$\boldsymbol{Grundtext}$} Umschrift $|$"Ubersetzung(en)\\
1.&12.&850.&52.&3381.&1.&2&30&20\_10 \textcolor{red}{\textcjheb{yk}} KJ $|$denn\\
2.&13.&851.&54.&3383.&3.&3&530&50\_80\_400 \textcolor{red}{\textcjheb{tpn}} NPT $|$Honig(seim)\\
3.&14.&852.&57.&3386.&6.&5&544&400\_9\_80\_50\_5 \textcolor{red}{\textcjheb{hnp.tt}} TtPNH $|$(sie) tr"aufeln\\
4.&15.&853.&62.&3391.&11.&4&790&300\_80\_400\_10 \textcolor{red}{\textcjheb{ytp+s}} SPTJ $|$die Lippen/(beide) Lippen\\
5.&16.&854.&66.&3395.&15.&3&212&7\_200\_5 \textcolor{red}{\textcjheb{hrz}} ZRH $|$der Fremden/(einer) fremden (Frau)\\
6.&17.&855.&69.&3398.&18.&4&144&6\_8\_30\_100 \textcolor{red}{\textcjheb{ql.hw}} WCLQ $|$und glatt(er) (ist)\\
7.&18.&856.&73.&3402.&22.&4&430&40\_300\_40\_50 \textcolor{red}{\textcjheb{nm+sm}} MSMN $|$(mehr) als "Ol\\
8.&19.&857.&77.&3406.&26.&3&33&8\_20\_5 \textcolor{red}{\textcjheb{hk.h}} CKH $|$ihr Gaumen\\
\end{tabular}\medskip \\
Ende des Verses 5.3\\
Verse: 120, Buchstaben: 28, 79, 3408, Totalwerte: 2713, 7404, 242872\\
\\
Denn Honigseim tr"aufeln die Lippen der Fremden, und glatter als "Ol ist ihr Gaumen;\\
\newpage 
{\bf -- 5.4}\\
\medskip \\
\begin{tabular}{rrrrrrrrp{120mm}}
WV&WK&WB&ABK&ABB&ABV&AnzB&TW&Zahlencode \textcolor{red}{$\boldsymbol{Grundtext}$} Umschrift $|$"Ubersetzung(en)\\
1.&20.&858.&80.&3409.&1.&7&630&6\_1\_8\_200\_10\_400\_5 \textcolor{red}{\textcjheb{htyr.h'w}} WACRJTH $|$aber ihr letztes/und ihr Ende\\
2.&21.&859.&87.&3416.&8.&3&245&40\_200\_5 \textcolor{red}{\textcjheb{hrm}} MRH $|$(ist) bitter\\
3.&22.&860.&90.&3419.&11.&5&175&20\_30\_70\_50\_5 \textcolor{red}{\textcjheb{hn`lk}} KLaNH $|$wie (der) Wermut\\
4.&23.&861.&95.&3424.&16.&3&17&8\_4\_5 \textcolor{red}{\textcjheb{hd.h}} CDH $|$scharf\\
5.&24.&862.&98.&3427.&19.&4&230&20\_8\_200\_2 \textcolor{red}{\textcjheb{br.hk}} KCRB $|$wie ein Schwert\\
6.&25.&863.&102.&3431.&23.&4&496&80\_10\_6\_400 \textcolor{red}{\textcjheb{twyp}} PJWT $|$zweischneidiges/(mit) Schneiden\\
\end{tabular}\medskip \\
Ende des Verses 5.4\\
Verse: 121, Buchstaben: 26, 105, 3434, Totalwerte: 1793, 9197, 244665\\
\\
aber ihr Letztes ist bitter wie Wermut, scharf wie ein zweischneidiges Schwert.\\
\newpage 
{\bf -- 5.5}\\
\medskip \\
\begin{tabular}{rrrrrrrrp{120mm}}
WV&WK&WB&ABK&ABB&ABV&AnzB&TW&Zahlencode \textcolor{red}{$\boldsymbol{Grundtext}$} Umschrift $|$"Ubersetzung(en)\\
1.&26.&864.&106.&3435.&1.&5&248&200\_3\_30\_10\_5 \textcolor{red}{\textcjheb{hylgr}} RGLJH $|$ihre(r) F"u"se\\
2.&27.&865.&111.&3440.&6.&5&620&10\_200\_4\_6\_400 \textcolor{red}{\textcjheb{twdry}} JRDWT $|$steigen hinab/(sind) niedersteigend(e)\\
3.&28.&866.&116.&3445.&11.&3&446&40\_6\_400 \textcolor{red}{\textcjheb{twm}} MWT $|$(zum) Tod\\
4.&29.&867.&119.&3448.&14.&4&337&300\_1\_6\_30 \textcolor{red}{\textcjheb{lw'+s}} SAWL $|$an dem Scheol/(an der) Unterwelt\\
5.&30.&868.&123.&3452.&18.&5&179&90\_70\_4\_10\_5 \textcolor{red}{\textcjheb{hyd`.s}} "saDJH $|$ihre Schritte\\
6.&31.&869.&128.&3457.&23.&5&476&10\_400\_40\_20\_6 \textcolor{red}{\textcjheb{wkmty}} JTMKW $|$haften/sie halten fest\\
\end{tabular}\medskip \\
Ende des Verses 5.5\\
Verse: 122, Buchstaben: 27, 132, 3461, Totalwerte: 2306, 11503, 246971\\
\\
Ihre F"u"se steigen hinab zum Tode, an dem Scheol haften ihre Schritte.\\
\newpage 
{\bf -- 5.6}\\
\medskip \\
\begin{tabular}{rrrrrrrrp{120mm}}
WV&WK&WB&ABK&ABB&ABV&AnzB&TW&Zahlencode \textcolor{red}{$\boldsymbol{Grundtext}$} Umschrift $|$"Ubersetzung(en)\\
1.&32.&870.&133.&3462.&1.&3&209&1\_200\_8 \textcolor{red}{\textcjheb{.hr'}} ARC $|$den Weg/(den) Pfad\\
2.&33.&871.&136.&3465.&4.&4&68&8\_10\_10\_40 \textcolor{red}{\textcjheb{myy.h}} CJJM $|$des Lebens/(der) Lebenden\\
3.&34.&872.&140.&3469.&8.&2&130&80\_50 \textcolor{red}{\textcjheb{np}} PN $|$damit nicht/dass nicht\\
4.&35.&873.&142.&3471.&10.&4&570&400\_80\_30\_60 \textcolor{red}{\textcjheb{slpt}} TPLs $|$sie einschlage/du m"ogest beachten\\
5.&36.&874.&146.&3475.&14.&3&126&50\_70\_6 \textcolor{red}{\textcjheb{w`n}} NaW $|$schweifen/sie (=es) schwankten\\
6.&37.&875.&149.&3478.&17.&7&558&40\_70\_3\_30\_400\_10\_5 \textcolor{red}{\textcjheb{hytlg`m}} MaGLTJH $|$ihre Bahnen\\
7.&38.&876.&156.&3485.&24.&2&31&30\_1 \textcolor{red}{\textcjheb{'l}} LA $|$ohne/nicht\\
8.&39.&877.&158.&3487.&26.&3&474&400\_4\_70 \textcolor{red}{\textcjheb{`dt}} TDa $|$dass sie es wei"s/du merkst (es)\\
\end{tabular}\medskip \\
Ende des Verses 5.6\\
Verse: 123, Buchstaben: 28, 160, 3489, Totalwerte: 2166, 13669, 249137\\
\\
Damit sie nicht den Weg des Lebens einschlage, schweifen ihre Bahnen, ohne da"s sie es wei"s.\\
\newpage 
{\bf -- 5.7}\\
\medskip \\
\begin{tabular}{rrrrrrrrp{120mm}}
WV&WK&WB&ABK&ABB&ABV&AnzB&TW&Zahlencode \textcolor{red}{$\boldsymbol{Grundtext}$} Umschrift $|$"Ubersetzung(en)\\
1.&40.&878.&161.&3490.&1.&4&481&6\_70\_400\_5 \textcolor{red}{\textcjheb{ht`w}} WaTH $|$nun denn/und nun\\
2.&41.&879.&165.&3494.&5.&4&102&2\_50\_10\_40 \textcolor{red}{\textcjheb{mynb}} BNJM $|$(ihr) S"ohne\\
3.&42.&880.&169.&3498.&9.&4&416&300\_40\_70\_6 \textcolor{red}{\textcjheb{w`m+s}} SMaW $|$h"oret\\
4.&43.&881.&173.&3502.&13.&2&40&30\_10 \textcolor{red}{\textcjheb{yl}} LJ $|$auf mich\\
5.&44.&882.&175.&3504.&15.&3&37&6\_1\_30 \textcolor{red}{\textcjheb{l'w}} WAL $|$und nicht\\
6.&45.&883.&178.&3507.&18.&5&672&400\_60\_6\_200\_6 \textcolor{red}{\textcjheb{wrwst}} TsWRW $|$weicht ab/ihr sollt weichen\\
7.&46.&884.&183.&3512.&23.&5&291&40\_1\_40\_200\_10 \textcolor{red}{\textcjheb{yrm'm}} MAMRJ $|$von den Worten\\
8.&47.&885.&188.&3517.&28.&2&90&80\_10 \textcolor{red}{\textcjheb{yp}} PJ $|$meines Mundes\\
\end{tabular}\medskip \\
Ende des Verses 5.7\\
Verse: 124, Buchstaben: 29, 189, 3518, Totalwerte: 2129, 15798, 251266\\
\\
Nun denn, ihr S"ohne, h"oret auf mich, und weichet nicht ab von den Worten meines Mundes!\\
\newpage 
{\bf -- 5.8}\\
\medskip \\
\begin{tabular}{rrrrrrrrp{120mm}}
WV&WK&WB&ABK&ABB&ABV&AnzB&TW&Zahlencode \textcolor{red}{$\boldsymbol{Grundtext}$} Umschrift $|$"Ubersetzung(en)\\
1.&48.&886.&190.&3519.&1.&4&313&5\_200\_8\_100 \textcolor{red}{\textcjheb{q.hrh}} HRCQ $|$halte fern\\
2.&49.&887.&194.&3523.&5.&5&155&40\_70\_30\_10\_5 \textcolor{red}{\textcjheb{hyl`m}} MaLJH $|$von ihr\\
3.&50.&888.&199.&3528.&10.&4&244&4\_200\_20\_20 \textcolor{red}{\textcjheb{kkrd}} DRKK $|$deinen Weg\\
4.&51.&889.&203.&3532.&14.&3&37&6\_1\_30 \textcolor{red}{\textcjheb{l'w}} WAL $|$und nicht\\
5.&52.&890.&206.&3535.&17.&4&702&400\_100\_200\_2 \textcolor{red}{\textcjheb{brqt}} TQRB $|$nahe/sollst du nahen\\
6.&53.&891.&210.&3539.&21.&2&31&1\_30 \textcolor{red}{\textcjheb{l'}} AL $|$zu/dem\\
7.&54.&892.&212.&3541.&23.&3&488&80\_400\_8 \textcolor{red}{\textcjheb{.htp}} PTC $|$der T"ur/Eingang\\
8.&55.&893.&215.&3544.&26.&4&417&2\_10\_400\_5 \textcolor{red}{\textcjheb{htyb}} BJTH $|$ihres Hauses\\
\end{tabular}\medskip \\
Ende des Verses 5.8\\
Verse: 125, Buchstaben: 29, 218, 3547, Totalwerte: 2387, 18185, 253653\\
\\
Halte fern von ihr deinen Weg, und nahe nicht zu der T"ur ihres Hauses:\\
\newpage 
{\bf -- 5.9}\\
\medskip \\
\begin{tabular}{rrrrrrrrp{120mm}}
WV&WK&WB&ABK&ABB&ABV&AnzB&TW&Zahlencode \textcolor{red}{$\boldsymbol{Grundtext}$} Umschrift $|$"Ubersetzung(en)\\
1.&56.&894.&219.&3548.&1.&2&130&80\_50 \textcolor{red}{\textcjheb{np}} PN $|$damit nicht/dass nicht\\
2.&57.&895.&221.&3550.&3.&3&850&400\_400\_50 \textcolor{red}{\textcjheb{ntt}} TTN $|$du gebest/du gibst\\
3.&58.&896.&224.&3553.&6.&6&289&30\_1\_8\_200\_10\_40 \textcolor{red}{\textcjheb{myr.h'l}} LACRJM $|$(an) andere(n)\\
4.&59.&897.&230.&3559.&12.&4&35&5\_6\_4\_20 \textcolor{red}{\textcjheb{kdwh}} HWDK $|$deine Bl"ute/deinen Glanz\\
5.&60.&898.&234.&3563.&16.&6&786&6\_300\_50\_400\_10\_20 \textcolor{red}{\textcjheb{kytn+sw}} WSNTJK $|$und deine Jahre\\
6.&61.&899.&240.&3569.&22.&6&268&30\_1\_20\_7\_200\_10 \textcolor{red}{\textcjheb{yrzk'l}} LAKZRJ $|$dem Grausamen/an einen Grausamen\\
\end{tabular}\medskip \\
Ende des Verses 5.9\\
Verse: 126, Buchstaben: 27, 245, 3574, Totalwerte: 2358, 20543, 256011\\
\\
damit du nicht anderen deine Bl"ute gebest, und deine Jahre dem Grausamen;\\
\newpage 
{\bf -- 5.10}\\
\medskip \\
\begin{tabular}{rrrrrrrrp{120mm}}
WV&WK&WB&ABK&ABB&ABV&AnzB&TW&Zahlencode \textcolor{red}{$\boldsymbol{Grundtext}$} Umschrift $|$"Ubersetzung(en)\\
1.&62.&900.&246.&3575.&1.&2&130&80\_50 \textcolor{red}{\textcjheb{np}} PN $|$damit nicht/dass nicht\\
2.&63.&901.&248.&3577.&3.&5&388&10\_300\_2\_70\_6 \textcolor{red}{\textcjheb{w`b+sy}} JSBaW $|$(sie) s"attigen sich\\
3.&64.&902.&253.&3582.&8.&4&257&7\_200\_10\_40 \textcolor{red}{\textcjheb{myrz}} ZRJM $|$Fremde\\
4.&65.&903.&257.&3586.&12.&3&48&20\_8\_20 \textcolor{red}{\textcjheb{k.hk}} KCK $|$an deinem Verm"ogen/an deiner Kraft\\
5.&66.&904.&260.&3589.&15.&6&198&6\_70\_90\_2\_10\_20 \textcolor{red}{\textcjheb{kyb.s`w}} Wa"sBJK $|$und dein (m"uhsam) Erworbenes (nicht komme)\\
6.&67.&905.&266.&3595.&21.&4&414&2\_2\_10\_400 \textcolor{red}{\textcjheb{tybb}} BBJT $|$in (das) Haus/im Haus\\
7.&68.&906.&270.&3599.&25.&4&280&50\_20\_200\_10 \textcolor{red}{\textcjheb{yrkn}} NKRJ $|$(eines) Ausl"anders\\
\end{tabular}\medskip \\
Ende des Verses 5.10\\
Verse: 127, Buchstaben: 28, 273, 3602, Totalwerte: 1715, 22258, 257726\\
\\
damit nicht Fremde sich s"attigen an deinem Verm"ogen, und dein m"uhsam Erworbenes nicht komme in eines Ausl"anders Haus;\\
\newpage 
{\bf -- 5.11}\\
\medskip \\
\begin{tabular}{rrrrrrrrp{120mm}}
WV&WK&WB&ABK&ABB&ABV&AnzB&TW&Zahlencode \textcolor{red}{$\boldsymbol{Grundtext}$} Umschrift $|$"Ubersetzung(en)\\
1.&69.&907.&274.&3603.&1.&5&501&6\_50\_5\_40\_400 \textcolor{red}{\textcjheb{tmhnw}} WNHMT $|$und nicht du st"ohnst/und du m"usstest st"ohnen\\
2.&70.&908.&279.&3608.&6.&7&641&2\_1\_8\_200\_10\_400\_20 \textcolor{red}{\textcjheb{ktyr.h'b}} BACRJTK $|$bei deinem Ende/an deinem Ende\\
3.&71.&909.&286.&3615.&13.&5&458&2\_20\_30\_6\_400 \textcolor{red}{\textcjheb{twlkb}} BKLWT $|$wenn dahinschwinden/in Schwinden\\
4.&72.&910.&291.&3620.&18.&4&522&2\_300\_200\_20 \textcolor{red}{\textcjheb{kr+sb}} BSRK $|$dein Fleisch/dein Leib\\
5.&73.&911.&295.&3624.&22.&5&527&6\_300\_1\_200\_20 \textcolor{red}{\textcjheb{kr'+sw}} WSARK $|$und dein Leib/und dein Fleisch\\
\end{tabular}\medskip \\
Ende des Verses 5.11\\
Verse: 128, Buchstaben: 26, 299, 3628, Totalwerte: 2649, 24907, 260375\\
\\
und du nicht st"ohnest bei deinem Ende, wenn dein Fleisch und dein Leib dahinschwinden, und sagest:\\
\newpage 
{\bf -- 5.12}\\
\medskip \\
\begin{tabular}{rrrrrrrrp{120mm}}
WV&WK&WB&ABK&ABB&ABV&AnzB&TW&Zahlencode \textcolor{red}{$\boldsymbol{Grundtext}$} Umschrift $|$"Ubersetzung(en)\\
1.&74.&912.&300.&3629.&1.&5&647&6\_1\_40\_200\_400 \textcolor{red}{\textcjheb{trm'w}} WAMRT $|$und sagest/und du sag(e)st\\
2.&75.&913.&305.&3634.&6.&3&31&1\_10\_20 \textcolor{red}{\textcjheb{ky'}} AJK $|$wie\\
3.&76.&914.&308.&3637.&9.&5&761&300\_50\_1\_400\_10 \textcolor{red}{\textcjheb{yt'n+s}} SNATJ $|$habe ich gehasst/ich hasste\\
4.&77.&915.&313.&3642.&14.&4&306&40\_6\_60\_200 \textcolor{red}{\textcjheb{rswm}} MWsR $|$die Unterweisung/Zucht\\
5.&78.&916.&317.&3646.&18.&6&840&6\_400\_6\_20\_8\_400 \textcolor{red}{\textcjheb{t.hkwtw}} WTWKCT $|$und die Zucht/und Zurechtweisung\\
6.&79.&917.&323.&3652.&24.&3&141&50\_1\_90 \textcolor{red}{\textcjheb{.s'n}} NA"s $|$hat verschm"aht/er (=es) verschm"ahte\\
7.&80.&918.&326.&3655.&27.&3&42&30\_2\_10 \textcolor{red}{\textcjheb{ybl}} LBJ $|$mein Herz\\
\end{tabular}\medskip \\
Ende des Verses 5.12\\
Verse: 129, Buchstaben: 29, 328, 3657, Totalwerte: 2768, 27675, 263143\\
\\
Wie habe ich die Unterweisung geha"st, und mein Herz hat die Zucht verschm"aht!\\
\newpage 
{\bf -- 5.13}\\
\medskip \\
\begin{tabular}{rrrrrrrrp{120mm}}
WV&WK&WB&ABK&ABB&ABV&AnzB&TW&Zahlencode \textcolor{red}{$\boldsymbol{Grundtext}$} Umschrift $|$"Ubersetzung(en)\\
1.&81.&919.&329.&3658.&1.&3&37&6\_30\_1 \textcolor{red}{\textcjheb{'lw}} WLA $|$und nicht\\
2.&82.&920.&332.&3661.&4.&5&820&300\_40\_70\_400\_10 \textcolor{red}{\textcjheb{yt`m+s}} SMaTJ $|$habe ich geh"ort/ich h"orte\\
3.&83.&921.&337.&3666.&9.&4&138&2\_100\_6\_30 \textcolor{red}{\textcjheb{lwqb}} BQWL $|$auf die Stimme\\
4.&84.&922.&341.&3670.&13.&4&256&40\_6\_200\_10 \textcolor{red}{\textcjheb{yrwm}} MWRJ $|$meiner Unterweiser/meiner Unterweisenden\\
5.&85.&923.&345.&3674.&17.&7&160&6\_30\_40\_30\_40\_4\_10 \textcolor{red}{\textcjheb{ydmlmlw}} WLMLMDJ $|$und meinen Lehrern/und zu meinen Lehrenden\\
6.&86.&924.&352.&3681.&24.&2&31&30\_1 \textcolor{red}{\textcjheb{'l}} LA $|$nicht\\
7.&87.&925.&354.&3683.&26.&5&434&5\_9\_10\_400\_10 \textcolor{red}{\textcjheb{yty.th}} HtJTJ $|$zugeneigt/ich neigte\\
8.&88.&926.&359.&3688.&31.&4&68&1\_7\_50\_10 \textcolor{red}{\textcjheb{ynz'}} AZNJ $|$mein Ohr\\
\end{tabular}\medskip \\
Ende des Verses 5.13\\
Verse: 130, Buchstaben: 34, 362, 3691, Totalwerte: 1944, 29619, 265087\\
\\
Und ich habe nicht geh"ort auf die Stimme meiner Unterweiser, und mein Ohr nicht zugeneigt meinen Lehrern.\\
\newpage 
{\bf -- 5.14}\\
\medskip \\
\begin{tabular}{rrrrrrrrp{120mm}}
WV&WK&WB&ABK&ABB&ABV&AnzB&TW&Zahlencode \textcolor{red}{$\boldsymbol{Grundtext}$} Umschrift $|$"Ubersetzung(en)\\
1.&89.&927.&363.&3692.&1.&4&139&20\_40\_70\_9 \textcolor{red}{\textcjheb{.t`mk}} KMat $|$wenig fehlte/beinahe\\
2.&90.&928.&367.&3696.&5.&5&435&5\_10\_10\_400\_10 \textcolor{red}{\textcjheb{ytyyh}} HJJTJ $|$so w"are ich gewesen/ich w"are geraten\\
3.&91.&929.&372.&3701.&10.&3&52&2\_20\_30 \textcolor{red}{\textcjheb{lkb}} BKL $|$in all(em)\\
4.&92.&930.&375.&3704.&13.&2&270&200\_70 \textcolor{red}{\textcjheb{`r}} Ra $|$B"osen/Ungl"uck\\
5.&93.&931.&377.&3706.&15.&4&428&2\_400\_6\_20 \textcolor{red}{\textcjheb{kwtb}} BTWK $|$inmitten\\
6.&94.&932.&381.&3710.&19.&3&135&100\_5\_30 \textcolor{red}{\textcjheb{lhq}} QHL $|$(der) Versammlung\\
7.&95.&933.&384.&3713.&22.&4&85&6\_70\_4\_5 \textcolor{red}{\textcjheb{hd`w}} WaDH $|$und (in) der Gemeinde\\
\end{tabular}\medskip \\
Ende des Verses 5.14\\
Verse: 131, Buchstaben: 25, 387, 3716, Totalwerte: 1544, 31163, 266631\\
\\
Wenig fehlte, so w"are ich in allem B"osen gewesen, inmitten der Versammlung und der Gemeinde.\\
\newpage 
{\bf -- 5.15}\\
\medskip \\
\begin{tabular}{rrrrrrrrp{120mm}}
WV&WK&WB&ABK&ABB&ABV&AnzB&TW&Zahlencode \textcolor{red}{$\boldsymbol{Grundtext}$} Umschrift $|$"Ubersetzung(en)\\
1.&96.&934.&388.&3717.&1.&3&705&300\_400\_5 \textcolor{red}{\textcjheb{ht+s}} STH $|$trinke\\
2.&97.&935.&391.&3720.&4.&3&90&40\_10\_40 \textcolor{red}{\textcjheb{mym}} MJM $|$Wasser\\
3.&98.&936.&394.&3723.&7.&5&268&40\_2\_6\_200\_20 \textcolor{red}{\textcjheb{krwbm}} MBWRK $|$aus deiner Zisterne\\
4.&99.&937.&399.&3728.&12.&6&143&6\_50\_7\_30\_10\_40 \textcolor{red}{\textcjheb{mylznw}} WNZLJM $|$und Flie"sendes\\
5.&100.&938.&405.&3734.&18.&4&466&40\_400\_6\_20 \textcolor{red}{\textcjheb{kwtm}} MTWK $|$aus/aus den Innern\\
6.&101.&939.&409.&3738.&22.&4&223&2\_1\_200\_20 \textcolor{red}{\textcjheb{kr'b}} BARK $|$deinem Brunnen/deines Brunnens\\
\end{tabular}\medskip \\
Ende des Verses 5.15\\
Verse: 132, Buchstaben: 25, 412, 3741, Totalwerte: 1895, 33058, 268526\\
\\
Trinke Wasser aus deiner Zisterne und Flie"sendes aus deinem Brunnen.\\
\newpage 
{\bf -- 5.16}\\
\medskip \\
\begin{tabular}{rrrrrrrrp{120mm}}
WV&WK&WB&ABK&ABB&ABV&AnzB&TW&Zahlencode \textcolor{red}{$\boldsymbol{Grundtext}$} Umschrift $|$"Ubersetzung(en)\\
1.&102.&940.&413.&3742.&1.&5&192&10\_80\_6\_90\_6 \textcolor{red}{\textcjheb{w.swpy}} JPW"sW $|$m"ogen sich ergie"sen/sie (=es) sollen sich ausbreiten\\
2.&103.&941.&418.&3747.&6.&7&600&40\_70\_10\_50\_400\_10\_20 \textcolor{red}{\textcjheb{kytny`m}} MaJNTJK $|$deine Quellen\\
3.&104.&942.&425.&3754.&13.&4&109&8\_6\_90\_5 \textcolor{red}{\textcjheb{h.sw.h}} CW"sH $|$nach (dr)au"sen (hin)\\
4.&105.&943.&429.&3758.&17.&6&618&2\_200\_8\_2\_6\_400 \textcolor{red}{\textcjheb{twb.hrb}} BRCBWT $|$auf die Stra"sen/in die offenen Pl"atze\\
5.&106.&944.&435.&3764.&23.&4&123&80\_30\_3\_10 \textcolor{red}{\textcjheb{yglp}} PLGJ $|$(deine) B"ache\\
6.&107.&945.&439.&3768.&27.&3&90&40\_10\_40 \textcolor{red}{\textcjheb{mym}} MJM $|$(von) Wasser\\
\end{tabular}\medskip \\
Ende des Verses 5.16\\
Verse: 133, Buchstaben: 29, 441, 3770, Totalwerte: 1732, 34790, 270258\\
\\
M"ogen nach au"sen sich ergie"sen deine Quellen, deine Wasserb"ache auf die Stra"sen.\\
\newpage 
{\bf -- 5.17}\\
\medskip \\
\begin{tabular}{rrrrrrrrp{120mm}}
WV&WK&WB&ABK&ABB&ABV&AnzB&TW&Zahlencode \textcolor{red}{$\boldsymbol{Grundtext}$} Umschrift $|$"Ubersetzung(en)\\
1.&108.&946.&442.&3771.&1.&4&31&10\_5\_10\_6 \textcolor{red}{\textcjheb{wyhy}} JHJW $|$geh"oren sollen sie/sie sollen sein\\
2.&109.&947.&446.&3775.&5.&2&50&30\_20 \textcolor{red}{\textcjheb{kl}} LK $|$(zu) dir\\
3.&110.&948.&448.&3777.&7.&4&56&30\_2\_4\_20 \textcolor{red}{\textcjheb{kdbl}} LBDK $|$(zu) (dir) allein\\
4.&111.&949.&452.&3781.&11.&4&67&6\_1\_10\_50 \textcolor{red}{\textcjheb{ny'w}} WAJN $|$und nicht\\
5.&112.&950.&456.&3785.&15.&5&287&30\_7\_200\_10\_40 \textcolor{red}{\textcjheb{myrzl}} LZRJM $|$(den) Fremden\\
6.&113.&951.&461.&3790.&20.&3&421&1\_400\_20 \textcolor{red}{\textcjheb{kt'}} ATK $|$mit dir/bei dir\\
\end{tabular}\medskip \\
Ende des Verses 5.17\\
Verse: 134, Buchstaben: 22, 463, 3792, Totalwerte: 912, 35702, 271170\\
\\
Dir allein sollen sie geh"oren, und nicht Fremden mit dir.\\
\newpage 
{\bf -- 5.18}\\
\medskip \\
\begin{tabular}{rrrrrrrrp{120mm}}
WV&WK&WB&ABK&ABB&ABV&AnzB&TW&Zahlencode \textcolor{red}{$\boldsymbol{Grundtext}$} Umschrift $|$"Ubersetzung(en)\\
1.&114.&952.&464.&3793.&1.&3&25&10\_5\_10 \textcolor{red}{\textcjheb{yhy}} JHJ $|$(sie (=es)) sei\\
2.&115.&953.&467.&3796.&4.&5&366&40\_100\_6\_200\_20 \textcolor{red}{\textcjheb{krwqm}} MQWRK $|$deine Quelle\\
3.&116.&954.&472.&3801.&9.&4&228&2\_200\_6\_20 \textcolor{red}{\textcjheb{kwrb}} BRWK $|$gesegnet\\
4.&117.&955.&476.&3805.&13.&4&354&6\_300\_40\_8 \textcolor{red}{\textcjheb{.hm+sw}} WSMC $|$und (er)freue dich\\
5.&118.&956.&480.&3809.&17.&4&741&40\_1\_300\_400 \textcolor{red}{\textcjheb{t+s'm}} MAST $|$an der Frau\\
6.&119.&957.&484.&3813.&21.&5&346&50\_70\_6\_200\_20 \textcolor{red}{\textcjheb{krw`n}} NaWRK $|$deiner Jugend(zeit)\\
\end{tabular}\medskip \\
Ende des Verses 5.18\\
Verse: 135, Buchstaben: 25, 488, 3817, Totalwerte: 2060, 37762, 273230\\
\\
Deine Quelle sei gesegnet, und erfreue dich an dem Weibe deiner Jugend;\\
\newpage 
{\bf -- 5.19}\\
\medskip \\
\begin{tabular}{rrrrrrrrp{120mm}}
WV&WK&WB&ABK&ABB&ABV&AnzB&TW&Zahlencode \textcolor{red}{$\boldsymbol{Grundtext}$} Umschrift $|$"Ubersetzung(en)\\
1.&120.&958.&489.&3818.&1.&4&441&1\_10\_30\_400 \textcolor{red}{\textcjheb{tly'}} AJLT $|$(die) Hindin\\
2.&121.&959.&493.&3822.&5.&5&58&1\_5\_2\_10\_40 \textcolor{red}{\textcjheb{mybh'}} AHBJM $|$liebliche\\
3.&122.&960.&498.&3827.&10.&5&516&6\_10\_70\_30\_400 \textcolor{red}{\textcjheb{tl`yw}} WJaLT $|$und (die) Gemse\\
4.&123.&961.&503.&3832.&15.&2&58&8\_50 \textcolor{red}{\textcjheb{n.h}} CN $|$anmutige/(der) Anmut\\
5.&124.&962.&505.&3834.&17.&4&23&4\_4\_10\_5 \textcolor{red}{\textcjheb{hydd}} DDJH $|$ihre Br"uste\\
6.&125.&963.&509.&3838.&21.&4&236&10\_200\_6\_20 \textcolor{red}{\textcjheb{kwry}} JRWK $|$m"ogen berauschen dich/sie erquicken dich\\
7.&126.&964.&513.&3842.&25.&3&52&2\_20\_30 \textcolor{red}{\textcjheb{lkb}} BKL $|$zu aller/in aller\\
8.&127.&965.&516.&3845.&28.&2&470&70\_400 \textcolor{red}{\textcjheb{t`}} aT $|$Zeit\\
9.&128.&966.&518.&3847.&30.&6&415&2\_1\_5\_2\_400\_5 \textcolor{red}{\textcjheb{htbh'b}} BAHBTH $|$in ihrer Liebe/ob ihrer Liebe\\
10.&129.&967.&524.&3853.&36.&4&708&400\_300\_3\_5 \textcolor{red}{\textcjheb{hg+st}} TSGH $|$taumle/du bist berauscht\\
11.&130.&968.&528.&3857.&40.&4&454&400\_40\_10\_4 \textcolor{red}{\textcjheb{dymt}} TMJD $|$stets/best"andig\\
\end{tabular}\medskip \\
Ende des Verses 5.19\\
Verse: 136, Buchstaben: 43, 531, 3860, Totalwerte: 3431, 41193, 276661\\
\\
die liebliche Hindin und anmutige Gemse-ihre Br"uste m"ogen dich berauschen zu aller Zeit, taumle stets in ihrer Liebe.\\
\newpage 
{\bf -- 5.20}\\
\medskip \\
\begin{tabular}{rrrrrrrrp{120mm}}
WV&WK&WB&ABK&ABB&ABV&AnzB&TW&Zahlencode \textcolor{red}{$\boldsymbol{Grundtext}$} Umschrift $|$"Ubersetzung(en)\\
1.&131.&969.&532.&3861.&1.&4&81&6\_30\_40\_5 \textcolor{red}{\textcjheb{hmlw}} WLMH $|$und warum\\
2.&132.&970.&536.&3865.&5.&4&708&400\_300\_3\_5 \textcolor{red}{\textcjheb{hg+st}} TSGH $|$solltest du taumeln/du willst berauschen dich\\
3.&133.&971.&540.&3869.&9.&3&62&2\_50\_10 \textcolor{red}{\textcjheb{ynb}} BNJ $|$mein Sohn\\
4.&134.&972.&543.&3872.&12.&4&214&2\_7\_200\_5 \textcolor{red}{\textcjheb{hrzb}} BZRH $|$an einer Fremden\\
5.&135.&973.&547.&3876.&16.&5&516&6\_400\_8\_2\_100 \textcolor{red}{\textcjheb{qb.htw}} WTCBQ $|$und umarmen/und umarmst\\
6.&136.&974.&552.&3881.&21.&2&108&8\_100 \textcolor{red}{\textcjheb{q.h}} CQ $|$den Busen\\
7.&137.&975.&554.&3883.&23.&5&285&50\_20\_200\_10\_5 \textcolor{red}{\textcjheb{hyrkn}} NKRJH $|$einer Fremden/(einer) Ausw"artigen\\
\end{tabular}\medskip \\
Ende des Verses 5.20\\
Verse: 137, Buchstaben: 27, 558, 3887, Totalwerte: 1974, 43167, 278635\\
\\
Und warum solltest du, mein Sohn, an einer Fremden taumeln, und den Busen einer Fremden umarmen? -\\
\newpage 
{\bf -- 5.21}\\
\medskip \\
\begin{tabular}{rrrrrrrrp{120mm}}
WV&WK&WB&ABK&ABB&ABV&AnzB&TW&Zahlencode \textcolor{red}{$\boldsymbol{Grundtext}$} Umschrift $|$"Ubersetzung(en)\\
1.&138.&976.&559.&3888.&1.&2&30&20\_10 \textcolor{red}{\textcjheb{yk}} KJ $|$denn\\
2.&139.&977.&561.&3890.&3.&3&78&50\_20\_8 \textcolor{red}{\textcjheb{.hkn}} NKC $|$vor/gegen"uber\\
3.&140.&978.&564.&3893.&6.&4&140&70\_10\_50\_10 \textcolor{red}{\textcjheb{yny`}} aJNJ $|$den Augen\\
4.&141.&979.&568.&3897.&10.&4&26&10\_5\_6\_5 \textcolor{red}{\textcjheb{hwhy}} JHWH $|$Jahwe(s)\\
5.&142.&980.&572.&3901.&14.&4&234&4\_200\_20\_10 \textcolor{red}{\textcjheb{ykrd}} DRKJ $|$(sind) (die) Wege\\
6.&143.&981.&576.&3905.&18.&3&311&1\_10\_300 \textcolor{red}{\textcjheb{+sy'}} AJS $|$eines jeden/(eines) Mannes\\
7.&144.&982.&579.&3908.&21.&3&56&6\_20\_30 \textcolor{red}{\textcjheb{lkw}} WKL $|$und alle\\
8.&145.&983.&582.&3911.&24.&7&559&40\_70\_3\_30\_400\_10\_6 \textcolor{red}{\textcjheb{wytlg`m}} MaGLTJW $|$seine Geleise/seine Bahnen\\
9.&146.&984.&589.&3918.&31.&4&210&40\_80\_30\_60 \textcolor{red}{\textcjheb{slpm}} MPLs $|$w"agt er ab/(ist er) ebnend\\
\end{tabular}\medskip \\
Ende des Verses 5.21\\
Verse: 138, Buchstaben: 34, 592, 3921, Totalwerte: 1644, 44811, 280279\\
\\
Denn vor den Augen Jahwes sind eines jeden Wege, und alle seine Geleise w"agt er ab.\\
\newpage 
{\bf -- 5.22}\\
\medskip \\
\begin{tabular}{rrrrrrrrp{120mm}}
WV&WK&WB&ABK&ABB&ABV&AnzB&TW&Zahlencode \textcolor{red}{$\boldsymbol{Grundtext}$} Umschrift $|$"Ubersetzung(en)\\
1.&147.&985.&593.&3922.&1.&8&554&70\_6\_6\_50\_6\_400\_10\_6 \textcolor{red}{\textcjheb{wytwnww`}} aWWNWTJW $|$die eigenen Missetaten/seine Missetaten\\
2.&148.&986.&601.&3930.&9.&6&120&10\_30\_20\_4\_50\_6 \textcolor{red}{\textcjheb{wndkly}} JLKDNW $|$werden ihn fangen/(sie) fangen ihn\\
3.&149.&987.&607.&3936.&15.&2&401&1\_400 \textcolor{red}{\textcjheb{t'}} AT $|$**\\
4.&150.&988.&609.&3938.&17.&4&575&5\_200\_300\_70 \textcolor{red}{\textcjheb{`+srh}} HRSa $|$den Gesetzlosen/den Frevler\\
5.&151.&989.&613.&3942.&21.&6&58&6\_2\_8\_2\_30\_10 \textcolor{red}{\textcjheb{ylb.hbw}} WBCBLJ $|$und in Banden/und durch die Stricke\\
6.&152.&990.&619.&3948.&27.&5&424&8\_9\_1\_400\_6 \textcolor{red}{\textcjheb{wt'.t.h}} CtATW $|$seiner S"unde(n)\\
7.&153.&991.&624.&3953.&32.&4&470&10\_400\_40\_20 \textcolor{red}{\textcjheb{kmty}} JTMK $|$er wird festgehalten (werden)\\
\end{tabular}\medskip \\
Ende des Verses 5.22\\
Verse: 139, Buchstaben: 35, 627, 3956, Totalwerte: 2602, 47413, 282881\\
\\
Die eigenen Missetaten werden ihn, den Gesetzlosen, fangen, und in seiner S"unde Banden wird er festgehalten werden.\\
\newpage 
{\bf -- 5.23}\\
\medskip \\
\begin{tabular}{rrrrrrrrp{120mm}}
WV&WK&WB&ABK&ABB&ABV&AnzB&TW&Zahlencode \textcolor{red}{$\boldsymbol{Grundtext}$} Umschrift $|$"Ubersetzung(en)\\
1.&154.&992.&628.&3957.&1.&3&12&5\_6\_1 \textcolor{red}{\textcjheb{'wh}} HWA $|$er\\
2.&155.&993.&631.&3960.&4.&4&456&10\_40\_6\_400 \textcolor{red}{\textcjheb{twmy}} JMWT $|$wird sterben/er stirbt\\
3.&156.&994.&635.&3964.&8.&4&63&2\_1\_10\_50 \textcolor{red}{\textcjheb{ny'b}} BAJN $|$weil ihm mangelt/aus Mangel an\\
4.&157.&995.&639.&3968.&12.&4&306&40\_6\_60\_200 \textcolor{red}{\textcjheb{rswm}} MWsR $|$Zucht\\
5.&158.&996.&643.&3972.&16.&4&210&6\_2\_200\_2 \textcolor{red}{\textcjheb{brbw}} WBRB $|$und in der Gr"o"se/und wegen der Menge\\
6.&159.&997.&647.&3976.&20.&5&443&1\_6\_30\_400\_6 \textcolor{red}{\textcjheb{wtlw'}} AWLTW $|$seiner Torheit\\
7.&160.&998.&652.&3981.&25.&4&318&10\_300\_3\_5 \textcolor{red}{\textcjheb{hg+sy}} JSGH $|$wird er dahintaumeln/er irrt umher\\
\end{tabular}\medskip \\
Ende des Verses 5.23\\
Verse: 140, Buchstaben: 28, 655, 3984, Totalwerte: 1808, 49221, 284689\\
\\
Sterben wird er, weil ihm Zucht mangelt, und in der Gr"o"se seiner Torheit wird er dahintaumeln.\\
\\
{\bf Ende des Kapitels 5}\\
\newpage 
{\bf -- 6.1}\\
\medskip \\
\begin{tabular}{rrrrrrrrp{120mm}}
WV&WK&WB&ABK&ABB&ABV&AnzB&TW&Zahlencode \textcolor{red}{$\boldsymbol{Grundtext}$} Umschrift $|$"Ubersetzung(en)\\
1.&1.&999.&1.&3985.&1.&3&62&2\_50\_10 \textcolor{red}{\textcjheb{ynb}} BNJ $|$mein Sohn\\
2.&2.&1000.&4.&3988.&4.&2&41&1\_40 \textcolor{red}{\textcjheb{m'}} AM $|$wenn\\
3.&3.&1001.&6.&3990.&6.&4&672&70\_200\_2\_400 \textcolor{red}{\textcjheb{tbr`}} aRBT $|$du B"urge geworden bist/du hast geb"urgt\\
4.&4.&1002.&10.&3994.&10.&4&320&30\_200\_70\_20 \textcolor{red}{\textcjheb{k`rl}} LRaK $|$(f"ur) deinen N"achsten\\
5.&5.&1003.&14.&3998.&14.&4&970&400\_100\_70\_400 \textcolor{red}{\textcjheb{t`qt}} TQaT $|$hast eingeschlagen\\
6.&6.&1004.&18.&4002.&18.&3&237&30\_7\_200 \textcolor{red}{\textcjheb{rzl}} LZR $|$f"ur einen anderen/(f"ur) einen Fremden\\
7.&7.&1005.&21.&4005.&21.&4&130&20\_80\_10\_20 \textcolor{red}{\textcjheb{kypk}} KPJK $|$deine Hand/(mit) deinen H"anden\\
\end{tabular}\medskip \\
Ende des Verses 6.1\\
Verse: 141, Buchstaben: 24, 24, 4008, Totalwerte: 2432, 2432, 287121\\
\\
Mein Sohn, wenn du B"urge geworden bist f"ur deinen N"achsten, f"ur einen anderen deine Hand eingeschlagen hast;\\
\newpage 
{\bf -- 6.2}\\
\medskip \\
\begin{tabular}{rrrrrrrrp{120mm}}
WV&WK&WB&ABK&ABB&ABV&AnzB&TW&Zahlencode \textcolor{red}{$\boldsymbol{Grundtext}$} Umschrift $|$"Ubersetzung(en)\\
1.&8.&1006.&25.&4009.&1.&5&856&50\_6\_100\_300\_400 \textcolor{red}{\textcjheb{t+sqwn}} NWQST $|$du bist verstrickt\\
2.&9.&1007.&30.&4014.&6.&5&253&2\_1\_40\_200\_10 \textcolor{red}{\textcjheb{yrm'b}} BAMRJ $|$durch die Worte/ob der Worte\\
3.&10.&1008.&35.&4019.&11.&3&110&80\_10\_20 \textcolor{red}{\textcjheb{kyp}} PJK $|$deines Mundes\\
4.&11.&1009.&38.&4022.&14.&5&504&50\_30\_20\_4\_400 \textcolor{red}{\textcjheb{tdkln}} NLKDT $|$(du bist) gefangen\\
5.&12.&1010.&43.&4027.&19.&5&253&2\_1\_40\_200\_10 \textcolor{red}{\textcjheb{yrm'b}} BAMRJ $|$durch die Worte/ob der Worte\\
6.&13.&1011.&48.&4032.&24.&3&110&80\_10\_20 \textcolor{red}{\textcjheb{kyp}} PJK $|$deines Mundes\\
\end{tabular}\medskip \\
Ende des Verses 6.2\\
Verse: 142, Buchstaben: 26, 50, 4034, Totalwerte: 2086, 4518, 289207\\
\\
bist du verstrickt durch die Worte deines Mundes, gefangen durch die Worte deines Mundes:\\
\newpage 
{\bf -- 6.3}\\
\medskip \\
\begin{tabular}{rrrrrrrrp{120mm}}
WV&WK&WB&ABK&ABB&ABV&AnzB&TW&Zahlencode \textcolor{red}{$\boldsymbol{Grundtext}$} Umschrift $|$"Ubersetzung(en)\\
1.&14.&1012.&51.&4035.&1.&3&375&70\_300\_5 \textcolor{red}{\textcjheb{h+s`}} aSH $|$tue\\
2.&15.&1013.&54.&4038.&4.&3&408&7\_1\_400 \textcolor{red}{\textcjheb{t'z}} ZAT $|$dies\\
3.&16.&1014.&57.&4041.&7.&4&88&1\_80\_6\_1 \textcolor{red}{\textcjheb{'wp'}} APWA $|$denn/also\\
4.&17.&1015.&61.&4045.&11.&3&62&2\_50\_10 \textcolor{red}{\textcjheb{ynb}} BNJ $|$mein Sohn\\
5.&18.&1016.&64.&4048.&14.&5&181&6\_5\_50\_90\_30 \textcolor{red}{\textcjheb{l.snhw}} WHN"sL $|$und rei"se dich los/und rette dich\\
6.&19.&1017.&69.&4053.&19.&2&30&20\_10 \textcolor{red}{\textcjheb{yk}} KJ $|$da/denn\\
7.&20.&1018.&71.&4055.&21.&3&403&2\_1\_400 \textcolor{red}{\textcjheb{t'b}} BAT $|$du bist gekommen\\
8.&21.&1019.&74.&4058.&24.&3&102&2\_20\_80 \textcolor{red}{\textcjheb{pkb}} BKP $|$in die Hand/in die Faust\\
9.&22.&1020.&77.&4061.&27.&3&290&200\_70\_20 \textcolor{red}{\textcjheb{k`r}} RaK $|$deines N"achsten\\
10.&23.&1021.&80.&4064.&30.&2&50&30\_20 \textcolor{red}{\textcjheb{kl}} LK $|$geh (hin)\\
11.&24.&1022.&82.&4066.&32.&5&745&5\_400\_200\_80\_60 \textcolor{red}{\textcjheb{sprth}} HTRPs $|$wirf dich nieder\\
12.&25.&1023.&87.&4071.&37.&4&213&6\_200\_5\_2 \textcolor{red}{\textcjheb{bhrw}} WRHB $|$und best"urme\\
13.&26.&1024.&91.&4075.&41.&4&300&200\_70\_10\_20 \textcolor{red}{\textcjheb{ky`r}} RaJK $|$deine(n) N"achsten\\
\end{tabular}\medskip \\
Ende des Verses 6.3\\
Verse: 143, Buchstaben: 44, 94, 4078, Totalwerte: 3247, 7765, 292454\\
\\
tue denn dieses, mein Sohn, und rei"se dich los, da du in deines N"achsten Hand gekommen bist; geh hin, wirf dich nieder, und best"urme deinen N"achsten;\\
\newpage 
{\bf -- 6.4}\\
\medskip \\
\begin{tabular}{rrrrrrrrp{120mm}}
WV&WK&WB&ABK&ABB&ABV&AnzB&TW&Zahlencode \textcolor{red}{$\boldsymbol{Grundtext}$} Umschrift $|$"Ubersetzung(en)\\
1.&27.&1025.&95.&4079.&1.&2&31&1\_30 \textcolor{red}{\textcjheb{l'}} AL $|$nicht\\
2.&28.&1026.&97.&4081.&3.&3&850&400\_400\_50 \textcolor{red}{\textcjheb{ntt}} TTN $|$gestatte/sollst du geben\\
3.&29.&1027.&100.&4084.&6.&3&355&300\_50\_5 \textcolor{red}{\textcjheb{hn+s}} SNH $|$Schlaf\\
4.&30.&1028.&103.&4087.&9.&6&190&30\_70\_10\_50\_10\_20 \textcolor{red}{\textcjheb{kyny`l}} LaJNJK $|$(in) deinen Augen\\
5.&31.&1029.&109.&4093.&15.&6&507&6\_400\_50\_6\_40\_5 \textcolor{red}{\textcjheb{hmwntw}} WTNWMH $|$und (nicht) Schlummer\\
6.&32.&1030.&115.&4099.&21.&7&360&30\_70\_80\_70\_80\_10\_20 \textcolor{red}{\textcjheb{kyp`p`l}} LaPaPJK $|$(in) deine(n) Wimpern\\
\end{tabular}\medskip \\
Ende des Verses 6.4\\
Verse: 144, Buchstaben: 27, 121, 4105, Totalwerte: 2293, 10058, 294747\\
\\
gestatte deinen Augen keinen Schlaf, und keinen Schlummer deinen Wimpern;\\
\newpage 
{\bf -- 6.5}\\
\medskip \\
\begin{tabular}{rrrrrrrrp{120mm}}
WV&WK&WB&ABK&ABB&ABV&AnzB&TW&Zahlencode \textcolor{red}{$\boldsymbol{Grundtext}$} Umschrift $|$"Ubersetzung(en)\\
1.&33.&1031.&122.&4106.&1.&4&175&5\_50\_90\_30 \textcolor{red}{\textcjheb{l.snh}} HN"sL $|$rei"se dich los/rette dich\\
2.&34.&1032.&126.&4110.&5.&4&122&20\_90\_2\_10 \textcolor{red}{\textcjheb{yb.sk}} K"sBJ $|$wie eine Gazelle\\
3.&35.&1033.&130.&4114.&9.&3&54&40\_10\_4 \textcolor{red}{\textcjheb{dym}} MJD $|$aus der Hand\\
4.&36.&1034.&133.&4117.&12.&6&402&6\_20\_90\_80\_6\_200 \textcolor{red}{\textcjheb{rwp.skw}} WK"sPWR $|$und wie ein Vogel\\
5.&37.&1035.&139.&4123.&18.&3&54&40\_10\_4 \textcolor{red}{\textcjheb{dym}} MJD $|$aus der Hand\\
6.&38.&1036.&142.&4126.&21.&4&416&10\_100\_6\_300 \textcolor{red}{\textcjheb{+swqy}} JQWS $|$des Vogelstellers\\
\end{tabular}\medskip \\
Ende des Verses 6.5\\
Verse: 145, Buchstaben: 24, 145, 4129, Totalwerte: 1223, 11281, 295970\\
\\
rei"se dich los wie eine Gazelle aus der Hand, und wie ein Vogel aus der Hand des Vogelstellers.\\
\newpage 
{\bf -- 6.6}\\
\medskip \\
\begin{tabular}{rrrrrrrrp{120mm}}
WV&WK&WB&ABK&ABB&ABV&AnzB&TW&Zahlencode \textcolor{red}{$\boldsymbol{Grundtext}$} Umschrift $|$"Ubersetzung(en)\\
1.&39.&1037.&146.&4130.&1.&2&50&30\_20 \textcolor{red}{\textcjheb{kl}} LK $|$geh (hin)\\
2.&40.&1038.&148.&4132.&3.&2&31&1\_30 \textcolor{red}{\textcjheb{l'}} AL $|$zur\\
3.&41.&1039.&150.&4134.&5.&4&125&50\_40\_30\_5 \textcolor{red}{\textcjheb{hlmn}} NMLH $|$Ameise\\
4.&42.&1040.&154.&4138.&9.&3&190&70\_90\_30 \textcolor{red}{\textcjheb{l.s`}} a"sL $|$du Fauler/Faulpelz\\
5.&43.&1041.&157.&4141.&12.&3&206&200\_1\_5 \textcolor{red}{\textcjheb{h'r}} RAH $|$sieh (an)\\
6.&44.&1042.&160.&4144.&15.&5&239&4\_200\_20\_10\_5 \textcolor{red}{\textcjheb{hykrd}} DRKJH $|$ihre Wege\\
7.&45.&1043.&165.&4149.&20.&4&74&6\_8\_20\_40 \textcolor{red}{\textcjheb{mk.hw}} WCKM $|$und werde weise\\
\end{tabular}\medskip \\
Ende des Verses 6.6\\
Verse: 146, Buchstaben: 23, 168, 4152, Totalwerte: 915, 12196, 296885\\
\\
Geh hin zur Ameise, du Fauler, sieh ihre Wege und werde weise.\\
\newpage 
{\bf -- 6.7}\\
\medskip \\
\begin{tabular}{rrrrrrrrp{120mm}}
WV&WK&WB&ABK&ABB&ABV&AnzB&TW&Zahlencode \textcolor{red}{$\boldsymbol{Grundtext}$} Umschrift $|$"Ubersetzung(en)\\
1.&46.&1044.&169.&4153.&1.&3&501&1\_300\_200 \textcolor{red}{\textcjheb{r+s'}} ASR $|$die\\
2.&47.&1045.&172.&4156.&4.&3&61&1\_10\_50 \textcolor{red}{\textcjheb{ny'}} AJN $|$nicht hat\\
3.&48.&1046.&175.&4159.&7.&2&35&30\_5 \textcolor{red}{\textcjheb{hl}} LH $|$sie/f"ur sich\\
4.&49.&1047.&177.&4161.&9.&4&250&100\_90\_10\_50 \textcolor{red}{\textcjheb{ny.sq}} Q"sJN $|$einen Richter/Vorsteher\\
5.&50.&1048.&181.&4165.&13.&3&509&300\_9\_200 \textcolor{red}{\textcjheb{r.t+s}} StR $|$Vorsteher/Antreibenden\\
6.&51.&1049.&184.&4168.&16.&4&376&6\_40\_300\_30 \textcolor{red}{\textcjheb{l+smw}} WMSL $|$und Gebieter/und Herrschenden\\
\end{tabular}\medskip \\
Ende des Verses 6.7\\
Verse: 147, Buchstaben: 19, 187, 4171, Totalwerte: 1732, 13928, 298617\\
\\
Sie, die keinen Richter, Vorsteher und Gebieter hat,\\
\newpage 
{\bf -- 6.8}\\
\medskip \\
\begin{tabular}{rrrrrrrrp{120mm}}
WV&WK&WB&ABK&ABB&ABV&AnzB&TW&Zahlencode \textcolor{red}{$\boldsymbol{Grundtext}$} Umschrift $|$"Ubersetzung(en)\\
1.&52.&1050.&188.&4172.&1.&4&480&400\_20\_10\_50 \textcolor{red}{\textcjheb{nykt}} TKJN $|$sie bereitet\\
2.&53.&1051.&192.&4176.&5.&4&202&2\_100\_10\_90 \textcolor{red}{\textcjheb{.syqb}} BQJ"s $|$im Sommer\\
3.&54.&1052.&196.&4180.&9.&4&83&30\_8\_40\_5 \textcolor{red}{\textcjheb{hm.hl}} LCMH $|$ihr Brot\\
4.&55.&1053.&200.&4184.&13.&4&209&1\_3\_200\_5 \textcolor{red}{\textcjheb{hrg'}} AGRH $|$hat eingesammelt/sie sammelt ein\\
5.&56.&1054.&204.&4188.&17.&5&402&2\_100\_90\_10\_200 \textcolor{red}{\textcjheb{ry.sqb}} BQ"sJR $|$in der Ernte\\
6.&57.&1055.&209.&4193.&22.&5&96&40\_1\_20\_30\_5 \textcolor{red}{\textcjheb{hlk'm}} MAKLH $|$ihre Nahrung/ihre Speise\\
\end{tabular}\medskip \\
Ende des Verses 6.8\\
Verse: 148, Buchstaben: 26, 213, 4197, Totalwerte: 1472, 15400, 300089\\
\\
sie bereitet im Sommer ihr Brot, hat in der Ernte ihre Nahrung eingesammelt.\\
\newpage 
{\bf -- 6.9}\\
\medskip \\
\begin{tabular}{rrrrrrrrp{120mm}}
WV&WK&WB&ABK&ABB&ABV&AnzB&TW&Zahlencode \textcolor{red}{$\boldsymbol{Grundtext}$} Umschrift $|$"Ubersetzung(en)\\
1.&58.&1056.&214.&4198.&1.&2&74&70\_4 \textcolor{red}{\textcjheb{d`}} aD $|$bis\\
2.&59.&1057.&216.&4200.&3.&3&450&40\_400\_10 \textcolor{red}{\textcjheb{ytm}} MTJ $|$wann\\
3.&60.&1058.&219.&4203.&6.&3&190&70\_90\_30 \textcolor{red}{\textcjheb{l.s`}} a"sL $|$du Fauler/Faulpelz\\
4.&61.&1059.&222.&4206.&9.&4&722&400\_300\_20\_2 \textcolor{red}{\textcjheb{bk+st}} TSKB $|$du willst liegen (bleiben)\\
5.&62.&1060.&226.&4210.&13.&3&450&40\_400\_10 \textcolor{red}{\textcjheb{ytm}} MTJ $|$wann\\
6.&63.&1061.&229.&4213.&16.&4&546&400\_100\_6\_40 \textcolor{red}{\textcjheb{mwqt}} TQWM $|$willst du aufstehen/wirst du aufstehen\\
7.&64.&1062.&233.&4217.&20.&5&810&40\_300\_50\_400\_20 \textcolor{red}{\textcjheb{ktn+sm}} MSNTK $|$von deinem Schlaf\\
\end{tabular}\medskip \\
Ende des Verses 6.9\\
Verse: 149, Buchstaben: 24, 237, 4221, Totalwerte: 3242, 18642, 303331\\
\\
Bis wann willst du liegen, du Fauler? Wann willst du von deinem Schlafe aufstehen?\\
\newpage 
{\bf -- 6.10}\\
\medskip \\
\begin{tabular}{rrrrrrrrp{120mm}}
WV&WK&WB&ABK&ABB&ABV&AnzB&TW&Zahlencode \textcolor{red}{$\boldsymbol{Grundtext}$} Umschrift $|$"Ubersetzung(en)\\
1.&65.&1063.&238.&4222.&1.&3&119&40\_70\_9 \textcolor{red}{\textcjheb{.t`m}} Mat $|$(ein) wenig\\
2.&66.&1064.&241.&4225.&4.&4&756&300\_50\_6\_400 \textcolor{red}{\textcjheb{twn+s}} SNWT $|$Schlaf\\
3.&67.&1065.&245.&4229.&8.&3&119&40\_70\_9 \textcolor{red}{\textcjheb{.t`m}} Mat $|$(ein) wenig\\
4.&68.&1066.&248.&4232.&11.&6&902&400\_50\_6\_40\_6\_400 \textcolor{red}{\textcjheb{twmwnt}} TNWMWT $|$Schlummer\\
5.&69.&1067.&254.&4238.&17.&3&119&40\_70\_9 \textcolor{red}{\textcjheb{.t`m}} Mat $|$(ein) wenig\\
6.&70.&1068.&257.&4241.&20.&3&110&8\_2\_100 \textcolor{red}{\textcjheb{qb.h}} CBQ $|$Falten/Verschr"anken\\
7.&71.&1069.&260.&4244.&23.&4&64&10\_4\_10\_40 \textcolor{red}{\textcjheb{mydy}} JDJM $|$(beider) H"ande\\
8.&72.&1070.&264.&4248.&27.&4&352&30\_300\_20\_2 \textcolor{red}{\textcjheb{bk+sl}} LSKB $|$um auszuruhen/zum Ruhen\\
\end{tabular}\medskip \\
Ende des Verses 6.10\\
Verse: 150, Buchstaben: 30, 267, 4251, Totalwerte: 2541, 21183, 305872\\
\\
Ein wenig Schlaf, ein wenig Schlummer, ein wenig H"andefalten, um auszuruhen:\\
\newpage 
{\bf -- 6.11}\\
\medskip \\
\begin{tabular}{rrrrrrrrp{120mm}}
WV&WK&WB&ABK&ABB&ABV&AnzB&TW&Zahlencode \textcolor{red}{$\boldsymbol{Grundtext}$} Umschrift $|$"Ubersetzung(en)\\
1.&73.&1071.&268.&4252.&1.&3&9&6\_2\_1 \textcolor{red}{\textcjheb{'bw}} WBA $|$und kommen wird/und kommend(er) (ist)\\
2.&74.&1072.&271.&4255.&4.&5&115&20\_40\_5\_30\_20 \textcolor{red}{\textcjheb{klhmk}} KMHLK $|$wie ein r"ustig Zuschreitender/wie ein Gehender\\
3.&75.&1073.&276.&4260.&9.&4&521&200\_1\_300\_20 \textcolor{red}{\textcjheb{k+s'r}} RASK $|$deine Armut\\
4.&76.&1074.&280.&4264.&13.&6&334&6\_40\_8\_60\_200\_20 \textcolor{red}{\textcjheb{krs.hmw}} WMCsRK $|$und deine Not/und dein Mangel\\
5.&77.&1075.&286.&4270.&19.&4&331&20\_1\_10\_300 \textcolor{red}{\textcjheb{+sy'k}} KAJS $|$wie (ein) Mann\\
6.&78.&1076.&290.&4274.&23.&3&93&40\_3\_50 \textcolor{red}{\textcjheb{ngm}} MGN $|$gewappneter/(mit) Schild\\
\end{tabular}\medskip \\
Ende des Verses 6.11\\
Verse: 151, Buchstaben: 25, 292, 4276, Totalwerte: 1403, 22586, 307275\\
\\
und deine Armut wird kommen wie ein r"ustig Zuschreitender, und deine Not wie ein gewappneter Mann.\\
\newpage 
{\bf -- 6.12}\\
\medskip \\
\begin{tabular}{rrrrrrrrp{120mm}}
WV&WK&WB&ABK&ABB&ABV&AnzB&TW&Zahlencode \textcolor{red}{$\boldsymbol{Grundtext}$} Umschrift $|$"Ubersetzung(en)\\
1.&79.&1077.&293.&4277.&1.&3&45&1\_4\_40 \textcolor{red}{\textcjheb{md'}} ADM $|$(ein) Mensch\\
2.&80.&1078.&296.&4280.&4.&5&142&2\_30\_10\_70\_30 \textcolor{red}{\textcjheb{l`ylb}} BLJaL $|$Belilas-/nichtsw"urdiger\\
3.&81.&1079.&301.&4285.&9.&3&311&1\_10\_300 \textcolor{red}{\textcjheb{+sy'}} AJS $|$(ist) (ein) Mann\\
4.&82.&1080.&304.&4288.&12.&3&57&1\_6\_50 \textcolor{red}{\textcjheb{nw'}} AWN $|$heilloser/des Frevels\\
5.&83.&1081.&307.&4291.&15.&4&61&5\_6\_30\_20 \textcolor{red}{\textcjheb{klwh}} HWLK $|$wer umhergeht/(ist) gehend(er)\\
6.&84.&1082.&311.&4295.&19.&5&876&70\_100\_300\_6\_400 \textcolor{red}{\textcjheb{tw+sq`}} aQSWT $|$mit Verkehrtheit/(mit) Falschheit\\
7.&85.&1083.&316.&4300.&24.&2&85&80\_5 \textcolor{red}{\textcjheb{hp}} PH $|$des Mundes/(im) Mund\\
\end{tabular}\medskip \\
Ende des Verses 6.12\\
Verse: 152, Buchstaben: 25, 317, 4301, Totalwerte: 1577, 24163, 308852\\
\\
Ein Belialsmensch, ein heilloser Mann ist, wer umhergeht mit Verkehrtheit des Mundes,\\
\newpage 
{\bf -- 6.13}\\
\medskip \\
\begin{tabular}{rrrrrrrrp{120mm}}
WV&WK&WB&ABK&ABB&ABV&AnzB&TW&Zahlencode \textcolor{red}{$\boldsymbol{Grundtext}$} Umschrift $|$"Ubersetzung(en)\\
1.&86.&1084.&318.&4302.&1.&3&390&100\_200\_90 \textcolor{red}{\textcjheb{.srq}} QR"s $|$zwinkt/(ist) zwinkernd\\
2.&87.&1085.&321.&4305.&4.&5&138&2\_70\_10\_50\_6 \textcolor{red}{\textcjheb{wny`b}} BaJNW $|$mit seinen Augen\\
3.&88.&1086.&326.&4310.&9.&3&100&40\_30\_30 \textcolor{red}{\textcjheb{llm}} MLL $|$scharrt/(ist) scharrend\\
4.&89.&1087.&329.&4313.&12.&5&241&2\_200\_3\_30\_6 \textcolor{red}{\textcjheb{wlgrb}} BRGLW $|$mit seinen F"u"sen\\
5.&90.&1088.&334.&4318.&17.&3&245&40\_200\_5 \textcolor{red}{\textcjheb{hrm}} MRH $|$deutet/(ist) weisend\\
6.&91.&1089.&337.&4321.&20.&8&581&2\_1\_90\_2\_70\_400\_10\_6 \textcolor{red}{\textcjheb{wyt`b.s'b}} BA"sBaTJW $|$mit seinen Fingern\\
\end{tabular}\medskip \\
Ende des Verses 6.13\\
Verse: 153, Buchstaben: 27, 344, 4328, Totalwerte: 1695, 25858, 310547\\
\\
mit seinen Augen zwinkt, mit seinen F"u"sen scharrt, mit seinen Fingern deutet.\\
\newpage 
{\bf -- 6.14}\\
\medskip \\
\begin{tabular}{rrrrrrrrp{120mm}}
WV&WK&WB&ABK&ABB&ABV&AnzB&TW&Zahlencode \textcolor{red}{$\boldsymbol{Grundtext}$} Umschrift $|$"Ubersetzung(en)\\
1.&92.&1090.&345.&4329.&1.&6&911&400\_5\_80\_20\_6\_400 \textcolor{red}{\textcjheb{twkpht}} THPKWT $|$Verkehrtheiten/(hat) R"anke\\
2.&93.&1091.&351.&4335.&7.&4&40&2\_30\_2\_6 \textcolor{red}{\textcjheb{wblb}} BLBW $|$(sind) in seinem Herzen\\
3.&94.&1092.&355.&4339.&11.&3&508&8\_200\_300 \textcolor{red}{\textcjheb{+sr.h}} CRS $|$er schmiedet/(ist) einschneidend\\
4.&95.&1093.&358.&4342.&14.&2&270&200\_70 \textcolor{red}{\textcjheb{`r}} Ra $|$B"oses/Unheil\\
5.&96.&1094.&360.&4344.&16.&3&52&2\_20\_30 \textcolor{red}{\textcjheb{lkb}} BKL $|$zu aller\\
6.&97.&1095.&363.&4347.&19.&2&470&70\_400 \textcolor{red}{\textcjheb{t`}} aT $|$Zeit\\
7.&98.&1096.&365.&4349.&21.&5&144&40\_4\_50\_10\_40 \textcolor{red}{\textcjheb{myndm}} MDNJM $|$Zwietracht/Z"ankereien\\
8.&99.&1097.&370.&4354.&26.&4&348&10\_300\_30\_8 \textcolor{red}{\textcjheb{.hl+sy}} JSLC $|$streut aus/er schickt\\
\end{tabular}\medskip \\
Ende des Verses 6.14\\
Verse: 154, Buchstaben: 29, 373, 4357, Totalwerte: 2743, 28601, 313290\\
\\
Verkehrtheiten sind in seinem Herzen; er schmiedet B"oses zu aller Zeit, streut Zwietracht aus.\\
\newpage 
{\bf -- 6.15}\\
\medskip \\
\begin{tabular}{rrrrrrrrp{120mm}}
WV&WK&WB&ABK&ABB&ABV&AnzB&TW&Zahlencode \textcolor{red}{$\boldsymbol{Grundtext}$} Umschrift $|$"Ubersetzung(en)\\
1.&100.&1098.&374.&4358.&1.&2&100&70\_30 \textcolor{red}{\textcjheb{l`}} aL $|$wegen\\
2.&101.&1099.&376.&4360.&3.&2&70&20\_50 \textcolor{red}{\textcjheb{nk}} KN $|$diesem\\
3.&102.&1100.&378.&4362.&5.&4&521&80\_400\_1\_40 \textcolor{red}{\textcjheb{m'tp}} PTAM $|$pl"otzlich\\
4.&103.&1101.&382.&4366.&9.&4&19&10\_2\_6\_1 \textcolor{red}{\textcjheb{'wby}} JBWA $|$wird kommen/er (=es) kommt\\
5.&104.&1102.&386.&4370.&13.&4&21&1\_10\_4\_6 \textcolor{red}{\textcjheb{wdy'}} AJDW $|$sein Verderben\\
6.&105.&1103.&390.&4374.&17.&3&550&80\_400\_70 \textcolor{red}{\textcjheb{`tp}} PTa $|$im Augenblick/augenblicklich\\
7.&106.&1104.&393.&4377.&20.&4&512&10\_300\_2\_200 \textcolor{red}{\textcjheb{rb+sy}} JSBR $|$wird er zerschmettert werden/er wird zerbrochen\\
8.&107.&1105.&397.&4381.&24.&4&67&6\_1\_10\_50 \textcolor{red}{\textcjheb{ny'w}} WAJN $|$ohne/und nicht gibt es\\
9.&108.&1106.&401.&4385.&28.&4&321&40\_200\_80\_1 \textcolor{red}{\textcjheb{'prm}} MRPA $|$Heilung\\
\end{tabular}\medskip \\
Ende des Verses 6.15\\
Verse: 155, Buchstaben: 31, 404, 4388, Totalwerte: 2181, 30782, 315471\\
\\
Darum wird pl"otzlich sein Verderben kommen; im Augenblick wird er zerschmettert werden ohne Heilung. -\\
\newpage 
{\bf -- 6.16}\\
\medskip \\
\begin{tabular}{rrrrrrrrp{120mm}}
WV&WK&WB&ABK&ABB&ABV&AnzB&TW&Zahlencode \textcolor{red}{$\boldsymbol{Grundtext}$} Umschrift $|$"Ubersetzung(en)\\
1.&109.&1107.&405.&4389.&1.&2&600&300\_300 \textcolor{red}{\textcjheb{+s+s}} SS $|$sechs\\
2.&110.&1108.&407.&4391.&3.&3&60&5\_50\_5 \textcolor{red}{\textcjheb{hnh}} HNH $|$sind es/sie (sind)\\
3.&111.&1109.&410.&4394.&6.&3&351&300\_50\_1 \textcolor{red}{\textcjheb{'n+s}} SNA $|$die hasst/er (=es) hasst\\
4.&112.&1110.&413.&4397.&9.&4&26&10\_5\_6\_5 \textcolor{red}{\textcjheb{hwhy}} JHWH $|$Jahwe\\
5.&113.&1111.&417.&4401.&13.&4&378&6\_300\_2\_70 \textcolor{red}{\textcjheb{`b+sw}} WSBa $|$und sieben\\
6.&114.&1112.&421.&4405.&17.&6&884&400\_6\_70\_2\_6\_400 \textcolor{red}{\textcjheb{twb`wt}} TWaBWT $|$sind (ein) Gr"auel\\
7.&115.&1113.&427.&4411.&23.&4&436&50\_80\_300\_6 \textcolor{red}{\textcjheb{w+spn}} NPSW $|$(f"ur) seine(r) Seele\\
\end{tabular}\medskip \\
Ende des Verses 6.16\\
Verse: 156, Buchstaben: 26, 430, 4414, Totalwerte: 2735, 33517, 318206\\
\\
Sechs sind es, die Jahwe ha"st, und sieben sind seiner Seele ein Greuel:\\
\newpage 
{\bf -- 6.17}\\
\medskip \\
\begin{tabular}{rrrrrrrrp{120mm}}
WV&WK&WB&ABK&ABB&ABV&AnzB&TW&Zahlencode \textcolor{red}{$\boldsymbol{Grundtext}$} Umschrift $|$"Ubersetzung(en)\\
1.&116.&1114.&431.&4415.&1.&5&180&70\_10\_50\_10\_40 \textcolor{red}{\textcjheb{myny`}} aJNJM $|$Augen\\
2.&117.&1115.&436.&4420.&6.&4&646&200\_40\_6\_400 \textcolor{red}{\textcjheb{twmr}} RMWT $|$hohe/stolze\\
3.&118.&1116.&440.&4424.&10.&4&386&30\_300\_6\_50 \textcolor{red}{\textcjheb{nw+sl}} LSWN $|$(eine) Zunge\\
4.&119.&1117.&444.&4428.&14.&3&600&300\_100\_200 \textcolor{red}{\textcjheb{rq+s}} SQR $|$(der) L"uge(n)\\
5.&120.&1118.&447.&4431.&17.&5&70&6\_10\_4\_10\_40 \textcolor{red}{\textcjheb{mydyw}} WJDJM $|$und H"ande\\
6.&121.&1119.&452.&4436.&22.&5&806&300\_80\_20\_6\_400 \textcolor{red}{\textcjheb{twkp+s}} SPKWT $|$(die) vergie"sen(d(e))\\
7.&122.&1120.&457.&4441.&27.&2&44&4\_40 \textcolor{red}{\textcjheb{md}} DM $|$Blut\\
8.&123.&1121.&459.&4443.&29.&3&160&50\_100\_10 \textcolor{red}{\textcjheb{yqn}} NQJ $|$unschuldiges\\
\end{tabular}\medskip \\
Ende des Verses 6.17\\
Verse: 157, Buchstaben: 31, 461, 4445, Totalwerte: 2892, 36409, 321098\\
\\
Hohe Augen, eine L"ugenzunge, und H"ande, die unschuldiges Blut vergie"sen;\\
\newpage 
{\bf -- 6.18}\\
\medskip \\
\begin{tabular}{rrrrrrrrp{120mm}}
WV&WK&WB&ABK&ABB&ABV&AnzB&TW&Zahlencode \textcolor{red}{$\boldsymbol{Grundtext}$} Umschrift $|$"Ubersetzung(en)\\
1.&124.&1122.&462.&4446.&1.&2&32&30\_2 \textcolor{red}{\textcjheb{bl}} LB $|$(ein) Herz\\
2.&125.&1123.&464.&4448.&3.&3&508&8\_200\_300 \textcolor{red}{\textcjheb{+sr.h}} CRS $|$welches schmiedet/ersinnend\\
3.&126.&1124.&467.&4451.&6.&6&756&40\_8\_300\_2\_6\_400 \textcolor{red}{\textcjheb{twb+s.hm}} MCSBWT $|$Anschl"age/Gedanken\\
4.&127.&1125.&473.&4457.&12.&3&57&1\_6\_50 \textcolor{red}{\textcjheb{nw'}} AWN $|$heillose/des Frevels\\
5.&128.&1126.&476.&4460.&15.&5&283&200\_3\_30\_10\_40 \textcolor{red}{\textcjheb{mylgr}} RGLJM $|$F"u"se\\
6.&129.&1127.&481.&4465.&20.&6&691&40\_40\_5\_200\_6\_400 \textcolor{red}{\textcjheb{twrhmm}} MMHRWT $|$eilends/eilend(e)\\
7.&130.&1128.&487.&4471.&26.&4&326&30\_200\_6\_90 \textcolor{red}{\textcjheb{.swrl}} LRW"s $|$die hinlaufen/um zu laufen\\
8.&131.&1129.&491.&4475.&30.&4&305&30\_200\_70\_5 \textcolor{red}{\textcjheb{h`rl}} LRaH $|$zum B"osen/nach dem B"osen\\
\end{tabular}\medskip \\
Ende des Verses 6.18\\
Verse: 158, Buchstaben: 33, 494, 4478, Totalwerte: 2958, 39367, 324056\\
\\
ein Herz, welches heillose Anschl"age schmiedet, F"u"se, die eilends zum B"osen hinlaufen;\\
\newpage 
{\bf -- 6.19}\\
\medskip \\
\begin{tabular}{rrrrrrrrp{120mm}}
WV&WK&WB&ABK&ABB&ABV&AnzB&TW&Zahlencode \textcolor{red}{$\boldsymbol{Grundtext}$} Umschrift $|$"Ubersetzung(en)\\
1.&132.&1130.&495.&4479.&1.&4&108&10\_80\_10\_8 \textcolor{red}{\textcjheb{.hypy}} JPJC $|$wer ausspricht/er schnaubt\\
2.&133.&1131.&499.&4483.&5.&5&79&20\_7\_2\_10\_40 \textcolor{red}{\textcjheb{mybzk}} KZBJM $|$als falscher/L"ugen\\
3.&134.&1132.&504.&4488.&10.&2&74&70\_4 \textcolor{red}{\textcjheb{d`}} aD $|$(als) Zeuge\\
4.&135.&1133.&506.&4490.&12.&3&600&300\_100\_200 \textcolor{red}{\textcjheb{rq+s}} SQR $|$L"ugen/des Trugs\\
5.&136.&1134.&509.&4493.&15.&5&384&6\_40\_300\_30\_8 \textcolor{red}{\textcjheb{.hl+smw}} WMSLC $|$und wer ausstreut/und (er) entfesselt\\
6.&137.&1135.&514.&4498.&20.&5&144&40\_4\_50\_10\_40 \textcolor{red}{\textcjheb{myndm}} MDNJM $|$Zwietracht/Z"ankereien\\
7.&138.&1136.&519.&4503.&25.&3&62&2\_10\_50 \textcolor{red}{\textcjheb{nyb}} BJN $|$zwischen\\
8.&139.&1137.&522.&4506.&28.&4&59&1\_8\_10\_40 \textcolor{red}{\textcjheb{my.h'}} ACJM $|$Br"udern\\
\end{tabular}\medskip \\
Ende des Verses 6.19\\
Verse: 159, Buchstaben: 31, 525, 4509, Totalwerte: 1510, 40877, 325566\\
\\
wer L"ugen ausspricht als falscher Zeuge, und wer Zwietracht ausstreut zwischen Br"udern.\\
\newpage 
{\bf -- 6.20}\\
\medskip \\
\begin{tabular}{rrrrrrrrp{120mm}}
WV&WK&WB&ABK&ABB&ABV&AnzB&TW&Zahlencode \textcolor{red}{$\boldsymbol{Grundtext}$} Umschrift $|$"Ubersetzung(en)\\
1.&140.&1138.&526.&4510.&1.&3&340&50\_90\_200 \textcolor{red}{\textcjheb{r.sn}} N"sR $|$bewahre/beobachte\\
2.&141.&1139.&529.&4513.&4.&3&62&2\_50\_10 \textcolor{red}{\textcjheb{ynb}} BNJ $|$mein Sohn\\
3.&142.&1140.&532.&4516.&7.&4&536&40\_90\_6\_400 \textcolor{red}{\textcjheb{tw.sm}} M"sWT $|$das Gebot\\
4.&143.&1141.&536.&4520.&11.&4&33&1\_2\_10\_20 \textcolor{red}{\textcjheb{kyb'}} ABJK $|$deines Vaters\\
5.&144.&1142.&540.&4524.&15.&3&37&6\_1\_30 \textcolor{red}{\textcjheb{l'w}} WAL $|$und nicht\\
6.&145.&1143.&543.&4527.&18.&3&709&400\_9\_300 \textcolor{red}{\textcjheb{+s.tt}} TtS $|$verlass/du sollst verwerfen\\
7.&146.&1144.&546.&4530.&21.&4&1006&400\_6\_200\_400 \textcolor{red}{\textcjheb{trwt}} TWRT $|$die Belehrung/die Weisung\\
8.&147.&1145.&550.&4534.&25.&3&61&1\_40\_20 \textcolor{red}{\textcjheb{km'}} AMK $|$deiner Mutter\\
\end{tabular}\medskip \\
Ende des Verses 6.20\\
Verse: 160, Buchstaben: 27, 552, 4536, Totalwerte: 2784, 43661, 328350\\
\\
Mein Sohn, bewahre das Gebot deines Vaters, und verla"s nicht die Belehrung deiner Mutter;\\
\newpage 
{\bf -- 6.21}\\
\medskip \\
\begin{tabular}{rrrrrrrrp{120mm}}
WV&WK&WB&ABK&ABB&ABV&AnzB&TW&Zahlencode \textcolor{red}{$\boldsymbol{Grundtext}$} Umschrift $|$"Ubersetzung(en)\\
1.&148.&1146.&553.&4537.&1.&4&640&100\_300\_200\_40 \textcolor{red}{\textcjheb{mr+sq}} QSRM $|$binde sie\\
2.&149.&1147.&557.&4541.&5.&2&100&70\_30 \textcolor{red}{\textcjheb{l`}} aL $|$auf/an\\
3.&150.&1148.&559.&4543.&7.&3&52&30\_2\_20 \textcolor{red}{\textcjheb{kbl}} LBK $|$dein Herz\\
4.&151.&1149.&562.&4546.&10.&4&454&400\_40\_10\_4 \textcolor{red}{\textcjheb{dymt}} TMJD $|$stets/st"andig\\
5.&152.&1150.&566.&4550.&14.&4&164&70\_50\_4\_40 \textcolor{red}{\textcjheb{mdn`}} aNDM $|$kn"upfe sie/lege sie\\
6.&153.&1151.&570.&4554.&18.&2&100&70\_30 \textcolor{red}{\textcjheb{l`}} aL $|$um\\
7.&154.&1152.&572.&4556.&20.&6&826&3\_200\_3\_200\_400\_20 \textcolor{red}{\textcjheb{ktrgrg}} GRGRTK $|$deinen Hals\\
\end{tabular}\medskip \\
Ende des Verses 6.21\\
Verse: 161, Buchstaben: 25, 577, 4561, Totalwerte: 2336, 45997, 330686\\
\\
binde sie stets auf dein Herz, kn"upfe sie um deinen Hals.\\
\newpage 
{\bf -- 6.22}\\
\medskip \\
\begin{tabular}{rrrrrrrrp{120mm}}
WV&WK&WB&ABK&ABB&ABV&AnzB&TW&Zahlencode \textcolor{red}{$\boldsymbol{Grundtext}$} Umschrift $|$"Ubersetzung(en)\\
1.&155.&1153.&578.&4562.&1.&7&482&2\_5\_400\_5\_30\_20\_20 \textcolor{red}{\textcjheb{kklhthb}} BHTHLKK $|$wenn du einhergehst/w"ahrend deines Gehens\\
2.&156.&1154.&585.&4569.&8.&4&463&400\_50\_8\_5 \textcolor{red}{\textcjheb{h.hnt}} TNCH $|$sie wird leiten/sie m"oge leiten\\
3.&157.&1155.&589.&4573.&12.&3&421&1\_400\_20 \textcolor{red}{\textcjheb{kt'}} ATK $|$dich\\
4.&158.&1156.&592.&4576.&15.&5&344&2\_300\_20\_2\_20 \textcolor{red}{\textcjheb{kbk+sb}} BSKBK $|$wenn du dich niederlegst/w"ahrend deines Liegens\\
5.&159.&1157.&597.&4581.&20.&4&940&400\_300\_40\_200 \textcolor{red}{\textcjheb{rm+st}} TSMR $|$wird sie wachen/sie m"oge wachen\\
6.&160.&1158.&601.&4585.&24.&4&130&70\_30\_10\_20 \textcolor{red}{\textcjheb{kyl`}} aLJK $|$"uber dich\\
7.&161.&1159.&605.&4589.&28.&7&617&6\_5\_100\_10\_90\_6\_400 \textcolor{red}{\textcjheb{tw.syqhw}} WHQJ"sWT $|$und (wenn) du erwachst\\
8.&162.&1160.&612.&4596.&35.&3&16&5\_10\_1 \textcolor{red}{\textcjheb{'yh}} HJA $|$(so) sie\\
9.&163.&1161.&615.&4599.&38.&5&738&400\_300\_10\_8\_20 \textcolor{red}{\textcjheb{k.hy+st}} TSJCK $|$wird mit dir reden/sie m"oge anreden dich\\
\end{tabular}\medskip \\
Ende des Verses 6.22\\
Verse: 162, Buchstaben: 42, 619, 4603, Totalwerte: 4151, 50148, 334837\\
\\
Wenn du einhergehst, wird sie dich leiten; wenn du dich niederlegst, wird sie "uber dich wachen; und erwachst du, so wird sie mit dir reden.\\
\newpage 
{\bf -- 6.23}\\
\medskip \\
\begin{tabular}{rrrrrrrrp{120mm}}
WV&WK&WB&ABK&ABB&ABV&AnzB&TW&Zahlencode \textcolor{red}{$\boldsymbol{Grundtext}$} Umschrift $|$"Ubersetzung(en)\\
1.&164.&1162.&620.&4604.&1.&2&30&20\_10 \textcolor{red}{\textcjheb{yk}} KJ $|$denn\\
2.&165.&1163.&622.&4606.&3.&2&250&50\_200 \textcolor{red}{\textcjheb{rn}} NR $|$(eine) Leuchte\\
3.&166.&1164.&624.&4608.&5.&4&141&40\_90\_6\_5 \textcolor{red}{\textcjheb{hw.sm}} M"sWH $|$(ist das) Gebot\\
4.&167.&1165.&628.&4612.&9.&5&617&6\_400\_6\_200\_5 \textcolor{red}{\textcjheb{hrwtw}} WTWRH $|$und die Belehrung/und (die) Weisung\\
5.&168.&1166.&633.&4617.&14.&3&207&1\_6\_200 \textcolor{red}{\textcjheb{rw'}} AWR $|$(ein) Licht\\
6.&169.&1167.&636.&4620.&17.&4&230&6\_4\_200\_20 \textcolor{red}{\textcjheb{krdw}} WDRK $|$und der Weg/und ein Weg\\
7.&170.&1168.&640.&4624.&21.&4&68&8\_10\_10\_40 \textcolor{red}{\textcjheb{myy.h}} CJJM $|$des Lebens/zum Leben\\
8.&171.&1169.&644.&4628.&25.&6&840&400\_6\_20\_8\_6\_400 \textcolor{red}{\textcjheb{tw.hkwt}} TWKCWT $|$(sind) die Zurechtweisung(en)\\
9.&172.&1170.&650.&4634.&31.&4&306&40\_6\_60\_200 \textcolor{red}{\textcjheb{rswm}} MWsR $|$der Zucht/(zur) Zucht\\
\end{tabular}\medskip \\
Ende des Verses 6.23\\
Verse: 163, Buchstaben: 34, 653, 4637, Totalwerte: 2689, 52837, 337526\\
\\
Denn das Gebot ist eine Leuchte, und die Belehrung ein Licht; und die Zurechtweisung der Zucht sind der Weg des Lebens:\\
\newpage 
{\bf -- 6.24}\\
\medskip \\
\begin{tabular}{rrrrrrrrp{120mm}}
WV&WK&WB&ABK&ABB&ABV&AnzB&TW&Zahlencode \textcolor{red}{$\boldsymbol{Grundtext}$} Umschrift $|$"Ubersetzung(en)\\
1.&173.&1171.&654.&4638.&1.&5&590&30\_300\_40\_200\_20 \textcolor{red}{\textcjheb{krm+sl}} LSMRK $|$um dich zu bewahren/zu beh"uten dich\\
2.&174.&1172.&659.&4643.&6.&4&741&40\_1\_300\_400 \textcolor{red}{\textcjheb{t+s'm}} MAST $|$vor der Frau/vor einer Frau\\
3.&175.&1173.&663.&4647.&10.&2&270&200\_70 \textcolor{red}{\textcjheb{`r}} Ra $|$b"osen/(des) B"osen\\
4.&176.&1174.&665.&4649.&12.&5&578&40\_8\_30\_100\_400 \textcolor{red}{\textcjheb{tql.hm}} MCLQT $|$vor der Gl"atte\\
5.&177.&1175.&670.&4654.&17.&4&386&30\_300\_6\_50 \textcolor{red}{\textcjheb{nw+sl}} LSWN $|$der Zunge\\
6.&178.&1176.&674.&4658.&21.&5&285&50\_20\_200\_10\_5 \textcolor{red}{\textcjheb{hyrkn}} NKRJH $|$(einer) Fremden\\
\end{tabular}\medskip \\
Ende des Verses 6.24\\
Verse: 164, Buchstaben: 25, 678, 4662, Totalwerte: 2850, 55687, 340376\\
\\
um dich zu bewahren vor dem b"osen Weibe, vor der Gl"atte der Zunge einer Fremden.\\
\newpage 
{\bf -- 6.25}\\
\medskip \\
\begin{tabular}{rrrrrrrrp{120mm}}
WV&WK&WB&ABK&ABB&ABV&AnzB&TW&Zahlencode \textcolor{red}{$\boldsymbol{Grundtext}$} Umschrift $|$"Ubersetzung(en)\\
1.&179.&1177.&679.&4663.&1.&2&31&1\_30 \textcolor{red}{\textcjheb{l'}} AL $|$nicht\\
2.&180.&1178.&681.&4665.&3.&4&452&400\_8\_40\_4 \textcolor{red}{\textcjheb{dm.ht}} TCMD $|$begehre/sollst du begehren\\
3.&181.&1179.&685.&4669.&7.&4&105&10\_80\_10\_5 \textcolor{red}{\textcjheb{hypy}} JPJH $|$(nach) ihre(r) Sch"onheit\\
4.&182.&1180.&689.&4673.&11.&5&56&2\_30\_2\_2\_20 \textcolor{red}{\textcjheb{kbblb}} BLBBK $|$in deinem Herzen\\
5.&183.&1181.&694.&4678.&16.&3&37&6\_1\_30 \textcolor{red}{\textcjheb{l'w}} WAL $|$und nicht\\
6.&184.&1182.&697.&4681.&19.&4&528&400\_100\_8\_20 \textcolor{red}{\textcjheb{k.hqt}} TQCK $|$sie fange dich/sie nehme (ein) dich\\
7.&185.&1183.&701.&4685.&23.&7&317&2\_70\_80\_70\_80\_10\_5 \textcolor{red}{\textcjheb{hyp`p`b}} BaPaPJH $|$mit ihren Wimpern\\
\end{tabular}\medskip \\
Ende des Verses 6.25\\
Verse: 165, Buchstaben: 29, 707, 4691, Totalwerte: 1526, 57213, 341902\\
\\
Begehre nicht in deinem Herzen nach ihrer Sch"onheit, und sie fange dich nicht mit ihren Wimpern!\\
\newpage 
{\bf -- 6.26}\\
\medskip \\
\begin{tabular}{rrrrrrrrp{120mm}}
WV&WK&WB&ABK&ABB&ABV&AnzB&TW&Zahlencode \textcolor{red}{$\boldsymbol{Grundtext}$} Umschrift $|$"Ubersetzung(en)\\
1.&186.&1184.&708.&4692.&1.&2&30&20\_10 \textcolor{red}{\textcjheb{yk}} KJ $|$denn\\
2.&187.&1185.&710.&4694.&3.&3&76&2\_70\_4 \textcolor{red}{\textcjheb{d`b}} BaD $|$um/f"ur\\
3.&188.&1186.&713.&4697.&6.&3&306&1\_300\_5 \textcolor{red}{\textcjheb{h+s'}} ASH $|$(eine) Frau\\
4.&189.&1187.&716.&4700.&9.&4&68&7\_6\_50\_5 \textcolor{red}{\textcjheb{hnwz}} ZWNH $|$hurerische\\
5.&190.&1188.&720.&4704.&13.&2&74&70\_4 \textcolor{red}{\textcjheb{d`}} aD $|$kommt man bis auf/(es dreht sich) bis zu\\
6.&191.&1189.&722.&4706.&15.&3&240&20\_20\_200 \textcolor{red}{\textcjheb{rkk}} KKR $|$einen Laib/(einem) Laib\\
7.&192.&1190.&725.&4709.&18.&3&78&30\_8\_40 \textcolor{red}{\textcjheb{m.hl}} LCM $|$Brot\\
8.&193.&1191.&728.&4712.&21.&4&707&6\_1\_300\_400 \textcolor{red}{\textcjheb{t+s'w}} WAST $|$und (die) Frau\\
9.&194.&1192.&732.&4716.&25.&3&311&1\_10\_300 \textcolor{red}{\textcjheb{+sy'}} AJS $|$eines (Ehe)Mannes\\
10.&195.&1193.&735.&4719.&28.&3&430&50\_80\_300 \textcolor{red}{\textcjheb{+spn}} NPS $|$einer Seele/dem Leben\\
11.&196.&1194.&738.&4722.&31.&4&315&10\_100\_200\_5 \textcolor{red}{\textcjheb{hrqy}} JQRH $|$kostbaren\\
12.&197.&1195.&742.&4726.&35.&4&500&400\_90\_6\_4 \textcolor{red}{\textcjheb{dw.st}} T"sWD $|$stellt nach/sie jagt nach\\
\end{tabular}\medskip \\
Ende des Verses 6.26\\
Verse: 166, Buchstaben: 38, 745, 4729, Totalwerte: 3135, 60348, 345037\\
\\
Denn um eines hurerischen Weibes willen kommt man bis auf einen Laib Brot, und eines Mannes Weib stellt einer kostbaren Seele nach. -\\
\newpage 
{\bf -- 6.27}\\
\medskip \\
\begin{tabular}{rrrrrrrrp{120mm}}
WV&WK&WB&ABK&ABB&ABV&AnzB&TW&Zahlencode \textcolor{red}{$\boldsymbol{Grundtext}$} Umschrift $|$"Ubersetzung(en)\\
1.&198.&1196.&746.&4730.&1.&5&428&5\_10\_8\_400\_5 \textcolor{red}{\textcjheb{ht.hyh}} HJCTH $|$sollte nehmen/etwa er (=es) tr"agt\\
2.&199.&1197.&751.&4735.&6.&3&311&1\_10\_300 \textcolor{red}{\textcjheb{+sy'}} AJS $|$jemand\\
3.&200.&1198.&754.&4738.&9.&2&301&1\_300 \textcolor{red}{\textcjheb{+s'}} AS $|$Feuer\\
4.&201.&1199.&756.&4740.&11.&5&126&2\_8\_10\_100\_6 \textcolor{red}{\textcjheb{wqy.hb}} BCJQW $|$in seinen Busen\\
5.&202.&1200.&761.&4745.&16.&6&31&6\_2\_3\_4\_10\_6 \textcolor{red}{\textcjheb{wydgbw}} WBGDJW $|$ohne dass seine Kleider/und seine Kleider\\
6.&203.&1201.&767.&4751.&22.&2&31&30\_1 \textcolor{red}{\textcjheb{'l}} LA $|$/nicht\\
7.&204.&1202.&769.&4753.&24.&6&1035&400\_300\_200\_80\_50\_5 \textcolor{red}{\textcjheb{hnpr+st}} TSRPNH $|$verbrennten/sie verbrennen\\
\end{tabular}\medskip \\
Ende des Verses 6.27\\
Verse: 167, Buchstaben: 29, 774, 4758, Totalwerte: 2263, 62611, 347300\\
\\
Sollte jemand Feuer in seinen Busen nehmen, ohne da"s seine Kleider verbrennten?\\
\newpage 
{\bf -- 6.28}\\
\medskip \\
\begin{tabular}{rrrrrrrrp{120mm}}
WV&WK&WB&ABK&ABB&ABV&AnzB&TW&Zahlencode \textcolor{red}{$\boldsymbol{Grundtext}$} Umschrift $|$"Ubersetzung(en)\\
1.&205.&1203.&775.&4759.&1.&2&41&1\_40 \textcolor{red}{\textcjheb{m'}} AM $|$oder\\
2.&206.&1204.&777.&4761.&3.&4&65&10\_5\_30\_20 \textcolor{red}{\textcjheb{klhy}} JHLK $|$sollte gehen/er (=es) geht\\
3.&207.&1205.&781.&4765.&7.&3&311&1\_10\_300 \textcolor{red}{\textcjheb{+sy'}} AJS $|$jemand\\
4.&208.&1206.&784.&4768.&10.&2&100&70\_30 \textcolor{red}{\textcjheb{l`}} aL $|$"uber\\
5.&209.&1207.&786.&4770.&12.&6&96&5\_3\_8\_30\_10\_40 \textcolor{red}{\textcjheb{myl.hgh}} HGCLJM $|$gl"uhende Kohlen/die Kohlenglut\\
6.&210.&1208.&792.&4776.&18.&6&255&6\_200\_3\_30\_10\_6 \textcolor{red}{\textcjheb{wylgrw}} WRGLJW $|$ohne dass seine F"u"se/und seine F"u"se\\
7.&211.&1209.&798.&4782.&24.&2&31&30\_1 \textcolor{red}{\textcjheb{'l}} LA $|$/nicht\\
8.&212.&1210.&800.&4784.&26.&6&491&400\_20\_6\_10\_50\_5 \textcolor{red}{\textcjheb{hnywkt}} TKWJNH $|$versengt w"urden/(sie) werden versengt\\
\end{tabular}\medskip \\
Ende des Verses 6.28\\
Verse: 168, Buchstaben: 31, 805, 4789, Totalwerte: 1390, 64001, 348690\\
\\
Oder sollte jemand "uber gl"uhende Kohlen gehen, ohne da"s seine F"u"se versengt w"urden?\\
\newpage 
{\bf -- 6.29}\\
\medskip \\
\begin{tabular}{rrrrrrrrp{120mm}}
WV&WK&WB&ABK&ABB&ABV&AnzB&TW&Zahlencode \textcolor{red}{$\boldsymbol{Grundtext}$} Umschrift $|$"Ubersetzung(en)\\
1.&213.&1211.&806.&4790.&1.&2&70&20\_50 \textcolor{red}{\textcjheb{nk}} KN $|$so\\
2.&214.&1212.&808.&4792.&3.&3&8&5\_2\_1 \textcolor{red}{\textcjheb{'bh}} HBA $|$der welcher eingeht/der Kommende\\
3.&215.&1213.&811.&4795.&6.&2&31&1\_30 \textcolor{red}{\textcjheb{l'}} AL $|$zur\\
4.&216.&1214.&813.&4797.&8.&3&701&1\_300\_400 \textcolor{red}{\textcjheb{t+s'}} AST $|$Frau\\
5.&217.&1215.&816.&4800.&11.&4&281&200\_70\_5\_6 \textcolor{red}{\textcjheb{wh`r}} RaHW $|$seines N"achsten/seines Gef"ahrten\\
6.&218.&1216.&820.&4804.&15.&2&31&30\_1 \textcolor{red}{\textcjheb{'l}} LA $|$nicht\\
7.&219.&1217.&822.&4806.&17.&4&165&10\_50\_100\_5 \textcolor{red}{\textcjheb{hqny}} JNQH $|$wird f"ur schuldlos gehalten werden/er (=es) ist unschuldig\\
8.&220.&1218.&826.&4810.&21.&2&50&20\_30 \textcolor{red}{\textcjheb{lk}} KL $|$einer/jeder\\
9.&221.&1219.&828.&4812.&23.&4&128&5\_50\_3\_70 \textcolor{red}{\textcjheb{`gnh}} HNGa $|$der ber"uhrt/der r"uhrend(e) (ist)\\
10.&222.&1220.&832.&4816.&27.&2&7&2\_5 \textcolor{red}{\textcjheb{hb}} BH $|$(an) sie\\
\end{tabular}\medskip \\
Ende des Verses 6.29\\
Verse: 169, Buchstaben: 28, 833, 4817, Totalwerte: 1472, 65473, 350162\\
\\
So der, welcher zu dem Weibe seines N"achsten eingeht: keiner, der sie ber"uhrt, wird f"ur schuldlos gehalten werden. -\\
\newpage 
{\bf -- 6.30}\\
\medskip \\
\begin{tabular}{rrrrrrrrp{120mm}}
WV&WK&WB&ABK&ABB&ABV&AnzB&TW&Zahlencode \textcolor{red}{$\boldsymbol{Grundtext}$} Umschrift $|$"Ubersetzung(en)\\
1.&223.&1221.&834.&4818.&1.&2&31&30\_1 \textcolor{red}{\textcjheb{'l}} LA $|$nicht\\
2.&224.&1222.&836.&4820.&3.&5&31&10\_2\_6\_7\_6 \textcolor{red}{\textcjheb{wzwby}} JBWZW $|$man verachtet/sie verachten\\
3.&225.&1223.&841.&4825.&8.&4&85&30\_3\_50\_2 \textcolor{red}{\textcjheb{bngl}} LGNB $|$den Dieb\\
4.&226.&1224.&845.&4829.&12.&2&30&20\_10 \textcolor{red}{\textcjheb{yk}} KJ $|$wenn\\
5.&227.&1225.&847.&4831.&14.&5&71&10\_3\_50\_6\_2 \textcolor{red}{\textcjheb{bwngy}} JGNWB $|$er stiehlt\\
6.&228.&1226.&852.&4836.&19.&4&101&30\_40\_30\_1 \textcolor{red}{\textcjheb{'lml}} LMLA $|$um zu stillen/(um) zu f"ullen\\
7.&229.&1227.&856.&4840.&23.&4&436&50\_80\_300\_6 \textcolor{red}{\textcjheb{w+spn}} NPSW $|$seine Gier/seine Seele\\
8.&230.&1228.&860.&4844.&27.&2&30&20\_10 \textcolor{red}{\textcjheb{yk}} KJ $|$weil/wenn\\
9.&231.&1229.&862.&4846.&29.&4&282&10\_200\_70\_2 \textcolor{red}{\textcjheb{b`ry}} JRaB $|$ihn hungert/er hungert\\
\end{tabular}\medskip \\
Ende des Verses 6.30\\
Verse: 170, Buchstaben: 32, 865, 4849, Totalwerte: 1097, 66570, 351259\\
\\
Man verachtet den Dieb nicht, wenn er stiehlt, um seine Gier zu stillen, weil ihn hungert;\\
\newpage 
{\bf -- 6.31}\\
\medskip \\
\begin{tabular}{rrrrrrrrp{120mm}}
WV&WK&WB&ABK&ABB&ABV&AnzB&TW&Zahlencode \textcolor{red}{$\boldsymbol{Grundtext}$} Umschrift $|$"Ubersetzung(en)\\
1.&232.&1230.&866.&4850.&1.&5&187&6\_50\_40\_90\_1 \textcolor{red}{\textcjheb{'.smnw}} WNM"sA $|$und (wenn) er wird gefunden\\
2.&233.&1231.&871.&4855.&6.&4&380&10\_300\_30\_40 \textcolor{red}{\textcjheb{ml+sy}} JSLM $|$kann er erstatten/muss er ersetzen (es)\\
3.&234.&1232.&875.&4859.&10.&6&822&300\_2\_70\_400\_10\_40 \textcolor{red}{\textcjheb{myt`b+s}} SBaTJM $|$siebenfach\\
4.&235.&1233.&881.&4865.&16.&2&401&1\_400 \textcolor{red}{\textcjheb{t'}} AT $|$**\\
5.&236.&1234.&883.&4867.&18.&2&50&20\_30 \textcolor{red}{\textcjheb{lk}} KL $|$all(es)\\
6.&237.&1235.&885.&4869.&20.&3&61&5\_6\_50 \textcolor{red}{\textcjheb{nwh}} HWN $|$Gut/die Habe\\
7.&238.&1236.&888.&4872.&23.&4&418&2\_10\_400\_6 \textcolor{red}{\textcjheb{wtyb}} BJTW $|$seines Hauses\\
8.&239.&1237.&892.&4876.&27.&3&460&10\_400\_50 \textcolor{red}{\textcjheb{nty}} JTN $|$kann (er) hingeben/er muss hergeben\\
\end{tabular}\medskip \\
Ende des Verses 6.31\\
Verse: 171, Buchstaben: 29, 894, 4878, Totalwerte: 2779, 69349, 354038\\
\\
und wenn er gefunden wird, kann er siebenfach erstatten, kann alles Gut seines Hauses hingeben.\\
\newpage 
{\bf -- 6.32}\\
\medskip \\
\begin{tabular}{rrrrrrrrp{120mm}}
WV&WK&WB&ABK&ABB&ABV&AnzB&TW&Zahlencode \textcolor{red}{$\boldsymbol{Grundtext}$} Umschrift $|$"Ubersetzung(en)\\
1.&240.&1238.&895.&4879.&1.&3&131&50\_1\_80 \textcolor{red}{\textcjheb{p'n}} NAP $|$wer begeht Ehebruch/(der) Ehebrechende\\
2.&241.&1239.&898.&4882.&4.&3&306&1\_300\_5 \textcolor{red}{\textcjheb{h+s'}} ASH $|$(mit einer) Frau\\
3.&242.&1240.&901.&4885.&7.&3&268&8\_60\_200 \textcolor{red}{\textcjheb{rs.h}} CsR $|$ist un-/(ist) ermangelnd\\
4.&243.&1241.&904.&4888.&10.&2&32&30\_2 \textcolor{red}{\textcjheb{bl}} LB $|$sinnig/Verstand\\
5.&244.&1242.&906.&4890.&12.&5&758&40\_300\_8\_10\_400 \textcolor{red}{\textcjheb{ty.h+sm}} MSCJT $|$wer verderben will/(ein) Verderbender\\
6.&245.&1243.&911.&4895.&17.&4&436&50\_80\_300\_6 \textcolor{red}{\textcjheb{w+spn}} NPSW $|$seine Seele/sich selbst\\
7.&246.&1244.&915.&4899.&21.&3&12&5\_6\_1 \textcolor{red}{\textcjheb{'wh}} HWA $|$(d)er\\
8.&247.&1245.&918.&4902.&24.&5&435&10\_70\_300\_50\_5 \textcolor{red}{\textcjheb{hn+s`y}} JaSNH $|$tut solches/(er) tut es\\
\end{tabular}\medskip \\
Ende des Verses 6.32\\
Verse: 172, Buchstaben: 28, 922, 4906, Totalwerte: 2378, 71727, 356416\\
\\
Wer mit einem Weibe Ehebruch begeht, ist unsinnig; wer seine Seele verderben will, der tut solches.\\
\newpage 
{\bf -- 6.33}\\
\medskip \\
\begin{tabular}{rrrrrrrrp{120mm}}
WV&WK&WB&ABK&ABB&ABV&AnzB&TW&Zahlencode \textcolor{red}{$\boldsymbol{Grundtext}$} Umschrift $|$"Ubersetzung(en)\\
1.&248.&1246.&923.&4907.&1.&3&123&50\_3\_70 \textcolor{red}{\textcjheb{`gn}} NGa $|$Plage\\
2.&249.&1247.&926.&4910.&4.&5&192&6\_100\_30\_6\_50 \textcolor{red}{\textcjheb{nwlqw}} WQLWN $|$und Schande\\
3.&250.&1248.&931.&4915.&9.&4&141&10\_40\_90\_1 \textcolor{red}{\textcjheb{'.smy}} JM"sA $|$wird er finden/er findet\\
4.&251.&1249.&935.&4919.&13.&6&700&6\_8\_200\_80\_400\_6 \textcolor{red}{\textcjheb{wtpr.hw}} WCRPTW $|$und seine Schmach\\
5.&252.&1250.&941.&4925.&19.&2&31&30\_1 \textcolor{red}{\textcjheb{'l}} LA $|$nicht\\
6.&253.&1251.&943.&4927.&21.&4&453&400\_40\_8\_5 \textcolor{red}{\textcjheb{h.hmt}} TMCH $|$(sie) wird ausgel"oscht (werden)\\
\end{tabular}\medskip \\
Ende des Verses 6.33\\
Verse: 173, Buchstaben: 24, 946, 4930, Totalwerte: 1640, 73367, 358056\\
\\
Plage und Schande wird er finden, und seine Schmach wird nicht ausgel"oscht werden.\\
\newpage 
{\bf -- 6.34}\\
\medskip \\
\begin{tabular}{rrrrrrrrp{120mm}}
WV&WK&WB&ABK&ABB&ABV&AnzB&TW&Zahlencode \textcolor{red}{$\boldsymbol{Grundtext}$} Umschrift $|$"Ubersetzung(en)\\
1.&254.&1252.&947.&4931.&1.&2&30&20\_10 \textcolor{red}{\textcjheb{yk}} KJ $|$denn\\
2.&255.&1253.&949.&4933.&3.&4&156&100\_50\_1\_5 \textcolor{red}{\textcjheb{h'nq}} QNAH $|$Eifersucht\\
3.&256.&1254.&953.&4937.&7.&3&448&8\_40\_400 \textcolor{red}{\textcjheb{tm.h}} CMT $|$ist Grimm/(ist) die Zornesglut\\
4.&257.&1255.&956.&4940.&10.&3&205&3\_2\_200 \textcolor{red}{\textcjheb{rbg}} GBR $|$eines Mannes/des Mannes\\
5.&258.&1256.&959.&4943.&13.&3&37&6\_30\_1 \textcolor{red}{\textcjheb{'lw}} WLA $|$und nicht\\
6.&259.&1257.&962.&4946.&16.&5&94&10\_8\_40\_6\_30 \textcolor{red}{\textcjheb{lwm.hy}} JCMWL $|$er schont/er "ubt Schonung\\
7.&260.&1258.&967.&4951.&21.&4&58&2\_10\_6\_40 \textcolor{red}{\textcjheb{mwyb}} BJWM $|$am Tag\\
8.&261.&1259.&971.&4955.&25.&3&190&50\_100\_40 \textcolor{red}{\textcjheb{mqn}} NQM $|$(der) Rache\\
\end{tabular}\medskip \\
Ende des Verses 6.34\\
Verse: 174, Buchstaben: 27, 973, 4957, Totalwerte: 1218, 74585, 359274\\
\\
Denn Eifersucht ist eines Mannes Grimm, und am Tage der Rache schont er nicht.\\
\newpage 
{\bf -- 6.35}\\
\medskip \\
\begin{tabular}{rrrrrrrrp{120mm}}
WV&WK&WB&ABK&ABB&ABV&AnzB&TW&Zahlencode \textcolor{red}{$\boldsymbol{Grundtext}$} Umschrift $|$"Ubersetzung(en)\\
1.&262.&1260.&974.&4958.&1.&2&31&30\_1 \textcolor{red}{\textcjheb{'l}} LA $|$nicht\\
2.&263.&1261.&976.&4960.&3.&3&311&10\_300\_1 \textcolor{red}{\textcjheb{'+sy}} JSA $|$er nimmt/er wird erheben\\
3.&264.&1262.&979.&4963.&6.&3&140&80\_50\_10 \textcolor{red}{\textcjheb{ynp}} PNJ $|$R"ucksicht/Gesichter\\
4.&265.&1263.&982.&4966.&9.&2&50&20\_30 \textcolor{red}{\textcjheb{lk}} KL $|$auf irgendeine/jegliches\\
5.&266.&1264.&984.&4968.&11.&3&300&20\_80\_200 \textcolor{red}{\textcjheb{rpk}} KPR $|$S"uhne/L"osegeld\\
6.&267.&1265.&987.&4971.&14.&3&37&6\_30\_1 \textcolor{red}{\textcjheb{'lw}} WLA $|$und nicht\\
7.&268.&1266.&990.&4974.&17.&4&18&10\_1\_2\_5 \textcolor{red}{\textcjheb{hb'y}} JABH $|$willigt ein/er wird willens sein\\
8.&269.&1267.&994.&4978.&21.&2&30&20\_10 \textcolor{red}{\textcjheb{yk}} KJ $|$auch/wenn\\
9.&270.&1268.&996.&4980.&23.&4&607&400\_200\_2\_5 \textcolor{red}{\textcjheb{hbrt}} TRBH $|$magst du vergr"o"sern/du bietest viele\\
10.&271.&1269.&1000.&4984.&27.&3&312&300\_8\_4 \textcolor{red}{\textcjheb{d.h+s}} SCD $|$(das) Geschenk(e)\\
\end{tabular}\medskip \\
Ende des Verses 6.35\\
Verse: 175, Buchstaben: 29, 1002, 4986, Totalwerte: 1836, 76421, 361110\\
\\
Er nimmt keine R"ucksicht auf irgendwelche S"uhne und willigt nicht ein, magst du auch das Geschenk vergr"o"sern.\\
\\
{\bf Ende des Kapitels 6}\\
\newpage 
{\bf -- 7.1}\\
\medskip \\
\begin{tabular}{rrrrrrrrp{120mm}}
WV&WK&WB&ABK&ABB&ABV&AnzB&TW&Zahlencode \textcolor{red}{$\boldsymbol{Grundtext}$} Umschrift $|$"Ubersetzung(en)\\
1.&1.&1270.&1.&4987.&1.&3&62&2\_50\_10 \textcolor{red}{\textcjheb{ynb}} BNJ $|$mein Sohn\\
2.&2.&1271.&4.&4990.&4.&3&540&300\_40\_200 \textcolor{red}{\textcjheb{rm+s}} SMR $|$bewahre\\
3.&3.&1272.&7.&4993.&7.&4&251&1\_40\_200\_10 \textcolor{red}{\textcjheb{yrm'}} AMRJ $|$(meine) Worte\\
4.&4.&1273.&11.&4997.&11.&6&552&6\_40\_90\_6\_400\_10 \textcolor{red}{\textcjheb{ytw.smw}} WM"sWTJ $|$und meine Gebote\\
5.&5.&1274.&17.&5003.&17.&4&620&400\_90\_80\_50 \textcolor{red}{\textcjheb{np.st}} T"sPN $|$birg/du sollst bergen\\
6.&6.&1275.&21.&5007.&21.&3&421&1\_400\_20 \textcolor{red}{\textcjheb{kt'}} ATK $|$bei dir\\
\end{tabular}\medskip \\
Ende des Verses 7.1\\
Verse: 176, Buchstaben: 23, 23, 5009, Totalwerte: 2446, 2446, 363556\\
\\
Mein Sohn, bewahre meine Worte, und birg bei dir meine Gebote;\\
\newpage 
{\bf -- 7.2}\\
\medskip \\
\begin{tabular}{rrrrrrrrp{120mm}}
WV&WK&WB&ABK&ABB&ABV&AnzB&TW&Zahlencode \textcolor{red}{$\boldsymbol{Grundtext}$} Umschrift $|$"Ubersetzung(en)\\
1.&7.&1276.&24.&5010.&1.&3&540&300\_40\_200 \textcolor{red}{\textcjheb{rm+s}} SMR $|$bewahre/beachte\\
2.&8.&1277.&27.&5013.&4.&5&546&40\_90\_6\_400\_10 \textcolor{red}{\textcjheb{ytw.sm}} M"sWTJ $|$meine Gebote\\
3.&9.&1278.&32.&5018.&9.&4&29&6\_8\_10\_5 \textcolor{red}{\textcjheb{hy.hw}} WCJH $|$und lebe\\
4.&10.&1279.&36.&5022.&13.&6&1022&6\_400\_6\_200\_400\_10 \textcolor{red}{\textcjheb{ytrwtw}} WTWRTJ $|$und meine Belehrung/meine Weisungen\\
5.&11.&1280.&42.&5028.&19.&6&387&20\_1\_10\_300\_6\_50 \textcolor{red}{\textcjheb{nw+sy'k}} KAJSWN $|$wie (den) Apfel/wie die Pupille\\
6.&12.&1281.&48.&5034.&25.&5&160&70\_10\_50\_10\_20 \textcolor{red}{\textcjheb{kyny`}} aJNJK $|$(deiner) Aug(en)\\
\end{tabular}\medskip \\
Ende des Verses 7.2\\
Verse: 177, Buchstaben: 29, 52, 5038, Totalwerte: 2684, 5130, 366240\\
\\
bewahre meine Gebote und lebe, und meine Belehrung wie deinen Augapfel.\\
\newpage 
{\bf -- 7.3}\\
\medskip \\
\begin{tabular}{rrrrrrrrp{120mm}}
WV&WK&WB&ABK&ABB&ABV&AnzB&TW&Zahlencode \textcolor{red}{$\boldsymbol{Grundtext}$} Umschrift $|$"Ubersetzung(en)\\
1.&13.&1282.&53.&5039.&1.&4&640&100\_300\_200\_40 \textcolor{red}{\textcjheb{mr+sq}} QSRM $|$binde sie\\
2.&14.&1283.&57.&5043.&5.&2&100&70\_30 \textcolor{red}{\textcjheb{l`}} aL $|$um/an\\
3.&15.&1284.&59.&5045.&7.&7&593&1\_90\_2\_70\_400\_10\_20 \textcolor{red}{\textcjheb{kyt`b.s'}} A"sBaTJK $|$deine Finger\\
4.&16.&1285.&66.&5052.&14.&4&462&20\_400\_2\_40 \textcolor{red}{\textcjheb{mbtk}} KTBM $|$schreibe sie\\
5.&17.&1286.&70.&5056.&18.&2&100&70\_30 \textcolor{red}{\textcjheb{l`}} aL $|$auf\\
6.&18.&1287.&72.&5058.&20.&3&44&30\_6\_8 \textcolor{red}{\textcjheb{.hwl}} LWC $|$die Tafel\\
7.&19.&1288.&75.&5061.&23.&3&52&30\_2\_20 \textcolor{red}{\textcjheb{kbl}} LBK $|$deines Herzens\\
\end{tabular}\medskip \\
Ende des Verses 7.3\\
Verse: 178, Buchstaben: 25, 77, 5063, Totalwerte: 1991, 7121, 368231\\
\\
Binde sie um deine Finger, schreibe sie auf die Tafel deines Herzens.\\
\newpage 
{\bf -- 7.4}\\
\medskip \\
\begin{tabular}{rrrrrrrrp{120mm}}
WV&WK&WB&ABK&ABB&ABV&AnzB&TW&Zahlencode \textcolor{red}{$\boldsymbol{Grundtext}$} Umschrift $|$"Ubersetzung(en)\\
1.&20.&1289.&78.&5064.&1.&3&241&1\_40\_200 \textcolor{red}{\textcjheb{rm'}} AMR $|$sprich\\
2.&21.&1290.&81.&5067.&4.&5&103&30\_8\_20\_40\_5 \textcolor{red}{\textcjheb{hmk.hl}} LCKMH $|$zur Weisheit\\
3.&22.&1291.&86.&5072.&9.&4&419&1\_8\_400\_10 \textcolor{red}{\textcjheb{yt.h'}} ACTJ $|$meine Schwester\\
4.&23.&1292.&90.&5076.&13.&2&401&1\_400 \textcolor{red}{\textcjheb{t'}} AT $|$(bist) du\\
5.&24.&1293.&92.&5078.&15.&4&120&6\_40\_4\_70 \textcolor{red}{\textcjheb{`dmw}} WMDa $|$und deinen Verwandten/und Bekannte\\
6.&25.&1294.&96.&5082.&19.&5&97&30\_2\_10\_50\_5 \textcolor{red}{\textcjheb{hnybl}} LBJNH $|$den Verstand/zu der Einsicht\\
7.&26.&1295.&101.&5087.&24.&4&701&400\_100\_200\_1 \textcolor{red}{\textcjheb{'rqt}} TQRA $|$(du sollst sie) nenne(n)\\
\end{tabular}\medskip \\
Ende des Verses 7.4\\
Verse: 179, Buchstaben: 27, 104, 5090, Totalwerte: 2082, 9203, 370313\\
\\
Sprich zur Weisheit: Du bist meine Schwester! Und nenne den Verstand deinen Verwandten;\\
\newpage 
{\bf -- 7.5}\\
\medskip \\
\begin{tabular}{rrrrrrrrp{120mm}}
WV&WK&WB&ABK&ABB&ABV&AnzB&TW&Zahlencode \textcolor{red}{$\boldsymbol{Grundtext}$} Umschrift $|$"Ubersetzung(en)\\
1.&27.&1296.&105.&5091.&1.&5&590&30\_300\_40\_200\_20 \textcolor{red}{\textcjheb{krm+sl}} LSMRK $|$damit sie dich bewahre/zu bewahren dich\\
2.&28.&1297.&110.&5096.&6.&4&346&40\_1\_300\_5 \textcolor{red}{\textcjheb{h+s'm}} MASH $|$vor der Frau/vor (einer) Frau\\
3.&29.&1298.&114.&5100.&10.&3&212&7\_200\_5 \textcolor{red}{\textcjheb{hrz}} ZRH $|$fremden\\
4.&30.&1299.&117.&5103.&13.&6&325&40\_50\_20\_200\_10\_5 \textcolor{red}{\textcjheb{hyrknm}} MNKRJH $|$vor der Fremden/vor einer Ausl"anderin\\
5.&31.&1300.&123.&5109.&19.&5&256&1\_40\_200\_10\_5 \textcolor{red}{\textcjheb{hyrm'}} AMRJH $|$(die) ihre Worte\\
6.&32.&1301.&128.&5114.&24.&6&158&5\_8\_30\_10\_100\_5 \textcolor{red}{\textcjheb{hqyl.hh}} HCLJQH $|$(sie) gl"attet\\
\end{tabular}\medskip \\
Ende des Verses 7.5\\
Verse: 180, Buchstaben: 29, 133, 5119, Totalwerte: 1887, 11090, 372200\\
\\
damit sie dich bewahre vor dem fremden Weibe, vor der Fremden, die ihre Worte gl"attet. -\\
\newpage 
{\bf -- 7.6}\\
\medskip \\
\begin{tabular}{rrrrrrrrp{120mm}}
WV&WK&WB&ABK&ABB&ABV&AnzB&TW&Zahlencode \textcolor{red}{$\boldsymbol{Grundtext}$} Umschrift $|$"Ubersetzung(en)\\
1.&33.&1302.&134.&5120.&1.&2&30&20\_10 \textcolor{red}{\textcjheb{yk}} KJ $|$denn\\
2.&34.&1303.&136.&5122.&3.&5&96&2\_8\_30\_6\_50 \textcolor{red}{\textcjheb{nwl.hb}} BCLWN $|$an dem Fenster/durch das Fenster\\
3.&35.&1304.&141.&5127.&8.&4&422&2\_10\_400\_10 \textcolor{red}{\textcjheb{ytyb}} BJTJ $|$meines Hauses\\
4.&36.&1305.&145.&5131.&12.&3&76&2\_70\_4 \textcolor{red}{\textcjheb{d`b}} BaD $|$durch\\
5.&37.&1306.&148.&5134.&15.&5&363&1\_300\_50\_2\_10 \textcolor{red}{\textcjheb{ybn+s'}} ASNBJ $|$mein Gitter\\
6.&38.&1307.&153.&5139.&20.&6&940&50\_300\_100\_80\_400\_10 \textcolor{red}{\textcjheb{ytpq+sn}} NSQPTJ $|$schaute ich hinaus/ich blickte hinaus\\
\end{tabular}\medskip \\
Ende des Verses 7.6\\
Verse: 181, Buchstaben: 25, 158, 5144, Totalwerte: 1927, 13017, 374127\\
\\
Denn an dem Fenster meines Hauses schaute ich durch mein Gitter hinaus;\\
\newpage 
{\bf -- 7.7}\\
\medskip \\
\begin{tabular}{rrrrrrrrp{120mm}}
WV&WK&WB&ABK&ABB&ABV&AnzB&TW&Zahlencode \textcolor{red}{$\boldsymbol{Grundtext}$} Umschrift $|$"Ubersetzung(en)\\
1.&39.&1308.&159.&5145.&1.&4&208&6\_1\_200\_1 \textcolor{red}{\textcjheb{'r'w}} WARA $|$und ich sah\\
2.&40.&1309.&163.&5149.&5.&6&533&2\_80\_400\_1\_10\_40 \textcolor{red}{\textcjheb{my'tpb}} BPTAJM $|$unter den Einf"altigen/bei den Unerfahrenen\\
3.&41.&1310.&169.&5155.&11.&5&68&1\_2\_10\_50\_5 \textcolor{red}{\textcjheb{hnyb'}} ABJNH $|$gewahrte/ich bemerkte\\
4.&42.&1311.&174.&5160.&16.&5&104&2\_2\_50\_10\_40 \textcolor{red}{\textcjheb{mynbb}} BBNJM $|$unter den S"ohnen\\
5.&43.&1312.&179.&5165.&21.&3&320&50\_70\_200 \textcolor{red}{\textcjheb{r`n}} NaR $|$einen J"ungling/(dass ein) Knabe\\
6.&44.&1313.&182.&5168.&24.&3&268&8\_60\_200 \textcolor{red}{\textcjheb{rs.h}} CsR $|$un-/(war) ermangelnd(er)\\
7.&45.&1314.&185.&5171.&27.&2&32&30\_2 \textcolor{red}{\textcjheb{bl}} LB $|$verst"andigen/Herz (=Verstand)\\
\end{tabular}\medskip \\
Ende des Verses 7.7\\
Verse: 182, Buchstaben: 28, 186, 5172, Totalwerte: 1533, 14550, 375660\\
\\
und ich sah unter den Einf"altigen, gewahrte unter den S"ohnen einen unverst"andigen J"ungling,\\
\newpage 
{\bf -- 7.8}\\
\medskip \\
\begin{tabular}{rrrrrrrrp{120mm}}
WV&WK&WB&ABK&ABB&ABV&AnzB&TW&Zahlencode \textcolor{red}{$\boldsymbol{Grundtext}$} Umschrift $|$"Ubersetzung(en)\\
1.&46.&1315.&187.&5173.&1.&3&272&70\_2\_200 \textcolor{red}{\textcjheb{rb`}} aBR $|$der hin und her ging/(er war) hin"ubergehend(er)\\
2.&47.&1316.&190.&5176.&4.&4&408&2\_300\_6\_100 \textcolor{red}{\textcjheb{qw+sb}} BSWQ $|$auf der Stra"se/durch den Markt\\
3.&48.&1317.&194.&5180.&8.&3&121&1\_90\_30 \textcolor{red}{\textcjheb{l.s'}} A"sL $|$neben\\
4.&49.&1318.&197.&5183.&11.&3&135&80\_50\_5 \textcolor{red}{\textcjheb{hnp}} PNH $|$ihrer Ecke\\
5.&50.&1319.&200.&5186.&14.&4&230&6\_4\_200\_20 \textcolor{red}{\textcjheb{krdw}} WDRK $|$und den Weg\\
6.&51.&1320.&204.&5190.&18.&4&417&2\_10\_400\_5 \textcolor{red}{\textcjheb{htyb}} BJTH $|$nach ihrem Haus/ihres Hauses\\
7.&52.&1321.&208.&5194.&22.&4&174&10\_90\_70\_4 \textcolor{red}{\textcjheb{d`.sy}} J"saD $|$(er) schritt\\
\end{tabular}\medskip \\
Ende des Verses 7.8\\
Verse: 183, Buchstaben: 25, 211, 5197, Totalwerte: 1757, 16307, 377417\\
\\
der hin und her ging auf der Stra"se, neben ihrer Ecke, und den Weg nach ihrem Hause schritt,\\
\newpage 
{\bf -- 7.9}\\
\medskip \\
\begin{tabular}{rrrrrrrrp{120mm}}
WV&WK&WB&ABK&ABB&ABV&AnzB&TW&Zahlencode \textcolor{red}{$\boldsymbol{Grundtext}$} Umschrift $|$"Ubersetzung(en)\\
1.&53.&1322.&212.&5198.&1.&4&432&2\_50\_300\_80 \textcolor{red}{\textcjheb{p+snb}} BNSP $|$in (der) D"ammerung\\
2.&54.&1323.&216.&5202.&5.&4&274&2\_70\_200\_2 \textcolor{red}{\textcjheb{br`b}} BaRB $|$am Abend\\
3.&55.&1324.&220.&5206.&9.&3&56&10\_6\_40 \textcolor{red}{\textcjheb{mwy}} JWM $|$des Tages\\
4.&56.&1325.&223.&5209.&12.&6&369&2\_1\_10\_300\_6\_50 \textcolor{red}{\textcjheb{nw+sy'b}} BAJSWN $|$in der Mitte\\
5.&57.&1326.&229.&5215.&18.&4&75&30\_10\_30\_5 \textcolor{red}{\textcjheb{hlyl}} LJLH $|$der Nacht\\
6.&58.&1327.&233.&5219.&22.&5&122&6\_1\_80\_30\_5 \textcolor{red}{\textcjheb{hlp'w}} WAPLH $|$und (in der) Dunkelheit\\
\end{tabular}\medskip \\
Ende des Verses 7.9\\
Verse: 184, Buchstaben: 26, 237, 5223, Totalwerte: 1328, 17635, 378745\\
\\
in der D"ammerung, am Abend des Tages, in der Mitte der Nacht und in der Dunkelheit.\\
\newpage 
{\bf -- 7.10}\\
\medskip \\
\begin{tabular}{rrrrrrrrp{120mm}}
WV&WK&WB&ABK&ABB&ABV&AnzB&TW&Zahlencode \textcolor{red}{$\boldsymbol{Grundtext}$} Umschrift $|$"Ubersetzung(en)\\
1.&59.&1328.&238.&5224.&1.&4&66&6\_5\_50\_5 \textcolor{red}{\textcjheb{hnhw}} WHNH $|$und siehe\\
2.&60.&1329.&242.&5228.&5.&3&306&1\_300\_5 \textcolor{red}{\textcjheb{h+s'}} ASH $|$(eine) Frau\\
3.&61.&1330.&245.&5231.&8.&6&737&30\_100\_200\_1\_400\_6 \textcolor{red}{\textcjheb{wt'rql}} LQRATW $|$(kam) ihm entgegen\\
4.&62.&1331.&251.&5237.&14.&3&710&300\_10\_400 \textcolor{red}{\textcjheb{ty+s}} SJT $|$im Anzug/(im) Gewand\\
5.&63.&1332.&254.&5240.&17.&4&68&7\_6\_50\_5 \textcolor{red}{\textcjheb{hnwz}} ZWNH $|$einer Hure/(der) Hure\\
6.&64.&1333.&258.&5244.&21.&5&746&6\_50\_90\_200\_400 \textcolor{red}{\textcjheb{tr.snw}} WN"sRT $|$und mit verstecktem/mit zielstrebigem\\
7.&65.&1334.&263.&5249.&26.&2&32&30\_2 \textcolor{red}{\textcjheb{bl}} LB $|$Herzen\\
\end{tabular}\medskip \\
Ende des Verses 7.10\\
Verse: 185, Buchstaben: 27, 264, 5250, Totalwerte: 2665, 20300, 381410\\
\\
Und siehe, ein Weib kam ihm entgegen im Anzug einer Hure und mit verstecktem Herzen. -\\
\newpage 
{\bf -- 7.11}\\
\medskip \\
\begin{tabular}{rrrrrrrrp{120mm}}
WV&WK&WB&ABK&ABB&ABV&AnzB&TW&Zahlencode \textcolor{red}{$\boldsymbol{Grundtext}$} Umschrift $|$"Ubersetzung(en)\\
1.&66.&1335.&265.&5251.&1.&4&60&5\_40\_10\_5 \textcolor{red}{\textcjheb{hymh}} HMJH $|$leidenschaftlich/l"armend\\
2.&67.&1336.&269.&5255.&5.&3&16&5\_10\_1 \textcolor{red}{\textcjheb{'yh}} HJA $|$sie (ist)\\
3.&68.&1337.&272.&5258.&8.&5&866&6\_60\_200\_200\_400 \textcolor{red}{\textcjheb{trrsw}} WsRRT $|$und unb"andig/und widerspenstig\\
4.&69.&1338.&277.&5263.&13.&5&419&2\_2\_10\_400\_5 \textcolor{red}{\textcjheb{htybb}} BBJTH $|$in ihrem Haus\\
5.&70.&1339.&282.&5268.&18.&2&31&30\_1 \textcolor{red}{\textcjheb{'l}} LA $|$nicht\\
6.&71.&1340.&284.&5270.&20.&5&386&10\_300\_20\_50\_6 \textcolor{red}{\textcjheb{wnk+sy}} JSKNW $|$bleiben/sie halten es aus\\
7.&72.&1341.&289.&5275.&25.&5&248&200\_3\_30\_10\_5 \textcolor{red}{\textcjheb{hylgr}} RGLJH $|$ihre F"u"se\\
\end{tabular}\medskip \\
Ende des Verses 7.11\\
Verse: 186, Buchstaben: 29, 293, 5279, Totalwerte: 2026, 22326, 383436\\
\\
Sie ist leidenschaftlich und unb"andig, ihre F"u"se bleiben nicht in ihrem Hause;\\
\newpage 
{\bf -- 7.12}\\
\medskip \\
\begin{tabular}{rrrrrrrrp{120mm}}
WV&WK&WB&ABK&ABB&ABV&AnzB&TW&Zahlencode \textcolor{red}{$\boldsymbol{Grundtext}$} Umschrift $|$"Ubersetzung(en)\\
1.&73.&1342.&294.&5280.&1.&3&190&80\_70\_40 \textcolor{red}{\textcjheb{m`p}} PaM $|$bald/einmal\\
2.&74.&1343.&297.&5283.&4.&4&106&2\_8\_6\_90 \textcolor{red}{\textcjheb{.sw.hb}} BCW"s $|$ist sie drau"sen/auf der Gasse\\
3.&75.&1344.&301.&5287.&8.&3&190&80\_70\_40 \textcolor{red}{\textcjheb{m`p}} PaM $|$bald/einmal\\
4.&76.&1345.&304.&5290.&11.&6&618&2\_200\_8\_2\_6\_400 \textcolor{red}{\textcjheb{twb.hrb}} BRCBWT $|$auf den Stra"sen/auf den Pl"atzen\\
5.&77.&1346.&310.&5296.&17.&4&127&6\_1\_90\_30 \textcolor{red}{\textcjheb{l.s'w}} WA"sL $|$und neben\\
6.&78.&1347.&314.&5300.&21.&2&50&20\_30 \textcolor{red}{\textcjheb{lk}} KL $|$jeder\\
7.&79.&1348.&316.&5302.&23.&3&135&80\_50\_5 \textcolor{red}{\textcjheb{hnp}} PNH $|$Ecke\\
8.&80.&1349.&319.&5305.&26.&4&603&400\_1\_200\_2 \textcolor{red}{\textcjheb{br't}} TARB $|$sie lauert\\
\end{tabular}\medskip \\
Ende des Verses 7.12\\
Verse: 187, Buchstaben: 29, 322, 5308, Totalwerte: 2019, 24345, 385455\\
\\
bald ist sie drau"sen, bald auf den Stra"sen, und neben jeder Ecke lauert sie. -\\
\newpage 
{\bf -- 7.13}\\
\medskip \\
\begin{tabular}{rrrrrrrrp{120mm}}
WV&WK&WB&ABK&ABB&ABV&AnzB&TW&Zahlencode \textcolor{red}{$\boldsymbol{Grundtext}$} Umschrift $|$"Ubersetzung(en)\\
1.&81.&1350.&323.&5309.&1.&7&141&6\_5\_8\_7\_10\_100\_5 \textcolor{red}{\textcjheb{hqyz.hhw}} WHCZJQH $|$und sie ergriff/und sie fasste fest\\
2.&82.&1351.&330.&5316.&8.&2&8&2\_6 \textcolor{red}{\textcjheb{wb}} BW $|$ihn\\
3.&83.&1352.&332.&5318.&10.&5&461&6\_50\_300\_100\_5 \textcolor{red}{\textcjheb{hq+snw}} WNSQH $|$und (sie) k"usste\\
4.&84.&1353.&337.&5323.&15.&2&36&30\_6 \textcolor{red}{\textcjheb{wl}} LW $|$ihn\\
5.&85.&1354.&339.&5325.&17.&4&87&5\_70\_7\_5 \textcolor{red}{\textcjheb{hz`h}} HaZH $|$mit unversch"amtem/sie zeigte frech\\
6.&86.&1355.&343.&5329.&21.&4&145&80\_50\_10\_5 \textcolor{red}{\textcjheb{hynp}} PNJH $|$(ihr) (An)Gesicht\\
7.&87.&1356.&347.&5333.&25.&5&647&6\_400\_1\_40\_200 \textcolor{red}{\textcjheb{rm'tw}} WTAMR $|$sprach sie/und sie sagte\\
8.&88.&1357.&352.&5338.&30.&2&36&30\_6 \textcolor{red}{\textcjheb{wl}} LW $|$zu ihm\\
\end{tabular}\medskip \\
Ende des Verses 7.13\\
Verse: 188, Buchstaben: 31, 353, 5339, Totalwerte: 1561, 25906, 387016\\
\\
Und sie ergriff ihn und k"u"ste ihn, und mit unversch"amtem Angesicht sprach sie zu ihm:\\
\newpage 
{\bf -- 7.14}\\
\medskip \\
\begin{tabular}{rrrrrrrrp{120mm}}
WV&WK&WB&ABK&ABB&ABV&AnzB&TW&Zahlencode \textcolor{red}{$\boldsymbol{Grundtext}$} Umschrift $|$"Ubersetzung(en)\\
1.&89.&1358.&354.&5340.&1.&4&27&7\_2\_8\_10 \textcolor{red}{\textcjheb{y.hbz}} ZBCJ $|$Opfer\\
2.&90.&1359.&358.&5344.&5.&5&420&300\_30\_40\_10\_40 \textcolor{red}{\textcjheb{myml+s}} SLMJM $|$(des) Friedens\\
3.&91.&1360.&363.&5349.&10.&3&110&70\_30\_10 \textcolor{red}{\textcjheb{yl`}} aLJ $|$lagen mir ob/obliegen mir\\
4.&92.&1361.&366.&5352.&13.&4&61&5\_10\_6\_40 \textcolor{red}{\textcjheb{mwyh}} HJWM $|$heute\\
5.&93.&1362.&370.&5356.&17.&5&780&300\_30\_40\_400\_10 \textcolor{red}{\textcjheb{ytml+s}} SLMTJ $|$habe ich bezahlt/ich habe erf"ullt\\
6.&94.&1363.&375.&5361.&22.&4&264&50\_4\_200\_10 \textcolor{red}{\textcjheb{yrdn}} NDRJ $|$meine Gel"ubde/meine Gel"obnisse\\
\end{tabular}\medskip \\
Ende des Verses 7.14\\
Verse: 189, Buchstaben: 25, 378, 5364, Totalwerte: 1662, 27568, 388678\\
\\
Friedensopfer lagen mir ob, heute habe ich meine Gel"ubde bezahlt;\\
\newpage 
{\bf -- 7.15}\\
\medskip \\
\begin{tabular}{rrrrrrrrp{120mm}}
WV&WK&WB&ABK&ABB&ABV&AnzB&TW&Zahlencode \textcolor{red}{$\boldsymbol{Grundtext}$} Umschrift $|$"Ubersetzung(en)\\
1.&95.&1364.&379.&5365.&1.&2&100&70\_30 \textcolor{red}{\textcjheb{l`}} aL $|$wegen\\
2.&96.&1365.&381.&5367.&3.&2&70&20\_50 \textcolor{red}{\textcjheb{nk}} KN $|$diesem\\
3.&97.&1366.&383.&5369.&5.&5&511&10\_90\_1\_400\_10 \textcolor{red}{\textcjheb{yt'.sy}} J"sATJ $|$bin ich ausgegangen/ich ging aus\\
4.&98.&1367.&388.&5374.&10.&6&751&30\_100\_200\_1\_400\_20 \textcolor{red}{\textcjheb{kt'rql}} LQRATK $|$dir entgegen/um zu begegnen dir\\
5.&99.&1368.&394.&5380.&16.&4&538&30\_300\_8\_200 \textcolor{red}{\textcjheb{r.h+sl}} LSCR $|$(um) zu suchen\\
6.&100.&1369.&398.&5384.&20.&4&160&80\_50\_10\_20 \textcolor{red}{\textcjheb{kynp}} PNJK $|$dein Antlitz/dich\\
7.&101.&1370.&402.&5388.&24.&6&158&6\_1\_40\_90\_1\_20 \textcolor{red}{\textcjheb{k'.sm'w}} WAM"sAK $|$und ich habe gefunden dich\\
\end{tabular}\medskip \\
Ende des Verses 7.15\\
Verse: 190, Buchstaben: 29, 407, 5393, Totalwerte: 2288, 29856, 390966\\
\\
darum bin ich ausgegangen, dir entgegen, um dein Antlitz zu suchen, und ich habe dich gefunden.\\
\newpage 
{\bf -- 7.16}\\
\medskip \\
\begin{tabular}{rrrrrrrrp{120mm}}
WV&WK&WB&ABK&ABB&ABV&AnzB&TW&Zahlencode \textcolor{red}{$\boldsymbol{Grundtext}$} Umschrift $|$"Ubersetzung(en)\\
1.&102.&1371.&408.&5394.&1.&6&296&40\_200\_2\_4\_10\_40 \textcolor{red}{\textcjheb{mydbrm}} MRBDJM $|$mit Teppichen/mit Decken\\
2.&103.&1372.&414.&5400.&7.&5&616&200\_2\_4\_400\_10 \textcolor{red}{\textcjheb{ytdbr}} RBDTJ $|$habe ich bereitet/ich bereitete\\
3.&104.&1373.&419.&5405.&12.&4&580&70\_200\_300\_10 \textcolor{red}{\textcjheb{y+sr`}} aRSJ $|$mein Bett\\
4.&105.&1374.&423.&5409.&16.&5&425&8\_9\_2\_6\_400 \textcolor{red}{\textcjheb{twb.t.h}} CtBWT $|$mit bunten Decken/mit bunt gestreiften\\
5.&106.&1375.&428.&5414.&21.&4&66&1\_9\_6\_50 \textcolor{red}{\textcjheb{nw.t'}} AtWN $|$von Garn/Linnen\\
6.&107.&1376.&432.&5418.&25.&5&380&40\_90\_200\_10\_40 \textcolor{red}{\textcjheb{myr.sm}} M"sRJM $|$"agyptischem/"Agyptens\\
\end{tabular}\medskip \\
Ende des Verses 7.16\\
Verse: 191, Buchstaben: 29, 436, 5422, Totalwerte: 2363, 32219, 393329\\
\\
Mit Teppichen habe ich mein Bett bereitet, mit bunten Decken von "agyptischem Garne;\\
\newpage 
{\bf -- 7.17}\\
\medskip \\
\begin{tabular}{rrrrrrrrp{120mm}}
WV&WK&WB&ABK&ABB&ABV&AnzB&TW&Zahlencode \textcolor{red}{$\boldsymbol{Grundtext}$} Umschrift $|$"Ubersetzung(en)\\
1.&108.&1377.&437.&5423.&1.&4&540&50\_80\_400\_10 \textcolor{red}{\textcjheb{ytpn}} NPTJ $|$ich habe benetzt/ich besprengte\\
2.&109.&1378.&441.&5427.&5.&5&372&40\_300\_20\_2\_10 \textcolor{red}{\textcjheb{ybk+sm}} MSKBJ $|$mein Lager\\
3.&110.&1379.&446.&5432.&10.&2&240&40\_200 \textcolor{red}{\textcjheb{rm}} MR $|$(mit) Myrrhe\\
4.&111.&1380.&448.&5434.&12.&5&86&1\_5\_30\_10\_40 \textcolor{red}{\textcjheb{mylh'}} AHLJM $|$Aloe\\
5.&112.&1381.&453.&5439.&17.&6&252&6\_100\_50\_40\_6\_50 \textcolor{red}{\textcjheb{nwmnqw}} WQNMWN $|$und Zimt\\
\end{tabular}\medskip \\
Ende des Verses 7.17\\
Verse: 192, Buchstaben: 22, 458, 5444, Totalwerte: 1490, 33709, 394819\\
\\
ich habe mein Lager benetzt mit Myrrhe, Aloe und Zimmet.\\
\newpage 
{\bf -- 7.18}\\
\medskip \\
\begin{tabular}{rrrrrrrrp{120mm}}
WV&WK&WB&ABK&ABB&ABV&AnzB&TW&Zahlencode \textcolor{red}{$\boldsymbol{Grundtext}$} Umschrift $|$"Ubersetzung(en)\\
1.&113.&1382.&459.&5445.&1.&3&55&30\_20\_5 \textcolor{red}{\textcjheb{hkl}} LKH $|$komm/gehe\\
2.&114.&1383.&462.&5448.&4.&4&261&50\_200\_6\_5 \textcolor{red}{\textcjheb{hwrn}} NRWH $|$wir wollen uns berauschen/wir wollen uns satttrinken\\
3.&115.&1384.&466.&5452.&8.&4&58&4\_4\_10\_40 \textcolor{red}{\textcjheb{mydd}} DDJM $|$in Liebe/in Geliebten\\
4.&116.&1385.&470.&5456.&12.&2&74&70\_4 \textcolor{red}{\textcjheb{d`}} aD $|$bis\\
5.&117.&1386.&472.&5458.&14.&4&307&5\_2\_100\_200 \textcolor{red}{\textcjheb{rqbh}} HBQR $|$an den Morgen/zum Morgen\\
6.&118.&1387.&476.&5462.&18.&6&615&50\_400\_70\_30\_60\_5 \textcolor{red}{\textcjheb{hsl`tn}} NTaLsH $|$(wir wollen) uns erg"otzen\\
7.&119.&1388.&482.&5468.&24.&6&60&2\_1\_5\_2\_10\_40 \textcolor{red}{\textcjheb{mybh'b}} BAHBJM $|$an Liebkosungen/in Buhlschaften\\
\end{tabular}\medskip \\
Ende des Verses 7.18\\
Verse: 193, Buchstaben: 29, 487, 5473, Totalwerte: 1430, 35139, 396249\\
\\
Komm, wir wollen uns in Liebe berauschen bis an den Morgen, an Liebkosungen uns erg"otzen.\\
\newpage 
{\bf -- 7.19}\\
\medskip \\
\begin{tabular}{rrrrrrrrp{120mm}}
WV&WK&WB&ABK&ABB&ABV&AnzB&TW&Zahlencode \textcolor{red}{$\boldsymbol{Grundtext}$} Umschrift $|$"Ubersetzung(en)\\
1.&120.&1389.&488.&5474.&1.&2&30&20\_10 \textcolor{red}{\textcjheb{yk}} KJ $|$denn\\
2.&121.&1390.&490.&5476.&3.&3&61&1\_10\_50 \textcolor{red}{\textcjheb{ny'}} AJN $|$nicht ist\\
3.&122.&1391.&493.&5479.&6.&4&316&5\_1\_10\_300 \textcolor{red}{\textcjheb{+sy'h}} HAJS $|$der (Ehe)Mann\\
4.&123.&1392.&497.&5483.&10.&5&420&2\_2\_10\_400\_6 \textcolor{red}{\textcjheb{wtybb}} BBJTW $|$zu Hause/in sein(em) Haus\\
5.&124.&1393.&502.&5488.&15.&3&55&5\_30\_20 \textcolor{red}{\textcjheb{klh}} HLK $|$(er) ist gegangen\\
6.&125.&1394.&505.&5491.&18.&4&226&2\_4\_200\_20 \textcolor{red}{\textcjheb{krdb}} BDRK $|$auf (eine) Reise\\
7.&126.&1395.&509.&5495.&22.&5&354&40\_200\_8\_6\_100 \textcolor{red}{\textcjheb{qw.hrm}} MRCWQ $|$weite/in die Ferne\\
\end{tabular}\medskip \\
Ende des Verses 7.19\\
Verse: 194, Buchstaben: 26, 513, 5499, Totalwerte: 1462, 36601, 397711\\
\\
Denn der Mann ist nicht zu Hause, er ist auf eine weite Reise gegangen;\\
\newpage 
{\bf -- 7.20}\\
\medskip \\
\begin{tabular}{rrrrrrrrp{120mm}}
WV&WK&WB&ABK&ABB&ABV&AnzB&TW&Zahlencode \textcolor{red}{$\boldsymbol{Grundtext}$} Umschrift $|$"Ubersetzung(en)\\
1.&127.&1396.&514.&5500.&1.&4&496&90\_200\_6\_200 \textcolor{red}{\textcjheb{rwr.s}} "sRWR $|$(den) Beutel\\
2.&128.&1397.&518.&5504.&5.&4&165&5\_20\_60\_80 \textcolor{red}{\textcjheb{pskh}} HKsP $|$(des) Geld(es)\\
3.&129.&1398.&522.&5508.&9.&3&138&30\_100\_8 \textcolor{red}{\textcjheb{.hql}} LQC $|$hat er genommen/er nahm\\
4.&130.&1399.&525.&5511.&12.&4&22&2\_10\_4\_6 \textcolor{red}{\textcjheb{wdyb}} BJDW $|$in seine Hand\\
5.&131.&1400.&529.&5515.&16.&4&86&30\_10\_6\_40 \textcolor{red}{\textcjheb{mwyl}} LJWM $|$am Tag/zum Tag\\
6.&132.&1401.&533.&5519.&20.&4&86&5\_20\_60\_1 \textcolor{red}{\textcjheb{'skh}} HKsA $|$des Vollmondes\\
7.&133.&1402.&537.&5523.&24.&3&13&10\_2\_1 \textcolor{red}{\textcjheb{'by}} JBA $|$wird er heimkehren/er wird heimkommen\\
8.&134.&1403.&540.&5526.&27.&4&418&2\_10\_400\_6 \textcolor{red}{\textcjheb{wtyb}} BJTW $|$/in sein Haus\\
\end{tabular}\medskip \\
Ende des Verses 7.20\\
Verse: 195, Buchstaben: 30, 543, 5529, Totalwerte: 1424, 38025, 399135\\
\\
er hat den Geldbeutel in seine Hand genommen, am Tage des Vollmondes wird er heimkehren.\\
\newpage 
{\bf -- 7.21}\\
\medskip \\
\begin{tabular}{rrrrrrrrp{120mm}}
WV&WK&WB&ABK&ABB&ABV&AnzB&TW&Zahlencode \textcolor{red}{$\boldsymbol{Grundtext}$} Umschrift $|$"Ubersetzung(en)\\
1.&135.&1404.&544.&5530.&1.&4&420&5\_9\_400\_6 \textcolor{red}{\textcjheb{wt.th}} HtTW $|$sie verleitete ihn/sie hat verleitet ihn\\
2.&136.&1405.&548.&5534.&5.&3&204&2\_200\_2 \textcolor{red}{\textcjheb{brb}} BRB $|$durch ihr vieles/durch die Menge\\
3.&137.&1406.&551.&5537.&8.&4&143&30\_100\_8\_5 \textcolor{red}{\textcjheb{h.hql}} LQCH $|$Bereden/ihrer "Uberredung\\
4.&138.&1407.&555.&5541.&12.&4&140&2\_8\_30\_100 \textcolor{red}{\textcjheb{ql.hb}} BCLQ $|$durch die Gl"atte\\
5.&139.&1408.&559.&5545.&16.&5&795&300\_80\_400\_10\_5 \textcolor{red}{\textcjheb{hytp+s}} SPTJH $|$ihrer Lippen\\
6.&140.&1409.&564.&5550.&21.&6&478&400\_4\_10\_8\_50\_6 \textcolor{red}{\textcjheb{wn.hydt}} TDJCNW $|$riss ihn fort/sie verf"uhrte ihn\\
\end{tabular}\medskip \\
Ende des Verses 7.21\\
Verse: 196, Buchstaben: 26, 569, 5555, Totalwerte: 2180, 40205, 401315\\
\\
Sie verleitete ihn durch ihr vieles Bereden, ri"s ihn fort durch die Gl"atte ihrer Lippen.\\
\newpage 
{\bf -- 7.22}\\
\medskip \\
\begin{tabular}{rrrrrrrrp{120mm}}
WV&WK&WB&ABK&ABB&ABV&AnzB&TW&Zahlencode \textcolor{red}{$\boldsymbol{Grundtext}$} Umschrift $|$"Ubersetzung(en)\\
1.&141.&1410.&570.&5556.&1.&4&61&5\_6\_30\_20 \textcolor{red}{\textcjheb{klwh}} HWLK $|$er ging/(er war) gehend(er)\\
2.&142.&1411.&574.&5560.&5.&5&224&1\_8\_200\_10\_5 \textcolor{red}{\textcjheb{hyr.h'}} ACRJH $|$ihr nach/hinter ihr\\
3.&143.&1412.&579.&5565.&10.&4&521&80\_400\_1\_40 \textcolor{red}{\textcjheb{m'tp}} PTAM $|$auf einmal/augenblicklich\\
4.&144.&1413.&583.&5569.&14.&4&526&20\_300\_6\_200 \textcolor{red}{\textcjheb{rw+sk}} KSWR $|$wie ein Ochs/wie ein Stier\\
5.&145.&1414.&587.&5573.&18.&2&31&1\_30 \textcolor{red}{\textcjheb{l'}} AL $|$zur\\
6.&146.&1415.&589.&5575.&20.&3&19&9\_2\_8 \textcolor{red}{\textcjheb{.hb.t}} tBC $|$Schlachtbank/Schlachtung\\
7.&147.&1416.&592.&5578.&23.&4&19&10\_2\_6\_1 \textcolor{red}{\textcjheb{'wby}} JBWA $|$geht/(er) kommt\\
8.&148.&1417.&596.&5582.&27.&5&176&6\_20\_70\_20\_60 \textcolor{red}{\textcjheb{sk`kw}} WKaKs $|$und wie (in) Fu"sfessel(n)\\
9.&149.&1418.&601.&5587.&32.&2&31&1\_30 \textcolor{red}{\textcjheb{l'}} AL $|$zur\\
10.&150.&1419.&603.&5589.&34.&4&306&40\_6\_60\_200 \textcolor{red}{\textcjheb{rswm}} MWsR $|$Z"uchtigung/Strafe\\
11.&151.&1420.&607.&5593.&38.&4&47&1\_6\_10\_30 \textcolor{red}{\textcjheb{lyw'}} AWJL $|$des Narren dienen/(ein) Tor\\
\end{tabular}\medskip \\
Ende des Verses 7.22\\
Verse: 197, Buchstaben: 41, 610, 5596, Totalwerte: 1961, 42166, 403276\\
\\
Auf einmal ging er ihr nach, wie ein Ochs zur Schlachtbank geht, und wie Fu"sfesseln zur Z"uchtigung des Narren dienen,\\
\newpage 
{\bf -- 7.23}\\
\medskip \\
\begin{tabular}{rrrrrrrrp{120mm}}
WV&WK&WB&ABK&ABB&ABV&AnzB&TW&Zahlencode \textcolor{red}{$\boldsymbol{Grundtext}$} Umschrift $|$"Ubersetzung(en)\\
1.&152.&1421.&611.&5597.&1.&2&74&70\_4 \textcolor{red}{\textcjheb{d`}} aD $|$bis\\
2.&153.&1422.&613.&5599.&3.&4&128&10\_80\_30\_8 \textcolor{red}{\textcjheb{.hlpy}} JPLC $|$(er (=es)) zerspaltet\\
3.&154.&1423.&617.&5603.&7.&2&98&8\_90 \textcolor{red}{\textcjheb{.s.h}} C"s $|$(ein) Pfeil\\
4.&155.&1424.&619.&5605.&9.&4&32&20\_2\_4\_6 \textcolor{red}{\textcjheb{wdbk}} KBDW $|$seine Leber\\
5.&156.&1425.&623.&5609.&13.&4&265&20\_40\_5\_200 \textcolor{red}{\textcjheb{rhmk}} KMHR $|$wie eilt\\
6.&157.&1426.&627.&5613.&17.&4&376&90\_80\_6\_200 \textcolor{red}{\textcjheb{rwp.s}} "sPWR $|$(ein) Vogel\\
7.&158.&1427.&631.&5617.&21.&2&31&1\_30 \textcolor{red}{\textcjheb{l'}} AL $|$zur\\
8.&159.&1428.&633.&5619.&23.&2&88&80\_8 \textcolor{red}{\textcjheb{.hp}} PC $|$Schlinge\\
9.&160.&1429.&635.&5621.&25.&3&37&6\_30\_1 \textcolor{red}{\textcjheb{'lw}} WLA $|$und nicht\\
10.&161.&1430.&638.&5624.&28.&3&84&10\_4\_70 \textcolor{red}{\textcjheb{`dy}} JDa $|$(er) wei"s\\
11.&162.&1431.&641.&5627.&31.&2&30&20\_10 \textcolor{red}{\textcjheb{yk}} KJ $|$dass\\
12.&163.&1432.&643.&5629.&33.&5&438&2\_50\_80\_300\_6 \textcolor{red}{\textcjheb{w+spnb}} BNPSW $|$(gegen) sein Leben\\
13.&164.&1433.&648.&5634.&38.&3&12&5\_6\_1 \textcolor{red}{\textcjheb{'wh}} HWA $|$es gilt/er (=es) ist\\
\end{tabular}\medskip \\
Ende des Verses 7.23\\
Verse: 198, Buchstaben: 40, 650, 5636, Totalwerte: 1693, 43859, 404969\\
\\
bis ein Pfeil seine Leber zerspaltet; wie ein Vogel zur Schlinge eilt und nicht wei"s, da"s es sein Leben gilt. -\\
\newpage 
{\bf -- 7.24}\\
\medskip \\
\begin{tabular}{rrrrrrrrp{120mm}}
WV&WK&WB&ABK&ABB&ABV&AnzB&TW&Zahlencode \textcolor{red}{$\boldsymbol{Grundtext}$} Umschrift $|$"Ubersetzung(en)\\
1.&165.&1434.&651.&5637.&1.&4&481&6\_70\_400\_5 \textcolor{red}{\textcjheb{ht`w}} WaTH $|$nun denn/und nun\\
2.&166.&1435.&655.&5641.&5.&4&102&2\_50\_10\_40 \textcolor{red}{\textcjheb{mynb}} BNJM $|$(ihr) S"ohne\\
3.&167.&1436.&659.&5645.&9.&4&416&300\_40\_70\_6 \textcolor{red}{\textcjheb{w`m+s}} SMaW $|$h"oret\\
4.&168.&1437.&663.&5649.&13.&2&40&30\_10 \textcolor{red}{\textcjheb{yl}} LJ $|$auf mich\\
5.&169.&1438.&665.&5651.&15.&7&429&6\_5\_100\_300\_10\_2\_6 \textcolor{red}{\textcjheb{wby+sqhw}} WHQSJBW $|$und horcht auf/und merket auf\\
6.&170.&1439.&672.&5658.&22.&5&281&30\_1\_40\_200\_10 \textcolor{red}{\textcjheb{yrm'l}} LAMRJ $|$auf die Worte\\
7.&171.&1440.&677.&5663.&27.&2&90&80\_10 \textcolor{red}{\textcjheb{yp}} PJ $|$meines Mundes\\
\end{tabular}\medskip \\
Ende des Verses 7.24\\
Verse: 199, Buchstaben: 28, 678, 5664, Totalwerte: 1839, 45698, 406808\\
\\
Nun denn, ihr S"ohne, h"oret auf mich, und horchet auf die Worte meines Mundes!\\
\newpage 
{\bf -- 7.25}\\
\medskip \\
\begin{tabular}{rrrrrrrrp{120mm}}
WV&WK&WB&ABK&ABB&ABV&AnzB&TW&Zahlencode \textcolor{red}{$\boldsymbol{Grundtext}$} Umschrift $|$"Ubersetzung(en)\\
1.&172.&1441.&679.&5665.&1.&2&31&1\_30 \textcolor{red}{\textcjheb{l'}} AL $|$nicht\\
2.&173.&1442.&681.&5667.&3.&3&319&10\_300\_9 \textcolor{red}{\textcjheb{.t+sy}} JSt $|$wende sich ab/er (=es) soll abweichen\\
3.&174.&1443.&684.&5670.&6.&2&31&1\_30 \textcolor{red}{\textcjheb{l'}} AL $|$nach/zu\\
4.&175.&1444.&686.&5672.&8.&5&239&4\_200\_20\_10\_5 \textcolor{red}{\textcjheb{hykrd}} DRKJH $|$ihren Wegen\\
5.&176.&1445.&691.&5677.&13.&3&52&30\_2\_20 \textcolor{red}{\textcjheb{kbl}} LBK $|$dein Herz\\
6.&177.&1446.&694.&5680.&16.&2&31&1\_30 \textcolor{red}{\textcjheb{l'}} AL $|$(und) nicht\\
7.&178.&1447.&696.&5682.&18.&3&870&400\_400\_70 \textcolor{red}{\textcjheb{`tt}} TTa $|$irre umher/du sollst umherirren\\
8.&179.&1448.&699.&5685.&21.&9&885&2\_50\_400\_10\_2\_6\_400\_10\_5 \textcolor{red}{\textcjheb{hytwbytnb}} BNTJBWTJH $|$auf ihren Pfaden\\
\end{tabular}\medskip \\
Ende des Verses 7.25\\
Verse: 200, Buchstaben: 29, 707, 5693, Totalwerte: 2458, 48156, 409266\\
\\
Dein Herz wende sich nicht ab nach ihren Wegen, und irre nicht umher auf ihren Pfaden!\\
\newpage 
{\bf -- 7.26}\\
\medskip \\
\begin{tabular}{rrrrrrrrp{120mm}}
WV&WK&WB&ABK&ABB&ABV&AnzB&TW&Zahlencode \textcolor{red}{$\boldsymbol{Grundtext}$} Umschrift $|$"Ubersetzung(en)\\
1.&180.&1449.&708.&5694.&1.&2&30&20\_10 \textcolor{red}{\textcjheb{yk}} KJ $|$denn\\
2.&181.&1450.&710.&5696.&3.&4&252&200\_2\_10\_40 \textcolor{red}{\textcjheb{mybr}} RBJM $|$viele\\
3.&182.&1451.&714.&5700.&7.&5&118&8\_30\_30\_10\_40 \textcolor{red}{\textcjheb{myll.h}} CLLJM $|$Erschlagene/Durchbohrte\\
4.&183.&1452.&719.&5705.&12.&5&130&5\_80\_10\_30\_5 \textcolor{red}{\textcjheb{hlyph}} HPJLH $|$hat sie niedergestreckt/sie machte fallen\\
5.&184.&1453.&724.&5710.&17.&6&256&6\_70\_90\_40\_10\_40 \textcolor{red}{\textcjheb{mym.s`w}} Wa"sMJM $|$und zahlreich sind\\
6.&185.&1454.&730.&5716.&23.&2&50&20\_30 \textcolor{red}{\textcjheb{lk}} KL $|$all(s)\\
7.&186.&1455.&732.&5718.&25.&5&223&5\_200\_3\_10\_5 \textcolor{red}{\textcjheb{hygrh}} HRGJH $|$ihre Ermordeten/ihre Get"oteten\\
\end{tabular}\medskip \\
Ende des Verses 7.26\\
Verse: 201, Buchstaben: 29, 736, 5722, Totalwerte: 1059, 49215, 410325\\
\\
Denn viele Erschlagene hat sie niedergestreckt, und zahlreich sind alle ihre Ermordeten.\\
\newpage 
{\bf -- 7.27}\\
\medskip \\
\begin{tabular}{rrrrrrrrp{120mm}}
WV&WK&WB&ABK&ABB&ABV&AnzB&TW&Zahlencode \textcolor{red}{$\boldsymbol{Grundtext}$} Umschrift $|$"Ubersetzung(en)\\
1.&187.&1456.&737.&5723.&1.&4&234&4\_200\_20\_10 \textcolor{red}{\textcjheb{ykrd}} DRKJ $|$(es) sind Wege/(die) Wege\\
2.&188.&1457.&741.&5727.&5.&4&337&300\_1\_6\_30 \textcolor{red}{\textcjheb{lw'+s}} SAWL $|$zum Scheol/der Unterwelt\\
3.&189.&1458.&745.&5731.&9.&4&417&2\_10\_400\_5 \textcolor{red}{\textcjheb{htyb}} BJTH $|$ihr Haus\\
4.&190.&1459.&749.&5735.&13.&5&620&10\_200\_4\_6\_400 \textcolor{red}{\textcjheb{twdry}} JRDWT $|$die hinabf"uhren/hinabsteigend(e)\\
5.&191.&1460.&754.&5740.&18.&2&31&1\_30 \textcolor{red}{\textcjheb{l'}} AL $|$zu\\
6.&192.&1461.&756.&5742.&20.&4&222&8\_4\_200\_10 \textcolor{red}{\textcjheb{yrd.h}} CDRJ $|$(den) Kammern\\
7.&193.&1462.&760.&5746.&24.&3&446&40\_6\_400 \textcolor{red}{\textcjheb{twm}} MWT $|$des Todes\\
\end{tabular}\medskip \\
Ende des Verses 7.27\\
Verse: 202, Buchstaben: 26, 762, 5748, Totalwerte: 2307, 51522, 412632\\
\\
Ihr Haus sind Wege zum Scheol, die hinabf"uhren zu den Kammern des Todes.\\
\\
{\bf Ende des Kapitels 7}\\
\newpage 
{\bf -- 8.1}\\
\medskip \\
\begin{tabular}{rrrrrrrrp{120mm}}
WV&WK&WB&ABK&ABB&ABV&AnzB&TW&Zahlencode \textcolor{red}{$\boldsymbol{Grundtext}$} Umschrift $|$"Ubersetzung(en)\\
1.&1.&1463.&1.&5749.&1.&3&36&5\_30\_1 \textcolor{red}{\textcjheb{'lh}} HLA $|$(etwa) nicht\\
2.&2.&1464.&4.&5752.&4.&4&73&8\_20\_40\_5 \textcolor{red}{\textcjheb{hmk.h}} CKMH $|$die Weisheit\\
3.&3.&1465.&8.&5756.&8.&4&701&400\_100\_200\_1 \textcolor{red}{\textcjheb{'rqt}} TQRA $|$(sie) ruft\\
4.&4.&1466.&12.&5760.&12.&6&469&6\_400\_2\_6\_50\_5 \textcolor{red}{\textcjheb{hnwbtw}} WTBWNH $|$und die Einsicht\\
5.&5.&1467.&18.&5766.&18.&3&850&400\_400\_50 \textcolor{red}{\textcjheb{ntt}} TTN $|$l"asst nicht erschallen/sie erhebt\\
6.&6.&1468.&21.&5769.&21.&4&141&100\_6\_30\_5 \textcolor{red}{\textcjheb{hlwq}} QWLH $|$ihre Stimme\\
\end{tabular}\medskip \\
Ende des Verses 8.1\\
Verse: 203, Buchstaben: 24, 24, 5772, Totalwerte: 2270, 2270, 414902\\
\\
Ruft nicht die Weisheit, und l"a"st nicht die Einsicht ihre Stimme erschallen?\\
\newpage 
{\bf -- 8.2}\\
\medskip \\
\begin{tabular}{rrrrrrrrp{120mm}}
WV&WK&WB&ABK&ABB&ABV&AnzB&TW&Zahlencode \textcolor{red}{$\boldsymbol{Grundtext}$} Umschrift $|$"Ubersetzung(en)\\
1.&7.&1469.&25.&5773.&1.&4&503&2\_200\_1\_300 \textcolor{red}{\textcjheb{+s'rb}} BRAS $|$oben auf/auf der Kuppe\\
2.&8.&1470.&29.&5777.&5.&6&336&40\_200\_6\_40\_10\_40 \textcolor{red}{\textcjheb{mymwrm}} MRWMJM $|$den Erh"ohungen/der H"ohen\\
3.&9.&1471.&35.&5783.&11.&3&110&70\_30\_10 \textcolor{red}{\textcjheb{yl`}} aLJ $|$am\\
4.&10.&1472.&38.&5786.&14.&3&224&4\_200\_20 \textcolor{red}{\textcjheb{krd}} DRK $|$Weg\\
5.&11.&1473.&41.&5789.&17.&3&412&2\_10\_400 \textcolor{red}{\textcjheb{tyb}} BJT $|$da wo zusammensto"sen/(im) Haus\\
6.&12.&1474.&44.&5792.&20.&6&868&50\_400\_10\_2\_6\_400 \textcolor{red}{\textcjheb{twbytn}} NTJBWT $|$(der) Pfade\\
7.&13.&1475.&50.&5798.&26.&4&147&50\_90\_2\_5 \textcolor{red}{\textcjheb{hb.sn}} N"sBH $|$hat sie sich aufgestellt/sie stellte sich hin\\
\end{tabular}\medskip \\
Ende des Verses 8.2\\
Verse: 204, Buchstaben: 29, 53, 5801, Totalwerte: 2600, 4870, 417502\\
\\
Oben auf den Erh"ohungen am Wege, da wo Pfade zusammensto"sen, hat sie sich aufgestellt.\\
\newpage 
{\bf -- 8.3}\\
\medskip \\
\begin{tabular}{rrrrrrrrp{120mm}}
WV&WK&WB&ABK&ABB&ABV&AnzB&TW&Zahlencode \textcolor{red}{$\boldsymbol{Grundtext}$} Umschrift $|$"Ubersetzung(en)\\
1.&14.&1476.&54.&5802.&1.&3&44&30\_10\_4 \textcolor{red}{\textcjheb{dyl}} LJD $|$zur Seite/an der Seite\\
2.&15.&1477.&57.&5805.&4.&5&620&300\_70\_200\_10\_40 \textcolor{red}{\textcjheb{myr`+s}} SaRJM $|$(der) Tore\\
3.&16.&1478.&62.&5810.&9.&3&120&30\_80\_10 \textcolor{red}{\textcjheb{ypl}} LPJ $|$wo sich auftut/an der "Offnung\\
4.&17.&1479.&65.&5813.&12.&3&700&100\_200\_400 \textcolor{red}{\textcjheb{trq}} QRT $|$die Stadt/(der) Stadt\\
5.&18.&1480.&68.&5816.&15.&4&49&40\_2\_6\_1 \textcolor{red}{\textcjheb{'wbm}} MBWA $|$am Eingang\\
6.&19.&1481.&72.&5820.&19.&5&538&80\_400\_8\_10\_40 \textcolor{red}{\textcjheb{my.htp}} PTCJM $|$der Pforten\\
7.&20.&1482.&77.&5825.&24.&4&655&400\_200\_50\_5 \textcolor{red}{\textcjheb{hnrt}} TRNH $|$sie schreit\\
\end{tabular}\medskip \\
Ende des Verses 8.3\\
Verse: 205, Buchstaben: 27, 80, 5828, Totalwerte: 2726, 7596, 420228\\
\\
Zur Seite der Tore, wo die Stadt sich auftut, am Eingang der Pforten schreit sie:\\
\newpage 
{\bf -- 8.4}\\
\medskip \\
\begin{tabular}{rrrrrrrrp{120mm}}
WV&WK&WB&ABK&ABB&ABV&AnzB&TW&Zahlencode \textcolor{red}{$\boldsymbol{Grundtext}$} Umschrift $|$"Ubersetzung(en)\\
1.&21.&1483.&81.&5829.&1.&5&101&1\_30\_10\_20\_40 \textcolor{red}{\textcjheb{mkyl'}} ALJKM $|$zu euch\\
2.&22.&1484.&86.&5834.&6.&5&361&1\_10\_300\_10\_40 \textcolor{red}{\textcjheb{my+sy'}} AJSJM $|$(ihr) M"anner\\
3.&23.&1485.&91.&5839.&11.&4&302&1\_100\_200\_1 \textcolor{red}{\textcjheb{'rq'}} AQRA $|$ich rufe\\
4.&24.&1486.&95.&5843.&15.&5&152&6\_100\_6\_30\_10 \textcolor{red}{\textcjheb{ylwqw}} WQWLJ $|$und meine Stimme\\
5.&25.&1487.&100.&5848.&20.&2&31&1\_30 \textcolor{red}{\textcjheb{l'}} AL $|$ergeht an/zu\\
6.&26.&1488.&102.&5850.&22.&3&62&2\_50\_10 \textcolor{red}{\textcjheb{ynb}} BNJ $|$die Kinder/den S"ohnen\\
7.&27.&1489.&105.&5853.&25.&3&45&1\_4\_40 \textcolor{red}{\textcjheb{md'}} ADM $|$(der) Menschen\\
\end{tabular}\medskip \\
Ende des Verses 8.4\\
Verse: 206, Buchstaben: 27, 107, 5855, Totalwerte: 1054, 8650, 421282\\
\\
Zu euch, ihr M"anner, rufe ich, und meine Stimme ergeht an die Menschenkinder.\\
\newpage 
{\bf -- 8.5}\\
\medskip \\
\begin{tabular}{rrrrrrrrp{120mm}}
WV&WK&WB&ABK&ABB&ABV&AnzB&TW&Zahlencode \textcolor{red}{$\boldsymbol{Grundtext}$} Umschrift $|$"Ubersetzung(en)\\
1.&28.&1490.&108.&5856.&1.&5&73&5\_2\_10\_50\_6 \textcolor{red}{\textcjheb{wnybh}} HBJNW $|$lernt/begreift\\
2.&29.&1491.&113.&5861.&6.&5&531&80\_400\_1\_10\_40 \textcolor{red}{\textcjheb{my'tp}} PTAJM $|$(ihr) Einf"altige(n)\\
3.&30.&1492.&118.&5866.&11.&4&315&70\_200\_40\_5 \textcolor{red}{\textcjheb{hmr`}} aRMH $|$Klugheit\\
4.&31.&1493.&122.&5870.&15.&7&176&6\_20\_60\_10\_30\_10\_40 \textcolor{red}{\textcjheb{mylyskw}} WKsJLJM $|$und ihr Toren/und ihr Dummen\\
5.&32.&1494.&129.&5877.&22.&5&73&5\_2\_10\_50\_6 \textcolor{red}{\textcjheb{wnybh}} HBJNW $|$lernt\\
6.&33.&1495.&134.&5882.&27.&2&32&30\_2 \textcolor{red}{\textcjheb{bl}} LB $|$Verstand/Herz\\
\end{tabular}\medskip \\
Ende des Verses 8.5\\
Verse: 207, Buchstaben: 28, 135, 5883, Totalwerte: 1200, 9850, 422482\\
\\
Lernet Klugheit, ihr Einf"altigen, und ihr Toren, lernet Verstand!\\
\newpage 
{\bf -- 8.6}\\
\medskip \\
\begin{tabular}{rrrrrrrrp{120mm}}
WV&WK&WB&ABK&ABB&ABV&AnzB&TW&Zahlencode \textcolor{red}{$\boldsymbol{Grundtext}$} Umschrift $|$"Ubersetzung(en)\\
1.&34.&1496.&136.&5884.&1.&4&416&300\_40\_70\_6 \textcolor{red}{\textcjheb{w`m+s}} SMaW $|$h"oret\\
2.&35.&1497.&140.&5888.&5.&2&30&20\_10 \textcolor{red}{\textcjheb{yk}} KJ $|$denn\\
3.&36.&1498.&142.&5890.&7.&6&117&50\_3\_10\_4\_10\_40 \textcolor{red}{\textcjheb{mydygn}} NGJDJM $|$Vortreffliches/edle Dinge\\
4.&37.&1499.&148.&5896.&13.&4&207&1\_4\_2\_200 \textcolor{red}{\textcjheb{rbd'}} ADBR $|$ich will reden\\
5.&38.&1500.&152.&5900.&17.&5&534&6\_40\_80\_400\_8 \textcolor{red}{\textcjheb{.htpmw}} WMPTC $|$und das Auftun\\
6.&39.&1501.&157.&5905.&22.&4&790&300\_80\_400\_10 \textcolor{red}{\textcjheb{ytp+s}} SPTJ $|$meiner Lippen\\
7.&40.&1502.&161.&5909.&26.&6&600&40\_10\_300\_200\_10\_40 \textcolor{red}{\textcjheb{myr+sym}} MJSRJM $|$(soll sein) Geradheit(en)\\
\end{tabular}\medskip \\
Ende des Verses 8.6\\
Verse: 208, Buchstaben: 31, 166, 5914, Totalwerte: 2694, 12544, 425176\\
\\
H"oret! Denn Vortreffliches will ich reden, und das Auftun meiner Lippen soll Geradheit sein.\\
\newpage 
{\bf -- 8.7}\\
\medskip \\
\begin{tabular}{rrrrrrrrp{120mm}}
WV&WK&WB&ABK&ABB&ABV&AnzB&TW&Zahlencode \textcolor{red}{$\boldsymbol{Grundtext}$} Umschrift $|$"Ubersetzung(en)\\
1.&41.&1503.&167.&5915.&1.&2&30&20\_10 \textcolor{red}{\textcjheb{yk}} KJ $|$denn\\
2.&42.&1504.&169.&5917.&3.&3&441&1\_40\_400 \textcolor{red}{\textcjheb{tm'}} AMT $|$Wahrheit\\
3.&43.&1505.&172.&5920.&6.&4&23&10\_5\_3\_5 \textcolor{red}{\textcjheb{hghy}} JHGH $|$spricht aus/er (=es) murmelt\\
4.&44.&1506.&176.&5924.&10.&3&38&8\_20\_10 \textcolor{red}{\textcjheb{yk.h}} CKJ $|$mein Gaumen\\
5.&45.&1507.&179.&5927.&13.&6&884&6\_400\_6\_70\_2\_400 \textcolor{red}{\textcjheb{tb`wtw}} WTWaBT $|$und ein Gr"auel/und Abscheu\\
6.&46.&1508.&185.&5933.&19.&4&790&300\_80\_400\_10 \textcolor{red}{\textcjheb{ytp+s}} SPTJ $|$ist meinen Lippen/(f"ur) meine Lippen\\
7.&47.&1509.&189.&5937.&23.&3&570&200\_300\_70 \textcolor{red}{\textcjheb{`+sr}} RSa $|$Gesetzlosigkeit/(ist) Frevel\\
\end{tabular}\medskip \\
Ende des Verses 8.7\\
Verse: 209, Buchstaben: 25, 191, 5939, Totalwerte: 2776, 15320, 427952\\
\\
Denn mein Gaumen spricht Wahrheit aus, und Gesetzlosigkeit ist meinen Lippen ein Greuel.\\
\newpage 
{\bf -- 8.8}\\
\medskip \\
\begin{tabular}{rrrrrrrrp{120mm}}
WV&WK&WB&ABK&ABB&ABV&AnzB&TW&Zahlencode \textcolor{red}{$\boldsymbol{Grundtext}$} Umschrift $|$"Ubersetzung(en)\\
1.&48.&1510.&192.&5940.&1.&4&196&2\_90\_4\_100 \textcolor{red}{\textcjheb{qd.sb}} B"sDQ $|$in Gerechtigkeit/in Geradheit\\
2.&49.&1511.&196.&5944.&5.&2&50&20\_30 \textcolor{red}{\textcjheb{lk}} KL $|$(sind) alle\\
3.&50.&1512.&198.&5946.&7.&4&251&1\_40\_200\_10 \textcolor{red}{\textcjheb{yrm'}} AMRJ $|$Worte\\
4.&51.&1513.&202.&5950.&11.&2&90&80\_10 \textcolor{red}{\textcjheb{yp}} PJ $|$meines Mundes\\
5.&52.&1514.&204.&5952.&13.&3&61&1\_10\_50 \textcolor{red}{\textcjheb{ny'}} AJN $|$nichts ist (es)\\
6.&53.&1515.&207.&5955.&16.&3&47&2\_5\_40 \textcolor{red}{\textcjheb{mhb}} BHM $|$in ihnen\\
7.&54.&1516.&210.&5958.&19.&4&560&50\_80\_400\_30 \textcolor{red}{\textcjheb{ltpn}} NPTL $|$Verdrehtes/verschlungen\\
8.&55.&1517.&214.&5962.&23.&4&476&6\_70\_100\_300 \textcolor{red}{\textcjheb{+sq`w}} WaQS $|$und Verkehrtes/und verkehrt\\
\end{tabular}\medskip \\
Ende des Verses 8.8\\
Verse: 210, Buchstaben: 26, 217, 5965, Totalwerte: 1731, 17051, 429683\\
\\
Alle Worte meines Mundes sind in Gerechtigkeit; es ist nichts Verdrehtes und Verkehrtes in ihnen.\\
\newpage 
{\bf -- 8.9}\\
\medskip \\
\begin{tabular}{rrrrrrrrp{120mm}}
WV&WK&WB&ABK&ABB&ABV&AnzB&TW&Zahlencode \textcolor{red}{$\boldsymbol{Grundtext}$} Umschrift $|$"Ubersetzung(en)\\
1.&56.&1518.&218.&5966.&1.&3&90&20\_30\_40 \textcolor{red}{\textcjheb{mlk}} KLM $|$alle sie\\
2.&57.&1519.&221.&5969.&4.&5&128&50\_20\_8\_10\_40 \textcolor{red}{\textcjheb{my.hkn}} NKCJM $|$sind richtig/sind gerade\\
3.&58.&1520.&226.&5974.&9.&5&132&30\_40\_2\_10\_50 \textcolor{red}{\textcjheb{nybml}} LMBJN $|$dem Verst"andigen/dem Einsichtigen\\
4.&59.&1521.&231.&5979.&14.&6&566&6\_10\_300\_200\_10\_40 \textcolor{red}{\textcjheb{myr+syw}} WJSRJM $|$und gerade denen/und Geradheiten\\
5.&60.&1522.&237.&5985.&20.&5&171&30\_40\_90\_1\_10 \textcolor{red}{\textcjheb{y'.sml}} LM"sAJ $|$die erlangt haben/den Findenden\\
6.&61.&1523.&242.&5990.&25.&3&474&4\_70\_400 \textcolor{red}{\textcjheb{t`d}} DaT $|$Erkenntnis\\
\end{tabular}\medskip \\
Ende des Verses 8.9\\
Verse: 211, Buchstaben: 27, 244, 5992, Totalwerte: 1561, 18612, 431244\\
\\
Sie alle sind richtig dem Verst"andigen, und gerade denen, die Erkenntnis erlangt haben.\\
\newpage 
{\bf -- 8.10}\\
\medskip \\
\begin{tabular}{rrrrrrrrp{120mm}}
WV&WK&WB&ABK&ABB&ABV&AnzB&TW&Zahlencode \textcolor{red}{$\boldsymbol{Grundtext}$} Umschrift $|$"Ubersetzung(en)\\
1.&62.&1524.&245.&5993.&1.&3&114&100\_8\_6 \textcolor{red}{\textcjheb{w.hq}} QCW $|$nehmt an\\
2.&63.&1525.&248.&5996.&4.&5&316&40\_6\_60\_200\_10 \textcolor{red}{\textcjheb{yrswm}} MWsRJ $|$meine Unterweisung/meine Zucht\\
3.&64.&1526.&253.&6001.&9.&3&37&6\_1\_30 \textcolor{red}{\textcjheb{l'w}} WAL $|$und nicht\\
4.&65.&1527.&256.&6004.&12.&3&160&20\_60\_80 \textcolor{red}{\textcjheb{psk}} KsP $|$Silber\\
5.&66.&1528.&259.&6007.&15.&4&480&6\_4\_70\_400 \textcolor{red}{\textcjheb{t`dw}} WDaT $|$und (Er)Kenntnis\\
6.&67.&1529.&263.&6011.&19.&5&344&40\_8\_200\_6\_90 \textcolor{red}{\textcjheb{.swr.hm}} MCRW"s $|$lieber als Gold/mehr als Gold\\
7.&68.&1530.&268.&6016.&24.&4&260&50\_2\_8\_200 \textcolor{red}{\textcjheb{r.hbn}} NBCR $|$(aus)erlesenes (feines)\\
\end{tabular}\medskip \\
Ende des Verses 8.10\\
Verse: 212, Buchstaben: 27, 271, 6019, Totalwerte: 1711, 20323, 432955\\
\\
Nehmet an meine Unterweisung, und nicht Silber, und Erkenntnis lieber als auserlesenes, feines Gold.\\
\newpage 
{\bf -- 8.11}\\
\medskip \\
\begin{tabular}{rrrrrrrrp{120mm}}
WV&WK&WB&ABK&ABB&ABV&AnzB&TW&Zahlencode \textcolor{red}{$\boldsymbol{Grundtext}$} Umschrift $|$"Ubersetzung(en)\\
1.&69.&1531.&272.&6020.&1.&2&30&20\_10 \textcolor{red}{\textcjheb{yk}} KJ $|$denn\\
2.&70.&1532.&274.&6022.&3.&4&22&9\_6\_2\_5 \textcolor{red}{\textcjheb{hbw.t}} tWBH $|$besser ist/gut (ist)\\
3.&71.&1533.&278.&6026.&7.&4&73&8\_20\_40\_5 \textcolor{red}{\textcjheb{hmk.h}} CKMH $|$Weisheit\\
4.&72.&1534.&282.&6030.&11.&7&280&40\_80\_50\_10\_50\_10\_40 \textcolor{red}{\textcjheb{mynynpm}} MPNJNJM $|$(mehr) als Korallen\\
5.&73.&1535.&289.&6037.&18.&3&56&6\_20\_30 \textcolor{red}{\textcjheb{lkw}} WKL $|$und alle(s)\\
6.&74.&1536.&292.&6040.&21.&5&228&8\_80\_90\_10\_40 \textcolor{red}{\textcjheb{my.sp.h}} CP"sJM $|$was man begehren mag/begehrten Dinge\\
7.&75.&1537.&297.&6045.&26.&2&31&30\_1 \textcolor{red}{\textcjheb{'l}} LA $|$nicht\\
8.&76.&1538.&299.&6047.&28.&4&322&10\_300\_6\_6 \textcolor{red}{\textcjheb{ww+sy}} JSWW $|$kommt gleich/sie sind gleich\\
9.&77.&1539.&303.&6051.&32.&2&7&2\_5 \textcolor{red}{\textcjheb{hb}} BH $|$ihr/wie sie\\
\end{tabular}\medskip \\
Ende des Verses 8.11\\
Verse: 213, Buchstaben: 33, 304, 6052, Totalwerte: 1049, 21372, 434004\\
\\
Denn Weisheit ist besser als Korallen, und alles, was man begehren mag, kommt ihr nicht gleich. -\\
\newpage 
{\bf -- 8.12}\\
\medskip \\
\begin{tabular}{rrrrrrrrp{120mm}}
WV&WK&WB&ABK&ABB&ABV&AnzB&TW&Zahlencode \textcolor{red}{$\boldsymbol{Grundtext}$} Umschrift $|$"Ubersetzung(en)\\
1.&78.&1540.&305.&6053.&1.&3&61&1\_50\_10 \textcolor{red}{\textcjheb{yn'}} ANJ $|$ich\\
2.&79.&1541.&308.&6056.&4.&4&73&8\_20\_40\_5 \textcolor{red}{\textcjheb{hmk.h}} CKMH $|$(die) Weisheit\\
3.&80.&1542.&312.&6060.&8.&5&780&300\_20\_50\_400\_10 \textcolor{red}{\textcjheb{ytnk+s}} SKNTJ $|$bewohne/ich bin vertraut mit\\
4.&81.&1543.&317.&6065.&13.&4&315&70\_200\_40\_5 \textcolor{red}{\textcjheb{hmr`}} aRMH $|$die Klugheit/(der) Klugheit\\
5.&82.&1544.&321.&6069.&17.&4&480&6\_4\_70\_400 \textcolor{red}{\textcjheb{t`dw}} WDaT $|$und (die) (Er)Kenntnis\\
6.&83.&1545.&325.&6073.&21.&5&493&40\_7\_40\_6\_400 \textcolor{red}{\textcjheb{twmzm}} MZMWT $|$(der) Besonnenheit/(von) (sinnreichen) Gedanken\\
7.&84.&1546.&330.&6078.&26.&4&132&1\_40\_90\_1 \textcolor{red}{\textcjheb{'.sm'}} AM"sA $|$(ich) finde\\
\end{tabular}\medskip \\
Ende des Verses 8.12\\
Verse: 214, Buchstaben: 29, 333, 6081, Totalwerte: 2334, 23706, 436338\\
\\
Ich, Weisheit, bewohne die Klugheit, und finde die Erkenntnis der Besonnenheit.\\
\newpage 
{\bf -- 8.13}\\
\medskip \\
\begin{tabular}{rrrrrrrrp{120mm}}
WV&WK&WB&ABK&ABB&ABV&AnzB&TW&Zahlencode \textcolor{red}{$\boldsymbol{Grundtext}$} Umschrift $|$"Ubersetzung(en)\\
1.&85.&1547.&334.&6082.&1.&4&611&10\_200\_1\_400 \textcolor{red}{\textcjheb{t'ry}} JRAT $|$(die) Furcht\\
2.&86.&1548.&338.&6086.&5.&4&26&10\_5\_6\_5 \textcolor{red}{\textcjheb{hwhy}} JHWH $|$(vor) Jahwe(s)\\
3.&87.&1549.&342.&6090.&9.&4&751&300\_50\_1\_400 \textcolor{red}{\textcjheb{t'n+s}} SNAT $|$(ist) (ein) Hassen\\
4.&88.&1550.&346.&6094.&13.&2&270&200\_70 \textcolor{red}{\textcjheb{`r}} Ra $|$das B"ose/B"oses\\
5.&89.&1551.&348.&6096.&15.&3&9&3\_1\_5 \textcolor{red}{\textcjheb{h'g}} GAH $|$Hoffart/Hochmut\\
6.&90.&1552.&351.&6099.&18.&5&66&6\_3\_1\_6\_50 \textcolor{red}{\textcjheb{nw'gw}} WGAWN $|$und Hochmut/und Stolz\\
7.&91.&1553.&356.&6104.&23.&4&230&6\_4\_200\_20 \textcolor{red}{\textcjheb{krdw}} WDRK $|$und den Weg/und einen Weg\\
8.&92.&1554.&360.&6108.&27.&2&270&200\_70 \textcolor{red}{\textcjheb{`r}} Ra $|$des B"osen/b"osen\\
9.&93.&1555.&362.&6110.&29.&3&96&6\_80\_10 \textcolor{red}{\textcjheb{ypw}} WPJ $|$und den Mund/und einen Mund\\
10.&94.&1556.&365.&6113.&32.&6&911&400\_5\_80\_20\_6\_400 \textcolor{red}{\textcjheb{twkpht}} THPKWT $|$der Verkehrtheit/der R"anke\\
11.&95.&1557.&371.&6119.&38.&5&761&300\_50\_1\_400\_10 \textcolor{red}{\textcjheb{yt'n+s}} SNATJ $|$ich hass(t)e\\
\end{tabular}\medskip \\
Ende des Verses 8.13\\
Verse: 215, Buchstaben: 42, 375, 6123, Totalwerte: 4001, 27707, 440339\\
\\
Die Furcht Jahwes ist: das B"ose hassen. Hoffart und Hochmut und den Weg des B"osen und den Mund der Verkehrtheit hasse ich.\\
\newpage 
{\bf -- 8.14}\\
\medskip \\
\begin{tabular}{rrrrrrrrp{120mm}}
WV&WK&WB&ABK&ABB&ABV&AnzB&TW&Zahlencode \textcolor{red}{$\boldsymbol{Grundtext}$} Umschrift $|$"Ubersetzung(en)\\
1.&96.&1558.&376.&6124.&1.&2&40&30\_10 \textcolor{red}{\textcjheb{yl}} LJ $|$mein sind/(bei) mir\\
2.&97.&1559.&378.&6126.&3.&3&165&70\_90\_5 \textcolor{red}{\textcjheb{h.s`}} a"sH $|$(ist) Rat\\
3.&98.&1560.&381.&6129.&6.&6&727&6\_400\_6\_300\_10\_5 \textcolor{red}{\textcjheb{hy+swtw}} WTWSJH $|$und Einsicht/und Gelingen\\
4.&99.&1561.&387.&6135.&12.&3&61&1\_50\_10 \textcolor{red}{\textcjheb{yn'}} ANJ $|$ich\\
5.&100.&1562.&390.&6138.&15.&4&67&2\_10\_50\_5 \textcolor{red}{\textcjheb{hnyb}} BJNH $|$bin der Verstand/Einsicht\\
6.&101.&1563.&394.&6142.&19.&2&40&30\_10 \textcolor{red}{\textcjheb{yl}} LJ $|$mein ist/ich habe\\
7.&102.&1564.&396.&6144.&21.&5&216&3\_2\_6\_200\_5 \textcolor{red}{\textcjheb{hrwbg}} GBWRH $|$(die) St"arke\\
\end{tabular}\medskip \\
Ende des Verses 8.14\\
Verse: 216, Buchstaben: 25, 400, 6148, Totalwerte: 1316, 29023, 441655\\
\\
Mein sind Rat und Einsicht; ich bin der Verstand, mein ist die St"arke.\\
\newpage 
{\bf -- 8.15}\\
\medskip \\
\begin{tabular}{rrrrrrrrp{120mm}}
WV&WK&WB&ABK&ABB&ABV&AnzB&TW&Zahlencode \textcolor{red}{$\boldsymbol{Grundtext}$} Umschrift $|$"Ubersetzung(en)\\
1.&103.&1565.&401.&6149.&1.&2&12&2\_10 \textcolor{red}{\textcjheb{yb}} BJ $|$durch mich\\
2.&104.&1566.&403.&6151.&3.&5&140&40\_30\_20\_10\_40 \textcolor{red}{\textcjheb{myklm}} MLKJM $|$K"onige\\
3.&105.&1567.&408.&6156.&8.&5&106&10\_40\_30\_20\_6 \textcolor{red}{\textcjheb{wklmy}} JMLKW $|$(sie) regieren\\
4.&106.&1568.&413.&6161.&13.&7&319&6\_200\_6\_7\_50\_10\_40 \textcolor{red}{\textcjheb{mynzwrw}} WRWZNJM $|$und F"ursten/und W"urdentr"ager\\
5.&107.&1569.&420.&6168.&20.&5&224&10\_8\_100\_100\_6 \textcolor{red}{\textcjheb{wqq.hy}} JCQQW $|$treffen Entscheidungen/(sie) ordnen an\\
6.&108.&1570.&425.&6173.&25.&3&194&90\_4\_100 \textcolor{red}{\textcjheb{qd.s}} "sDQ $|$gerechte/Rechtes\\
\end{tabular}\medskip \\
Ende des Verses 8.15\\
Verse: 217, Buchstaben: 27, 427, 6175, Totalwerte: 995, 30018, 442650\\
\\
Durch mich regieren K"onige, und F"ursten treffen gerechte Entscheidungen;\\
\newpage 
{\bf -- 8.16}\\
\medskip \\
\begin{tabular}{rrrrrrrrp{120mm}}
WV&WK&WB&ABK&ABB&ABV&AnzB&TW&Zahlencode \textcolor{red}{$\boldsymbol{Grundtext}$} Umschrift $|$"Ubersetzung(en)\\
1.&109.&1571.&428.&6176.&1.&2&12&2\_10 \textcolor{red}{\textcjheb{yb}} BJ $|$durch mich\\
2.&110.&1572.&430.&6178.&3.&4&550&300\_200\_10\_40 \textcolor{red}{\textcjheb{myr+s}} SRJM $|$Herrscher/F"ursten\\
3.&111.&1573.&434.&6182.&7.&4&516&10\_300\_200\_6 \textcolor{red}{\textcjheb{wr+sy}} JSRW $|$herrschen/(sie) regieren\\
4.&112.&1574.&438.&6186.&11.&7&122&6\_50\_4\_10\_2\_10\_40 \textcolor{red}{\textcjheb{mybydnw}} WNDJBJM $|$und Edle/und Machthaber\\
5.&113.&1575.&445.&6193.&18.&2&50&20\_30 \textcolor{red}{\textcjheb{lk}} KL $|$(sind) alle\\
6.&114.&1576.&447.&6195.&20.&4&399&300\_80\_9\_10 \textcolor{red}{\textcjheb{y.tp+s}} SPtJ $|$Richter/Rechtsprechende\\
7.&115.&1577.&451.&6199.&24.&3&194&90\_4\_100 \textcolor{red}{\textcjheb{qd.s}} "sDQ $|$der Erde/Gerechtigkeit\\
\end{tabular}\medskip \\
Ende des Verses 8.16\\
Verse: 218, Buchstaben: 26, 453, 6201, Totalwerte: 1843, 31861, 444493\\
\\
durch mich herrschen Herrscher und Edle, alle Richter der Erde.\\
\newpage 
{\bf -- 8.17}\\
\medskip \\
\begin{tabular}{rrrrrrrrp{120mm}}
WV&WK&WB&ABK&ABB&ABV&AnzB&TW&Zahlencode \textcolor{red}{$\boldsymbol{Grundtext}$} Umschrift $|$"Ubersetzung(en)\\
1.&116.&1578.&454.&6202.&1.&3&61&1\_50\_10 \textcolor{red}{\textcjheb{yn'}} ANJ $|$ich\\
2.&117.&1579.&457.&6205.&4.&5&23&1\_5\_2\_10\_5 \textcolor{red}{\textcjheb{hybh'}} AHBJH $|$die mich lieben\\
3.&118.&1580.&462.&6210.&9.&3&8&1\_5\_2 \textcolor{red}{\textcjheb{bh'}} AHB $|$(ich) liebe\\
4.&119.&1581.&465.&6213.&12.&6&564&6\_40\_300\_8\_200\_10 \textcolor{red}{\textcjheb{yr.h+smw}} WMSCRJ $|$und die mich fr"uh suchen/und meine Suchenden\\
5.&120.&1582.&471.&6219.&18.&7&251&10\_40\_90\_1\_50\_50\_10 \textcolor{red}{\textcjheb{ynn'.smy}} JM"sANNJ $|$(sie) werden finden mich\\
\end{tabular}\medskip \\
Ende des Verses 8.17\\
Verse: 219, Buchstaben: 24, 477, 6225, Totalwerte: 907, 32768, 445400\\
\\
Ich liebe, die mich lieben; und die mich fr"uh suchen, werden mich finden.\\
\newpage 
{\bf -- 8.18}\\
\medskip \\
\begin{tabular}{rrrrrrrrp{120mm}}
WV&WK&WB&ABK&ABB&ABV&AnzB&TW&Zahlencode \textcolor{red}{$\boldsymbol{Grundtext}$} Umschrift $|$"Ubersetzung(en)\\
1.&121.&1583.&478.&6226.&1.&3&570&70\_300\_200 \textcolor{red}{\textcjheb{r+s`}} aSR $|$Reichtum\\
2.&122.&1584.&481.&6229.&4.&5&38&6\_20\_2\_6\_4 \textcolor{red}{\textcjheb{dwbkw}} WKBWD $|$und Ehre\\
3.&123.&1585.&486.&6234.&9.&3&411&1\_400\_10 \textcolor{red}{\textcjheb{yt'}} ATJ $|$(sind) bei mir\\
4.&124.&1586.&489.&6237.&12.&3&61&5\_6\_50 \textcolor{red}{\textcjheb{nwh}} HWN $|$Gut\\
5.&125.&1587.&492.&6240.&15.&3&570&70\_400\_100 \textcolor{red}{\textcjheb{qt`}} aTQ $|$bleibendes/stattliches\\
6.&126.&1588.&495.&6243.&18.&5&205&6\_90\_4\_100\_5 \textcolor{red}{\textcjheb{hqd.sw}} W"sDQH $|$und Gerechtigkeit/und Recht\\
\end{tabular}\medskip \\
Ende des Verses 8.18\\
Verse: 220, Buchstaben: 22, 499, 6247, Totalwerte: 1855, 34623, 447255\\
\\
Reichtum und Ehre sind bei mir, bleibendes Gut und Gerechtigkeit.\\
\newpage 
{\bf -- 8.19}\\
\medskip \\
\begin{tabular}{rrrrrrrrp{120mm}}
WV&WK&WB&ABK&ABB&ABV&AnzB&TW&Zahlencode \textcolor{red}{$\boldsymbol{Grundtext}$} Umschrift $|$"Ubersetzung(en)\\
1.&127.&1589.&500.&6248.&1.&3&17&9\_6\_2 \textcolor{red}{\textcjheb{bw.t}} tWB $|$besser ist/gut (ist)\\
2.&128.&1590.&503.&6251.&4.&4&300&80\_200\_10\_10 \textcolor{red}{\textcjheb{yyrp}} PRJJ $|$meine Frucht\\
3.&129.&1591.&507.&6255.&8.&5&344&40\_8\_200\_6\_90 \textcolor{red}{\textcjheb{.swr.hm}} MCRW"s $|$(mehr) als (feines) Gold\\
4.&130.&1592.&512.&6260.&13.&4&133&6\_40\_80\_7 \textcolor{red}{\textcjheb{zpmw}} WMPZ $|$und (mehr als) gediegenes Gold\\
5.&131.&1593.&516.&6264.&17.&7&825&6\_400\_2\_6\_1\_400\_10 \textcolor{red}{\textcjheb{yt'wbtw}} WTBWATJ $|$und mein Ertrag\\
6.&132.&1594.&523.&6271.&24.&4&200&40\_20\_60\_80 \textcolor{red}{\textcjheb{pskm}} MKsP $|$(mehr) als Silber\\
7.&133.&1595.&527.&6275.&28.&4&260&50\_2\_8\_200 \textcolor{red}{\textcjheb{r.hbn}} NBCR $|$(aus)erlesenes\\
\end{tabular}\medskip \\
Ende des Verses 8.19\\
Verse: 221, Buchstaben: 31, 530, 6278, Totalwerte: 2079, 36702, 449334\\
\\
Meine Frucht ist besser als feines Gold und gediegenes Gold, und mein Ertrag als auserlesenes Silber.\\
\newpage 
{\bf -- 8.20}\\
\medskip \\
\begin{tabular}{rrrrrrrrp{120mm}}
WV&WK&WB&ABK&ABB&ABV&AnzB&TW&Zahlencode \textcolor{red}{$\boldsymbol{Grundtext}$} Umschrift $|$"Ubersetzung(en)\\
1.&134.&1596.&531.&6279.&1.&4&211&2\_1\_200\_8 \textcolor{red}{\textcjheb{.hr'b}} BARC $|$auf dem Pfad\\
2.&135.&1597.&535.&6283.&5.&4&199&90\_4\_100\_5 \textcolor{red}{\textcjheb{hqd.s}} "sDQH $|$der Gerechtigkeit\\
3.&136.&1598.&539.&6287.&9.&4&56&1\_5\_30\_20 \textcolor{red}{\textcjheb{klh'}} AHLK $|$ich wandle/ich gehe\\
4.&137.&1599.&543.&6291.&13.&4&428&2\_400\_6\_20 \textcolor{red}{\textcjheb{kwtb}} BTWK $|$mitten auf/inmitten\\
5.&138.&1600.&547.&6295.&17.&6&868&50\_400\_10\_2\_6\_400 \textcolor{red}{\textcjheb{twbytn}} NTJBWT $|$den Steigen/(der) Pfade\\
6.&139.&1601.&553.&6301.&23.&4&429&40\_300\_80\_9 \textcolor{red}{\textcjheb{.tp+sm}} MSPt $|$des Rechts\\
\end{tabular}\medskip \\
Ende des Verses 8.20\\
Verse: 222, Buchstaben: 26, 556, 6304, Totalwerte: 2191, 38893, 451525\\
\\
Ich wandle auf dem Pfade der Gerechtigkeit, mitten auf den Steigen des Rechts;\\
\newpage 
{\bf -- 8.21}\\
\medskip \\
\begin{tabular}{rrrrrrrrp{120mm}}
WV&WK&WB&ABK&ABB&ABV&AnzB&TW&Zahlencode \textcolor{red}{$\boldsymbol{Grundtext}$} Umschrift $|$"Ubersetzung(en)\\
1.&140.&1602.&557.&6305.&1.&6&133&30\_5\_50\_8\_10\_30 \textcolor{red}{\textcjheb{ly.hnhl}} LHNCJL $|$um die erben zu lassen/um zu vererben\\
2.&141.&1603.&563.&6311.&7.&4&18&1\_5\_2\_10 \textcolor{red}{\textcjheb{ybh'}} AHBJ $|$die mich lieben/meinen Liebenden\\
3.&142.&1604.&567.&6315.&11.&2&310&10\_300 \textcolor{red}{\textcjheb{+sy}} JS $|$best"andiges Gut/was ist vorhanden\\
4.&143.&1605.&569.&6317.&13.&8&752&6\_1\_90\_200\_400\_10\_5\_40 \textcolor{red}{\textcjheb{mhytr.s'w}} WA"sRTJHM $|$und um ihre Vorratskammern/und ihre Speicher\\
5.&144.&1606.&577.&6325.&21.&4&72&1\_40\_30\_1 \textcolor{red}{\textcjheb{'lm'}} AMLA $|$zu f"ullen/ich f"ulle\\
\end{tabular}\medskip \\
Ende des Verses 8.21\\
Verse: 223, Buchstaben: 24, 580, 6328, Totalwerte: 1285, 40178, 452810\\
\\
um die, die mich lieben, best"andiges Gut erben zu lassen, und um ihre Vorratskammern zu f"ullen.\\
\newpage 
{\bf -- 8.22}\\
\medskip \\
\begin{tabular}{rrrrrrrrp{120mm}}
WV&WK&WB&ABK&ABB&ABV&AnzB&TW&Zahlencode \textcolor{red}{$\boldsymbol{Grundtext}$} Umschrift $|$"Ubersetzung(en)\\
1.&145.&1607.&581.&6329.&1.&4&26&10\_5\_6\_5 \textcolor{red}{\textcjheb{hwhy}} JHWH $|$Jahwe\\
2.&146.&1608.&585.&6333.&5.&4&210&100\_50\_50\_10 \textcolor{red}{\textcjheb{ynnq}} QNNJ $|$besa"s mich/schuf mich\\
3.&147.&1609.&589.&6337.&9.&5&911&200\_1\_300\_10\_400 \textcolor{red}{\textcjheb{ty+s'r}} RASJT $|$im Anfang/(als) Anfang\\
4.&148.&1610.&594.&6342.&14.&4&230&4\_200\_20\_6 \textcolor{red}{\textcjheb{wkrd}} DRKW $|$seines Weges\\
5.&149.&1611.&598.&6346.&18.&3&144&100\_4\_40 \textcolor{red}{\textcjheb{mdq}} QDM $|$vor/fr"uher\\
6.&150.&1612.&601.&6349.&21.&6&236&40\_80\_70\_30\_10\_6 \textcolor{red}{\textcjheb{wyl`pm}} MPaLJW $|$seinen Werken/als seine Werke\\
7.&151.&1613.&607.&6355.&27.&3&48&40\_1\_7 \textcolor{red}{\textcjheb{z'm}} MAZ $|$von jeher/von damals\\
\end{tabular}\medskip \\
Ende des Verses 8.22\\
Verse: 224, Buchstaben: 29, 609, 6357, Totalwerte: 1805, 41983, 454615\\
\\
Jahwe besa"s mich im Anfang seines Weges, vor seinen Werken von jeher.\\
\newpage 
{\bf -- 8.23}\\
\medskip \\
\begin{tabular}{rrrrrrrrp{120mm}}
WV&WK&WB&ABK&ABB&ABV&AnzB&TW&Zahlencode \textcolor{red}{$\boldsymbol{Grundtext}$} Umschrift $|$"Ubersetzung(en)\\
1.&152.&1614.&610.&6358.&1.&5&186&40\_70\_6\_30\_40 \textcolor{red}{\textcjheb{mlw`m}} MaWLM $|$von Ewigkeit (her)\\
2.&153.&1615.&615.&6363.&6.&5&540&50\_60\_20\_400\_10 \textcolor{red}{\textcjheb{ytksn}} NsKTJ $|$ich war eingesetzt/ich wurde eingesetzt\\
3.&154.&1616.&620.&6368.&11.&4&541&40\_200\_1\_300 \textcolor{red}{\textcjheb{+s'rm}} MRAS $|$von Anbeginn\\
4.&155.&1617.&624.&6372.&15.&5&194&40\_100\_4\_40\_10 \textcolor{red}{\textcjheb{ymdqm}} MQDMJ $|$vor den Uranf"angen/vor den Urzeiten\\
5.&156.&1618.&629.&6377.&20.&3&291&1\_200\_90 \textcolor{red}{\textcjheb{.sr'}} AR"s $|$der Erde\\
\end{tabular}\medskip \\
Ende des Verses 8.23\\
Verse: 225, Buchstaben: 22, 631, 6379, Totalwerte: 1752, 43735, 456367\\
\\
Ich war eingesetzt von Ewigkeit her, von Anbeginn, vor den Uranf"angen der Erde.\\
\newpage 
{\bf -- 8.24}\\
\medskip \\
\begin{tabular}{rrrrrrrrp{120mm}}
WV&WK&WB&ABK&ABB&ABV&AnzB&TW&Zahlencode \textcolor{red}{$\boldsymbol{Grundtext}$} Umschrift $|$"Ubersetzung(en)\\
1.&157.&1619.&632.&6380.&1.&4&63&2\_1\_10\_50 \textcolor{red}{\textcjheb{ny'b}} BAJN $|$als (noch) nicht waren\\
2.&158.&1620.&636.&6384.&5.&5&851&400\_5\_40\_6\_400 \textcolor{red}{\textcjheb{twmht}} THMWT $|$die Tiefen/(die) (Ur)Fluten\\
3.&159.&1621.&641.&6389.&10.&6&484&8\_6\_30\_30\_400\_10 \textcolor{red}{\textcjheb{ytllw.h}} CWLLTJ $|$ich war geboren/ich wurde geboren\\
4.&160.&1622.&647.&6395.&16.&4&63&2\_1\_10\_50 \textcolor{red}{\textcjheb{ny'b}} BAJN $|$als (noch) nicht waren\\
5.&161.&1623.&651.&6399.&20.&6&576&40\_70\_10\_50\_6\_400 \textcolor{red}{\textcjheb{twny`m}} MaJNWT $|$Quellen\\
6.&162.&1624.&657.&6405.&26.&5&86&50\_20\_2\_4\_10 \textcolor{red}{\textcjheb{ydbkn}} NKBDJ $|$reich\\
7.&163.&1625.&662.&6410.&31.&3&90&40\_10\_40 \textcolor{red}{\textcjheb{mym}} MJM $|$an Wasser\\
\end{tabular}\medskip \\
Ende des Verses 8.24\\
Verse: 226, Buchstaben: 33, 664, 6412, Totalwerte: 2213, 45948, 458580\\
\\
Ich war geboren, als die Tiefen noch nicht waren, als noch keine Quellen waren, reich an Wasser.\\
\newpage 
{\bf -- 8.25}\\
\medskip \\
\begin{tabular}{rrrrrrrrp{120mm}}
WV&WK&WB&ABK&ABB&ABV&AnzB&TW&Zahlencode \textcolor{red}{$\boldsymbol{Grundtext}$} Umschrift $|$"Ubersetzung(en)\\
1.&164.&1626.&665.&6413.&1.&4&251&2\_9\_200\_40 \textcolor{red}{\textcjheb{mr.tb}} BtRM $|$ehe/bevor\\
2.&165.&1627.&669.&6417.&5.&4&255&5\_200\_10\_40 \textcolor{red}{\textcjheb{myrh}} HRJM $|$die Berge\\
3.&166.&1628.&673.&6421.&9.&5&92&5\_9\_2\_70\_6 \textcolor{red}{\textcjheb{w`b.th}} HtBaW $|$(sie) wurden eingesenkt\\
4.&167.&1629.&678.&6426.&14.&4&170&30\_80\_50\_10 \textcolor{red}{\textcjheb{ynpl}} LPNJ $|$vor\\
5.&168.&1630.&682.&6430.&18.&5&481&3\_2\_70\_6\_400 \textcolor{red}{\textcjheb{tw`bg}} GBaWT $|$den H"ugeln\\
6.&169.&1631.&687.&6435.&23.&6&484&8\_6\_30\_30\_400\_10 \textcolor{red}{\textcjheb{ytllw.h}} CWLLTJ $|$war ich geboren/ich wurde geboren\\
\end{tabular}\medskip \\
Ende des Verses 8.25\\
Verse: 227, Buchstaben: 28, 692, 6440, Totalwerte: 1733, 47681, 460313\\
\\
Ehe die Berge eingesenkt wurden, vor den H"ugeln war ich geboren;\\
\newpage 
{\bf -- 8.26}\\
\medskip \\
\begin{tabular}{rrrrrrrrp{120mm}}
WV&WK&WB&ABK&ABB&ABV&AnzB&TW&Zahlencode \textcolor{red}{$\boldsymbol{Grundtext}$} Umschrift $|$"Ubersetzung(en)\\
1.&170.&1632.&693.&6441.&1.&2&74&70\_4 \textcolor{red}{\textcjheb{d`}} aD $|$(als) noch\\
2.&171.&1633.&695.&6443.&3.&2&31&30\_1 \textcolor{red}{\textcjheb{'l}} LA $|$nicht\\
3.&172.&1634.&697.&6445.&5.&3&375&70\_300\_5 \textcolor{red}{\textcjheb{h+s`}} aSH $|$er hatte gemacht\\
4.&173.&1635.&700.&6448.&8.&3&291&1\_200\_90 \textcolor{red}{\textcjheb{.sr'}} AR"s $|$die Erde\\
5.&174.&1636.&703.&6451.&11.&6&516&6\_8\_6\_90\_6\_400 \textcolor{red}{\textcjheb{tw.sw.hw}} WCW"sWT $|$und (die) Fluren\\
6.&175.&1637.&709.&6457.&17.&4&507&6\_200\_1\_300 \textcolor{red}{\textcjheb{+s'rw}} WRAS $|$und den Beginn/und den Anfang\\
7.&176.&1638.&713.&6461.&21.&5&756&70\_80\_200\_6\_400 \textcolor{red}{\textcjheb{twrp`}} aPRWT $|$der Schollen/der Bestandteile\\
8.&177.&1639.&718.&6466.&26.&3&432&400\_2\_30 \textcolor{red}{\textcjheb{lbt}} TBL $|$des Erdkreises/des Erdenrunds\\
\end{tabular}\medskip \\
Ende des Verses 8.26\\
Verse: 228, Buchstaben: 28, 720, 6468, Totalwerte: 2982, 50663, 463295\\
\\
als er die Erde und die Fluren noch nicht gemacht hatte, und den Beginn der Schollen des Erdkreises.\\
\newpage 
{\bf -- 8.27}\\
\medskip \\
\begin{tabular}{rrrrrrrrp{120mm}}
WV&WK&WB&ABK&ABB&ABV&AnzB&TW&Zahlencode \textcolor{red}{$\boldsymbol{Grundtext}$} Umschrift $|$"Ubersetzung(en)\\
1.&178.&1640.&721.&6469.&1.&6&93&2\_5\_20\_10\_50\_6 \textcolor{red}{\textcjheb{wnykhb}} BHKJNW $|$als er feststellte/bei seinem Bereiten\\
2.&179.&1641.&727.&6475.&7.&4&390&300\_40\_10\_40 \textcolor{red}{\textcjheb{mym+s}} SMJM $|$die Himmel\\
3.&180.&1642.&731.&6479.&11.&2&340&300\_40 \textcolor{red}{\textcjheb{m+s}} SM $|$da/dabei\\
4.&181.&1643.&733.&6481.&13.&3&61&1\_50\_10 \textcolor{red}{\textcjheb{yn'}} ANJ $|$ich (war)\\
5.&182.&1644.&736.&6484.&16.&5&122&2\_8\_6\_100\_6 \textcolor{red}{\textcjheb{wqw.hb}} BCWQW $|$als er abma"s/bei seinem Festsetzen\\
6.&183.&1645.&741.&6489.&21.&3&17&8\_6\_3 \textcolor{red}{\textcjheb{gw.h}} CWG $|$einen Kreis/den Erdkreis\\
7.&184.&1646.&744.&6492.&24.&2&100&70\_30 \textcolor{red}{\textcjheb{l`}} aL $|$"uber\\
8.&185.&1647.&746.&6494.&26.&3&140&80\_50\_10 \textcolor{red}{\textcjheb{ynp}} PNJ $|$der Fl"ache\\
9.&186.&1648.&749.&6497.&29.&4&451&400\_5\_6\_40 \textcolor{red}{\textcjheb{mwht}} THWM $|$der Tiefe/der Urflut\\
\end{tabular}\medskip \\
Ende des Verses 8.27\\
Verse: 229, Buchstaben: 32, 752, 6500, Totalwerte: 1714, 52377, 465009\\
\\
Als er die Himmel feststellte, war ich da, als er einen Kreis abma"s "uber der Fl"ache der Tiefe;\\
\newpage 
{\bf -- 8.28}\\
\medskip \\
\begin{tabular}{rrrrrrrrp{120mm}}
WV&WK&WB&ABK&ABB&ABV&AnzB&TW&Zahlencode \textcolor{red}{$\boldsymbol{Grundtext}$} Umschrift $|$"Ubersetzung(en)\\
1.&187.&1649.&753.&6501.&1.&5&139&2\_1\_40\_90\_6 \textcolor{red}{\textcjheb{w.sm'b}} BAM"sW $|$als er befestigte/bei seinem Befestigen\\
2.&188.&1650.&758.&6506.&6.&5&458&300\_8\_100\_10\_40 \textcolor{red}{\textcjheb{myq.h+s}} SCQJM $|$die Wolken\\
3.&189.&1651.&763.&6511.&11.&4&180&40\_40\_70\_30 \textcolor{red}{\textcjheb{l`mm}} MMaL $|$droben\\
4.&190.&1652.&767.&6515.&15.&5&92&2\_70\_7\_6\_7 \textcolor{red}{\textcjheb{zwz`b}} BaZWZ $|$als er Festigkeit gab/in Starksein\\
5.&191.&1653.&772.&6520.&20.&5&536&70\_10\_50\_6\_400 \textcolor{red}{\textcjheb{twny`}} aJNWT $|$den Quellen/die Quellen\\
6.&192.&1654.&777.&6525.&25.&4&451&400\_5\_6\_40 \textcolor{red}{\textcjheb{mwht}} THWM $|$der (Ur)Tiefe\\
\end{tabular}\medskip \\
Ende des Verses 8.28\\
Verse: 230, Buchstaben: 28, 780, 6528, Totalwerte: 1856, 54233, 466865\\
\\
als er die Wolken droben befestigte, als er Festigkeit gab den Quellen der Tiefe;\\
\newpage 
{\bf -- 8.29}\\
\medskip \\
\begin{tabular}{rrrrrrrrp{120mm}}
WV&WK&WB&ABK&ABB&ABV&AnzB&TW&Zahlencode \textcolor{red}{$\boldsymbol{Grundtext}$} Umschrift $|$"Ubersetzung(en)\\
1.&193.&1655.&781.&6529.&1.&5&354&2\_300\_6\_40\_6 \textcolor{red}{\textcjheb{wmw+sb}} BSWMW $|$als er setzte/bei seinem Setzen\\
2.&194.&1656.&786.&6534.&6.&3&80&30\_10\_40 \textcolor{red}{\textcjheb{myl}} LJM $|$dem Meer\\
3.&195.&1657.&789.&6537.&9.&3&114&8\_100\_6 \textcolor{red}{\textcjheb{wq.h}} CQW $|$seine Schranken/seine Grenzen\\
4.&196.&1658.&792.&6540.&12.&4&96&6\_40\_10\_40 \textcolor{red}{\textcjheb{mymw}} WMJM $|$dass die Wasser/und die Wasser\\
5.&197.&1659.&796.&6544.&16.&2&31&30\_1 \textcolor{red}{\textcjheb{'l}} LA $|$nicht\\
6.&198.&1660.&798.&6546.&18.&5&288&10\_70\_2\_200\_6 \textcolor{red}{\textcjheb{wrb`y}} JaBRW $|$"uberschritten/(sie) "ubertreten\\
7.&199.&1661.&803.&6551.&23.&3&96&80\_10\_6 \textcolor{red}{\textcjheb{wyp}} PJW $|$seinen Befehl/seinen Rand\\
8.&200.&1662.&806.&6554.&26.&5&122&2\_8\_6\_100\_6 \textcolor{red}{\textcjheb{wqw.hb}} BCWQW $|$als er feststellte/bei seinem Festsetzen\\
9.&201.&1663.&811.&6559.&31.&5&120&40\_6\_60\_4\_10 \textcolor{red}{\textcjheb{ydswm}} MWsDJ $|$die Grundfesten/die Grundlagen\\
10.&202.&1664.&816.&6564.&36.&3&291&1\_200\_90 \textcolor{red}{\textcjheb{.sr'}} AR"s $|$der Erde\\
\end{tabular}\medskip \\
Ende des Verses 8.29\\
Verse: 231, Buchstaben: 38, 818, 6566, Totalwerte: 1592, 55825, 468457\\
\\
als er dem Meere seine Schranken setzte, da"s die Wasser seinen Befehl nicht "uberschritten, als er die Grundfesten der Erde feststellte:\\
\newpage 
{\bf -- 8.30}\\
\medskip \\
\begin{tabular}{rrrrrrrrp{120mm}}
WV&WK&WB&ABK&ABB&ABV&AnzB&TW&Zahlencode \textcolor{red}{$\boldsymbol{Grundtext}$} Umschrift $|$"Ubersetzung(en)\\
1.&203.&1665.&819.&6567.&1.&5&27&6\_1\_5\_10\_5 \textcolor{red}{\textcjheb{hyh'w}} WAHJH $|$da war ich/und ich war\\
2.&204.&1666.&824.&6572.&6.&4&127&1\_90\_30\_6 \textcolor{red}{\textcjheb{wl.s'}} A"sLW $|$bei ihm/an seiner Seite\\
3.&205.&1667.&828.&6576.&10.&4&97&1\_40\_6\_50 \textcolor{red}{\textcjheb{nwm'}} AMWN $|$Schosskind/als Pflegling\\
4.&206.&1668.&832.&6580.&14.&5&27&6\_1\_5\_10\_5 \textcolor{red}{\textcjheb{hyh'w}} WAHJH $|$und (ich) war\\
5.&207.&1669.&837.&6585.&19.&6&790&300\_70\_300\_70\_10\_40 \textcolor{red}{\textcjheb{my`+s`+s}} SaSaJM $|$(s)eine Wonne\\
6.&208.&1670.&843.&6591.&25.&3&56&10\_6\_40 \textcolor{red}{\textcjheb{mwy}} JWM $|$Tag\\
7.&209.&1671.&846.&6594.&28.&3&56&10\_6\_40 \textcolor{red}{\textcjheb{mwy}} JWM $|$f"ur Tag/(um) Tag\\
8.&210.&1672.&849.&6597.&31.&5&848&40\_300\_8\_100\_400 \textcolor{red}{\textcjheb{tq.h+sm}} MSCQT $|$mich erg"otzend/spielend\\
9.&211.&1673.&854.&6602.&36.&5&176&30\_80\_50\_10\_6 \textcolor{red}{\textcjheb{wynpl}} LPNJW $|$vor ihm\\
10.&212.&1674.&859.&6607.&41.&3&52&2\_20\_30 \textcolor{red}{\textcjheb{lkb}} BKL $|$alle/in all\\
11.&213.&1675.&862.&6610.&44.&2&470&70\_400 \textcolor{red}{\textcjheb{t`}} aT $|$(der) Zeit\\
\end{tabular}\medskip \\
Ende des Verses 8.30\\
Verse: 232, Buchstaben: 45, 863, 6611, Totalwerte: 2726, 58551, 471183\\
\\
da war ich Scho"skind bei ihm, und war Tag f"ur Tag seine Wonne, vor ihm mich erg"otzend allezeit,\\
\newpage 
{\bf -- 8.31}\\
\medskip \\
\begin{tabular}{rrrrrrrrp{120mm}}
WV&WK&WB&ABK&ABB&ABV&AnzB&TW&Zahlencode \textcolor{red}{$\boldsymbol{Grundtext}$} Umschrift $|$"Ubersetzung(en)\\
1.&214.&1676.&864.&6612.&1.&5&848&40\_300\_8\_100\_400 \textcolor{red}{\textcjheb{tq.h+sm}} MSCQT $|$mich erg"otzend/(ich war) spielend\\
2.&215.&1677.&869.&6617.&6.&4&434&2\_400\_2\_30 \textcolor{red}{\textcjheb{lbtb}} BTBL $|$auf dem bewohnten Teil/auf dem Festland\\
3.&216.&1678.&873.&6621.&10.&4&297&1\_200\_90\_6 \textcolor{red}{\textcjheb{w.sr'}} AR"sW $|$seiner Erde\\
4.&217.&1679.&877.&6625.&14.&6&756&6\_300\_70\_300\_70\_10 \textcolor{red}{\textcjheb{y`+s`+sw}} WSaSaJ $|$und meine Wonne war (es)\\
5.&218.&1680.&883.&6631.&20.&2&401&1\_400 \textcolor{red}{\textcjheb{t'}} AT $|$bei/mit\\
6.&219.&1681.&885.&6633.&22.&3&62&2\_50\_10 \textcolor{red}{\textcjheb{ynb}} BNJ $|$den Kindern/den S"ohnen\\
7.&220.&1682.&888.&6636.&25.&3&45&1\_4\_40 \textcolor{red}{\textcjheb{md'}} ADM $|$(des) Menschen\\
\end{tabular}\medskip \\
Ende des Verses 8.31\\
Verse: 233, Buchstaben: 27, 890, 6638, Totalwerte: 2843, 61394, 474026\\
\\
mich erg"otzend auf dem bewohnten Teile seiner Erde; und meine Wonne war bei den Menschenkindern.\\
\newpage 
{\bf -- 8.32}\\
\medskip \\
\begin{tabular}{rrrrrrrrp{120mm}}
WV&WK&WB&ABK&ABB&ABV&AnzB&TW&Zahlencode \textcolor{red}{$\boldsymbol{Grundtext}$} Umschrift $|$"Ubersetzung(en)\\
1.&221.&1683.&891.&6639.&1.&4&481&6\_70\_400\_5 \textcolor{red}{\textcjheb{ht`w}} WaTH $|$nun denn/und nun\\
2.&222.&1684.&895.&6643.&5.&4&102&2\_50\_10\_40 \textcolor{red}{\textcjheb{mynb}} BNJM $|$(ihr) S"ohne\\
3.&223.&1685.&899.&6647.&9.&4&416&300\_40\_70\_6 \textcolor{red}{\textcjheb{w`m+s}} SMaW $|$h"oret\\
4.&224.&1686.&903.&6651.&13.&2&40&30\_10 \textcolor{red}{\textcjheb{yl}} LJ $|$auf mich\\
5.&225.&1687.&905.&6653.&15.&5&517&6\_1\_300\_200\_10 \textcolor{red}{\textcjheb{yr+s'w}} WASRJ $|$gl"uckselig sind/und Seligkeiten\\
6.&226.&1688.&910.&6658.&20.&4&234&4\_200\_20\_10 \textcolor{red}{\textcjheb{ykrd}} DRKJ $|$(die) (meine) Wege\\
7.&227.&1689.&914.&6662.&24.&5&556&10\_300\_40\_200\_6 \textcolor{red}{\textcjheb{wrm+sy}} JSMRW $|$(sie) bewahren\\
\end{tabular}\medskip \\
Ende des Verses 8.32\\
Verse: 234, Buchstaben: 28, 918, 6666, Totalwerte: 2346, 63740, 476372\\
\\
Nun denn, ihr S"ohne, h"oret auf mich: Gl"uckselig sind, die meine Wege bewahren!\\
\newpage 
{\bf -- 8.33}\\
\medskip \\
\begin{tabular}{rrrrrrrrp{120mm}}
WV&WK&WB&ABK&ABB&ABV&AnzB&TW&Zahlencode \textcolor{red}{$\boldsymbol{Grundtext}$} Umschrift $|$"Ubersetzung(en)\\
1.&228.&1690.&919.&6667.&1.&4&416&300\_40\_70\_6 \textcolor{red}{\textcjheb{w`m+s}} SMaW $|$h"oret\\
2.&229.&1691.&923.&6671.&5.&4&306&40\_6\_60\_200 \textcolor{red}{\textcjheb{rswm}} MWsR $|$Unterweisung/die Zurechtweisung\\
3.&230.&1692.&927.&6675.&9.&5&80&6\_8\_20\_40\_6 \textcolor{red}{\textcjheb{wmk.hw}} WCKMW $|$und werdet weise/und seid weise\\
4.&231.&1693.&932.&6680.&14.&3&37&6\_1\_30 \textcolor{red}{\textcjheb{l'w}} WAL $|$und nicht(s)\\
5.&232.&1694.&935.&6683.&17.&5&756&400\_80\_200\_70\_6 \textcolor{red}{\textcjheb{w`rpt}} TPRaW $|$verwerft sie/ihr sollt unbeachtet lassen\\
\end{tabular}\medskip \\
Ende des Verses 8.33\\
Verse: 235, Buchstaben: 21, 939, 6687, Totalwerte: 1595, 65335, 477967\\
\\
H"oret Unterweisung und werdet weise, und verwerfet sie nicht!\\
\newpage 
{\bf -- 8.34}\\
\medskip \\
\begin{tabular}{rrrrrrrrp{120mm}}
WV&WK&WB&ABK&ABB&ABV&AnzB&TW&Zahlencode \textcolor{red}{$\boldsymbol{Grundtext}$} Umschrift $|$"Ubersetzung(en)\\
1.&233.&1695.&940.&6688.&1.&4&511&1\_300\_200\_10 \textcolor{red}{\textcjheb{yr+s'}} ASRJ $|$gl"uckselig/Seligkeiten (=Heil)\\
2.&234.&1696.&944.&6692.&5.&3&45&1\_4\_40 \textcolor{red}{\textcjheb{md'}} ADM $|$der Mensch/(dem) Menschen\\
3.&235.&1697.&947.&6695.&8.&3&410&300\_40\_70 \textcolor{red}{\textcjheb{`m+s}} SMa $|$der h"ort/(der ist) h"orend(er)\\
4.&236.&1698.&950.&6698.&11.&2&40&30\_10 \textcolor{red}{\textcjheb{yl}} LJ $|$auf mich\\
5.&237.&1699.&952.&6700.&13.&4&434&30\_300\_100\_4 \textcolor{red}{\textcjheb{dq+sl}} LSQD $|$indem er wacht/zu wachen\\
6.&238.&1700.&956.&6704.&17.&2&100&70\_30 \textcolor{red}{\textcjheb{l`}} aL $|$an\\
7.&239.&1701.&958.&6706.&19.&5&844&4\_30\_400\_400\_10 \textcolor{red}{\textcjheb{yttld}} DLTTJ $|$meinen T"uren\\
8.&240.&1702.&963.&6711.&24.&3&56&10\_6\_40 \textcolor{red}{\textcjheb{mwy}} JWM $|$Tag\\
9.&241.&1703.&966.&6714.&27.&3&56&10\_6\_40 \textcolor{red}{\textcjheb{mwy}} JWM $|$(f"ur) Tag\\
10.&242.&1704.&969.&6717.&30.&4&570&30\_300\_40\_200 \textcolor{red}{\textcjheb{rm+sl}} LSMR $|$h"utet/zu h"uten\\
11.&243.&1705.&973.&6721.&34.&5&460&40\_7\_6\_7\_400 \textcolor{red}{\textcjheb{tzwzm}} MZWZT $|$die Pfosten\\
12.&244.&1706.&978.&6726.&39.&4&498&80\_400\_8\_10 \textcolor{red}{\textcjheb{y.htp}} PTCJ $|$meiner Tore\\
\end{tabular}\medskip \\
Ende des Verses 8.34\\
Verse: 236, Buchstaben: 42, 981, 6729, Totalwerte: 4024, 69359, 481991\\
\\
Gl"uckselig der Mensch, der auf mich h"ort, indem er an meinen T"uren wacht Tag f"ur Tag, die Pfosten meiner Tore h"utet!\\
\newpage 
{\bf -- 8.35}\\
\medskip \\
\begin{tabular}{rrrrrrrrp{120mm}}
WV&WK&WB&ABK&ABB&ABV&AnzB&TW&Zahlencode \textcolor{red}{$\boldsymbol{Grundtext}$} Umschrift $|$"Ubersetzung(en)\\
1.&245.&1707.&982.&6730.&1.&2&30&20\_10 \textcolor{red}{\textcjheb{yk}} KJ $|$denn\\
2.&246.&1708.&984.&6732.&3.&4&141&40\_90\_1\_10 \textcolor{red}{\textcjheb{y'.sm}} M"sAJ $|$wer mich findet\\
3.&247.&1709.&988.&6736.&7.&4&141&40\_90\_1\_10 \textcolor{red}{\textcjheb{y'.sm}} M"sAJ $|$hat gefunden/(ist) findend\\
4.&248.&1710.&992.&6740.&11.&4&68&8\_10\_10\_40 \textcolor{red}{\textcjheb{myy.h}} CJJM $|$(das) Leben\\
5.&249.&1711.&996.&6744.&15.&4&196&6\_10\_80\_100 \textcolor{red}{\textcjheb{qpyw}} WJPQ $|$und (er) erlangt\\
6.&250.&1712.&1000.&6748.&19.&4&346&200\_90\_6\_50 \textcolor{red}{\textcjheb{nw.sr}} R"sWN $|$Wohlgefallen\\
7.&251.&1713.&1004.&6752.&23.&5&66&40\_10\_5\_6\_5 \textcolor{red}{\textcjheb{hwhym}} MJHWH $|$von Jahwe/vor Jahwe\\
\end{tabular}\medskip \\
Ende des Verses 8.35\\
Verse: 237, Buchstaben: 27, 1008, 6756, Totalwerte: 988, 70347, 482979\\
\\
Denn wer mich findet, hat das Leben gefunden und Wohlgefallen erlangt von Jahwe.\\
\newpage 
{\bf -- 8.36}\\
\medskip \\
\begin{tabular}{rrrrrrrrp{120mm}}
WV&WK&WB&ABK&ABB&ABV&AnzB&TW&Zahlencode \textcolor{red}{$\boldsymbol{Grundtext}$} Umschrift $|$"Ubersetzung(en)\\
1.&252.&1714.&1009.&6757.&1.&5&34&6\_8\_9\_1\_10 \textcolor{red}{\textcjheb{y'.t.hw}} WCtAJ $|$war aber s"undigt an mir/und wer sich verfehlt gegen mich\\
2.&253.&1715.&1014.&6762.&6.&3&108&8\_40\_60 \textcolor{red}{\textcjheb{sm.h}} CMs $|$tut Gewalt an/(ist) Gewalt antuend(er)\\
3.&254.&1716.&1017.&6765.&9.&4&436&50\_80\_300\_6 \textcolor{red}{\textcjheb{w+spn}} NPSW $|$seiner Seele/sich selbst\\
4.&255.&1717.&1021.&6769.&13.&2&50&20\_30 \textcolor{red}{\textcjheb{lk}} KL $|$alle\\
5.&256.&1718.&1023.&6771.&15.&5&401&40\_300\_50\_1\_10 \textcolor{red}{\textcjheb{y'n+sm}} MSNAJ $|$die mich hassen/meine Hassenden\\
6.&257.&1719.&1028.&6776.&20.&4&14&1\_5\_2\_6 \textcolor{red}{\textcjheb{wbh'}} AHBW $|$(sie) lieb(t)en\\
7.&258.&1720.&1032.&6780.&24.&3&446&40\_6\_400 \textcolor{red}{\textcjheb{twm}} MWT $|$den Tod\\
\end{tabular}\medskip \\
Ende des Verses 8.36\\
Verse: 238, Buchstaben: 26, 1034, 6782, Totalwerte: 1489, 71836, 484468\\
\\
Wer aber an mir s"undigt, tut seiner Seele Gewalt an; alle, die mich hassen, lieben den Tod.\\
\\
{\bf Ende des Kapitels 8}\\
\newpage 
{\bf -- 9.1}\\
\medskip \\
\begin{tabular}{rrrrrrrrp{120mm}}
WV&WK&WB&ABK&ABB&ABV&AnzB&TW&Zahlencode \textcolor{red}{$\boldsymbol{Grundtext}$} Umschrift $|$"Ubersetzung(en)\\
1.&1.&1721.&1.&6783.&1.&5&474&8\_20\_40\_6\_400 \textcolor{red}{\textcjheb{twmk.h}} CKMWT $|$(die) Weisheit(en)\\
2.&2.&1722.&6.&6788.&6.&4&457&2\_50\_400\_5 \textcolor{red}{\textcjheb{htnb}} BNTH $|$hat gebaut/(sie) hat erbaut\\
3.&3.&1723.&10.&6792.&10.&4&417&2\_10\_400\_5 \textcolor{red}{\textcjheb{htyb}} BJTH $|$ihr Haus\\
4.&4.&1724.&14.&6796.&14.&4&105&8\_90\_2\_5 \textcolor{red}{\textcjheb{hb.s.h}} C"sBH $|$(sie) hat ausgehauen\\
5.&5.&1725.&18.&6800.&18.&6&135&70\_40\_6\_4\_10\_5 \textcolor{red}{\textcjheb{hydwm`}} aMWDJH $|$ihre S"aulen\\
6.&6.&1726.&24.&6806.&24.&4&377&300\_2\_70\_5 \textcolor{red}{\textcjheb{h`b+s}} SBaH $|$sieben\\
\end{tabular}\medskip \\
Ende des Verses 9.1\\
Verse: 239, Buchstaben: 27, 27, 6809, Totalwerte: 1965, 1965, 486433\\
\\
Die Weisheit hat ihr Haus gebaut, hat ihre sieben S"aulen ausgehauen;\\
\newpage 
{\bf -- 9.2}\\
\medskip \\
\begin{tabular}{rrrrrrrrp{120mm}}
WV&WK&WB&ABK&ABB&ABV&AnzB&TW&Zahlencode \textcolor{red}{$\boldsymbol{Grundtext}$} Umschrift $|$"Ubersetzung(en)\\
1.&7.&1727.&28.&6810.&1.&4&24&9\_2\_8\_5 \textcolor{red}{\textcjheb{h.hb.t}} tBCH $|$sie hat geschlachtet\\
2.&8.&1728.&32.&6814.&5.&4&24&9\_2\_8\_5 \textcolor{red}{\textcjheb{h.hb.t}} tBCH $|$ihr Schlachtvieh/ihr Geschlachtetes\\
3.&9.&1729.&36.&6818.&9.&4&125&40\_60\_20\_5 \textcolor{red}{\textcjheb{hksm}} MsKH $|$(sie) (hat) gemischt\\
4.&10.&1730.&40.&6822.&13.&4&75&10\_10\_50\_5 \textcolor{red}{\textcjheb{hnyy}} JJNH $|$ihren Wein\\
5.&11.&1731.&44.&6826.&17.&2&81&1\_80 \textcolor{red}{\textcjheb{p'}} AP $|$auch\\
6.&12.&1732.&46.&6828.&19.&4&295&70\_200\_20\_5 \textcolor{red}{\textcjheb{hkr`}} aRKH $|$gedeckt/(sie hat) hergerichtet\\
7.&13.&1733.&50.&6832.&23.&5&393&300\_30\_8\_50\_5 \textcolor{red}{\textcjheb{hn.hl+s}} SLCNH $|$ihren Tisch\\
\end{tabular}\medskip \\
Ende des Verses 9.2\\
Verse: 240, Buchstaben: 27, 54, 6836, Totalwerte: 1017, 2982, 487450\\
\\
sie hat ihr Schlachtvieh geschlachtet, ihren Wein gemischt, auch ihren Tisch gedeckt;\\
\newpage 
{\bf -- 9.3}\\
\medskip \\
\begin{tabular}{rrrrrrrrp{120mm}}
WV&WK&WB&ABK&ABB&ABV&AnzB&TW&Zahlencode \textcolor{red}{$\boldsymbol{Grundtext}$} Umschrift $|$"Ubersetzung(en)\\
1.&14.&1734.&55.&6837.&1.&4&343&300\_30\_8\_5 \textcolor{red}{\textcjheb{h.hl+s}} SLCH $|$sie hat ausgesandt\\
2.&15.&1735.&59.&6841.&5.&6&735&50\_70\_200\_400\_10\_5 \textcolor{red}{\textcjheb{hytr`n}} NaRTJH $|$ihre M"agde/ihre M"adchen\\
3.&16.&1736.&65.&6847.&11.&4&701&400\_100\_200\_1 \textcolor{red}{\textcjheb{'rqt}} TQRA $|$ladet ein/(und) sie l"asst rufen\\
4.&17.&1737.&69.&6851.&15.&2&100&70\_30 \textcolor{red}{\textcjheb{l`}} aL $|$auf\\
5.&18.&1738.&71.&6853.&17.&3&93&3\_80\_10 \textcolor{red}{\textcjheb{ypg}} GPJ $|$die (R"ucken)\\
6.&19.&1739.&74.&6856.&20.&4&290&40\_200\_40\_10 \textcolor{red}{\textcjheb{ymrm}} MRMJ $|$(der) H"ohen\\
7.&20.&1740.&78.&6860.&24.&3&700&100\_200\_400 \textcolor{red}{\textcjheb{trq}} QRT $|$(der) Stadt\\
\end{tabular}\medskip \\
Ende des Verses 9.3\\
Verse: 241, Buchstaben: 26, 80, 6862, Totalwerte: 2962, 5944, 490412\\
\\
sie hat ihre M"agde ausgesandt, ladet ein auf den H"ohen der Stadt:\\
\newpage 
{\bf -- 9.4}\\
\medskip \\
\begin{tabular}{rrrrrrrrp{120mm}}
WV&WK&WB&ABK&ABB&ABV&AnzB&TW&Zahlencode \textcolor{red}{$\boldsymbol{Grundtext}$} Umschrift $|$"Ubersetzung(en)\\
1.&21.&1741.&81.&6863.&1.&2&50&40\_10 \textcolor{red}{\textcjheb{ym}} MJ $|$wer\\
2.&22.&1742.&83.&6865.&3.&3&490&80\_400\_10 \textcolor{red}{\textcjheb{ytp}} PTJ $|$einf"altig (ist)\\
3.&23.&1743.&86.&6868.&6.&3&270&10\_60\_200 \textcolor{red}{\textcjheb{rsy}} JsR $|$er wende sich/(d)er soll kehren\\
4.&24.&1744.&89.&6871.&9.&3&60&5\_50\_5 \textcolor{red}{\textcjheb{hnh}} HNH $|$hierher\\
5.&25.&1745.&92.&6874.&12.&3&268&8\_60\_200 \textcolor{red}{\textcjheb{rs.h}} CsR $|$zu den Un-/(wer ist) ermangelnd\\
6.&26.&1746.&95.&6877.&15.&2&32&30\_2 \textcolor{red}{\textcjheb{bl}} LB $|$verst"andigen/Herz (=Verstand)\\
7.&27.&1747.&97.&6879.&17.&4&246&1\_40\_200\_5 \textcolor{red}{\textcjheb{hrm'}} AMRH $|$sie spricht\\
8.&28.&1748.&101.&6883.&21.&2&36&30\_6 \textcolor{red}{\textcjheb{wl}} LW $|$/zu ihm\\
\end{tabular}\medskip \\
Ende des Verses 9.4\\
Verse: 242, Buchstaben: 22, 102, 6884, Totalwerte: 1452, 7396, 491864\\
\\
'Wer ist einf"altig? Er wende sich hierher!' Zu den Unverst"andigen spricht sie:\\
\newpage 
{\bf -- 9.5}\\
\medskip \\
\begin{tabular}{rrrrrrrrp{120mm}}
WV&WK&WB&ABK&ABB&ABV&AnzB&TW&Zahlencode \textcolor{red}{$\boldsymbol{Grundtext}$} Umschrift $|$"Ubersetzung(en)\\
1.&29.&1749.&103.&6885.&1.&3&56&30\_20\_6 \textcolor{red}{\textcjheb{wkl}} LKW $|$kommt\\
2.&30.&1750.&106.&6888.&4.&4&84&30\_8\_40\_6 \textcolor{red}{\textcjheb{wm.hl}} LCMW $|$esset\\
3.&31.&1751.&110.&6892.&8.&5&90&2\_30\_8\_40\_10 \textcolor{red}{\textcjheb{ym.hlb}} BLCMJ $|$von meinem Brot\\
4.&32.&1752.&115.&6897.&13.&4&712&6\_300\_400\_6 \textcolor{red}{\textcjheb{wt+sw}} WSTW $|$und trinkt\\
5.&33.&1753.&119.&6901.&17.&4&72&2\_10\_10\_50 \textcolor{red}{\textcjheb{nyyb}} BJJN $|$vom Wein\\
6.&34.&1754.&123.&6905.&21.&5&530&40\_60\_20\_400\_10 \textcolor{red}{\textcjheb{ytksm}} MsKTJ $|$(den) ich gemischt (habe)\\
\end{tabular}\medskip \\
Ende des Verses 9.5\\
Verse: 243, Buchstaben: 25, 127, 6909, Totalwerte: 1544, 8940, 493408\\
\\
'Kommet, esset von meinem Brote, und trinket von dem Weine, den ich gemischt habe!\\
\newpage 
{\bf -- 9.6}\\
\medskip \\
\begin{tabular}{rrrrrrrrp{120mm}}
WV&WK&WB&ABK&ABB&ABV&AnzB&TW&Zahlencode \textcolor{red}{$\boldsymbol{Grundtext}$} Umschrift $|$"Ubersetzung(en)\\
1.&35.&1755.&128.&6910.&1.&4&85&70\_7\_2\_6 \textcolor{red}{\textcjheb{wbz`}} aZBW $|$lasst ab/verlasst\\
2.&36.&1756.&132.&6914.&5.&5&531&80\_400\_1\_10\_40 \textcolor{red}{\textcjheb{my'tp}} PTAJM $|$von der Einf"altigkeit/(die) Einf"altigkeiten\\
3.&37.&1757.&137.&6919.&10.&4&30&6\_8\_10\_6 \textcolor{red}{\textcjheb{wy.hw}} WCJW $|$und lebt/dass ihr lebet\\
4.&38.&1758.&141.&6923.&14.&5&513&6\_1\_300\_200\_6 \textcolor{red}{\textcjheb{wr+s'w}} WASRW $|$und schreitet einher/und geht (gerade)\\
5.&39.&1759.&146.&6928.&19.&4&226&2\_4\_200\_20 \textcolor{red}{\textcjheb{krdb}} BDRK $|$auf dem Weg\\
6.&40.&1760.&150.&6932.&23.&4&67&2\_10\_50\_5 \textcolor{red}{\textcjheb{hnyb}} BJNH $|$des Verstands/(der) Einsicht\\
\end{tabular}\medskip \\
Ende des Verses 9.6\\
Verse: 244, Buchstaben: 26, 153, 6935, Totalwerte: 1452, 10392, 494860\\
\\
Lasset ab von der Einf"altigkeit und lebet, und schreitet einher auf dem Wege des Verstandes!' -\\
\newpage 
{\bf -- 9.7}\\
\medskip \\
\begin{tabular}{rrrrrrrrp{120mm}}
WV&WK&WB&ABK&ABB&ABV&AnzB&TW&Zahlencode \textcolor{red}{$\boldsymbol{Grundtext}$} Umschrift $|$"Ubersetzung(en)\\
1.&41.&1761.&154.&6936.&1.&3&270&10\_60\_200 \textcolor{red}{\textcjheb{rsy}} JsR $|$wer zurechtweist/ein Zurechtweisender\\
2.&42.&1762.&157.&6939.&4.&2&120&30\_90 \textcolor{red}{\textcjheb{.sl}} L"s $|$den Sp"otter/einen Sp"otter\\
3.&43.&1763.&159.&6941.&6.&3&138&30\_100\_8 \textcolor{red}{\textcjheb{.hql}} LQC $|$zieht zu/(ist) nehmend(er)\\
4.&44.&1764.&162.&6944.&9.&2&36&30\_6 \textcolor{red}{\textcjheb{wl}} LW $|$sich\\
5.&45.&1765.&164.&6946.&11.&4&186&100\_30\_6\_50 \textcolor{red}{\textcjheb{nwlq}} QLWN $|$Schande/Schimpf\\
6.&46.&1766.&168.&6950.&15.&6&90&6\_40\_6\_20\_10\_8 \textcolor{red}{\textcjheb{.hykwmw}} WMWKJC $|$und wer straft/und ein Tadelnder\\
7.&47.&1767.&174.&6956.&21.&4&600&30\_200\_300\_70 \textcolor{red}{\textcjheb{`+srl}} LRSa $|$den Gesetzlosen/(den) Frevler\\
8.&48.&1768.&178.&6960.&25.&4&92&40\_6\_40\_6 \textcolor{red}{\textcjheb{wmwm}} MWMW $|$sein Schandfleck ist es/seinen (eigenen) Schandfleck\\
\end{tabular}\medskip \\
Ende des Verses 9.7\\
Verse: 245, Buchstaben: 28, 181, 6963, Totalwerte: 1532, 11924, 496392\\
\\
Wer den Sp"otter zurechtweist, zieht sich Schande zu; und wer den Gesetzlosen straft, sein Schandfleck ist es.\\
\newpage 
{\bf -- 9.8}\\
\medskip \\
\begin{tabular}{rrrrrrrrp{120mm}}
WV&WK&WB&ABK&ABB&ABV&AnzB&TW&Zahlencode \textcolor{red}{$\boldsymbol{Grundtext}$} Umschrift $|$"Ubersetzung(en)\\
1.&49.&1769.&182.&6964.&1.&2&31&1\_30 \textcolor{red}{\textcjheb{l'}} AL $|$nicht\\
2.&50.&1770.&184.&6966.&3.&4&434&400\_6\_20\_8 \textcolor{red}{\textcjheb{.hkwt}} TWKC $|$strafe/du sollst tadeln\\
3.&51.&1771.&188.&6970.&7.&2&120&30\_90 \textcolor{red}{\textcjheb{.sl}} L"s $|$den Sp"otter/einen Sp"otter\\
4.&52.&1772.&190.&6972.&9.&2&130&80\_50 \textcolor{red}{\textcjheb{np}} PN $|$dass nicht\\
5.&53.&1773.&192.&6974.&11.&5&381&10\_300\_50\_1\_20 \textcolor{red}{\textcjheb{k'n+sy}} JSNAK $|$er dich hasse/er hasst dich\\
6.&54.&1774.&197.&6979.&16.&4&39&5\_6\_20\_8 \textcolor{red}{\textcjheb{.hkwh}} HWKC $|$strafe/tadle\\
7.&55.&1775.&201.&6983.&20.&4&98&30\_8\_20\_40 \textcolor{red}{\textcjheb{mk.hl}} LCKM $|$den Weisen\\
8.&56.&1776.&205.&6987.&24.&6&44&6\_10\_1\_5\_2\_20 \textcolor{red}{\textcjheb{kbh'yw}} WJAHBK $|$und er wird lieben dich\\
\end{tabular}\medskip \\
Ende des Verses 9.8\\
Verse: 246, Buchstaben: 29, 210, 6992, Totalwerte: 1277, 13201, 497669\\
\\
Strafe den Sp"otter nicht, da"s er dich nicht hasse; strafe den Weisen, und er wird dich lieben.\\
\newpage 
{\bf -- 9.9}\\
\medskip \\
\begin{tabular}{rrrrrrrrp{120mm}}
WV&WK&WB&ABK&ABB&ABV&AnzB&TW&Zahlencode \textcolor{red}{$\boldsymbol{Grundtext}$} Umschrift $|$"Ubersetzung(en)\\
1.&57.&1777.&211.&6993.&1.&2&450&400\_50 \textcolor{red}{\textcjheb{nt}} TN $|$gib\\
2.&58.&1778.&213.&6995.&3.&4&98&30\_8\_20\_40 \textcolor{red}{\textcjheb{mk.hl}} LCKM $|$dem Weisen\\
3.&59.&1779.&217.&6999.&7.&5&84&6\_10\_8\_20\_40 \textcolor{red}{\textcjheb{mk.hyw}} WJCKM $|$und er wird weiser\\
4.&60.&1780.&222.&7004.&12.&3&80&70\_6\_4 \textcolor{red}{\textcjheb{dw`}} aWD $|$noch\\
5.&61.&1781.&225.&7007.&15.&4&85&5\_6\_4\_70 \textcolor{red}{\textcjheb{`dwh}} HWDa $|$belehre\\
6.&62.&1782.&229.&7011.&19.&5&234&30\_90\_4\_10\_100 \textcolor{red}{\textcjheb{qyd.sl}} L"sDJQ $|$den Gerechten\\
7.&63.&1783.&234.&7016.&24.&5&162&6\_10\_6\_60\_80 \textcolor{red}{\textcjheb{pswyw}} WJWsP $|$so wird er zunehen an/und er macht hinzuf"ugen\\
8.&64.&1784.&239.&7021.&29.&3&138&30\_100\_8 \textcolor{red}{\textcjheb{.hql}} LQC $|$Kenntnis/Einsicht\\
\end{tabular}\medskip \\
Ende des Verses 9.9\\
Verse: 247, Buchstaben: 31, 241, 7023, Totalwerte: 1331, 14532, 499000\\
\\
Gib dem Weisen, so wird er noch weiser; belehre den Gerechten, so wird er an Kenntnis zunehmen. -\\
\newpage 
{\bf -- 9.10}\\
\medskip \\
\begin{tabular}{rrrrrrrrp{120mm}}
WV&WK&WB&ABK&ABB&ABV&AnzB&TW&Zahlencode \textcolor{red}{$\boldsymbol{Grundtext}$} Umschrift $|$"Ubersetzung(en)\\
1.&65.&1785.&242.&7024.&1.&4&838&400\_8\_30\_400 \textcolor{red}{\textcjheb{tl.ht}} TCLT $|$(der) Anfang\\
2.&66.&1786.&246.&7028.&5.&4&73&8\_20\_40\_5 \textcolor{red}{\textcjheb{hmk.h}} CKMH $|$der Weisheit\\
3.&67.&1787.&250.&7032.&9.&4&611&10\_200\_1\_400 \textcolor{red}{\textcjheb{t'ry}} JRAT $|$(ist die) Furcht\\
4.&68.&1788.&254.&7036.&13.&4&26&10\_5\_6\_5 \textcolor{red}{\textcjheb{hwhy}} JHWH $|$(vor) Jahwe(s)\\
5.&69.&1789.&258.&7040.&17.&4&480&6\_4\_70\_400 \textcolor{red}{\textcjheb{t`dw}} WDaT $|$und die Erkenntnis/und ein Erkennen\\
6.&70.&1790.&262.&7044.&21.&5&454&100\_4\_300\_10\_40 \textcolor{red}{\textcjheb{my+sdq}} QDSJM $|$(des) Heiligen\\
7.&71.&1791.&267.&7049.&26.&4&67&2\_10\_50\_5 \textcolor{red}{\textcjheb{hnyb}} BJNH $|$ist Verstand/(ist) Einsicht\\
\end{tabular}\medskip \\
Ende des Verses 9.10\\
Verse: 248, Buchstaben: 29, 270, 7052, Totalwerte: 2549, 17081, 501549\\
\\
Die Furcht Jahwes ist der Weisheit Anfang; und die Erkenntnis des Heiligen ist Verstand.\\
\newpage 
{\bf -- 9.11}\\
\medskip \\
\begin{tabular}{rrrrrrrrp{120mm}}
WV&WK&WB&ABK&ABB&ABV&AnzB&TW&Zahlencode \textcolor{red}{$\boldsymbol{Grundtext}$} Umschrift $|$"Ubersetzung(en)\\
1.&72.&1792.&271.&7053.&1.&2&30&20\_10 \textcolor{red}{\textcjheb{yk}} KJ $|$denn\\
2.&73.&1793.&273.&7055.&3.&2&12&2\_10 \textcolor{red}{\textcjheb{yb}} BJ $|$durch mich\\
3.&74.&1794.&275.&7057.&5.&4&218&10\_200\_2\_6 \textcolor{red}{\textcjheb{wbry}} JRBW $|$werden sich mehren/sie (=es) werden viel sein\\
4.&75.&1795.&279.&7061.&9.&4&80&10\_40\_10\_20 \textcolor{red}{\textcjheb{kymy}} JMJK $|$deine Tage\\
5.&76.&1796.&283.&7065.&13.&7&178&6\_10\_6\_60\_10\_80\_6 \textcolor{red}{\textcjheb{wpyswyw}} WJWsJPW $|$und hinzugef"ugt werden/und sie (=es) werden sich mehren\\
6.&77.&1797.&290.&7072.&20.&2&50&30\_20 \textcolor{red}{\textcjheb{kl}} LK $|$dir/f"ur dich\\
7.&78.&1798.&292.&7074.&22.&4&756&300\_50\_6\_400 \textcolor{red}{\textcjheb{twn+s}} SNWT $|$Jahre\\
8.&79.&1799.&296.&7078.&26.&4&68&8\_10\_10\_40 \textcolor{red}{\textcjheb{myy.h}} CJJM $|$des Lebens/der Lebenden\\
\end{tabular}\medskip \\
Ende des Verses 9.11\\
Verse: 249, Buchstaben: 29, 299, 7081, Totalwerte: 1392, 18473, 502941\\
\\
Denn durch mich werden deine Tage sich mehren, und Jahre des Lebens werden dir hinzugef"ugt werden.\\
\newpage 
{\bf -- 9.12}\\
\medskip \\
\begin{tabular}{rrrrrrrrp{120mm}}
WV&WK&WB&ABK&ABB&ABV&AnzB&TW&Zahlencode \textcolor{red}{$\boldsymbol{Grundtext}$} Umschrift $|$"Ubersetzung(en)\\
1.&80.&1800.&300.&7082.&1.&2&41&1\_40 \textcolor{red}{\textcjheb{m'}} AM $|$wenn\\
2.&81.&1801.&302.&7084.&3.&4&468&8\_20\_40\_400 \textcolor{red}{\textcjheb{tmk.h}} CKMT $|$du bist weise\\
3.&82.&1802.&306.&7088.&7.&4&468&8\_20\_40\_400 \textcolor{red}{\textcjheb{tmk.h}} CKMT $|$so bist du weise/du bist weise\\
4.&83.&1803.&310.&7092.&11.&2&50&30\_20 \textcolor{red}{\textcjheb{kl}} LK $|$f"ur dich/zu dir\\
5.&84.&1804.&312.&7094.&13.&4&526&6\_30\_90\_400 \textcolor{red}{\textcjheb{t.slw}} WL"sT $|$und spottest du/und wenn "uberm"utig (du bist)\\
6.&85.&1805.&316.&7098.&17.&4&56&30\_2\_4\_20 \textcolor{red}{\textcjheb{kdbl}} LBDK $|$(so) du allein\\
7.&86.&1806.&320.&7102.&21.&3&701&400\_300\_1 \textcolor{red}{\textcjheb{'+st}} TSA $|$wirst es tragen/du musst es tragen\\
\end{tabular}\medskip \\
Ende des Verses 9.12\\
Verse: 250, Buchstaben: 23, 322, 7104, Totalwerte: 2310, 20783, 505251\\
\\
Wenn du weise bist, so bist du weise f"ur dich; und spottest du, so wirst du allein es tragen.\\
\newpage 
{\bf -- 9.13}\\
\medskip \\
\begin{tabular}{rrrrrrrrp{120mm}}
WV&WK&WB&ABK&ABB&ABV&AnzB&TW&Zahlencode \textcolor{red}{$\boldsymbol{Grundtext}$} Umschrift $|$"Ubersetzung(en)\\
1.&87.&1807.&323.&7105.&1.&3&701&1\_300\_400 \textcolor{red}{\textcjheb{t+s'}} AST $|$(eine) Frau\\
2.&88.&1808.&326.&7108.&4.&6&526&20\_60\_10\_30\_6\_400 \textcolor{red}{\textcjheb{twlysk}} KsJLWT $|$(von) Torheit\\
3.&89.&1809.&332.&7114.&10.&4&60&5\_40\_10\_5 \textcolor{red}{\textcjheb{hymh}} HMJH $|$ist leidenschaftlich/(ist) l"armend\\
4.&90.&1810.&336.&7118.&14.&5&896&80\_400\_10\_6\_400 \textcolor{red}{\textcjheb{twytp}} PTJWT $|$sie ist lauter Einf"altigkeit/Einfalt\\
5.&91.&1811.&341.&7123.&19.&3&38&6\_2\_30 \textcolor{red}{\textcjheb{lbw}} WBL $|$und nicht(s)\\
6.&92.&1812.&344.&7126.&22.&4&89&10\_4\_70\_5 \textcolor{red}{\textcjheb{h`dy}} JDaH $|$(sie) wei"s\\
7.&93.&1813.&348.&7130.&26.&2&45&40\_5 \textcolor{red}{\textcjheb{hm}} MH $|$gar/was (es ist)\\
\end{tabular}\medskip \\
Ende des Verses 9.13\\
Verse: 251, Buchstaben: 27, 349, 7131, Totalwerte: 2355, 23138, 507606\\
\\
Frau Torheit ist leidenschaftlich; sie ist lauter Einf"altigkeit und wei"s gar nichts.\\
\newpage 
{\bf -- 9.14}\\
\medskip \\
\begin{tabular}{rrrrrrrrp{120mm}}
WV&WK&WB&ABK&ABB&ABV&AnzB&TW&Zahlencode \textcolor{red}{$\boldsymbol{Grundtext}$} Umschrift $|$"Ubersetzung(en)\\
1.&94.&1814.&350.&7132.&1.&5&323&6\_10\_300\_2\_5 \textcolor{red}{\textcjheb{hb+syw}} WJSBH $|$und sie sitzt\\
2.&95.&1815.&355.&7137.&6.&4&518&30\_80\_400\_8 \textcolor{red}{\textcjheb{.htpl}} LPTC $|$am Eingang\\
3.&96.&1816.&359.&7141.&10.&4&417&2\_10\_400\_5 \textcolor{red}{\textcjheb{htyb}} BJTH $|$ihres Hauses\\
4.&97.&1817.&363.&7145.&14.&2&100&70\_30 \textcolor{red}{\textcjheb{l`}} aL $|$auf\\
5.&98.&1818.&365.&7147.&16.&3&81&20\_60\_1 \textcolor{red}{\textcjheb{'sk}} KsA $|$einem Sitz/einem Sessel\\
6.&99.&1819.&368.&7150.&19.&4&290&40\_200\_40\_10 \textcolor{red}{\textcjheb{ymrm}} MRMJ $|$an hochgelegenen Stellen/auf den H"ohen\\
7.&100.&1820.&372.&7154.&23.&3&700&100\_200\_400 \textcolor{red}{\textcjheb{trq}} QRT $|$(der) Stadt\\
\end{tabular}\medskip \\
Ende des Verses 9.14\\
Verse: 252, Buchstaben: 25, 374, 7156, Totalwerte: 2429, 25567, 510035\\
\\
Und sie sitzt am Eingang ihres Hauses, auf einem Sitze an hochgelegenen Stellen der Stadt,\\
\newpage 
{\bf -- 9.15}\\
\medskip \\
\begin{tabular}{rrrrrrrrp{120mm}}
WV&WK&WB&ABK&ABB&ABV&AnzB&TW&Zahlencode \textcolor{red}{$\boldsymbol{Grundtext}$} Umschrift $|$"Ubersetzung(en)\\
1.&101.&1821.&375.&7157.&1.&4&331&30\_100\_200\_1 \textcolor{red}{\textcjheb{'rql}} LQRA $|$um einzuladen/um zu rufen\\
2.&102.&1822.&379.&7161.&5.&5&312&30\_70\_2\_200\_10 \textcolor{red}{\textcjheb{yrb`l}} LaBRJ $|$die vor"ubergehen/die Vorbeiziehenden\\
3.&103.&1823.&384.&7166.&10.&3&224&4\_200\_20 \textcolor{red}{\textcjheb{krd}} DRK $|$des Wegs\\
4.&104.&1824.&387.&7169.&13.&7&605&5\_40\_10\_300\_200\_10\_40 \textcolor{red}{\textcjheb{myr+symh}} HMJSRJM $|$die gerade halten/die geradeaus Gehenden\\
5.&105.&1825.&394.&7176.&20.&6&655&1\_200\_8\_6\_400\_40 \textcolor{red}{\textcjheb{mtw.hr'}} ARCWTM $|$ihre Pfade\\
\end{tabular}\medskip \\
Ende des Verses 9.15\\
Verse: 253, Buchstaben: 25, 399, 7181, Totalwerte: 2127, 27694, 512162\\
\\
um einzuladen, die des Weges vor"ubergehen, die ihre Pfade gerade halten:\\
\newpage 
{\bf -- 9.16}\\
\medskip \\
\begin{tabular}{rrrrrrrrp{120mm}}
WV&WK&WB&ABK&ABB&ABV&AnzB&TW&Zahlencode \textcolor{red}{$\boldsymbol{Grundtext}$} Umschrift $|$"Ubersetzung(en)\\
1.&106.&1826.&400.&7182.&1.&2&50&40\_10 \textcolor{red}{\textcjheb{ym}} MJ $|$wer\\
2.&107.&1827.&402.&7184.&3.&3&490&80\_400\_10 \textcolor{red}{\textcjheb{ytp}} PTJ $|$einf"altig (ist)\\
3.&108.&1828.&405.&7187.&6.&3&270&10\_60\_200 \textcolor{red}{\textcjheb{rsy}} JsR $|$er wende sich/(er) m"oge abbiegen\\
4.&109.&1829.&408.&7190.&9.&3&60&5\_50\_5 \textcolor{red}{\textcjheb{hnh}} HNH $|$hierher\\
5.&110.&1830.&411.&7193.&12.&4&274&6\_8\_60\_200 \textcolor{red}{\textcjheb{rs.hw}} WCsR $|$und zu dem Un-//und der ermangelnd ist\\
6.&111.&1831.&415.&7197.&16.&2&32&30\_2 \textcolor{red}{\textcjheb{bl}} LB $|$verst"andigen/Herz (=Verstand)\\
7.&112.&1832.&417.&7199.&18.&5&252&6\_1\_40\_200\_5 \textcolor{red}{\textcjheb{hrm'w}} WAMRH $|$sie spricht/und sie sagte\\
8.&113.&1833.&422.&7204.&23.&2&36&30\_6 \textcolor{red}{\textcjheb{wl}} LW $|$/zu ihm\\
\end{tabular}\medskip \\
Ende des Verses 9.16\\
Verse: 254, Buchstaben: 24, 423, 7205, Totalwerte: 1464, 29158, 513626\\
\\
'Wer ist einf"altig? Er wende sich hierher!' Und zu dem Unverst"andigen spricht sie:\\
\newpage 
{\bf -- 9.17}\\
\medskip \\
\begin{tabular}{rrrrrrrrp{120mm}}
WV&WK&WB&ABK&ABB&ABV&AnzB&TW&Zahlencode \textcolor{red}{$\boldsymbol{Grundtext}$} Umschrift $|$"Ubersetzung(en)\\
1.&114.&1834.&424.&7206.&1.&3&90&40\_10\_40 \textcolor{red}{\textcjheb{mym}} MJM $|$Wasser\\
2.&115.&1835.&427.&7209.&4.&6&111&3\_50\_6\_2\_10\_40 \textcolor{red}{\textcjheb{mybwng}} GNWBJM $|$gestohlene(s)\\
3.&116.&1836.&433.&7215.&10.&5&556&10\_40\_400\_100\_6 \textcolor{red}{\textcjheb{wqtmy}} JMTQW $|$(sie) sind s"u"s\\
4.&117.&1837.&438.&7220.&15.&4&84&6\_30\_8\_40 \textcolor{red}{\textcjheb{m.hlw}} WLCM $|$und Brot\\
5.&118.&1838.&442.&7224.&19.&5&710&60\_400\_200\_10\_40 \textcolor{red}{\textcjheb{myrts}} sTRJM $|$heimliches/verborgenes\\
6.&119.&1839.&447.&7229.&24.&4&170&10\_50\_70\_40 \textcolor{red}{\textcjheb{m`ny}} JNaM $|$ist lieblich/er (=es) ist angenehm\\
\end{tabular}\medskip \\
Ende des Verses 9.17\\
Verse: 255, Buchstaben: 27, 450, 7232, Totalwerte: 1721, 30879, 515347\\
\\
"Gestohlene Wasser sind s"u"s, und heimliches Brot ist lieblich".\\
\newpage 
{\bf -- 9.18}\\
\medskip \\
\begin{tabular}{rrrrrrrrp{120mm}}
WV&WK&WB&ABK&ABB&ABV&AnzB&TW&Zahlencode \textcolor{red}{$\boldsymbol{Grundtext}$} Umschrift $|$"Ubersetzung(en)\\
1.&120.&1840.&451.&7233.&1.&3&37&6\_30\_1 \textcolor{red}{\textcjheb{'lw}} WLA $|$und nicht\\
2.&121.&1841.&454.&7236.&4.&3&84&10\_4\_70 \textcolor{red}{\textcjheb{`dy}} JDa $|$er wei"s/er merkt\\
3.&122.&1842.&457.&7239.&7.&2&30&20\_10 \textcolor{red}{\textcjheb{yk}} KJ $|$dass\\
4.&123.&1843.&459.&7241.&9.&5&331&200\_80\_1\_10\_40 \textcolor{red}{\textcjheb{my'pr}} RPAJM $|$Schatten/Verstorbene\\
5.&124.&1844.&464.&7246.&14.&2&340&300\_40 \textcolor{red}{\textcjheb{m+s}} SM $|$dort (sind)\\
6.&125.&1845.&466.&7248.&16.&5&222&2\_70\_40\_100\_10 \textcolor{red}{\textcjheb{yqm`b}} BaMQJ $|$in den Tiefen\\
7.&126.&1846.&471.&7253.&21.&4&337&300\_1\_6\_30 \textcolor{red}{\textcjheb{lw'+s}} SAWL $|$des Scheols/(des) Totenreichs\\
8.&127.&1847.&475.&7257.&25.&5&316&100\_200\_1\_10\_5 \textcolor{red}{\textcjheb{hy'rq}} QRAJH $|$ihre Geladenen/ihre Gerufenen\\
\end{tabular}\medskip \\
Ende des Verses 9.18\\
Verse: 256, Buchstaben: 29, 479, 7261, Totalwerte: 1697, 32576, 517044\\
\\
Und er wei"s nicht, da"s dort die Schatten sind, in den Tiefen des Scheols ihre Geladenen.\\
\\
{\bf Ende des Kapitels 9}\\
\newpage 
{\bf -- 10.1}\\
\medskip \\
\begin{tabular}{rrrrrrrrp{120mm}}
WV&WK&WB&ABK&ABB&ABV&AnzB&TW&Zahlencode \textcolor{red}{$\boldsymbol{Grundtext}$} Umschrift $|$"Ubersetzung(en)\\
1.&1.&1848.&1.&7262.&1.&4&380&40\_300\_30\_10 \textcolor{red}{\textcjheb{yl+sm}} MSLJ $|$Spr"uche///---"Uberschrift\\
2.&2.&1849.&5.&7266.&5.&4&375&300\_30\_40\_5 \textcolor{red}{\textcjheb{hml+s}} SLMH $|$(von) Salomo(s)\\
3.&3.&1850.&9.&7270.&9.&2&52&2\_50 \textcolor{red}{\textcjheb{nb}} BN $|$(ein) Sohn///---Versanfang\\
4.&4.&1851.&11.&7272.&11.&3&68&8\_20\_40 \textcolor{red}{\textcjheb{mk.h}} CKM $|$weiser\\
5.&5.&1852.&14.&7275.&14.&4&358&10\_300\_40\_8 \textcolor{red}{\textcjheb{.hm+sy}} JSMC $|$(er) erfreut\\
6.&6.&1853.&18.&7279.&18.&2&3&1\_2 \textcolor{red}{\textcjheb{b'}} AB $|$den Vater\\
7.&7.&1854.&20.&7281.&20.&3&58&6\_2\_50 \textcolor{red}{\textcjheb{nbw}} WBN $|$aber ein Sohn/und (ein) Sohn\\
8.&8.&1855.&23.&7284.&23.&4&120&20\_60\_10\_30 \textcolor{red}{\textcjheb{lysk}} KsJL $|$t"orichter\\
9.&9.&1856.&27.&7288.&27.&4&809&400\_6\_3\_400 \textcolor{red}{\textcjheb{tgwt}} TWGT $|$(ist) Kummer\\
10.&10.&1857.&31.&7292.&31.&3&47&1\_40\_6 \textcolor{red}{\textcjheb{wm'}} AMW $|$seiner Mutter/f"ur seine Mutter\\
\end{tabular}\medskip \\
Ende des Verses 10.1\\
Verse: 257, Buchstaben: 33, 33, 7294, Totalwerte: 2270, 2270, 519314\\
\\
Ein weiser Sohn erfreut den Vater, aber ein t"orichter Sohn ist seiner Mutter Kummer.\\
\newpage 
{\bf -- 10.2}\\
\medskip \\
\begin{tabular}{rrrrrrrrp{120mm}}
WV&WK&WB&ABK&ABB&ABV&AnzB&TW&Zahlencode \textcolor{red}{$\boldsymbol{Grundtext}$} Umschrift $|$"Ubersetzung(en)\\
1.&11.&1858.&34.&7295.&1.&2&31&30\_1 \textcolor{red}{\textcjheb{'l}} LA $|$nichts\\
2.&12.&1859.&36.&7297.&3.&6&132&10\_6\_70\_10\_30\_6 \textcolor{red}{\textcjheb{wly`wy}} JWaJLW $|$(sie (=es)) n"utzen\\
3.&13.&1860.&42.&7303.&9.&6&703&1\_6\_90\_200\_6\_400 \textcolor{red}{\textcjheb{twr.sw'}} AW"sRWT $|$Sch"atze\\
4.&14.&1861.&48.&7309.&15.&3&570&200\_300\_70 \textcolor{red}{\textcjheb{`+sr}} RSa $|$der Gesetzlosigkeit/des Unrechts\\
5.&15.&1862.&51.&7312.&18.&5&205&6\_90\_4\_100\_5 \textcolor{red}{\textcjheb{hqd.sw}} W"sDQH $|$aber Gerechtigkeit/und Gerechtigkeit\\
6.&16.&1863.&56.&7317.&23.&4&530&400\_90\_10\_30 \textcolor{red}{\textcjheb{ly.st}} T"sJL $|$(sie) (er)rettet\\
7.&17.&1864.&60.&7321.&27.&4&486&40\_40\_6\_400 \textcolor{red}{\textcjheb{twmm}} MMWT $|$von (=vor) (dem) Tod\\
\end{tabular}\medskip \\
Ende des Verses 10.2\\
Verse: 258, Buchstaben: 30, 63, 7324, Totalwerte: 2657, 4927, 521971\\
\\
Sch"atze der Gesetzlosigkeit n"utzen nichts, aber Gerechtigkeit errettet vom Tode.\\
\newpage 
{\bf -- 10.3}\\
\medskip \\
\begin{tabular}{rrrrrrrrp{120mm}}
WV&WK&WB&ABK&ABB&ABV&AnzB&TW&Zahlencode \textcolor{red}{$\boldsymbol{Grundtext}$} Umschrift $|$"Ubersetzung(en)\\
1.&18.&1865.&64.&7325.&1.&2&31&30\_1 \textcolor{red}{\textcjheb{'l}} LA $|$nicht\\
2.&19.&1866.&66.&7327.&3.&5&292&10\_200\_70\_10\_2 \textcolor{red}{\textcjheb{by`ry}} JRaJB $|$(er (=es)) l"asst hungern\\
3.&20.&1867.&71.&7332.&8.&4&26&10\_5\_6\_5 \textcolor{red}{\textcjheb{hwhy}} JHWH $|$Jahwe\\
4.&21.&1868.&75.&7336.&12.&3&430&50\_80\_300 \textcolor{red}{\textcjheb{+spn}} NPS $|$die Seele\\
5.&22.&1869.&78.&7339.&15.&4&204&90\_4\_10\_100 \textcolor{red}{\textcjheb{qyd.s}} "sDJQ $|$des Gerechten/(eines) Gerechten\\
6.&23.&1870.&82.&7343.&19.&4&417&6\_5\_6\_400 \textcolor{red}{\textcjheb{twhw}} WHWT $|$aber die Gier/und die Gier\\
7.&24.&1871.&86.&7347.&23.&5&620&200\_300\_70\_10\_40 \textcolor{red}{\textcjheb{my`+sr}} RSaJM $|$der Gesetzlosen/(der) Frevler\\
8.&25.&1872.&91.&7352.&28.&4&99&10\_5\_4\_80 \textcolor{red}{\textcjheb{pdhy}} JHDP $|$st"o"st er hinweg/er st"o"st zur"uck\\
\end{tabular}\medskip \\
Ende des Verses 10.3\\
Verse: 259, Buchstaben: 31, 94, 7355, Totalwerte: 2119, 7046, 524090\\
\\
Jahwe l"a"st die Seele des Gerechten nicht hungern, aber die Gier der Gesetzlosen st"o"st er hinweg.\\
\newpage 
{\bf -- 10.4}\\
\medskip \\
\begin{tabular}{rrrrrrrrp{120mm}}
WV&WK&WB&ABK&ABB&ABV&AnzB&TW&Zahlencode \textcolor{red}{$\boldsymbol{Grundtext}$} Umschrift $|$"Ubersetzung(en)\\
1.&26.&1873.&95.&7356.&1.&3&501&200\_1\_300 \textcolor{red}{\textcjheb{+s'r}} RAS $|$(es) wird arm/er (=es) ist darbend\\
2.&27.&1874.&98.&7359.&4.&3&375&70\_300\_5 \textcolor{red}{\textcjheb{h+s`}} aSH $|$wer schafft/(ein) Machender\\
3.&28.&1875.&101.&7362.&7.&2&100&20\_80 \textcolor{red}{\textcjheb{pk}} KP $|$(mit) Hand\\
4.&29.&1876.&103.&7364.&9.&4&255&200\_40\_10\_5 \textcolor{red}{\textcjheb{hymr}} RMJH $|$l"assiger\\
5.&30.&1877.&107.&7368.&13.&3&20&6\_10\_4 \textcolor{red}{\textcjheb{dyw}} WJD $|$aber die Hand/und die Hand\\
6.&31.&1878.&110.&7371.&16.&6&354&8\_200\_6\_90\_10\_40 \textcolor{red}{\textcjheb{my.swr.h}} CRW"sJM $|$der Flei"sigen\\
7.&32.&1879.&116.&7377.&22.&5&980&400\_70\_300\_10\_200 \textcolor{red}{\textcjheb{ry+s`t}} TaSJR $|$(sie) macht reich\\
\end{tabular}\medskip \\
Ende des Verses 10.4\\
Verse: 260, Buchstaben: 26, 120, 7381, Totalwerte: 2585, 9631, 526675\\
\\
Wer mit l"assiger Hand schafft, wird arm; aber die Hand der Flei"sigen macht reich.\\
\newpage 
{\bf -- 10.5}\\
\medskip \\
\begin{tabular}{rrrrrrrrp{120mm}}
WV&WK&WB&ABK&ABB&ABV&AnzB&TW&Zahlencode \textcolor{red}{$\boldsymbol{Grundtext}$} Umschrift $|$"Ubersetzung(en)\\
1.&33.&1880.&121.&7382.&1.&3&204&1\_3\_200 \textcolor{red}{\textcjheb{rg'}} AGR $|$wer einsammelt/Sammelnder\\
2.&34.&1881.&124.&7385.&4.&4&202&2\_100\_10\_90 \textcolor{red}{\textcjheb{.syqb}} BQJ"s $|$(ist) im Sommer\\
3.&35.&1882.&128.&7389.&8.&2&52&2\_50 \textcolor{red}{\textcjheb{nb}} BN $|$(ist) (ein) Sohn\\
4.&36.&1883.&130.&7391.&10.&5&400&40\_300\_20\_10\_30 \textcolor{red}{\textcjheb{lyk+sm}} MSKJL $|$einsichtsvoller/verst"andiger\\
5.&37.&1884.&135.&7396.&15.&4&294&50\_200\_4\_40 \textcolor{red}{\textcjheb{mdrn}} NRDM $|$wer im tiefen Schlaf liegt/(ein) tief Schlafender\\
6.&38.&1885.&139.&7400.&19.&5&402&2\_100\_90\_10\_200 \textcolor{red}{\textcjheb{ry.sqb}} BQ"sJR $|$zur Erntezeit/w"ahrend der Ernte\\
7.&39.&1886.&144.&7405.&24.&2&52&2\_50 \textcolor{red}{\textcjheb{nb}} BN $|$(ist) (ein) Sohn\\
8.&40.&1887.&146.&7407.&26.&4&352&40\_2\_10\_300 \textcolor{red}{\textcjheb{+sybm}} MBJS $|$der Schande bringt/Schande machender\\
\end{tabular}\medskip \\
Ende des Verses 10.5\\
Verse: 261, Buchstaben: 29, 149, 7410, Totalwerte: 1958, 11589, 528633\\
\\
Wer im Sommer einsammelt, ist ein einsichtsvoller Sohn; wer zur Erntezeit in tiefem Schlafe liegt, ist ein Sohn, der Schande bringt.\\
\newpage 
{\bf -- 10.6}\\
\medskip \\
\begin{tabular}{rrrrrrrrp{120mm}}
WV&WK&WB&ABK&ABB&ABV&AnzB&TW&Zahlencode \textcolor{red}{$\boldsymbol{Grundtext}$} Umschrift $|$"Ubersetzung(en)\\
1.&41.&1888.&150.&7411.&1.&5&628&2\_200\_20\_6\_400 \textcolor{red}{\textcjheb{twkrb}} BRKWT $|$Segnungen\\
2.&42.&1889.&155.&7416.&6.&4&531&30\_200\_1\_300 \textcolor{red}{\textcjheb{+s'rl}} LRAS $|$(werden zuteil) dem Haupt\\
3.&43.&1890.&159.&7420.&10.&4&204&90\_4\_10\_100 \textcolor{red}{\textcjheb{qyd.s}} "sDJQ $|$des Gerechten/gerechten\\
4.&44.&1891.&163.&7424.&14.&3&96&6\_80\_10 \textcolor{red}{\textcjheb{ypw}} WPJ $|$aber den Mund/und der Mund\\
5.&45.&1892.&166.&7427.&17.&5&620&200\_300\_70\_10\_40 \textcolor{red}{\textcjheb{my`+sr}} RSaJM $|$der Gesetzlosen/der B"osen\\
6.&46.&1893.&171.&7432.&22.&4&95&10\_20\_60\_5 \textcolor{red}{\textcjheb{hsky}} JKsH $|$(er) bedeckt\\
7.&47.&1894.&175.&7436.&26.&3&108&8\_40\_60 \textcolor{red}{\textcjheb{sm.h}} CMs $|$Gewalttat/Unrecht\\
\end{tabular}\medskip \\
Ende des Verses 10.6\\
Verse: 262, Buchstaben: 28, 177, 7438, Totalwerte: 2282, 13871, 530915\\
\\
Dem Haupte des Gerechten werden Segnungen zuteil, aber den Mund der Gesetzlosen bedeckt Gewalttat.\\
\newpage 
{\bf -- 10.7}\\
\medskip \\
\begin{tabular}{rrrrrrrrp{120mm}}
WV&WK&WB&ABK&ABB&ABV&AnzB&TW&Zahlencode \textcolor{red}{$\boldsymbol{Grundtext}$} Umschrift $|$"Ubersetzung(en)\\
1.&48.&1895.&178.&7439.&1.&3&227&7\_20\_200 \textcolor{red}{\textcjheb{rkz}} ZKR $|$das Ged"achtnis/(das) Gedenken\\
2.&49.&1896.&181.&7442.&4.&4&204&90\_4\_10\_100 \textcolor{red}{\textcjheb{qyd.s}} "sDJQ $|$(des) Gerechten\\
3.&50.&1897.&185.&7446.&8.&5&257&30\_2\_200\_20\_5 \textcolor{red}{\textcjheb{hkrbl}} LBRKH $|$ist zum Segen/(wird) zu Segnung\\
4.&51.&1898.&190.&7451.&13.&3&346&6\_300\_40 \textcolor{red}{\textcjheb{m+sw}} WSM $|$aber der Name/und der Name\\
5.&52.&1899.&193.&7454.&16.&5&620&200\_300\_70\_10\_40 \textcolor{red}{\textcjheb{my`+sr}} RSaJM $|$der Gesetzlosen/(der) Frevler\\
6.&53.&1900.&198.&7459.&21.&4&312&10\_200\_100\_2 \textcolor{red}{\textcjheb{bqry}} JRQB $|$verwest/(er) verfault\\
\end{tabular}\medskip \\
Ende des Verses 10.7\\
Verse: 263, Buchstaben: 24, 201, 7462, Totalwerte: 1966, 15837, 532881\\
\\
Das Ged"achtnis des Gerechten ist zum Segen, aber der Name der Gesetzlosen verwest.\\
\newpage 
{\bf -- 10.8}\\
\medskip \\
\begin{tabular}{rrrrrrrrp{120mm}}
WV&WK&WB&ABK&ABB&ABV&AnzB&TW&Zahlencode \textcolor{red}{$\boldsymbol{Grundtext}$} Umschrift $|$"Ubersetzung(en)\\
1.&54.&1901.&202.&7463.&1.&3&68&8\_20\_40 \textcolor{red}{\textcjheb{mk.h}} CKM $|$wer weisen\\
2.&55.&1902.&205.&7466.&4.&2&32&30\_2 \textcolor{red}{\textcjheb{bl}} LB $|$Herzens (ist)\\
3.&56.&1903.&207.&7468.&6.&3&118&10\_100\_8 \textcolor{red}{\textcjheb{.hqy}} JQC $|$((d)er) nimmt an\\
4.&57.&1904.&210.&7471.&9.&4&536&40\_90\_6\_400 \textcolor{red}{\textcjheb{tw.sm}} M"sWT $|$(die) Gebote\\
5.&58.&1905.&214.&7475.&13.&5&53&6\_1\_6\_10\_30 \textcolor{red}{\textcjheb{lyw'w}} WAWJL $|$aber ein n"arrischer/und ein T"orichter\\
6.&59.&1906.&219.&7480.&18.&5&830&300\_80\_400\_10\_40 \textcolor{red}{\textcjheb{mytp+s}} SPTJM $|$Schw"atzer/mit Lippen\\
7.&60.&1907.&224.&7485.&23.&4&51&10\_30\_2\_9 \textcolor{red}{\textcjheb{.tbly}} JLBt $|$(er) kommt zu Fall\\
\end{tabular}\medskip \\
Ende des Verses 10.8\\
Verse: 264, Buchstaben: 26, 227, 7488, Totalwerte: 1688, 17525, 534569\\
\\
Wer weisen Herzens ist, nimmt Gebote an; aber ein n"arrischer Schw"atzer kommt zu Fall.\\
\newpage 
{\bf -- 10.9}\\
\medskip \\
\begin{tabular}{rrrrrrrrp{120mm}}
WV&WK&WB&ABK&ABB&ABV&AnzB&TW&Zahlencode \textcolor{red}{$\boldsymbol{Grundtext}$} Umschrift $|$"Ubersetzung(en)\\
1.&61.&1908.&228.&7489.&1.&4&61&5\_6\_30\_20 \textcolor{red}{\textcjheb{klwh}} HWLK $|$wer wandelt/(ein) Wandelnder\\
2.&62.&1909.&232.&7493.&5.&3&442&2\_400\_40 \textcolor{red}{\textcjheb{mtb}} BTM $|$in Vollkommenheit/in der Lauterkeit\\
3.&63.&1910.&235.&7496.&8.&3&60&10\_30\_20 \textcolor{red}{\textcjheb{kly}} JLK $|$(er) wandelt\\
4.&64.&1911.&238.&7499.&11.&3&19&2\_9\_8 \textcolor{red}{\textcjheb{.h.tb}} BtC $|$sicher/in Sicherheit\\
5.&65.&1912.&241.&7502.&14.&5&516&6\_40\_70\_100\_300 \textcolor{red}{\textcjheb{+sq`mw}} WMaQS $|$wer aber kr"ummt/und ein Machender krumm\\
6.&66.&1913.&246.&7507.&19.&5&240&4\_200\_20\_10\_6 \textcolor{red}{\textcjheb{wykrd}} DRKJW $|$seine Wege\\
7.&67.&1914.&251.&7512.&24.&4&90&10\_6\_4\_70 \textcolor{red}{\textcjheb{`dwy}} JWDa $|$wird bekannt werden/(er) wird erkannt (werden)\\
\end{tabular}\medskip \\
Ende des Verses 10.9\\
Verse: 265, Buchstaben: 27, 254, 7515, Totalwerte: 1428, 18953, 535997\\
\\
Wer in Vollkommenheit wandelt, wandelt sicher; wer aber seine Wege kr"ummt, wird bekannt werden.\\
\newpage 
{\bf -- 10.10}\\
\medskip \\
\begin{tabular}{rrrrrrrrp{120mm}}
WV&WK&WB&ABK&ABB&ABV&AnzB&TW&Zahlencode \textcolor{red}{$\boldsymbol{Grundtext}$} Umschrift $|$"Ubersetzung(en)\\
1.&68.&1915.&255.&7516.&1.&3&390&100\_200\_90 \textcolor{red}{\textcjheb{.srq}} QR"s $|$wer zwinkt/ein Zusammenkneifender\\
2.&69.&1916.&258.&7519.&4.&3&130&70\_10\_50 \textcolor{red}{\textcjheb{ny`}} aJN $|$mit den Augen/das Auge\\
3.&70.&1917.&261.&7522.&7.&3&460&10\_400\_50 \textcolor{red}{\textcjheb{nty}} JTN $|$(er) verursacht\\
4.&71.&1918.&264.&7525.&10.&4&562&70\_90\_2\_400 \textcolor{red}{\textcjheb{tb.s`}} a"sBT $|$Kr"ankung/Schmerz\\
5.&72.&1919.&268.&7529.&14.&5&53&6\_1\_6\_10\_30 \textcolor{red}{\textcjheb{lyw'w}} WAWJL $|$und ein n"arrischer/und ein T"orichter\\
6.&73.&1920.&273.&7534.&19.&5&830&300\_80\_400\_10\_40 \textcolor{red}{\textcjheb{mytp+s}} SPTJM $|$Schw"atzer/der Lippen\\
7.&74.&1921.&278.&7539.&24.&4&51&10\_30\_2\_9 \textcolor{red}{\textcjheb{.tbly}} JLBt $|$(er) kommt zu Fall\\
\end{tabular}\medskip \\
Ende des Verses 10.10\\
Verse: 266, Buchstaben: 27, 281, 7542, Totalwerte: 2476, 21429, 538473\\
\\
Wer mit den Augen zwinkt, verursacht Kr"ankung; und ein n"arrischer Schw"atzer kommt zu Fall.\\
\newpage 
{\bf -- 10.11}\\
\medskip \\
\begin{tabular}{rrrrrrrrp{120mm}}
WV&WK&WB&ABK&ABB&ABV&AnzB&TW&Zahlencode \textcolor{red}{$\boldsymbol{Grundtext}$} Umschrift $|$"Ubersetzung(en)\\
1.&75.&1922.&282.&7543.&1.&4&346&40\_100\_6\_200 \textcolor{red}{\textcjheb{rwqm}} MQWR $|$ein Born/ein Quell\\
2.&76.&1923.&286.&7547.&5.&4&68&8\_10\_10\_40 \textcolor{red}{\textcjheb{myy.h}} CJJM $|$des Lebens\\
3.&77.&1924.&290.&7551.&9.&2&90&80\_10 \textcolor{red}{\textcjheb{yp}} PJ $|$(ist) der Mund\\
4.&78.&1925.&292.&7553.&11.&4&204&90\_4\_10\_100 \textcolor{red}{\textcjheb{qyd.s}} "sDJQ $|$(des) Gerechten\\
5.&79.&1926.&296.&7557.&15.&3&96&6\_80\_10 \textcolor{red}{\textcjheb{ypw}} WPJ $|$aber den Mund/und der Mund\\
6.&80.&1927.&299.&7560.&18.&5&620&200\_300\_70\_10\_40 \textcolor{red}{\textcjheb{my`+sr}} RSaJM $|$der Gesetzlosen/der B"osen\\
7.&81.&1928.&304.&7565.&23.&4&95&10\_20\_60\_5 \textcolor{red}{\textcjheb{hsky}} JKsH $|$(er) bedeckt\\
8.&82.&1929.&308.&7569.&27.&3&108&8\_40\_60 \textcolor{red}{\textcjheb{sm.h}} CMs $|$Gewalttat/Unrecht\\
\end{tabular}\medskip \\
Ende des Verses 10.11\\
Verse: 267, Buchstaben: 29, 310, 7571, Totalwerte: 1627, 23056, 540100\\
\\
Ein Born des Lebens ist der Mund des Gerechten, aber den Mund der Gesetzlosen bedeckt Gewalttat.\\
\newpage 
{\bf -- 10.12}\\
\medskip \\
\begin{tabular}{rrrrrrrrp{120mm}}
WV&WK&WB&ABK&ABB&ABV&AnzB&TW&Zahlencode \textcolor{red}{$\boldsymbol{Grundtext}$} Umschrift $|$"Ubersetzung(en)\\
1.&83.&1930.&311.&7572.&1.&4&356&300\_50\_1\_5 \textcolor{red}{\textcjheb{h'n+s}} SNAH $|$Hass\\
2.&84.&1931.&315.&7576.&5.&5&876&400\_70\_6\_200\_200 \textcolor{red}{\textcjheb{rrw`t}} TaWRR $|$(er) erregt\\
3.&85.&1932.&320.&7581.&10.&5&144&40\_4\_50\_10\_40 \textcolor{red}{\textcjheb{myndm}} MDNJM $|$Zwietracht/Streitigkeiten\\
4.&86.&1933.&325.&7586.&15.&3&106&6\_70\_30 \textcolor{red}{\textcjheb{l`w}} WaL $|$aber/und "uber\\
5.&87.&1934.&328.&7589.&18.&2&50&20\_30 \textcolor{red}{\textcjheb{lk}} KL $|$alle\\
6.&88.&1935.&330.&7591.&20.&5&500&80\_300\_70\_10\_40 \textcolor{red}{\textcjheb{my`+sp}} PSaJM $|$"Ubertretungen/Vergehen\\
7.&89.&1936.&335.&7596.&25.&4&485&400\_20\_60\_5 \textcolor{red}{\textcjheb{hskt}} TKsH $|$(sie (=es)) deckt zu\\
8.&90.&1937.&339.&7600.&29.&4&13&1\_5\_2\_5 \textcolor{red}{\textcjheb{hbh'}} AHBH $|$(die) Liebe\\
\end{tabular}\medskip \\
Ende des Verses 10.12\\
Verse: 268, Buchstaben: 32, 342, 7603, Totalwerte: 2530, 25586, 542630\\
\\
Ha"s erregt Zwietracht, aber Liebe deckt alle "Ubertretungen zu.\\
\newpage 
{\bf -- 10.13}\\
\medskip \\
\begin{tabular}{rrrrrrrrp{120mm}}
WV&WK&WB&ABK&ABB&ABV&AnzB&TW&Zahlencode \textcolor{red}{$\boldsymbol{Grundtext}$} Umschrift $|$"Ubersetzung(en)\\
1.&91.&1938.&343.&7604.&1.&5&792&2\_300\_80\_400\_10 \textcolor{red}{\textcjheb{ytp+sb}} BSPTJ $|$auf den Lippen\\
2.&92.&1939.&348.&7609.&6.&4&108&50\_2\_6\_50 \textcolor{red}{\textcjheb{nwbn}} NBWN $|$des Verst"andigen/des Klugen\\
3.&93.&1940.&352.&7613.&10.&4&531&400\_40\_90\_1 \textcolor{red}{\textcjheb{'.smt}} TM"sA $|$(sie (=es)) wird gefunden\\
4.&94.&1941.&356.&7617.&14.&4&73&8\_20\_40\_5 \textcolor{red}{\textcjheb{hmk.h}} CKMH $|$Weisheit\\
5.&95.&1942.&360.&7621.&18.&4&317&6\_300\_2\_9 \textcolor{red}{\textcjheb{.tb+sw}} WSBt $|$aber der Stock geb"uhrt/und eine Rute\\
6.&96.&1943.&364.&7625.&22.&3&39&30\_3\_6 \textcolor{red}{\textcjheb{wgl}} LGW $|$dem R"ucken/f"ur den R"ucken\\
7.&97.&1944.&367.&7628.&25.&3&268&8\_60\_200 \textcolor{red}{\textcjheb{rs.h}} CsR $|$des Un-/(des) Ermangelnden\\
8.&98.&1945.&370.&7631.&28.&2&32&30\_2 \textcolor{red}{\textcjheb{bl}} LB $|$verst"andigen/Herz\\
\end{tabular}\medskip \\
Ende des Verses 10.13\\
Verse: 269, Buchstaben: 29, 371, 7632, Totalwerte: 2160, 27746, 544790\\
\\
Auf den Lippen des Verst"andigen wird Weisheit gefunden; aber der Stock geb"uhrt dem R"ucken des Unverst"andigen.\\
\newpage 
{\bf -- 10.14}\\
\medskip \\
\begin{tabular}{rrrrrrrrp{120mm}}
WV&WK&WB&ABK&ABB&ABV&AnzB&TW&Zahlencode \textcolor{red}{$\boldsymbol{Grundtext}$} Umschrift $|$"Ubersetzung(en)\\
1.&99.&1946.&372.&7633.&1.&5&118&8\_20\_40\_10\_40 \textcolor{red}{\textcjheb{mymk.h}} CKMJM $|$die Weisen/Weise\\
2.&100.&1947.&377.&7638.&6.&5&236&10\_90\_80\_50\_6 \textcolor{red}{\textcjheb{wnp.sy}} J"sPNW $|$bewahren auf/(sie) bergen\\
3.&101.&1948.&382.&7643.&11.&3&474&4\_70\_400 \textcolor{red}{\textcjheb{t`d}} DaT $|$Erkenntnis\\
4.&102.&1949.&385.&7646.&14.&3&96&6\_80\_10 \textcolor{red}{\textcjheb{ypw}} WPJ $|$aber der Mund/und der Mund\\
5.&103.&1950.&388.&7649.&17.&4&47&1\_6\_10\_30 \textcolor{red}{\textcjheb{lyw'}} AWJL $|$des Narren/eines Toren\\
6.&104.&1951.&392.&7653.&21.&4&453&40\_8\_400\_5 \textcolor{red}{\textcjheb{ht.hm}} MCTH $|$ist Ungl"ucksfall/(ist) Verderben\\
7.&105.&1952.&396.&7657.&25.&4&307&100\_200\_2\_5 \textcolor{red}{\textcjheb{hbrq}} QRBH $|$drohender/nahe(s)\\
\end{tabular}\medskip \\
Ende des Verses 10.14\\
Verse: 270, Buchstaben: 28, 399, 7660, Totalwerte: 1731, 29477, 546521\\
\\
Die Weisen bewahren Erkenntnis auf, aber der Mund des Narren ist drohender Ungl"ucksfall.\\
\newpage 
{\bf -- 10.15}\\
\medskip \\
\begin{tabular}{rrrrrrrrp{120mm}}
WV&WK&WB&ABK&ABB&ABV&AnzB&TW&Zahlencode \textcolor{red}{$\boldsymbol{Grundtext}$} Umschrift $|$"Ubersetzung(en)\\
1.&106.&1953.&400.&7661.&1.&3&61&5\_6\_50 \textcolor{red}{\textcjheb{nwh}} HWN $|$der Wohlstand/(das) Verm"ogen\\
2.&107.&1954.&403.&7664.&4.&4&580&70\_300\_10\_200 \textcolor{red}{\textcjheb{ry+s`}} aSJR $|$des Reichen/(eines) Reichen\\
3.&108.&1955.&407.&7668.&8.&4&710&100\_200\_10\_400 \textcolor{red}{\textcjheb{tyrq}} QRJT $|$(ist) (die) Stadt\\
4.&109.&1956.&411.&7672.&12.&3&83&70\_7\_6 \textcolor{red}{\textcjheb{wz`}} aZW $|$seine feste/seiner Macht\\
5.&110.&1957.&414.&7675.&15.&4&848&40\_8\_400\_400 \textcolor{red}{\textcjheb{tt.hm}} MCTT $|$der Ungl"ucksfall/der Untergang\\
6.&111.&1958.&418.&7679.&19.&4&84&4\_30\_10\_40 \textcolor{red}{\textcjheb{myld}} DLJM $|$der Geringen/der Armen\\
7.&112.&1959.&422.&7683.&23.&4&550&200\_10\_300\_40 \textcolor{red}{\textcjheb{m+syr}} RJSM $|$(ist) ihre Armut\\
\end{tabular}\medskip \\
Ende des Verses 10.15\\
Verse: 271, Buchstaben: 26, 425, 7686, Totalwerte: 2916, 32393, 549437\\
\\
Der Wohlstand des Reichen ist seine feste Stadt, der Ungl"ucksfall der Geringen ihre Armut.\\
\newpage 
{\bf -- 10.16}\\
\medskip \\
\begin{tabular}{rrrrrrrrp{120mm}}
WV&WK&WB&ABK&ABB&ABV&AnzB&TW&Zahlencode \textcolor{red}{$\boldsymbol{Grundtext}$} Umschrift $|$"Ubersetzung(en)\\
1.&113.&1960.&426.&7687.&1.&4&580&80\_70\_30\_400 \textcolor{red}{\textcjheb{tl`p}} PaLT $|$der Erwerb\\
2.&114.&1961.&430.&7691.&5.&4&204&90\_4\_10\_100 \textcolor{red}{\textcjheb{qyd.s}} "sDJQ $|$(des) Gerechten\\
3.&115.&1962.&434.&7695.&9.&5&98&30\_8\_10\_10\_40 \textcolor{red}{\textcjheb{myy.hl}} LCJJM $|$gereicht zum Leben/ist f"ur das Leben\\
4.&116.&1963.&439.&7700.&14.&5&809&400\_2\_6\_1\_400 \textcolor{red}{\textcjheb{t'wbt}} TBWAT $|$der Ertrag\\
5.&117.&1964.&444.&7705.&19.&3&570&200\_300\_70 \textcolor{red}{\textcjheb{`+sr}} RSa $|$des Gesetzlosen/(des) Frevlers\\
6.&118.&1965.&447.&7708.&22.&5&448&30\_8\_9\_1\_400 \textcolor{red}{\textcjheb{t'.t.hl}} LCtAT $|$zur S"unde/f"ur die S"unde\\
\end{tabular}\medskip \\
Ende des Verses 10.16\\
Verse: 272, Buchstaben: 26, 451, 7712, Totalwerte: 2709, 35102, 552146\\
\\
Der Erwerb des Gerechten gereicht zum Leben, der Ertrag des Gesetzlosen zur S"unde.\\
\newpage 
{\bf -- 10.17}\\
\medskip \\
\begin{tabular}{rrrrrrrrp{120mm}}
WV&WK&WB&ABK&ABB&ABV&AnzB&TW&Zahlencode \textcolor{red}{$\boldsymbol{Grundtext}$} Umschrift $|$"Ubersetzung(en)\\
1.&119.&1966.&452.&7713.&1.&3&209&1\_200\_8 \textcolor{red}{\textcjheb{.hr'}} ARC $|$es ist der Pfad/(den) Pfad\\
2.&120.&1967.&455.&7716.&4.&5&98&30\_8\_10\_10\_40 \textcolor{red}{\textcjheb{myy.hl}} LCJJM $|$zum Leben\\
3.&121.&1968.&460.&7721.&9.&4&546&300\_6\_40\_200 \textcolor{red}{\textcjheb{rmw+s}} SWMR $|$wenn einer beachtet/(geht) der Beachtende\\
4.&122.&1969.&464.&7725.&13.&4&306&40\_6\_60\_200 \textcolor{red}{\textcjheb{rswm}} MWsR $|$Unterweisung/Z"uchtigung\\
5.&123.&1970.&468.&7729.&17.&5&91&6\_70\_6\_7\_2 \textcolor{red}{\textcjheb{bzw`w}} WaWZB $|$war aber unbeachtet l"asst/und ein Verlassender\\
6.&124.&1971.&473.&7734.&22.&5&834&400\_6\_20\_8\_400 \textcolor{red}{\textcjheb{t.hkwt}} TWKCT $|$Zucht/Zurechtweisung\\
7.&125.&1972.&478.&7739.&27.&4&515&40\_400\_70\_5 \textcolor{red}{\textcjheb{h`tm}} MTaH $|$geht irre/(ist) irref"uhrend(er)\\
\end{tabular}\medskip \\
Ende des Verses 10.17\\
Verse: 273, Buchstaben: 30, 481, 7742, Totalwerte: 2599, 37701, 554745\\
\\
Es ist der Pfad zum Leben, wenn einer Unterweisung beachtet; wer aber Zucht unbeachtet l"a"st, geht irre.\\
\newpage 
{\bf -- 10.18}\\
\medskip \\
\begin{tabular}{rrrrrrrrp{120mm}}
WV&WK&WB&ABK&ABB&ABV&AnzB&TW&Zahlencode \textcolor{red}{$\boldsymbol{Grundtext}$} Umschrift $|$"Ubersetzung(en)\\
1.&126.&1973.&482.&7743.&1.&4&125&40\_20\_60\_5 \textcolor{red}{\textcjheb{hskm}} MKsH $|$wer verbirgt/ein Verbergender\\
2.&127.&1974.&486.&7747.&5.&4&356&300\_50\_1\_5 \textcolor{red}{\textcjheb{h'n+s}} SNAH $|$Hass\\
3.&128.&1975.&490.&7751.&9.&4&790&300\_80\_400\_10 \textcolor{red}{\textcjheb{ytp+s}} SPTJ $|$(hat) Lippen\\
4.&129.&1976.&494.&7755.&13.&3&600&300\_100\_200 \textcolor{red}{\textcjheb{rq+s}} SQR $|$(der) L"uge(n)\\
5.&130.&1977.&497.&7758.&16.&5&143&6\_40\_6\_90\_1 \textcolor{red}{\textcjheb{'.swmw}} WMW"sA $|$und wer ausbringt/und ein Hervorbringender\\
6.&131.&1978.&502.&7763.&21.&3&11&4\_2\_5 \textcolor{red}{\textcjheb{hbd}} DBH $|$Verleumdung\\
7.&132.&1979.&505.&7766.&24.&3&12&5\_6\_1 \textcolor{red}{\textcjheb{'wh}} HWA $|$((d)er) (ist)\\
8.&133.&1980.&508.&7769.&27.&4&120&20\_60\_10\_30 \textcolor{red}{\textcjheb{lysk}} KsJL $|$(ein) Tor\\
\end{tabular}\medskip \\
Ende des Verses 10.18\\
Verse: 274, Buchstaben: 30, 511, 7772, Totalwerte: 2157, 39858, 556902\\
\\
Wer Ha"s verbirgt, hat L"ugenlippen; und wer Verleumdung ausbringt, ist ein Tor.\\
\newpage 
{\bf -- 10.19}\\
\medskip \\
\begin{tabular}{rrrrrrrrp{120mm}}
WV&WK&WB&ABK&ABB&ABV&AnzB&TW&Zahlencode \textcolor{red}{$\boldsymbol{Grundtext}$} Umschrift $|$"Ubersetzung(en)\\
1.&134.&1981.&512.&7773.&1.&3&204&2\_200\_2 \textcolor{red}{\textcjheb{brb}} BRB $|$bei der Menge/bei vielen\\
2.&135.&1982.&515.&7776.&4.&5&256&4\_2\_200\_10\_40 \textcolor{red}{\textcjheb{myrbd}} DBRJM $|$der Worte/Reden\\
3.&136.&1983.&520.&7781.&9.&2&31&30\_1 \textcolor{red}{\textcjheb{'l}} LA $|$nicht\\
4.&137.&1984.&522.&7783.&11.&4&52&10\_8\_4\_30 \textcolor{red}{\textcjheb{ld.hy}} JCDL $|$fehlt/er (=es) bleibt aus\\
5.&138.&1985.&526.&7787.&15.&3&450&80\_300\_70 \textcolor{red}{\textcjheb{`+sp}} PSa $|$"Ubertretung/Verfehlung\\
6.&139.&1986.&529.&7790.&18.&4&334&6\_8\_300\_20 \textcolor{red}{\textcjheb{k+s.hw}} WCSK $|$wer aber zur"uckh"alt/und ein Zur"uckhaltender\\
7.&140.&1987.&533.&7794.&22.&5&796&300\_80\_400\_10\_6 \textcolor{red}{\textcjheb{wytp+s}} SPTJW $|$seine Lippen\\
8.&141.&1988.&538.&7799.&27.&5&400&40\_300\_20\_10\_30 \textcolor{red}{\textcjheb{lyk+sm}} MSKJL $|$ist einsichtsvoll/(ist ein) Verst"andiger\\
\end{tabular}\medskip \\
Ende des Verses 10.19\\
Verse: 275, Buchstaben: 31, 542, 7803, Totalwerte: 2523, 42381, 559425\\
\\
Bei der Menge der Worte fehlt "Ubertretung nicht; wer aber seine Lippen zur"uckh"alt, ist einsichtsvoll.\\
\newpage 
{\bf -- 10.20}\\
\medskip \\
\begin{tabular}{rrrrrrrrp{120mm}}
WV&WK&WB&ABK&ABB&ABV&AnzB&TW&Zahlencode \textcolor{red}{$\boldsymbol{Grundtext}$} Umschrift $|$"Ubersetzung(en)\\
1.&142.&1989.&543.&7804.&1.&3&160&20\_60\_80 \textcolor{red}{\textcjheb{psk}} KsP $|$Silber\\
2.&143.&1990.&546.&7807.&4.&4&260&50\_2\_8\_200 \textcolor{red}{\textcjheb{r.hbn}} NBCR $|$auserlesenes ist/erlesenem\\
3.&144.&1991.&550.&7811.&8.&4&386&30\_300\_6\_50 \textcolor{red}{\textcjheb{nw+sl}} LSWN $|$(gleicht) die Zunge\\
4.&145.&1992.&554.&7815.&12.&4&204&90\_4\_10\_100 \textcolor{red}{\textcjheb{qyd.s}} "sDJQ $|$(des) Gerechten\\
5.&146.&1993.&558.&7819.&16.&2&32&30\_2 \textcolor{red}{\textcjheb{bl}} LB $|$der Verstand/das Herz\\
6.&147.&1994.&560.&7821.&18.&5&620&200\_300\_70\_10\_40 \textcolor{red}{\textcjheb{my`+sr}} RSaJM $|$der Gesetzlosen/(der) Frevler\\
7.&148.&1995.&565.&7826.&23.&4&139&20\_40\_70\_9 \textcolor{red}{\textcjheb{.t`mk}} KMat $|$(wie) ist (es) wenig (wert)\\
\end{tabular}\medskip \\
Ende des Verses 10.20\\
Verse: 276, Buchstaben: 26, 568, 7829, Totalwerte: 1801, 44182, 561226\\
\\
Die Zunge des Gerechten ist auserlesenes Silber, der Verstand der Gesetzlosen ist wenig wert.\\
\newpage 
{\bf -- 10.21}\\
\medskip \\
\begin{tabular}{rrrrrrrrp{120mm}}
WV&WK&WB&ABK&ABB&ABV&AnzB&TW&Zahlencode \textcolor{red}{$\boldsymbol{Grundtext}$} Umschrift $|$"Ubersetzung(en)\\
1.&149.&1996.&569.&7830.&1.&4&790&300\_80\_400\_10 \textcolor{red}{\textcjheb{ytp+s}} SPTJ $|$die Lippen\\
2.&150.&1997.&573.&7834.&5.&4&204&90\_4\_10\_100 \textcolor{red}{\textcjheb{qyd.s}} "sDJQ $|$(des) Gerechten\\
3.&151.&1998.&577.&7838.&9.&4&286&10\_200\_70\_6 \textcolor{red}{\textcjheb{w`ry}} JRaW $|$(sie) weiden\\
4.&152.&1999.&581.&7842.&13.&4&252&200\_2\_10\_40 \textcolor{red}{\textcjheb{mybr}} RBJM $|$viele\\
5.&153.&2000.&585.&7846.&17.&7&103&6\_1\_6\_10\_30\_10\_40 \textcolor{red}{\textcjheb{mylyw'w}} WAWJLJM $|$aber die Narren/und die Toren\\
6.&154.&2001.&592.&7853.&24.&4&270&2\_8\_60\_200 \textcolor{red}{\textcjheb{rs.hb}} BCsR $|$durch Mangel\\
7.&155.&2002.&596.&7857.&28.&2&32&30\_2 \textcolor{red}{\textcjheb{bl}} LB $|$(an) Verstand\\
8.&156.&2003.&598.&7859.&30.&5&462&10\_40\_6\_400\_6 \textcolor{red}{\textcjheb{wtwmy}} JMWTW $|$(sie) sterben\\
\end{tabular}\medskip \\
Ende des Verses 10.21\\
Verse: 277, Buchstaben: 34, 602, 7863, Totalwerte: 2399, 46581, 563625\\
\\
Die Lippen des Gerechten weiden viele, aber die Narren sterben durch Mangel an Verstand.\\
\newpage 
{\bf -- 10.22}\\
\medskip \\
\begin{tabular}{rrrrrrrrp{120mm}}
WV&WK&WB&ABK&ABB&ABV&AnzB&TW&Zahlencode \textcolor{red}{$\boldsymbol{Grundtext}$} Umschrift $|$"Ubersetzung(en)\\
1.&157.&2004.&603.&7864.&1.&4&622&2\_200\_20\_400 \textcolor{red}{\textcjheb{tkrb}} BRKT $|$der Segen/die Segnung\\
2.&158.&2005.&607.&7868.&5.&4&26&10\_5\_6\_5 \textcolor{red}{\textcjheb{hwhy}} JHWH $|$Jahwe(s)\\
3.&159.&2006.&611.&7872.&9.&3&16&5\_10\_1 \textcolor{red}{\textcjheb{'yh}} HJA $|$er/sie\\
4.&160.&2007.&614.&7875.&12.&5&980&400\_70\_300\_10\_200 \textcolor{red}{\textcjheb{ry+s`t}} TaSJR $|$(sie) macht reich\\
5.&161.&2008.&619.&7880.&17.&3&37&6\_30\_1 \textcolor{red}{\textcjheb{'lw}} WLA $|$und nicht(s)\\
6.&162.&2009.&622.&7883.&20.&4&156&10\_6\_60\_80 \textcolor{red}{\textcjheb{pswy}} JWsP $|$(er (=es))) f"ugt hinzu\\
7.&163.&2010.&626.&7887.&24.&3&162&70\_90\_2 \textcolor{red}{\textcjheb{b.s`}} a"sB $|$Anstrengung/(eigene) M"uhe\\
8.&164.&2011.&629.&7890.&27.&3&115&70\_40\_5 \textcolor{red}{\textcjheb{hm`}} aMH $|$neben ihm/zu ihr\\
\end{tabular}\medskip \\
Ende des Verses 10.22\\
Verse: 278, Buchstaben: 29, 631, 7892, Totalwerte: 2114, 48695, 565739\\
\\
Der Segen Jahwes, er macht reich, und Anstrengung f"ugt neben ihm nichts hinzu.\\
\newpage 
{\bf -- 10.23}\\
\medskip \\
\begin{tabular}{rrrrrrrrp{120mm}}
WV&WK&WB&ABK&ABB&ABV&AnzB&TW&Zahlencode \textcolor{red}{$\boldsymbol{Grundtext}$} Umschrift $|$"Ubersetzung(en)\\
1.&165.&2012.&632.&7893.&1.&5&434&20\_300\_8\_6\_100 \textcolor{red}{\textcjheb{qw.h+sk}} KSCWQ $|$wie ein Spiel/wie Scherz\\
2.&166.&2013.&637.&7898.&6.&5&150&30\_20\_60\_10\_30 \textcolor{red}{\textcjheb{lyskl}} LKsJL $|$dem Toren/(ist) f"ur den Toren\\
3.&167.&2014.&642.&7903.&11.&4&776&70\_300\_6\_400 \textcolor{red}{\textcjheb{tw+s`}} aSWT $|$ist es zu ver"uben/(ein) Tun\\
4.&168.&2015.&646.&7907.&15.&3&52&7\_40\_5 \textcolor{red}{\textcjheb{hmz}} ZMH $|$Schandtat\\
5.&169.&2016.&649.&7910.&18.&5&79&6\_8\_20\_40\_5 \textcolor{red}{\textcjheb{hmk.hw}} WCKMH $|$und Weisheit\\
6.&170.&2017.&654.&7915.&23.&4&341&30\_1\_10\_300 \textcolor{red}{\textcjheb{+sy'l}} LAJS $|$zu "uben dem Mann/(geh"ort zu) dem Mann\\
7.&171.&2018.&658.&7919.&27.&5&463&400\_2\_6\_50\_5 \textcolor{red}{\textcjheb{hnwbt}} TBWNH $|$verst"andigen/(mit) Einsicht\\
\end{tabular}\medskip \\
Ende des Verses 10.23\\
Verse: 279, Buchstaben: 31, 662, 7923, Totalwerte: 2295, 50990, 568034\\
\\
Dem Toren ist es wie ein Spiel, Schandtat zu ver"uben, und Weisheit zu "uben dem verst"andigen Manne.\\
\newpage 
{\bf -- 10.24}\\
\medskip \\
\begin{tabular}{rrrrrrrrp{120mm}}
WV&WK&WB&ABK&ABB&ABV&AnzB&TW&Zahlencode \textcolor{red}{$\boldsymbol{Grundtext}$} Umschrift $|$"Ubersetzung(en)\\
1.&172.&2019.&663.&7924.&1.&5&649&40\_3\_6\_200\_400 \textcolor{red}{\textcjheb{trwgm}} MGWRT $|$wovor bangt/die Bef"urchtung\\
2.&173.&2020.&668.&7929.&6.&3&570&200\_300\_70 \textcolor{red}{\textcjheb{`+sr}} RSa $|$dem Gesetzlosen/(des) Frevlers\\
3.&174.&2021.&671.&7932.&9.&3&16&5\_10\_1 \textcolor{red}{\textcjheb{'yh}} HJA $|$das/sie\\
4.&175.&2022.&674.&7935.&12.&6&465&400\_2\_6\_1\_50\_6 \textcolor{red}{\textcjheb{wn'wbt}} TBWANW $|$wird kommen "uber ihn/sie "uberkommt ihn\\
5.&176.&2023.&680.&7941.&18.&5&813&6\_400\_1\_6\_400 \textcolor{red}{\textcjheb{tw'tw}} WTAWT $|$und das Begehren/und einen Wunsch\\
6.&177.&2024.&685.&7946.&23.&6&254&90\_4\_10\_100\_10\_40 \textcolor{red}{\textcjheb{myqyd.s}} "sDJQJM $|$der Gerechten\\
7.&178.&2025.&691.&7952.&29.&3&460&10\_400\_50 \textcolor{red}{\textcjheb{nty}} JTN $|$wird gew"ahrt/er wird erf"ullen\\
\end{tabular}\medskip \\
Ende des Verses 10.24\\
Verse: 280, Buchstaben: 31, 693, 7954, Totalwerte: 3227, 54217, 571261\\
\\
Wovor dem Gesetzlosen bangt, das wird "uber ihn kommen, und das Begehren der Gerechten wird gew"ahrt.\\
\newpage 
{\bf -- 10.25}\\
\medskip \\
\begin{tabular}{rrrrrrrrp{120mm}}
WV&WK&WB&ABK&ABB&ABV&AnzB&TW&Zahlencode \textcolor{red}{$\boldsymbol{Grundtext}$} Umschrift $|$"Ubersetzung(en)\\
1.&179.&2026.&694.&7955.&1.&5&298&20\_70\_2\_6\_200 \textcolor{red}{\textcjheb{rwb`k}} KaBWR $|$wie daherf"ahrt/wie ein Vor"ubergehen\\
2.&180.&2027.&699.&7960.&6.&4&151&60\_6\_80\_5 \textcolor{red}{\textcjheb{hpws}} sWPH $|$(ein) Sturmwind(s)\\
3.&181.&2028.&703.&7964.&10.&4&67&6\_1\_10\_50 \textcolor{red}{\textcjheb{ny'w}} WAJN $|$so ist nicht mehr/und nicht ist mehr da\\
4.&182.&2029.&707.&7968.&14.&3&570&200\_300\_70 \textcolor{red}{\textcjheb{`+sr}} RSa $|$der Gesetzlose/(der) Frevler\\
5.&183.&2030.&710.&7971.&17.&5&210&6\_90\_4\_10\_100 \textcolor{red}{\textcjheb{qyd.sw}} W"sDJQ $|$aber der Gerechte/und der Gerechte\\
6.&184.&2031.&715.&7976.&22.&4&80&10\_60\_6\_4 \textcolor{red}{\textcjheb{dwsy}} JsWD $|$ist ein fester Grund/(ist) Fundament\\
7.&185.&2032.&719.&7980.&26.&4&146&70\_6\_30\_40 \textcolor{red}{\textcjheb{mlw`}} aWLM $|$ewig/f"ur immer\\
\end{tabular}\medskip \\
Ende des Verses 10.25\\
Verse: 281, Buchstaben: 29, 722, 7983, Totalwerte: 1522, 55739, 572783\\
\\
Wie ein Sturmwind daherf"ahrt, so ist der Gesetzlose nicht mehr; aber der Gerechte ist ein ewig fester Grund.\\
\newpage 
{\bf -- 10.26}\\
\medskip \\
\begin{tabular}{rrrrrrrrp{120mm}}
WV&WK&WB&ABK&ABB&ABV&AnzB&TW&Zahlencode \textcolor{red}{$\boldsymbol{Grundtext}$} Umschrift $|$"Ubersetzung(en)\\
1.&186.&2033.&723.&7984.&1.&4&158&20\_8\_40\_90 \textcolor{red}{\textcjheb{.sm.hk}} KCM"s $|$wie (der) Essig\\
2.&187.&2034.&727.&7988.&5.&5&430&30\_300\_50\_10\_40 \textcolor{red}{\textcjheb{myn+sl}} LSNJM $|$den Z"ahnen/f"ur die Zahnreihen\\
3.&188.&2035.&732.&7993.&10.&5&446&6\_20\_70\_300\_50 \textcolor{red}{\textcjheb{n+s`kw}} WKaSN $|$und wie der Rauch\\
4.&189.&2036.&737.&7998.&15.&6&210&30\_70\_10\_50\_10\_40 \textcolor{red}{\textcjheb{myny`l}} LaJNJM $|$den Augen/f"ur die Augen\\
5.&190.&2037.&743.&8004.&21.&2&70&20\_50 \textcolor{red}{\textcjheb{nk}} KN $|$so (ist)\\
6.&191.&2038.&745.&8006.&23.&4&195&5\_70\_90\_30 \textcolor{red}{\textcjheb{l.s`h}} Ha"sL $|$der Faule denen/der Faulpelz \\
7.&192.&2039.&749.&8010.&27.&6&384&30\_300\_30\_8\_10\_6 \textcolor{red}{\textcjheb{wy.hl+sl}} LSLCJW $|$die ihn senden/f"ur seine Auftraggeber\\
\end{tabular}\medskip \\
Ende des Verses 10.26\\
Verse: 282, Buchstaben: 32, 754, 8015, Totalwerte: 1893, 57632, 574676\\
\\
Wie der Essig den Z"ahnen, und wie der Rauch den Augen, so ist der Faule denen, die ihn senden.\\
\newpage 
{\bf -- 10.27}\\
\medskip \\
\begin{tabular}{rrrrrrrrp{120mm}}
WV&WK&WB&ABK&ABB&ABV&AnzB&TW&Zahlencode \textcolor{red}{$\boldsymbol{Grundtext}$} Umschrift $|$"Ubersetzung(en)\\
1.&193.&2040.&755.&8016.&1.&4&611&10\_200\_1\_400 \textcolor{red}{\textcjheb{t'ry}} JRAT $|$(die) (Ehr)Furcht\\
2.&194.&2041.&759.&8020.&5.&4&26&10\_5\_6\_5 \textcolor{red}{\textcjheb{hwhy}} JHWH $|$(vor) Jahwe(s)\\
3.&195.&2042.&763.&8024.&9.&5&556&400\_6\_60\_10\_80 \textcolor{red}{\textcjheb{pyswt}} TWsJP $|$(sie) (ver)mehrt\\
4.&196.&2043.&768.&8029.&14.&4&100&10\_40\_10\_40 \textcolor{red}{\textcjheb{mymy}} JMJM $|$(die) Tage\\
5.&197.&2044.&772.&8033.&18.&5&762&6\_300\_50\_6\_400 \textcolor{red}{\textcjheb{twn+sw}} WSNWT $|$aber die Jahre/und die Jahre\\
6.&198.&2045.&777.&8038.&23.&5&620&200\_300\_70\_10\_40 \textcolor{red}{\textcjheb{my`+sr}} RSaJM $|$der Gesetzlosen/(der) Frevler\\
7.&199.&2046.&782.&8043.&28.&6&845&400\_100\_90\_200\_50\_5 \textcolor{red}{\textcjheb{hnr.sqt}} TQ"sRNH $|$werden verk"urzt/(sie) werden kurz\\
\end{tabular}\medskip \\
Ende des Verses 10.27\\
Verse: 283, Buchstaben: 33, 787, 8048, Totalwerte: 3520, 61152, 578196\\
\\
Die Furcht Jahwes mehrt die Tage, aber die Jahre der Gesetzlosen werden verk"urzt.\\
\newpage 
{\bf -- 10.28}\\
\medskip \\
\begin{tabular}{rrrrrrrrp{120mm}}
WV&WK&WB&ABK&ABB&ABV&AnzB&TW&Zahlencode \textcolor{red}{$\boldsymbol{Grundtext}$} Umschrift $|$"Ubersetzung(en)\\
1.&200.&2047.&788.&8049.&1.&5&844&400\_6\_8\_30\_400 \textcolor{red}{\textcjheb{tl.hwt}} TWCLT $|$das Harren/die Hoffnung\\
2.&201.&2048.&793.&8054.&6.&6&254&90\_4\_10\_100\_10\_40 \textcolor{red}{\textcjheb{myqyd.s}} "sDJQJM $|$der Gerechten\\
3.&202.&2049.&799.&8060.&12.&4&353&300\_40\_8\_5 \textcolor{red}{\textcjheb{h.hm+s}} SMCH $|$wird Freude/(bringt) Freude\\
4.&203.&2050.&803.&8064.&16.&5&912&6\_400\_100\_6\_400 \textcolor{red}{\textcjheb{twqtw}} WTQWT $|$aber die Hoffnung/und die Hoffnung\\
5.&204.&2051.&808.&8069.&21.&5&620&200\_300\_70\_10\_40 \textcolor{red}{\textcjheb{my`+sr}} RSaJM $|$der Gesetzlosen/(der) Frevler\\
6.&205.&2052.&813.&8074.&26.&4&407&400\_1\_2\_4 \textcolor{red}{\textcjheb{db't}} TABD $|$wird zunichte/(sie) geht unter\\
\end{tabular}\medskip \\
Ende des Verses 10.28\\
Verse: 284, Buchstaben: 29, 816, 8077, Totalwerte: 3390, 64542, 581586\\
\\
Das Harren der Gerechten wird Freude, aber die Hoffnung der Gesetzlosen wird zunichte.\\
\newpage 
{\bf -- 10.29}\\
\medskip \\
\begin{tabular}{rrrrrrrrp{120mm}}
WV&WK&WB&ABK&ABB&ABV&AnzB&TW&Zahlencode \textcolor{red}{$\boldsymbol{Grundtext}$} Umschrift $|$"Ubersetzung(en)\\
1.&206.&2053.&817.&8078.&1.&4&123&40\_70\_6\_7 \textcolor{red}{\textcjheb{zw`m}} MaWZ $|$eine Feste/Schutzwehr\\
2.&207.&2054.&821.&8082.&5.&3&470&30\_400\_40 \textcolor{red}{\textcjheb{mtl}} LTM $|$f"ur die Vollkommenheit/dem Lauteren\\
3.&208.&2055.&824.&8085.&8.&3&224&4\_200\_20 \textcolor{red}{\textcjheb{krd}} DRK $|$(ist) der Weg\\
4.&209.&2056.&827.&8088.&11.&4&26&10\_5\_6\_5 \textcolor{red}{\textcjheb{hwhy}} JHWH $|$Jahwe(s)\\
5.&210.&2057.&831.&8092.&15.&5&459&6\_40\_8\_400\_5 \textcolor{red}{\textcjheb{ht.hmw}} WMCTH $|$aber Untergang/und Untergang\\
6.&211.&2058.&836.&8097.&20.&5&220&30\_80\_70\_30\_10 \textcolor{red}{\textcjheb{yl`pl}} LPaLJ $|$f"ur die welche tun/f"ur Tuende\\
7.&212.&2059.&841.&8102.&25.&3&57&1\_6\_50 \textcolor{red}{\textcjheb{nw'}} AWN $|$Frevel/"Ubel\\
\end{tabular}\medskip \\
Ende des Verses 10.29\\
Verse: 285, Buchstaben: 27, 843, 8104, Totalwerte: 1579, 66121, 583165\\
\\
Der Weg Jahwes ist eine Feste f"ur die Vollkommenheit, aber Untergang f"ur die, welche Frevel tun.\\
\newpage 
{\bf -- 10.30}\\
\medskip \\
\begin{tabular}{rrrrrrrrp{120mm}}
WV&WK&WB&ABK&ABB&ABV&AnzB&TW&Zahlencode \textcolor{red}{$\boldsymbol{Grundtext}$} Umschrift $|$"Ubersetzung(en)\\
1.&213.&2060.&844.&8105.&1.&4&204&90\_4\_10\_100 \textcolor{red}{\textcjheb{qyd.s}} "sDJQ $|$der Gerechte/(ein) Gerechter\\
2.&214.&2061.&848.&8109.&5.&5&176&30\_70\_6\_30\_40 \textcolor{red}{\textcjheb{mlw`l}} LaWLM $|$in Ewigkeit\\
3.&215.&2062.&853.&8114.&10.&2&32&2\_30 \textcolor{red}{\textcjheb{lb}} BL $|$nicht\\
4.&216.&2063.&855.&8116.&12.&4&65&10\_40\_6\_9 \textcolor{red}{\textcjheb{.twmy}} JMWt $|$(er) wird wanken\\
5.&217.&2064.&859.&8120.&16.&6&626&6\_200\_300\_70\_10\_40 \textcolor{red}{\textcjheb{my`+srw}} WRSaJM $|$aber die Gesetzlosen/und Frevler\\
6.&218.&2065.&865.&8126.&22.&2&31&30\_1 \textcolor{red}{\textcjheb{'l}} LA $|$nicht\\
7.&219.&2066.&867.&8128.&24.&5&386&10\_300\_20\_50\_6 \textcolor{red}{\textcjheb{wnk+sy}} JSKNW $|$(sie) werden (be)wohnen\\
8.&220.&2067.&872.&8133.&29.&3&291&1\_200\_90 \textcolor{red}{\textcjheb{.sr'}} AR"s $|$das Land/(im) Land\\
\end{tabular}\medskip \\
Ende des Verses 10.30\\
Verse: 286, Buchstaben: 31, 874, 8135, Totalwerte: 1811, 67932, 584976\\
\\
Der Gerechte wird nicht wanken in Ewigkeit, aber die Gesetzlosen werden das Land nicht bewohnen.\\
\newpage 
{\bf -- 10.31}\\
\medskip \\
\begin{tabular}{rrrrrrrrp{120mm}}
WV&WK&WB&ABK&ABB&ABV&AnzB&TW&Zahlencode \textcolor{red}{$\boldsymbol{Grundtext}$} Umschrift $|$"Ubersetzung(en)\\
1.&221.&2068.&875.&8136.&1.&2&90&80\_10 \textcolor{red}{\textcjheb{yp}} PJ $|$der Mund\\
2.&222.&2069.&877.&8138.&3.&4&204&90\_4\_10\_100 \textcolor{red}{\textcjheb{qyd.s}} "sDJQ $|$(des) Gerechten\\
3.&223.&2070.&881.&8142.&7.&4&68&10\_50\_6\_2 \textcolor{red}{\textcjheb{bwny}} JNWB $|$sprosst/er tr"agt\\
4.&224.&2071.&885.&8146.&11.&4&73&8\_20\_40\_5 \textcolor{red}{\textcjheb{hmk.h}} CKMH $|$Weisheit\\
5.&225.&2072.&889.&8150.&15.&5&392&6\_30\_300\_6\_50 \textcolor{red}{\textcjheb{nw+slw}} WLSWN $|$aber die Zunge/und die Zunge\\
6.&226.&2073.&894.&8155.&20.&6&911&400\_5\_80\_20\_6\_400 \textcolor{red}{\textcjheb{twkpht}} THPKWT $|$der Verkehrtheit/(von) Verkehrtheiten\\
7.&227.&2074.&900.&8161.&26.&4&1020&400\_20\_200\_400 \textcolor{red}{\textcjheb{trkt}} TKRT $|$(sie) wird ausgerottet (werden)\\
\end{tabular}\medskip \\
Ende des Verses 10.31\\
Verse: 287, Buchstaben: 29, 903, 8164, Totalwerte: 2758, 70690, 587734\\
\\
Der Mund des Gerechten spro"st Weisheit, aber die Zunge der Verkehrtheit wird ausgerottet werden.\\
\newpage 
{\bf -- 10.32}\\
\medskip \\
\begin{tabular}{rrrrrrrrp{120mm}}
WV&WK&WB&ABK&ABB&ABV&AnzB&TW&Zahlencode \textcolor{red}{$\boldsymbol{Grundtext}$} Umschrift $|$"Ubersetzung(en)\\
1.&228.&2075.&904.&8165.&1.&4&790&300\_80\_400\_10 \textcolor{red}{\textcjheb{ytp+s}} SPTJ $|$die Lippen\\
2.&229.&2076.&908.&8169.&5.&4&204&90\_4\_10\_100 \textcolor{red}{\textcjheb{qyd.s}} "sDJQ $|$(des) Gerechten\\
3.&230.&2077.&912.&8173.&9.&5&140&10\_4\_70\_6\_50 \textcolor{red}{\textcjheb{nw`dy}} JDaWN $|$verstehen sich auf/(sie) kennen\\
4.&231.&2078.&917.&8178.&14.&4&346&200\_90\_6\_50 \textcolor{red}{\textcjheb{nw.sr}} R"sWN $|$Wohlgef"alliges/Wohlgefallen\\
5.&232.&2079.&921.&8182.&18.&3&96&6\_80\_10 \textcolor{red}{\textcjheb{ypw}} WPJ $|$aber der Mund/und der Mund\\
6.&233.&2080.&924.&8185.&21.&5&620&200\_300\_70\_10\_40 \textcolor{red}{\textcjheb{my`+sr}} RSaJM $|$der Gesetzlosen/(von) Frevler(n)\\
7.&234.&2081.&929.&8190.&26.&6&911&400\_5\_80\_20\_6\_400 \textcolor{red}{\textcjheb{twkpht}} THPKWT $|$ist Verkehrtheit/(spricht aus) Verkehrtheiten\\
\end{tabular}\medskip \\
Ende des Verses 10.32\\
Verse: 288, Buchstaben: 31, 934, 8195, Totalwerte: 3107, 73797, 590841\\
\\
Die Lippen des Gerechten verstehen sich auf Wohlgef"alliges, aber der Mund der Gesetzlosen ist Verkehrtheit.\\
\\
{\bf Ende des Kapitels 10}\\
\newpage 
{\bf -- 11.1}\\
\medskip \\
\begin{tabular}{rrrrrrrrp{120mm}}
WV&WK&WB&ABK&ABB&ABV&AnzB&TW&Zahlencode \textcolor{red}{$\boldsymbol{Grundtext}$} Umschrift $|$"Ubersetzung(en)\\
1.&1.&2082.&1.&8196.&1.&5&108&40\_1\_7\_50\_10 \textcolor{red}{\textcjheb{ynz'm}} MAZNJ $|$(zwei) Waagschalen\\
2.&2.&2083.&6.&8201.&6.&4&285&40\_200\_40\_5 \textcolor{red}{\textcjheb{hmrm}} MRMH $|$tr"ugerische/des Betrugs\\
3.&3.&2084.&10.&8205.&10.&5&878&400\_6\_70\_2\_400 \textcolor{red}{\textcjheb{tb`wt}} TWaBT $|$(sind ein) Gr"auel\\
4.&4.&2085.&15.&8210.&15.&4&26&10\_5\_6\_5 \textcolor{red}{\textcjheb{hwhy}} JHWH $|$(f"ur) Jahwe\\
5.&5.&2086.&19.&8214.&19.&4&59&6\_1\_2\_50 \textcolor{red}{\textcjheb{nb'w}} WABN $|$aber Gewicht/und (ein) Stein\\
6.&6.&2087.&23.&8218.&23.&4&375&300\_30\_40\_5 \textcolor{red}{\textcjheb{hml+s}} SLMH $|$voll(st"andig)es\\
7.&7.&2088.&27.&8222.&27.&5&352&200\_90\_6\_50\_6 \textcolor{red}{\textcjheb{wnw.sr}} R"sWNW $|$(ist) sein Wohlgefallen\\
\end{tabular}\medskip \\
Ende des Verses 11.1\\
Verse: 289, Buchstaben: 31, 31, 8226, Totalwerte: 2083, 2083, 592924\\
\\
Tr"ugerische Waagschalen sind Jahwe ein Greuel, aber volles Gewicht ist sein Wohlgefallen.\\
\newpage 
{\bf -- 11.2}\\
\medskip \\
\begin{tabular}{rrrrrrrrp{120mm}}
WV&WK&WB&ABK&ABB&ABV&AnzB&TW&Zahlencode \textcolor{red}{$\boldsymbol{Grundtext}$} Umschrift $|$"Ubersetzung(en)\\
1.&8.&2089.&32.&8227.&1.&2&3&2\_1 \textcolor{red}{\textcjheb{'b}} BA $|$(er (=es)) kommt\\
2.&9.&2090.&34.&8229.&3.&4&67&7\_4\_6\_50 \textcolor{red}{\textcjheb{nwdz}} ZDWN $|$"Ubermut\\
3.&10.&2091.&38.&8233.&7.&4&19&6\_10\_2\_1 \textcolor{red}{\textcjheb{'byw}} WJBA $|$so kommt auch/und er (=es) kommt\\
4.&11.&2092.&42.&8237.&11.&4&186&100\_30\_6\_50 \textcolor{red}{\textcjheb{nwlq}} QLWN $|$Schande\\
5.&12.&2093.&46.&8241.&15.&3&407&6\_1\_400 \textcolor{red}{\textcjheb{t'w}} WAT $|$aber bei/und bei\\
6.&13.&2094.&49.&8244.&18.&6&266&90\_50\_6\_70\_10\_40 \textcolor{red}{\textcjheb{my`wn.s}} "sNWaJM $|$(den) Bescheidenen\\
7.&14.&2095.&55.&8250.&24.&4&73&8\_20\_40\_5 \textcolor{red}{\textcjheb{hmk.h}} CKMH $|$(ist) Weisheit\\
\end{tabular}\medskip \\
Ende des Verses 11.2\\
Verse: 290, Buchstaben: 27, 58, 8253, Totalwerte: 1021, 3104, 593945\\
\\
Kommt "Ubermut, so kommt auch Schande; bei den Bescheidenen aber ist Weisheit.\\
\newpage 
{\bf -- 11.3}\\
\medskip \\
\begin{tabular}{rrrrrrrrp{120mm}}
WV&WK&WB&ABK&ABB&ABV&AnzB&TW&Zahlencode \textcolor{red}{$\boldsymbol{Grundtext}$} Umschrift $|$"Ubersetzung(en)\\
1.&15.&2096.&59.&8254.&1.&3&840&400\_40\_400 \textcolor{red}{\textcjheb{tmt}} TMT $|$die Unstr"aflichkeit/Lauterkeit\\
2.&16.&2097.&62.&8257.&4.&5&560&10\_300\_200\_10\_40 \textcolor{red}{\textcjheb{myr+sy}} JSRJM $|$der Aufrichtigen/der Geraden\\
3.&17.&2098.&67.&8262.&9.&4&498&400\_50\_8\_40 \textcolor{red}{\textcjheb{m.hnt}} TNCM $|$(sie) leitet sie\\
4.&18.&2099.&71.&8266.&13.&4&176&6\_60\_30\_80 \textcolor{red}{\textcjheb{plsw}} WsLP $|$aber die Verkehrtheit/und (die) Verdrehtheit\\
5.&19.&2100.&75.&8270.&17.&6&65&2\_6\_3\_4\_10\_40 \textcolor{red}{\textcjheb{mydgwb}} BWGDJM $|$(der) Treulosen\\
6.&20.&2101.&81.&8276.&23.&4&350&6\_300\_4\_40 \textcolor{red}{\textcjheb{md+sw}} WSDM $|$(und er) zerst"ort sie\\
\end{tabular}\medskip \\
Ende des Verses 11.3\\
Verse: 291, Buchstaben: 26, 84, 8279, Totalwerte: 2489, 5593, 596434\\
\\
Die Unstr"aflichkeit der Aufrichtigen leitet sie, aber Treulosen Verkehrtheit zerst"ort sie.\\
\newpage 
{\bf -- 11.4}\\
\medskip \\
\begin{tabular}{rrrrrrrrp{120mm}}
WV&WK&WB&ABK&ABB&ABV&AnzB&TW&Zahlencode \textcolor{red}{$\boldsymbol{Grundtext}$} Umschrift $|$"Ubersetzung(en)\\
1.&21.&2102.&85.&8280.&1.&2&31&30\_1 \textcolor{red}{\textcjheb{'l}} LA $|$nicht(s)\\
2.&22.&2103.&87.&8282.&3.&5&126&10\_6\_70\_10\_30 \textcolor{red}{\textcjheb{ly`wy}} JWaJL $|$(er (=es)) n"utzt\\
3.&23.&2104.&92.&8287.&8.&3&61&5\_6\_50 \textcolor{red}{\textcjheb{nwh}} HWN $|$Verm"ogen/Reichtum\\
4.&24.&2105.&95.&8290.&11.&4&58&2\_10\_6\_40 \textcolor{red}{\textcjheb{mwyb}} BJWM $|$am Tag\\
5.&25.&2106.&99.&8294.&15.&4&277&70\_2\_200\_5 \textcolor{red}{\textcjheb{hrb`}} aBRH $|$des Zornes\\
6.&26.&2107.&103.&8298.&19.&5&205&6\_90\_4\_100\_5 \textcolor{red}{\textcjheb{hqd.sw}} W"sDQH $|$aber Gerechtigkeit/und Gerechtigkeit\\
7.&27.&2108.&108.&8303.&24.&4&530&400\_90\_10\_30 \textcolor{red}{\textcjheb{ly.st}} T"sJL $|$(sie) (er)rettet\\
8.&28.&2109.&112.&8307.&28.&4&486&40\_40\_6\_400 \textcolor{red}{\textcjheb{twmm}} MMWT $|$vom Tod/von (=vor dem) Tod\\
\end{tabular}\medskip \\
Ende des Verses 11.4\\
Verse: 292, Buchstaben: 31, 115, 8310, Totalwerte: 1774, 7367, 598208\\
\\
Verm"ogen n"utzt nichts am Tage des Zornes, aber Gerechtigkeit errettet vom Tode.\\
\newpage 
{\bf -- 11.5}\\
\medskip \\
\begin{tabular}{rrrrrrrrp{120mm}}
WV&WK&WB&ABK&ABB&ABV&AnzB&TW&Zahlencode \textcolor{red}{$\boldsymbol{Grundtext}$} Umschrift $|$"Ubersetzung(en)\\
1.&29.&2110.&116.&8311.&1.&4&594&90\_4\_100\_400 \textcolor{red}{\textcjheb{tqd.s}} "sDQT $|$(die) Gerechtigkeit\\
2.&30.&2111.&120.&8315.&5.&4&490&400\_40\_10\_40 \textcolor{red}{\textcjheb{mymt}} TMJM $|$des Vollkommen(en)/des Lauteren\\
3.&31.&2112.&124.&8319.&9.&4&910&400\_10\_300\_200 \textcolor{red}{\textcjheb{r+syt}} TJSR $|$(sie) macht gerade\\
4.&32.&2113.&128.&8323.&13.&4&230&4\_200\_20\_6 \textcolor{red}{\textcjheb{wkrd}} DRKW $|$seinen Weg\\
5.&33.&2114.&132.&8327.&17.&7&984&6\_2\_200\_300\_70\_400\_6 \textcolor{red}{\textcjheb{wt`+srbw}} WBRSaTW $|$aber durch seine Gesetzlosigkeit/und durch seinen Frevel\\
6.&34.&2115.&139.&8334.&24.&3&120&10\_80\_30 \textcolor{red}{\textcjheb{lpy}} JPL $|$(er (=es)) f"allt\\
7.&35.&2116.&142.&8337.&27.&3&570&200\_300\_70 \textcolor{red}{\textcjheb{`+sr}} RSa $|$der Gesetzlose/(der) Frevler\\
\end{tabular}\medskip \\
Ende des Verses 11.5\\
Verse: 293, Buchstaben: 29, 144, 8339, Totalwerte: 3898, 11265, 602106\\
\\
Des Vollkommenen Gerechtigkeit macht seinen Weg gerade, aber der Gesetzlose f"allt durch seine Gesetzlosigkeit.\\
\newpage 
{\bf -- 11.6}\\
\medskip \\
\begin{tabular}{rrrrrrrrp{120mm}}
WV&WK&WB&ABK&ABB&ABV&AnzB&TW&Zahlencode \textcolor{red}{$\boldsymbol{Grundtext}$} Umschrift $|$"Ubersetzung(en)\\
1.&36.&2117.&145.&8340.&1.&4&594&90\_4\_100\_400 \textcolor{red}{\textcjheb{tqd.s}} "sDQT $|$(die) Gerechtigkeit\\
2.&37.&2118.&149.&8344.&5.&5&560&10\_300\_200\_10\_40 \textcolor{red}{\textcjheb{myr+sy}} JSRJM $|$der Aufrichtigen/der Geraden\\
3.&38.&2119.&154.&8349.&10.&5&570&400\_90\_10\_30\_40 \textcolor{red}{\textcjheb{mly.st}} T"sJLM $|$(sie) (er)rettet sie\\
4.&39.&2120.&159.&8354.&15.&5&419&6\_2\_5\_6\_400 \textcolor{red}{\textcjheb{twhbw}} WBHWT $|$aber in ihrer Gier/und in der Gier\\
5.&40.&2121.&164.&8359.&20.&5&59&2\_3\_4\_10\_40 \textcolor{red}{\textcjheb{mydgb}} BGDJM $|$die Treulosen/(der) Treulosen\\
6.&41.&2122.&169.&8364.&25.&5&70&10\_30\_20\_4\_6 \textcolor{red}{\textcjheb{wdkly}} JLKDW $|$(sie) werden gefangen\\
\end{tabular}\medskip \\
Ende des Verses 11.6\\
Verse: 294, Buchstaben: 29, 173, 8368, Totalwerte: 2272, 13537, 604378\\
\\
Der Aufrichtigen Gerechtigkeit errettet sie, aber die Treulosen werden gefangen in ihrer Gier.\\
\newpage 
{\bf -- 11.7}\\
\medskip \\
\begin{tabular}{rrrrrrrrp{120mm}}
WV&WK&WB&ABK&ABB&ABV&AnzB&TW&Zahlencode \textcolor{red}{$\boldsymbol{Grundtext}$} Umschrift $|$"Ubersetzung(en)\\
1.&42.&2123.&174.&8369.&1.&4&448&2\_40\_6\_400 \textcolor{red}{\textcjheb{twmb}} BMWT $|$wenn stirbt/mit dem Tod\\
2.&43.&2124.&178.&8373.&5.&3&45&1\_4\_40 \textcolor{red}{\textcjheb{md'}} ADM $|$(ein(es)) Mensch(en)\\
3.&44.&2125.&181.&8376.&8.&3&570&200\_300\_70 \textcolor{red}{\textcjheb{`+sr}} RSa $|$gesetzloser/b"osen\\
4.&45.&2126.&184.&8379.&11.&4&407&400\_1\_2\_4 \textcolor{red}{\textcjheb{db't}} TABD $|$wird zunichte/sie (=es) geht unter\\
5.&46.&2127.&188.&8383.&15.&4&511&400\_100\_6\_5 \textcolor{red}{\textcjheb{hwqt}} TQWH $|$seine Hoffnung/(die) Hoffnung\\
6.&47.&2128.&192.&8387.&19.&6&850&6\_400\_6\_8\_30\_400 \textcolor{red}{\textcjheb{tl.hwtw}} WTWCLT $|$und die Erwartung\\
7.&48.&2129.&198.&8393.&25.&5&107&1\_6\_50\_10\_40 \textcolor{red}{\textcjheb{mynw'}} AWNJM $|$der Frevler/von L"ugen\\
8.&49.&2130.&203.&8398.&30.&4&12&1\_2\_4\_5 \textcolor{red}{\textcjheb{hdb'}} ABDH $|$ist zunichte geworden/(sie) geht zugrunde\\
\end{tabular}\medskip \\
Ende des Verses 11.7\\
Verse: 295, Buchstaben: 33, 206, 8401, Totalwerte: 2950, 16487, 607328\\
\\
Wenn ein gesetzloser Mensch stirbt, wird seine Hoffnung zunichte, und die Erwartung der Frevler ist zunichte geworden.\\
\newpage 
{\bf -- 11.8}\\
\medskip \\
\begin{tabular}{rrrrrrrrp{120mm}}
WV&WK&WB&ABK&ABB&ABV&AnzB&TW&Zahlencode \textcolor{red}{$\boldsymbol{Grundtext}$} Umschrift $|$"Ubersetzung(en)\\
1.&50.&2131.&207.&8402.&1.&4&204&90\_4\_10\_100 \textcolor{red}{\textcjheb{qyd.s}} "sDJQ $|$(der) Gerechte\\
2.&51.&2132.&211.&8406.&5.&4&335&40\_90\_200\_5 \textcolor{red}{\textcjheb{hr.sm}} M"sRH $|$aus der Drangsal/aus der Bedr"angnis\\
3.&52.&2133.&215.&8410.&9.&4&178&50\_8\_30\_90 \textcolor{red}{\textcjheb{.sl.hn}} NCL"s $|$wird befreit/(er) wird gerettet\\
4.&53.&2134.&219.&8414.&13.&4&19&6\_10\_2\_1 \textcolor{red}{\textcjheb{'byw}} WJBA $|$und (es) tritt/und er (=es) kommt\\
5.&54.&2135.&223.&8418.&17.&3&570&200\_300\_70 \textcolor{red}{\textcjheb{`+sr}} RSa $|$der Gesetzlose/(der) Frevler\\
6.&55.&2136.&226.&8421.&20.&5&824&400\_8\_400\_10\_6 \textcolor{red}{\textcjheb{wyt.ht}} TCTJW $|$an seine Stelle\\
\end{tabular}\medskip \\
Ende des Verses 11.8\\
Verse: 296, Buchstaben: 24, 230, 8425, Totalwerte: 2130, 18617, 609458\\
\\
Der Gerechte wird aus der Drangsal befreit, und der Gesetzlose tritt an seine Stelle.\\
\newpage 
{\bf -- 11.9}\\
\medskip \\
\begin{tabular}{rrrrrrrrp{120mm}}
WV&WK&WB&ABK&ABB&ABV&AnzB&TW&Zahlencode \textcolor{red}{$\boldsymbol{Grundtext}$} Umschrift $|$"Ubersetzung(en)\\
1.&56.&2137.&231.&8426.&1.&3&87&2\_80\_5 \textcolor{red}{\textcjheb{hpb}} BPH $|$mit dem Mund\\
2.&57.&2138.&234.&8429.&4.&3&138&8\_50\_80 \textcolor{red}{\textcjheb{pn.h}} CNP $|$der Ruchlose\\
3.&58.&2139.&237.&8432.&7.&4&718&10\_300\_8\_400 \textcolor{red}{\textcjheb{t.h+sy}} JSCT $|$(er) verdirbt\\
4.&59.&2140.&241.&8436.&11.&4&281&200\_70\_5\_6 \textcolor{red}{\textcjheb{wh`r}} RaHW $|$seinen N"achsten\\
5.&60.&2141.&245.&8440.&15.&5&482&6\_2\_4\_70\_400 \textcolor{red}{\textcjheb{t`dbw}} WBDaT $|$aber durch Erkenntnis/und durch Erkenntnis\\
6.&61.&2142.&250.&8445.&20.&6&254&90\_4\_10\_100\_10\_40 \textcolor{red}{\textcjheb{myqyd.s}} "sDJQJM $|$(die) Gerechte(n)\\
7.&62.&2143.&256.&8451.&26.&5&144&10\_8\_30\_90\_6 \textcolor{red}{\textcjheb{w.sl.hy}} JCL"sW $|$werden befreit/(sie) werden gerettet\\
\end{tabular}\medskip \\
Ende des Verses 11.9\\
Verse: 297, Buchstaben: 30, 260, 8455, Totalwerte: 2104, 20721, 611562\\
\\
Mit dem Munde verdirbt der Ruchlose seinen N"achsten, aber durch Erkenntnis werden die Gerechten befreit.\\
\newpage 
{\bf -- 11.10}\\
\medskip \\
\begin{tabular}{rrrrrrrrp{120mm}}
WV&WK&WB&ABK&ABB&ABV&AnzB&TW&Zahlencode \textcolor{red}{$\boldsymbol{Grundtext}$} Umschrift $|$"Ubersetzung(en)\\
1.&63.&2144.&261.&8456.&1.&4&19&2\_9\_6\_2 \textcolor{red}{\textcjheb{bw.tb}} BtWB $|$beim Wohl/ob des Gl"ucks\\
2.&64.&2145.&265.&8460.&5.&6&254&90\_4\_10\_100\_10\_40 \textcolor{red}{\textcjheb{myqyd.s}} "sDJQJM $|$der Gerechten\\
3.&65.&2146.&271.&8466.&11.&4&590&400\_70\_30\_90 \textcolor{red}{\textcjheb{.sl`t}} TaL"s $|$(sie (=es)) frohlockt\\
4.&66.&2147.&275.&8470.&15.&4&315&100\_200\_10\_5 \textcolor{red}{\textcjheb{hyrq}} QRJH $|$die Stadt\\
5.&67.&2148.&279.&8474.&19.&5&15&6\_2\_1\_2\_4 \textcolor{red}{\textcjheb{db'bw}} WBABD $|$und beim Untergang/und beim Untergehen\\
6.&68.&2149.&284.&8479.&24.&5&620&200\_300\_70\_10\_40 \textcolor{red}{\textcjheb{my`+sr}} RSaJM $|$der Gesetzlosen/(der) Frevler\\
7.&69.&2150.&289.&8484.&29.&3&255&200\_50\_5 \textcolor{red}{\textcjheb{hnr}} RNH $|$ist Jubel/(erschallt) Jubel\\
\end{tabular}\medskip \\
Ende des Verses 11.10\\
Verse: 298, Buchstaben: 31, 291, 8486, Totalwerte: 2068, 22789, 613630\\
\\
Die Stadt frohlockt beim Wohle der Gerechten, und beim Untergang der Gesetzlosen ist Jubel.\\
\newpage 
{\bf -- 11.11}\\
\medskip \\
\begin{tabular}{rrrrrrrrp{120mm}}
WV&WK&WB&ABK&ABB&ABV&AnzB&TW&Zahlencode \textcolor{red}{$\boldsymbol{Grundtext}$} Umschrift $|$"Ubersetzung(en)\\
1.&70.&2151.&292.&8487.&1.&5&624&2\_2\_200\_20\_400 \textcolor{red}{\textcjheb{tkrbb}} BBRKT $|$durch den Segen/durch die Segnung\\
2.&71.&2152.&297.&8492.&6.&5&560&10\_300\_200\_10\_40 \textcolor{red}{\textcjheb{myr+sy}} JSRJM $|$der Aufrichtigen/der Geraden\\
3.&72.&2153.&302.&8497.&11.&4&646&400\_200\_6\_40 \textcolor{red}{\textcjheb{mwrt}} TRWM $|$kommt empor/(sie) wird erh"oht\\
4.&73.&2154.&306.&8501.&15.&3&700&100\_200\_400 \textcolor{red}{\textcjheb{trq}} QRT $|$eine Stadt/(die) Stadt\\
5.&74.&2155.&309.&8504.&18.&4&98&6\_2\_80\_10 \textcolor{red}{\textcjheb{ypbw}} WBPJ $|$aber durch den Mund/und durch den Mund\\
6.&75.&2156.&313.&8508.&22.&5&620&200\_300\_70\_10\_40 \textcolor{red}{\textcjheb{my`+sr}} RSaJM $|$der Gesetzlosen/(der) Frevler\\
7.&76.&2157.&318.&8513.&27.&4&665&400\_5\_200\_60 \textcolor{red}{\textcjheb{srht}} THRs $|$sie wird niedergerissen\\
\end{tabular}\medskip \\
Ende des Verses 11.11\\
Verse: 299, Buchstaben: 30, 321, 8516, Totalwerte: 3913, 26702, 617543\\
\\
Durch den Segen der Aufrichtigen kommt eine Stadt empor, aber durch den Mund der Gesetzlosen wird sie niedergerissen.\\
\newpage 
{\bf -- 11.12}\\
\medskip \\
\begin{tabular}{rrrrrrrrp{120mm}}
WV&WK&WB&ABK&ABB&ABV&AnzB&TW&Zahlencode \textcolor{red}{$\boldsymbol{Grundtext}$} Umschrift $|$"Ubersetzung(en)\\
1.&77.&2158.&322.&8517.&1.&2&9&2\_7 \textcolor{red}{\textcjheb{zb}} BZ $|$wer verachtet/ein Verachtender\\
2.&78.&2159.&324.&8519.&3.&5&311&30\_200\_70\_5\_6 \textcolor{red}{\textcjheb{wh`rl}} LRaHW $|$seinen N"achsten\\
3.&79.&2160.&329.&8524.&8.&3&268&8\_60\_200 \textcolor{red}{\textcjheb{rs.h}} CsR $|$hat keinen/(ist ein) Ermangelnder\\
4.&80.&2161.&332.&8527.&11.&2&32&30\_2 \textcolor{red}{\textcjheb{bl}} LB $|$Herz (=Verstand)\\
5.&81.&2162.&334.&8529.&13.&4&317&6\_1\_10\_300 \textcolor{red}{\textcjheb{+sy'w}} WAJS $|$aber ein Mann/und der Mann\\
6.&82.&2163.&338.&8533.&17.&6&864&400\_2\_6\_50\_6\_400 \textcolor{red}{\textcjheb{twnwbt}} TBWNWT $|$verst"andiger/(mit) Einsicht(en)\\
7.&83.&2164.&344.&8539.&23.&5&528&10\_8\_200\_10\_300 \textcolor{red}{\textcjheb{+syr.hy}} JCRJS $|$(er) schweigt (still)\\
\end{tabular}\medskip \\
Ende des Verses 11.12\\
Verse: 300, Buchstaben: 27, 348, 8543, Totalwerte: 2329, 29031, 619872\\
\\
Wer seinen N"achsten verachtet, hat keinen Verstand; aber ein verst"andiger Mann schweigt still.\\
\newpage 
{\bf -- 11.13}\\
\medskip \\
\begin{tabular}{rrrrrrrrp{120mm}}
WV&WK&WB&ABK&ABB&ABV&AnzB&TW&Zahlencode \textcolor{red}{$\boldsymbol{Grundtext}$} Umschrift $|$"Ubersetzung(en)\\
1.&84.&2165.&349.&8544.&1.&4&61&5\_6\_30\_20 \textcolor{red}{\textcjheb{klwh}} HWLK $|$wer umhergeht/(ein) Umhergehender\\
2.&85.&2166.&353.&8548.&5.&4&260&200\_20\_10\_30 \textcolor{red}{\textcjheb{lykr}} RKJL $|$(als) Verleumder\\
3.&86.&2167.&357.&8552.&9.&4&78&40\_3\_30\_5 \textcolor{red}{\textcjheb{hlgm}} MGLH $|$(er) deckt auf\\
4.&87.&2168.&361.&8556.&13.&3&70&60\_6\_4 \textcolor{red}{\textcjheb{dws}} sWD $|$(anvertrautes) Geheimnis\\
5.&88.&2169.&364.&8559.&16.&5&147&6\_50\_1\_40\_50 \textcolor{red}{\textcjheb{nm'nw}} WNAMN $|$wer aber ist treuen/und ein Zuverl"assiger\\
6.&89.&2170.&369.&8564.&21.&3&214&200\_6\_8 \textcolor{red}{\textcjheb{.hwr}} RWC $|$Geistes/(in seinem) Sinn\\
7.&90.&2171.&372.&8567.&24.&4&125&40\_20\_60\_5 \textcolor{red}{\textcjheb{hskm}} MKsH $|$deckt zu/h"alt geheim\\
8.&91.&2172.&376.&8571.&28.&3&206&4\_2\_200 \textcolor{red}{\textcjheb{rbd}} DBR $|$(die) Sache\\
\end{tabular}\medskip \\
Ende des Verses 11.13\\
Verse: 301, Buchstaben: 30, 378, 8573, Totalwerte: 1161, 30192, 621033\\
\\
Wer als Verleumder umhergeht, deckt das Geheimnis auf; wer aber treuen Geistes ist, deckt die Sache zu.\\
\newpage 
{\bf -- 11.14}\\
\medskip \\
\begin{tabular}{rrrrrrrrp{120mm}}
WV&WK&WB&ABK&ABB&ABV&AnzB&TW&Zahlencode \textcolor{red}{$\boldsymbol{Grundtext}$} Umschrift $|$"Ubersetzung(en)\\
1.&92.&2173.&379.&8574.&1.&4&63&2\_1\_10\_50 \textcolor{red}{\textcjheb{ny'b}} BAJN $|$wo nicht ist/ohne\\
2.&93.&2174.&383.&8578.&5.&6&846&400\_8\_2\_30\_6\_400 \textcolor{red}{\textcjheb{twlb.ht}} TCBLWT $|$F"uhrung/"Uberlegungen\\
3.&94.&2175.&389.&8584.&11.&3&120&10\_80\_30 \textcolor{red}{\textcjheb{lpy}} JPL $|$verf"allt/er (=es) zerf"allt\\
4.&95.&2176.&392.&8587.&14.&2&110&70\_40 \textcolor{red}{\textcjheb{m`}} aM $|$(ein) Volk\\
5.&96.&2177.&394.&8589.&16.&6&787&6\_400\_300\_6\_70\_5 \textcolor{red}{\textcjheb{h`w+stw}} WTSWaH $|$aber Heil ist/und Erfolg (ist)\\
6.&97.&2178.&400.&8595.&22.&3&204&2\_200\_2 \textcolor{red}{\textcjheb{brb}} BRB $|$bei der Menge/durch eine Menge\\
7.&98.&2179.&403.&8598.&25.&4&176&10\_6\_70\_90 \textcolor{red}{\textcjheb{.s`wy}} JWa"s $|$(der) Ratgeber\\
\end{tabular}\medskip \\
Ende des Verses 11.14\\
Verse: 302, Buchstaben: 28, 406, 8601, Totalwerte: 2306, 32498, 623339\\
\\
Wo keine F"uhrung ist, verf"allt ein Volk; aber Heil ist bei der Menge der Ratgeber.\\
\newpage 
{\bf -- 11.15}\\
\medskip \\
\begin{tabular}{rrrrrrrrp{120mm}}
WV&WK&WB&ABK&ABB&ABV&AnzB&TW&Zahlencode \textcolor{red}{$\boldsymbol{Grundtext}$} Umschrift $|$"Ubersetzung(en)\\
1.&99.&2180.&407.&8602.&1.&2&270&200\_70 \textcolor{red}{\textcjheb{`r}} Ra $|$sehr schlecht/schlimm\\
2.&100.&2181.&409.&8604.&3.&4&286&10\_200\_6\_70 \textcolor{red}{\textcjheb{`wry}} JRWa $|$ergeht es einem/es wird "ubel behandelt\\
3.&101.&2182.&413.&8608.&7.&2&30&20\_10 \textcolor{red}{\textcjheb{yk}} KJ $|$wenn\\
4.&102.&2183.&415.&8610.&9.&3&272&70\_200\_2 \textcolor{red}{\textcjheb{br`}} aRB $|$er B"urge geworden ist/einer sich verb"urgt\\
5.&103.&2184.&418.&8613.&12.&2&207&7\_200 \textcolor{red}{\textcjheb{rz}} ZR $|$f"ur einen anderen/(f"ur einen) Fremden\\
6.&104.&2185.&420.&8615.&14.&4&357&6\_300\_50\_1 \textcolor{red}{\textcjheb{'n+sw}} WSNA $|$wer aber hasst/und ein Hassender\\
7.&105.&2186.&424.&8619.&18.&5&620&400\_100\_70\_10\_40 \textcolor{red}{\textcjheb{my`qt}} TQaJM $|$(das) Hand(ein)schlag(en)\\
8.&106.&2187.&429.&8624.&23.&4&25&2\_6\_9\_8 \textcolor{red}{\textcjheb{.h.twb}} BWtC $|$(ist) (ge)sicher(t)\\
\end{tabular}\medskip \\
Ende des Verses 11.15\\
Verse: 303, Buchstaben: 26, 432, 8627, Totalwerte: 2067, 34565, 625406\\
\\
Sehr schlecht ergeht's einem, wenn er f"ur einen anderen B"urge geworden ist; wer aber das Handeinschlagen ha"st, ist sicher.\\
\newpage 
{\bf -- 11.16}\\
\medskip \\
\begin{tabular}{rrrrrrrrp{120mm}}
WV&WK&WB&ABK&ABB&ABV&AnzB&TW&Zahlencode \textcolor{red}{$\boldsymbol{Grundtext}$} Umschrift $|$"Ubersetzung(en)\\
1.&107.&2188.&433.&8628.&1.&3&701&1\_300\_400 \textcolor{red}{\textcjheb{t+s'}} AST $|$(eine) Frau\\
2.&108.&2189.&436.&8631.&4.&2&58&8\_50 \textcolor{red}{\textcjheb{n.h}} CN $|$anmutige/(von) Anmut\\
3.&109.&2190.&438.&8633.&6.&4&860&400\_400\_40\_20 \textcolor{red}{\textcjheb{kmtt}} TTMK $|$(sie) erlangt\\
4.&110.&2191.&442.&8637.&10.&4&32&20\_2\_6\_4 \textcolor{red}{\textcjheb{dwbk}} KBWD $|$Ehre\\
5.&111.&2192.&446.&8641.&14.&7&426&6\_70\_200\_10\_90\_10\_40 \textcolor{red}{\textcjheb{my.syr`w}} WaRJ"sJM $|$und Gewaltt"atige\\
6.&112.&2193.&453.&8648.&21.&5&476&10\_400\_40\_20\_6 \textcolor{red}{\textcjheb{wkmty}} JTMKW $|$(sie) erlangen\\
7.&113.&2194.&458.&8653.&26.&3&570&70\_300\_200 \textcolor{red}{\textcjheb{r+s`}} aSR $|$Reichtum\\
\end{tabular}\medskip \\
Ende des Verses 11.16\\
Verse: 304, Buchstaben: 28, 460, 8655, Totalwerte: 3123, 37688, 628529\\
\\
Ein anmutiges Weib erlangt Ehre, und Gewaltt"atige erlangen Reichtum.\\
\newpage 
{\bf -- 11.17}\\
\medskip \\
\begin{tabular}{rrrrrrrrp{120mm}}
WV&WK&WB&ABK&ABB&ABV&AnzB&TW&Zahlencode \textcolor{red}{$\boldsymbol{Grundtext}$} Umschrift $|$"Ubersetzung(en)\\
1.&114.&2195.&461.&8656.&1.&3&73&3\_40\_30 \textcolor{red}{\textcjheb{lmg}} GML $|$wohl tut/ein Guttuender\\
2.&115.&2196.&464.&8659.&4.&4&436&50\_80\_300\_6 \textcolor{red}{\textcjheb{w+spn}} NPSW $|$sich selbst\\
3.&116.&2197.&468.&8663.&8.&3&311&1\_10\_300 \textcolor{red}{\textcjheb{+sy'}} AJS $|$der Mildt"atige/(ist ein) Mann\\
4.&117.&2198.&471.&8666.&11.&3&72&8\_60\_4 \textcolor{red}{\textcjheb{ds.h}} CsD $|$/(der) G"ute\\
5.&118.&2199.&474.&8669.&14.&4&296&6\_70\_20\_200 \textcolor{red}{\textcjheb{rk`w}} WaKR $|$aber wehe tut/und ein Betr"ubender\\
6.&119.&2200.&478.&8673.&18.&4&507&300\_1\_200\_6 \textcolor{red}{\textcjheb{wr'+s}} SARW $|$seinem Fleisch\\
7.&120.&2201.&482.&8677.&22.&5&238&1\_20\_7\_200\_10 \textcolor{red}{\textcjheb{yrzk'}} AKZRJ $|$der Unbarmherzige/(ist) unbarmherzig\\
\end{tabular}\medskip \\
Ende des Verses 11.17\\
Verse: 305, Buchstaben: 26, 486, 8681, Totalwerte: 1933, 39621, 630462\\
\\
Sich selbst tut der Mildt"atige wohl, der Unbarmherzige aber tut seinem Fleische wehe.\\
\newpage 
{\bf -- 11.18}\\
\medskip \\
\begin{tabular}{rrrrrrrrp{120mm}}
WV&WK&WB&ABK&ABB&ABV&AnzB&TW&Zahlencode \textcolor{red}{$\boldsymbol{Grundtext}$} Umschrift $|$"Ubersetzung(en)\\
1.&121.&2202.&487.&8682.&1.&3&570&200\_300\_70 \textcolor{red}{\textcjheb{`+sr}} RSa $|$der Gesetzlose/(ein) Frevler\\
2.&122.&2203.&490.&8685.&4.&3&375&70\_300\_5 \textcolor{red}{\textcjheb{h+s`}} aSH $|$schafft/(ist) machend\\
3.&123.&2204.&493.&8688.&7.&4&580&80\_70\_30\_400 \textcolor{red}{\textcjheb{tl`p}} PaLT $|$sich Gewinn/(einen) Gewinn\\
4.&124.&2205.&497.&8692.&11.&3&600&300\_100\_200 \textcolor{red}{\textcjheb{rq+s}} SQR $|$tr"uglichen/des Trugs\\
5.&125.&2206.&500.&8695.&14.&4&283&6\_7\_200\_70 \textcolor{red}{\textcjheb{`rzw}} WZRa $|$wer aber s"at/und ein S"aender\\
6.&126.&2207.&504.&8699.&18.&4&199&90\_4\_100\_5 \textcolor{red}{\textcjheb{hqd.s}} "sDQH $|$Gerechtigkeit\\
7.&127.&2208.&508.&8703.&22.&3&520&300\_20\_200 \textcolor{red}{\textcjheb{rk+s}} SKR $|$(einen) Lohn\\
8.&128.&2209.&511.&8706.&25.&3&441&1\_40\_400 \textcolor{red}{\textcjheb{tm'}} AMT $|$wahrhaftigen/(von) Wahrheit\\
\end{tabular}\medskip \\
Ende des Verses 11.18\\
Verse: 306, Buchstaben: 27, 513, 8708, Totalwerte: 3568, 43189, 634030\\
\\
Der Gesetzlose schafft sich tr"uglichen Gewinn, wer aber Gerechtigkeit s"at, wahrhaftigen Lohn.\\
\newpage 
{\bf -- 11.19}\\
\medskip \\
\begin{tabular}{rrrrrrrrp{120mm}}
WV&WK&WB&ABK&ABB&ABV&AnzB&TW&Zahlencode \textcolor{red}{$\boldsymbol{Grundtext}$} Umschrift $|$"Ubersetzung(en)\\
1.&129.&2210.&514.&8709.&1.&2&70&20\_50 \textcolor{red}{\textcjheb{nk}} KN $|$wie gereicht/Rechte\\
2.&130.&2211.&516.&8711.&3.&4&199&90\_4\_100\_5 \textcolor{red}{\textcjheb{hqd.s}} "sDQH $|$(die) Gerechtigkeit\\
3.&131.&2212.&520.&8715.&7.&5&98&30\_8\_10\_10\_40 \textcolor{red}{\textcjheb{myy.hl}} LCJJM $|$(f"uhrt) zum Leben\\
4.&132.&2213.&525.&8720.&12.&5&330&6\_40\_200\_4\_80 \textcolor{red}{\textcjheb{pdrmw}} WMRDP $|$so es dem der nachjagt/und ein Nachjagender\\
5.&133.&2214.&530.&8725.&17.&3&275&200\_70\_5 \textcolor{red}{\textcjheb{h`r}} RaH $|$B"osem\\
6.&134.&2215.&533.&8728.&20.&5&482&30\_40\_6\_400\_6 \textcolor{red}{\textcjheb{wtwml}} LMWTW $|$zu seinem Tod\\
\end{tabular}\medskip \\
Ende des Verses 11.19\\
Verse: 307, Buchstaben: 24, 537, 8732, Totalwerte: 1454, 44643, 635484\\
\\
Wie die Gerechtigkeit zum Leben, so gereicht es dem, der B"osem nachjagt, zu seinem Tode.\\
\newpage 
{\bf -- 11.20}\\
\medskip \\
\begin{tabular}{rrrrrrrrp{120mm}}
WV&WK&WB&ABK&ABB&ABV&AnzB&TW&Zahlencode \textcolor{red}{$\boldsymbol{Grundtext}$} Umschrift $|$"Ubersetzung(en)\\
1.&135.&2216.&538.&8733.&1.&5&878&400\_6\_70\_2\_400 \textcolor{red}{\textcjheb{tb`wt}} TWaBT $|$ein Gr"auel sind/Abscheu\\
2.&136.&2217.&543.&8738.&6.&4&26&10\_5\_6\_5 \textcolor{red}{\textcjheb{hwhy}} JHWH $|$(f"ur) Jahwe\\
3.&137.&2218.&547.&8742.&10.&4&480&70\_100\_300\_10 \textcolor{red}{\textcjheb{y+sq`}} aQSJ $|$die verkehrten/(sind) Verkehrte\\
4.&138.&2219.&551.&8746.&14.&2&32&30\_2 \textcolor{red}{\textcjheb{bl}} LB $|$(im) Herzen(s) (sind)\\
5.&139.&2220.&553.&8748.&16.&6&358&6\_200\_90\_6\_50\_6 \textcolor{red}{\textcjheb{wnw.srw}} WR"sWNW $|$aber sein Wohlgefallen/und sein Wohlgefallen\\
6.&140.&2221.&559.&8754.&22.&5&500&400\_40\_10\_40\_10 \textcolor{red}{\textcjheb{ymymt}} TMJMJ $|$sind die Vollkommenen/die vollendeten\\
7.&141.&2222.&564.&8759.&27.&3&224&4\_200\_20 \textcolor{red}{\textcjheb{krd}} DRK $|$im Weg/Weges\\
\end{tabular}\medskip \\
Ende des Verses 11.20\\
Verse: 308, Buchstaben: 29, 566, 8761, Totalwerte: 2498, 47141, 637982\\
\\
Die verkehrten Herzens sind, sind Jahwe ein Greuel; aber sein Wohlgefallen sind die im Wege Vollkommenen.\\
\newpage 
{\bf -- 11.21}\\
\medskip \\
\begin{tabular}{rrrrrrrrp{120mm}}
WV&WK&WB&ABK&ABB&ABV&AnzB&TW&Zahlencode \textcolor{red}{$\boldsymbol{Grundtext}$} Umschrift $|$"Ubersetzung(en)\\
1.&142.&2223.&567.&8762.&1.&2&14&10\_4 \textcolor{red}{\textcjheb{dy}} JD $|$(die) Hand\\
2.&143.&2224.&569.&8764.&3.&3&44&30\_10\_4 \textcolor{red}{\textcjheb{dyl}} LJD $|$darauf/auf Hand\\
3.&144.&2225.&572.&8767.&6.&2&31&30\_1 \textcolor{red}{\textcjheb{'l}} LA $|$nicht\\
4.&145.&2226.&574.&8769.&8.&4&165&10\_50\_100\_5 \textcolor{red}{\textcjheb{hqny}} JNQH $|$wird f"ur schuldlos gehalten werden/er (=es) bleibt ungestraft\\
5.&146.&2227.&578.&8773.&12.&2&270&200\_70 \textcolor{red}{\textcjheb{`r}} Ra $|$der B"ose/(ein) B"oser\\
6.&147.&2228.&580.&8775.&14.&4&283&6\_7\_200\_70 \textcolor{red}{\textcjheb{`rzw}} WZRa $|$aber der Same/und der Same\\
7.&148.&2229.&584.&8779.&18.&6&254&90\_4\_10\_100\_10\_40 \textcolor{red}{\textcjheb{myqyd.s}} "sDJQJM $|$der Gerechten\\
8.&149.&2230.&590.&8785.&24.&4&129&50\_40\_30\_9 \textcolor{red}{\textcjheb{.tlmn}} NMLt $|$wird entrinnen/er rettet sich\\
\end{tabular}\medskip \\
Ende des Verses 11.21\\
Verse: 309, Buchstaben: 27, 593, 8788, Totalwerte: 1190, 48331, 639172\\
\\
Die Hand darauf! Der B"ose wird nicht f"ur schuldlos gehalten werden; aber der Same der Gerechten wird entrinnen.\\
\newpage 
{\bf -- 11.22}\\
\medskip \\
\begin{tabular}{rrrrrrrrp{120mm}}
WV&WK&WB&ABK&ABB&ABV&AnzB&TW&Zahlencode \textcolor{red}{$\boldsymbol{Grundtext}$} Umschrift $|$"Ubersetzung(en)\\
1.&150.&2231.&594.&8789.&1.&3&97&50\_7\_40 \textcolor{red}{\textcjheb{mzn}} NZM $|$(wie) (ein) Ring\\
2.&151.&2232.&597.&8792.&4.&3&14&7\_5\_2 \textcolor{red}{\textcjheb{bhz}} ZHB $|$goldener/(von) Gold\\
3.&152.&2233.&600.&8795.&7.&3&83&2\_1\_80 \textcolor{red}{\textcjheb{p'b}} BAP $|$in (der) Nase\\
4.&153.&2234.&603.&8798.&10.&4&225&8\_7\_10\_200 \textcolor{red}{\textcjheb{ryz.h}} CZJR $|$(eines) (Wild)Schweins\\
5.&154.&2235.&607.&8802.&14.&3&306&1\_300\_5 \textcolor{red}{\textcjheb{h+s'}} ASH $|$(so ist eine) Frau\\
6.&155.&2236.&610.&8805.&17.&3&95&10\_80\_5 \textcolor{red}{\textcjheb{hpy}} JPH $|$sch"one\\
7.&156.&2237.&613.&8808.&20.&4&666&6\_60\_200\_400 \textcolor{red}{\textcjheb{trsw}} WsRT $|$ohne/und abweichend (ist)\\
8.&157.&2238.&617.&8812.&24.&3&119&9\_70\_40 \textcolor{red}{\textcjheb{m`.t}} taM $|$Anstand/vom Geschmack\\
\end{tabular}\medskip \\
Ende des Verses 11.22\\
Verse: 310, Buchstaben: 26, 619, 8814, Totalwerte: 1605, 49936, 640777\\
\\
Ein goldener Ring in der Nase eines Schweines, so ist ein sch"ones Weib ohne Anstand.\\
\newpage 
{\bf -- 11.23}\\
\medskip \\
\begin{tabular}{rrrrrrrrp{120mm}}
WV&WK&WB&ABK&ABB&ABV&AnzB&TW&Zahlencode \textcolor{red}{$\boldsymbol{Grundtext}$} Umschrift $|$"Ubersetzung(en)\\
1.&158.&2239.&620.&8815.&1.&4&807&400\_1\_6\_400 \textcolor{red}{\textcjheb{tw't}} TAWT $|$das Begehren/das Verlangen\\
2.&159.&2240.&624.&8819.&5.&6&254&90\_4\_10\_100\_10\_40 \textcolor{red}{\textcjheb{myqyd.s}} "sDJQJM $|$der Gerechten\\
3.&160.&2241.&630.&8825.&11.&2&21&1\_20 \textcolor{red}{\textcjheb{k'}} AK $|$ist nur/(bringt) nur\\
4.&161.&2242.&632.&8827.&13.&3&17&9\_6\_2 \textcolor{red}{\textcjheb{bw.t}} tWB $|$Gutes\\
5.&162.&2243.&635.&8830.&16.&4&906&400\_100\_6\_400 \textcolor{red}{\textcjheb{twqt}} TQWT $|$die Hoffnung\\
6.&163.&2244.&639.&8834.&20.&5&620&200\_300\_70\_10\_40 \textcolor{red}{\textcjheb{my`+sr}} RSaJM $|$der Gesetzlosen/(der) Frevler\\
7.&164.&2245.&644.&8839.&25.&4&277&70\_2\_200\_5 \textcolor{red}{\textcjheb{hrb`}} aBRH $|$ist der Grimm/Zorn\\
\end{tabular}\medskip \\
Ende des Verses 11.23\\
Verse: 311, Buchstaben: 28, 647, 8842, Totalwerte: 2902, 52838, 643679\\
\\
Das Begehren der Gerechten ist nur Gutes; die Hoffnung der Gesetzlosen ist der Grimm.\\
\newpage 
{\bf -- 11.24}\\
\medskip \\
\begin{tabular}{rrrrrrrrp{120mm}}
WV&WK&WB&ABK&ABB&ABV&AnzB&TW&Zahlencode \textcolor{red}{$\boldsymbol{Grundtext}$} Umschrift $|$"Ubersetzung(en)\\
1.&165.&2246.&648.&8843.&1.&2&310&10\_300 \textcolor{red}{\textcjheb{+sy}} JS $|$(da) ist\\
2.&166.&2247.&650.&8845.&3.&4&327&40\_80\_7\_200 \textcolor{red}{\textcjheb{rzpm}} MPZR $|$einer der ausstreut/einer Ausstreuender\\
3.&167.&2248.&654.&8849.&7.&5&202&6\_50\_6\_60\_80 \textcolor{red}{\textcjheb{pswnw}} WNWsP $|$und er bekommt mehr/und gewinnend\\
4.&168.&2249.&659.&8854.&12.&3&80&70\_6\_4 \textcolor{red}{\textcjheb{dw`}} aWD $|$noch\\
5.&169.&2250.&662.&8857.&15.&5&340&6\_8\_6\_300\_20 \textcolor{red}{\textcjheb{k+sw.hw}} WCWSK $|$und einer der mehr spart/und ein Zur"uckhaltender\\
6.&170.&2251.&667.&8862.&20.&4&550&40\_10\_300\_200 \textcolor{red}{\textcjheb{r+sym}} MJSR $|$als recht ist/Geradheit\\
7.&171.&2252.&671.&8866.&24.&2&21&1\_20 \textcolor{red}{\textcjheb{k'}} AK $|$(und) nur\\
8.&172.&2253.&673.&8868.&26.&6&344&30\_40\_8\_60\_6\_200 \textcolor{red}{\textcjheb{rws.hml}} LMCsWR $|$(ist es zum) Mangel\\
\end{tabular}\medskip \\
Ende des Verses 11.24\\
Verse: 312, Buchstaben: 31, 678, 8873, Totalwerte: 2174, 55012, 645853\\
\\
Da ist einer, der ausstreut, und er bekommt noch mehr; und einer, der mehr spart als recht ist, und es ist nur zum Mangel.\\
\newpage 
{\bf -- 11.25}\\
\medskip \\
\begin{tabular}{rrrrrrrrp{120mm}}
WV&WK&WB&ABK&ABB&ABV&AnzB&TW&Zahlencode \textcolor{red}{$\boldsymbol{Grundtext}$} Umschrift $|$"Ubersetzung(en)\\
1.&173.&2254.&679.&8874.&1.&3&430&50\_80\_300 \textcolor{red}{\textcjheb{+spn}} NPS $|$die Seele/(eine) Person\\
2.&174.&2255.&682.&8877.&4.&4&227&2\_200\_20\_5 \textcolor{red}{\textcjheb{hkrb}} BRKH $|$segnende/(von) Segenswunsch\\
3.&175.&2256.&686.&8881.&8.&4&754&400\_4\_300\_50 \textcolor{red}{\textcjheb{n+sdt}} TDSN $|$(sie) wird (reichlich) ges"attigt\\
4.&176.&2257.&690.&8885.&12.&5&257&6\_40\_200\_6\_5 \textcolor{red}{\textcjheb{hwrmw}} WMRWH $|$und der Tr"ankende/und ein Tr"ankender\\
5.&177.&2258.&695.&8890.&17.&2&43&3\_40 \textcolor{red}{\textcjheb{mg}} GM $|$auch (selbst)\\
6.&178.&2259.&697.&8892.&19.&3&12&5\_6\_1 \textcolor{red}{\textcjheb{'wh}} HWA $|$/er\\
7.&179.&2260.&700.&8895.&22.&4&217&10\_6\_200\_1 \textcolor{red}{\textcjheb{'rwy}} JWRA $|$wird getr"ankt/er wird gelabt\\
\end{tabular}\medskip \\
Ende des Verses 11.25\\
Verse: 313, Buchstaben: 25, 703, 8898, Totalwerte: 1940, 56952, 647793\\
\\
Die segnende Seele wird reichlich ges"attigt, und der Tr"ankende wird auch selbst getr"ankt.\\
\newpage 
{\bf -- 11.26}\\
\medskip \\
\begin{tabular}{rrrrrrrrp{120mm}}
WV&WK&WB&ABK&ABB&ABV&AnzB&TW&Zahlencode \textcolor{red}{$\boldsymbol{Grundtext}$} Umschrift $|$"Ubersetzung(en)\\
1.&180.&2261.&704.&8899.&1.&3&160&40\_50\_70 \textcolor{red}{\textcjheb{`nm}} MNa $|$wer zur"uckh"alt/den Zur"uckhaltenden\\
2.&181.&2262.&707.&8902.&4.&2&202&2\_200 \textcolor{red}{\textcjheb{rb}} BR $|$Korn\\
3.&182.&2263.&709.&8904.&6.&5&123&10\_100\_2\_5\_6 \textcolor{red}{\textcjheb{whbqy}} JQBHW $|$den verflucht/sie (=es) werden verfluchen\\
4.&183.&2264.&714.&8909.&11.&4&77&30\_1\_6\_40 \textcolor{red}{\textcjheb{mw'l}} LAWM $|$(das) Volk\\
5.&184.&2265.&718.&8913.&15.&5&233&6\_2\_200\_20\_5 \textcolor{red}{\textcjheb{hkrbw}} WBRKH $|$aber Segen/und Segen\\
6.&185.&2266.&723.&8918.&20.&4&531&30\_200\_1\_300 \textcolor{red}{\textcjheb{+s'rl}} LRAS $|$wird dem Haupt dessen zuteil/"uber das Haupt\\
7.&186.&2267.&727.&8922.&24.&5&552&40\_300\_2\_10\_200 \textcolor{red}{\textcjheb{ryb+sm}} MSBJR $|$der Getreide verkauft/eines Getreide Verkaufenden\\
\end{tabular}\medskip \\
Ende des Verses 11.26\\
Verse: 314, Buchstaben: 28, 731, 8926, Totalwerte: 1878, 58830, 649671\\
\\
Wer Korn zur"uckh"alt, den verflucht das Volk; aber Segen wird dem Haupte dessen zuteil, der Getreide verkauft.\\
\newpage 
{\bf -- 11.27}\\
\medskip \\
\begin{tabular}{rrrrrrrrp{120mm}}
WV&WK&WB&ABK&ABB&ABV&AnzB&TW&Zahlencode \textcolor{red}{$\boldsymbol{Grundtext}$} Umschrift $|$"Ubersetzung(en)\\
1.&187.&2268.&732.&8927.&1.&3&508&300\_8\_200 \textcolor{red}{\textcjheb{r.h+s}} SCR $|$wer sucht eifrig/ein Suchender\\
2.&188.&2269.&735.&8930.&4.&3&17&9\_6\_2 \textcolor{red}{\textcjheb{bw.t}} tWB $|$(das) Gute(s)\\
3.&189.&2270.&738.&8933.&7.&4&412&10\_2\_100\_300 \textcolor{red}{\textcjheb{+sqby}} JBQS $|$((d)er) sucht\\
4.&190.&2271.&742.&8937.&11.&4&346&200\_90\_6\_50 \textcolor{red}{\textcjheb{nw.sr}} R"sWN $|$Wohlgefallen\\
5.&191.&2272.&746.&8941.&15.&4&510&6\_4\_200\_300 \textcolor{red}{\textcjheb{+srdw}} WDRS $|$wer aber trachtet/und ein Trachtender\\
6.&192.&2273.&750.&8945.&19.&3&275&200\_70\_5 \textcolor{red}{\textcjheb{h`r}} RaH $|$(nach) B"osem\\
7.&193.&2274.&753.&8948.&22.&6&465&400\_2\_6\_1\_50\_6 \textcolor{red}{\textcjheb{wn'wbt}} TBWANW $|$"uber ihn wird es kommen/er (=es) "uberkommt ihn\\
\end{tabular}\medskip \\
Ende des Verses 11.27\\
Verse: 315, Buchstaben: 27, 758, 8953, Totalwerte: 2533, 61363, 652204\\
\\
Wer das Gute eifrig sucht, sucht Wohlgefallen; wer aber nach B"osem trachtet, "uber ihn wird es kommen.\\
\newpage 
{\bf -- 11.28}\\
\medskip \\
\begin{tabular}{rrrrrrrrp{120mm}}
WV&WK&WB&ABK&ABB&ABV&AnzB&TW&Zahlencode \textcolor{red}{$\boldsymbol{Grundtext}$} Umschrift $|$"Ubersetzung(en)\\
1.&194.&2275.&759.&8954.&1.&4&25&2\_6\_9\_8 \textcolor{red}{\textcjheb{.h.twb}} BWtC $|$wer vertraut/(ein) Vertrauender\\
2.&195.&2276.&763.&8958.&5.&5&578&2\_70\_300\_200\_6 \textcolor{red}{\textcjheb{wr+s`b}} BaSRW $|$auf seinen Reichtum\\
3.&196.&2277.&768.&8963.&10.&3&12&5\_6\_1 \textcolor{red}{\textcjheb{'wh}} HWA $|$(d)er\\
4.&197.&2278.&771.&8966.&13.&3&120&10\_80\_30 \textcolor{red}{\textcjheb{lpy}} JPL $|$wird fallen/(er) st"urzt\\
5.&198.&2279.&774.&8969.&16.&5&131&6\_20\_70\_30\_5 \textcolor{red}{\textcjheb{hl`kw}} WKaLH $|$aber wie Laub/und wie das Laub\\
6.&199.&2280.&779.&8974.&21.&6&254&90\_4\_10\_100\_10\_40 \textcolor{red}{\textcjheb{myqyd.s}} "sDJQJM $|$(die) Gerechten\\
7.&200.&2281.&785.&8980.&27.&5&304&10\_80\_200\_8\_6 \textcolor{red}{\textcjheb{w.hrpy}} JPRCW $|$werden sprossen/(sie) sprie"sen\\
\end{tabular}\medskip \\
Ende des Verses 11.28\\
Verse: 316, Buchstaben: 31, 789, 8984, Totalwerte: 1424, 62787, 653628\\
\\
Wer auf seinen Reichtum vertraut, der wird fallen; aber die Gerechten werden sprossen wie Laub.\\
\newpage 
{\bf -- 11.29}\\
\medskip \\
\begin{tabular}{rrrrrrrrp{120mm}}
WV&WK&WB&ABK&ABB&ABV&AnzB&TW&Zahlencode \textcolor{red}{$\boldsymbol{Grundtext}$} Umschrift $|$"Ubersetzung(en)\\
1.&201.&2282.&790.&8985.&1.&4&296&70\_6\_20\_200 \textcolor{red}{\textcjheb{rkw`}} aWKR $|$wer verst"ort/wer in Unordnung bringt\\
2.&202.&2283.&794.&8989.&5.&4&418&2\_10\_400\_6 \textcolor{red}{\textcjheb{wtyb}} BJTW $|$sein Haus\\
3.&203.&2284.&798.&8993.&9.&4&98&10\_50\_8\_30 \textcolor{red}{\textcjheb{l.hny}} JNCL $|$wird erben/(er) erbt\\
4.&204.&2285.&802.&8997.&13.&3&214&200\_6\_8 \textcolor{red}{\textcjheb{.hwr}} RWC $|$Wind\\
5.&205.&2286.&805.&9000.&16.&4&82&6\_70\_2\_4 \textcolor{red}{\textcjheb{db`w}} WaBD $|$und ein Knecht dessen/und Knecht\\
6.&206.&2287.&809.&9004.&20.&4&47&1\_6\_10\_30 \textcolor{red}{\textcjheb{lyw'}} AWJL $|$(wird) der Tor\\
7.&207.&2288.&813.&9008.&24.&4&98&30\_8\_20\_40 \textcolor{red}{\textcjheb{mk.hl}} LCKM $|$der ist weisen/dem Weisen\\
8.&208.&2289.&817.&9012.&28.&2&32&30\_2 \textcolor{red}{\textcjheb{bl}} LB $|$(im) Herzen(s)\\
\end{tabular}\medskip \\
Ende des Verses 11.29\\
Verse: 317, Buchstaben: 29, 818, 9013, Totalwerte: 1285, 64072, 654913\\
\\
Wer sein Haus verst"ort, wird Wind erben; und der Narr wird ein Knecht dessen, der weisen Herzens ist.\\
\newpage 
{\bf -- 11.30}\\
\medskip \\
\begin{tabular}{rrrrrrrrp{120mm}}
WV&WK&WB&ABK&ABB&ABV&AnzB&TW&Zahlencode \textcolor{red}{$\boldsymbol{Grundtext}$} Umschrift $|$"Ubersetzung(en)\\
1.&209.&2290.&819.&9014.&1.&3&290&80\_200\_10 \textcolor{red}{\textcjheb{yrp}} PRJ $|$die Frucht\\
2.&210.&2291.&822.&9017.&4.&4&204&90\_4\_10\_100 \textcolor{red}{\textcjheb{qyd.s}} "sDJQ $|$(des) Gerechten\\
3.&211.&2292.&826.&9021.&8.&2&160&70\_90 \textcolor{red}{\textcjheb{.s`}} a"s $|$(ist) (ein) Baum\\
4.&212.&2293.&828.&9023.&10.&4&68&8\_10\_10\_40 \textcolor{red}{\textcjheb{myy.h}} CJJM $|$des Lebens\\
5.&213.&2294.&832.&9027.&14.&4&144&6\_30\_100\_8 \textcolor{red}{\textcjheb{.hqlw}} WLQC $|$und (es) gewinnt/und gewinnend\\
6.&214.&2295.&836.&9031.&18.&5&836&50\_80\_300\_6\_400 \textcolor{red}{\textcjheb{tw+spn}} NPSWT $|$Seelen\\
7.&215.&2296.&841.&9036.&23.&3&68&8\_20\_40 \textcolor{red}{\textcjheb{mk.h}} CKM $|$(ist) (der) Weise\\
\end{tabular}\medskip \\
Ende des Verses 11.30\\
Verse: 318, Buchstaben: 25, 843, 9038, Totalwerte: 1770, 65842, 656683\\
\\
Die Frucht des Gerechten ist ein Baum des Lebens, und der Weise gewinnt Seelen.\\
\newpage 
{\bf -- 11.31}\\
\medskip \\
\begin{tabular}{rrrrrrrrp{120mm}}
WV&WK&WB&ABK&ABB&ABV&AnzB&TW&Zahlencode \textcolor{red}{$\boldsymbol{Grundtext}$} Umschrift $|$"Ubersetzung(en)\\
1.&216.&2297.&844.&9039.&1.&2&55&5\_50 \textcolor{red}{\textcjheb{nh}} HN $|$siehe\\
2.&217.&2298.&846.&9041.&3.&4&204&90\_4\_10\_100 \textcolor{red}{\textcjheb{qyd.s}} "sDJQ $|$dem Gerechten/(einem) Rechtschaffenen\\
3.&218.&2299.&850.&9045.&7.&4&293&2\_1\_200\_90 \textcolor{red}{\textcjheb{.sr'b}} BAR"s $|$auf (der) Erde(n)\\
4.&219.&2300.&854.&9049.&11.&4&380&10\_300\_30\_40 \textcolor{red}{\textcjheb{ml+sy}} JSLM $|$(er) wird vergelten\\
5.&220.&2301.&858.&9053.&15.&2&81&1\_80 \textcolor{red}{\textcjheb{p'}} AP $|$wie viel\\
6.&221.&2302.&860.&9055.&17.&2&30&20\_10 \textcolor{red}{\textcjheb{yk}} KJ $|$mehr\\
7.&222.&2303.&862.&9057.&19.&3&570&200\_300\_70 \textcolor{red}{\textcjheb{`+sr}} RSa $|$dem Gesetzlosen/(einem) B"osen\\
8.&223.&2304.&865.&9060.&22.&5&30&6\_8\_6\_9\_1 \textcolor{red}{\textcjheb{'.tw.hw}} WCWtA $|$und S"under\\
\end{tabular}\medskip \\
Ende des Verses 11.31\\
Verse: 319, Buchstaben: 26, 869, 9064, Totalwerte: 1643, 67485, 658326\\
\\
Siehe, dem Gerechten wird auf Erden vergolten, wieviel mehr dem Gesetzlosen und S"under!\\
\\
{\bf Ende des Kapitels 11}\\
\newpage 
{\bf -- 12.1}\\
\medskip \\
\begin{tabular}{rrrrrrrrp{120mm}}
WV&WK&WB&ABK&ABB&ABV&AnzB&TW&Zahlencode \textcolor{red}{$\boldsymbol{Grundtext}$} Umschrift $|$"Ubersetzung(en)\\
1.&1.&2305.&1.&9065.&1.&3&8&1\_5\_2 \textcolor{red}{\textcjheb{bh'}} AHB $|$wer liebt/(ein) Liebender\\
2.&2.&2306.&4.&9068.&4.&4&306&40\_6\_60\_200 \textcolor{red}{\textcjheb{rswm}} MWsR $|$Unterweisung/Zucht\\
3.&3.&2307.&8.&9072.&8.&3&8&1\_5\_2 \textcolor{red}{\textcjheb{bh'}} AHB $|$liebt/(ist) liebend\\
4.&4.&2308.&11.&9075.&11.&3&474&4\_70\_400 \textcolor{red}{\textcjheb{t`d}} DaT $|$Erkenntnis\\
5.&5.&2309.&14.&9078.&14.&4&357&6\_300\_50\_1 \textcolor{red}{\textcjheb{'n+sw}} WSNA $|$und wer hasst/und ein Hassender\\
6.&6.&2310.&18.&9082.&18.&5&834&400\_6\_20\_8\_400 \textcolor{red}{\textcjheb{t.hkwt}} TWKCT $|$Zucht/Zurechtweisung\\
7.&7.&2311.&23.&9087.&23.&3&272&2\_70\_200 \textcolor{red}{\textcjheb{r`b}} BaR $|$ist dumm/(ist) wie Vieh\\
\end{tabular}\medskip \\
Ende des Verses 12.1\\
Verse: 320, Buchstaben: 25, 25, 9089, Totalwerte: 2259, 2259, 660585\\
\\
Wer Unterweisung liebt, liebt Erkenntnis; und wer Zucht ha"st, ist dumm.\\
\newpage 
{\bf -- 12.2}\\
\medskip \\
\begin{tabular}{rrrrrrrrp{120mm}}
WV&WK&WB&ABK&ABB&ABV&AnzB&TW&Zahlencode \textcolor{red}{$\boldsymbol{Grundtext}$} Umschrift $|$"Ubersetzung(en)\\
1.&8.&2312.&26.&9090.&1.&3&17&9\_6\_2 \textcolor{red}{\textcjheb{bw.t}} tWB $|$der G"utige/(ein) Guter\\
2.&9.&2313.&29.&9093.&4.&4&200&10\_80\_10\_100 \textcolor{red}{\textcjheb{qypy}} JPJQ $|$(er) erlangt\\
3.&10.&2314.&33.&9097.&8.&4&346&200\_90\_6\_50 \textcolor{red}{\textcjheb{nw.sr}} R"sWN $|$Wohlgefallen\\
4.&11.&2315.&37.&9101.&12.&5&66&40\_10\_5\_6\_5 \textcolor{red}{\textcjheb{hwhym}} MJHWH $|$von Jahwe\\
5.&12.&2316.&42.&9106.&17.&4&317&6\_1\_10\_300 \textcolor{red}{\textcjheb{+sy'w}} WAJS $|$aber den Mann/und (einen) Mann\\
6.&13.&2317.&46.&9110.&21.&5&493&40\_7\_40\_6\_400 \textcolor{red}{\textcjheb{twmzm}} MZMWT $|$der R"anke/(von) R"anken\\
7.&14.&2318.&51.&9115.&26.&5&590&10\_200\_300\_10\_70 \textcolor{red}{\textcjheb{`y+sry}} JRSJa $|$spricht er schuldig/er erkl"art f"ur schuldig\\
\end{tabular}\medskip \\
Ende des Verses 12.2\\
Verse: 321, Buchstaben: 30, 55, 9119, Totalwerte: 2029, 4288, 662614\\
\\
Der G"utige erlangt Wohlgefallen von Jahwe, aber den Mann der R"anke spricht er schuldig.\\
\newpage 
{\bf -- 12.3}\\
\medskip \\
\begin{tabular}{rrrrrrrrp{120mm}}
WV&WK&WB&ABK&ABB&ABV&AnzB&TW&Zahlencode \textcolor{red}{$\boldsymbol{Grundtext}$} Umschrift $|$"Ubersetzung(en)\\
1.&15.&2319.&56.&9120.&1.&2&31&30\_1 \textcolor{red}{\textcjheb{'l}} LA $|$nicht\\
2.&16.&2320.&58.&9122.&3.&4&86&10\_20\_6\_50 \textcolor{red}{\textcjheb{nwky}} JKWN $|$wird bestehen/er (=es) hat Bestand\\
3.&17.&2321.&62.&9126.&7.&3&45&1\_4\_40 \textcolor{red}{\textcjheb{md'}} ADM $|$(ein) Mensch\\
4.&18.&2322.&65.&9129.&10.&4&572&2\_200\_300\_70 \textcolor{red}{\textcjheb{`+srb}} BRSa $|$durch Gesetzlosigkeit/durch Frevel\\
5.&19.&2323.&69.&9133.&14.&4&806&6\_300\_200\_300 \textcolor{red}{\textcjheb{+sr+sw}} WSRS $|$aber die Wurzel/und die Wurzel\\
6.&20.&2324.&73.&9137.&18.&6&254&90\_4\_10\_100\_10\_40 \textcolor{red}{\textcjheb{myqyd.s}} "sDJQJM $|$der Gerechten\\
7.&21.&2325.&79.&9143.&24.&2&32&2\_30 \textcolor{red}{\textcjheb{lb}} BL $|$nicht\\
8.&22.&2326.&81.&9145.&26.&4&65&10\_40\_6\_9 \textcolor{red}{\textcjheb{.twmy}} JMWt $|$wird ersch"uttert werden/sie (=es) wankt\\
\end{tabular}\medskip \\
Ende des Verses 12.3\\
Verse: 322, Buchstaben: 29, 84, 9148, Totalwerte: 1891, 6179, 664505\\
\\
Ein Mensch wird nicht bestehen durch Gesetzlosigkeit, aber die Wurzel der Gerechten wird nicht ersch"uttert werden.\\
\newpage 
{\bf -- 12.4}\\
\medskip \\
\begin{tabular}{rrrrrrrrp{120mm}}
WV&WK&WB&ABK&ABB&ABV&AnzB&TW&Zahlencode \textcolor{red}{$\boldsymbol{Grundtext}$} Umschrift $|$"Ubersetzung(en)\\
1.&23.&2327.&85.&9149.&1.&3&701&1\_300\_400 \textcolor{red}{\textcjheb{t+s'}} AST $|$(eine) Frau\\
2.&24.&2328.&88.&9152.&4.&3&48&8\_10\_30 \textcolor{red}{\textcjheb{ly.h}} CJL $|$wackere/tatkr"aftige\\
3.&25.&2329.&91.&9155.&7.&4&679&70\_9\_200\_400 \textcolor{red}{\textcjheb{tr.t`}} atRT $|$(ist) die Krone\\
4.&26.&2330.&95.&9159.&11.&4&107&2\_70\_30\_5 \textcolor{red}{\textcjheb{hl`b}} BaLH $|$ihres Mannes/ihres Eheherrn\\
5.&27.&2331.&99.&9163.&15.&5&328&6\_20\_200\_100\_2 \textcolor{red}{\textcjheb{bqrkw}} WKRQB $|$aber wie F"aulnis/und wie F"aulnis\\
6.&28.&2332.&104.&9168.&20.&8&624&2\_70\_90\_40\_6\_400\_10\_6 \textcolor{red}{\textcjheb{wytwm.s`b}} Ba"sMWTJW $|$in seinen Gebeinen/in seinen Knochen\\
7.&29.&2333.&112.&9176.&28.&5&357&40\_2\_10\_300\_5 \textcolor{red}{\textcjheb{h+sybm}} MBJSH $|$eine Sch"andliche\\
\end{tabular}\medskip \\
Ende des Verses 12.4\\
Verse: 323, Buchstaben: 32, 116, 9180, Totalwerte: 2844, 9023, 667349\\
\\
Ein wackeres Weib ist ihres Mannes Krone, aber wie F"aulnis in seinen Gebeinen ist ein sch"andliches.\\
\newpage 
{\bf -- 12.5}\\
\medskip \\
\begin{tabular}{rrrrrrrrp{120mm}}
WV&WK&WB&ABK&ABB&ABV&AnzB&TW&Zahlencode \textcolor{red}{$\boldsymbol{Grundtext}$} Umschrift $|$"Ubersetzung(en)\\
1.&30.&2334.&117.&9181.&1.&6&756&40\_8\_300\_2\_6\_400 \textcolor{red}{\textcjheb{twb+s.hm}} MCSBWT $|$die Gedanken\\
2.&31.&2335.&123.&9187.&7.&6&254&90\_4\_10\_100\_10\_40 \textcolor{red}{\textcjheb{myqyd.s}} "sDJQJM $|$der Gerechten\\
3.&32.&2336.&129.&9193.&13.&4&429&40\_300\_80\_9 \textcolor{red}{\textcjheb{.tp+sm}} MSPt $|$(sind) das Recht\\
4.&33.&2337.&133.&9197.&17.&6&846&400\_8\_2\_30\_6\_400 \textcolor{red}{\textcjheb{twlb.ht}} TCBLWT $|$(die) "Uberlegungen\\
5.&34.&2338.&139.&9203.&23.&5&620&200\_300\_70\_10\_40 \textcolor{red}{\textcjheb{my`+sr}} RSaJM $|$der Gesetzlosen/(der) Frevler\\
6.&35.&2339.&144.&9208.&28.&4&285&40\_200\_40\_5 \textcolor{red}{\textcjheb{hmrm}} MRMH $|$(sind) Betrug\\
\end{tabular}\medskip \\
Ende des Verses 12.5\\
Verse: 324, Buchstaben: 31, 147, 9211, Totalwerte: 3190, 12213, 670539\\
\\
Die Gedanken der Gerechten sind Recht, die "Uberlegungen der Gesetzlosen sind Betrug.\\
\newpage 
{\bf -- 12.6}\\
\medskip \\
\begin{tabular}{rrrrrrrrp{120mm}}
WV&WK&WB&ABK&ABB&ABV&AnzB&TW&Zahlencode \textcolor{red}{$\boldsymbol{Grundtext}$} Umschrift $|$"Ubersetzung(en)\\
1.&36.&2340.&148.&9212.&1.&4&216&4\_2\_200\_10 \textcolor{red}{\textcjheb{yrbd}} DBRJ $|$die Worte\\
2.&37.&2341.&152.&9216.&5.&5&620&200\_300\_70\_10\_40 \textcolor{red}{\textcjheb{my`+sr}} RSaJM $|$der Gesetzlosen/(von) Frevlern\\
3.&38.&2342.&157.&9221.&10.&3&203&1\_200\_2 \textcolor{red}{\textcjheb{br'}} ARB $|$(sind ein) Lauern\\
4.&39.&2343.&160.&9224.&13.&2&44&4\_40 \textcolor{red}{\textcjheb{md}} DM $|$(auf) Blut\\
5.&40.&2344.&162.&9226.&15.&3&96&6\_80\_10 \textcolor{red}{\textcjheb{ypw}} WPJ $|$aber der Mund/und der Mund\\
6.&41.&2345.&165.&9229.&18.&5&560&10\_300\_200\_10\_40 \textcolor{red}{\textcjheb{myr+sy}} JSRJM $|$der Aufrichtigen/der Geraden\\
7.&42.&2346.&170.&9234.&23.&5&180&10\_90\_10\_30\_40 \textcolor{red}{\textcjheb{mly.sy}} J"sJLM $|$(er) errettet sie\\
\end{tabular}\medskip \\
Ende des Verses 12.6\\
Verse: 325, Buchstaben: 27, 174, 9238, Totalwerte: 1919, 14132, 672458\\
\\
Die Worte der Gesetzlosen sind ein Lauern auf Blut; aber der Mund der Aufrichtigen errettet sie.\\
\newpage 
{\bf -- 12.7}\\
\medskip \\
\begin{tabular}{rrrrrrrrp{120mm}}
WV&WK&WB&ABK&ABB&ABV&AnzB&TW&Zahlencode \textcolor{red}{$\boldsymbol{Grundtext}$} Umschrift $|$"Ubersetzung(en)\\
1.&43.&2347.&175.&9239.&1.&4&111&5\_80\_6\_20 \textcolor{red}{\textcjheb{kwph}} HPWK $|$man kehrt um/es st"urzen\\
2.&44.&2348.&179.&9243.&5.&5&620&200\_300\_70\_10\_40 \textcolor{red}{\textcjheb{my`+sr}} RSaJM $|$die Gesetzlosen/(die) Frevler\\
3.&45.&2349.&184.&9248.&10.&5&107&6\_1\_10\_50\_40 \textcolor{red}{\textcjheb{mny'w}} WAJNM $|$und nicht sind sie (mehr)\\
4.&46.&2350.&189.&9253.&15.&4&418&6\_2\_10\_400 \textcolor{red}{\textcjheb{tybw}} WBJT $|$aber das Haus/und das Haus\\
5.&47.&2351.&193.&9257.&19.&6&254&90\_4\_10\_100\_10\_40 \textcolor{red}{\textcjheb{myqyd.s}} "sDJQJM $|$der Gerechten\\
6.&48.&2352.&199.&9263.&25.&4&124&10\_70\_40\_4 \textcolor{red}{\textcjheb{dm`y}} JaMD $|$(er (=es)) bleibt bestehen\\
\end{tabular}\medskip \\
Ende des Verses 12.7\\
Verse: 326, Buchstaben: 28, 202, 9266, Totalwerte: 1634, 15766, 674092\\
\\
Man kehrt die Gesetzlosen um, und sie sind nicht mehr; aber das Haus der Gerechten bleibt bestehen.\\
\newpage 
{\bf -- 12.8}\\
\medskip \\
\begin{tabular}{rrrrrrrrp{120mm}}
WV&WK&WB&ABK&ABB&ABV&AnzB&TW&Zahlencode \textcolor{red}{$\boldsymbol{Grundtext}$} Umschrift $|$"Ubersetzung(en)\\
1.&49.&2353.&203.&9267.&1.&3&120&30\_80\_10 \textcolor{red}{\textcjheb{ypl}} LPJ $|$gem"a"s (der Aussage)\\
2.&50.&2354.&206.&9270.&4.&4&356&300\_20\_30\_6 \textcolor{red}{\textcjheb{wlk+s}} SKLW $|$seiner Einsicht/seines Verstandes\\
3.&51.&2355.&210.&9274.&8.&4&75&10\_5\_30\_30 \textcolor{red}{\textcjheb{llhy}} JHLL $|$wird gelobt/er (=es) wird gepriesen\\
4.&52.&2356.&214.&9278.&12.&3&311&1\_10\_300 \textcolor{red}{\textcjheb{+sy'}} AJS $|$(ein) Mann\\
5.&53.&2357.&217.&9281.&15.&5&137&6\_50\_70\_6\_5 \textcolor{red}{\textcjheb{hw`nw}} WNaWH $|$wer aber ist verkehrten/und ein Verkehrter\\
6.&54.&2358.&222.&9286.&20.&2&32&30\_2 \textcolor{red}{\textcjheb{bl}} LB $|$(im) Herzen(s)\\
7.&55.&2359.&224.&9288.&22.&4&30&10\_5\_10\_5 \textcolor{red}{\textcjheb{hyhy}} JHJH $|$(er) wird sein\\
8.&56.&2360.&228.&9292.&26.&4&45&30\_2\_6\_7 \textcolor{red}{\textcjheb{zwbl}} LBWZ $|$(zu) der Verachtung\\
\end{tabular}\medskip \\
Ende des Verses 12.8\\
Verse: 327, Buchstaben: 29, 231, 9295, Totalwerte: 1106, 16872, 675198\\
\\
Gem"a"s seiner Einsicht wird ein Mann gelobt; wer aber verkehrten Herzens ist, wird zur Verachtung sein.\\
\newpage 
{\bf -- 12.9}\\
\medskip \\
\begin{tabular}{rrrrrrrrp{120mm}}
WV&WK&WB&ABK&ABB&ABV&AnzB&TW&Zahlencode \textcolor{red}{$\boldsymbol{Grundtext}$} Umschrift $|$"Ubersetzung(en)\\
1.&57.&2361.&232.&9296.&1.&3&17&9\_6\_2 \textcolor{red}{\textcjheb{bw.t}} tWB $|$besser/gut (ist)\\
2.&58.&2362.&235.&9299.&4.&4&185&50\_100\_30\_5 \textcolor{red}{\textcjheb{hlqn}} NQLH $|$wer gering ist/ein gering Gesch"atzter\\
3.&59.&2363.&239.&9303.&8.&4&82&6\_70\_2\_4 \textcolor{red}{\textcjheb{db`w}} WaBD $|$und (der) (einen) Knecht\\
4.&60.&2364.&243.&9307.&12.&2&36&30\_6 \textcolor{red}{\textcjheb{wl}} LW $|$hat/f"ur sich (hat)\\
5.&61.&2365.&245.&9309.&14.&6&506&40\_40\_400\_20\_2\_4 \textcolor{red}{\textcjheb{dbktmm}} MMTKBD $|$als wer vornehm tut/mehr als ein sich Br"ustender\\
6.&62.&2366.&251.&9315.&20.&4&274&6\_8\_60\_200 \textcolor{red}{\textcjheb{rs.hw}} WCsR $|$und Mangel hat an/und Ermangelnder (ist)\\
7.&63.&2367.&255.&9319.&24.&3&78&30\_8\_40 \textcolor{red}{\textcjheb{m.hl}} LCM $|$Brot\\
\end{tabular}\medskip \\
Ende des Verses 12.9\\
Verse: 328, Buchstaben: 26, 257, 9321, Totalwerte: 1178, 18050, 676376\\
\\
Besser, wer gering ist und einen Knecht hat, als wer vornehm tut und hat Mangel an Brot.\\
\newpage 
{\bf -- 12.10}\\
\medskip \\
\begin{tabular}{rrrrrrrrp{120mm}}
WV&WK&WB&ABK&ABB&ABV&AnzB&TW&Zahlencode \textcolor{red}{$\boldsymbol{Grundtext}$} Umschrift $|$"Ubersetzung(en)\\
1.&64.&2368.&258.&9322.&1.&4&90&10\_6\_4\_70 \textcolor{red}{\textcjheb{`dwy}} JWDa $|$(es) k"ummert sich/kennend (ist)\\
2.&65.&2369.&262.&9326.&5.&4&204&90\_4\_10\_100 \textcolor{red}{\textcjheb{qyd.s}} "sDJQ $|$der Gerechte/(ein) Rechtschaffener\\
3.&66.&2370.&266.&9330.&9.&3&430&50\_80\_300 \textcolor{red}{\textcjheb{+spn}} NPS $|$um das Leben/die Seele\\
4.&67.&2371.&269.&9333.&12.&5&453&2\_5\_40\_400\_6 \textcolor{red}{\textcjheb{wtmhb}} BHMTW $|$seines Viehs\\
5.&68.&2372.&274.&9338.&17.&5&264&6\_200\_8\_40\_10 \textcolor{red}{\textcjheb{ym.hrw}} WRCMJ $|$aber das Herz/und das Innere\\
6.&69.&2373.&279.&9343.&22.&5&620&200\_300\_70\_10\_40 \textcolor{red}{\textcjheb{my`+sr}} RSaJM $|$der Gesetzlosen/(der) Frevler\\
7.&70.&2374.&284.&9348.&27.&5&238&1\_20\_7\_200\_10 \textcolor{red}{\textcjheb{yrzk'}} AKZRJ $|$(ist) grausam\\
\end{tabular}\medskip \\
Ende des Verses 12.10\\
Verse: 329, Buchstaben: 31, 288, 9352, Totalwerte: 2299, 20349, 678675\\
\\
Der Gerechte k"ummert sich um das Leben seines Viehes, aber das Herz der Gesetzlosen ist grausam.\\
\newpage 
{\bf -- 12.11}\\
\medskip \\
\begin{tabular}{rrrrrrrrp{120mm}}
WV&WK&WB&ABK&ABB&ABV&AnzB&TW&Zahlencode \textcolor{red}{$\boldsymbol{Grundtext}$} Umschrift $|$"Ubersetzung(en)\\
1.&71.&2375.&289.&9353.&1.&3&76&70\_2\_4 \textcolor{red}{\textcjheb{db`}} aBD $|$wer bebaut/(ein) Bestellender\\
2.&72.&2376.&292.&9356.&4.&5&451&1\_4\_40\_400\_6 \textcolor{red}{\textcjheb{wtmd'}} ADMTW $|$sein Land/seinen Boden\\
3.&73.&2377.&297.&9361.&9.&4&382&10\_300\_2\_70 \textcolor{red}{\textcjheb{`b+sy}} JSBa $|$wird ges"attigt werden/(er) wird satt\\
4.&74.&2378.&301.&9365.&13.&3&78&30\_8\_40 \textcolor{red}{\textcjheb{m.hl}} LCM $|$(mit) Brot\\
5.&75.&2379.&304.&9368.&16.&5&330&6\_40\_200\_4\_80 \textcolor{red}{\textcjheb{pdrmw}} WMRDP $|$wer aber nachjagt/und ein Nachjagender\\
6.&76.&2380.&309.&9373.&21.&5&360&200\_10\_100\_10\_40 \textcolor{red}{\textcjheb{myqyr}} RJQJM $|$nichtigen Dingen/leeren Dingen\\
7.&77.&2381.&314.&9378.&26.&3&268&8\_60\_200 \textcolor{red}{\textcjheb{rs.h}} CsR $|$ist un-/(ist) ermangelnd\\
8.&78.&2382.&317.&9381.&29.&2&32&30\_2 \textcolor{red}{\textcjheb{bl}} LB $|$verst"andig/Herz (=Verstand)\\
\end{tabular}\medskip \\
Ende des Verses 12.11\\
Verse: 330, Buchstaben: 30, 318, 9382, Totalwerte: 1977, 22326, 680652\\
\\
Wer sein Land bebaut, wird mit Brot ges"attigt werden; wer aber nichtigen Dingen nachjagt, ist unverst"andig.\\
\newpage 
{\bf -- 12.12}\\
\medskip \\
\begin{tabular}{rrrrrrrrp{120mm}}
WV&WK&WB&ABK&ABB&ABV&AnzB&TW&Zahlencode \textcolor{red}{$\boldsymbol{Grundtext}$} Umschrift $|$"Ubersetzung(en)\\
1.&79.&2383.&319.&9383.&1.&3&52&8\_40\_4 \textcolor{red}{\textcjheb{dm.h}} CMD $|$(es) gel"ustete/er (=es) begehrt\\
2.&80.&2384.&322.&9386.&4.&3&570&200\_300\_70 \textcolor{red}{\textcjheb{`+sr}} RSa $|$den Gesetzlosen/(ein) Frevler\\
3.&81.&2385.&325.&9389.&7.&4&140&40\_90\_6\_4 \textcolor{red}{\textcjheb{dw.sm}} M"sWD $|$nach dem Raub/das Fangnetz\\
4.&82.&2386.&329.&9393.&11.&4&320&200\_70\_10\_40 \textcolor{red}{\textcjheb{my`r}} RaJM $|$der B"osen\\
5.&83.&2387.&333.&9397.&15.&4&806&6\_300\_200\_300 \textcolor{red}{\textcjheb{+sr+sw}} WSRS $|$aber die Wurzel/und die Wurzel\\
6.&84.&2388.&337.&9401.&19.&6&254&90\_4\_10\_100\_10\_40 \textcolor{red}{\textcjheb{myqyd.s}} "sDJQJM $|$der Gerechten/der Rechtschaffenen\\
7.&85.&2389.&343.&9407.&25.&3&460&10\_400\_50 \textcolor{red}{\textcjheb{nty}} JTN $|$tr"agt ein/er (=sie) ist ergiebig\\
\end{tabular}\medskip \\
Ende des Verses 12.12\\
Verse: 331, Buchstaben: 27, 345, 9409, Totalwerte: 2602, 24928, 683254\\
\\
Den Gesetzlosen gel"ustete nach dem Raube der B"osen, aber die Wurzel der Gerechten tr"agt ein.\\
\newpage 
{\bf -- 12.13}\\
\medskip \\
\begin{tabular}{rrrrrrrrp{120mm}}
WV&WK&WB&ABK&ABB&ABV&AnzB&TW&Zahlencode \textcolor{red}{$\boldsymbol{Grundtext}$} Umschrift $|$"Ubersetzung(en)\\
1.&86.&2390.&346.&9410.&1.&4&452&2\_80\_300\_70 \textcolor{red}{\textcjheb{`+spb}} BPSa $|$in der "Ubertretung/in der Verfehlung\\
2.&87.&2391.&350.&9414.&5.&5&830&300\_80\_400\_10\_40 \textcolor{red}{\textcjheb{mytp+s}} SPTJM $|$der Lippen\\
3.&88.&2392.&355.&9419.&10.&4&446&40\_6\_100\_300 \textcolor{red}{\textcjheb{+sqwm}} MWQS $|$ist ein Fallstrick/(ist) (der) Fallstrick\\
4.&89.&2393.&359.&9423.&14.&2&270&200\_70 \textcolor{red}{\textcjheb{`r}} Ra $|$b"oser/(des) B"osen\\
5.&90.&2394.&361.&9425.&16.&4&107&6\_10\_90\_1 \textcolor{red}{\textcjheb{'.syw}} WJ"sA $|$aber (es) entgeht/und er (=es) kommt heraus\\
6.&91.&2395.&365.&9429.&20.&4&335&40\_90\_200\_5 \textcolor{red}{\textcjheb{hr.sm}} M"sRH $|$der Drangsal/aus der Bedr"angnis\\
7.&92.&2396.&369.&9433.&24.&4&204&90\_4\_10\_100 \textcolor{red}{\textcjheb{qyd.s}} "sDJQ $|$der Gerechte/(ein) Rechtschaffener\\
\end{tabular}\medskip \\
Ende des Verses 12.13\\
Verse: 332, Buchstaben: 27, 372, 9436, Totalwerte: 2644, 27572, 685898\\
\\
In der "Ubertretung der Lippen ist ein b"oser Fallstrick, aber der Gerechte entgeht der Drangsal.\\
\newpage 
{\bf -- 12.14}\\
\medskip \\
\begin{tabular}{rrrrrrrrp{120mm}}
WV&WK&WB&ABK&ABB&ABV&AnzB&TW&Zahlencode \textcolor{red}{$\boldsymbol{Grundtext}$} Umschrift $|$"Ubersetzung(en)\\
1.&93.&2397.&373.&9437.&1.&4&330&40\_80\_200\_10 \textcolor{red}{\textcjheb{yrpm}} MPRJ $|$von der Frucht\\
2.&94.&2398.&377.&9441.&5.&2&90&80\_10 \textcolor{red}{\textcjheb{yp}} PJ $|$seines Mundes/des Mundes\\
3.&95.&2399.&379.&9443.&7.&3&311&1\_10\_300 \textcolor{red}{\textcjheb{+sy'}} AJS $|$(ein(es)) Mann(es)\\
4.&96.&2400.&382.&9446.&10.&4&382&10\_300\_2\_70 \textcolor{red}{\textcjheb{`b+sy}} JSBa $|$wird ges"attigt/er wird satt\\
5.&97.&2401.&386.&9450.&14.&3&17&9\_6\_2 \textcolor{red}{\textcjheb{bw.t}} tWB $|$(mit) Gutem\\
6.&98.&2402.&389.&9453.&17.&5&85&6\_3\_40\_6\_30 \textcolor{red}{\textcjheb{lwmgw}} WGMWL $|$und das Tun/und die Tat\\
7.&99.&2403.&394.&9458.&22.&3&24&10\_4\_10 \textcolor{red}{\textcjheb{ydy}} JDJ $|$der H"ande\\
8.&100.&2404.&397.&9461.&25.&3&45&1\_4\_40 \textcolor{red}{\textcjheb{md'}} ADM $|$(eines) Menschen\\
9.&101.&2405.&400.&9464.&28.&4&318&10\_300\_6\_2 \textcolor{red}{\textcjheb{bw+sy}} JSWB $|$kehrt zur"uck/er (=sie) f"allt zur"uck\\
10.&102.&2406.&404.&9468.&32.&2&36&30\_6 \textcolor{red}{\textcjheb{wl}} LW $|$zu ihm\\
\end{tabular}\medskip \\
Ende des Verses 12.14\\
Verse: 333, Buchstaben: 33, 405, 9469, Totalwerte: 1638, 29210, 687536\\
\\
Von der Frucht seines Mundes wird ein Mann mit Gutem ges"attigt, und das Tun der H"ande eines Menschen kehrt zu ihm zur"uck.\\
\newpage 
{\bf -- 12.15}\\
\medskip \\
\begin{tabular}{rrrrrrrrp{120mm}}
WV&WK&WB&ABK&ABB&ABV&AnzB&TW&Zahlencode \textcolor{red}{$\boldsymbol{Grundtext}$} Umschrift $|$"Ubersetzung(en)\\
1.&103.&2407.&406.&9470.&1.&3&224&4\_200\_20 \textcolor{red}{\textcjheb{krd}} DRK $|$der Weg\\
2.&104.&2408.&409.&9473.&4.&4&47&1\_6\_10\_30 \textcolor{red}{\textcjheb{lyw'}} AWJL $|$des Narren/(eines) Toren\\
3.&105.&2409.&413.&9477.&8.&3&510&10\_300\_200 \textcolor{red}{\textcjheb{r+sy}} JSR $|$ist richtig/ist gerade\\
4.&106.&2410.&416.&9480.&11.&6&148&2\_70\_10\_50\_10\_6 \textcolor{red}{\textcjheb{wyny`b}} BaJNJW $|$in seinen Augen\\
5.&107.&2411.&422.&9486.&17.&4&416&6\_300\_40\_70 \textcolor{red}{\textcjheb{`m+sw}} WSMa $|$aber (es) h"ort/und h"orend(er)\\
6.&108.&2412.&426.&9490.&21.&4&195&30\_70\_90\_5 \textcolor{red}{\textcjheb{h.s`l}} La"sH $|$auf Rat\\
7.&109.&2413.&430.&9494.&25.&3&68&8\_20\_40 \textcolor{red}{\textcjheb{mk.h}} CKM $|$(ist) (der) Weise\\
\end{tabular}\medskip \\
Ende des Verses 12.15\\
Verse: 334, Buchstaben: 27, 432, 9496, Totalwerte: 1608, 30818, 689144\\
\\
Der Weg des Narren ist richtig in seinen Augen, aber der Weise h"ort auf Rat.\\
\newpage 
{\bf -- 12.16}\\
\medskip \\
\begin{tabular}{rrrrrrrrp{120mm}}
WV&WK&WB&ABK&ABB&ABV&AnzB&TW&Zahlencode \textcolor{red}{$\boldsymbol{Grundtext}$} Umschrift $|$"Ubersetzung(en)\\
1.&110.&2414.&433.&9497.&1.&4&47&1\_6\_10\_30 \textcolor{red}{\textcjheb{lyw'}} AWJL $|$des Narren/(einem) Toren\\
2.&111.&2415.&437.&9501.&5.&4&58&2\_10\_6\_40 \textcolor{red}{\textcjheb{mwyb}} BJWM $|$am (selben) Tag\\
3.&112.&2416.&441.&9505.&9.&4&90&10\_6\_4\_70 \textcolor{red}{\textcjheb{`dwy}} JWDa $|$(er (=es)) tut sich kund\\
4.&113.&2417.&445.&9509.&13.&4&156&20\_70\_60\_6 \textcolor{red}{\textcjheb{ws`k}} KasW $|$der Unmut/sein "Arger\\
5.&114.&2418.&449.&9513.&17.&4&91&6\_20\_60\_5 \textcolor{red}{\textcjheb{hskw}} WKsH $|$aber (es) verbirgt/und es verbirgt\\
6.&115.&2419.&453.&9517.&21.&4&186&100\_30\_6\_50 \textcolor{red}{\textcjheb{nwlq}} QLWN $|$den Schimpf/Schmach\\
7.&116.&2420.&457.&9521.&25.&4&316&70\_200\_6\_40 \textcolor{red}{\textcjheb{mwr`}} aRWM $|$der Kluge\\
\end{tabular}\medskip \\
Ende des Verses 12.16\\
Verse: 335, Buchstaben: 28, 460, 9524, Totalwerte: 944, 31762, 690088\\
\\
Der Unmut des Narren tut sich am selben Tage kund, aber der Kluge verbirgt den Schimpf.\\
\newpage 
{\bf -- 12.17}\\
\medskip \\
\begin{tabular}{rrrrrrrrp{120mm}}
WV&WK&WB&ABK&ABB&ABV&AnzB&TW&Zahlencode \textcolor{red}{$\boldsymbol{Grundtext}$} Umschrift $|$"Ubersetzung(en)\\
1.&117.&2421.&461.&9525.&1.&4&108&10\_80\_10\_8 \textcolor{red}{\textcjheb{.hypy}} JPJC $|$wer ausspricht/(w)er hervorbringt\\
2.&118.&2422.&465.&9529.&5.&5&102&1\_40\_6\_50\_5 \textcolor{red}{\textcjheb{hnwm'}} AMWNH $|$Wahrheit\\
3.&119.&2423.&470.&9534.&10.&4&27&10\_3\_10\_4 \textcolor{red}{\textcjheb{dygy}} JGJD $|$tut kund/(er) sagt aus\\
4.&120.&2424.&474.&9538.&14.&3&194&90\_4\_100 \textcolor{red}{\textcjheb{qd.s}} "sDQ $|$Gerechtigkeit/Rechtes\\
5.&121.&2425.&477.&9541.&17.&3&80&6\_70\_4 \textcolor{red}{\textcjheb{d`w}} WaD $|$aber ein Zeuge/und ein Zeuge\\
6.&122.&2426.&480.&9544.&20.&5&650&300\_100\_200\_10\_40 \textcolor{red}{\textcjheb{myrq+s}} SQRJM $|$falscher/(von) L"ugen\\
7.&123.&2427.&485.&9549.&25.&4&285&40\_200\_40\_5 \textcolor{red}{\textcjheb{hmrm}} MRMH $|$Trug/Betrug\\
\end{tabular}\medskip \\
Ende des Verses 12.17\\
Verse: 336, Buchstaben: 28, 488, 9552, Totalwerte: 1446, 33208, 691534\\
\\
Wer Wahrheit ausspricht, tut Gerechtigkeit kund, aber ein falscher Zeuge Trug.\\
\newpage 
{\bf -- 12.18}\\
\medskip \\
\begin{tabular}{rrrrrrrrp{120mm}}
WV&WK&WB&ABK&ABB&ABV&AnzB&TW&Zahlencode \textcolor{red}{$\boldsymbol{Grundtext}$} Umschrift $|$"Ubersetzung(en)\\
1.&124.&2428.&489.&9553.&1.&2&310&10\_300 \textcolor{red}{\textcjheb{+sy}} JS $|$(da) ist\\
2.&125.&2429.&491.&9555.&3.&4&22&2\_6\_9\_5 \textcolor{red}{\textcjheb{h.twb}} BWtH $|$einer der redet unbesonnene Worte/ein Schwatzender\\
3.&126.&2430.&495.&9559.&7.&7&770&20\_40\_4\_100\_200\_6\_400 \textcolor{red}{\textcjheb{twrqdmk}} KMDQRWT $|$gleichen Stichen/wie Stiche\\
4.&127.&2431.&502.&9566.&14.&3&210&8\_200\_2 \textcolor{red}{\textcjheb{br.h}} CRB $|$(des) Schwert(es)\\
5.&128.&2432.&505.&9569.&17.&5&392&6\_30\_300\_6\_50 \textcolor{red}{\textcjheb{nw+slw}} WLSWN $|$aber die Zunge/und die Zunge\\
6.&129.&2433.&510.&9574.&22.&5&118&8\_20\_40\_10\_40 \textcolor{red}{\textcjheb{mymk.h}} CKMJM $|$der Weisen/(von) Weisen\\
7.&130.&2434.&515.&9579.&27.&4&321&40\_200\_80\_1 \textcolor{red}{\textcjheb{'prm}} MRPA $|$ist Heilung/(bringt) Heilung\\
\end{tabular}\medskip \\
Ende des Verses 12.18\\
Verse: 337, Buchstaben: 30, 518, 9582, Totalwerte: 2143, 35351, 693677\\
\\
Da ist einer, der unbesonnene Worte redet gleich Schwertstichen; aber die Zunge der Weisen ist Heilung.\\
\newpage 
{\bf -- 12.19}\\
\medskip \\
\begin{tabular}{rrrrrrrrp{120mm}}
WV&WK&WB&ABK&ABB&ABV&AnzB&TW&Zahlencode \textcolor{red}{$\boldsymbol{Grundtext}$} Umschrift $|$"Ubersetzung(en)\\
1.&131.&2435.&519.&9583.&1.&3&780&300\_80\_400 \textcolor{red}{\textcjheb{tp+s}} SPT $|$die Lippe\\
2.&132.&2436.&522.&9586.&4.&3&441&1\_40\_400 \textcolor{red}{\textcjheb{tm'}} AMT $|$(der) Wahrheit\\
3.&133.&2437.&525.&9589.&7.&4&476&400\_20\_6\_50 \textcolor{red}{\textcjheb{nwkt}} TKWN $|$besteht/ist gegr"undet\\
4.&134.&2438.&529.&9593.&11.&3&104&30\_70\_4 \textcolor{red}{\textcjheb{d`l}} LaD $|$ewiglich/f"ur immer\\
5.&135.&2439.&532.&9596.&14.&3&80&6\_70\_4 \textcolor{red}{\textcjheb{d`w}} WaD $|$aber nur/und bis\\
6.&136.&2440.&535.&9599.&17.&6&289&1\_200\_3\_10\_70\_5 \textcolor{red}{\textcjheb{h`ygr'}} ARGJaH $|$einen Augenblick/ich Ruhe habe\\
7.&137.&2441.&541.&9605.&23.&4&386&30\_300\_6\_50 \textcolor{red}{\textcjheb{nw+sl}} LSWN $|$die Zunge/eine Zunge\\
8.&138.&2442.&545.&9609.&27.&3&600&300\_100\_200 \textcolor{red}{\textcjheb{rq+s}} SQR $|$der L"uge\\
\end{tabular}\medskip \\
Ende des Verses 12.19\\
Verse: 338, Buchstaben: 29, 547, 9611, Totalwerte: 3156, 38507, 696833\\
\\
Die Lippe der Wahrheit besteht ewiglich, aber nur einen Augenblick die Zunge der L"uge.\\
\newpage 
{\bf -- 12.20}\\
\medskip \\
\begin{tabular}{rrrrrrrrp{120mm}}
WV&WK&WB&ABK&ABB&ABV&AnzB&TW&Zahlencode \textcolor{red}{$\boldsymbol{Grundtext}$} Umschrift $|$"Ubersetzung(en)\\
1.&139.&2443.&548.&9612.&1.&4&285&40\_200\_40\_5 \textcolor{red}{\textcjheb{hmrm}} MRMH $|$Trug\\
2.&140.&2444.&552.&9616.&5.&3&34&2\_30\_2 \textcolor{red}{\textcjheb{blb}} BLB $|$(ist) im Herzen\\
3.&141.&2445.&555.&9619.&8.&4&518&8\_200\_300\_10 \textcolor{red}{\textcjheb{y+sr.h}} CRSJ $|$derer die schmieden/der Ersinnenden\\
4.&142.&2446.&559.&9623.&12.&2&270&200\_70 \textcolor{red}{\textcjheb{`r}} Ra $|$B"oses\\
5.&143.&2447.&561.&9625.&14.&6&216&6\_30\_10\_70\_90\_10 \textcolor{red}{\textcjheb{y.s`ylw}} WLJa"sJ $|$bei denen aber die planen/und bei Ratenden\\
6.&144.&2448.&567.&9631.&20.&4&376&300\_30\_6\_40 \textcolor{red}{\textcjheb{mwl+s}} SLWM $|$(zum) Frieden\\
7.&145.&2449.&571.&9635.&24.&4&353&300\_40\_8\_5 \textcolor{red}{\textcjheb{h.hm+s}} SMCH $|$(ist) Freude\\
\end{tabular}\medskip \\
Ende des Verses 12.20\\
Verse: 339, Buchstaben: 27, 574, 9638, Totalwerte: 2052, 40559, 698885\\
\\
Trug ist im Herzen derer, die B"oses schmieden; bei denen aber, die Frieden planen, ist Freude.\\
\newpage 
{\bf -- 12.21}\\
\medskip \\
\begin{tabular}{rrrrrrrrp{120mm}}
WV&WK&WB&ABK&ABB&ABV&AnzB&TW&Zahlencode \textcolor{red}{$\boldsymbol{Grundtext}$} Umschrift $|$"Ubersetzung(en)\\
1.&146.&2450.&575.&9639.&1.&2&31&30\_1 \textcolor{red}{\textcjheb{'l}} LA $|$nicht\\
2.&147.&2451.&577.&9641.&3.&4&66&10\_1\_50\_5 \textcolor{red}{\textcjheb{hn'y}} JANH $|$wird widerfahren/er (=es) begegnet\\
3.&148.&2452.&581.&9645.&7.&5&234&30\_90\_4\_10\_100 \textcolor{red}{\textcjheb{qyd.sl}} L"sDJQ $|$dem Gerechten/dem Rechtschaffenen\\
4.&149.&2453.&586.&9650.&12.&2&50&20\_30 \textcolor{red}{\textcjheb{lk}} KL $|$irgendein\\
5.&150.&2454.&588.&9652.&14.&3&57&1\_6\_50 \textcolor{red}{\textcjheb{nw'}} AWN $|$Unheil\\
6.&151.&2455.&591.&9655.&17.&6&626&6\_200\_300\_70\_10\_40 \textcolor{red}{\textcjheb{my`+srw}} WRSaJM $|$aber die Gesetzlosen/und Frevler\\
7.&152.&2456.&597.&9661.&23.&4&77&40\_30\_1\_6 \textcolor{red}{\textcjheb{w'lm}} MLAW $|$haben die F"ulle/sie sind voll\\
8.&153.&2457.&601.&9665.&27.&2&270&200\_70 \textcolor{red}{\textcjheb{`r}} Ra $|$"Ubel/(des) B"osen\\
\end{tabular}\medskip \\
Ende des Verses 12.21\\
Verse: 340, Buchstaben: 28, 602, 9666, Totalwerte: 1411, 41970, 700296\\
\\
Dem Gerechten wird keinerlei Unheil widerfahren, aber die Gesetzlosen haben "Ubel die F"ulle.\\
\newpage 
{\bf -- 12.22}\\
\medskip \\
\begin{tabular}{rrrrrrrrp{120mm}}
WV&WK&WB&ABK&ABB&ABV&AnzB&TW&Zahlencode \textcolor{red}{$\boldsymbol{Grundtext}$} Umschrift $|$"Ubersetzung(en)\\
1.&154.&2458.&603.&9667.&1.&5&878&400\_6\_70\_2\_400 \textcolor{red}{\textcjheb{tb`wt}} TWaBT $|$ein Gr"auel/Abscheu\\
2.&155.&2459.&608.&9672.&6.&4&26&10\_5\_6\_5 \textcolor{red}{\textcjheb{hwhy}} JHWH $|$(f"ur) Jahwe\\
3.&156.&2460.&612.&9676.&10.&4&790&300\_80\_400\_10 \textcolor{red}{\textcjheb{ytp+s}} SPTJ $|$(sind) die Lippen\\
4.&157.&2461.&616.&9680.&14.&3&600&300\_100\_200 \textcolor{red}{\textcjheb{rq+s}} SQR $|$der L"uge\\
5.&158.&2462.&619.&9683.&17.&4&386&6\_70\_300\_10 \textcolor{red}{\textcjheb{y+s`w}} WaSJ $|$die aber welche "uben/und Tuende\\
6.&159.&2463.&623.&9687.&21.&5&102&1\_40\_6\_50\_5 \textcolor{red}{\textcjheb{hnwm'}} AMWNH $|$Wahrheit/Wahrhaftigkeit\\
7.&160.&2464.&628.&9692.&26.&5&352&200\_90\_6\_50\_6 \textcolor{red}{\textcjheb{wnw.sr}} R"sWNW $|$(haben) sein Wohlgefallen\\
\end{tabular}\medskip \\
Ende des Verses 12.22\\
Verse: 341, Buchstaben: 30, 632, 9696, Totalwerte: 3134, 45104, 703430\\
\\
Die Lippen der L"uge sind Jahwe ein Greuel, die aber, welche Wahrheit "uben, sein Wohlgefallen.\\
\newpage 
{\bf -- 12.23}\\
\medskip \\
\begin{tabular}{rrrrrrrrp{120mm}}
WV&WK&WB&ABK&ABB&ABV&AnzB&TW&Zahlencode \textcolor{red}{$\boldsymbol{Grundtext}$} Umschrift $|$"Ubersetzung(en)\\
1.&161.&2465.&633.&9697.&1.&3&45&1\_4\_40 \textcolor{red}{\textcjheb{md'}} ADM $|$(ein) Mensch\\
2.&162.&2466.&636.&9700.&4.&4&316&70\_200\_6\_40 \textcolor{red}{\textcjheb{mwr`}} aRWM $|$kluger\\
3.&163.&2467.&640.&9704.&8.&3&85&20\_60\_5 \textcolor{red}{\textcjheb{hsk}} KsH $|$h"alt verborgen/ist bedeckend\\
4.&164.&2468.&643.&9707.&11.&3&474&4\_70\_400 \textcolor{red}{\textcjheb{t`d}} DaT $|$Erkenntnis/Wissen\\
5.&165.&2469.&646.&9710.&14.&3&38&6\_30\_2 \textcolor{red}{\textcjheb{blw}} WLB $|$aber das Herz/und das Herz\\
6.&166.&2470.&649.&9713.&17.&6&170&20\_60\_10\_30\_10\_40 \textcolor{red}{\textcjheb{mylysk}} KsJLJM $|$der Toren\\
7.&167.&2471.&655.&9719.&23.&4&311&10\_100\_200\_1 \textcolor{red}{\textcjheb{'rqy}} JQRA $|$(er (=es)) ruft aus\\
8.&168.&2472.&659.&9723.&27.&4&437&1\_6\_30\_400 \textcolor{red}{\textcjheb{tlw'}} AWLT $|$Narrheit\\
\end{tabular}\medskip \\
Ende des Verses 12.23\\
Verse: 342, Buchstaben: 30, 662, 9726, Totalwerte: 1876, 46980, 705306\\
\\
Ein kluger Mensch h"alt die Erkenntnis verborgen, aber das Herz der Toren ruft Narrheit aus.\\
\newpage 
{\bf -- 12.24}\\
\medskip \\
\begin{tabular}{rrrrrrrrp{120mm}}
WV&WK&WB&ABK&ABB&ABV&AnzB&TW&Zahlencode \textcolor{red}{$\boldsymbol{Grundtext}$} Umschrift $|$"Ubersetzung(en)\\
1.&169.&2473.&663.&9727.&1.&2&14&10\_4 \textcolor{red}{\textcjheb{dy}} JD $|$die Hand\\
2.&170.&2474.&665.&9729.&3.&6&354&8\_200\_6\_90\_10\_40 \textcolor{red}{\textcjheb{my.swr.h}} CRW"sJM $|$der Flei"sigen\\
3.&171.&2475.&671.&9735.&9.&5&776&400\_40\_300\_6\_30 \textcolor{red}{\textcjheb{lw+smt}} TMSWL $|$(sie) wird herrschen\\
4.&172.&2476.&676.&9740.&14.&5&261&6\_200\_40\_10\_5 \textcolor{red}{\textcjheb{hymrw}} WRMJH $|$aber die l"assige/und L"assigkeit\\
5.&173.&2477.&681.&9745.&19.&4&420&400\_5\_10\_5 \textcolor{red}{\textcjheb{hyht}} THJH $|$(sie) wird sein\\
6.&174.&2478.&685.&9749.&23.&3&130&30\_40\_60 \textcolor{red}{\textcjheb{sml}} LMs $|$fronpflichtig/(zu) Fronarbeit\\
\end{tabular}\medskip \\
Ende des Verses 12.24\\
Verse: 343, Buchstaben: 25, 687, 9751, Totalwerte: 1955, 48935, 707261\\
\\
Die Hand der Flei"sigen wird herrschen, aber die l"assige wird fronpflichtig sein.\\
\newpage 
{\bf -- 12.25}\\
\medskip \\
\begin{tabular}{rrrrrrrrp{120mm}}
WV&WK&WB&ABK&ABB&ABV&AnzB&TW&Zahlencode \textcolor{red}{$\boldsymbol{Grundtext}$} Umschrift $|$"Ubersetzung(en)\\
1.&175.&2479.&688.&9752.&1.&4&13&4\_1\_3\_5 \textcolor{red}{\textcjheb{hg'd}} DAGH $|$Kummer\\
2.&176.&2480.&692.&9756.&5.&3&34&2\_30\_2 \textcolor{red}{\textcjheb{blb}} BLB $|$im Herzen\\
3.&177.&2481.&695.&9759.&8.&3&311&1\_10\_300 \textcolor{red}{\textcjheb{+sy'}} AJS $|$des Mannes/eines Mannes\\
4.&178.&2482.&698.&9762.&11.&5&373&10\_300\_8\_50\_5 \textcolor{red}{\textcjheb{hn.h+sy}} JSCNH $|$beugt es nieder/(er) macht beugen sie (=es)\\
5.&179.&2483.&703.&9767.&16.&4&212&6\_4\_2\_200 \textcolor{red}{\textcjheb{rbdw}} WDBR $|$aber ein Wort/und ein Wort\\
6.&180.&2484.&707.&9771.&20.&3&17&9\_6\_2 \textcolor{red}{\textcjheb{bw.t}} tWB $|$gutes\\
7.&181.&2485.&710.&9774.&23.&6&413&10\_300\_40\_8\_50\_5 \textcolor{red}{\textcjheb{hn.hm+sy}} JSMCNH $|$(er (=es)) erfreut sie (=es)\\
\end{tabular}\medskip \\
Ende des Verses 12.25\\
Verse: 344, Buchstaben: 28, 715, 9779, Totalwerte: 1373, 50308, 708634\\
\\
Kummer im Herzen des Mannes beugt es nieder, aber ein gutes Wort erfreut es.\\
\newpage 
{\bf -- 12.26}\\
\medskip \\
\begin{tabular}{rrrrrrrrp{120mm}}
WV&WK&WB&ABK&ABB&ABV&AnzB&TW&Zahlencode \textcolor{red}{$\boldsymbol{Grundtext}$} Umschrift $|$"Ubersetzung(en)\\
1.&182.&2486.&716.&9780.&1.&3&610&10\_400\_200 \textcolor{red}{\textcjheb{rty}} JTR $|$(es) weist den Weg/er (=es) m"oge erkunden\\
2.&183.&2487.&719.&9783.&4.&5&321&40\_200\_70\_5\_6 \textcolor{red}{\textcjheb{wh`rm}} MRaHW $|$seinem N"achsten/von seinem Gef"ahrten\\
3.&184.&2488.&724.&9788.&9.&4&204&90\_4\_10\_100 \textcolor{red}{\textcjheb{qyd.s}} "sDJQ $|$(der) Gerechte\\
4.&185.&2489.&728.&9792.&13.&4&230&6\_4\_200\_20 \textcolor{red}{\textcjheb{krdw}} WDRK $|$aber der Weg/und der Weg\\
5.&186.&2490.&732.&9796.&17.&5&620&200\_300\_70\_10\_40 \textcolor{red}{\textcjheb{my`+sr}} RSaJM $|$der Gesetzlosen/(der) Frevler\\
6.&187.&2491.&737.&9801.&22.&4&910&400\_400\_70\_40 \textcolor{red}{\textcjheb{m`tt}} TTaM $|$(sie (=er)) f"uhrt sie irre\\
\end{tabular}\medskip \\
Ende des Verses 12.26\\
Verse: 345, Buchstaben: 25, 740, 9804, Totalwerte: 2895, 53203, 711529\\
\\
Der Gerechte weist seinem N"achsten den Weg, aber der Weg der Gesetzlosen f"uhrt sie irre.\\
\newpage 
{\bf -- 12.27}\\
\medskip \\
\begin{tabular}{rrrrrrrrp{120mm}}
WV&WK&WB&ABK&ABB&ABV&AnzB&TW&Zahlencode \textcolor{red}{$\boldsymbol{Grundtext}$} Umschrift $|$"Ubersetzung(en)\\
1.&188.&2492.&741.&9805.&1.&2&31&30\_1 \textcolor{red}{\textcjheb{'l}} LA $|$nicht\\
2.&189.&2493.&743.&9807.&3.&4&238&10\_8\_200\_20 \textcolor{red}{\textcjheb{kr.hy}} JCRK $|$erjagt/er (=es) wird erlangen\\
3.&190.&2494.&747.&9811.&7.&4&255&200\_40\_10\_5 \textcolor{red}{\textcjheb{hymr}} RMJH $|$der L"assige\\
4.&191.&2495.&751.&9815.&11.&4&110&90\_10\_4\_6 \textcolor{red}{\textcjheb{wdy.s}} "sJDW $|$sein Wild\\
5.&192.&2496.&755.&9819.&15.&4&67&6\_5\_6\_50 \textcolor{red}{\textcjheb{nwhw}} WHWN $|$aber Gut/und Gut\\
6.&193.&2497.&759.&9823.&19.&3&45&1\_4\_40 \textcolor{red}{\textcjheb{md'}} ADM $|$(ein(es)) Mensch (ist es)\\
7.&194.&2498.&762.&9826.&22.&3&310&10\_100\_200 \textcolor{red}{\textcjheb{rqy}} JQR $|$kostbares/wertvoller\\
8.&195.&2499.&765.&9829.&25.&4&304&8\_200\_6\_90 \textcolor{red}{\textcjheb{.swr.h}} CRW"s $|$wenn er flei"sig ist/(und) flei"siger\\
\end{tabular}\medskip \\
Ende des Verses 12.27\\
Verse: 346, Buchstaben: 28, 768, 9832, Totalwerte: 1360, 54563, 712889\\
\\
Nicht erjagt der L"assige sein Wild; aber kostbares Gut eines Menschen ist es, wenn er flei"sig ist.\\
\newpage 
{\bf -- 12.28}\\
\medskip \\
\begin{tabular}{rrrrrrrrp{120mm}}
WV&WK&WB&ABK&ABB&ABV&AnzB&TW&Zahlencode \textcolor{red}{$\boldsymbol{Grundtext}$} Umschrift $|$"Ubersetzung(en)\\
1.&196.&2500.&769.&9833.&1.&4&211&2\_1\_200\_8 \textcolor{red}{\textcjheb{.hr'b}} BARC $|$auf dem Pfad\\
2.&197.&2501.&773.&9837.&5.&4&199&90\_4\_100\_5 \textcolor{red}{\textcjheb{hqd.s}} "sDQH $|$der Gerechtigkeit\\
3.&198.&2502.&777.&9841.&9.&4&68&8\_10\_10\_40 \textcolor{red}{\textcjheb{myy.h}} CJJM $|$(ist) Leben\\
4.&199.&2503.&781.&9845.&13.&4&230&6\_4\_200\_20 \textcolor{red}{\textcjheb{krdw}} WDRK $|$und auf dem Weg/und der Weg\\
5.&200.&2504.&785.&9849.&17.&5&467&50\_400\_10\_2\_5 \textcolor{red}{\textcjheb{hbytn}} NTJBH $|$ihres Steiges/ist Pfad\\
6.&201.&2505.&790.&9854.&22.&2&31&1\_30 \textcolor{red}{\textcjheb{l'}} AL $|$nicht\\
7.&202.&2506.&792.&9856.&24.&3&446&40\_6\_400 \textcolor{red}{\textcjheb{twm}} MWT $|$(zum) Tod\\
\end{tabular}\medskip \\
Ende des Verses 12.28\\
Verse: 347, Buchstaben: 26, 794, 9858, Totalwerte: 1652, 56215, 714541\\
\\
Auf dem Pfade der Gerechtigkeit ist Leben, und kein Tod auf dem Wege ihres Steiges.\\
\\
{\bf Ende des Kapitels 12}\\
\newpage 
{\bf -- 13.1}\\
\medskip \\
\begin{tabular}{rrrrrrrrp{120mm}}
WV&WK&WB&ABK&ABB&ABV&AnzB&TW&Zahlencode \textcolor{red}{$\boldsymbol{Grundtext}$} Umschrift $|$"Ubersetzung(en)\\
1.&1.&2507.&1.&9859.&1.&2&52&2\_50 \textcolor{red}{\textcjheb{nb}} BN $|$(ein) Sohn\\
2.&2.&2508.&3.&9861.&3.&3&68&8\_20\_40 \textcolor{red}{\textcjheb{mk.h}} CKM $|$weiser\\
3.&3.&2509.&6.&9864.&6.&4&306&40\_6\_60\_200 \textcolor{red}{\textcjheb{rswm}} MWsR $|$h"ort auf die Unterweisung/(h"ort auf) die Mahnung\\
4.&4.&2510.&10.&9868.&10.&2&3&1\_2 \textcolor{red}{\textcjheb{b'}} AB $|$des Vaters\\
5.&5.&2511.&12.&9870.&12.&3&126&6\_30\_90 \textcolor{red}{\textcjheb{.slw}} WL"s $|$aber ein Sp"otter/und ein Sp"otter\\
6.&6.&2512.&15.&9873.&15.&2&31&30\_1 \textcolor{red}{\textcjheb{'l}} LA $|$nicht\\
7.&7.&2513.&17.&9875.&17.&3&410&300\_40\_70 \textcolor{red}{\textcjheb{`m+s}} SMa $|$(er) h"ort\\
8.&8.&2514.&20.&9878.&20.&4&278&3\_70\_200\_5 \textcolor{red}{\textcjheb{hr`g}} GaRH $|$auf Schelten/(auf einen) Tadel\\
\end{tabular}\medskip \\
Ende des Verses 13.1\\
Verse: 348, Buchstaben: 23, 23, 9881, Totalwerte: 1274, 1274, 715815\\
\\
Ein weiser Sohn h"ort auf die Unterweisung des Vaters, aber ein Sp"otter h"ort nicht auf Schelten.\\
\newpage 
{\bf -- 13.2}\\
\medskip \\
\begin{tabular}{rrrrrrrrp{120mm}}
WV&WK&WB&ABK&ABB&ABV&AnzB&TW&Zahlencode \textcolor{red}{$\boldsymbol{Grundtext}$} Umschrift $|$"Ubersetzung(en)\\
1.&9.&2515.&24.&9882.&1.&4&330&40\_80\_200\_10 \textcolor{red}{\textcjheb{yrpm}} MPRJ $|$von der Frucht\\
2.&10.&2516.&28.&9886.&5.&2&90&80\_10 \textcolor{red}{\textcjheb{yp}} PJ $|$seines Mundes/des Mundes\\
3.&11.&2517.&30.&9888.&7.&3&311&1\_10\_300 \textcolor{red}{\textcjheb{+sy'}} AJS $|$ein(es) Mannes\\
4.&12.&2518.&33.&9891.&10.&4&61&10\_1\_20\_30 \textcolor{red}{\textcjheb{lk'y}} JAKL $|$(er (=es)) isst\\
5.&13.&2519.&37.&9895.&14.&3&17&9\_6\_2 \textcolor{red}{\textcjheb{bw.t}} tWB $|$Gutes/der Gute\\
6.&14.&2520.&40.&9898.&17.&4&436&6\_50\_80\_300 \textcolor{red}{\textcjheb{+spnw}} WNPS $|$aber die Seele/und die Seele\\
7.&15.&2521.&44.&9902.&21.&5&59&2\_3\_4\_10\_40 \textcolor{red}{\textcjheb{mydgb}} BGDJM $|$(der) Treulosen\\
8.&16.&2522.&49.&9907.&26.&3&108&8\_40\_60 \textcolor{red}{\textcjheb{sm.h}} CMs $|$(is(s)t) Gewalttat\\
\end{tabular}\medskip \\
Ende des Verses 13.2\\
Verse: 349, Buchstaben: 28, 51, 9909, Totalwerte: 1412, 2686, 717227\\
\\
Von der Frucht seines Mundes i"st ein Mann Gutes, aber die Seele der Treulosen i"st Gewalttat.\\
\newpage 
{\bf -- 13.3}\\
\medskip \\
\begin{tabular}{rrrrrrrrp{120mm}}
WV&WK&WB&ABK&ABB&ABV&AnzB&TW&Zahlencode \textcolor{red}{$\boldsymbol{Grundtext}$} Umschrift $|$"Ubersetzung(en)\\
1.&17.&2523.&52.&9910.&1.&3&340&50\_90\_200 \textcolor{red}{\textcjheb{r.sn}} N"sR $|$wer bewahrt/(ein) Bewachender\\
2.&18.&2524.&55.&9913.&4.&3&96&80\_10\_6 \textcolor{red}{\textcjheb{wyp}} PJW $|$seinen Mund\\
3.&19.&2525.&58.&9916.&7.&3&540&300\_40\_200 \textcolor{red}{\textcjheb{rm+s}} SMR $|$beh"utet/(ist) bewahrend\\
4.&20.&2526.&61.&9919.&10.&4&436&50\_80\_300\_6 \textcolor{red}{\textcjheb{w+spn}} NPSW $|$seine Seele\\
5.&21.&2527.&65.&9923.&14.&3&480&80\_300\_100 \textcolor{red}{\textcjheb{q+sp}} PSQ $|$wer aufrei"st/ein Aufrei"sender\\
6.&22.&2528.&68.&9926.&17.&5&796&300\_80\_400\_10\_6 \textcolor{red}{\textcjheb{wytp+s}} SPTJW $|$seine Lippen\\
7.&23.&2529.&73.&9931.&22.&4&453&40\_8\_400\_5 \textcolor{red}{\textcjheb{ht.hm}} MCTH $|$zum Untergang wird es/Verderben\\
8.&24.&2530.&77.&9935.&26.&2&36&30\_6 \textcolor{red}{\textcjheb{wl}} LW $|$dem/zu ihn\\
\end{tabular}\medskip \\
Ende des Verses 13.3\\
Verse: 350, Buchstaben: 27, 78, 9936, Totalwerte: 3177, 5863, 720404\\
\\
Wer seinen Mund bewahrt, beh"utet seine Seele; wer seine Lippen aufrei"st, dem wird's zum Untergang.\\
\newpage 
{\bf -- 13.4}\\
\medskip \\
\begin{tabular}{rrrrrrrrp{120mm}}
WV&WK&WB&ABK&ABB&ABV&AnzB&TW&Zahlencode \textcolor{red}{$\boldsymbol{Grundtext}$} Umschrift $|$"Ubersetzung(en)\\
1.&25.&2531.&79.&9937.&1.&5&452&40\_400\_1\_6\_5 \textcolor{red}{\textcjheb{hw'tm}} MTAWH $|$(es) begehrt/sich begierig zeigend\\
2.&26.&2532.&84.&9942.&6.&4&67&6\_1\_10\_50 \textcolor{red}{\textcjheb{ny'w}} WAJN $|$und nicht(s) ist (da)\\
3.&27.&2533.&88.&9946.&10.&4&436&50\_80\_300\_6 \textcolor{red}{\textcjheb{w+spn}} NPSW $|$/sein Begehren\\
4.&28.&2534.&92.&9950.&14.&3&190&70\_90\_30 \textcolor{red}{\textcjheb{l.s`}} a"sL $|$(des) Faulen\\
5.&29.&2535.&95.&9953.&17.&4&436&6\_50\_80\_300 \textcolor{red}{\textcjheb{+spnw}} WNPS $|$aber die Seele/und die Seele\\
6.&30.&2536.&99.&9957.&21.&5&348&8\_200\_90\_10\_40 \textcolor{red}{\textcjheb{my.sr.h}} CR"sJM $|$der Flei"sigen\\
7.&31.&2537.&104.&9962.&26.&4&754&400\_4\_300\_50 \textcolor{red}{\textcjheb{n+sdt}} TDSN $|$(sie (=es)) wird (reichlich) ges"attigt\\
\end{tabular}\medskip \\
Ende des Verses 13.4\\
Verse: 351, Buchstaben: 29, 107, 9965, Totalwerte: 2683, 8546, 723087\\
\\
Die Seele des Faulen begehrt, und nichts ist da; aber die Seele der Flei"sigen wird reichlich ges"attigt.\\
\newpage 
{\bf -- 13.5}\\
\medskip \\
\begin{tabular}{rrrrrrrrp{120mm}}
WV&WK&WB&ABK&ABB&ABV&AnzB&TW&Zahlencode \textcolor{red}{$\boldsymbol{Grundtext}$} Umschrift $|$"Ubersetzung(en)\\
1.&32.&2538.&108.&9966.&1.&3&206&4\_2\_200 \textcolor{red}{\textcjheb{rbd}} DBR $|$Rede/(ein) Wort\\
2.&33.&2539.&111.&9969.&4.&3&600&300\_100\_200 \textcolor{red}{\textcjheb{rq+s}} SQR $|$(der) L"uge(n)\\
3.&34.&2540.&114.&9972.&7.&4&361&10\_300\_50\_1 \textcolor{red}{\textcjheb{'n+sy}} JSNA $|$(er (=es)) hasst\\
4.&35.&2541.&118.&9976.&11.&4&204&90\_4\_10\_100 \textcolor{red}{\textcjheb{qyd.s}} "sDJQ $|$der Gerechte/(ein) Rechtschaffener\\
5.&36.&2542.&122.&9980.&15.&4&576&6\_200\_300\_70 \textcolor{red}{\textcjheb{`+srw}} WRSa $|$aber der Gesetzlose/und (ein) Frevler\\
6.&37.&2543.&126.&9984.&19.&5&323&10\_2\_1\_10\_300 \textcolor{red}{\textcjheb{+sy'by}} JBAJS $|$handelt sch"andlich/(er) macht stinkend\\
7.&38.&2544.&131.&9989.&24.&6&314&6\_10\_8\_80\_10\_200 \textcolor{red}{\textcjheb{ryp.hyw}} WJCPJR $|$und schm"ahlich/und (er) handelt sch"andlich\\
\end{tabular}\medskip \\
Ende des Verses 13.5\\
Verse: 352, Buchstaben: 29, 136, 9994, Totalwerte: 2584, 11130, 725671\\
\\
Der Gerechte ha"st L"ugenrede, aber der Gesetzlose handelt sch"andlich und schm"ahlich.\\
\newpage 
{\bf -- 13.6}\\
\medskip \\
\begin{tabular}{rrrrrrrrp{120mm}}
WV&WK&WB&ABK&ABB&ABV&AnzB&TW&Zahlencode \textcolor{red}{$\boldsymbol{Grundtext}$} Umschrift $|$"Ubersetzung(en)\\
1.&39.&2545.&137.&9995.&1.&4&199&90\_4\_100\_5 \textcolor{red}{\textcjheb{hqd.s}} "sDQH $|$(die) Gerechtigkeit\\
2.&40.&2546.&141.&9999.&5.&3&690&400\_90\_200 \textcolor{red}{\textcjheb{r.st}} T"sR $|$beh"utet/(sie) bewahrt\\
3.&41.&2547.&144.&10002.&8.&2&440&400\_40 \textcolor{red}{\textcjheb{mt}} TM $|$den Vollkommenen/(den) Vollendeten\\
4.&42.&2548.&146.&10004.&10.&3&224&4\_200\_20 \textcolor{red}{\textcjheb{krd}} DRK $|$im Weg/(des) Wegs\\
5.&43.&2549.&149.&10007.&13.&5&581&6\_200\_300\_70\_5 \textcolor{red}{\textcjheb{h`+srw}} WRSaH $|$und die Gesetzlosigkeit/und Ruchlosigkeit\\
6.&44.&2550.&154.&10012.&18.&4&570&400\_60\_30\_80 \textcolor{red}{\textcjheb{plst}} TsLP $|$kehrt um/(sie) verkehrt\\
7.&45.&2551.&158.&10016.&22.&4&418&8\_9\_1\_400 \textcolor{red}{\textcjheb{t'.t.h}} CtAT $|$den S"under/(in) S"unde\\
\end{tabular}\medskip \\
Ende des Verses 13.6\\
Verse: 353, Buchstaben: 25, 161, 10019, Totalwerte: 3122, 14252, 728793\\
\\
Die Gerechtigkeit beh"utet den im Wege Vollkommenen, und die Gesetzlosigkeit kehrt den S"under um.\\
\newpage 
{\bf -- 13.7}\\
\medskip \\
\begin{tabular}{rrrrrrrrp{120mm}}
WV&WK&WB&ABK&ABB&ABV&AnzB&TW&Zahlencode \textcolor{red}{$\boldsymbol{Grundtext}$} Umschrift $|$"Ubersetzung(en)\\
1.&46.&2552.&162.&10020.&1.&2&310&10\_300 \textcolor{red}{\textcjheb{+sy}} JS $|$da gibt es einen/es gibt\\
2.&47.&2553.&164.&10022.&3.&5&1010&40\_400\_70\_300\_200 \textcolor{red}{\textcjheb{r+s`tm}} MTaSR $|$der sich reich stellt/einen sich reich Stellenden\\
3.&48.&2554.&169.&10027.&8.&4&67&6\_1\_10\_50 \textcolor{red}{\textcjheb{ny'w}} WAJN $|$und nichts hat/und nicht gibt es\\
4.&49.&2555.&173.&10031.&12.&2&50&20\_30 \textcolor{red}{\textcjheb{lk}} KL $|$gar/alles\\
5.&50.&2556.&175.&10033.&14.&6&1246&40\_400\_200\_6\_300\_300 \textcolor{red}{\textcjheb{+s+swrtm}} MTRWSS $|$und einer der sich arm stellt/einem sich arm Stellenden\\
6.&51.&2557.&181.&10039.&20.&4&67&6\_5\_6\_50 \textcolor{red}{\textcjheb{nwhw}} WHWN $|$und (er) (hat) Verm"ogen\\
7.&52.&2558.&185.&10043.&24.&2&202&200\_2 \textcolor{red}{\textcjheb{br}} RB $|$viel\\
\end{tabular}\medskip \\
Ende des Verses 13.7\\
Verse: 354, Buchstaben: 25, 186, 10044, Totalwerte: 2952, 17204, 731745\\
\\
Da ist einer, der sich reich stellt und hat gar nichts, und einer, der sich arm stellt und hat viel Verm"ogen.\\
\newpage 
{\bf -- 13.8}\\
\medskip \\
\begin{tabular}{rrrrrrrrp{120mm}}
WV&WK&WB&ABK&ABB&ABV&AnzB&TW&Zahlencode \textcolor{red}{$\boldsymbol{Grundtext}$} Umschrift $|$"Ubersetzung(en)\\
1.&53.&2559.&187.&10045.&1.&3&300&20\_80\_200 \textcolor{red}{\textcjheb{rpk}} KPR $|$L"osegeld\\
2.&54.&2560.&190.&10048.&4.&3&430&50\_80\_300 \textcolor{red}{\textcjheb{+spn}} NPS $|$(f"ur) das Leben\\
3.&55.&2561.&193.&10051.&7.&3&311&1\_10\_300 \textcolor{red}{\textcjheb{+sy'}} AJS $|$eines Mannes\\
4.&56.&2562.&196.&10054.&10.&4&576&70\_300\_200\_6 \textcolor{red}{\textcjheb{wr+s`}} aSRW $|$(ist) sein Reichtum\\
5.&57.&2563.&200.&10058.&14.&3&506&6\_200\_300 \textcolor{red}{\textcjheb{+srw}} WRS $|$aber der Arme/und ein Armer\\
6.&58.&2564.&203.&10061.&17.&2&31&30\_1 \textcolor{red}{\textcjheb{'l}} LA $|$nicht\\
7.&59.&2565.&205.&10063.&19.&3&410&300\_40\_70 \textcolor{red}{\textcjheb{`m+s}} SMa $|$(er) h"ort(e)\\
8.&60.&2566.&208.&10066.&22.&4&278&3\_70\_200\_5 \textcolor{red}{\textcjheb{hr`g}} GaRH $|$(eine) Drohung\\
\end{tabular}\medskip \\
Ende des Verses 13.8\\
Verse: 355, Buchstaben: 25, 211, 10069, Totalwerte: 2842, 20046, 734587\\
\\
L"osegeld f"ur das Leben eines Mannes ist sein Reichtum, aber der Arme h"ort keine Drohung.\\
\newpage 
{\bf -- 13.9}\\
\medskip \\
\begin{tabular}{rrrrrrrrp{120mm}}
WV&WK&WB&ABK&ABB&ABV&AnzB&TW&Zahlencode \textcolor{red}{$\boldsymbol{Grundtext}$} Umschrift $|$"Ubersetzung(en)\\
1.&61.&2567.&212.&10070.&1.&3&207&1\_6\_200 \textcolor{red}{\textcjheb{rw'}} AWR $|$(das) Licht\\
2.&62.&2568.&215.&10073.&4.&6&254&90\_4\_10\_100\_10\_40 \textcolor{red}{\textcjheb{myqyd.s}} "sDJQJM $|$der Gerechten\\
3.&63.&2569.&221.&10079.&10.&4&358&10\_300\_40\_8 \textcolor{red}{\textcjheb{.hm+sy}} JSMC $|$brennt fr"ohlich/er (=es) gedeiht\\
4.&64.&2570.&225.&10083.&14.&3&256&6\_50\_200 \textcolor{red}{\textcjheb{rnw}} WNR $|$aber die Leuchte/und die Leuchte\\
5.&65.&2571.&228.&10086.&17.&5&620&200\_300\_70\_10\_40 \textcolor{red}{\textcjheb{my`+sr}} RSaJM $|$der Gesetzlosen/(der) Frevler\\
6.&66.&2572.&233.&10091.&22.&4&104&10\_4\_70\_20 \textcolor{red}{\textcjheb{k`dy}} JDaK $|$(er (=sie)) erlischt\\
\end{tabular}\medskip \\
Ende des Verses 13.9\\
Verse: 356, Buchstaben: 25, 236, 10094, Totalwerte: 1799, 21845, 736386\\
\\
Das Licht der Gerechten brennt fr"ohlich, aber die Leuchte der Gesetzlosen erlischt.\\
\newpage 
{\bf -- 13.10}\\
\medskip \\
\begin{tabular}{rrrrrrrrp{120mm}}
WV&WK&WB&ABK&ABB&ABV&AnzB&TW&Zahlencode \textcolor{red}{$\boldsymbol{Grundtext}$} Umschrift $|$"Ubersetzung(en)\\
1.&67.&2573.&237.&10095.&1.&2&300&200\_100 \textcolor{red}{\textcjheb{qr}} RQ $|$nur\\
2.&68.&2574.&239.&10097.&3.&5&69&2\_7\_4\_6\_50 \textcolor{red}{\textcjheb{nwdzb}} BZDWN $|$durch "Ubermut/durch Vermessenheit\\
3.&69.&2575.&244.&10102.&8.&3&460&10\_400\_50 \textcolor{red}{\textcjheb{nty}} JTN $|$er (=es) gibt\\
4.&70.&2576.&247.&10105.&11.&3&135&40\_90\_5 \textcolor{red}{\textcjheb{h.sm}} M"sH $|$Zank/Streit\\
5.&71.&2577.&250.&10108.&14.&3&407&6\_1\_400 \textcolor{red}{\textcjheb{t'w}} WAT $|$bei denen aber/und mit\\
6.&72.&2578.&253.&10111.&17.&6&266&50\_6\_70\_90\_10\_40 \textcolor{red}{\textcjheb{my.s`wn}} NWa"sJM $|$die sich raten lassen/(sich) Beratenden\\
7.&73.&2579.&259.&10117.&23.&4&73&8\_20\_40\_5 \textcolor{red}{\textcjheb{hmk.h}} CKMH $|$(ist) Weisheit\\
\end{tabular}\medskip \\
Ende des Verses 13.10\\
Verse: 357, Buchstaben: 26, 262, 10120, Totalwerte: 1710, 23555, 738096\\
\\
Durch "Ubermut gibt es nur Zank; bei denen aber, die sich raten lassen, Weisheit.\\
\newpage 
{\bf -- 13.11}\\
\medskip \\
\begin{tabular}{rrrrrrrrp{120mm}}
WV&WK&WB&ABK&ABB&ABV&AnzB&TW&Zahlencode \textcolor{red}{$\boldsymbol{Grundtext}$} Umschrift $|$"Ubersetzung(en)\\
1.&74.&2580.&263.&10121.&1.&3&61&5\_6\_50 \textcolor{red}{\textcjheb{nwh}} HWN $|$Verm"ogen\\
2.&75.&2581.&266.&10124.&4.&4&77&40\_5\_2\_30 \textcolor{red}{\textcjheb{lbhm}} MHBL $|$das auf nichtige Weise erworben ist/aus der Nichtigkeit\\
3.&76.&2582.&270.&10128.&8.&4&129&10\_40\_70\_9 \textcolor{red}{\textcjheb{.t`my}} JMat $|$vermindert sich/er (=es) wird wenig\\
4.&77.&2583.&274.&10132.&12.&4&198&6\_100\_2\_90 \textcolor{red}{\textcjheb{.sbqw}} WQB"s $|$wer aber ansammelt/und ein Ansammeln\\
5.&78.&2584.&278.&10136.&16.&2&100&70\_30 \textcolor{red}{\textcjheb{l`}} aL $|$/auf\\
6.&79.&2585.&280.&10138.&18.&2&14&10\_4 \textcolor{red}{\textcjheb{dy}} JD $|$allm"ahlich/die Hand\\
7.&80.&2586.&282.&10140.&20.&4&217&10\_200\_2\_5 \textcolor{red}{\textcjheb{hbry}} JRBH $|$vermehrt es/er (=es) wird vermehrt\\
\end{tabular}\medskip \\
Ende des Verses 13.11\\
Verse: 358, Buchstaben: 23, 285, 10143, Totalwerte: 796, 24351, 738892\\
\\
Verm"ogen, das auf nichtige Weise erworben ist, vermindert sich; wer aber allm"ahlich sammelt, vermehrt es.\\
\newpage 
{\bf -- 13.12}\\
\medskip \\
\begin{tabular}{rrrrrrrrp{120mm}}
WV&WK&WB&ABK&ABB&ABV&AnzB&TW&Zahlencode \textcolor{red}{$\boldsymbol{Grundtext}$} Umschrift $|$"Ubersetzung(en)\\
1.&81.&2587.&286.&10144.&1.&5&844&400\_6\_8\_30\_400 \textcolor{red}{\textcjheb{tl.hwt}} TWCLT $|$Harren/eine Erwartung\\
2.&82.&2588.&291.&10149.&6.&5&405&40\_40\_300\_20\_5 \textcolor{red}{\textcjheb{hk+smm}} MMSKH $|$lang hingezogenes/hingehaltene\\
3.&83.&2589.&296.&10154.&11.&4&83&40\_8\_30\_5 \textcolor{red}{\textcjheb{hl.hm}} MCLH $|$macht krank/(bedeutet) Krankheit\\
4.&84.&2590.&300.&10158.&15.&2&32&30\_2 \textcolor{red}{\textcjheb{bl}} LB $|$(f"ur) (das) Herz\\
5.&85.&2591.&302.&10160.&17.&3&166&6\_70\_90 \textcolor{red}{\textcjheb{.s`w}} Wa"s $|$aber ein Baum/und ein Baum\\
6.&86.&2592.&305.&10163.&20.&4&68&8\_10\_10\_40 \textcolor{red}{\textcjheb{myy.h}} CJJM $|$des Lebens\\
7.&87.&2593.&309.&10167.&24.&4&412&400\_1\_6\_5 \textcolor{red}{\textcjheb{hw't}} TAWH $|$ist ein Wunsch/(ist) (ein) Verlangen\\
8.&88.&2594.&313.&10171.&28.&3&8&2\_1\_5 \textcolor{red}{\textcjheb{h'b}} BAH $|$eingetroffener/erf"ulltes\\
\end{tabular}\medskip \\
Ende des Verses 13.12\\
Verse: 359, Buchstaben: 30, 315, 10173, Totalwerte: 2018, 26369, 740910\\
\\
Lang hingezogenes Harren macht das Herz krank, aber ein eingetroffener Wunsch ist ein Baum des Lebens.\\
\newpage 
{\bf -- 13.13}\\
\medskip \\
\begin{tabular}{rrrrrrrrp{120mm}}
WV&WK&WB&ABK&ABB&ABV&AnzB&TW&Zahlencode \textcolor{red}{$\boldsymbol{Grundtext}$} Umschrift $|$"Ubersetzung(en)\\
1.&89.&2595.&316.&10174.&1.&2&9&2\_7 \textcolor{red}{\textcjheb{zb}} BZ $|$wer verachtet/ein Verachtender\\
2.&90.&2596.&318.&10176.&3.&4&236&30\_4\_2\_200 \textcolor{red}{\textcjheb{rbdl}} LDBR $|$das Wort\\
3.&91.&2597.&322.&10180.&7.&4&50&10\_8\_2\_30 \textcolor{red}{\textcjheb{lb.hy}} JCBL $|$wird gepf"andet/er (=es) wird gepf"andet werden\\
4.&92.&2598.&326.&10184.&11.&2&36&30\_6 \textcolor{red}{\textcjheb{wl}} LW $|$von ihm/bei ihm\\
5.&93.&2599.&328.&10186.&13.&4&217&6\_10\_200\_1 \textcolor{red}{\textcjheb{'ryw}} WJRA $|$wer aber f"urchtet/und wer f"urchtet\\
6.&94.&2600.&332.&10190.&17.&4&141&40\_90\_6\_5 \textcolor{red}{\textcjheb{hw.sm}} M"sWH $|$(das) Gebot\\
7.&95.&2601.&336.&10194.&21.&3&12&5\_6\_1 \textcolor{red}{\textcjheb{'wh}} HWA $|$dem/(d)er\\
8.&96.&2602.&339.&10197.&24.&4&380&10\_300\_30\_40 \textcolor{red}{\textcjheb{ml+sy}} JSLM $|$wird vergolten werden/(er) wird belohnt\\
\end{tabular}\medskip \\
Ende des Verses 13.13\\
Verse: 360, Buchstaben: 27, 342, 10200, Totalwerte: 1081, 27450, 741991\\
\\
Wer das Wort verachtet, wird von ihm gepf"andet; wer aber das Gebot f"urchtet, dem wird vergolten werden.\\
\newpage 
{\bf -- 13.14}\\
\medskip \\
\begin{tabular}{rrrrrrrrp{120mm}}
WV&WK&WB&ABK&ABB&ABV&AnzB&TW&Zahlencode \textcolor{red}{$\boldsymbol{Grundtext}$} Umschrift $|$"Ubersetzung(en)\\
1.&97.&2603.&343.&10201.&1.&4&1006&400\_6\_200\_400 \textcolor{red}{\textcjheb{trwt}} TWRT $|$die Belehrung/die Weisung\\
2.&98.&2604.&347.&10205.&5.&3&68&8\_20\_40 \textcolor{red}{\textcjheb{mk.h}} CKM $|$(des) Weisen\\
3.&99.&2605.&350.&10208.&8.&4&346&40\_100\_6\_200 \textcolor{red}{\textcjheb{rwqm}} MQWR $|$ist ein Born/(ist) eine Quelle\\
4.&100.&2606.&354.&10212.&12.&4&68&8\_10\_10\_40 \textcolor{red}{\textcjheb{myy.h}} CJJM $|$des Lebens\\
5.&101.&2607.&358.&10216.&16.&4&296&30\_60\_6\_200 \textcolor{red}{\textcjheb{rwsl}} LsWR $|$um zu entgehen/zu weichen\\
6.&102.&2608.&362.&10220.&20.&5&490&40\_40\_100\_300\_10 \textcolor{red}{\textcjheb{y+sqmm}} MMQSJ $|$den Fallstricken/von den Schlingen\\
7.&103.&2609.&367.&10225.&25.&3&446&40\_6\_400 \textcolor{red}{\textcjheb{twm}} MWT $|$des Todes\\
\end{tabular}\medskip \\
Ende des Verses 13.14\\
Verse: 361, Buchstaben: 27, 369, 10227, Totalwerte: 2720, 30170, 744711\\
\\
Die Belehrung des Weisen ist ein Born des Lebens, um zu entgehen den Fallstricken des Todes.\\
\newpage 
{\bf -- 13.15}\\
\medskip \\
\begin{tabular}{rrrrrrrrp{120mm}}
WV&WK&WB&ABK&ABB&ABV&AnzB&TW&Zahlencode \textcolor{red}{$\boldsymbol{Grundtext}$} Umschrift $|$"Ubersetzung(en)\\
1.&104.&2610.&370.&10228.&1.&3&350&300\_20\_30 \textcolor{red}{\textcjheb{lk+s}} SKL $|$Einsicht\\
2.&105.&2611.&373.&10231.&4.&3&17&9\_6\_2 \textcolor{red}{\textcjheb{bw.t}} tWB $|$gute\\
3.&106.&2612.&376.&10234.&7.&3&460&10\_400\_50 \textcolor{red}{\textcjheb{nty}} JTN $|$verschafft/er (=sie) bringt ein\\
4.&107.&2613.&379.&10237.&10.&2&58&8\_50 \textcolor{red}{\textcjheb{n.h}} CN $|$Gunst\\
5.&108.&2614.&381.&10239.&12.&4&230&6\_4\_200\_20 \textcolor{red}{\textcjheb{krdw}} WDRK $|$aber der Weg/und der Weg\\
6.&109.&2615.&385.&10243.&16.&5&59&2\_3\_4\_10\_40 \textcolor{red}{\textcjheb{mydgb}} BGDJM $|$(der) Treulosen\\
7.&110.&2616.&390.&10248.&21.&4&461&1\_10\_400\_50 \textcolor{red}{\textcjheb{nty'}} AJTN $|$ist hart/(ist) fest\\
\end{tabular}\medskip \\
Ende des Verses 13.15\\
Verse: 362, Buchstaben: 24, 393, 10251, Totalwerte: 1635, 31805, 746346\\
\\
Gute Einsicht verschafft Gunst, aber der Treulosen Weg ist hart.\\
\newpage 
{\bf -- 13.16}\\
\medskip \\
\begin{tabular}{rrrrrrrrp{120mm}}
WV&WK&WB&ABK&ABB&ABV&AnzB&TW&Zahlencode \textcolor{red}{$\boldsymbol{Grundtext}$} Umschrift $|$"Ubersetzung(en)\\
1.&111.&2617.&394.&10252.&1.&2&50&20\_30 \textcolor{red}{\textcjheb{lk}} KL $|$jeder\\
2.&112.&2618.&396.&10254.&3.&4&316&70\_200\_6\_40 \textcolor{red}{\textcjheb{mwr`}} aRWM $|$Kluge\\
3.&113.&2619.&400.&10258.&7.&4&385&10\_70\_300\_5 \textcolor{red}{\textcjheb{h+s`y}} JaSH $|$(er) handelt\\
4.&114.&2620.&404.&10262.&11.&4&476&2\_4\_70\_400 \textcolor{red}{\textcjheb{t`db}} BDaT $|$mit Bedacht/nach Erkenntnis\\
5.&115.&2621.&408.&10266.&15.&5&126&6\_20\_60\_10\_30 \textcolor{red}{\textcjheb{lyskw}} WKsJL $|$aber ein Tor/und ein Tor\\
6.&116.&2622.&413.&10271.&20.&4&590&10\_80\_200\_300 \textcolor{red}{\textcjheb{+srpy}} JPRS $|$breitet aus/(er) verbreitet\\
7.&117.&2623.&417.&10275.&24.&4&437&1\_6\_30\_400 \textcolor{red}{\textcjheb{tlw'}} AWLT $|$Narrheit\\
\end{tabular}\medskip \\
Ende des Verses 13.16\\
Verse: 363, Buchstaben: 27, 420, 10278, Totalwerte: 2380, 34185, 748726\\
\\
Jeder Kluge handelt mit Bedacht; ein Tor aber breitet Narrheit aus.\\
\newpage 
{\bf -- 13.17}\\
\medskip \\
\begin{tabular}{rrrrrrrrp{120mm}}
WV&WK&WB&ABK&ABB&ABV&AnzB&TW&Zahlencode \textcolor{red}{$\boldsymbol{Grundtext}$} Umschrift $|$"Ubersetzung(en)\\
1.&118.&2624.&421.&10279.&1.&4&91&40\_30\_1\_20 \textcolor{red}{\textcjheb{k'lm}} MLAK $|$(ein) Bote\\
2.&119.&2625.&425.&10283.&5.&3&570&200\_300\_70 \textcolor{red}{\textcjheb{`+sr}} RSa $|$gottloser/ruchloser\\
3.&120.&2626.&428.&10286.&8.&3&120&10\_80\_30 \textcolor{red}{\textcjheb{lpy}} JPL $|$(er) f"allt\\
4.&121.&2627.&431.&10289.&11.&3&272&2\_200\_70 \textcolor{red}{\textcjheb{`rb}} BRa $|$in(s) Ungl"uck\\
5.&122.&2628.&434.&10292.&14.&4&306&6\_90\_10\_200 \textcolor{red}{\textcjheb{ry.sw}} W"sJR $|$aber ein Gesandter/und ein Gesandter\\
6.&123.&2629.&438.&10296.&18.&6&147&1\_40\_6\_50\_10\_40 \textcolor{red}{\textcjheb{mynwm'}} AMWNJM $|$treuer\\
7.&124.&2630.&444.&10302.&24.&4&321&40\_200\_80\_1 \textcolor{red}{\textcjheb{'prm}} MRPA $|$ist Gesundheit/(bringt) Heilung\\
\end{tabular}\medskip \\
Ende des Verses 13.17\\
Verse: 364, Buchstaben: 27, 447, 10305, Totalwerte: 1827, 36012, 750553\\
\\
Ein gottloser Bote f"allt in Ungl"uck, aber ein treuer Gesandter ist Gesundheit.\\
\newpage 
{\bf -- 13.18}\\
\medskip \\
\begin{tabular}{rrrrrrrrp{120mm}}
WV&WK&WB&ABK&ABB&ABV&AnzB&TW&Zahlencode \textcolor{red}{$\boldsymbol{Grundtext}$} Umschrift $|$"Ubersetzung(en)\\
1.&125.&2631.&448.&10306.&1.&3&510&200\_10\_300 \textcolor{red}{\textcjheb{+syr}} RJS $|$Armut\\
2.&126.&2632.&451.&10309.&4.&5&192&6\_100\_30\_6\_50 \textcolor{red}{\textcjheb{nwlqw}} WQLWN $|$und Schande\\
3.&127.&2633.&456.&10314.&9.&4&356&80\_6\_200\_70 \textcolor{red}{\textcjheb{`rwp}} PWRa $|$dem der verwirft/einem sich Entziehenden\\
4.&128.&2634.&460.&10318.&13.&4&306&40\_6\_60\_200 \textcolor{red}{\textcjheb{rswm}} MWsR $|$Unterweisung/(der) Zucht\\
5.&129.&2635.&464.&10322.&17.&5&552&6\_300\_6\_40\_200 \textcolor{red}{\textcjheb{rmw+sw}} WSWMR $|$wer aber beachtet/und ein Beachtender\\
6.&130.&2636.&469.&10327.&22.&5&834&400\_6\_20\_8\_400 \textcolor{red}{\textcjheb{t.hkwt}} TWKCT $|$Zucht/Zurechtweisung\\
7.&131.&2637.&474.&10332.&27.&4&36&10\_20\_2\_4 \textcolor{red}{\textcjheb{dbky}} JKBD $|$(er) wird geehrt\\
\end{tabular}\medskip \\
Ende des Verses 13.18\\
Verse: 365, Buchstaben: 30, 477, 10335, Totalwerte: 2786, 38798, 753339\\
\\
Armut und Schande dem, der Unterweisung verwirft; wer aber Zucht beachtet wird geehrt.\\
\newpage 
{\bf -- 13.19}\\
\medskip \\
\begin{tabular}{rrrrrrrrp{120mm}}
WV&WK&WB&ABK&ABB&ABV&AnzB&TW&Zahlencode \textcolor{red}{$\boldsymbol{Grundtext}$} Umschrift $|$"Ubersetzung(en)\\
1.&132.&2638.&478.&10336.&1.&4&412&400\_1\_6\_5 \textcolor{red}{\textcjheb{hw't}} TAWH $|$ein Begehren/(ein) Verlangen\\
2.&133.&2639.&482.&10340.&5.&4&70&50\_5\_10\_5 \textcolor{red}{\textcjheb{hyhn}} NHJH $|$erf"ulltes\\
3.&134.&2640.&486.&10344.&9.&4&672&400\_70\_200\_2 \textcolor{red}{\textcjheb{br`t}} TaRB $|$ist s"u"s/sie (=er) ist angenehm\\
4.&135.&2641.&490.&10348.&13.&4&460&30\_50\_80\_300 \textcolor{red}{\textcjheb{+spnl}} LNPS $|$der Seele/f"ur die Seele\\
5.&136.&2642.&494.&10352.&17.&6&884&6\_400\_6\_70\_2\_400 \textcolor{red}{\textcjheb{tb`wtw}} WTWaBT $|$und ein Gr"auel/und Abscheu\\
6.&137.&2643.&500.&10358.&23.&6&170&20\_60\_10\_30\_10\_40 \textcolor{red}{\textcjheb{mylysk}} KsJLJM $|$den Toren/der Toren\\
7.&138.&2644.&506.&10364.&29.&3&266&60\_6\_200 \textcolor{red}{\textcjheb{rws}} sWR $|$ist es zu weichen/(ist ein) Weichen\\
8.&139.&2645.&509.&10367.&32.&3&310&40\_200\_70 \textcolor{red}{\textcjheb{`rm}} MRa $|$vom B"osen\\
\end{tabular}\medskip \\
Ende des Verses 13.19\\
Verse: 366, Buchstaben: 34, 511, 10369, Totalwerte: 3244, 42042, 756583\\
\\
Ein erf"ulltes Begehren ist der Seele s"u"s, und den Toren ist's ein Greuel, vom B"osen zu weichen.\\
\newpage 
{\bf -- 13.20}\\
\medskip \\
\begin{tabular}{rrrrrrrrp{120mm}}
WV&WK&WB&ABK&ABB&ABV&AnzB&TW&Zahlencode \textcolor{red}{$\boldsymbol{Grundtext}$} Umschrift $|$"Ubersetzung(en)\\
1.&140.&2646.&512.&10370.&1.&4&61&5\_30\_6\_20 \textcolor{red}{\textcjheb{kwlh}} HLWK $|$wer umgeht\\
2.&141.&2647.&516.&10374.&5.&2&401&1\_400 \textcolor{red}{\textcjheb{t'}} AT $|$mit\\
3.&142.&2648.&518.&10376.&7.&5&118&8\_20\_40\_10\_40 \textcolor{red}{\textcjheb{mymk.h}} CKMJM $|$Weisen\\
4.&143.&2649.&523.&10381.&12.&4&74&6\_8\_20\_40 \textcolor{red}{\textcjheb{mk.hw}} WCKM $|$(der) wird weise\\
5.&144.&2650.&527.&10385.&16.&4&281&6\_200\_70\_5 \textcolor{red}{\textcjheb{h`rw}} WRaH $|$aber wer sich gesellt/und ein Verkehrender\\
6.&145.&2651.&531.&10389.&20.&6&170&20\_60\_10\_30\_10\_40 \textcolor{red}{\textcjheb{mylysk}} KsJLJM $|$zu Toren/(mit) Toren\\
7.&146.&2652.&537.&10395.&26.&4&286&10\_200\_6\_70 \textcolor{red}{\textcjheb{`wry}} JRWa $|$wird schlecht/der wird "ubel behandelt\\
\end{tabular}\medskip \\
Ende des Verses 13.20\\
Verse: 367, Buchstaben: 29, 540, 10398, Totalwerte: 1391, 43433, 757974\\
\\
Wer mit Weisen umgeht, wird weise; aber wer sich zu Toren gesellt, wird schlecht.\\
\newpage 
{\bf -- 13.21}\\
\medskip \\
\begin{tabular}{rrrrrrrrp{120mm}}
WV&WK&WB&ABK&ABB&ABV&AnzB&TW&Zahlencode \textcolor{red}{$\boldsymbol{Grundtext}$} Umschrift $|$"Ubersetzung(en)\\
1.&147.&2653.&541.&10399.&1.&5&68&8\_9\_1\_10\_40 \textcolor{red}{\textcjheb{my'.t.h}} CtAJM $|$(die) S"under\\
2.&148.&2654.&546.&10404.&6.&4&684&400\_200\_4\_80 \textcolor{red}{\textcjheb{pdrt}} TRDP $|$(sie (=es)) verfolgt\\
3.&149.&2655.&550.&10408.&10.&3&275&200\_70\_5 \textcolor{red}{\textcjheb{h`r}} RaH $|$das B"ose\\
4.&150.&2656.&553.&10411.&13.&3&407&6\_1\_400 \textcolor{red}{\textcjheb{t'w}} WAT $|$aber den/und **\\
5.&151.&2657.&556.&10414.&16.&6&254&90\_4\_10\_100\_10\_40 \textcolor{red}{\textcjheb{myqyd.s}} "sDJQJM $|$Gerechten\\
6.&152.&2658.&562.&10420.&22.&4&380&10\_300\_30\_40 \textcolor{red}{\textcjheb{ml+sy}} JSLM $|$wird man vergelten/er vergilt\\
7.&153.&2659.&566.&10424.&26.&3&17&9\_6\_2 \textcolor{red}{\textcjheb{bw.t}} tWB $|$(mit) Gutem\\
\end{tabular}\medskip \\
Ende des Verses 13.21\\
Verse: 368, Buchstaben: 28, 568, 10426, Totalwerte: 2085, 45518, 760059\\
\\
Das B"ose verfolgt die S"under, aber den Gerechten wird man mit Gutem vergelten.\\
\newpage 
{\bf -- 13.22}\\
\medskip \\
\begin{tabular}{rrrrrrrrp{120mm}}
WV&WK&WB&ABK&ABB&ABV&AnzB&TW&Zahlencode \textcolor{red}{$\boldsymbol{Grundtext}$} Umschrift $|$"Ubersetzung(en)\\
1.&154.&2660.&569.&10427.&1.&3&17&9\_6\_2 \textcolor{red}{\textcjheb{bw.t}} tWB $|$der Gute\\
2.&155.&2661.&572.&10430.&4.&5&108&10\_50\_8\_10\_30 \textcolor{red}{\textcjheb{ly.hny}} JNCJL $|$vererbt/(er) macht erben\\
3.&156.&2662.&577.&10435.&9.&3&62&2\_50\_10 \textcolor{red}{\textcjheb{ynb}} BNJ $|$auf Kindes-/S"ohne\\
4.&157.&2663.&580.&10438.&12.&4&102&2\_50\_10\_40 \textcolor{red}{\textcjheb{mynb}} BNJM $|$kinder/(von) S"ohnen\\
5.&158.&2664.&584.&10442.&16.&5&232&6\_90\_80\_6\_50 \textcolor{red}{\textcjheb{nwp.sw}} W"sPWN $|$aber (es) ist aufbewahrt/und er (=es) ist verwahrt\\
6.&159.&2665.&589.&10447.&21.&5&234&30\_90\_4\_10\_100 \textcolor{red}{\textcjheb{qyd.sl}} L"sDJQ $|$f"ur den Gerechten\\
7.&160.&2666.&594.&10452.&26.&3&48&8\_10\_30 \textcolor{red}{\textcjheb{ly.h}} CJL $|$Reichtum/(das) Verm"ogen\\
8.&161.&2667.&597.&10455.&29.&4&24&8\_6\_9\_1 \textcolor{red}{\textcjheb{'.tw.h}} CWtA $|$des S"unders\\
\end{tabular}\medskip \\
Ende des Verses 13.22\\
Verse: 369, Buchstaben: 32, 600, 10458, Totalwerte: 827, 46345, 760886\\
\\
Der Gute vererbt auf Kindeskinder, aber des S"unders Reichtum ist aufbewahrt f"ur den Gerechten.\\
\newpage 
{\bf -- 13.23}\\
\medskip \\
\begin{tabular}{rrrrrrrrp{120mm}}
WV&WK&WB&ABK&ABB&ABV&AnzB&TW&Zahlencode \textcolor{red}{$\boldsymbol{Grundtext}$} Umschrift $|$"Ubersetzung(en)\\
1.&162.&2668.&601.&10459.&1.&2&202&200\_2 \textcolor{red}{\textcjheb{br}} RB $|$viel\\
2.&163.&2669.&603.&10461.&3.&3&51&1\_20\_30 \textcolor{red}{\textcjheb{lk'}} AKL $|$Speise/Nahrung\\
3.&164.&2670.&606.&10464.&6.&3&260&50\_10\_200 \textcolor{red}{\textcjheb{ryn}} NJR $|$gibt der Neubruch/(bringt der) Neubruch\\
4.&165.&2671.&609.&10467.&9.&5&551&200\_1\_300\_10\_40 \textcolor{red}{\textcjheb{my+s'r}} RASJM $|$der Armen/(der) H"aupter\\
5.&166.&2672.&614.&10472.&14.&3&316&6\_10\_300 \textcolor{red}{\textcjheb{+syw}} WJS $|$aber (es) geht mancher/und er (=es) ist\\
6.&167.&2673.&617.&10475.&17.&4&195&50\_60\_80\_5 \textcolor{red}{\textcjheb{hpsn}} NsPH $|$zu Grunde/weggerafft werdend\\
7.&168.&2674.&621.&10479.&21.&3&33&2\_30\_1 \textcolor{red}{\textcjheb{'lb}} BLA $|$durch Un-\\
8.&169.&2675.&624.&10482.&24.&4&429&40\_300\_80\_9 \textcolor{red}{\textcjheb{.tp+sm}} MSPt $|$Redlichkeit/Recht\\
\end{tabular}\medskip \\
Ende des Verses 13.23\\
Verse: 370, Buchstaben: 27, 627, 10485, Totalwerte: 2037, 48382, 762923\\
\\
Der Neubruch der Armen gibt viel Speise, aber mancher geht zu Grunde durch Unrechtlichkeit.\\
\newpage 
{\bf -- 13.24}\\
\medskip \\
\begin{tabular}{rrrrrrrrp{120mm}}
WV&WK&WB&ABK&ABB&ABV&AnzB&TW&Zahlencode \textcolor{red}{$\boldsymbol{Grundtext}$} Umschrift $|$"Ubersetzung(en)\\
1.&170.&2676.&628.&10486.&1.&4&334&8\_6\_300\_20 \textcolor{red}{\textcjheb{k+sw.h}} CWSK $|$wer spart/ein Zur"uckhaltender\\
2.&171.&2677.&632.&10490.&5.&4&317&300\_2\_9\_6 \textcolor{red}{\textcjheb{w.tb+s}} SBtW $|$seine Rute\\
3.&172.&2678.&636.&10494.&9.&4&357&300\_6\_50\_1 \textcolor{red}{\textcjheb{'nw+s}} SWNA $|$hasst/(ist) hassender\\
4.&173.&2679.&640.&10498.&13.&3&58&2\_50\_6 \textcolor{red}{\textcjheb{wnb}} BNW $|$seinen Sohn\\
5.&174.&2680.&643.&10501.&16.&5&20&6\_1\_5\_2\_6 \textcolor{red}{\textcjheb{wbh'w}} WAHBW $|$aber wer ihn lieb hat/und sein Liebender\\
6.&175.&2681.&648.&10506.&21.&4&514&300\_8\_200\_6 \textcolor{red}{\textcjheb{wr.h+s}} SCRW $|$sucht ihn fr"uh heim/der heimsucht ihn\\
7.&176.&2682.&652.&10510.&25.&4&306&40\_6\_60\_200 \textcolor{red}{\textcjheb{rswm}} MWsR $|$mit Z"uchtigung\\
\end{tabular}\medskip \\
Ende des Verses 13.24\\
Verse: 371, Buchstaben: 28, 655, 10513, Totalwerte: 1906, 50288, 764829\\
\\
Wer seine Rute spart, ha"st seinen Sohn, aber wer ihn lieb hat, sucht ihn fr"uh heim mit Z"uchtigung.\\
\newpage 
{\bf -- 13.25}\\
\medskip \\
\begin{tabular}{rrrrrrrrp{120mm}}
WV&WK&WB&ABK&ABB&ABV&AnzB&TW&Zahlencode \textcolor{red}{$\boldsymbol{Grundtext}$} Umschrift $|$"Ubersetzung(en)\\
1.&177.&2683.&656.&10514.&1.&4&204&90\_4\_10\_100 \textcolor{red}{\textcjheb{qyd.s}} "sDJQ $|$der Gerechte/(ein) Rechtschaffener\\
2.&178.&2684.&660.&10518.&5.&3&51&1\_20\_30 \textcolor{red}{\textcjheb{lk'}} AKL $|$isst/(ist) essend\\
3.&179.&2685.&663.&10521.&8.&4&402&30\_300\_2\_70 \textcolor{red}{\textcjheb{`b+sl}} LSBa $|$bis zur S"attigung/bis zum S"attigen\\
4.&180.&2686.&667.&10525.&12.&4&436&50\_80\_300\_6 \textcolor{red}{\textcjheb{w+spn}} NPSW $|$seine(r) Seele\\
5.&181.&2687.&671.&10529.&16.&4&67&6\_2\_9\_50 \textcolor{red}{\textcjheb{n.tbw}} WBtN $|$aber der Leib/und der Bauch\\
6.&182.&2688.&675.&10533.&20.&5&620&200\_300\_70\_10\_40 \textcolor{red}{\textcjheb{my`+sr}} RSaJM $|$der Gesetzlosen/(der) Frevler\\
7.&183.&2689.&680.&10538.&25.&4&668&400\_8\_60\_200 \textcolor{red}{\textcjheb{rs.ht}} TCsR $|$(er) muss darben\\
\end{tabular}\medskip \\
Ende des Verses 13.25\\
Verse: 372, Buchstaben: 28, 683, 10541, Totalwerte: 2448, 52736, 767277\\
\\
Der Gerechte i"st bis zur S"attigung seiner Seele, aber der Leib der Gesetzlosen mu"s darben.\\
\\
{\bf Ende des Kapitels 13}\\
\newpage 
{\bf -- 14.1}\\
\medskip \\
\begin{tabular}{rrrrrrrrp{120mm}}
WV&WK&WB&ABK&ABB&ABV&AnzB&TW&Zahlencode \textcolor{red}{$\boldsymbol{Grundtext}$} Umschrift $|$"Ubersetzung(en)\\
1.&1.&2690.&1.&10542.&1.&5&474&8\_20\_40\_6\_400 \textcolor{red}{\textcjheb{twmk.h}} CKMWT $|$(die) Weisheit\\
2.&2.&2691.&6.&10547.&6.&4&400&50\_300\_10\_40 \textcolor{red}{\textcjheb{my+sn}} NSJM $|$der Frauen/(von) Frauen\\
3.&3.&2692.&10.&10551.&10.&4&457&2\_50\_400\_5 \textcolor{red}{\textcjheb{htnb}} BNTH $|$(sie) (er)baut\\
4.&4.&2693.&14.&10555.&14.&4&417&2\_10\_400\_5 \textcolor{red}{\textcjheb{htyb}} BJTH $|$ihr Haus\\
5.&5.&2694.&18.&10559.&18.&5&443&6\_1\_6\_30\_400 \textcolor{red}{\textcjheb{tlw'w}} WAWLT $|$und ihre Narrheit/und die Torheit\\
6.&6.&2695.&23.&10564.&23.&5&31&2\_10\_4\_10\_5 \textcolor{red}{\textcjheb{hydyb}} BJDJH $|$mit eigenen H"anden/mit ihren H"anden\\
7.&7.&2696.&28.&10569.&28.&6&721&400\_5\_200\_60\_50\_6 \textcolor{red}{\textcjheb{wnsrht}} THRsNW $|$(sie) rei"st nieder ihn (=es)\\
\end{tabular}\medskip \\
Ende des Verses 14.1\\
Verse: 373, Buchstaben: 33, 33, 10574, Totalwerte: 2943, 2943, 770220\\
\\
Der Weiber Weisheit baut ihr Haus, und ihre Narrheit rei"st es mit eigenen H"anden nieder.\\
\newpage 
{\bf -- 14.2}\\
\medskip \\
\begin{tabular}{rrrrrrrrp{120mm}}
WV&WK&WB&ABK&ABB&ABV&AnzB&TW&Zahlencode \textcolor{red}{$\boldsymbol{Grundtext}$} Umschrift $|$"Ubersetzung(en)\\
1.&8.&2697.&34.&10575.&1.&4&61&5\_6\_30\_20 \textcolor{red}{\textcjheb{klwh}} HWLK $|$wer wandelt/(ein) Wandelnder\\
2.&9.&2698.&38.&10579.&5.&5&518&2\_10\_300\_200\_6 \textcolor{red}{\textcjheb{wr+syb}} BJSRW $|$in seiner Geradheit\\
3.&10.&2699.&43.&10584.&10.&3&211&10\_200\_1 \textcolor{red}{\textcjheb{'ry}} JRA $|$f"urchtet/(ist) f"urchtend\\
4.&11.&2700.&46.&10587.&13.&4&26&10\_5\_6\_5 \textcolor{red}{\textcjheb{hwhy}} JHWH $|$Jahwe\\
5.&12.&2701.&50.&10591.&17.&5&99&6\_50\_30\_6\_7 \textcolor{red}{\textcjheb{zwlnw}} WNLWZ $|$war aber verkehrt ist/und ein Verdrehter\\
6.&13.&2702.&55.&10596.&22.&5&240&4\_200\_20\_10\_6 \textcolor{red}{\textcjheb{wykrd}} DRKJW $|$in seinen Wegen/(auf) seinen Wegen\\
7.&14.&2703.&60.&10601.&27.&5&26&2\_6\_7\_5\_6 \textcolor{red}{\textcjheb{whzwb}} BWZHW $|$verachtet ihn/(ist) verachtend ihn\\
\end{tabular}\medskip \\
Ende des Verses 14.2\\
Verse: 374, Buchstaben: 31, 64, 10605, Totalwerte: 1181, 4124, 771401\\
\\
Wer in seiner Geradheit wandelt, f"urchtet Jahwe; wer aber in seinen Wegen verkehrt ist, verachtet ihn.\\
\newpage 
{\bf -- 14.3}\\
\medskip \\
\begin{tabular}{rrrrrrrrp{120mm}}
WV&WK&WB&ABK&ABB&ABV&AnzB&TW&Zahlencode \textcolor{red}{$\boldsymbol{Grundtext}$} Umschrift $|$"Ubersetzung(en)\\
1.&15.&2704.&65.&10606.&1.&3&92&2\_80\_10 \textcolor{red}{\textcjheb{ypb}} BPJ $|$im Mund\\
2.&16.&2705.&68.&10609.&4.&4&47&1\_6\_10\_30 \textcolor{red}{\textcjheb{lyw'}} AWJL $|$des Narren\\
3.&17.&2706.&72.&10613.&8.&3&217&8\_9\_200 \textcolor{red}{\textcjheb{r.t.h}} CtR $|$ist eine Gerte/(ist ein) Zweig\\
4.&18.&2707.&75.&10616.&11.&4&15&3\_1\_6\_5 \textcolor{red}{\textcjheb{hw'g}} GAWH $|$(des) Hochmut(s)\\
5.&19.&2708.&79.&10620.&15.&5&796&6\_300\_80\_400\_10 \textcolor{red}{\textcjheb{ytp+sw}} WSPTJ $|$aber die Lippen/und (die) Lippen\\
6.&20.&2709.&84.&10625.&20.&5&118&8\_20\_40\_10\_40 \textcolor{red}{\textcjheb{mymk.h}} CKMJM $|$der Weisen\\
7.&21.&2710.&89.&10630.&25.&6&986&400\_300\_40\_6\_200\_40 \textcolor{red}{\textcjheb{mrwm+st}} TSMWRM $|$(sie) bewahren sie (davor)\\
\end{tabular}\medskip \\
Ende des Verses 14.3\\
Verse: 375, Buchstaben: 30, 94, 10635, Totalwerte: 2271, 6395, 773672\\
\\
Im Munde des Narren ist eine Gerte des Hochmuts; aber die Lippen der Weisen, sie bewahren sie.\\
\newpage 
{\bf -- 14.4}\\
\medskip \\
\begin{tabular}{rrrrrrrrp{120mm}}
WV&WK&WB&ABK&ABB&ABV&AnzB&TW&Zahlencode \textcolor{red}{$\boldsymbol{Grundtext}$} Umschrift $|$"Ubersetzung(en)\\
1.&22.&2711.&95.&10636.&1.&4&63&2\_1\_10\_50 \textcolor{red}{\textcjheb{ny'b}} BAJN $|$wo sind keine/ohne\\
2.&23.&2712.&99.&10640.&5.&5&161&1\_30\_80\_10\_40 \textcolor{red}{\textcjheb{mypl'}} ALPJM $|$Rinder\\
3.&24.&2713.&104.&10645.&10.&4&69&1\_2\_6\_60 \textcolor{red}{\textcjheb{swb'}} ABWs $|$ist die Krippe/(ist ein) Futtertrog\\
4.&25.&2714.&108.&10649.&14.&2&202&2\_200 \textcolor{red}{\textcjheb{rb}} BR $|$rein\\
5.&26.&2715.&110.&10651.&16.&3&208&6\_200\_2 \textcolor{red}{\textcjheb{brw}} WRB $|$aber viel/und viel\\
6.&27.&2716.&113.&10654.&19.&6&815&400\_2\_6\_1\_6\_400 \textcolor{red}{\textcjheb{tw'wbt}} TBWAWT $|$Ertrag\\
7.&28.&2717.&119.&10660.&25.&3&30&2\_20\_8 \textcolor{red}{\textcjheb{.hkb}} BKC $|$(ist) durch (die) Kraft\\
8.&29.&2718.&122.&10663.&28.&3&506&300\_6\_200 \textcolor{red}{\textcjheb{rw+s}} SWR $|$(eines) Stiers\\
\end{tabular}\medskip \\
Ende des Verses 14.4\\
Verse: 376, Buchstaben: 30, 124, 10665, Totalwerte: 2054, 8449, 775726\\
\\
Wo keine Rinder sind, ist die Krippe rein; aber viel Ertrag ist durch des Stieres Kraft.\\
\newpage 
{\bf -- 14.5}\\
\medskip \\
\begin{tabular}{rrrrrrrrp{120mm}}
WV&WK&WB&ABK&ABB&ABV&AnzB&TW&Zahlencode \textcolor{red}{$\boldsymbol{Grundtext}$} Umschrift $|$"Ubersetzung(en)\\
1.&30.&2719.&125.&10666.&1.&2&74&70\_4 \textcolor{red}{\textcjheb{d`}} aD $|$(ein) Zeuge\\
2.&31.&2720.&127.&10668.&3.&6&147&1\_40\_6\_50\_10\_40 \textcolor{red}{\textcjheb{mynwm'}} AMWNJM $|$treuer\\
3.&32.&2721.&133.&10674.&9.&2&31&30\_1 \textcolor{red}{\textcjheb{'l}} LA $|$nicht\\
4.&33.&2722.&135.&10676.&11.&4&39&10\_20\_7\_2 \textcolor{red}{\textcjheb{bzky}} JKZB $|$(er) l"ugt\\
5.&34.&2723.&139.&10680.&15.&5&114&6\_10\_80\_10\_8 \textcolor{red}{\textcjheb{.hypyw}} WJPJC $|$aber (es) spricht aus/und wer verbreitet\\
6.&35.&2724.&144.&10685.&20.&5&79&20\_7\_2\_10\_40 \textcolor{red}{\textcjheb{mybzk}} KZBJM $|$L"ugen\\
7.&36.&2725.&149.&10690.&25.&2&74&70\_4 \textcolor{red}{\textcjheb{d`}} aD $|$(ist) (ein) Zeuge\\
8.&37.&2726.&151.&10692.&27.&3&600&300\_100\_200 \textcolor{red}{\textcjheb{rq+s}} SQR $|$falscher/von Trug\\
\end{tabular}\medskip \\
Ende des Verses 14.5\\
Verse: 377, Buchstaben: 29, 153, 10694, Totalwerte: 1158, 9607, 776884\\
\\
Ein treuer Zeuge l"ugt nicht, aber ein falscher Zeuge spricht L"ugen aus.\\
\newpage 
{\bf -- 14.6}\\
\medskip \\
\begin{tabular}{rrrrrrrrp{120mm}}
WV&WK&WB&ABK&ABB&ABV&AnzB&TW&Zahlencode \textcolor{red}{$\boldsymbol{Grundtext}$} Umschrift $|$"Ubersetzung(en)\\
1.&38.&2727.&154.&10695.&1.&3&402&2\_100\_300 \textcolor{red}{\textcjheb{+sqb}} BQS $|$(er (=es)) sucht\\
2.&39.&2728.&157.&10698.&4.&2&120&30\_90 \textcolor{red}{\textcjheb{.sl}} L"s $|$der Sp"otter\\
3.&40.&2729.&159.&10700.&6.&4&73&8\_20\_40\_5 \textcolor{red}{\textcjheb{hmk.h}} CKMH $|$(nach) Weisheit\\
4.&41.&2730.&163.&10704.&10.&4&67&6\_1\_10\_50 \textcolor{red}{\textcjheb{ny'w}} WAJN $|$und nicht (sie) ist (da)\\
5.&42.&2731.&167.&10708.&14.&4&480&6\_4\_70\_400 \textcolor{red}{\textcjheb{t`dw}} WDaT $|$aber Erkenntnis/und Erkenntnis\\
6.&43.&2732.&171.&10712.&18.&5&138&30\_50\_2\_6\_50 \textcolor{red}{\textcjheb{nwbnl}} LNBWN $|$f"ur den Verst"andigen/f"ur einen Einsichtsvollen\\
7.&44.&2733.&176.&10717.&23.&3&180&50\_100\_30 \textcolor{red}{\textcjheb{lqn}} NQL $|$(ist) leicht\\
\end{tabular}\medskip \\
Ende des Verses 14.6\\
Verse: 378, Buchstaben: 25, 178, 10719, Totalwerte: 1460, 11067, 778344\\
\\
Der Sp"otter sucht Weisheit, und sie ist nicht da; aber f"ur den Verst"andigen ist Erkenntnis leicht.\\
\newpage 
{\bf -- 14.7}\\
\medskip \\
\begin{tabular}{rrrrrrrrp{120mm}}
WV&WK&WB&ABK&ABB&ABV&AnzB&TW&Zahlencode \textcolor{red}{$\boldsymbol{Grundtext}$} Umschrift $|$"Ubersetzung(en)\\
1.&45.&2734.&179.&10720.&1.&2&50&30\_20 \textcolor{red}{\textcjheb{kl}} LK $|$geh\\
2.&46.&2735.&181.&10722.&3.&4&97&40\_50\_3\_4 \textcolor{red}{\textcjheb{dgnm}} MNGD $|$(hinweg) von\\
3.&47.&2736.&185.&10726.&7.&4&341&30\_1\_10\_300 \textcolor{red}{\textcjheb{+sy'l}} LAJS $|$einem Mann\\
4.&48.&2737.&189.&10730.&11.&4&120&20\_60\_10\_30 \textcolor{red}{\textcjheb{lysk}} KsJL $|$t"orichten\\
5.&49.&2738.&193.&10734.&15.&3&38&6\_2\_30 \textcolor{red}{\textcjheb{lbw}} WBL $|$und nicht\\
6.&50.&2739.&196.&10737.&18.&4&484&10\_4\_70\_400 \textcolor{red}{\textcjheb{t`dy}} JDaT $|$bei wem du merkst/du erkennst\\
7.&51.&2740.&200.&10741.&22.&4&790&300\_80\_400\_10 \textcolor{red}{\textcjheb{ytp+s}} SPTJ $|$Lippen\\
8.&52.&2741.&204.&10745.&26.&3&474&4\_70\_400 \textcolor{red}{\textcjheb{t`d}} DaT $|$der Erkenntnis/(von) Erkenntnis\\
\end{tabular}\medskip \\
Ende des Verses 14.7\\
Verse: 379, Buchstaben: 28, 206, 10747, Totalwerte: 2394, 13461, 780738\\
\\
Geh hinweg von einem t"orichten Manne und bei wem du nicht Lippen der Erkenntnis merkst.\\
\newpage 
{\bf -- 14.8}\\
\medskip \\
\begin{tabular}{rrrrrrrrp{120mm}}
WV&WK&WB&ABK&ABB&ABV&AnzB&TW&Zahlencode \textcolor{red}{$\boldsymbol{Grundtext}$} Umschrift $|$"Ubersetzung(en)\\
1.&53.&2742.&207.&10748.&1.&4&468&8\_20\_40\_400 \textcolor{red}{\textcjheb{tmk.h}} CKMT $|$(die) Weisheit\\
2.&54.&2743.&211.&10752.&5.&4&316&70\_200\_6\_40 \textcolor{red}{\textcjheb{mwr`}} aRWM $|$des Klugen\\
3.&55.&2744.&215.&10756.&9.&4&67&5\_2\_10\_50 \textcolor{red}{\textcjheb{nybh}} HBJN $|$zu merken auf/(ist) Erkennen\\
4.&56.&2745.&219.&10760.&13.&4&230&4\_200\_20\_6 \textcolor{red}{\textcjheb{wkrd}} DRKW $|$seinen Weg\\
5.&57.&2746.&223.&10764.&17.&5&443&6\_1\_6\_30\_400 \textcolor{red}{\textcjheb{tlw'w}} WAWLT $|$und die Narrheit\\
6.&58.&2747.&228.&10769.&22.&6&170&20\_60\_10\_30\_10\_40 \textcolor{red}{\textcjheb{mylysk}} KsJLJM $|$der Toren\\
7.&59.&2748.&234.&10775.&28.&4&285&40\_200\_40\_5 \textcolor{red}{\textcjheb{hmrm}} MRMH $|$(ist) (Be)Trug\\
\end{tabular}\medskip \\
Ende des Verses 14.8\\
Verse: 380, Buchstaben: 31, 237, 10778, Totalwerte: 1979, 15440, 782717\\
\\
Die Weisheit des Klugen ist, auf seinen Weg zu merken, und die Narrheit der Toren ist Betrug.\\
\newpage 
{\bf -- 14.9}\\
\medskip \\
\begin{tabular}{rrrrrrrrp{120mm}}
WV&WK&WB&ABK&ABB&ABV&AnzB&TW&Zahlencode \textcolor{red}{$\boldsymbol{Grundtext}$} Umschrift $|$"Ubersetzung(en)\\
1.&60.&2749.&238.&10779.&1.&5&87&1\_6\_30\_10\_40 \textcolor{red}{\textcjheb{mylw'}} AWLJM $|$der Narren/die Narren\\
2.&61.&2750.&243.&10784.&6.&4&140&10\_30\_10\_90 \textcolor{red}{\textcjheb{.syly}} JLJ"s $|$spottet/es macht zum Gesp"ott\\
3.&62.&2751.&247.&10788.&10.&3&341&1\_300\_40 \textcolor{red}{\textcjheb{m+s'}} ASM $|$die Schuld/Verschuldung\\
4.&63.&2752.&250.&10791.&13.&4&68&6\_2\_10\_50 \textcolor{red}{\textcjheb{nybw}} WBJN $|$aber unter/und zwischen\\
5.&64.&2753.&254.&10795.&17.&5&560&10\_300\_200\_10\_40 \textcolor{red}{\textcjheb{myr+sy}} JSRJM $|$den Aufrichtigen/Geraden\\
6.&65.&2754.&259.&10800.&22.&4&346&200\_90\_6\_50 \textcolor{red}{\textcjheb{nw.sr}} R"sWN $|$ist Wohlwollen/(herrscht) Wohlgefallen\\
\end{tabular}\medskip \\
Ende des Verses 14.9\\
Verse: 381, Buchstaben: 25, 262, 10803, Totalwerte: 1542, 16982, 784259\\
\\
Die Schuld spottet der Narren, aber unter den Aufrichtigen ist Wohlwollen.\\
\newpage 
{\bf -- 14.10}\\
\medskip \\
\begin{tabular}{rrrrrrrrp{120mm}}
WV&WK&WB&ABK&ABB&ABV&AnzB&TW&Zahlencode \textcolor{red}{$\boldsymbol{Grundtext}$} Umschrift $|$"Ubersetzung(en)\\
1.&66.&2755.&263.&10804.&1.&2&32&30\_2 \textcolor{red}{\textcjheb{bl}} LB $|$das Herz\\
2.&67.&2756.&265.&10806.&3.&4&90&10\_6\_4\_70 \textcolor{red}{\textcjheb{`dwy}} JWDa $|$kennt/(ist) wissend\\
3.&68.&2757.&269.&10810.&7.&3&640&40\_200\_400 \textcolor{red}{\textcjheb{trm}} MRT $|$(die) Bitterkeit/um die Bitternis\\
4.&69.&2758.&272.&10813.&10.&4&436&50\_80\_300\_6 \textcolor{red}{\textcjheb{w+spn}} NPSW $|$seine eigene/seiner selbst\\
5.&70.&2759.&276.&10817.&14.&7&762&6\_2\_300\_40\_8\_400\_6 \textcolor{red}{\textcjheb{wt.hm+sbw}} WBSMCTW $|$und in seine Freude\\
6.&71.&2760.&283.&10824.&21.&2&31&30\_1 \textcolor{red}{\textcjheb{'l}} LA $|$nicht\\
7.&72.&2761.&285.&10826.&23.&5&682&10\_400\_70\_200\_2 \textcolor{red}{\textcjheb{br`ty}} JTaRB $|$kann sich mischen/er (=es) mischt sich ein\\
8.&73.&2762.&290.&10831.&28.&2&207&7\_200 \textcolor{red}{\textcjheb{rz}} ZR $|$(ein) Fremder\\
\end{tabular}\medskip \\
Ende des Verses 14.10\\
Verse: 382, Buchstaben: 29, 291, 10832, Totalwerte: 2880, 19862, 787139\\
\\
Das Herz kennt seine eigene Bitterkeit, und kein Fremder kann sich in seine Freude mischen.\\
\newpage 
{\bf -- 14.11}\\
\medskip \\
\begin{tabular}{rrrrrrrrp{120mm}}
WV&WK&WB&ABK&ABB&ABV&AnzB&TW&Zahlencode \textcolor{red}{$\boldsymbol{Grundtext}$} Umschrift $|$"Ubersetzung(en)\\
1.&74.&2763.&292.&10833.&1.&3&412&2\_10\_400 \textcolor{red}{\textcjheb{tyb}} BJT $|$das Haus\\
2.&75.&2764.&295.&10836.&4.&5&620&200\_300\_70\_10\_40 \textcolor{red}{\textcjheb{my`+sr}} RSaJM $|$der Gesetzlosen/(der) Frevler\\
3.&76.&2765.&300.&10841.&9.&4&354&10\_300\_40\_4 \textcolor{red}{\textcjheb{dm+sy}} JSMD $|$wird vertilgt werden/er (=es) wird vernichtet\\
4.&77.&2766.&304.&10845.&13.&4&42&6\_1\_5\_30 \textcolor{red}{\textcjheb{lh'w}} WAHL $|$aber das Zelt/und (das) Zelt\\
5.&78.&2767.&308.&10849.&17.&5&560&10\_300\_200\_10\_40 \textcolor{red}{\textcjheb{myr+sy}} JSRJM $|$der Aufrichtigen/der Geraden\\
6.&79.&2768.&313.&10854.&22.&5&308&10\_80\_200\_10\_8 \textcolor{red}{\textcjheb{.hyrpy}} JPRJC $|$wird emporbl"uhen/er (=es) gedeiht\\
\end{tabular}\medskip \\
Ende des Verses 14.11\\
Verse: 383, Buchstaben: 26, 317, 10858, Totalwerte: 2296, 22158, 789435\\
\\
Das Haus der Gesetzlosen wird vertilgt werden, aber das Zelt der Aufrichtigen wird emporbl"uhen.\\
\newpage 
{\bf -- 14.12}\\
\medskip \\
\begin{tabular}{rrrrrrrrp{120mm}}
WV&WK&WB&ABK&ABB&ABV&AnzB&TW&Zahlencode \textcolor{red}{$\boldsymbol{Grundtext}$} Umschrift $|$"Ubersetzung(en)\\
1.&80.&2769.&318.&10859.&1.&2&310&10\_300 \textcolor{red}{\textcjheb{+sy}} JS $|$da ist/es ist\\
2.&81.&2770.&320.&10861.&3.&3&224&4\_200\_20 \textcolor{red}{\textcjheb{krd}} DRK $|$der Weg/(mancher) Weg\\
3.&82.&2771.&323.&10864.&6.&3&510&10\_300\_200 \textcolor{red}{\textcjheb{r+sy}} JSR $|$der gerade erscheint/ein gerader\\
4.&83.&2772.&326.&10867.&9.&4&170&30\_80\_50\_10 \textcolor{red}{\textcjheb{ynpl}} LPNJ $|$/vor\\
5.&84.&2773.&330.&10871.&13.&3&311&1\_10\_300 \textcolor{red}{\textcjheb{+sy'}} AJS $|$einem Menschen/(einem) Mann\\
6.&85.&2774.&333.&10874.&16.&7&630&6\_1\_8\_200\_10\_400\_5 \textcolor{red}{\textcjheb{htyr.h'w}} WACRJTH $|$aber sein Ende/und sein Ende\\
7.&86.&2775.&340.&10881.&23.&4&234&4\_200\_20\_10 \textcolor{red}{\textcjheb{ykrd}} DRKJ $|$(sind) Wege\\
8.&87.&2776.&344.&10885.&27.&3&446&40\_6\_400 \textcolor{red}{\textcjheb{twm}} MWT $|$des Todes\\
\end{tabular}\medskip \\
Ende des Verses 14.12\\
Verse: 384, Buchstaben: 29, 346, 10887, Totalwerte: 2835, 24993, 792270\\
\\
Da ist der Weg, der einem Menschen gerade erscheint, aber sein Ende sind Wege des Todes.\\
\newpage 
{\bf -- 14.13}\\
\medskip \\
\begin{tabular}{rrrrrrrrp{120mm}}
WV&WK&WB&ABK&ABB&ABV&AnzB&TW&Zahlencode \textcolor{red}{$\boldsymbol{Grundtext}$} Umschrift $|$"Ubersetzung(en)\\
1.&88.&2777.&347.&10888.&1.&2&43&3\_40 \textcolor{red}{\textcjheb{mg}} GM $|$auch\\
2.&89.&2778.&349.&10890.&3.&5&416&2\_300\_8\_6\_100 \textcolor{red}{\textcjheb{qw.h+sb}} BSCWQ $|$beim Lachen\\
3.&90.&2779.&354.&10895.&8.&4&33&10\_20\_1\_2 \textcolor{red}{\textcjheb{b'ky}} JKAB $|$hat Kummer/er (=es) kann schmerzen\\
4.&91.&2780.&358.&10899.&12.&2&32&30\_2 \textcolor{red}{\textcjheb{bl}} LB $|$das Herz\\
5.&92.&2781.&360.&10901.&14.&7&630&6\_1\_8\_200\_10\_400\_5 \textcolor{red}{\textcjheb{htyr.h'w}} WACRJTH $|$und ihr Ende/und am Ende\\
6.&93.&2782.&367.&10908.&21.&4&353&300\_40\_8\_5 \textcolor{red}{\textcjheb{h.hm+s}} SMCH $|$der Freude/(der) Fr"ohlichkeit\\
7.&94.&2783.&371.&10912.&25.&4&414&400\_6\_3\_5 \textcolor{red}{\textcjheb{hgwt}} TWGH $|$ist Traurigkeit/(ist) Kummer\\
\end{tabular}\medskip \\
Ende des Verses 14.13\\
Verse: 385, Buchstaben: 28, 374, 10915, Totalwerte: 1921, 26914, 794191\\
\\
Auch beim Lachen hat das Herz Kummer, und ihr, der Freude, Ende ist Traurigkeit.\\
\newpage 
{\bf -- 14.14}\\
\medskip \\
\begin{tabular}{rrrrrrrrp{120mm}}
WV&WK&WB&ABK&ABB&ABV&AnzB&TW&Zahlencode \textcolor{red}{$\boldsymbol{Grundtext}$} Umschrift $|$"Ubersetzung(en)\\
1.&95.&2784.&375.&10916.&1.&6&280&40\_4\_200\_20\_10\_6 \textcolor{red}{\textcjheb{wykrdm}} MDRKJW $|$von seinen Wegen\\
2.&96.&2785.&381.&10922.&7.&4&382&10\_300\_2\_70 \textcolor{red}{\textcjheb{`b+sy}} JSBa $|$wird ges"attigt/er (=es) wird satt\\
3.&97.&2786.&385.&10926.&11.&3&69&60\_6\_3 \textcolor{red}{\textcjheb{gws}} sWG $|$wer abtr"unnigen\\
4.&98.&2787.&388.&10929.&14.&2&32&30\_2 \textcolor{red}{\textcjheb{bl}} LB $|$Herzens (ist)\\
5.&99.&2788.&390.&10931.&16.&6&162&6\_40\_70\_30\_10\_6 \textcolor{red}{\textcjheb{wyl`mw}} WMaLJW $|$und von dem was in ihm ist/und fern von ihm\\
6.&100.&2789.&396.&10937.&22.&3&311&1\_10\_300 \textcolor{red}{\textcjheb{+sy'}} AJS $|$der Mann/(ein) Mann\\
7.&101.&2790.&399.&10940.&25.&3&17&9\_6\_2 \textcolor{red}{\textcjheb{bw.t}} tWB $|$gute(r)\\
\end{tabular}\medskip \\
Ende des Verses 14.14\\
Verse: 386, Buchstaben: 27, 401, 10942, Totalwerte: 1253, 28167, 795444\\
\\
Von seinen Wegen wird ges"attigt, wer abtr"unnigen Herzens ist, und von dem, was in ihm ist, der gute Mann.\\
\newpage 
{\bf -- 14.15}\\
\medskip \\
\begin{tabular}{rrrrrrrrp{120mm}}
WV&WK&WB&ABK&ABB&ABV&AnzB&TW&Zahlencode \textcolor{red}{$\boldsymbol{Grundtext}$} Umschrift $|$"Ubersetzung(en)\\
1.&102.&2791.&402.&10943.&1.&3&490&80\_400\_10 \textcolor{red}{\textcjheb{ytp}} PTJ $|$der Einf"altige\\
2.&103.&2792.&405.&10946.&4.&5&111&10\_1\_40\_10\_50 \textcolor{red}{\textcjheb{nym'y}} JAMJN $|$(er) glaubt\\
3.&104.&2793.&410.&10951.&9.&3&80&30\_20\_30 \textcolor{red}{\textcjheb{lkl}} LKL $|$jedem/an jedes\\
4.&105.&2794.&413.&10954.&12.&3&206&4\_2\_200 \textcolor{red}{\textcjheb{rbd}} DBR $|$Wort\\
5.&106.&2795.&416.&10957.&15.&5&322&6\_70\_200\_6\_40 \textcolor{red}{\textcjheb{mwr`w}} WaRWM $|$aber der Kluge/und der Kluge\\
6.&107.&2796.&421.&10962.&20.&4&72&10\_2\_10\_50 \textcolor{red}{\textcjheb{nyby}} JBJN $|$(er) merkt\\
7.&108.&2797.&425.&10966.&24.&5&537&30\_1\_300\_200\_6 \textcolor{red}{\textcjheb{wr+s'l}} LASRW $|$auf seine(n) Schritt(e)\\
\end{tabular}\medskip \\
Ende des Verses 14.15\\
Verse: 387, Buchstaben: 28, 429, 10970, Totalwerte: 1818, 29985, 797262\\
\\
Der Einf"altige glaubt jedem Worte, aber der Kluge merkt auf seine Schritte.\\
\newpage 
{\bf -- 14.16}\\
\medskip \\
\begin{tabular}{rrrrrrrrp{120mm}}
WV&WK&WB&ABK&ABB&ABV&AnzB&TW&Zahlencode \textcolor{red}{$\boldsymbol{Grundtext}$} Umschrift $|$"Ubersetzung(en)\\
1.&109.&2798.&430.&10971.&1.&3&68&8\_20\_40 \textcolor{red}{\textcjheb{mk.h}} CKM $|$(der) Weise\\
2.&110.&2799.&433.&10974.&4.&3&211&10\_200\_1 \textcolor{red}{\textcjheb{'ry}} JRA $|$(er) f"urchtet (sich)\\
3.&111.&2800.&436.&10977.&7.&3&266&6\_60\_200 \textcolor{red}{\textcjheb{rsw}} WsR $|$und meidet/und ist ausweichend\\
4.&112.&2801.&439.&10980.&10.&3&310&40\_200\_70 \textcolor{red}{\textcjheb{`rm}} MRa $|$das B"ose/vor B"osem\\
5.&113.&2802.&442.&10983.&13.&5&126&6\_20\_60\_10\_30 \textcolor{red}{\textcjheb{lyskw}} WKsJL $|$aber der Tor/und der Tor\\
6.&114.&2803.&447.&10988.&18.&5&712&40\_400\_70\_2\_200 \textcolor{red}{\textcjheb{rb`tm}} MTaBR $|$braust auf/(ist) sich gehen lassend\\
7.&115.&2804.&452.&10993.&23.&5&31&6\_2\_6\_9\_8 \textcolor{red}{\textcjheb{.h.twbw}} WBWtC $|$und ist sorglos/und sich sicher w"ahnend\\
\end{tabular}\medskip \\
Ende des Verses 14.16\\
Verse: 388, Buchstaben: 27, 456, 10997, Totalwerte: 1724, 31709, 798986\\
\\
Der Weise f"urchtet sich und meidet das B"ose, aber der Tor braust auf und ist sorglos.\\
\newpage 
{\bf -- 14.17}\\
\medskip \\
\begin{tabular}{rrrrrrrrp{120mm}}
WV&WK&WB&ABK&ABB&ABV&AnzB&TW&Zahlencode \textcolor{red}{$\boldsymbol{Grundtext}$} Umschrift $|$"Ubersetzung(en)\\
1.&116.&2805.&457.&10998.&1.&3&390&100\_90\_200 \textcolor{red}{\textcjheb{r.sq}} Q"sR $|$der (kurz)\\
2.&117.&2806.&460.&11001.&4.&4&131&1\_80\_10\_40 \textcolor{red}{\textcjheb{myp'}} APJM $|$J"ahzornige/(zwei) Nasenl"ocher\\
3.&118.&2807.&464.&11005.&8.&4&385&10\_70\_300\_5 \textcolor{red}{\textcjheb{h+s`y}} JaSH $|$(er) begeht\\
4.&119.&2808.&468.&11009.&12.&4&437&1\_6\_30\_400 \textcolor{red}{\textcjheb{tlw'}} AWLT $|$Narrheit/Torheit\\
5.&120.&2809.&472.&11013.&16.&4&317&6\_1\_10\_300 \textcolor{red}{\textcjheb{+sy'w}} WAJS $|$und der Mann/und (ein) Mann\\
6.&121.&2810.&476.&11017.&20.&5&493&40\_7\_40\_6\_400 \textcolor{red}{\textcjheb{twmzm}} MZMWT $|$von R"anken/von T"ucken\\
7.&122.&2811.&481.&11022.&25.&4&361&10\_300\_50\_1 \textcolor{red}{\textcjheb{'n+sy}} JSNA $|$(er) wird gehasst\\
\end{tabular}\medskip \\
Ende des Verses 14.17\\
Verse: 389, Buchstaben: 28, 484, 11025, Totalwerte: 2514, 34223, 801500\\
\\
Der J"ahzornige begeht Narrheit, und der Mann von R"anken wird geha"st.\\
\newpage 
{\bf -- 14.18}\\
\medskip \\
\begin{tabular}{rrrrrrrrp{120mm}}
WV&WK&WB&ABK&ABB&ABV&AnzB&TW&Zahlencode \textcolor{red}{$\boldsymbol{Grundtext}$} Umschrift $|$"Ubersetzung(en)\\
1.&123.&2812.&485.&11026.&1.&4&94&50\_8\_30\_6 \textcolor{red}{\textcjheb{wl.hn}} NCLW $|$(es) erben/sie (=es) erb(t)en\\
2.&124.&2813.&489.&11030.&5.&5&531&80\_400\_1\_10\_40 \textcolor{red}{\textcjheb{my'tp}} PTAJM $|$(die) Einf"altige(n)\\
3.&125.&2814.&494.&11035.&10.&4&437&1\_6\_30\_400 \textcolor{red}{\textcjheb{tlw'}} AWLT $|$Narrheit/Torheit\\
4.&126.&2815.&498.&11039.&14.&7&372&6\_70\_200\_6\_40\_10\_40 \textcolor{red}{\textcjheb{mymwr`w}} WaRWMJM $|$aber die Klugen/und Kluge\\
5.&127.&2816.&505.&11046.&21.&5&636&10\_20\_400\_200\_6 \textcolor{red}{\textcjheb{wrtky}} JKTRW $|$werden gekr"ont/sie schm"ucken sich\\
6.&128.&2817.&510.&11051.&26.&3&474&4\_70\_400 \textcolor{red}{\textcjheb{t`d}} DaT $|$(mit) Erkenntnis\\
\end{tabular}\medskip \\
Ende des Verses 14.18\\
Verse: 390, Buchstaben: 28, 512, 11053, Totalwerte: 2544, 36767, 804044\\
\\
Die Einf"altigen erben Narrheit, die Klugen aber werden mit Erkenntnis gekr"ont.\\
\newpage 
{\bf -- 14.19}\\
\medskip \\
\begin{tabular}{rrrrrrrrp{120mm}}
WV&WK&WB&ABK&ABB&ABV&AnzB&TW&Zahlencode \textcolor{red}{$\boldsymbol{Grundtext}$} Umschrift $|$"Ubersetzung(en)\\
1.&129.&2818.&513.&11054.&1.&3&314&300\_8\_6 \textcolor{red}{\textcjheb{w.h+s}} SCW $|$(sie (=es)) beug(t)en sich\\
2.&130.&2819.&516.&11057.&4.&4&320&200\_70\_10\_40 \textcolor{red}{\textcjheb{my`r}} RaJM $|$(die) B"ose(n)\\
3.&131.&2820.&520.&11061.&8.&4&170&30\_80\_50\_10 \textcolor{red}{\textcjheb{ynpl}} LPNJ $|$vor\\
4.&132.&2821.&524.&11065.&12.&5&67&9\_6\_2\_10\_40 \textcolor{red}{\textcjheb{mybw.t}} tWBJM $|$den Guten\\
5.&133.&2822.&529.&11070.&17.&6&626&6\_200\_300\_70\_10\_40 \textcolor{red}{\textcjheb{my`+srw}} WRSaJM $|$und die Gesetzlosen/und Frevler\\
6.&134.&2823.&535.&11076.&23.&2&100&70\_30 \textcolor{red}{\textcjheb{l`}} aL $|$an\\
7.&135.&2824.&537.&11078.&25.&4&580&300\_70\_200\_10 \textcolor{red}{\textcjheb{yr`+s}} SaRJ $|$(den) Toren\\
8.&136.&2825.&541.&11082.&29.&4&204&90\_4\_10\_100 \textcolor{red}{\textcjheb{qyd.s}} "sDJQ $|$(des) Gerechten (stehen)\\
\end{tabular}\medskip \\
Ende des Verses 14.19\\
Verse: 391, Buchstaben: 32, 544, 11085, Totalwerte: 2381, 39148, 806425\\
\\
Die B"osen beugen sich vor den Guten, und die Gesetzlosen stehen an den Toren des Gerechten.\\
\newpage 
{\bf -- 14.20}\\
\medskip \\
\begin{tabular}{rrrrrrrrp{120mm}}
WV&WK&WB&ABK&ABB&ABV&AnzB&TW&Zahlencode \textcolor{red}{$\boldsymbol{Grundtext}$} Umschrift $|$"Ubersetzung(en)\\
1.&137.&2826.&545.&11086.&1.&2&43&3\_40 \textcolor{red}{\textcjheb{mg}} GM $|$selbst von/auch\\
2.&138.&2827.&547.&11088.&3.&5&311&30\_200\_70\_5\_6 \textcolor{red}{\textcjheb{wh`rl}} LRaHW $|$seinem N"achsten/seinem Gef"ahrten\\
3.&139.&2828.&552.&11093.&8.&4&361&10\_300\_50\_1 \textcolor{red}{\textcjheb{'n+sy}} JSNA $|$wird gehasst/er (=es) wird verhasst\\
4.&140.&2829.&556.&11097.&12.&2&500&200\_300 \textcolor{red}{\textcjheb{+sr}} RS $|$der Arme/(ein) Armer\\
5.&141.&2830.&558.&11099.&14.&5&24&6\_1\_5\_2\_10 \textcolor{red}{\textcjheb{ybh'w}} WAHBJ $|$aber derer die lieben/und Liebende\\
6.&142.&2831.&563.&11104.&19.&4&580&70\_300\_10\_200 \textcolor{red}{\textcjheb{ry+s`}} aSJR $|$den Reichen/(des) Reichen\\
7.&143.&2832.&567.&11108.&23.&4&252&200\_2\_10\_40 \textcolor{red}{\textcjheb{mybr}} RBJM $|$(sind) viele\\
\end{tabular}\medskip \\
Ende des Verses 14.20\\
Verse: 392, Buchstaben: 26, 570, 11111, Totalwerte: 2071, 41219, 808496\\
\\
Selbst von seinem N"achsten wird der Arme geha"st; aber derer, die den Reichen lieben, sind viele.\\
\newpage 
{\bf -- 14.21}\\
\medskip \\
\begin{tabular}{rrrrrrrrp{120mm}}
WV&WK&WB&ABK&ABB&ABV&AnzB&TW&Zahlencode \textcolor{red}{$\boldsymbol{Grundtext}$} Umschrift $|$"Ubersetzung(en)\\
1.&144.&2833.&571.&11112.&1.&2&9&2\_7 \textcolor{red}{\textcjheb{zb}} BZ $|$wer verachtet/ein Verachtender\\
2.&145.&2834.&573.&11114.&3.&5&311&30\_200\_70\_5\_6 \textcolor{red}{\textcjheb{wh`rl}} LRaHW $|$seinen N"achsten\\
3.&146.&2835.&578.&11119.&8.&4&24&8\_6\_9\_1 \textcolor{red}{\textcjheb{'.tw.h}} CWtA $|$s"undigt/(ist) s"undigend\\
4.&147.&2836.&582.&11123.&12.&6&160&6\_40\_8\_6\_50\_50 \textcolor{red}{\textcjheb{nnw.hmw}} WMCWNN $|$wer aber sich erbarmt/und ein Habender Erbarmen\\
5.&148.&2837.&588.&11129.&18.&5&180&70\_50\_10\_10\_40 \textcolor{red}{\textcjheb{myyn`}} aNJJM $|$der Elenden/(mit) Elenden\\
6.&149.&2838.&593.&11134.&23.&5&517&1\_300\_200\_10\_6 \textcolor{red}{\textcjheb{wyr+s'}} ASRJW $|$ist gl"ucklich/Heil ihm\\
\end{tabular}\medskip \\
Ende des Verses 14.21\\
Verse: 393, Buchstaben: 27, 597, 11138, Totalwerte: 1201, 42420, 809697\\
\\
Wer seinen N"achsten verachtet, s"undigt; wer aber der Elenden sich erbarmt, ist gl"ucklich.\\
\newpage 
{\bf -- 14.22}\\
\medskip \\
\begin{tabular}{rrrrrrrrp{120mm}}
WV&WK&WB&ABK&ABB&ABV&AnzB&TW&Zahlencode \textcolor{red}{$\boldsymbol{Grundtext}$} Umschrift $|$"Ubersetzung(en)\\
1.&150.&2839.&598.&11139.&1.&4&42&5\_30\_6\_1 \textcolor{red}{\textcjheb{'wlh}} HLWA $|$werden nicht/etwa nicht\\
2.&151.&2840.&602.&11143.&5.&4&486&10\_400\_70\_6 \textcolor{red}{\textcjheb{w`ty}} JTaW $|$(sie (=es)) gehen irre\\
3.&152.&2841.&606.&11147.&9.&4&518&8\_200\_300\_10 \textcolor{red}{\textcjheb{y+sr.h}} CRSJ $|$die schmieden/die Ersinnenden\\
4.&153.&2842.&610.&11151.&13.&2&270&200\_70 \textcolor{red}{\textcjheb{`r}} Ra $|$B"oses\\
5.&154.&2843.&612.&11153.&15.&4&78&6\_8\_60\_4 \textcolor{red}{\textcjheb{ds.hw}} WCsD $|$aber G"ute/und Liebe\\
6.&155.&2844.&616.&11157.&19.&4&447&6\_1\_40\_400 \textcolor{red}{\textcjheb{tm'w}} WAMT $|$und Wahrheit/und Treue\\
7.&156.&2845.&620.&11161.&23.&4&518&8\_200\_300\_10 \textcolor{red}{\textcjheb{y+sr.h}} CRSJ $|$finden die schmieden/(erfahren) Schmiedende\\
8.&157.&2846.&624.&11165.&27.&3&17&9\_6\_2 \textcolor{red}{\textcjheb{bw.t}} tWB $|$Gutes\\
\end{tabular}\medskip \\
Ende des Verses 14.22\\
Verse: 394, Buchstaben: 29, 626, 11167, Totalwerte: 2376, 44796, 812073\\
\\
Werden nicht irregehen, die B"oses schmieden, aber G"ute und Wahrheit finden, die Gutes schmieden?\\
\newpage 
{\bf -- 14.23}\\
\medskip \\
\begin{tabular}{rrrrrrrrp{120mm}}
WV&WK&WB&ABK&ABB&ABV&AnzB&TW&Zahlencode \textcolor{red}{$\boldsymbol{Grundtext}$} Umschrift $|$"Ubersetzung(en)\\
1.&158.&2847.&627.&11168.&1.&3&52&2\_20\_30 \textcolor{red}{\textcjheb{lkb}} BKL $|$bei jeder/bei allem\\
2.&159.&2848.&630.&11171.&4.&3&162&70\_90\_2 \textcolor{red}{\textcjheb{b.s`}} a"sB $|$M"uhe/Schmerz\\
3.&160.&2849.&633.&11174.&7.&4&30&10\_5\_10\_5 \textcolor{red}{\textcjheb{hyhy}} JHJH $|$wird sein/er (=es) ist\\
4.&161.&2850.&637.&11178.&11.&4&646&40\_6\_400\_200 \textcolor{red}{\textcjheb{rtwm}} MWTR $|$Gewinn\\
5.&162.&2851.&641.&11182.&15.&4&212&6\_4\_2\_200 \textcolor{red}{\textcjheb{rbdw}} WDBR $|$aber Gerede/und ein Wort\\
6.&163.&2852.&645.&11186.&19.&5&830&300\_80\_400\_10\_40 \textcolor{red}{\textcjheb{mytp+s}} SPTJM $|$(von) Lippen\\
7.&164.&2853.&650.&11191.&24.&2&21&1\_20 \textcolor{red}{\textcjheb{k'}} AK $|$gereicht nur/(f"uhrt) nur\\
8.&165.&2854.&652.&11193.&26.&6&344&30\_40\_8\_60\_6\_200 \textcolor{red}{\textcjheb{rws.hml}} LMCsWR $|$(zu(m)) Mangel\\
\end{tabular}\medskip \\
Ende des Verses 14.23\\
Verse: 395, Buchstaben: 31, 657, 11198, Totalwerte: 2297, 47093, 814370\\
\\
Bei jeder M"uhe wird Gewinn sein, aber Lippengerede gereicht nur zum Mangel.\\
\newpage 
{\bf -- 14.24}\\
\medskip \\
\begin{tabular}{rrrrrrrrp{120mm}}
WV&WK&WB&ABK&ABB&ABV&AnzB&TW&Zahlencode \textcolor{red}{$\boldsymbol{Grundtext}$} Umschrift $|$"Ubersetzung(en)\\
1.&166.&2855.&658.&11199.&1.&4&679&70\_9\_200\_400 \textcolor{red}{\textcjheb{tr.t`}} atRT $|$(die) Krone\\
2.&167.&2856.&662.&11203.&5.&5&118&8\_20\_40\_10\_40 \textcolor{red}{\textcjheb{mymk.h}} CKMJM $|$der Weisen\\
3.&168.&2857.&667.&11208.&10.&4&610&70\_300\_200\_40 \textcolor{red}{\textcjheb{mr+s`}} aSRM $|$(ist) ihr Reichtum\\
4.&169.&2858.&671.&11212.&14.&4&437&1\_6\_30\_400 \textcolor{red}{\textcjheb{tlw'}} AWLT $|$die Narrheit\\
5.&170.&2859.&675.&11216.&18.&6&170&20\_60\_10\_30\_10\_40 \textcolor{red}{\textcjheb{mylysk}} KsJLJM $|$der Toren\\
6.&171.&2860.&681.&11222.&24.&4&437&1\_6\_30\_400 \textcolor{red}{\textcjheb{tlw'}} AWLT $|$ist Narrheit/(bleibt) Narrheit\\
\end{tabular}\medskip \\
Ende des Verses 14.24\\
Verse: 396, Buchstaben: 27, 684, 11225, Totalwerte: 2451, 49544, 816821\\
\\
Der Weisen Krone ist ihr Reichtum; die Narrheit der Toren ist Narrheit.\\
\newpage 
{\bf -- 14.25}\\
\medskip \\
\begin{tabular}{rrrrrrrrp{120mm}}
WV&WK&WB&ABK&ABB&ABV&AnzB&TW&Zahlencode \textcolor{red}{$\boldsymbol{Grundtext}$} Umschrift $|$"Ubersetzung(en)\\
1.&172.&2861.&685.&11226.&1.&4&170&40\_90\_10\_30 \textcolor{red}{\textcjheb{ly.sm}} M"sJL $|$(es) errettet/(ein) Rettender\\
2.&173.&2862.&689.&11230.&5.&5&836&50\_80\_300\_6\_400 \textcolor{red}{\textcjheb{tw+spn}} NPSWT $|$Seelen\\
3.&174.&2863.&694.&11235.&10.&2&74&70\_4 \textcolor{red}{\textcjheb{d`}} aD $|$(ist) (ein) Zeuge\\
4.&175.&2864.&696.&11237.&12.&3&441&1\_40\_400 \textcolor{red}{\textcjheb{tm'}} AMT $|$wahrhaftiger/(der) Wahrheit\\
5.&176.&2865.&699.&11240.&15.&4&104&6\_10\_80\_8 \textcolor{red}{\textcjheb{.hpyw}} WJPC $|$wer aber ausspricht/und wer macht hervorbringen\\
6.&177.&2866.&703.&11244.&19.&5&79&20\_7\_2\_10\_40 \textcolor{red}{\textcjheb{mybzk}} KZBJM $|$L"ugen\\
7.&178.&2867.&708.&11249.&24.&4&285&40\_200\_40\_5 \textcolor{red}{\textcjheb{hmrm}} MRMH $|$ist lauter Trug/(ist) ein Verr"ater\\
\end{tabular}\medskip \\
Ende des Verses 14.25\\
Verse: 397, Buchstaben: 27, 711, 11252, Totalwerte: 1989, 51533, 818810\\
\\
Ein wahrhaftiger Zeuge errettet Seelen; wer aber L"ugen ausspricht, ist lauter Trug.\\
\newpage 
{\bf -- 14.26}\\
\medskip \\
\begin{tabular}{rrrrrrrrp{120mm}}
WV&WK&WB&ABK&ABB&ABV&AnzB&TW&Zahlencode \textcolor{red}{$\boldsymbol{Grundtext}$} Umschrift $|$"Ubersetzung(en)\\
1.&179.&2868.&712.&11253.&1.&5&613&2\_10\_200\_1\_400 \textcolor{red}{\textcjheb{t'ryb}} BJRAT $|$in der Furcht\\
2.&180.&2869.&717.&11258.&6.&4&26&10\_5\_6\_5 \textcolor{red}{\textcjheb{hwhy}} JHWH $|$(vor) Jahwe(s)\\
3.&181.&2870.&721.&11262.&10.&4&59&40\_2\_9\_8 \textcolor{red}{\textcjheb{.h.tbm}} MBtC $|$ist ein Vertrauen/(besteht) Vertrauen\\
4.&182.&2871.&725.&11266.&14.&2&77&70\_7 \textcolor{red}{\textcjheb{z`}} aZ $|$starkes\\
5.&183.&2872.&727.&11268.&16.&6&104&6\_30\_2\_50\_10\_6 \textcolor{red}{\textcjheb{wynblw}} WLBNJW $|$und seine Kinder/und f"ur seine S"ohne\\
6.&184.&2873.&733.&11274.&22.&4&30&10\_5\_10\_5 \textcolor{red}{\textcjheb{hyhy}} JHJH $|$haben eine/er ist\\
7.&185.&2874.&737.&11278.&26.&4&113&40\_8\_60\_5 \textcolor{red}{\textcjheb{hs.hm}} MCsH $|$Zuflucht\\
\end{tabular}\medskip \\
Ende des Verses 14.26\\
Verse: 398, Buchstaben: 29, 740, 11281, Totalwerte: 1022, 52555, 819832\\
\\
In der Furcht Jahwes ist ein starkes Vertrauen, und seine Kinder haben eine Zuflucht.\\
\newpage 
{\bf -- 14.27}\\
\medskip \\
\begin{tabular}{rrrrrrrrp{120mm}}
WV&WK&WB&ABK&ABB&ABV&AnzB&TW&Zahlencode \textcolor{red}{$\boldsymbol{Grundtext}$} Umschrift $|$"Ubersetzung(en)\\
1.&186.&2875.&741.&11282.&1.&4&611&10\_200\_1\_400 \textcolor{red}{\textcjheb{t'ry}} JRAT $|$(die) Furcht\\
2.&187.&2876.&745.&11286.&5.&4&26&10\_5\_6\_5 \textcolor{red}{\textcjheb{hwhy}} JHWH $|$(vor) Jahwe(s)\\
3.&188.&2877.&749.&11290.&9.&4&346&40\_100\_6\_200 \textcolor{red}{\textcjheb{rwqm}} MQWR $|$ist ein Born/(ist) (eine) Quelle\\
4.&189.&2878.&753.&11294.&13.&4&68&8\_10\_10\_40 \textcolor{red}{\textcjheb{myy.h}} CJJM $|$des Lebens\\
5.&190.&2879.&757.&11298.&17.&4&296&30\_60\_6\_200 \textcolor{red}{\textcjheb{rwsl}} LsWR $|$um zu entgehen/(um) zu entweichen\\
6.&191.&2880.&761.&11302.&21.&5&490&40\_40\_100\_300\_10 \textcolor{red}{\textcjheb{y+sqmm}} MMQSJ $|$den Fallstricken/von den Schlingen\\
7.&192.&2881.&766.&11307.&26.&3&446&40\_6\_400 \textcolor{red}{\textcjheb{twm}} MWT $|$des Todes\\
\end{tabular}\medskip \\
Ende des Verses 14.27\\
Verse: 399, Buchstaben: 28, 768, 11309, Totalwerte: 2283, 54838, 822115\\
\\
Die Furcht Jahwes ist ein Born des Lebens, um zu entgehen den Fallstricken des Todes.\\
\newpage 
{\bf -- 14.28}\\
\medskip \\
\begin{tabular}{rrrrrrrrp{120mm}}
WV&WK&WB&ABK&ABB&ABV&AnzB&TW&Zahlencode \textcolor{red}{$\boldsymbol{Grundtext}$} Umschrift $|$"Ubersetzung(en)\\
1.&193.&2882.&769.&11310.&1.&3&204&2\_200\_2 \textcolor{red}{\textcjheb{brb}} BRB $|$in der Menge\\
2.&194.&2883.&772.&11313.&4.&2&110&70\_40 \textcolor{red}{\textcjheb{m`}} aM $|$des Volkes\\
3.&195.&2884.&774.&11315.&6.&4&609&5\_4\_200\_400 \textcolor{red}{\textcjheb{trdh}} HDRT $|$ist die Herrlichkeit/(beruht) die Herrlichkeit\\
4.&196.&2885.&778.&11319.&10.&3&90&40\_30\_20 \textcolor{red}{\textcjheb{klm}} MLK $|$des K"onigs\\
5.&197.&2886.&781.&11322.&13.&5&149&6\_2\_1\_80\_60 \textcolor{red}{\textcjheb{sp'bw}} WBAPs $|$aber im Schwinden/und im Mangel\\
6.&198.&2887.&786.&11327.&18.&3&71&30\_1\_40 \textcolor{red}{\textcjheb{m'l}} LAM $|$der Bev"olkerung/an Volk\\
7.&199.&2888.&789.&11330.&21.&4&848&40\_8\_400\_400 \textcolor{red}{\textcjheb{tt.hm}} MCTT $|$(liegt) (der) Untergang\\
8.&200.&2889.&793.&11334.&25.&4&263&200\_7\_6\_50 \textcolor{red}{\textcjheb{nwzr}} RZWN $|$eines F"ursten\\
\end{tabular}\medskip \\
Ende des Verses 14.28\\
Verse: 400, Buchstaben: 28, 796, 11337, Totalwerte: 2344, 57182, 824459\\
\\
In der Menge des Volkes ist die Herrlichkeit eines K"onigs, aber im Schwinden der Bev"olkerung eines F"ursten Untergang.\\
\newpage 
{\bf -- 14.29}\\
\medskip \\
\begin{tabular}{rrrrrrrrp{120mm}}
WV&WK&WB&ABK&ABB&ABV&AnzB&TW&Zahlencode \textcolor{red}{$\boldsymbol{Grundtext}$} Umschrift $|$"Ubersetzung(en)\\
1.&201.&2890.&797.&11338.&1.&3&221&1\_200\_20 \textcolor{red}{\textcjheb{kr'}} ARK $|$ein Lang-/(die) L"ange\\
2.&202.&2891.&800.&11341.&4.&4&131&1\_80\_10\_40 \textcolor{red}{\textcjheb{myp'}} APJM $|$m"utiger/(vom) Zorn\\
3.&203.&2892.&804.&11345.&8.&2&202&200\_2 \textcolor{red}{\textcjheb{br}} RB $|$(wer) viel\\
4.&204.&2893.&806.&11347.&10.&5&463&400\_2\_6\_50\_5 \textcolor{red}{\textcjheb{hnwbt}} TBWNH $|$Verstand hat/Einsicht (hat)\\
5.&205.&2894.&811.&11352.&15.&4&396&6\_100\_90\_200 \textcolor{red}{\textcjheb{r.sqw}} WQ"sR $|$aber ein J"ah-/und der kurze\\
6.&206.&2895.&815.&11356.&19.&3&214&200\_6\_8 \textcolor{red}{\textcjheb{.hwr}} RWC $|$zorniger/(an) Geist\\
7.&207.&2896.&818.&11359.&22.&4&290&40\_200\_10\_40 \textcolor{red}{\textcjheb{myrm}} MRJM $|$erh"oht/tr"agt davon\\
8.&208.&2897.&822.&11363.&26.&4&437&1\_6\_30\_400 \textcolor{red}{\textcjheb{tlw'}} AWLT $|$die Narrheit/Torheit\\
\end{tabular}\medskip \\
Ende des Verses 14.29\\
Verse: 401, Buchstaben: 29, 825, 11366, Totalwerte: 2354, 59536, 826813\\
\\
Ein Langm"utiger hat viel Verstand, aber ein J"ahzorniger erh"oht die Narrheit.\\
\newpage 
{\bf -- 14.30}\\
\medskip \\
\begin{tabular}{rrrrrrrrp{120mm}}
WV&WK&WB&ABK&ABB&ABV&AnzB&TW&Zahlencode \textcolor{red}{$\boldsymbol{Grundtext}$} Umschrift $|$"Ubersetzung(en)\\
1.&209.&2898.&826.&11367.&1.&3&28&8\_10\_10 \textcolor{red}{\textcjheb{yy.h}} CJJ $|$Leben\\
2.&210.&2899.&829.&11370.&4.&5&552&2\_300\_200\_10\_40 \textcolor{red}{\textcjheb{myr+sb}} BSRJM $|$des Leibes/f"ur den Leib\\
3.&211.&2900.&834.&11375.&9.&2&32&30\_2 \textcolor{red}{\textcjheb{bl}} LB $|$(ist) (ein) Herz\\
4.&212.&2901.&836.&11377.&11.&4&321&40\_200\_80\_1 \textcolor{red}{\textcjheb{'prm}} MRPA $|$gelassenes/der Gelassenheit\\
5.&213.&2902.&840.&11381.&15.&4&308&6\_200\_100\_2 \textcolor{red}{\textcjheb{bqrw}} WRQB $|$aber F"aulnis/und F"aulnis\\
6.&214.&2903.&844.&11385.&19.&5&606&70\_90\_40\_6\_400 \textcolor{red}{\textcjheb{twm.s`}} a"sMWT $|$der Gebeine/(f"ur) die Gebeine\\
7.&215.&2904.&849.&11390.&24.&4&156&100\_50\_1\_5 \textcolor{red}{\textcjheb{h'nq}} QNAH $|$Ereiferung/(ist) Eifersucht\\
\end{tabular}\medskip \\
Ende des Verses 14.30\\
Verse: 402, Buchstaben: 27, 852, 11393, Totalwerte: 2003, 61539, 828816\\
\\
Ein gelassenes Herz ist des Leibes Leben, aber Ereiferung ist F"aulnis der Gebeine.\\
\newpage 
{\bf -- 14.31}\\
\medskip \\
\begin{tabular}{rrrrrrrrp{120mm}}
WV&WK&WB&ABK&ABB&ABV&AnzB&TW&Zahlencode \textcolor{red}{$\boldsymbol{Grundtext}$} Umschrift $|$"Ubersetzung(en)\\
1.&216.&2905.&853.&11394.&1.&3&470&70\_300\_100 \textcolor{red}{\textcjheb{q+s`}} aSQ $|$wer bedr"uckt/ein Bedr"uckender\\
2.&217.&2906.&856.&11397.&4.&2&34&4\_30 \textcolor{red}{\textcjheb{ld}} DL $|$den Armen/(einen) Schwachen\\
3.&218.&2907.&858.&11399.&6.&3&288&8\_200\_80 \textcolor{red}{\textcjheb{pr.h}} CRP $|$verh"ohnt/der achtet(e) gering\\
4.&219.&2908.&861.&11402.&9.&4&381&70\_300\_5\_6 \textcolor{red}{\textcjheb{wh+s`}} aSHW $|$den der ihn gemachte hat/seinen Sch"opfer\\
5.&220.&2909.&865.&11406.&13.&6&78&6\_40\_20\_2\_4\_6 \textcolor{red}{\textcjheb{wdbkmw}} WMKBDW $|$wer aber (es) ehrt ihn/und ein Ehrender ihn\\
6.&221.&2910.&871.&11412.&19.&3&108&8\_50\_50 \textcolor{red}{\textcjheb{nn.h}} CNN $|$sich erbarmt/ist ein sich Erbarmender\\
7.&222.&2911.&874.&11415.&22.&5&69&1\_2\_10\_6\_50 \textcolor{red}{\textcjheb{nwyb'}} ABJWN $|$des D"urftigen/(eines) Bed"urftigen\\
\end{tabular}\medskip \\
Ende des Verses 14.31\\
Verse: 403, Buchstaben: 26, 878, 11419, Totalwerte: 1428, 62967, 830244\\
\\
Wer den Armen bedr"uckt, verh"ohnt den, der ihn gemacht hat; wer aber des D"urftigen sich erbarmt, ehrt ihn.\\
\newpage 
{\bf -- 14.32}\\
\medskip \\
\begin{tabular}{rrrrrrrrp{120mm}}
WV&WK&WB&ABK&ABB&ABV&AnzB&TW&Zahlencode \textcolor{red}{$\boldsymbol{Grundtext}$} Umschrift $|$"Ubersetzung(en)\\
1.&223.&2912.&879.&11420.&1.&5&678&2\_200\_70\_400\_6 \textcolor{red}{\textcjheb{wt`rb}} BRaTW $|$in seinem Ungl"uck/durch seine Bosheit\\
2.&224.&2913.&884.&11425.&6.&4&27&10\_4\_8\_5 \textcolor{red}{\textcjheb{h.hdy}} JDCH $|$wird umgesto"sen/er (=es) wird gest"urzt\\
3.&225.&2914.&888.&11429.&10.&3&570&200\_300\_70 \textcolor{red}{\textcjheb{`+sr}} RSa $|$der Gesetzlose/(der) Frevler\\
4.&226.&2915.&891.&11432.&13.&4&79&6\_8\_60\_5 \textcolor{red}{\textcjheb{hs.hw}} WCsH $|$aber (es) vertraut auch/und sich bergend (ist)\\
5.&227.&2916.&895.&11436.&17.&5&454&2\_40\_6\_400\_6 \textcolor{red}{\textcjheb{wtwmb}} BMWTW $|$in seinem Tod/bei seinem Tod\\
6.&228.&2917.&900.&11441.&22.&4&204&90\_4\_10\_100 \textcolor{red}{\textcjheb{qyd.s}} "sDJQ $|$der Gerechte/(ein) Rechtschaffener\\
\end{tabular}\medskip \\
Ende des Verses 14.32\\
Verse: 404, Buchstaben: 25, 903, 11444, Totalwerte: 2012, 64979, 832256\\
\\
In seinem Ungl"uck wird der Gesetzlose umgesto"sen, aber der Gerechte vertraut auch in seinem Tode.\\
\newpage 
{\bf -- 14.33}\\
\medskip \\
\begin{tabular}{rrrrrrrrp{120mm}}
WV&WK&WB&ABK&ABB&ABV&AnzB&TW&Zahlencode \textcolor{red}{$\boldsymbol{Grundtext}$} Umschrift $|$"Ubersetzung(en)\\
1.&229.&2918.&904.&11445.&1.&3&34&2\_30\_2 \textcolor{red}{\textcjheb{blb}} BLB $|$im Herzen\\
2.&230.&2919.&907.&11448.&4.&4&108&50\_2\_6\_50 \textcolor{red}{\textcjheb{nwbn}} NBWN $|$(des) Verst"andigen\\
3.&231.&2920.&911.&11452.&8.&4&464&400\_50\_6\_8 \textcolor{red}{\textcjheb{.hwnt}} TNWC $|$(sie (=es)) ruht\\
4.&232.&2921.&915.&11456.&12.&4&73&8\_20\_40\_5 \textcolor{red}{\textcjheb{hmk.h}} CKMH $|$die Weisheit\\
5.&233.&2922.&919.&11460.&16.&5&310&6\_2\_100\_200\_2 \textcolor{red}{\textcjheb{brqbw}} WBQRB $|$aber was im Inneren ist/und inmitten\\
6.&234.&2923.&924.&11465.&21.&6&170&20\_60\_10\_30\_10\_40 \textcolor{red}{\textcjheb{mylysk}} KsJLJM $|$der Toren\\
7.&235.&2924.&930.&11471.&27.&4&480&400\_6\_4\_70 \textcolor{red}{\textcjheb{`dwt}} TWDa $|$tut sich kund/sie wird bekannt\\
\end{tabular}\medskip \\
Ende des Verses 14.33\\
Verse: 405, Buchstaben: 30, 933, 11474, Totalwerte: 1639, 66618, 833895\\
\\
Die Weisheit ruht im Herzen des Verst"andigen; aber was im Inneren der Toren ist, tut sich kund.\\
\newpage 
{\bf -- 14.34}\\
\medskip \\
\begin{tabular}{rrrrrrrrp{120mm}}
WV&WK&WB&ABK&ABB&ABV&AnzB&TW&Zahlencode \textcolor{red}{$\boldsymbol{Grundtext}$} Umschrift $|$"Ubersetzung(en)\\
1.&236.&2925.&934.&11475.&1.&4&199&90\_4\_100\_5 \textcolor{red}{\textcjheb{hqd.s}} "sDQH $|$Gerechtigkeit\\
2.&237.&2926.&938.&11479.&5.&5&686&400\_200\_6\_40\_40 \textcolor{red}{\textcjheb{mmwrt}} TRWMM $|$(sie) erh"oht\\
3.&238.&2927.&943.&11484.&10.&3&19&3\_6\_10 \textcolor{red}{\textcjheb{ywg}} GWJ $|$eine Nation/(ein) Volk\\
4.&239.&2928.&946.&11487.&13.&4&78&6\_8\_60\_4 \textcolor{red}{\textcjheb{ds.hw}} WCsD $|$aber Schande/und Schande\\
5.&240.&2929.&950.&11491.&17.&5&121&30\_1\_40\_10\_40 \textcolor{red}{\textcjheb{mym'l}} LAMJM $|$der V"olker/(der) Nationen\\
6.&241.&2930.&955.&11496.&22.&4&418&8\_9\_1\_400 \textcolor{red}{\textcjheb{t'.t.h}} CtAT $|$(ist) (die) S"unde\\
\end{tabular}\medskip \\
Ende des Verses 14.34\\
Verse: 406, Buchstaben: 25, 958, 11499, Totalwerte: 1521, 68139, 835416\\
\\
Gerechtigkeit erh"oht eine Nation, aber S"unde ist der V"olker Schande.\\
\newpage 
{\bf -- 14.35}\\
\medskip \\
\begin{tabular}{rrrrrrrrp{120mm}}
WV&WK&WB&ABK&ABB&ABV&AnzB&TW&Zahlencode \textcolor{red}{$\boldsymbol{Grundtext}$} Umschrift $|$"Ubersetzung(en)\\
1.&242.&2931.&959.&11500.&1.&4&346&200\_90\_6\_50 \textcolor{red}{\textcjheb{nw.sr}} R"sWN $|$Gunst/Wohlgefallen\\
2.&243.&2932.&963.&11504.&5.&3&90&40\_30\_20 \textcolor{red}{\textcjheb{klm}} MLK $|$des K"onigs\\
3.&244.&2933.&966.&11507.&8.&4&106&30\_70\_2\_4 \textcolor{red}{\textcjheb{db`l}} LaBD $|$wird zuteil dem Knecht/(gilt) dem Diener\\
4.&245.&2934.&970.&11511.&12.&5&400&40\_300\_20\_10\_30 \textcolor{red}{\textcjheb{lyk+sm}} MSKJL $|$einsichtigen/verst"andnisvollen\\
5.&246.&2935.&975.&11516.&17.&6&684&6\_70\_2\_200\_400\_6 \textcolor{red}{\textcjheb{wtrb`w}} WaBRTW $|$aber Gegenstand seines Grimmes/und sein Zorn\\
6.&247.&2936.&981.&11522.&23.&4&420&400\_5\_10\_5 \textcolor{red}{\textcjheb{hyht}} THJH $|$wird sein/er trifft\\
7.&248.&2937.&985.&11526.&27.&4&352&40\_2\_10\_300 \textcolor{red}{\textcjheb{+sybm}} MBJS $|$der Sch"andliche/(dem) Schande Machenden\\
\end{tabular}\medskip \\
Ende des Verses 14.35\\
Verse: 407, Buchstaben: 30, 988, 11529, Totalwerte: 2398, 70537, 837814\\
\\
Des K"onigs Gunst wird dem einsichtigen Knechte zuteil; aber der Sch"andliche wird Gegenstand seines Grimmes sein.\\
\\
{\bf Ende des Kapitels 14}\\
\newpage 
{\bf -- 15.1}\\
\medskip \\
\begin{tabular}{rrrrrrrrp{120mm}}
WV&WK&WB&ABK&ABB&ABV&AnzB&TW&Zahlencode \textcolor{red}{$\boldsymbol{Grundtext}$} Umschrift $|$"Ubersetzung(en)\\
1.&1.&2938.&1.&11530.&1.&4&165&40\_70\_50\_5 \textcolor{red}{\textcjheb{hn`m}} MaNH $|$(eine) Antwort\\
2.&2.&2939.&5.&11534.&5.&2&220&200\_20 \textcolor{red}{\textcjheb{kr}} RK $|$gelinde/milde\\
3.&3.&2940.&7.&11536.&7.&4&322&10\_300\_10\_2 \textcolor{red}{\textcjheb{by+sy}} JSJB $|$wendet ab/er (=sie) macht abwenden\\
4.&4.&2941.&11.&11540.&11.&3&53&8\_40\_5 \textcolor{red}{\textcjheb{hm.h}} CMH $|$den Grimm/Glut\\
5.&5.&2942.&14.&11543.&14.&4&212&6\_4\_2\_200 \textcolor{red}{\textcjheb{rbdw}} WDBR $|$aber ein Wort/und ein Wort\\
6.&6.&2943.&18.&11547.&18.&3&162&70\_90\_2 \textcolor{red}{\textcjheb{b.s`}} a"sB $|$kr"ankendes/der Kr"ankung\\
7.&7.&2944.&21.&11550.&21.&4&115&10\_70\_30\_5 \textcolor{red}{\textcjheb{hl`y}} JaLH $|$erregt/er (=es) erweckt\\
8.&8.&2945.&25.&11554.&25.&2&81&1\_80 \textcolor{red}{\textcjheb{p'}} AP $|$(den) Zorn\\
\end{tabular}\medskip \\
Ende des Verses 15.1\\
Verse: 408, Buchstaben: 26, 26, 11555, Totalwerte: 1330, 1330, 839144\\
\\
Eine gelinde Antwort wendet den Grimm ab, aber ein kr"ankendes Wort erregt den Zorn.\\
\newpage 
{\bf -- 15.2}\\
\medskip \\
\begin{tabular}{rrrrrrrrp{120mm}}
WV&WK&WB&ABK&ABB&ABV&AnzB&TW&Zahlencode \textcolor{red}{$\boldsymbol{Grundtext}$} Umschrift $|$"Ubersetzung(en)\\
1.&9.&2946.&27.&11556.&1.&4&386&30\_300\_6\_50 \textcolor{red}{\textcjheb{nw+sl}} LSWN $|$die Zunge\\
2.&10.&2947.&31.&11560.&5.&5&118&8\_20\_40\_10\_40 \textcolor{red}{\textcjheb{mymk.h}} CKMJM $|$der Weisen\\
3.&11.&2948.&36.&11565.&10.&5&431&400\_10\_9\_10\_2 \textcolor{red}{\textcjheb{by.tyt}} TJtJB $|$spricht aus/sie macht gut\\
4.&12.&2949.&41.&11570.&15.&3&474&4\_70\_400 \textcolor{red}{\textcjheb{t`d}} DaT $|$t"uchtiges Wissen/Erkenntnis\\
5.&13.&2950.&44.&11573.&18.&3&96&6\_80\_10 \textcolor{red}{\textcjheb{ypw}} WPJ $|$aber der Mund/und der Mund\\
6.&14.&2951.&47.&11576.&21.&6&170&20\_60\_10\_30\_10\_40 \textcolor{red}{\textcjheb{mylysk}} KsJLJM $|$der Toren\\
7.&15.&2952.&53.&11582.&27.&4&92&10\_2\_10\_70 \textcolor{red}{\textcjheb{`yby}} JBJa $|$sprudelt/(er) macht hervorsprudeln\\
8.&16.&2953.&57.&11586.&31.&4&437&1\_6\_30\_400 \textcolor{red}{\textcjheb{tlw'}} AWLT $|$Narrheit\\
\end{tabular}\medskip \\
Ende des Verses 15.2\\
Verse: 409, Buchstaben: 34, 60, 11589, Totalwerte: 2204, 3534, 841348\\
\\
Die Zunge der Weisen spricht t"uchtiges Wissen aus, aber der Mund der Toren sprudelt Narrheit.\\
\newpage 
{\bf -- 15.3}\\
\medskip \\
\begin{tabular}{rrrrrrrrp{120mm}}
WV&WK&WB&ABK&ABB&ABV&AnzB&TW&Zahlencode \textcolor{red}{$\boldsymbol{Grundtext}$} Umschrift $|$"Ubersetzung(en)\\
1.&17.&2954.&61.&11590.&1.&3&52&2\_20\_30 \textcolor{red}{\textcjheb{lkb}} BKL $|$an jedem\\
2.&18.&2955.&64.&11593.&4.&4&186&40\_100\_6\_40 \textcolor{red}{\textcjheb{mwqm}} MQWM $|$Ort\\
3.&19.&2956.&68.&11597.&8.&4&140&70\_10\_50\_10 \textcolor{red}{\textcjheb{yny`}} aJNJ $|$(sind) die Augen\\
4.&20.&2957.&72.&11601.&12.&4&26&10\_5\_6\_5 \textcolor{red}{\textcjheb{hwhy}} JHWH $|$Jahwe(s)\\
5.&21.&2958.&76.&11605.&16.&4&576&90\_80\_6\_400 \textcolor{red}{\textcjheb{twp.s}} "sPWT $|$schauen auf/beobachtend\\
6.&22.&2959.&80.&11609.&20.&4&320&200\_70\_10\_40 \textcolor{red}{\textcjheb{my`r}} RaJM $|$B"ose\\
7.&23.&2960.&84.&11613.&24.&6&73&6\_9\_6\_2\_10\_40 \textcolor{red}{\textcjheb{mybw.tw}} WtWBJM $|$und (auf) Gute\\
\end{tabular}\medskip \\
Ende des Verses 15.3\\
Verse: 410, Buchstaben: 29, 89, 11618, Totalwerte: 1373, 4907, 842721\\
\\
Die Augen Jahwes sind an jedem Orte, schauen aus auf B"ose und auf Gute.\\
\newpage 
{\bf -- 15.4}\\
\medskip \\
\begin{tabular}{rrrrrrrrp{120mm}}
WV&WK&WB&ABK&ABB&ABV&AnzB&TW&Zahlencode \textcolor{red}{$\boldsymbol{Grundtext}$} Umschrift $|$"Ubersetzung(en)\\
1.&24.&2961.&90.&11619.&1.&4&321&40\_200\_80\_1 \textcolor{red}{\textcjheb{'prm}} MRPA $|$Lindigkeit/Linderung\\
2.&25.&2962.&94.&11623.&5.&4&386&30\_300\_6\_50 \textcolor{red}{\textcjheb{nw+sl}} LSWN $|$der Zunge\\
3.&26.&2963.&98.&11627.&9.&2&160&70\_90 \textcolor{red}{\textcjheb{.s`}} a"s $|$(ist) (ein) Baum\\
4.&27.&2964.&100.&11629.&11.&4&68&8\_10\_10\_40 \textcolor{red}{\textcjheb{myy.h}} CJJM $|$des Lebens\\
5.&28.&2965.&104.&11633.&15.&4&176&6\_60\_30\_80 \textcolor{red}{\textcjheb{plsw}} WsLP $|$aber Verkehrtheit/und Verdrehtheit\\
6.&29.&2966.&108.&11637.&19.&2&7&2\_5 \textcolor{red}{\textcjheb{hb}} BH $|$in ihr/an ihr\\
7.&30.&2967.&110.&11639.&21.&3&502&300\_2\_200 \textcolor{red}{\textcjheb{rb+s}} SBR $|$ist eine Verwundung/(ist) Zusammenbruch\\
8.&31.&2968.&113.&11642.&24.&4&216&2\_200\_6\_8 \textcolor{red}{\textcjheb{.hwrb}} BRWC $|$des Geistes/in Geist\\
\end{tabular}\medskip \\
Ende des Verses 15.4\\
Verse: 411, Buchstaben: 27, 116, 11645, Totalwerte: 1836, 6743, 844557\\
\\
Lindigkeit der Zunge ist ein Baum des Lebens, aber Verkehrtheit in ihr ist eine Verwundung des Geistes.\\
\newpage 
{\bf -- 15.5}\\
\medskip \\
\begin{tabular}{rrrrrrrrp{120mm}}
WV&WK&WB&ABK&ABB&ABV&AnzB&TW&Zahlencode \textcolor{red}{$\boldsymbol{Grundtext}$} Umschrift $|$"Ubersetzung(en)\\
1.&32.&2969.&117.&11646.&1.&4&47&1\_6\_10\_30 \textcolor{red}{\textcjheb{lyw'}} AWJL $|$(ein) Narr\\
2.&33.&2970.&121.&11650.&5.&4&151&10\_50\_1\_90 \textcolor{red}{\textcjheb{.s'ny}} JNA"s $|$verschm"aht/(er) verachtet\\
3.&34.&2971.&125.&11654.&9.&4&306&40\_6\_60\_200 \textcolor{red}{\textcjheb{rswm}} MWsR $|$die Unterweisung/die Z"uchtigung\\
4.&35.&2972.&129.&11658.&13.&4&19&1\_2\_10\_6 \textcolor{red}{\textcjheb{wyb'}} ABJW $|$seines Vaters\\
5.&36.&2973.&133.&11662.&17.&4&546&6\_300\_40\_200 \textcolor{red}{\textcjheb{rm+sw}} WSMR $|$wer aber beachtet/und ein Beachtender\\
6.&37.&2974.&137.&11666.&21.&5&834&400\_6\_20\_8\_400 \textcolor{red}{\textcjheb{t.hkwt}} TWKCT $|$die Zucht/Tadel\\
7.&38.&2975.&142.&11671.&26.&4&320&10\_70\_200\_40 \textcolor{red}{\textcjheb{mr`y}} JaRM $|$ist klug/er handelt klug\\
\end{tabular}\medskip \\
Ende des Verses 15.5\\
Verse: 412, Buchstaben: 29, 145, 11674, Totalwerte: 2223, 8966, 846780\\
\\
Ein Narr verschm"aht die Unterweisung seines Vaters; wer aber die Zucht beachtet, ist klug.\\
\newpage 
{\bf -- 15.6}\\
\medskip \\
\begin{tabular}{rrrrrrrrp{120mm}}
WV&WK&WB&ABK&ABB&ABV&AnzB&TW&Zahlencode \textcolor{red}{$\boldsymbol{Grundtext}$} Umschrift $|$"Ubersetzung(en)\\
1.&39.&2976.&146.&11675.&1.&3&412&2\_10\_400 \textcolor{red}{\textcjheb{tyb}} BJT $|$das Haus/(im) Haus\\
2.&40.&2977.&149.&11678.&4.&4&204&90\_4\_10\_100 \textcolor{red}{\textcjheb{qyd.s}} "sDJQ $|$des Gerechten/(eines) Rechtschaffenen\\
3.&41.&2978.&153.&11682.&8.&3&118&8\_60\_50 \textcolor{red}{\textcjheb{ns.h}} CsN $|$ist eine Schatzkammer/(ist) Verm"ogen\\
4.&42.&2979.&156.&11685.&11.&2&202&200\_2 \textcolor{red}{\textcjheb{br}} RB $|$gro"se/viel\\
5.&43.&2980.&158.&11687.&13.&7&817&6\_2\_400\_2\_6\_1\_400 \textcolor{red}{\textcjheb{t'wbtbw}} WBTBWAT $|$aber im Einkommen/und im Ertrag\\
6.&44.&2981.&165.&11694.&20.&3&570&200\_300\_70 \textcolor{red}{\textcjheb{`+sr}} RSa $|$des Gesetzlosen/(eines) Frevlers\\
7.&45.&2982.&168.&11697.&23.&5&740&50\_70\_20\_200\_400 \textcolor{red}{\textcjheb{trk`n}} NaKRT $|$(ist) Zerr"uttung\\
\end{tabular}\medskip \\
Ende des Verses 15.6\\
Verse: 413, Buchstaben: 27, 172, 11701, Totalwerte: 3063, 12029, 849843\\
\\
Das Haus des Gerechten ist eine gro"se Schatzkammer; aber im Einkommen des Gesetzlosen ist Zerr"uttung.\\
\newpage 
{\bf -- 15.7}\\
\medskip \\
\begin{tabular}{rrrrrrrrp{120mm}}
WV&WK&WB&ABK&ABB&ABV&AnzB&TW&Zahlencode \textcolor{red}{$\boldsymbol{Grundtext}$} Umschrift $|$"Ubersetzung(en)\\
1.&46.&2983.&173.&11702.&1.&4&790&300\_80\_400\_10 \textcolor{red}{\textcjheb{ytp+s}} SPTJ $|$die Lippen\\
2.&47.&2984.&177.&11706.&5.&5&118&8\_20\_40\_10\_40 \textcolor{red}{\textcjheb{mymk.h}} CKMJM $|$der Weisen\\
3.&48.&2985.&182.&11711.&10.&4&223&10\_7\_200\_6 \textcolor{red}{\textcjheb{wrzy}} JZRW $|$(sie) streuen aus\\
4.&49.&2986.&186.&11715.&14.&3&474&4\_70\_400 \textcolor{red}{\textcjheb{t`d}} DaT $|$Erkenntnis\\
5.&50.&2987.&189.&11718.&17.&3&38&6\_30\_2 \textcolor{red}{\textcjheb{blw}} WLB $|$aber das Herz/und das Herz\\
6.&51.&2988.&192.&11721.&20.&6&170&20\_60\_10\_30\_10\_40 \textcolor{red}{\textcjheb{mylysk}} KsJLJM $|$der Toren\\
7.&52.&2989.&198.&11727.&26.&2&31&30\_1 \textcolor{red}{\textcjheb{'l}} LA $|$nicht\\
8.&53.&2990.&200.&11729.&28.&2&70&20\_50 \textcolor{red}{\textcjheb{nk}} KN $|$also/richtig (ist)\\
\end{tabular}\medskip \\
Ende des Verses 15.7\\
Verse: 414, Buchstaben: 29, 201, 11730, Totalwerte: 1914, 13943, 851757\\
\\
Die Lippen der Weisen streuen Erkenntnis aus, aber nicht also das Herz der Toren.\\
\newpage 
{\bf -- 15.8}\\
\medskip \\
\begin{tabular}{rrrrrrrrp{120mm}}
WV&WK&WB&ABK&ABB&ABV&AnzB&TW&Zahlencode \textcolor{red}{$\boldsymbol{Grundtext}$} Umschrift $|$"Ubersetzung(en)\\
1.&54.&2991.&202.&11731.&1.&3&17&7\_2\_8 \textcolor{red}{\textcjheb{.hbz}} ZBC $|$das Opfer/(ein) Schlachtopfer\\
2.&55.&2992.&205.&11734.&4.&5&620&200\_300\_70\_10\_40 \textcolor{red}{\textcjheb{my`+sr}} RSaJM $|$der Gesetzlosen/(der) Frevler\\
3.&56.&2993.&210.&11739.&9.&5&878&400\_6\_70\_2\_400 \textcolor{red}{\textcjheb{tb`wt}} TWaBT $|$(ist ein) Gr"auel\\
4.&57.&2994.&215.&11744.&14.&4&26&10\_5\_6\_5 \textcolor{red}{\textcjheb{hwhy}} JHWH $|$Jahwe\\
5.&58.&2995.&219.&11748.&18.&5&916&6\_400\_80\_30\_400 \textcolor{red}{\textcjheb{tlptw}} WTPLT $|$aber das Gebet/und ein Gebet\\
6.&59.&2996.&224.&11753.&23.&5&560&10\_300\_200\_10\_40 \textcolor{red}{\textcjheb{myr+sy}} JSRJM $|$der Aufrichtigen/der Geraden\\
7.&60.&2997.&229.&11758.&28.&5&352&200\_90\_6\_50\_6 \textcolor{red}{\textcjheb{wnw.sr}} R"sWNW $|$(findet) sein Wohlgefallen\\
\end{tabular}\medskip \\
Ende des Verses 15.8\\
Verse: 415, Buchstaben: 32, 233, 11762, Totalwerte: 3369, 17312, 855126\\
\\
Das Opfer der Gesetzlosen ist Jahwe ein Greuel, aber das Gebet der Aufrichtigen sein Wohlgefallen.\\
\newpage 
{\bf -- 15.9}\\
\medskip \\
\begin{tabular}{rrrrrrrrp{120mm}}
WV&WK&WB&ABK&ABB&ABV&AnzB&TW&Zahlencode \textcolor{red}{$\boldsymbol{Grundtext}$} Umschrift $|$"Ubersetzung(en)\\
1.&61.&2998.&234.&11763.&1.&5&878&400\_6\_70\_2\_400 \textcolor{red}{\textcjheb{tb`wt}} TWaBT $|$(ein) Gr"auel\\
2.&62.&2999.&239.&11768.&6.&4&26&10\_5\_6\_5 \textcolor{red}{\textcjheb{hwhy}} JHWH $|$(f"ur) Jahwe\\
3.&63.&3000.&243.&11772.&10.&3&224&4\_200\_20 \textcolor{red}{\textcjheb{krd}} DRK $|$(ist) der Weg\\
4.&64.&3001.&246.&11775.&13.&3&570&200\_300\_70 \textcolor{red}{\textcjheb{`+sr}} RSa $|$des Gesetzlosen/(eines) Frevlers\\
5.&65.&3002.&249.&11778.&16.&5&330&6\_40\_200\_4\_80 \textcolor{red}{\textcjheb{pdrmw}} WMRDP $|$wer aber nachjagt/und einen Nachjagenden\\
6.&66.&3003.&254.&11783.&21.&4&199&90\_4\_100\_5 \textcolor{red}{\textcjheb{hqd.s}} "sDQH $|$(der) Gerechtigkeit\\
7.&67.&3004.&258.&11787.&25.&4&18&10\_1\_5\_2 \textcolor{red}{\textcjheb{bh'y}} JAHB $|$(den) liebt er\\
\end{tabular}\medskip \\
Ende des Verses 15.9\\
Verse: 416, Buchstaben: 28, 261, 11790, Totalwerte: 2245, 19557, 857371\\
\\
Der Weg des Gesetzlosen ist Jahwe ein Greuel; wer aber der Gerechtigkeit nachjagt, den liebt er.\\
\newpage 
{\bf -- 15.10}\\
\medskip \\
\begin{tabular}{rrrrrrrrp{120mm}}
WV&WK&WB&ABK&ABB&ABV&AnzB&TW&Zahlencode \textcolor{red}{$\boldsymbol{Grundtext}$} Umschrift $|$"Ubersetzung(en)\\
1.&68.&3005.&262.&11791.&1.&4&306&40\_6\_60\_200 \textcolor{red}{\textcjheb{rswm}} MWsR $|$Z"uchtigung\\
2.&69.&3006.&266.&11795.&5.&2&270&200\_70 \textcolor{red}{\textcjheb{`r}} Ra $|$schlimme/ist "ubel\\
3.&70.&3007.&268.&11797.&7.&4&109&30\_70\_7\_2 \textcolor{red}{\textcjheb{bz`l}} LaZB $|$wird dem zuteil der verl"asst/f"ur den Verlassenden\\
4.&71.&3008.&272.&11801.&11.&3&209&1\_200\_8 \textcolor{red}{\textcjheb{.hr'}} ARC $|$(den) Pfad\\
5.&72.&3009.&275.&11804.&14.&4&357&300\_6\_50\_1 \textcolor{red}{\textcjheb{'nw+s}} SWNA $|$wer hasst/(ein) Hassender\\
6.&73.&3010.&279.&11808.&18.&5&834&400\_6\_20\_8\_400 \textcolor{red}{\textcjheb{t.hkwt}} TWKCT $|$Zucht/Zurechtweisung\\
7.&74.&3011.&284.&11813.&23.&4&456&10\_40\_6\_400 \textcolor{red}{\textcjheb{twmy}} JMWT $|$(er) wird sterben\\
\end{tabular}\medskip \\
Ende des Verses 15.10\\
Verse: 417, Buchstaben: 26, 287, 11816, Totalwerte: 2541, 22098, 859912\\
\\
Schlimme Z"uchtigung wird dem zuteil, der den Pfad verl"a"st; wer Zucht ha"st, wird sterben.\\
\newpage 
{\bf -- 15.11}\\
\medskip \\
\begin{tabular}{rrrrrrrrp{120mm}}
WV&WK&WB&ABK&ABB&ABV&AnzB&TW&Zahlencode \textcolor{red}{$\boldsymbol{Grundtext}$} Umschrift $|$"Ubersetzung(en)\\
1.&75.&3012.&288.&11817.&1.&4&337&300\_1\_6\_30 \textcolor{red}{\textcjheb{lw'+s}} SAWL $|$Scheol/Totenreich\\
2.&76.&3013.&292.&11821.&5.&6&69&6\_1\_2\_4\_6\_50 \textcolor{red}{\textcjheb{nwdb'w}} WABDWN $|$und Abgrund/und Abaddon\\
3.&77.&3014.&298.&11827.&11.&3&57&50\_3\_4 \textcolor{red}{\textcjheb{dgn}} NGD $|$(sind) vor\\
4.&78.&3015.&301.&11830.&14.&4&26&10\_5\_6\_5 \textcolor{red}{\textcjheb{hwhy}} JHWH $|$Jahwe\\
5.&79.&3016.&305.&11834.&18.&2&81&1\_80 \textcolor{red}{\textcjheb{p'}} AP $|$wie viel\\
6.&80.&3017.&307.&11836.&20.&2&30&20\_10 \textcolor{red}{\textcjheb{yk}} KJ $|$mehr\\
7.&81.&3018.&309.&11838.&22.&4&438&30\_2\_6\_400 \textcolor{red}{\textcjheb{twbl}} LBWT $|$(die) Herzen\\
8.&82.&3019.&313.&11842.&26.&3&62&2\_50\_10 \textcolor{red}{\textcjheb{ynb}} BNJ $|$der Kinder/der S"ohne\\
9.&83.&3020.&316.&11845.&29.&3&45&1\_4\_40 \textcolor{red}{\textcjheb{md'}} ADM $|$(des) Menschen\\
\end{tabular}\medskip \\
Ende des Verses 15.11\\
Verse: 418, Buchstaben: 31, 318, 11847, Totalwerte: 1145, 23243, 861057\\
\\
Scheol und Abgrund sind vor Jahwe, wieviel mehr die Herzen der Menschenkinder!\\
\newpage 
{\bf -- 15.12}\\
\medskip \\
\begin{tabular}{rrrrrrrrp{120mm}}
WV&WK&WB&ABK&ABB&ABV&AnzB&TW&Zahlencode \textcolor{red}{$\boldsymbol{Grundtext}$} Umschrift $|$"Ubersetzung(en)\\
1.&84.&3021.&319.&11848.&1.&2&31&30\_1 \textcolor{red}{\textcjheb{'l}} LA $|$nicht\\
2.&85.&3022.&321.&11850.&3.&4&18&10\_1\_5\_2 \textcolor{red}{\textcjheb{bh'y}} JAHB $|$(er (=es)) liebt (es)\\
3.&86.&3023.&325.&11854.&7.&2&120&30\_90 \textcolor{red}{\textcjheb{.sl}} L"s $|$der Sp"otter\\
4.&87.&3024.&327.&11856.&9.&4&39&5\_6\_20\_8 \textcolor{red}{\textcjheb{.hkwh}} HWKC $|$dass man zurechtweise/(ein) Zurechtweisen\\
5.&88.&3025.&331.&11860.&13.&2&36&30\_6 \textcolor{red}{\textcjheb{wl}} LW $|$ihn\\
6.&89.&3026.&333.&11862.&15.&2&31&1\_30 \textcolor{red}{\textcjheb{l'}} AL $|$zu\\
7.&90.&3027.&335.&11864.&17.&5&118&8\_20\_40\_10\_40 \textcolor{red}{\textcjheb{mymk.h}} CKMJM $|$den Weisen\\
8.&91.&3028.&340.&11869.&22.&2&31&30\_1 \textcolor{red}{\textcjheb{'l}} LA $|$nicht\\
9.&92.&3029.&342.&11871.&24.&3&60&10\_30\_20 \textcolor{red}{\textcjheb{kly}} JLK $|$er geht\\
\end{tabular}\medskip \\
Ende des Verses 15.12\\
Verse: 419, Buchstaben: 26, 344, 11873, Totalwerte: 484, 23727, 861541\\
\\
Der Sp"otter liebt es nicht, da"s man ihn zurechtweise; zu den Weisen geht er nicht.\\
\newpage 
{\bf -- 15.13}\\
\medskip \\
\begin{tabular}{rrrrrrrrp{120mm}}
WV&WK&WB&ABK&ABB&ABV&AnzB&TW&Zahlencode \textcolor{red}{$\boldsymbol{Grundtext}$} Umschrift $|$"Ubersetzung(en)\\
1.&93.&3030.&345.&11874.&1.&2&32&30\_2 \textcolor{red}{\textcjheb{bl}} LB $|$(ein) Herz\\
2.&94.&3031.&347.&11876.&3.&3&348&300\_40\_8 \textcolor{red}{\textcjheb{.hm+s}} SMC $|$frohes/fr"ohliches\\
3.&95.&3032.&350.&11879.&6.&4&31&10\_10\_9\_2 \textcolor{red}{\textcjheb{b.tyy}} JJtB $|$(er (=es)) erheitert\\
4.&96.&3033.&354.&11883.&10.&4&180&80\_50\_10\_40 \textcolor{red}{\textcjheb{mynp}} PNJM $|$das Antlitz\\
5.&97.&3034.&358.&11887.&14.&6&570&6\_2\_70\_90\_2\_400 \textcolor{red}{\textcjheb{tb.s`bw}} WBa"sBT $|$aber bei Kummer/und durch Kummer\\
6.&98.&3035.&364.&11893.&20.&2&32&30\_2 \textcolor{red}{\textcjheb{bl}} LB $|$des Herzens\\
7.&99.&3036.&366.&11895.&22.&3&214&200\_6\_8 \textcolor{red}{\textcjheb{.hwr}} RWC $|$der Geist\\
8.&100.&3037.&369.&11898.&25.&4&76&50\_20\_1\_5 \textcolor{red}{\textcjheb{h'kn}} NKAH $|$ist zerschlagen/ist niedergeschlagen\\
\end{tabular}\medskip \\
Ende des Verses 15.13\\
Verse: 420, Buchstaben: 28, 372, 11901, Totalwerte: 1483, 25210, 863024\\
\\
Ein frohes Herz erheitert das Antlitz; aber bei Kummer des Herzens ist der Geist zerschlagen.\\
\newpage 
{\bf -- 15.14}\\
\medskip \\
\begin{tabular}{rrrrrrrrp{120mm}}
WV&WK&WB&ABK&ABB&ABV&AnzB&TW&Zahlencode \textcolor{red}{$\boldsymbol{Grundtext}$} Umschrift $|$"Ubersetzung(en)\\
1.&101.&3038.&373.&11902.&1.&2&32&30\_2 \textcolor{red}{\textcjheb{bl}} LB $|$(ein) Herz\\
2.&102.&3039.&375.&11904.&3.&4&108&50\_2\_6\_50 \textcolor{red}{\textcjheb{nwbn}} NBWN $|$des Verst"andigen/verst"andiges\\
3.&103.&3040.&379.&11908.&7.&4&412&10\_2\_100\_300 \textcolor{red}{\textcjheb{+sqby}} JBQS $|$(er (=es)) sucht\\
4.&104.&3041.&383.&11912.&11.&3&474&4\_70\_400 \textcolor{red}{\textcjheb{t`d}} DaT $|$Erkenntnis\\
5.&105.&3042.&386.&11915.&14.&4&146&6\_80\_50\_10 \textcolor{red}{\textcjheb{ynpw}} WPNJ $|$aber der Mund/und der Mund\\
6.&106.&3043.&390.&11919.&18.&6&170&20\_60\_10\_30\_10\_40 \textcolor{red}{\textcjheb{mylysk}} KsJLJM $|$der Toren\\
7.&107.&3044.&396.&11925.&24.&4&285&10\_200\_70\_5 \textcolor{red}{\textcjheb{h`ry}} JRaH $|$(er) weidet (sich an)\\
8.&108.&3045.&400.&11929.&28.&4&437&1\_6\_30\_400 \textcolor{red}{\textcjheb{tlw'}} AWLT $|$Narrheit\\
\end{tabular}\medskip \\
Ende des Verses 15.14\\
Verse: 421, Buchstaben: 31, 403, 11932, Totalwerte: 2064, 27274, 865088\\
\\
Des Verst"andigen Herz sucht Erkenntnis, aber der Mund der Toren weidet sich an Narrheit.\\
\newpage 
{\bf -- 15.15}\\
\medskip \\
\begin{tabular}{rrrrrrrrp{120mm}}
WV&WK&WB&ABK&ABB&ABV&AnzB&TW&Zahlencode \textcolor{red}{$\boldsymbol{Grundtext}$} Umschrift $|$"Ubersetzung(en)\\
1.&109.&3046.&404.&11933.&1.&2&50&20\_30 \textcolor{red}{\textcjheb{lk}} KL $|$alle\\
2.&110.&3047.&406.&11935.&3.&3&60&10\_40\_10 \textcolor{red}{\textcjheb{ymy}} JMJ $|$Tage\\
3.&111.&3048.&409.&11938.&6.&3&130&70\_50\_10 \textcolor{red}{\textcjheb{yn`}} aNJ $|$des Elenden/(eines) Elenden\\
4.&112.&3049.&412.&11941.&9.&4&320&200\_70\_10\_40 \textcolor{red}{\textcjheb{my`r}} RaJM $|$sind b"ose/(sind) schlimm\\
5.&113.&3050.&416.&11945.&13.&4&23&6\_9\_6\_2 \textcolor{red}{\textcjheb{bw.tw}} WtWB $|$aber ein fr"ohliches/und ein heiteres\\
6.&114.&3051.&420.&11949.&17.&2&32&30\_2 \textcolor{red}{\textcjheb{bl}} LB $|$Herz\\
7.&115.&3052.&422.&11951.&19.&4&745&40\_300\_400\_5 \textcolor{red}{\textcjheb{ht+sm}} MSTH $|$(ist) (ein) Festmahl\\
8.&116.&3053.&426.&11955.&23.&4&454&400\_40\_10\_4 \textcolor{red}{\textcjheb{dymt}} TMJD $|$(be)st"andiges\\
\end{tabular}\medskip \\
Ende des Verses 15.15\\
Verse: 422, Buchstaben: 26, 429, 11958, Totalwerte: 1814, 29088, 866902\\
\\
Alle Tage des Elenden sind b"ose, aber ein fr"ohliches Herz ist ein best"andiges Festmahl.\\
\newpage 
{\bf -- 15.16}\\
\medskip \\
\begin{tabular}{rrrrrrrrp{120mm}}
WV&WK&WB&ABK&ABB&ABV&AnzB&TW&Zahlencode \textcolor{red}{$\boldsymbol{Grundtext}$} Umschrift $|$"Ubersetzung(en)\\
1.&117.&3054.&430.&11959.&1.&3&17&9\_6\_2 \textcolor{red}{\textcjheb{bw.t}} tWB $|$besser/gut (ist)\\
2.&118.&3055.&433.&11962.&4.&3&119&40\_70\_9 \textcolor{red}{\textcjheb{.t`m}} Mat $|$(nur) (ein) wenig\\
3.&119.&3056.&436.&11965.&7.&5&613&2\_10\_200\_1\_400 \textcolor{red}{\textcjheb{t'ryb}} BJRAT $|$mit der Furcht/(an) Furcht\\
4.&120.&3057.&441.&11970.&12.&4&26&10\_5\_6\_5 \textcolor{red}{\textcjheb{hwhy}} JHWH $|$(vor) Jahwe(s)\\
5.&121.&3058.&445.&11974.&16.&5&337&40\_1\_6\_90\_200 \textcolor{red}{\textcjheb{r.sw'm}} MAW"sR $|$(mehr) als ein Schatz\\
6.&122.&3059.&450.&11979.&21.&2&202&200\_2 \textcolor{red}{\textcjheb{br}} RB $|$gro"ser\\
7.&123.&3060.&452.&11981.&23.&6&102&6\_40\_5\_6\_40\_5 \textcolor{red}{\textcjheb{hmwhmw}} WMHWMH $|$und Unruhe/und Beunruhigung\\
8.&124.&3061.&458.&11987.&29.&2&8&2\_6 \textcolor{red}{\textcjheb{wb}} BW $|$dabei\\
\end{tabular}\medskip \\
Ende des Verses 15.16\\
Verse: 423, Buchstaben: 30, 459, 11988, Totalwerte: 1424, 30512, 868326\\
\\
Besser wenig mit der Furcht Jahwes, als ein gro"ser Schatz und Unruhe dabei.\\
\newpage 
{\bf -- 15.17}\\
\medskip \\
\begin{tabular}{rrrrrrrrp{120mm}}
WV&WK&WB&ABK&ABB&ABV&AnzB&TW&Zahlencode \textcolor{red}{$\boldsymbol{Grundtext}$} Umschrift $|$"Ubersetzung(en)\\
1.&125.&3062.&460.&11989.&1.&3&17&9\_6\_2 \textcolor{red}{\textcjheb{bw.t}} tWB $|$besser/gut (ist)\\
2.&126.&3063.&463.&11992.&4.&4&609&1\_200\_8\_400 \textcolor{red}{\textcjheb{t.hr'}} ARCT $|$ein Gericht/eine Portion\\
3.&127.&3064.&467.&11996.&8.&3&310&10\_200\_100 \textcolor{red}{\textcjheb{qry}} JRQ $|$Gem"use\\
4.&128.&3065.&470.&11999.&11.&5&19&6\_1\_5\_2\_5 \textcolor{red}{\textcjheb{hbh'w}} WAHBH $|$und Liebe\\
5.&129.&3066.&475.&12004.&16.&2&340&300\_40 \textcolor{red}{\textcjheb{m+s}} SM $|$dabei/dort\\
6.&130.&3067.&477.&12006.&18.&4&546&40\_300\_6\_200 \textcolor{red}{\textcjheb{rw+sm}} MSWR $|$(als) ein Ochs\\
7.&131.&3068.&481.&12010.&22.&4&69&1\_2\_6\_60 \textcolor{red}{\textcjheb{swb'}} ABWs $|$gem"astet(er)\\
8.&132.&3069.&485.&12014.&26.&5&362&6\_300\_50\_1\_5 \textcolor{red}{\textcjheb{h'n+sw}} WSNAH $|$und Hass\\
9.&133.&3070.&490.&12019.&31.&2&8&2\_6 \textcolor{red}{\textcjheb{wb}} BW $|$dabei\\
\end{tabular}\medskip \\
Ende des Verses 15.17\\
Verse: 424, Buchstaben: 32, 491, 12020, Totalwerte: 2280, 32792, 870606\\
\\
Besser ein Gericht Gem"use und Liebe dabei, als ein gem"asteter Ochs und Ha"s dabei.\\
\newpage 
{\bf -- 15.18}\\
\medskip \\
\begin{tabular}{rrrrrrrrp{120mm}}
WV&WK&WB&ABK&ABB&ABV&AnzB&TW&Zahlencode \textcolor{red}{$\boldsymbol{Grundtext}$} Umschrift $|$"Ubersetzung(en)\\
1.&134.&3071.&492.&12021.&1.&3&311&1\_10\_300 \textcolor{red}{\textcjheb{+sy'}} AJS $|$(ein) Mann\\
2.&135.&3072.&495.&12024.&4.&3&53&8\_40\_5 \textcolor{red}{\textcjheb{hm.h}} CMH $|$zorniger/(der) Glut\\
3.&136.&3073.&498.&12027.&7.&4&218&10\_3\_200\_5 \textcolor{red}{\textcjheb{hrgy}} JGRH $|$(er) erregt\\
4.&137.&3074.&502.&12031.&11.&4&100&40\_4\_6\_50 \textcolor{red}{\textcjheb{nwdm}} MDWN $|$Zank\\
5.&138.&3075.&506.&12035.&15.&4&227&6\_1\_200\_20 \textcolor{red}{\textcjheb{kr'w}} WARK $|$aber ein Lang-/und lang\\
6.&139.&3076.&510.&12039.&19.&4&131&1\_80\_10\_40 \textcolor{red}{\textcjheb{myp'}} APJM $|$m"utiger/an Nasenl"ochern\\
7.&140.&3077.&514.&12043.&23.&5&429&10\_300\_100\_10\_9 \textcolor{red}{\textcjheb{.tyq+sy}} JSQJt $|$(er) beschwichtigt\\
8.&141.&3078.&519.&12048.&28.&3&212&200\_10\_2 \textcolor{red}{\textcjheb{byr}} RJB $|$den Streit/(einen) Streit\\
\end{tabular}\medskip \\
Ende des Verses 15.18\\
Verse: 425, Buchstaben: 30, 521, 12050, Totalwerte: 1681, 34473, 872287\\
\\
Ein zorniger Mann erregt Zank, aber ein Langm"utiger beschwichtigt den Streit.\\
\newpage 
{\bf -- 15.19}\\
\medskip \\
\begin{tabular}{rrrrrrrrp{120mm}}
WV&WK&WB&ABK&ABB&ABV&AnzB&TW&Zahlencode \textcolor{red}{$\boldsymbol{Grundtext}$} Umschrift $|$"Ubersetzung(en)\\
1.&142.&3079.&522.&12051.&1.&3&224&4\_200\_20 \textcolor{red}{\textcjheb{krd}} DRK $|$der Weg\\
2.&143.&3080.&525.&12054.&4.&3&190&70\_90\_30 \textcolor{red}{\textcjheb{l.s`}} a"sL $|$(des) Faulen\\
3.&144.&3081.&528.&12057.&7.&5&780&20\_40\_300\_20\_400 \textcolor{red}{\textcjheb{tk+smk}} KMSKT $|$ist wie eine Dornhecke/(ist) wie die Dornhecke\\
4.&145.&3082.&533.&12062.&12.&3&112&8\_4\_100 \textcolor{red}{\textcjheb{qd.h}} CDQ $|$/eines Strauchs\\
5.&146.&3083.&536.&12065.&15.&4&215&6\_1\_200\_8 \textcolor{red}{\textcjheb{.hr'w}} WARC $|$aber der Pfad/und der Pfad\\
6.&147.&3084.&540.&12069.&19.&5&560&10\_300\_200\_10\_40 \textcolor{red}{\textcjheb{myr+sy}} JSRJM $|$der Aufrichtigen/der Geraden\\
7.&148.&3085.&545.&12074.&24.&4&125&60\_30\_30\_5 \textcolor{red}{\textcjheb{hlls}} sLLH $|$(ist) gebahnt\\
\end{tabular}\medskip \\
Ende des Verses 15.19\\
Verse: 426, Buchstaben: 27, 548, 12077, Totalwerte: 2206, 36679, 874493\\
\\
Der Weg des Faulen ist wie eine Dornhecke, aber der Pfad der Aufrichtigen ist gebahnt.\\
\newpage 
{\bf -- 15.20}\\
\medskip \\
\begin{tabular}{rrrrrrrrp{120mm}}
WV&WK&WB&ABK&ABB&ABV&AnzB&TW&Zahlencode \textcolor{red}{$\boldsymbol{Grundtext}$} Umschrift $|$"Ubersetzung(en)\\
1.&149.&3086.&549.&12078.&1.&2&52&2\_50 \textcolor{red}{\textcjheb{nb}} BN $|$(ein) Sohn\\
2.&150.&3087.&551.&12080.&3.&3&68&8\_20\_40 \textcolor{red}{\textcjheb{mk.h}} CKM $|$weiser\\
3.&151.&3088.&554.&12083.&6.&4&358&10\_300\_40\_8 \textcolor{red}{\textcjheb{.hm+sy}} JSMC $|$(er) erfreut\\
4.&152.&3089.&558.&12087.&10.&2&3&1\_2 \textcolor{red}{\textcjheb{b'}} AB $|$den Vater\\
5.&153.&3090.&560.&12089.&12.&5&126&6\_20\_60\_10\_30 \textcolor{red}{\textcjheb{lyskw}} WKsJL $|$aber ein t"orichter/und ein t"orichter\\
6.&154.&3091.&565.&12094.&17.&3&45&1\_4\_40 \textcolor{red}{\textcjheb{md'}} ADM $|$Mensch\\
7.&155.&3092.&568.&12097.&20.&4&20&2\_6\_7\_5 \textcolor{red}{\textcjheb{hzwb}} BWZH $|$verachtet/ist verachtend\\
8.&156.&3093.&572.&12101.&24.&3&47&1\_40\_6 \textcolor{red}{\textcjheb{wm'}} AMW $|$seine Mutter\\
\end{tabular}\medskip \\
Ende des Verses 15.20\\
Verse: 427, Buchstaben: 26, 574, 12103, Totalwerte: 719, 37398, 875212\\
\\
Ein weiser Sohn erfreut den Vater, aber ein t"orichter Mensch verachtet seine Mutter.\\
\newpage 
{\bf -- 15.21}\\
\medskip \\
\begin{tabular}{rrrrrrrrp{120mm}}
WV&WK&WB&ABK&ABB&ABV&AnzB&TW&Zahlencode \textcolor{red}{$\boldsymbol{Grundtext}$} Umschrift $|$"Ubersetzung(en)\\
1.&157.&3094.&575.&12104.&1.&4&437&1\_6\_30\_400 \textcolor{red}{\textcjheb{tlw'}} AWLT $|$(die) Narrheit\\
2.&158.&3095.&579.&12108.&5.&4&353&300\_40\_8\_5 \textcolor{red}{\textcjheb{h.hm+s}} SMCH $|$(ist) Freude\\
3.&159.&3096.&583.&12112.&9.&4&298&30\_8\_60\_200 \textcolor{red}{\textcjheb{rs.hl}} LCsR $|$dem Un-/dem Ermangelnden\\
4.&160.&3097.&587.&12116.&13.&2&32&30\_2 \textcolor{red}{\textcjheb{bl}} LB $|$verst"andigen/Herz (=Verstand)\\
5.&161.&3098.&589.&12118.&15.&4&317&6\_1\_10\_300 \textcolor{red}{\textcjheb{+sy'w}} WAJS $|$aber ein Mann/und (ein) Mann\\
6.&162.&3099.&593.&12122.&19.&5&463&400\_2\_6\_50\_5 \textcolor{red}{\textcjheb{hnwbt}} TBWNH $|$verst"andiger/(mit) Einsicht\\
7.&163.&3100.&598.&12127.&24.&4&520&10\_10\_300\_200 \textcolor{red}{\textcjheb{r+syy}} JJSR $|$wandelt geradeaus/er geht geradeaus\\
8.&164.&3101.&602.&12131.&28.&3&450&30\_20\_400 \textcolor{red}{\textcjheb{tkl}} LKT $|$/beim Gehen\\
\end{tabular}\medskip \\
Ende des Verses 15.21\\
Verse: 428, Buchstaben: 30, 604, 12133, Totalwerte: 2870, 40268, 878082\\
\\
Die Narrheit ist dem Unverst"andigen Freude, aber ein verst"andiger Mann wandelt geradeaus.\\
\newpage 
{\bf -- 15.22}\\
\medskip \\
\begin{tabular}{rrrrrrrrp{120mm}}
WV&WK&WB&ABK&ABB&ABV&AnzB&TW&Zahlencode \textcolor{red}{$\boldsymbol{Grundtext}$} Umschrift $|$"Ubersetzung(en)\\
1.&165.&3102.&605.&12134.&1.&3&285&5\_80\_200 \textcolor{red}{\textcjheb{rph}} HPR $|$(es) scheitern/(ein) Scheitern\\
2.&166.&3103.&608.&12137.&4.&6&756&40\_8\_300\_2\_6\_400 \textcolor{red}{\textcjheb{twb+s.hm}} MCSBWT $|$(von) Pl"ane(n)\\
3.&167.&3104.&614.&12143.&10.&4&63&2\_1\_10\_50 \textcolor{red}{\textcjheb{ny'b}} BAJN $|$wo nicht ist/ohne\\
4.&168.&3105.&618.&12147.&14.&3&70&60\_6\_4 \textcolor{red}{\textcjheb{dws}} sWD $|$Besprechung/Beratschlagung\\
5.&169.&3106.&621.&12150.&17.&4&210&6\_2\_200\_2 \textcolor{red}{\textcjheb{brbw}} WBRB $|$aber durch viele/und mit einer Vielheit\\
6.&170.&3107.&625.&12154.&21.&6&226&10\_6\_70\_90\_10\_40 \textcolor{red}{\textcjheb{my.s`wy}} JWa"sJM $|$(von) Ratgeber(n)\\
7.&171.&3108.&631.&12160.&27.&4&546&400\_100\_6\_40 \textcolor{red}{\textcjheb{mwqt}} TQWM $|$kommen sie zustande/du kannst bestehen\\
\end{tabular}\medskip \\
Ende des Verses 15.22\\
Verse: 429, Buchstaben: 30, 634, 12163, Totalwerte: 2156, 42424, 880238\\
\\
Pl"ane scheitern, wo keine Besprechung ist; aber durch viele Ratgeber kommen sie zustande.\\
\newpage 
{\bf -- 15.23}\\
\medskip \\
\begin{tabular}{rrrrrrrrp{120mm}}
WV&WK&WB&ABK&ABB&ABV&AnzB&TW&Zahlencode \textcolor{red}{$\boldsymbol{Grundtext}$} Umschrift $|$"Ubersetzung(en)\\
1.&172.&3109.&635.&12164.&1.&4&353&300\_40\_8\_5 \textcolor{red}{\textcjheb{h.hm+s}} SMCH $|$Freude\\
2.&173.&3110.&639.&12168.&5.&4&341&30\_1\_10\_300 \textcolor{red}{\textcjheb{+sy'l}} LAJS $|$hat ein Mann\\
3.&174.&3111.&643.&12172.&9.&5&167&2\_40\_70\_50\_5 \textcolor{red}{\textcjheb{hn`mb}} BMaNH $|$an der Antwort/an einer Antwort\\
4.&175.&3112.&648.&12177.&14.&3&96&80\_10\_6 \textcolor{red}{\textcjheb{wyp}} PJW $|$seines Mundes\\
5.&176.&3113.&651.&12180.&17.&4&212&6\_4\_2\_200 \textcolor{red}{\textcjheb{rbdw}} WDBR $|$und ein Wort\\
6.&177.&3114.&655.&12184.&21.&4&478&2\_70\_400\_6 \textcolor{red}{\textcjheb{wt`b}} BaTW $|$zu seiner Zeit\\
7.&178.&3115.&659.&12188.&25.&2&45&40\_5 \textcolor{red}{\textcjheb{hm}} MH $|$wie\\
8.&179.&3116.&661.&12190.&27.&3&17&9\_6\_2 \textcolor{red}{\textcjheb{bw.t}} tWB $|$gut (ist es)\\
\end{tabular}\medskip \\
Ende des Verses 15.23\\
Verse: 430, Buchstaben: 29, 663, 12192, Totalwerte: 1709, 44133, 881947\\
\\
Ein Mann hat Freude an der Antwort seines Mundes; und ein Wort zu seiner Zeit, wie gut!\\
\newpage 
{\bf -- 15.24}\\
\medskip \\
\begin{tabular}{rrrrrrrrp{120mm}}
WV&WK&WB&ABK&ABB&ABV&AnzB&TW&Zahlencode \textcolor{red}{$\boldsymbol{Grundtext}$} Umschrift $|$"Ubersetzung(en)\\
1.&180.&3117.&664.&12193.&1.&3&209&1\_200\_8 \textcolor{red}{\textcjheb{.hr'}} ARC $|$der Weg/(der) Pfad\\
2.&181.&3118.&667.&12196.&4.&4&68&8\_10\_10\_40 \textcolor{red}{\textcjheb{myy.h}} CJJM $|$des Lebens\\
3.&182.&3119.&671.&12200.&8.&5&175&30\_40\_70\_30\_5 \textcolor{red}{\textcjheb{hl`ml}} LMaLH $|$ist aufw"arts/(f"uhrt) nach oben\\
4.&183.&3120.&676.&12205.&13.&6&430&30\_40\_300\_20\_10\_30 \textcolor{red}{\textcjheb{lyk+sml}} LMSKJL $|$f"ur den Einsichtigen/den Verst"andigen\\
5.&184.&3121.&682.&12211.&19.&4&190&30\_40\_70\_50 \textcolor{red}{\textcjheb{n`ml}} LMaN $|$damit\\
6.&185.&3122.&686.&12215.&23.&3&266&60\_6\_200 \textcolor{red}{\textcjheb{rws}} sWR $|$er entgehe/ein Abweichen\\
7.&186.&3123.&689.&12218.&26.&5&377&40\_300\_1\_6\_30 \textcolor{red}{\textcjheb{lw'+sm}} MSAWL $|$dem Scheol/sei von dem Totenreich\\
8.&187.&3124.&694.&12223.&31.&3&54&40\_9\_5 \textcolor{red}{\textcjheb{h.tm}} MtH $|$(dr)unten\\
\end{tabular}\medskip \\
Ende des Verses 15.24\\
Verse: 431, Buchstaben: 33, 696, 12225, Totalwerte: 1769, 45902, 883716\\
\\
Der Weg des Lebens ist f"ur den Einsichtigen aufw"arts, damit er dem Scheol unten entgehe.\\
\newpage 
{\bf -- 15.25}\\
\medskip \\
\begin{tabular}{rrrrrrrrp{120mm}}
WV&WK&WB&ABK&ABB&ABV&AnzB&TW&Zahlencode \textcolor{red}{$\boldsymbol{Grundtext}$} Umschrift $|$"Ubersetzung(en)\\
1.&188.&3125.&697.&12226.&1.&3&412&2\_10\_400 \textcolor{red}{\textcjheb{tyb}} BJT $|$das Haus\\
2.&189.&3126.&700.&12229.&4.&4&54&3\_1\_10\_40 \textcolor{red}{\textcjheb{my'g}} GAJM $|$der Hoff"artigen/(der) Hochm"utigen\\
3.&190.&3127.&704.&12233.&8.&3&78&10\_60\_8 \textcolor{red}{\textcjheb{.hsy}} JsC $|$rei"st nieder/er (=es) rei"st ein\\
4.&191.&3128.&707.&12236.&11.&4&26&10\_5\_6\_5 \textcolor{red}{\textcjheb{hwhy}} JHWH $|$Jahwe\\
5.&192.&3129.&711.&12240.&15.&4&108&6\_10\_90\_2 \textcolor{red}{\textcjheb{b.syw}} WJ"sB $|$aber er stellt fest/und er macht fest\\
6.&193.&3130.&715.&12244.&19.&4&41&3\_2\_6\_30 \textcolor{red}{\textcjheb{lwbg}} GBWL $|$(die) Grenze/das Gebiet\\
7.&194.&3131.&719.&12248.&23.&5&126&1\_30\_40\_50\_5 \textcolor{red}{\textcjheb{hnml'}} ALMNH $|$der Witwe/(einer) Witwe\\
\end{tabular}\medskip \\
Ende des Verses 15.25\\
Verse: 432, Buchstaben: 27, 723, 12252, Totalwerte: 845, 46747, 884561\\
\\
Das Haus der Hoff"artigen rei"st Jahwe nieder, aber der Witwe Grenze stellt er fest.\\
\newpage 
{\bf -- 15.26}\\
\medskip \\
\begin{tabular}{rrrrrrrrp{120mm}}
WV&WK&WB&ABK&ABB&ABV&AnzB&TW&Zahlencode \textcolor{red}{$\boldsymbol{Grundtext}$} Umschrift $|$"Ubersetzung(en)\\
1.&195.&3132.&724.&12253.&1.&5&878&400\_6\_70\_2\_400 \textcolor{red}{\textcjheb{tb`wt}} TWaBT $|$(ein) Gr"auel\\
2.&196.&3133.&729.&12258.&6.&4&26&10\_5\_6\_5 \textcolor{red}{\textcjheb{hwhy}} JHWH $|$(f"ur) Jahwe\\
3.&197.&3134.&733.&12262.&10.&6&756&40\_8\_300\_2\_6\_400 \textcolor{red}{\textcjheb{twb+s.hm}} MCSBWT $|$sind Anschl"age/(sind) die Gedanken\\
4.&198.&3135.&739.&12268.&16.&2&270&200\_70 \textcolor{red}{\textcjheb{`r}} Ra $|$b"ose/(des) B"osen\\
5.&199.&3136.&741.&12270.&18.&6&270&6\_9\_5\_200\_10\_40 \textcolor{red}{\textcjheb{myrh.tw}} WtHRJM $|$aber rein sind/und rein (sind)\\
6.&200.&3137.&747.&12276.&24.&4&251&1\_40\_200\_10 \textcolor{red}{\textcjheb{yrm'}} AMRJ $|$(die) Worte\\
7.&201.&3138.&751.&12280.&28.&3&160&50\_70\_40 \textcolor{red}{\textcjheb{m`n}} NaM $|$huldvolle/von Freundlichkeit\\
\end{tabular}\medskip \\
Ende des Verses 15.26\\
Verse: 433, Buchstaben: 30, 753, 12282, Totalwerte: 2611, 49358, 887172\\
\\
B"ose Anschl"age sind Jahwe ein Greuel, aber huldvolle Worte sind rein.\\
\newpage 
{\bf -- 15.27}\\
\medskip \\
\begin{tabular}{rrrrrrrrp{120mm}}
WV&WK&WB&ABK&ABB&ABV&AnzB&TW&Zahlencode \textcolor{red}{$\boldsymbol{Grundtext}$} Umschrift $|$"Ubersetzung(en)\\
1.&202.&3139.&754.&12283.&1.&3&290&70\_20\_200 \textcolor{red}{\textcjheb{rk`}} aKR $|$(es) verst"ort/verwirrend (ist)\\
2.&203.&3140.&757.&12286.&4.&4&418&2\_10\_400\_6 \textcolor{red}{\textcjheb{wtyb}} BJTW $|$sein Haus\\
3.&204.&3141.&761.&12290.&8.&4&168&2\_6\_90\_70 \textcolor{red}{\textcjheb{`.swb}} BW"sa $|$wer fr"ont/ein Abschneidender\\
4.&205.&3142.&765.&12294.&12.&3&162&2\_90\_70 \textcolor{red}{\textcjheb{`.sb}} B"sa $|$der Habsucht/Gewinn\\
5.&206.&3143.&768.&12297.&15.&5&363&6\_300\_6\_50\_1 \textcolor{red}{\textcjheb{'nw+sw}} WSWNA $|$wer aber hasst/und ein Hassender\\
6.&207.&3144.&773.&12302.&20.&4&890&40\_400\_50\_400 \textcolor{red}{\textcjheb{tntm}} MTNT $|$Geschenke/Bestechung\\
7.&208.&3145.&777.&12306.&24.&4&33&10\_8\_10\_5 \textcolor{red}{\textcjheb{hy.hy}} JCJH $|$(er) wird leben\\
\end{tabular}\medskip \\
Ende des Verses 15.27\\
Verse: 434, Buchstaben: 27, 780, 12309, Totalwerte: 2324, 51682, 889496\\
\\
Wer der Habsucht fr"ont, verst"ort sein Haus; wer aber Geschenke ha"st, wird leben.\\
\newpage 
{\bf -- 15.28}\\
\medskip \\
\begin{tabular}{rrrrrrrrp{120mm}}
WV&WK&WB&ABK&ABB&ABV&AnzB&TW&Zahlencode \textcolor{red}{$\boldsymbol{Grundtext}$} Umschrift $|$"Ubersetzung(en)\\
1.&209.&3146.&781.&12310.&1.&2&32&30\_2 \textcolor{red}{\textcjheb{bl}} LB $|$das Herz\\
2.&210.&3147.&783.&12312.&3.&4&204&90\_4\_10\_100 \textcolor{red}{\textcjheb{qyd.s}} "sDJQ $|$des Gerechten/(eines) Rechtschaffenen\\
3.&211.&3148.&787.&12316.&7.&4&23&10\_5\_3\_5 \textcolor{red}{\textcjheb{hghy}} JHGH $|$"uberlegt/er (=es) sinnt nach\\
4.&212.&3149.&791.&12320.&11.&5&556&30\_70\_50\_6\_400 \textcolor{red}{\textcjheb{twn`l}} LaNWT $|$um zu antworten\\
5.&213.&3150.&796.&12325.&16.&3&96&6\_80\_10 \textcolor{red}{\textcjheb{ypw}} WPJ $|$aber der Mund/und der Mund\\
6.&214.&3151.&799.&12328.&19.&5&620&200\_300\_70\_10\_40 \textcolor{red}{\textcjheb{my`+sr}} RSaJM $|$der Gesetzlosen/(der) Frevler\\
7.&215.&3152.&804.&12333.&24.&4&92&10\_2\_10\_70 \textcolor{red}{\textcjheb{`yby}} JBJa $|$sprudelt/er macht hervorsprudeln\\
8.&216.&3153.&808.&12337.&28.&4&676&200\_70\_6\_400 \textcolor{red}{\textcjheb{tw`r}} RaWT $|$Bosheiten\\
\end{tabular}\medskip \\
Ende des Verses 15.28\\
Verse: 435, Buchstaben: 31, 811, 12340, Totalwerte: 2299, 53981, 891795\\
\\
Das Herz des Gerechten "uberlegt, um zu antworten; aber der Mund der Gesetzlosen sprudelt Bosheiten.\\
\newpage 
{\bf -- 15.29}\\
\medskip \\
\begin{tabular}{rrrrrrrrp{120mm}}
WV&WK&WB&ABK&ABB&ABV&AnzB&TW&Zahlencode \textcolor{red}{$\boldsymbol{Grundtext}$} Umschrift $|$"Ubersetzung(en)\\
1.&217.&3154.&812.&12341.&1.&4&314&200\_8\_6\_100 \textcolor{red}{\textcjheb{qw.hr}} RCWQ $|$fern\\
2.&218.&3155.&816.&12345.&5.&4&26&10\_5\_6\_5 \textcolor{red}{\textcjheb{hwhy}} JHWH $|$(ist) Jahwe\\
3.&219.&3156.&820.&12349.&9.&6&660&40\_200\_300\_70\_10\_40 \textcolor{red}{\textcjheb{my`+srm}} MRSaJM $|$von den Gesetzlosen/den Gottlosen\\
4.&220.&3157.&826.&12355.&15.&5&916&6\_400\_80\_30\_400 \textcolor{red}{\textcjheb{tlptw}} WTPLT $|$aber das Gebet/und ein Gebet\\
5.&221.&3158.&831.&12360.&20.&6&254&90\_4\_10\_100\_10\_40 \textcolor{red}{\textcjheb{myqyd.s}} "sDJQJM $|$der Gerechten/der Rechtschaffenen\\
6.&222.&3159.&837.&12366.&26.&4&420&10\_300\_40\_70 \textcolor{red}{\textcjheb{`m+sy}} JSMa $|$er (er)h"ort\\
\end{tabular}\medskip \\
Ende des Verses 15.29\\
Verse: 436, Buchstaben: 29, 840, 12369, Totalwerte: 2590, 56571, 894385\\
\\
Jahwe ist fern von den Gesetzlosen, aber das Gebet der Gerechten h"ort er.\\
\newpage 
{\bf -- 15.30}\\
\medskip \\
\begin{tabular}{rrrrrrrrp{120mm}}
WV&WK&WB&ABK&ABB&ABV&AnzB&TW&Zahlencode \textcolor{red}{$\boldsymbol{Grundtext}$} Umschrift $|$"Ubersetzung(en)\\
1.&223.&3160.&841.&12370.&1.&4&247&40\_1\_6\_200 \textcolor{red}{\textcjheb{rw'm}} MAWR $|$das Leuchten\\
2.&224.&3161.&845.&12374.&5.&5&180&70\_10\_50\_10\_40 \textcolor{red}{\textcjheb{myny`}} aJNJM $|$der Augen/(zweier) Augen\\
3.&225.&3162.&850.&12379.&10.&4&358&10\_300\_40\_8 \textcolor{red}{\textcjheb{.hm+sy}} JSMC $|$(er (=es)) (er)freut\\
4.&226.&3163.&854.&12383.&14.&2&32&30\_2 \textcolor{red}{\textcjheb{bl}} LB $|$das Herz\\
5.&227.&3164.&856.&12385.&16.&5&421&300\_40\_6\_70\_5 \textcolor{red}{\textcjheb{h`wm+s}} SMWaH $|$eine Nachricht/eine Kunde\\
6.&228.&3165.&861.&12390.&21.&4&22&9\_6\_2\_5 \textcolor{red}{\textcjheb{hbw.t}} tWBH $|$gute\\
7.&229.&3166.&865.&12394.&25.&4&754&400\_4\_300\_50 \textcolor{red}{\textcjheb{n+sdt}} TDSN $|$labt/(sie) erquickt\\
8.&230.&3167.&869.&12398.&29.&3&200&70\_90\_40 \textcolor{red}{\textcjheb{m.s`}} a"sM $|$(das) Gebein\\
\end{tabular}\medskip \\
Ende des Verses 15.30\\
Verse: 437, Buchstaben: 31, 871, 12400, Totalwerte: 2214, 58785, 896599\\
\\
Das Leuchten der Augen erfreut das Herz; eine gute Nachricht labt das Gebein.\\
\newpage 
{\bf -- 15.31}\\
\medskip \\
\begin{tabular}{rrrrrrrrp{120mm}}
WV&WK&WB&ABK&ABB&ABV&AnzB&TW&Zahlencode \textcolor{red}{$\boldsymbol{Grundtext}$} Umschrift $|$"Ubersetzung(en)\\
1.&231.&3168.&872.&12401.&1.&3&58&1\_7\_50 \textcolor{red}{\textcjheb{nz'}} AZN $|$(ein) Ohr\\
2.&232.&3169.&875.&12404.&4.&4&810&300\_40\_70\_400 \textcolor{red}{\textcjheb{t`m+s}} SMaT $|$das h"ort auf/h"orend (auf eine)\\
3.&233.&3170.&879.&12408.&8.&5&834&400\_6\_20\_8\_400 \textcolor{red}{\textcjheb{t.hkwt}} TWKCT $|$die Zucht/Zurechtweisung\\
4.&234.&3171.&884.&12413.&13.&4&68&8\_10\_10\_40 \textcolor{red}{\textcjheb{myy.h}} CJJM $|$zum Leben/des Lebens\\
5.&235.&3172.&888.&12417.&17.&4&304&2\_100\_200\_2 \textcolor{red}{\textcjheb{brqb}} BQRB $|$inmitten (von)\\
6.&236.&3173.&892.&12421.&21.&5&118&8\_20\_40\_10\_40 \textcolor{red}{\textcjheb{mymk.h}} CKMJM $|$(der) Weisen\\
7.&237.&3174.&897.&12426.&26.&4&490&400\_30\_10\_50 \textcolor{red}{\textcjheb{nylt}} TLJN $|$(sie (=es)) wird weilen\\
\end{tabular}\medskip \\
Ende des Verses 15.31\\
Verse: 438, Buchstaben: 29, 900, 12429, Totalwerte: 2682, 61467, 899281\\
\\
Ein Ohr, das auf die Zucht zum Leben h"ort, wird inmitten der Weisen weilen.\\
\newpage 
{\bf -- 15.32}\\
\medskip \\
\begin{tabular}{rrrrrrrrp{120mm}}
WV&WK&WB&ABK&ABB&ABV&AnzB&TW&Zahlencode \textcolor{red}{$\boldsymbol{Grundtext}$} Umschrift $|$"Ubersetzung(en)\\
1.&238.&3175.&901.&12430.&1.&4&356&80\_6\_200\_70 \textcolor{red}{\textcjheb{`rwp}} PWRa $|$wer verwirft\\
2.&239.&3176.&905.&12434.&5.&4&306&40\_6\_60\_200 \textcolor{red}{\textcjheb{rswm}} MWsR $|$Unterweisung/(der) Zucht\\
3.&240.&3177.&909.&12438.&9.&4&107&40\_6\_1\_60 \textcolor{red}{\textcjheb{s'wm}} MWAs $|$verachtet/der missachtet\\
4.&241.&3178.&913.&12442.&13.&4&436&50\_80\_300\_6 \textcolor{red}{\textcjheb{w+spn}} NPSW $|$seine Seele/sich selbst\\
5.&242.&3179.&917.&12446.&17.&5&422&6\_300\_6\_40\_70 \textcolor{red}{\textcjheb{`mw+sw}} WSWMa $|$wer aber h"ort/und ein H"orender (auf eine)\\
6.&243.&3180.&922.&12451.&22.&5&834&400\_6\_20\_8\_400 \textcolor{red}{\textcjheb{t.hkwt}} TWKCT $|$auf Zucht/Zurechtweisung\\
7.&244.&3181.&927.&12456.&27.&4&161&100\_6\_50\_5 \textcolor{red}{\textcjheb{hnwq}} QWNH $|$erwirbt/(ist) erwerbend(er)\\
8.&245.&3182.&931.&12460.&31.&2&32&30\_2 \textcolor{red}{\textcjheb{bl}} LB $|$Herz (=Verstand)\\
\end{tabular}\medskip \\
Ende des Verses 15.32\\
Verse: 439, Buchstaben: 32, 932, 12461, Totalwerte: 2654, 64121, 901935\\
\\
Wer Unterweisung verwirft, verachtet seine Seele; wer aber auf Zucht h"ort, erwirbt Verstand.\\
\newpage 
{\bf -- 15.33}\\
\medskip \\
\begin{tabular}{rrrrrrrrp{120mm}}
WV&WK&WB&ABK&ABB&ABV&AnzB&TW&Zahlencode \textcolor{red}{$\boldsymbol{Grundtext}$} Umschrift $|$"Ubersetzung(en)\\
1.&246.&3183.&933.&12462.&1.&4&611&10\_200\_1\_400 \textcolor{red}{\textcjheb{t'ry}} JRAT $|$(die) Furcht\\
2.&247.&3184.&937.&12466.&5.&4&26&10\_5\_6\_5 \textcolor{red}{\textcjheb{hwhy}} JHWH $|$(vor) Jahwe(s)\\
3.&248.&3185.&941.&12470.&9.&4&306&40\_6\_60\_200 \textcolor{red}{\textcjheb{rswm}} MWsR $|$ist Unterweisung/(ist) Zucht\\
4.&249.&3186.&945.&12474.&13.&4&73&8\_20\_40\_5 \textcolor{red}{\textcjheb{hmk.h}} CKMH $|$(zu) der Weisheit\\
5.&250.&3187.&949.&12478.&17.&5&176&6\_30\_80\_50\_10 \textcolor{red}{\textcjheb{ynplw}} WLPNJ $|$und vor(aus)\\
6.&251.&3188.&954.&12483.&22.&4&32&20\_2\_6\_4 \textcolor{red}{\textcjheb{dwbk}} KBWD $|$der Ehre\\
7.&252.&3189.&958.&12487.&26.&4&131&70\_50\_6\_5 \textcolor{red}{\textcjheb{hwn`}} aNWH $|$geht Demut/(kommt) Demut\\
\end{tabular}\medskip \\
Ende des Verses 15.33\\
Verse: 440, Buchstaben: 29, 961, 12490, Totalwerte: 1355, 65476, 903290\\
\\
Die Furcht Jahwes ist Unterweisung zur Weisheit, und der Ehre geht Demut voraus.\\
\\
{\bf Ende des Kapitels 15}\\
\newpage 
{\bf -- 16.1}\\
\medskip \\
\begin{tabular}{rrrrrrrrp{120mm}}
WV&WK&WB&ABK&ABB&ABV&AnzB&TW&Zahlencode \textcolor{red}{$\boldsymbol{Grundtext}$} Umschrift $|$"Ubersetzung(en)\\
1.&1.&3190.&1.&12491.&1.&4&75&30\_1\_4\_40 \textcolor{red}{\textcjheb{md'l}} LADM $|$des Menschen/beim Menschen\\
2.&2.&3191.&5.&12495.&5.&5&340&40\_70\_200\_20\_10 \textcolor{red}{\textcjheb{ykr`m}} MaRKJ $|$sind die Entw"urfe/(sind) die "Uberlegungen\\
3.&3.&3192.&10.&12500.&10.&2&32&30\_2 \textcolor{red}{\textcjheb{bl}} LB $|$des Herzens\\
4.&4.&3193.&12.&12502.&12.&6&72&6\_40\_10\_5\_6\_5 \textcolor{red}{\textcjheb{hwhymw}} WMJHWH $|$aber von Jahwe/und von Jahwe\\
5.&5.&3194.&18.&12508.&18.&4&165&40\_70\_50\_5 \textcolor{red}{\textcjheb{hn`m}} MaNH $|$(kommt die) Antwort\\
6.&6.&3195.&22.&12512.&22.&4&386&30\_300\_6\_50 \textcolor{red}{\textcjheb{nw+sl}} LSWN $|$der Zunge\\
\end{tabular}\medskip \\
Ende des Verses 16.1\\
Verse: 441, Buchstaben: 25, 25, 12515, Totalwerte: 1070, 1070, 904360\\
\\
Die Entw"urfe des Herzens sind des Menschen, aber die Antwort der Zunge kommt von Jahwe.\\
\newpage 
{\bf -- 16.2}\\
\medskip \\
\begin{tabular}{rrrrrrrrp{120mm}}
WV&WK&WB&ABK&ABB&ABV&AnzB&TW&Zahlencode \textcolor{red}{$\boldsymbol{Grundtext}$} Umschrift $|$"Ubersetzung(en)\\
1.&7.&3196.&26.&12516.&1.&2&50&20\_30 \textcolor{red}{\textcjheb{lk}} KL $|$alle\\
2.&8.&3197.&28.&12518.&3.&4&234&4\_200\_20\_10 \textcolor{red}{\textcjheb{ykrd}} DRKJ $|$Wege\\
3.&9.&3198.&32.&12522.&7.&3&311&1\_10\_300 \textcolor{red}{\textcjheb{+sy'}} AJS $|$eines Mannes/(des) Mannes\\
4.&10.&3199.&35.&12525.&10.&2&27&7\_20 \textcolor{red}{\textcjheb{kz}} ZK $|$(sind) rein\\
5.&11.&3200.&37.&12527.&12.&6&148&2\_70\_10\_50\_10\_6 \textcolor{red}{\textcjheb{wyny`b}} BaJNJW $|$in seinen Augen\\
6.&12.&3201.&43.&12533.&18.&4&476&6\_400\_20\_50 \textcolor{red}{\textcjheb{nktw}} WTKN $|$aber (es) w"agt/und pr"ufend\\
7.&13.&3202.&47.&12537.&22.&5&620&200\_6\_8\_6\_400 \textcolor{red}{\textcjheb{tw.hwr}} RWCWT $|$die Geister\\
8.&14.&3203.&52.&12542.&27.&4&26&10\_5\_6\_5 \textcolor{red}{\textcjheb{hwhy}} JHWH $|$(ist) Jahwe\\
\end{tabular}\medskip \\
Ende des Verses 16.2\\
Verse: 442, Buchstaben: 30, 55, 12545, Totalwerte: 1892, 2962, 906252\\
\\
Alle Wege eines Mannes sind rein in seinen Augen, aber Jahwe w"agt die Geister.\\
\newpage 
{\bf -- 16.3}\\
\medskip \\
\begin{tabular}{rrrrrrrrp{120mm}}
WV&WK&WB&ABK&ABB&ABV&AnzB&TW&Zahlencode \textcolor{red}{$\boldsymbol{Grundtext}$} Umschrift $|$"Ubersetzung(en)\\
1.&15.&3204.&56.&12546.&1.&2&33&3\_30 \textcolor{red}{\textcjheb{lg}} GL $|$befiehl/w"alze\\
2.&16.&3205.&58.&12548.&3.&2&31&1\_30 \textcolor{red}{\textcjheb{l'}} AL $|$/auf\\
3.&17.&3206.&60.&12550.&5.&4&26&10\_5\_6\_5 \textcolor{red}{\textcjheb{hwhy}} JHWH $|$Jahwe\\
4.&18.&3207.&64.&12554.&9.&5&440&40\_70\_300\_10\_20 \textcolor{red}{\textcjheb{ky+s`m}} MaSJK $|$deine Werke\\
5.&19.&3208.&69.&12559.&14.&5&92&6\_10\_20\_50\_6 \textcolor{red}{\textcjheb{wnkyw}} WJKNW $|$und (es) werden zustande kommen/und sie (=es) werden bestehen\\
6.&20.&3209.&74.&12564.&19.&7&780&40\_8\_300\_2\_400\_10\_20 \textcolor{red}{\textcjheb{kytb+s.hm}} MCSBTJK $|$deine Gedanken/deine Planungen\\
\end{tabular}\medskip \\
Ende des Verses 16.3\\
Verse: 443, Buchstaben: 25, 80, 12570, Totalwerte: 1402, 4364, 907654\\
\\
Befiehl Jahwe deine Werke, und deine Gedanken werden zustande kommen.\\
\newpage 
{\bf -- 16.4}\\
\medskip \\
\begin{tabular}{rrrrrrrrp{120mm}}
WV&WK&WB&ABK&ABB&ABV&AnzB&TW&Zahlencode \textcolor{red}{$\boldsymbol{Grundtext}$} Umschrift $|$"Ubersetzung(en)\\
1.&21.&3210.&81.&12571.&1.&2&50&20\_30 \textcolor{red}{\textcjheb{lk}} KL $|$alles\\
2.&22.&3211.&83.&12573.&3.&3&180&80\_70\_30 \textcolor{red}{\textcjheb{l`p}} PaL $|$hat gemacht/er (=es) schuf\\
3.&23.&3212.&86.&12576.&6.&4&26&10\_5\_6\_5 \textcolor{red}{\textcjheb{hwhy}} JHWH $|$Jahwe\\
4.&24.&3213.&90.&12580.&10.&6&201&30\_40\_70\_50\_5\_6 \textcolor{red}{\textcjheb{whn`ml}} LMaNHW $|$zu seiner Absicht/zu seinem Zweck\\
5.&25.&3214.&96.&12586.&16.&3&49&6\_3\_40 \textcolor{red}{\textcjheb{mgw}} WGM $|$und auch\\
6.&26.&3215.&99.&12589.&19.&3&570&200\_300\_70 \textcolor{red}{\textcjheb{`+sr}} RSa $|$den Gesetzlosen/(den) Frevler\\
7.&27.&3216.&102.&12592.&22.&4&86&30\_10\_6\_40 \textcolor{red}{\textcjheb{mwyl}} LJWM $|$f"ur den Tag\\
8.&28.&3217.&106.&12596.&26.&3&275&200\_70\_5 \textcolor{red}{\textcjheb{h`r}} RaH $|$des Ungl"ucks/des Unheils\\
\end{tabular}\medskip \\
Ende des Verses 16.4\\
Verse: 444, Buchstaben: 28, 108, 12598, Totalwerte: 1437, 5801, 909091\\
\\
Jahwe hat alles zu seiner Absicht gemacht, und auch den Gesetzlosen f"ur den Tag des Ungl"ucks.\\
\newpage 
{\bf -- 16.5}\\
\medskip \\
\begin{tabular}{rrrrrrrrp{120mm}}
WV&WK&WB&ABK&ABB&ABV&AnzB&TW&Zahlencode \textcolor{red}{$\boldsymbol{Grundtext}$} Umschrift $|$"Ubersetzung(en)\\
1.&29.&3218.&109.&12599.&1.&5&878&400\_6\_70\_2\_400 \textcolor{red}{\textcjheb{tb`wt}} TWaBT $|$(ein) Gr"auel\\
2.&30.&3219.&114.&12604.&6.&4&26&10\_5\_6\_5 \textcolor{red}{\textcjheb{hwhy}} JHWH $|$(ist) (f"ur) Jahwe\\
3.&31.&3220.&118.&12608.&10.&2&50&20\_30 \textcolor{red}{\textcjheb{lk}} KL $|$jeder\\
4.&32.&3221.&120.&12610.&12.&3&10&3\_2\_5 \textcolor{red}{\textcjheb{hbg}} GBH $|$Hoch-/stolzen\\
5.&33.&3222.&123.&12613.&15.&2&32&30\_2 \textcolor{red}{\textcjheb{bl}} LB $|$m"utige/Herzens\\
6.&34.&3223.&125.&12615.&17.&2&14&10\_4 \textcolor{red}{\textcjheb{dy}} JD $|$die Hand\\
7.&35.&3224.&127.&12617.&19.&3&44&30\_10\_4 \textcolor{red}{\textcjheb{dyl}} LJD $|$darauf/zu Hand\\
8.&36.&3225.&130.&12620.&22.&2&31&30\_1 \textcolor{red}{\textcjheb{'l}} LA $|$nicht\\
9.&37.&3226.&132.&12622.&24.&4&165&10\_50\_100\_5 \textcolor{red}{\textcjheb{hqny}} JNQH $|$wird er f"ur schuldlos gehalten werden/er bleibt ungestraft\\
\end{tabular}\medskip \\
Ende des Verses 16.5\\
Verse: 445, Buchstaben: 27, 135, 12625, Totalwerte: 1250, 7051, 910341\\
\\
Jeder Hochm"utige ist Jahwe ein Greuel; die Hand darauf! Er wird nicht f"ur schuldlos gehalten werden.\\
\newpage 
{\bf -- 16.6}\\
\medskip \\
\begin{tabular}{rrrrrrrrp{120mm}}
WV&WK&WB&ABK&ABB&ABV&AnzB&TW&Zahlencode \textcolor{red}{$\boldsymbol{Grundtext}$} Umschrift $|$"Ubersetzung(en)\\
1.&38.&3227.&136.&12626.&1.&4&74&2\_8\_60\_4 \textcolor{red}{\textcjheb{ds.hb}} BCsD $|$durch G"ute/(durch) Liebe\\
2.&39.&3228.&140.&12630.&5.&4&447&6\_1\_40\_400 \textcolor{red}{\textcjheb{tm'w}} WAMT $|$und Wahrheit/und Treue\\
3.&40.&3229.&144.&12634.&9.&4&310&10\_20\_80\_200 \textcolor{red}{\textcjheb{rpky}} JKPR $|$(er (=es)) wird ges"uhnt\\
4.&41.&3230.&148.&12638.&13.&3&126&70\_6\_50 \textcolor{red}{\textcjheb{nw`}} aWN $|$die Missetat/eine Schuld\\
5.&42.&3231.&151.&12641.&16.&6&619&6\_2\_10\_200\_1\_400 \textcolor{red}{\textcjheb{t'rybw}} WBJRAT $|$und durch die Furcht/und in der Furcht\\
6.&43.&3232.&157.&12647.&22.&4&26&10\_5\_6\_5 \textcolor{red}{\textcjheb{hwhy}} JHWH $|$(vor) Jahwe(s)\\
7.&44.&3233.&161.&12651.&26.&3&266&60\_6\_200 \textcolor{red}{\textcjheb{rws}} sWR $|$weicht man/(ist ein) Weichen\\
8.&45.&3234.&164.&12654.&29.&3&310&40\_200\_70 \textcolor{red}{\textcjheb{`rm}} MRa $|$vom B"osen\\
\end{tabular}\medskip \\
Ende des Verses 16.6\\
Verse: 446, Buchstaben: 31, 166, 12656, Totalwerte: 2178, 9229, 912519\\
\\
Durch G"ute und Wahrheit wird die Missetat ges"uhnt, und durch die Furcht Jahwes weicht man vom B"osen.\\
\newpage 
{\bf -- 16.7}\\
\medskip \\
\begin{tabular}{rrrrrrrrp{120mm}}
WV&WK&WB&ABK&ABB&ABV&AnzB&TW&Zahlencode \textcolor{red}{$\boldsymbol{Grundtext}$} Umschrift $|$"Ubersetzung(en)\\
1.&46.&3235.&167.&12657.&1.&5&698&2\_200\_90\_6\_400 \textcolor{red}{\textcjheb{tw.srb}} BR"sWT $|$wenn wohlgefallen/wenn Wohlgefallen hat\\
2.&47.&3236.&172.&12662.&6.&4&26&10\_5\_6\_5 \textcolor{red}{\textcjheb{hwhy}} JHWH $|$Jahwe\\
3.&48.&3237.&176.&12666.&10.&4&234&4\_200\_20\_10 \textcolor{red}{\textcjheb{ykrd}} DRKJ $|$(an den) Wege(n)\\
4.&49.&3238.&180.&12670.&14.&3&311&1\_10\_300 \textcolor{red}{\textcjheb{+sy'}} AJS $|$(eines) Mannes\\
5.&50.&3239.&183.&12673.&17.&2&43&3\_40 \textcolor{red}{\textcjheb{mg}} GM $|$so l"asst er selbst/auch\\
6.&51.&3240.&185.&12675.&19.&6&35&1\_6\_10\_2\_10\_6 \textcolor{red}{\textcjheb{wybyw'}} AWJBJW $|$seine Feinde\\
7.&52.&3241.&191.&12681.&25.&4&380&10\_300\_30\_40 \textcolor{red}{\textcjheb{ml+sy}} JSLM $|$in Frieden sein/er l"asst Frieden machen\\
8.&53.&3242.&195.&12685.&29.&3&407&1\_400\_6 \textcolor{red}{\textcjheb{wt'}} ATW $|$mit ihm\\
\end{tabular}\medskip \\
Ende des Verses 16.7\\
Verse: 447, Buchstaben: 31, 197, 12687, Totalwerte: 2134, 11363, 914653\\
\\
Wenn eines Mannes Wege Jahwe wohlgefallen, so l"a"st er selbst seine Feinde mit ihm in Frieden sein.\\
\newpage 
{\bf -- 16.8}\\
\medskip \\
\begin{tabular}{rrrrrrrrp{120mm}}
WV&WK&WB&ABK&ABB&ABV&AnzB&TW&Zahlencode \textcolor{red}{$\boldsymbol{Grundtext}$} Umschrift $|$"Ubersetzung(en)\\
1.&54.&3243.&198.&12688.&1.&3&17&9\_6\_2 \textcolor{red}{\textcjheb{bw.t}} tWB $|$besser/gut (ist)\\
2.&55.&3244.&201.&12691.&4.&3&119&40\_70\_9 \textcolor{red}{\textcjheb{.t`m}} Mat $|$(ein) wenig\\
3.&56.&3245.&204.&12694.&7.&5&201&2\_90\_4\_100\_5 \textcolor{red}{\textcjheb{hqd.sb}} B"sDQH $|$mit Gerechtigkeit/durch Gerechtigkeit\\
4.&57.&3246.&209.&12699.&12.&3&242&40\_200\_2 \textcolor{red}{\textcjheb{brm}} MRB $|$als viel/als eine Vielheit\\
5.&58.&3247.&212.&12702.&15.&6&815&400\_2\_6\_1\_6\_400 \textcolor{red}{\textcjheb{tw'wbt}} TBWAWT $|$Einkommen/von Eink"unften\\
6.&59.&3248.&218.&12708.&21.&3&33&2\_30\_1 \textcolor{red}{\textcjheb{'lb}} BLA $|$mit Un-/ohne\\
7.&60.&3249.&221.&12711.&24.&4&429&40\_300\_80\_9 \textcolor{red}{\textcjheb{.tp+sm}} MSPt $|$Rechtlichkeit/Recht\\
\end{tabular}\medskip \\
Ende des Verses 16.8\\
Verse: 448, Buchstaben: 27, 224, 12714, Totalwerte: 1856, 13219, 916509\\
\\
Besser wenig mit Gerechtigkeit, als viel Einkommen mit Unrechtlichkeit.\\
\newpage 
{\bf -- 16.9}\\
\medskip \\
\begin{tabular}{rrrrrrrrp{120mm}}
WV&WK&WB&ABK&ABB&ABV&AnzB&TW&Zahlencode \textcolor{red}{$\boldsymbol{Grundtext}$} Umschrift $|$"Ubersetzung(en)\\
1.&61.&3250.&225.&12715.&1.&2&32&30\_2 \textcolor{red}{\textcjheb{bl}} LB $|$das Herz\\
2.&62.&3251.&227.&12717.&3.&3&45&1\_4\_40 \textcolor{red}{\textcjheb{md'}} ADM $|$des Menschen\\
3.&63.&3252.&230.&12720.&6.&4&320&10\_8\_300\_2 \textcolor{red}{\textcjheb{b+s.hy}} JCSB $|$erdenkt/er (=es) plant\\
4.&64.&3253.&234.&12724.&10.&4&230&4\_200\_20\_6 \textcolor{red}{\textcjheb{wkrd}} DRKW $|$seinen Weg\\
5.&65.&3254.&238.&12728.&14.&5&32&6\_10\_5\_6\_5 \textcolor{red}{\textcjheb{hwhyw}} WJHWH $|$aber Jahwe/und Jahwe\\
6.&66.&3255.&243.&12733.&19.&4&90&10\_20\_10\_50 \textcolor{red}{\textcjheb{nyky}} JKJN $|$(er) lenkt\\
7.&67.&3256.&247.&12737.&23.&4&170&90\_70\_4\_6 \textcolor{red}{\textcjheb{wd`.s}} "saDW $|$seine(n) Schritt(e)\\
\end{tabular}\medskip \\
Ende des Verses 16.9\\
Verse: 449, Buchstaben: 26, 250, 12740, Totalwerte: 919, 14138, 917428\\
\\
Das Herz des Menschen erdenkt seinen Weg, aber Jahwe lenkt seine Schritte.\\
\newpage 
{\bf -- 16.10}\\
\medskip \\
\begin{tabular}{rrrrrrrrp{120mm}}
WV&WK&WB&ABK&ABB&ABV&AnzB&TW&Zahlencode \textcolor{red}{$\boldsymbol{Grundtext}$} Umschrift $|$"Ubersetzung(en)\\
1.&68.&3257.&251.&12741.&1.&3&200&100\_60\_40 \textcolor{red}{\textcjheb{msq}} QsM $|$(ein) Orakel(spruch)\\
2.&69.&3258.&254.&12744.&4.&2&100&70\_30 \textcolor{red}{\textcjheb{l`}} aL $|$ist auf/(liegt) auf\\
3.&70.&3259.&256.&12746.&6.&4&790&300\_80\_400\_10 \textcolor{red}{\textcjheb{ytp+s}} SPTJ $|$den Lippen\\
4.&71.&3260.&260.&12750.&10.&3&90&40\_30\_20 \textcolor{red}{\textcjheb{klm}} MLK $|$des K"onigs\\
5.&72.&3261.&263.&12753.&13.&5&431&2\_40\_300\_80\_9 \textcolor{red}{\textcjheb{.tp+smb}} BMSPt $|$am Recht/beim Rechtsspruch\\
6.&73.&3262.&268.&12758.&18.&2&31&30\_1 \textcolor{red}{\textcjheb{'l}} LA $|$nicht\\
7.&74.&3263.&270.&12760.&20.&4&150&10\_40\_70\_30 \textcolor{red}{\textcjheb{l`my}} JMaL $|$vergeht sich/er (=es) geht fehl\\
8.&75.&3264.&274.&12764.&24.&3&96&80\_10\_6 \textcolor{red}{\textcjheb{wyp}} PJW $|$sein Mund\\
\end{tabular}\medskip \\
Ende des Verses 16.10\\
Verse: 450, Buchstaben: 26, 276, 12766, Totalwerte: 1888, 16026, 919316\\
\\
Ein Orakelspruch ist auf den Lippen des K"onigs: sein Mund vergeht sich nicht am Recht.\\
\newpage 
{\bf -- 16.11}\\
\medskip \\
\begin{tabular}{rrrrrrrrp{120mm}}
WV&WK&WB&ABK&ABB&ABV&AnzB&TW&Zahlencode \textcolor{red}{$\boldsymbol{Grundtext}$} Umschrift $|$"Ubersetzung(en)\\
1.&76.&3265.&277.&12767.&1.&3&170&80\_30\_60 \textcolor{red}{\textcjheb{slp}} PLs $|$Waage\\
2.&77.&3266.&280.&12770.&4.&6&114&6\_40\_1\_7\_50\_10 \textcolor{red}{\textcjheb{ynz'mw}} WMAZNJ $|$und Waagschalen\\
3.&78.&3267.&286.&12776.&10.&4&429&40\_300\_80\_9 \textcolor{red}{\textcjheb{.tp+sm}} MSPt $|$gerechte/des Rechts\\
4.&79.&3268.&290.&12780.&14.&5&56&30\_10\_5\_6\_5 \textcolor{red}{\textcjheb{hwhyl}} LJHWH $|$(sind) (bei) Jahwe(s)\\
5.&80.&3269.&295.&12785.&19.&5&421&40\_70\_300\_5\_6 \textcolor{red}{\textcjheb{wh+s`m}} MaSHW $|$sein Werk\\
6.&81.&3270.&300.&12790.&24.&2&50&20\_30 \textcolor{red}{\textcjheb{lk}} KL $|$(sind) alle\\
7.&82.&3271.&302.&12792.&26.&4&63&1\_2\_50\_10 \textcolor{red}{\textcjheb{ynb'}} ABNJ $|$(Gewichts)Steine\\
8.&83.&3272.&306.&12796.&30.&3&90&20\_10\_60 \textcolor{red}{\textcjheb{syk}} KJs $|$des Beutels/(im) Beutel\\
\end{tabular}\medskip \\
Ende des Verses 16.11\\
Verse: 451, Buchstaben: 32, 308, 12798, Totalwerte: 1393, 17419, 920709\\
\\
Gerechte Waage und Waagschalen sind Jahwes; sein Werk sind alle Gewichtsteine des Beutels.\\
\newpage 
{\bf -- 16.12}\\
\medskip \\
\begin{tabular}{rrrrrrrrp{120mm}}
WV&WK&WB&ABK&ABB&ABV&AnzB&TW&Zahlencode \textcolor{red}{$\boldsymbol{Grundtext}$} Umschrift $|$"Ubersetzung(en)\\
1.&84.&3273.&309.&12799.&1.&5&878&400\_6\_70\_2\_400 \textcolor{red}{\textcjheb{tb`wt}} TWaBT $|$(ein) Gr"auel/(ein) Abscheu\\
2.&85.&3274.&314.&12804.&6.&5&140&40\_30\_20\_10\_40 \textcolor{red}{\textcjheb{myklm}} MLKJM $|$der K"onige/(f"ur) K"onige\\
3.&86.&3275.&319.&12809.&11.&4&776&70\_300\_6\_400 \textcolor{red}{\textcjheb{tw+s`}} aSWT $|$ist zu tun/(sei ein) Tun\\
4.&87.&3276.&323.&12813.&15.&3&570&200\_300\_70 \textcolor{red}{\textcjheb{`+sr}} RSa $|$Gesetzlosigkeit/Frevel\\
5.&88.&3277.&326.&12816.&18.&2&30&20\_10 \textcolor{red}{\textcjheb{yk}} KJ $|$denn\\
6.&89.&3278.&328.&12818.&20.&5&201&2\_90\_4\_100\_5 \textcolor{red}{\textcjheb{hqd.sb}} B"sDQH $|$durch Gerechtigkeit\\
7.&90.&3279.&333.&12823.&25.&4&86&10\_20\_6\_50 \textcolor{red}{\textcjheb{nwky}} JKWN $|$steht fest/er (=es) hat Bestand\\
8.&91.&3280.&337.&12827.&29.&3&81&20\_60\_1 \textcolor{red}{\textcjheb{'sk}} KsA $|$ein Thron/der Thron\\
\end{tabular}\medskip \\
Ende des Verses 16.12\\
Verse: 452, Buchstaben: 31, 339, 12829, Totalwerte: 2762, 20181, 923471\\
\\
Der K"onige Greuel ist, Gesetzlosigkeit zu tun; denn durch Gerechtigkeit steht ein Thron fest.\\
\newpage 
{\bf -- 16.13}\\
\medskip \\
\begin{tabular}{rrrrrrrrp{120mm}}
WV&WK&WB&ABK&ABB&ABV&AnzB&TW&Zahlencode \textcolor{red}{$\boldsymbol{Grundtext}$} Umschrift $|$"Ubersetzung(en)\\
1.&92.&3281.&340.&12830.&1.&4&346&200\_90\_6\_50 \textcolor{red}{\textcjheb{nw.sr}} R"sWN $|$Wohlgefallen\\
2.&93.&3282.&344.&12834.&5.&5&140&40\_30\_20\_10\_40 \textcolor{red}{\textcjheb{myklm}} MLKJM $|$der K"onige/(f"ur) K"onige\\
3.&94.&3283.&349.&12839.&10.&4&790&300\_80\_400\_10 \textcolor{red}{\textcjheb{ytp+s}} SPTJ $|$(sind) Lippen\\
4.&95.&3284.&353.&12843.&14.&3&194&90\_4\_100 \textcolor{red}{\textcjheb{qd.s}} "sDQ $|$gerechte/der Gerechtigkeit\\
5.&96.&3285.&356.&12846.&17.&4&212&6\_4\_2\_200 \textcolor{red}{\textcjheb{rbdw}} WDBR $|$und wer redet/und (einen) Redenden\\
6.&97.&3286.&360.&12850.&21.&5&560&10\_300\_200\_10\_40 \textcolor{red}{\textcjheb{myr+sy}} JSRJM $|$Aufrichtiges/rechte (Dinge)\\
7.&98.&3287.&365.&12855.&26.&4&18&10\_1\_5\_2 \textcolor{red}{\textcjheb{bh'y}} JAHB $|$(den) er liebt\\
\end{tabular}\medskip \\
Ende des Verses 16.13\\
Verse: 453, Buchstaben: 29, 368, 12858, Totalwerte: 2260, 22441, 925731\\
\\
Der K"onige Wohlgefallen sind gerechte Lippen; und wer Aufrichtiges redet, den liebt er.\\
\newpage 
{\bf -- 16.14}\\
\medskip \\
\begin{tabular}{rrrrrrrrp{120mm}}
WV&WK&WB&ABK&ABB&ABV&AnzB&TW&Zahlencode \textcolor{red}{$\boldsymbol{Grundtext}$} Umschrift $|$"Ubersetzung(en)\\
1.&99.&3288.&369.&12859.&1.&3&448&8\_40\_400 \textcolor{red}{\textcjheb{tm.h}} CMT $|$Grimm/Zornesglut\\
2.&100.&3289.&372.&12862.&4.&3&90&40\_30\_20 \textcolor{red}{\textcjheb{klm}} MLK $|$des K"onigs/(f"ur den) K"onig\\
3.&101.&3290.&375.&12865.&7.&5&101&40\_30\_1\_20\_10 \textcolor{red}{\textcjheb{yk'lm}} MLAKJ $|$gleicht Boten/(sind) Boten\\
4.&102.&3291.&380.&12870.&12.&3&446&40\_6\_400 \textcolor{red}{\textcjheb{twm}} MWT $|$(des) Todes\\
5.&103.&3292.&383.&12873.&15.&4&317&6\_1\_10\_300 \textcolor{red}{\textcjheb{+sy'w}} WAJS $|$aber ein Mann/und (ein) Mann\\
6.&104.&3293.&387.&12877.&19.&3&68&8\_20\_40 \textcolor{red}{\textcjheb{mk.h}} CKM $|$weiser\\
7.&105.&3294.&390.&12880.&22.&6&365&10\_20\_80\_200\_50\_5 \textcolor{red}{\textcjheb{hnrpky}} JKPRNH $|$vers"ohnt ihn/(er) bes"anftigt sie\\
\end{tabular}\medskip \\
Ende des Verses 16.14\\
Verse: 454, Buchstaben: 27, 395, 12885, Totalwerte: 1835, 24276, 927566\\
\\
Des K"onigs Grimm gleicht Todesboten; aber ein weiser Mann vers"ohnt ihn.\\
\newpage 
{\bf -- 16.15}\\
\medskip \\
\begin{tabular}{rrrrrrrrp{120mm}}
WV&WK&WB&ABK&ABB&ABV&AnzB&TW&Zahlencode \textcolor{red}{$\boldsymbol{Grundtext}$} Umschrift $|$"Ubersetzung(en)\\
1.&106.&3295.&396.&12886.&1.&4&209&2\_1\_6\_200 \textcolor{red}{\textcjheb{rw'b}} BAWR $|$im Licht/im Leuchten\\
2.&107.&3296.&400.&12890.&5.&3&140&80\_50\_10 \textcolor{red}{\textcjheb{ynp}} PNJ $|$des Angesichts/des Antlitzes\\
3.&108.&3297.&403.&12893.&8.&3&90&40\_30\_20 \textcolor{red}{\textcjheb{klm}} MLK $|$des K"onigs\\
4.&109.&3298.&406.&12896.&11.&4&68&8\_10\_10\_40 \textcolor{red}{\textcjheb{myy.h}} CJJM $|$(ist) Leben\\
5.&110.&3299.&410.&12900.&15.&6&358&6\_200\_90\_6\_50\_6 \textcolor{red}{\textcjheb{wnw.srw}} WR"sWNW $|$und sein Wohlgefallen\\
6.&111.&3300.&416.&12906.&21.&3&92&20\_70\_2 \textcolor{red}{\textcjheb{b`k}} KaB $|$ist wie eine Wolke/(ist) wie Gew"olk\\
7.&112.&3301.&419.&12909.&24.&5&476&40\_30\_100\_6\_300 \textcolor{red}{\textcjheb{+swqlm}} MLQWS $|$des Sp"atregens\\
\end{tabular}\medskip \\
Ende des Verses 16.15\\
Verse: 455, Buchstaben: 28, 423, 12913, Totalwerte: 1433, 25709, 928999\\
\\
Im Lichte des Angesichts des K"onigs ist Leben, und sein Wohlgefallen ist wie eine Wolke des Sp"atregens.\\
\newpage 
{\bf -- 16.16}\\
\medskip \\
\begin{tabular}{rrrrrrrrp{120mm}}
WV&WK&WB&ABK&ABB&ABV&AnzB&TW&Zahlencode \textcolor{red}{$\boldsymbol{Grundtext}$} Umschrift $|$"Ubersetzung(en)\\
1.&113.&3302.&424.&12914.&1.&3&155&100\_50\_5 \textcolor{red}{\textcjheb{hnq}} QNH $|$(ein) Erwerben\\
2.&114.&3303.&427.&12917.&4.&4&73&8\_20\_40\_5 \textcolor{red}{\textcjheb{hmk.h}} CKMH $|$Weisheit\\
3.&115.&3304.&431.&12921.&8.&2&45&40\_5 \textcolor{red}{\textcjheb{hm}} MH $|$wieviel/das\\
4.&116.&3305.&433.&12923.&10.&3&17&9\_6\_2 \textcolor{red}{\textcjheb{bw.t}} tWB $|$besser ist es/gut (ist) (es)\\
5.&117.&3306.&436.&12926.&13.&5&344&40\_8\_200\_6\_90 \textcolor{red}{\textcjheb{.swr.hm}} MCRW"s $|$(mehr) als (feines) Gold\\
6.&118.&3307.&441.&12931.&18.&5&562&6\_100\_50\_6\_400 \textcolor{red}{\textcjheb{twnqw}} WQNWT $|$und (ein) Erwerben\\
7.&119.&3308.&446.&12936.&23.&4&67&2\_10\_50\_5 \textcolor{red}{\textcjheb{hnyb}} BJNH $|$Verstand/Einsicht\\
8.&120.&3309.&450.&12940.&27.&4&260&50\_2\_8\_200 \textcolor{red}{\textcjheb{r.hbn}} NBCR $|$wieviel vorz"uglicher/er (=es) wird vorgezogen\\
9.&121.&3310.&454.&12944.&31.&4&200&40\_20\_60\_80 \textcolor{red}{\textcjheb{pskm}} MKsP $|$(mehr) als Silber\\
\end{tabular}\medskip \\
Ende des Verses 16.16\\
Verse: 456, Buchstaben: 34, 457, 12947, Totalwerte: 1723, 27432, 930722\\
\\
Weisheit erwerben, wieviel besser ist es als feines Gold, und Verstand erwerben, wieviel vorz"uglicher als Silber!\\
\newpage 
{\bf -- 16.17}\\
\medskip \\
\begin{tabular}{rrrrrrrrp{120mm}}
WV&WK&WB&ABK&ABB&ABV&AnzB&TW&Zahlencode \textcolor{red}{$\boldsymbol{Grundtext}$} Umschrift $|$"Ubersetzung(en)\\
1.&122.&3311.&458.&12948.&1.&4&530&40\_60\_30\_400 \textcolor{red}{\textcjheb{tlsm}} MsLT $|$(die) Stra"se\\
2.&123.&3312.&462.&12952.&5.&5&560&10\_300\_200\_10\_40 \textcolor{red}{\textcjheb{myr+sy}} JSRJM $|$der Aufrichtigen/der Geraden\\
3.&124.&3313.&467.&12957.&10.&3&266&60\_6\_200 \textcolor{red}{\textcjheb{rws}} sWR $|$ist weichen/(ist ein) Ausweichen\\
4.&125.&3314.&470.&12960.&13.&3&310&40\_200\_70 \textcolor{red}{\textcjheb{`rm}} MRa $|$vom B"osen/vor B"osem\\
5.&126.&3315.&473.&12963.&16.&3&540&300\_40\_200 \textcolor{red}{\textcjheb{rm+s}} SMR $|$wer bewahrt/(ein) Bewahrender\\
6.&127.&3316.&476.&12966.&19.&4&436&50\_80\_300\_6 \textcolor{red}{\textcjheb{w+spn}} NPSW $|$seine Seele\\
7.&128.&3317.&480.&12970.&23.&3&340&50\_90\_200 \textcolor{red}{\textcjheb{r.sn}} N"sR $|$beh"utet/(ist ein) Achthabender\\
8.&129.&3318.&483.&12973.&26.&4&230&4\_200\_20\_6 \textcolor{red}{\textcjheb{wkrd}} DRKW $|$seinen Weg/auf seinem Weg\\
\end{tabular}\medskip \\
Ende des Verses 16.17\\
Verse: 457, Buchstaben: 29, 486, 12976, Totalwerte: 3212, 30644, 933934\\
\\
Der Aufrichtigen Stra"se ist: vom B"osen weichen; wer seinen Weg bewahrt, beh"utet seine Seele.\\
\newpage 
{\bf -- 16.18}\\
\medskip \\
\begin{tabular}{rrrrrrrrp{120mm}}
WV&WK&WB&ABK&ABB&ABV&AnzB&TW&Zahlencode \textcolor{red}{$\boldsymbol{Grundtext}$} Umschrift $|$"Ubersetzung(en)\\
1.&130.&3319.&487.&12977.&1.&4&170&30\_80\_50\_10 \textcolor{red}{\textcjheb{ynpl}} LPNJ $|$vor(aus)\\
2.&131.&3320.&491.&12981.&5.&3&502&300\_2\_200 \textcolor{red}{\textcjheb{rb+s}} SBR $|$dem Sturz/(dem) Zusammenbruch\\
3.&132.&3321.&494.&12984.&8.&4&60&3\_1\_6\_50 \textcolor{red}{\textcjheb{nw'g}} GAWN $|$geht Hoffahrt/(kommt) der Stolz\\
4.&133.&3322.&498.&12988.&12.&5&176&6\_30\_80\_50\_10 \textcolor{red}{\textcjheb{ynplw}} WLPNJ $|$und (vor)\\
5.&134.&3323.&503.&12993.&17.&5&406&20\_300\_30\_6\_50 \textcolor{red}{\textcjheb{nwl+sk}} KSLWN $|$dem Fall\\
6.&135.&3324.&508.&12998.&22.&3&10&3\_2\_5 \textcolor{red}{\textcjheb{hbg}} GBH $|$Hochmut\\
7.&136.&3325.&511.&13001.&25.&3&214&200\_6\_8 \textcolor{red}{\textcjheb{.hwr}} RWC $|$/des Geistes\\
\end{tabular}\medskip \\
Ende des Verses 16.18\\
Verse: 458, Buchstaben: 27, 513, 13003, Totalwerte: 1538, 32182, 935472\\
\\
Hoffart geht dem Sturze, und Hochmut dem Falle voraus.\\
\newpage 
{\bf -- 16.19}\\
\medskip \\
\begin{tabular}{rrrrrrrrp{120mm}}
WV&WK&WB&ABK&ABB&ABV&AnzB&TW&Zahlencode \textcolor{red}{$\boldsymbol{Grundtext}$} Umschrift $|$"Ubersetzung(en)\\
1.&137.&3326.&514.&13004.&1.&3&17&9\_6\_2 \textcolor{red}{\textcjheb{bw.t}} tWB $|$besser/er (=es) ist gut\\
2.&138.&3327.&517.&13007.&4.&3&410&300\_80\_30 \textcolor{red}{\textcjheb{lp+s}} SPL $|$niedrigen/dem"utig\\
3.&139.&3328.&520.&13010.&7.&3&214&200\_6\_8 \textcolor{red}{\textcjheb{.hwr}} RWC $|$(im) Geist(es) (zu) (sein)\\
4.&140.&3329.&523.&13013.&10.&2&401&1\_400 \textcolor{red}{\textcjheb{t'}} AT $|$mit\\
5.&141.&3330.&525.&13015.&12.&5&180&70\_50\_10\_10\_40 \textcolor{red}{\textcjheb{myyn`}} aNJJM $|$(den) Dem"utigen\\
6.&142.&3331.&530.&13020.&17.&4&178&40\_8\_30\_100 \textcolor{red}{\textcjheb{ql.hm}} MCLQ $|$als teilen/mehr als ein Teilen\\
7.&143.&3332.&534.&13024.&21.&3&360&300\_30\_30 \textcolor{red}{\textcjheb{ll+s}} SLL $|$Raub/Beute\\
8.&144.&3333.&537.&13027.&24.&2&401&1\_400 \textcolor{red}{\textcjheb{t'}} AT $|$mit\\
9.&145.&3334.&539.&13029.&26.&4&54&3\_1\_10\_40 \textcolor{red}{\textcjheb{my'g}} GAJM $|$den Hoff"artigen/Stolzen\\
\end{tabular}\medskip \\
Ende des Verses 16.19\\
Verse: 459, Buchstaben: 29, 542, 13032, Totalwerte: 2215, 34397, 937687\\
\\
Besser niedrigen Geistes sein mit den Dem"utigen, als Raub teilen mit den Hoff"artigen.\\
\newpage 
{\bf -- 16.20}\\
\medskip \\
\begin{tabular}{rrrrrrrrp{120mm}}
WV&WK&WB&ABK&ABB&ABV&AnzB&TW&Zahlencode \textcolor{red}{$\boldsymbol{Grundtext}$} Umschrift $|$"Ubersetzung(en)\\
1.&146.&3335.&543.&13033.&1.&5&400&40\_300\_20\_10\_30 \textcolor{red}{\textcjheb{lyk+sm}} MSKJL $|$wer achtet/(ein) einsichtig Machender\\
2.&147.&3336.&548.&13038.&6.&2&100&70\_30 \textcolor{red}{\textcjheb{l`}} aL $|$(in Bezug) auf\\
3.&148.&3337.&550.&13040.&8.&3&206&4\_2\_200 \textcolor{red}{\textcjheb{rbd}} DBR $|$das Wort/(ein) Wort\\
4.&149.&3338.&553.&13043.&11.&4&141&10\_40\_90\_1 \textcolor{red}{\textcjheb{'.smy}} JM"sA $|$wird erlangen/(er) findet\\
5.&150.&3339.&557.&13047.&15.&3&17&9\_6\_2 \textcolor{red}{\textcjheb{bw.t}} tWB $|$Gutes\\
6.&151.&3340.&560.&13050.&18.&5&31&6\_2\_6\_9\_8 \textcolor{red}{\textcjheb{.h.twbw}} WBWtC $|$und wer vertraut/und ein Vertrauender\\
7.&152.&3341.&565.&13055.&23.&5&28&2\_10\_5\_6\_5 \textcolor{red}{\textcjheb{hwhyb}} BJHWH $|$auf Jahwe\\
8.&153.&3342.&570.&13060.&28.&5&517&1\_300\_200\_10\_6 \textcolor{red}{\textcjheb{wyr+s'}} ASRJW $|$ist gl"uckselig/seine Seligkeiten\\
\end{tabular}\medskip \\
Ende des Verses 16.20\\
Verse: 460, Buchstaben: 32, 574, 13064, Totalwerte: 1440, 35837, 939127\\
\\
Wer auf das Wort achtet, wird Gutes erlangen; und wer auf Jahwe vertraut, ist gl"uckselig.\\
\newpage 
{\bf -- 16.21}\\
\medskip \\
\begin{tabular}{rrrrrrrrp{120mm}}
WV&WK&WB&ABK&ABB&ABV&AnzB&TW&Zahlencode \textcolor{red}{$\boldsymbol{Grundtext}$} Umschrift $|$"Ubersetzung(en)\\
1.&154.&3343.&575.&13065.&1.&4&98&30\_8\_20\_40 \textcolor{red}{\textcjheb{mk.hl}} LCKM $|$wer weisen/den Weisen\\
2.&155.&3344.&579.&13069.&5.&2&32&30\_2 \textcolor{red}{\textcjheb{bl}} LB $|$(des) Herzens (ist)\\
3.&156.&3345.&581.&13071.&7.&4&311&10\_100\_200\_1 \textcolor{red}{\textcjheb{'rqy}} JQRA $|$wird genannt/(er) wird gerufen\\
4.&157.&3346.&585.&13075.&11.&4&108&50\_2\_6\_50 \textcolor{red}{\textcjheb{nwbn}} NBWN $|$verst"andig/(einen) verst"andig Seienden\\
5.&158.&3347.&589.&13079.&15.&4&546&6\_40\_400\_100 \textcolor{red}{\textcjheb{qtmw}} WMTQ $|$und (einer mit) S"u"sigkeit\\
6.&159.&3348.&593.&13083.&19.&5&830&300\_80\_400\_10\_40 \textcolor{red}{\textcjheb{mytp+s}} SPTJM $|$der Lippen\\
7.&160.&3349.&598.&13088.&24.&4&160&10\_60\_10\_80 \textcolor{red}{\textcjheb{pysy}} JsJP $|$(er) mehrt\\
8.&161.&3350.&602.&13092.&28.&3&138&30\_100\_8 \textcolor{red}{\textcjheb{.hql}} LQC $|$die Lehre/Belehrung\\
\end{tabular}\medskip \\
Ende des Verses 16.21\\
Verse: 461, Buchstaben: 30, 604, 13094, Totalwerte: 2223, 38060, 941350\\
\\
Wer weisen Herzens ist, wird verst"andig genannt; und S"u"sigkeit der Lippen mehrt die Lehre.\\
\newpage 
{\bf -- 16.22}\\
\medskip \\
\begin{tabular}{rrrrrrrrp{120mm}}
WV&WK&WB&ABK&ABB&ABV&AnzB&TW&Zahlencode \textcolor{red}{$\boldsymbol{Grundtext}$} Umschrift $|$"Ubersetzung(en)\\
1.&162.&3351.&605.&13095.&1.&4&346&40\_100\_6\_200 \textcolor{red}{\textcjheb{rwqm}} MQWR $|$ein Born/eine Quelle\\
2.&163.&3352.&609.&13099.&5.&4&68&8\_10\_10\_40 \textcolor{red}{\textcjheb{myy.h}} CJJM $|$des Lebens\\
3.&164.&3353.&613.&13103.&9.&3&350&300\_20\_30 \textcolor{red}{\textcjheb{lk+s}} SKL $|$(ist) Einsicht/(ist der) Verstand\\
4.&165.&3354.&616.&13106.&12.&5&118&2\_70\_30\_10\_6 \textcolor{red}{\textcjheb{wyl`b}} BaLJW $|$f"ur ihre Besitzer/auf ihm\\
5.&166.&3355.&621.&13111.&17.&5&312&6\_40\_6\_60\_200 \textcolor{red}{\textcjheb{rswmw}} WMWsR $|$aber die Z"uchtigung/und Zurechtweisung\\
6.&167.&3356.&626.&13116.&22.&5&87&1\_6\_30\_10\_40 \textcolor{red}{\textcjheb{mylw'}} AWLJM $|$der Narren\\
7.&168.&3357.&631.&13121.&27.&4&437&1\_6\_30\_400 \textcolor{red}{\textcjheb{tlw'}} AWLT $|$(ist) (die) Narrheit\\
\end{tabular}\medskip \\
Ende des Verses 16.22\\
Verse: 462, Buchstaben: 30, 634, 13124, Totalwerte: 1718, 39778, 943068\\
\\
Einsicht ist f"ur ihre Besitzer ein Born des Lebens, aber die Z"uchtigung der Narren ist die Narrheit.\\
\newpage 
{\bf -- 16.23}\\
\medskip \\
\begin{tabular}{rrrrrrrrp{120mm}}
WV&WK&WB&ABK&ABB&ABV&AnzB&TW&Zahlencode \textcolor{red}{$\boldsymbol{Grundtext}$} Umschrift $|$"Ubersetzung(en)\\
1.&169.&3358.&635.&13125.&1.&2&32&30\_2 \textcolor{red}{\textcjheb{bl}} LB $|$das Herz\\
2.&170.&3359.&637.&13127.&3.&3&68&8\_20\_40 \textcolor{red}{\textcjheb{mk.h}} CKM $|$(des) Weisen\\
3.&171.&3360.&640.&13130.&6.&5&370&10\_300\_20\_10\_30 \textcolor{red}{\textcjheb{lyk+sy}} JSKJL $|$gibt Einsicht/er (=es) macht verst"andig\\
4.&172.&3361.&645.&13135.&11.&4&101&80\_10\_5\_6 \textcolor{red}{\textcjheb{whyp}} PJHW $|$seinen Mund\\
5.&173.&3362.&649.&13139.&15.&3&106&6\_70\_30 \textcolor{red}{\textcjheb{l`w}} WaL $|$und auf\\
6.&174.&3363.&652.&13142.&18.&5&796&300\_80\_400\_10\_6 \textcolor{red}{\textcjheb{wytp+s}} SPTJW $|$seinen Lippen\\
7.&175.&3364.&657.&13147.&23.&4&160&10\_60\_10\_80 \textcolor{red}{\textcjheb{pysy}} JsJP $|$(er) mehrt\\
8.&176.&3365.&661.&13151.&27.&3&138&30\_100\_8 \textcolor{red}{\textcjheb{.hql}} LQC $|$die Lehre/Belehrung\\
\end{tabular}\medskip \\
Ende des Verses 16.23\\
Verse: 463, Buchstaben: 29, 663, 13153, Totalwerte: 1771, 41549, 944839\\
\\
Das Herz des Weisen gibt seinem Munde Einsicht und mehrt auf seinen Lippen die Lehre.\\
\newpage 
{\bf -- 16.24}\\
\medskip \\
\begin{tabular}{rrrrrrrrp{120mm}}
WV&WK&WB&ABK&ABB&ABV&AnzB&TW&Zahlencode \textcolor{red}{$\boldsymbol{Grundtext}$} Umschrift $|$"Ubersetzung(en)\\
1.&177.&3366.&664.&13154.&1.&3&176&90\_6\_80 \textcolor{red}{\textcjheb{pw.s}} "sWP $|$(ein) Wabe\\
2.&178.&3367.&667.&13157.&4.&3&306&4\_2\_300 \textcolor{red}{\textcjheb{+sbd}} DBS $|$(von) Honig\\
3.&179.&3368.&670.&13160.&7.&4&251&1\_40\_200\_10 \textcolor{red}{\textcjheb{yrm'}} AMRJ $|$(sind) (die) Worte\\
4.&180.&3369.&674.&13164.&11.&3&160&50\_70\_40 \textcolor{red}{\textcjheb{m`n}} NaM $|$huldvolle/der Freundlichkeit\\
5.&181.&3370.&677.&13167.&14.&4&546&40\_400\_6\_100 \textcolor{red}{\textcjheb{qwtm}} MTWQ $|$S"u"ses/s"u"s\\
6.&182.&3371.&681.&13171.&18.&4&460&30\_50\_80\_300 \textcolor{red}{\textcjheb{+spnl}} LNPS $|$f"ur die Seele\\
7.&183.&3372.&685.&13175.&22.&5&327&6\_40\_200\_80\_1 \textcolor{red}{\textcjheb{'prmw}} WMRPA $|$und Gesundheit/und Erquickung\\
8.&184.&3373.&690.&13180.&27.&4&230&30\_70\_90\_40 \textcolor{red}{\textcjheb{m.s`l}} La"sM $|$f"ur das Gebein\\
\end{tabular}\medskip \\
Ende des Verses 16.24\\
Verse: 464, Buchstaben: 30, 693, 13183, Totalwerte: 2456, 44005, 947295\\
\\
Huldvolle Worte sind eine Honigwabe, S"u"ses f"ur die Seele und Gesundheit f"ur das Gebein.\\
\newpage 
{\bf -- 16.25}\\
\medskip \\
\begin{tabular}{rrrrrrrrp{120mm}}
WV&WK&WB&ABK&ABB&ABV&AnzB&TW&Zahlencode \textcolor{red}{$\boldsymbol{Grundtext}$} Umschrift $|$"Ubersetzung(en)\\
1.&185.&3374.&694.&13184.&1.&2&310&10\_300 \textcolor{red}{\textcjheb{+sy}} JS $|$da ist/(es) ist\\
2.&186.&3375.&696.&13186.&3.&3&224&4\_200\_20 \textcolor{red}{\textcjheb{krd}} DRK $|$ein Weg/(mancher) Weg\\
3.&187.&3376.&699.&13189.&6.&3&510&10\_300\_200 \textcolor{red}{\textcjheb{r+sy}} JSR $|$der gerade erscheint/(ein) gerader\\
4.&188.&3377.&702.&13192.&9.&4&170&30\_80\_50\_10 \textcolor{red}{\textcjheb{ynpl}} LPNJ $|$einem/vor\\
5.&189.&3378.&706.&13196.&13.&3&311&1\_10\_300 \textcolor{red}{\textcjheb{+sy'}} AJS $|$Menschen/jemandem\\
6.&190.&3379.&709.&13199.&16.&7&630&6\_1\_8\_200\_10\_400\_5 \textcolor{red}{\textcjheb{htyr.h'w}} WACRJTH $|$aber sein Ende/und sein Ende\\
7.&191.&3380.&716.&13206.&23.&4&234&4\_200\_20\_10 \textcolor{red}{\textcjheb{ykrd}} DRKJ $|$(sind) Wege\\
8.&192.&3381.&720.&13210.&27.&3&446&40\_6\_400 \textcolor{red}{\textcjheb{twm}} MWT $|$des Todes\\
\end{tabular}\medskip \\
Ende des Verses 16.25\\
Verse: 465, Buchstaben: 29, 722, 13212, Totalwerte: 2835, 46840, 950130\\
\\
Da ist ein Weg, der einem Menschen gerade erscheint, aber sein Ende sind Wege des Todes.\\
\newpage 
{\bf -- 16.26}\\
\medskip \\
\begin{tabular}{rrrrrrrrp{120mm}}
WV&WK&WB&ABK&ABB&ABV&AnzB&TW&Zahlencode \textcolor{red}{$\boldsymbol{Grundtext}$} Umschrift $|$"Ubersetzung(en)\\
1.&193.&3382.&723.&13213.&1.&3&430&50\_80\_300 \textcolor{red}{\textcjheb{+spn}} NPS $|$(der) Hunger/die Gier\\
2.&194.&3383.&726.&13216.&4.&3&140&70\_40\_30 \textcolor{red}{\textcjheb{lm`}} aML $|$des Arbeiters\\
3.&195.&3384.&729.&13219.&7.&4&145&70\_40\_30\_5 \textcolor{red}{\textcjheb{hlm`}} aMLH $|$arbeitet/(sie) bem"uht sich\\
4.&196.&3385.&733.&13223.&11.&2&36&30\_6 \textcolor{red}{\textcjheb{wl}} LW $|$f"ur ihn\\
5.&197.&3386.&735.&13225.&13.&2&30&20\_10 \textcolor{red}{\textcjheb{yk}} KJ $|$denn\\
6.&198.&3387.&737.&13227.&15.&3&101&1\_20\_80 \textcolor{red}{\textcjheb{pk'}} AKP $|$spornt/er (=es) treibt\\
7.&199.&3388.&740.&13230.&18.&4&116&70\_30\_10\_6 \textcolor{red}{\textcjheb{wyl`}} aLJW $|$an ihn\\
8.&200.&3389.&744.&13234.&22.&4&101&80\_10\_5\_6 \textcolor{red}{\textcjheb{whyp}} PJHW $|$sein Mund\\
\end{tabular}\medskip \\
Ende des Verses 16.26\\
Verse: 466, Buchstaben: 25, 747, 13237, Totalwerte: 1099, 47939, 951229\\
\\
Des Arbeiters Hunger arbeitet f"ur ihn, denn sein Mund spornt ihn an.\\
\newpage 
{\bf -- 16.27}\\
\medskip \\
\begin{tabular}{rrrrrrrrp{120mm}}
WV&WK&WB&ABK&ABB&ABV&AnzB&TW&Zahlencode \textcolor{red}{$\boldsymbol{Grundtext}$} Umschrift $|$"Ubersetzung(en)\\
1.&201.&3390.&748.&13238.&1.&3&311&1\_10\_300 \textcolor{red}{\textcjheb{+sy'}} AJS $|$(ein) Mann\\
2.&202.&3391.&751.&13241.&4.&5&142&2\_30\_10\_70\_30 \textcolor{red}{\textcjheb{l`ylb}} BLJaL $|$Belials/nichtsnutziger\\
3.&203.&3392.&756.&13246.&9.&3&225&20\_200\_5 \textcolor{red}{\textcjheb{hrk}} KRH $|$gr"abt nach/(ist) ausgrabend\\
4.&204.&3393.&759.&13249.&12.&3&275&200\_70\_5 \textcolor{red}{\textcjheb{h`r}} RaH $|$B"osem/Unheil\\
5.&205.&3394.&762.&13252.&15.&3&106&6\_70\_30 \textcolor{red}{\textcjheb{l`w}} WaL $|$und auf\\
6.&206.&3395.&765.&13255.&18.&5&796&300\_80\_400\_10\_6 \textcolor{red}{\textcjheb{wytp+s}} SPTJW $|$seinen Lippen\\
7.&207.&3396.&770.&13260.&23.&3&321&20\_1\_300 \textcolor{red}{\textcjheb{+s'k}} KAS $|$(ist es) wie (das) Feuer\\
8.&208.&3397.&773.&13263.&26.&4&692&90\_200\_2\_400 \textcolor{red}{\textcjheb{tbr.s}} "sRBT $|$brennendes/der Versengung\\
\end{tabular}\medskip \\
Ende des Verses 16.27\\
Verse: 467, Buchstaben: 29, 776, 13266, Totalwerte: 2868, 50807, 954097\\
\\
Ein Belialsmann gr"abt nach B"osem, und auf seinen Lippen ist es wie brennendes Feuer.\\
\newpage 
{\bf -- 16.28}\\
\medskip \\
\begin{tabular}{rrrrrrrrp{120mm}}
WV&WK&WB&ABK&ABB&ABV&AnzB&TW&Zahlencode \textcolor{red}{$\boldsymbol{Grundtext}$} Umschrift $|$"Ubersetzung(en)\\
1.&209.&3398.&777.&13267.&1.&3&311&1\_10\_300 \textcolor{red}{\textcjheb{+sy'}} AJS $|$(ein) Mann\\
2.&210.&3399.&780.&13270.&4.&6&911&400\_5\_80\_20\_6\_400 \textcolor{red}{\textcjheb{twkpht}} THPKWT $|$verkehrter/(der) Verkehrtheiten\\
3.&211.&3400.&786.&13276.&10.&4&348&10\_300\_30\_8 \textcolor{red}{\textcjheb{.hl+sy}} JSLC $|$streut aus/er (=es) stiftet\\
4.&212.&3401.&790.&13280.&14.&4&100&40\_4\_6\_50 \textcolor{red}{\textcjheb{nwdm}} MDWN $|$Zwietracht/Streit\\
5.&213.&3402.&794.&13284.&18.&5&309&6\_50\_200\_3\_50 \textcolor{red}{\textcjheb{ngrnw}} WNRGN $|$und ein Ohrenbl"aser/und ein Verleumder\\
6.&214.&3403.&799.&13289.&23.&5&334&40\_80\_200\_10\_4 \textcolor{red}{\textcjheb{dyrpm}} MPRJD $|$entzweit/er trennt sich\\
7.&215.&3404.&804.&13294.&28.&4&117&1\_30\_6\_80 \textcolor{red}{\textcjheb{pwl'}} ALWP $|$(von einem) Vertraute(n)\\
\end{tabular}\medskip \\
Ende des Verses 16.28\\
Verse: 468, Buchstaben: 31, 807, 13297, Totalwerte: 2430, 53237, 956527\\
\\
Ein verkehrter Mann streut Zwietracht aus, und ein Ohrenbl"aser entzweit Vertraute.\\
\newpage 
{\bf -- 16.29}\\
\medskip \\
\begin{tabular}{rrrrrrrrp{120mm}}
WV&WK&WB&ABK&ABB&ABV&AnzB&TW&Zahlencode \textcolor{red}{$\boldsymbol{Grundtext}$} Umschrift $|$"Ubersetzung(en)\\
1.&216.&3405.&808.&13298.&1.&3&311&1\_10\_300 \textcolor{red}{\textcjheb{+sy'}} AJS $|$(ein) Mann\\
2.&217.&3406.&811.&13301.&4.&3&108&8\_40\_60 \textcolor{red}{\textcjheb{sm.h}} CMs $|$(der) Gewalttat\\
3.&218.&3407.&814.&13304.&7.&4&495&10\_80\_400\_5 \textcolor{red}{\textcjheb{htpy}} JPTH $|$verlockt/(er) bet"ort\\
4.&219.&3408.&818.&13308.&11.&4&281&200\_70\_5\_6 \textcolor{red}{\textcjheb{wh`r}} RaHW $|$seinen N"achsten/seinen Gef"ahrten\\
5.&220.&3409.&822.&13312.&15.&7&83&6\_5\_6\_30\_10\_20\_6 \textcolor{red}{\textcjheb{wkylwhw}} WHWLJKW $|$und f"uhrt ihn/und er macht gehen ihn\\
6.&221.&3410.&829.&13319.&22.&4&226&2\_4\_200\_20 \textcolor{red}{\textcjheb{krdb}} BDRK $|$auf einem Weg/auf dem Weg\\
7.&222.&3411.&833.&13323.&26.&2&31&30\_1 \textcolor{red}{\textcjheb{'l}} LA $|$(der) nicht\\
8.&223.&3412.&835.&13325.&28.&3&17&9\_6\_2 \textcolor{red}{\textcjheb{bw.t}} tWB $|$gut(en) (ist)\\
\end{tabular}\medskip \\
Ende des Verses 16.29\\
Verse: 469, Buchstaben: 30, 837, 13327, Totalwerte: 1552, 54789, 958079\\
\\
Ein Mann der Gewalttat verlockt seinen N"achsten und f"uhrt ihn auf einen Weg, der nicht gut ist.\\
\newpage 
{\bf -- 16.30}\\
\medskip \\
\begin{tabular}{rrrrrrrrp{120mm}}
WV&WK&WB&ABK&ABB&ABV&AnzB&TW&Zahlencode \textcolor{red}{$\boldsymbol{Grundtext}$} Umschrift $|$"Ubersetzung(en)\\
1.&224.&3413.&838.&13328.&1.&3&165&70\_90\_5 \textcolor{red}{\textcjheb{h.s`}} a"sH $|$wer zudr"uckt/(ein) Schlie"sender\\
2.&225.&3414.&841.&13331.&4.&5&146&70\_10\_50\_10\_6 \textcolor{red}{\textcjheb{wyny`}} aJNJW $|$seine Augen\\
3.&226.&3415.&846.&13336.&9.&4&340&30\_8\_300\_2 \textcolor{red}{\textcjheb{b+s.hl}} LCSB $|$um zu ersinnen\\
4.&227.&3416.&850.&13340.&13.&6&911&400\_5\_80\_20\_6\_400 \textcolor{red}{\textcjheb{twkpht}} THPKWT $|$Verkehrtes/Verkehrtheiten\\
5.&228.&3417.&856.&13346.&19.&3&390&100\_200\_90 \textcolor{red}{\textcjheb{.srq}} QR"s $|$zusammenkneift/pressend\\
6.&229.&3418.&859.&13349.&22.&5&796&300\_80\_400\_10\_6 \textcolor{red}{\textcjheb{wytp+s}} SPTJW $|$seine Lippen\\
7.&230.&3419.&864.&13354.&27.&3&55&20\_30\_5 \textcolor{red}{\textcjheb{hlk}} KLH $|$hat beschlossen/er hat vollbracht\\
8.&231.&3420.&867.&13357.&30.&3&275&200\_70\_5 \textcolor{red}{\textcjheb{h`r}} RaH $|$das B"ose\\
\end{tabular}\medskip \\
Ende des Verses 16.30\\
Verse: 470, Buchstaben: 32, 869, 13359, Totalwerte: 3078, 57867, 961157\\
\\
Wer seine Augen zudr"uckt, um Verkehrtes zu ersinnen, seine Lippen zusammenkneift, hat das B"ose beschlossen.\\
\newpage 
{\bf -- 16.31}\\
\medskip \\
\begin{tabular}{rrrrrrrrp{120mm}}
WV&WK&WB&ABK&ABB&ABV&AnzB&TW&Zahlencode \textcolor{red}{$\boldsymbol{Grundtext}$} Umschrift $|$"Ubersetzung(en)\\
1.&232.&3421.&870.&13360.&1.&4&679&70\_9\_200\_400 \textcolor{red}{\textcjheb{tr.t`}} atRT $|$eine Krone\\
2.&233.&3422.&874.&13364.&5.&5&1081&400\_80\_1\_200\_400 \textcolor{red}{\textcjheb{tr'pt}} TPART $|$pr"achtige/(von) Ruhm\\
3.&234.&3423.&879.&13369.&10.&4&317&300\_10\_2\_5 \textcolor{red}{\textcjheb{hby+s}} SJBH $|$ist das graue Haar/(ist) Greisenhaar\\
4.&235.&3424.&883.&13373.&14.&4&226&2\_4\_200\_20 \textcolor{red}{\textcjheb{krdb}} BDRK $|$auf dem Weg\\
5.&236.&3425.&887.&13377.&18.&4&199&90\_4\_100\_5 \textcolor{red}{\textcjheb{hqd.s}} "sDQH $|$der Gerechtigkeit\\
6.&237.&3426.&891.&13381.&22.&4&531&400\_40\_90\_1 \textcolor{red}{\textcjheb{'.smt}} TM"sA $|$wird sie gefunden/sie l"asst sich finden\\
\end{tabular}\medskip \\
Ende des Verses 16.31\\
Verse: 471, Buchstaben: 25, 894, 13384, Totalwerte: 3033, 60900, 964190\\
\\
Das graue Haar ist eine pr"achtige Krone: auf dem Wege der Gerechtigkeit wird sie gefunden.\\
\newpage 
{\bf -- 16.32}\\
\medskip \\
\begin{tabular}{rrrrrrrrp{120mm}}
WV&WK&WB&ABK&ABB&ABV&AnzB&TW&Zahlencode \textcolor{red}{$\boldsymbol{Grundtext}$} Umschrift $|$"Ubersetzung(en)\\
1.&238.&3427.&895.&13385.&1.&3&17&9\_6\_2 \textcolor{red}{\textcjheb{bw.t}} tWB $|$besser/gut (ist)\\
2.&239.&3428.&898.&13388.&4.&3&221&1\_200\_20 \textcolor{red}{\textcjheb{kr'}} ARK $|$ein Lang-/(die) L"ange\\
3.&240.&3429.&901.&13391.&7.&4&131&1\_80\_10\_40 \textcolor{red}{\textcjheb{myp'}} APJM $|$m"utiger/zweier Nasenl"ocher\\
4.&241.&3430.&905.&13395.&11.&5&251&40\_3\_2\_6\_200 \textcolor{red}{\textcjheb{rwbgm}} MGBWR $|$(mehr) als ein (Kriegs)Held\\
5.&242.&3431.&910.&13400.&16.&4&376&6\_40\_300\_30 \textcolor{red}{\textcjheb{l+smw}} WMSL $|$und wer beherrscht/und (ein) Herrschender\\
6.&243.&3432.&914.&13404.&20.&5&222&2\_200\_6\_8\_6 \textcolor{red}{\textcjheb{w.hwrb}} BRWCW $|$seinen Geist/in seinem Geist\\
7.&244.&3433.&919.&13409.&25.&4&94&40\_30\_20\_4 \textcolor{red}{\textcjheb{dklm}} MLKD $|$als wer erobert/mehr als ein Bezwingender\\
8.&245.&3434.&923.&13413.&29.&3&280&70\_10\_200 \textcolor{red}{\textcjheb{ry`}} aJR $|$(eine) Stadt\\
\end{tabular}\medskip \\
Ende des Verses 16.32\\
Verse: 472, Buchstaben: 31, 925, 13415, Totalwerte: 1592, 62492, 965782\\
\\
Besser ein Langm"utiger als ein Held, und wer seinen Geist beherrscht, als wer eine Stadt erobert.\\
\newpage 
{\bf -- 16.33}\\
\medskip \\
\begin{tabular}{rrrrrrrrp{120mm}}
WV&WK&WB&ABK&ABB&ABV&AnzB&TW&Zahlencode \textcolor{red}{$\boldsymbol{Grundtext}$} Umschrift $|$"Ubersetzung(en)\\
1.&246.&3435.&926.&13416.&1.&4&120&2\_8\_10\_100 \textcolor{red}{\textcjheb{qy.hb}} BCJQ $|$im Busen\\
2.&247.&3436.&930.&13420.&5.&4&55&10\_6\_9\_30 \textcolor{red}{\textcjheb{l.twy}} JWtL $|$(er (=es)) wird geworfen\\
3.&248.&3437.&934.&13424.&9.&2&401&1\_400 \textcolor{red}{\textcjheb{t'}} AT $|$**\\
4.&249.&3438.&936.&13426.&11.&5&244&5\_3\_6\_200\_30 \textcolor{red}{\textcjheb{lrwgh}} HGWRL $|$das Los\\
5.&250.&3439.&941.&13431.&16.&6&72&6\_40\_10\_5\_6\_5 \textcolor{red}{\textcjheb{hwhymw}} WMJHWH $|$aber von Jahwe/und von Jahwe\\
6.&251.&3440.&947.&13437.&22.&2&50&20\_30 \textcolor{red}{\textcjheb{lk}} KL $|$kommt all/(geht aus) all\\
7.&252.&3441.&949.&13439.&24.&5&435&40\_300\_80\_9\_6 \textcolor{red}{\textcjheb{w.tp+sm}} MSPtW $|$seine Entscheidung/sein Rechtsentscheid\\
\end{tabular}\medskip \\
Ende des Verses 16.33\\
Verse: 473, Buchstaben: 28, 953, 13443, Totalwerte: 1377, 63869, 967159\\
\\
Das Los wird in dem Busen geworfen, aber all seine Entscheidung kommt von Jahwe.\\
\\
{\bf Ende des Kapitels 16}\\
\newpage 
{\bf -- 17.1}\\
\medskip \\
\begin{tabular}{rrrrrrrrp{120mm}}
WV&WK&WB&ABK&ABB&ABV&AnzB&TW&Zahlencode \textcolor{red}{$\boldsymbol{Grundtext}$} Umschrift $|$"Ubersetzung(en)\\
1.&1.&3442.&1.&13444.&1.&3&17&9\_6\_2 \textcolor{red}{\textcjheb{bw.t}} tWB $|$besser/gut (ist)\\
2.&2.&3443.&4.&13447.&4.&2&480&80\_400 \textcolor{red}{\textcjheb{tp}} PT $|$(ein) Bissen\\
3.&3.&3444.&6.&13449.&6.&4&215&8\_200\_2\_5 \textcolor{red}{\textcjheb{hbr.h}} CRBH $|$trockener\\
4.&4.&3445.&10.&13453.&10.&5&347&6\_300\_30\_6\_5 \textcolor{red}{\textcjheb{hwl+sw}} WSLWH $|$und Friede/und Ruhe\\
5.&5.&3446.&15.&13458.&15.&2&7&2\_5 \textcolor{red}{\textcjheb{hb}} BH $|$dabei\\
6.&6.&3447.&17.&13460.&17.&4&452&40\_2\_10\_400 \textcolor{red}{\textcjheb{tybm}} MBJT $|$als ein Haus\\
7.&7.&3448.&21.&13464.&21.&3&71&40\_30\_1 \textcolor{red}{\textcjheb{'lm}} MLA $|$voll\\
8.&8.&3449.&24.&13467.&24.&4&27&7\_2\_8\_10 \textcolor{red}{\textcjheb{y.hbz}} ZBCJ $|$Opferfleisch/Schlachtopfer\\
9.&9.&3450.&28.&13471.&28.&3&212&200\_10\_2 \textcolor{red}{\textcjheb{byr}} RJB $|$(mit) Zank\\
\end{tabular}\medskip \\
Ende des Verses 17.1\\
Verse: 474, Buchstaben: 30, 30, 13473, Totalwerte: 1828, 1828, 968987\\
\\
Besser ein trockener Bissen und Friede dabei, als ein Haus voll Opferfleisch mit Zank.\\
\newpage 
{\bf -- 17.2}\\
\medskip \\
\begin{tabular}{rrrrrrrrp{120mm}}
WV&WK&WB&ABK&ABB&ABV&AnzB&TW&Zahlencode \textcolor{red}{$\boldsymbol{Grundtext}$} Umschrift $|$"Ubersetzung(en)\\
1.&10.&3451.&31.&13474.&1.&3&76&70\_2\_4 \textcolor{red}{\textcjheb{db`}} aBD $|$(ein) Knecht\\
2.&11.&3452.&34.&13477.&4.&5&400&40\_300\_20\_10\_30 \textcolor{red}{\textcjheb{lyk+sm}} MSKJL $|$einsichtiger/verst"andiger\\
3.&12.&3453.&39.&13482.&9.&4&380&10\_40\_300\_30 \textcolor{red}{\textcjheb{l+smy}} JMSL $|$wird herrschen/(er) herrscht\\
4.&13.&3454.&43.&13486.&13.&3&54&2\_2\_50 \textcolor{red}{\textcjheb{nbb}} BBN $|$"uber den Sohn/"uber einen Sohn\\
5.&14.&3455.&46.&13489.&16.&4&352&40\_2\_10\_300 \textcolor{red}{\textcjheb{+sybm}} MBJS $|$sch"andlichen/Schande machenden\\
6.&15.&3456.&50.&13493.&20.&5&434&6\_2\_400\_6\_20 \textcolor{red}{\textcjheb{kwtbw}} WBTWK $|$und inmitten\\
7.&16.&3457.&55.&13498.&25.&4&59&1\_8\_10\_40 \textcolor{red}{\textcjheb{my.h'}} ACJM $|$(der) Br"uder\\
8.&17.&3458.&59.&13502.&29.&4&148&10\_8\_30\_100 \textcolor{red}{\textcjheb{ql.hy}} JCLQ $|$teilen/er teilt (zu)\\
9.&18.&3459.&63.&13506.&33.&4&93&50\_8\_30\_5 \textcolor{red}{\textcjheb{hl.hn}} NCLH $|$die Erbschaft/(das) Erbe\\
\end{tabular}\medskip \\
Ende des Verses 17.2\\
Verse: 475, Buchstaben: 36, 66, 13509, Totalwerte: 1996, 3824, 970983\\
\\
Ein einsichtiger Knecht wird "uber den sch"andlichen Sohn herrschen, und inmitten der Br"uder die Erbschaft teilen.\\
\newpage 
{\bf -- 17.3}\\
\medskip \\
\begin{tabular}{rrrrrrrrp{120mm}}
WV&WK&WB&ABK&ABB&ABV&AnzB&TW&Zahlencode \textcolor{red}{$\boldsymbol{Grundtext}$} Umschrift $|$"Ubersetzung(en)\\
1.&19.&3460.&67.&13510.&1.&4&410&40\_90\_200\_80 \textcolor{red}{\textcjheb{pr.sm}} M"sRP $|$der Schmelztiegel\\
2.&20.&3461.&71.&13514.&5.&4&190&30\_20\_60\_80 \textcolor{red}{\textcjheb{pskl}} LKsP $|$(ist) f"ur das Silber\\
3.&21.&3462.&75.&13518.&9.&4&232&6\_20\_6\_200 \textcolor{red}{\textcjheb{rwkw}} WKWR $|$und der (Schmelz)Ofen\\
4.&22.&3463.&79.&13522.&13.&4&44&30\_7\_5\_2 \textcolor{red}{\textcjheb{bhzl}} LZHB $|$f"ur das Gold\\
5.&23.&3464.&83.&13526.&17.&4&66&6\_2\_8\_50 \textcolor{red}{\textcjheb{n.hbw}} WBCN $|$aber Pr"ufer/und Pr"ufender\\
6.&24.&3465.&87.&13530.&21.&4&438&30\_2\_6\_400 \textcolor{red}{\textcjheb{twbl}} LBWT $|$der Herzen/(die) Herzen\\
7.&25.&3466.&91.&13534.&25.&4&26&10\_5\_6\_5 \textcolor{red}{\textcjheb{hwhy}} JHWH $|$(ist) Jahwe\\
\end{tabular}\medskip \\
Ende des Verses 17.3\\
Verse: 476, Buchstaben: 28, 94, 13537, Totalwerte: 1406, 5230, 972389\\
\\
Der Schmelztiegel f"ur das Silber, und der Ofen f"ur das Gold; aber Pr"ufer der Herzen ist Jahwe.\\
\newpage 
{\bf -- 17.4}\\
\medskip \\
\begin{tabular}{rrrrrrrrp{120mm}}
WV&WK&WB&ABK&ABB&ABV&AnzB&TW&Zahlencode \textcolor{red}{$\boldsymbol{Grundtext}$} Umschrift $|$"Ubersetzung(en)\\
1.&26.&3467.&95.&13538.&1.&3&310&40\_200\_70 \textcolor{red}{\textcjheb{`rm}} MRa $|$ein "Ubelt"ater/der B"ose\\
2.&27.&3468.&98.&13541.&4.&5&452&40\_100\_300\_10\_2 \textcolor{red}{\textcjheb{by+sqm}} MQSJB $|$horcht/ist achtend\\
3.&28.&3469.&103.&13546.&9.&2&100&70\_30 \textcolor{red}{\textcjheb{l`}} aL $|$auf\\
4.&29.&3470.&105.&13548.&11.&3&780&300\_80\_400 \textcolor{red}{\textcjheb{tp+s}} SPT $|$die Lippe\\
5.&30.&3471.&108.&13551.&14.&3&57&1\_6\_50 \textcolor{red}{\textcjheb{nw'}} AWN $|$des Unheils\\
6.&31.&3472.&111.&13554.&17.&3&600&300\_100\_200 \textcolor{red}{\textcjheb{rq+s}} SQR $|$ein L"ugner/die L"uge\\
7.&32.&3473.&114.&13557.&20.&4&107&40\_7\_10\_50 \textcolor{red}{\textcjheb{nyzm}} MZJN $|$gibt Geh"or/(ist) horchend\\
8.&33.&3474.&118.&13561.&24.&2&100&70\_30 \textcolor{red}{\textcjheb{l`}} aL $|$/auf\\
9.&34.&3475.&120.&13563.&26.&4&386&30\_300\_6\_50 \textcolor{red}{\textcjheb{nw+sl}} LSWN $|$der Zunge/die Zunge\\
10.&35.&3476.&124.&13567.&30.&3&411&5\_6\_400 \textcolor{red}{\textcjheb{twh}} HWT $|$des Verderbens\\
\end{tabular}\medskip \\
Ende des Verses 17.4\\
Verse: 477, Buchstaben: 32, 126, 13569, Totalwerte: 3303, 8533, 975692\\
\\
Ein "Ubelt"ater horcht auf die Lippe des Unheils, ein L"ugner gibt Geh"or der Zunge des Verderbens.\\
\newpage 
{\bf -- 17.5}\\
\medskip \\
\begin{tabular}{rrrrrrrrp{120mm}}
WV&WK&WB&ABK&ABB&ABV&AnzB&TW&Zahlencode \textcolor{red}{$\boldsymbol{Grundtext}$} Umschrift $|$"Ubersetzung(en)\\
1.&36.&3477.&127.&13570.&1.&3&103&30\_70\_3 \textcolor{red}{\textcjheb{g`l}} LaG $|$wer spottet/ein Spottender\\
2.&37.&3478.&130.&13573.&4.&3&530&30\_200\_300 \textcolor{red}{\textcjheb{+srl}} LRS $|$des Armen/"uber den Armen\\
3.&38.&3479.&133.&13576.&7.&3&288&8\_200\_80 \textcolor{red}{\textcjheb{pr.h}} CRP $|$(er) verh"ohnt(e) (den)\\
4.&39.&3480.&136.&13579.&10.&4&381&70\_300\_5\_6 \textcolor{red}{\textcjheb{wh+s`}} aSHW $|$der gemacht ihn hat/seinen Machenden\\
5.&40.&3481.&140.&13583.&14.&3&348&300\_40\_8 \textcolor{red}{\textcjheb{.hm+s}} SMC $|$wer sich freut(e)\\
6.&41.&3482.&143.&13586.&17.&4&45&30\_1\_10\_4 \textcolor{red}{\textcjheb{dy'l}} LAJD $|$"uber Ungl"uck/am Ungl"uck\\
7.&42.&3483.&147.&13590.&21.&2&31&30\_1 \textcolor{red}{\textcjheb{'l}} LA $|$nicht\\
8.&43.&3484.&149.&13592.&23.&4&165&10\_50\_100\_5 \textcolor{red}{\textcjheb{hqny}} JNQH $|$wird f"ur schuldlos gehalten werden/(er) bleibt ungestraft\\
\end{tabular}\medskip \\
Ende des Verses 17.5\\
Verse: 478, Buchstaben: 26, 152, 13595, Totalwerte: 1891, 10424, 977583\\
\\
Wer des Armen spottet, verh"ohnt den, der ihn gemacht hat; wer "uber Ungl"uck sich freut, wird nicht f"ur schuldlos gehalten werden.\\
\newpage 
{\bf -- 17.6}\\
\medskip \\
\begin{tabular}{rrrrrrrrp{120mm}}
WV&WK&WB&ABK&ABB&ABV&AnzB&TW&Zahlencode \textcolor{red}{$\boldsymbol{Grundtext}$} Umschrift $|$"Ubersetzung(en)\\
1.&44.&3485.&153.&13596.&1.&4&679&70\_9\_200\_400 \textcolor{red}{\textcjheb{tr.t`}} atRT $|$die Krone\\
2.&45.&3486.&157.&13600.&5.&5&207&7\_100\_50\_10\_40 \textcolor{red}{\textcjheb{mynqz}} ZQNJM $|$der Alten/der Greisen\\
3.&46.&3487.&162.&13605.&10.&3&62&2\_50\_10 \textcolor{red}{\textcjheb{ynb}} BNJ $|$sind Kinder/(sind) S"ohne\\
4.&47.&3488.&165.&13608.&13.&4&102&2\_50\_10\_40 \textcolor{red}{\textcjheb{mynb}} BNJM $|$(des) Kindes/(von) S"ohnen\\
5.&48.&3489.&169.&13612.&17.&6&1087&6\_400\_80\_1\_200\_400 \textcolor{red}{\textcjheb{tr'ptw}} WTPART $|$und Schmuck/und die Ehre\\
6.&49.&3490.&175.&13618.&23.&4&102&2\_50\_10\_40 \textcolor{red}{\textcjheb{mynb}} BNJM $|$der Kinder/(der) S"ohne\\
7.&50.&3491.&179.&13622.&27.&5&449&1\_2\_6\_400\_40 \textcolor{red}{\textcjheb{mtwb'}} ABWTM $|$(sind) ihre V"ater\\
\end{tabular}\medskip \\
Ende des Verses 17.6\\
Verse: 479, Buchstaben: 31, 183, 13626, Totalwerte: 2688, 13112, 980271\\
\\
Kindeskinder sind die Krone der Alten, und der Kinder Schmuck sind ihre V"ater.\\
\newpage 
{\bf -- 17.7}\\
\medskip \\
\begin{tabular}{rrrrrrrrp{120mm}}
WV&WK&WB&ABK&ABB&ABV&AnzB&TW&Zahlencode \textcolor{red}{$\boldsymbol{Grundtext}$} Umschrift $|$"Ubersetzung(en)\\
1.&51.&3492.&184.&13627.&1.&2&31&30\_1 \textcolor{red}{\textcjheb{'l}} LA $|$nicht\\
2.&52.&3493.&186.&13629.&3.&4&62&50\_1\_6\_5 \textcolor{red}{\textcjheb{hw'n}} NAWH $|$schickt sich/sie (=es) geziemt\\
3.&53.&3494.&190.&13633.&7.&4&112&30\_50\_2\_30 \textcolor{red}{\textcjheb{lbnl}} LNBL $|$f"ur einen gemeinen Menschen/dem Toren\\
4.&54.&3495.&194.&13637.&11.&3&780&300\_80\_400 \textcolor{red}{\textcjheb{tp+s}} SPT $|$Rede\\
5.&55.&3496.&197.&13640.&14.&3&610&10\_400\_200 \textcolor{red}{\textcjheb{rty}} JTR $|$vortreffliche/anma"sende\\
6.&56.&3497.&200.&13643.&17.&2&81&1\_80 \textcolor{red}{\textcjheb{p'}} AP $|$wie viel weniger\\
7.&57.&3498.&202.&13645.&19.&2&30&20\_10 \textcolor{red}{\textcjheb{yk}} KJ $|$/wenn\\
8.&58.&3499.&204.&13647.&21.&5&96&30\_50\_4\_10\_2 \textcolor{red}{\textcjheb{bydnl}} LNDJB $|$f"ur einen Edlen/dem Edlen\\
9.&59.&3500.&209.&13652.&26.&3&780&300\_80\_400 \textcolor{red}{\textcjheb{tp+s}} SPT $|$Rede/eine Lippe\\
10.&60.&3501.&212.&13655.&29.&3&600&300\_100\_200 \textcolor{red}{\textcjheb{rq+s}} SQR $|$(von) L"ugen/der L"uge\\
\end{tabular}\medskip \\
Ende des Verses 17.7\\
Verse: 480, Buchstaben: 31, 214, 13657, Totalwerte: 3182, 16294, 983453\\
\\
Vortreffliche Rede schickt sich nicht f"ur einen gemeinen Menschen; wieviel weniger L"ugenrede f"ur einen Edlen!\\
\newpage 
{\bf -- 17.8}\\
\medskip \\
\begin{tabular}{rrrrrrrrp{120mm}}
WV&WK&WB&ABK&ABB&ABV&AnzB&TW&Zahlencode \textcolor{red}{$\boldsymbol{Grundtext}$} Umschrift $|$"Ubersetzung(en)\\
1.&61.&3502.&215.&13658.&1.&3&53&1\_2\_50 \textcolor{red}{\textcjheb{nb'}} ABN $|$(ein) Stein\\
2.&62.&3503.&218.&13661.&4.&2&58&8\_50 \textcolor{red}{\textcjheb{n.h}} CN $|$Edel-/(der) Gunst\\
3.&63.&3504.&220.&13663.&6.&4&317&5\_300\_8\_4 \textcolor{red}{\textcjheb{d.h+sh}} HSCD $|$ist das Geschenk/(ist) die Bestechung\\
4.&64.&3505.&224.&13667.&10.&5&142&2\_70\_10\_50\_10 \textcolor{red}{\textcjheb{yny`b}} BaJNJ $|$in den Augen\\
5.&65.&3506.&229.&13672.&15.&5&118&2\_70\_30\_10\_6 \textcolor{red}{\textcjheb{wyl`b}} BaLJW $|$des Empf"angers/seiner Besitzer\\
6.&66.&3507.&234.&13677.&20.&2&31&1\_30 \textcolor{red}{\textcjheb{l'}} AL $|$wo-/zu\\
7.&67.&3508.&236.&13679.&22.&2&50&20\_30 \textcolor{red}{\textcjheb{lk}} KL $|$hin/all\\
8.&68.&3509.&238.&13681.&24.&3&501&1\_300\_200 \textcolor{red}{\textcjheb{r+s'}} ASR $|$er/welchem\\
9.&69.&3510.&241.&13684.&27.&4&145&10\_80\_50\_5 \textcolor{red}{\textcjheb{hnpy}} JPNH $|$(er) sich wendet \\
10.&70.&3511.&245.&13688.&31.&5&370&10\_300\_20\_10\_30 \textcolor{red}{\textcjheb{lyk+sy}} JSKJL $|$gelingt es ihm/er hat Erfolg\\
\end{tabular}\medskip \\
Ende des Verses 17.8\\
Verse: 481, Buchstaben: 35, 249, 13692, Totalwerte: 1785, 18079, 985238\\
\\
Das Geschenk ist ein Edelstein in den Augen des Empf"angers; wohin er sich wendet, gelingt es ihm.\\
\newpage 
{\bf -- 17.9}\\
\medskip \\
\begin{tabular}{rrrrrrrrp{120mm}}
WV&WK&WB&ABK&ABB&ABV&AnzB&TW&Zahlencode \textcolor{red}{$\boldsymbol{Grundtext}$} Umschrift $|$"Ubersetzung(en)\\
1.&71.&3512.&250.&13693.&1.&4&125&40\_20\_60\_5 \textcolor{red}{\textcjheb{hskm}} MKsH $|$(es) deckt zu/Bedeckender\\
2.&72.&3513.&254.&13697.&5.&3&450&80\_300\_70 \textcolor{red}{\textcjheb{`+sp}} PSa $|$die "Ubertretung/Missetat\\
3.&73.&3514.&257.&13700.&8.&4&442&40\_2\_100\_300 \textcolor{red}{\textcjheb{+sqbm}} MBQS $|$wer sucht/(ist) suchend\\
4.&74.&3515.&261.&13704.&12.&4&13&1\_5\_2\_5 \textcolor{red}{\textcjheb{hbh'}} AHBH $|$Liebe\\
5.&75.&3516.&265.&13708.&16.&4&361&6\_300\_50\_5 \textcolor{red}{\textcjheb{hn+sw}} WSNH $|$wer aber immer wieder anregt/und Wiederholender\\
6.&76.&3517.&269.&13712.&20.&4&208&2\_4\_2\_200 \textcolor{red}{\textcjheb{rbdb}} BDBR $|$eine Sache\\
7.&77.&3518.&273.&13716.&24.&5&334&40\_80\_200\_10\_4 \textcolor{red}{\textcjheb{dyrpm}} MPRJD $|$entzweit/(ist) trennender\\
8.&78.&3519.&278.&13721.&29.&4&117&1\_30\_6\_80 \textcolor{red}{\textcjheb{pwl'}} ALWP $|$Vertraute/den Freund\\
\end{tabular}\medskip \\
Ende des Verses 17.9\\
Verse: 482, Buchstaben: 32, 281, 13724, Totalwerte: 2050, 20129, 987288\\
\\
Wer Liebe sucht, deckt die "Ubertretung zu; wer aber eine Sache immer wieder anregt, entzweit Vertraute.\\
\newpage 
{\bf -- 17.10}\\
\medskip \\
\begin{tabular}{rrrrrrrrp{120mm}}
WV&WK&WB&ABK&ABB&ABV&AnzB&TW&Zahlencode \textcolor{red}{$\boldsymbol{Grundtext}$} Umschrift $|$"Ubersetzung(en)\\
1.&79.&3520.&282.&13725.&1.&3&808&400\_8\_400 \textcolor{red}{\textcjheb{t.ht}} TCT $|$(es) dringt tiefer ein/sie (=es) steigt hinab\\
2.&80.&3521.&285.&13728.&4.&4&278&3\_70\_200\_5 \textcolor{red}{\textcjheb{hr`g}} GaRH $|$ein Verweis/(ein) Tadel\\
3.&81.&3522.&289.&13732.&8.&5&104&2\_40\_2\_10\_50 \textcolor{red}{\textcjheb{nybmb}} BMBJN $|$bei einem Verst"andigen\\
4.&82.&3523.&294.&13737.&13.&5&471&40\_5\_20\_6\_400 \textcolor{red}{\textcjheb{twkhm}} MHKWT $|$als Schl"age/(mehr als) Schlagen\\
5.&83.&3524.&299.&13742.&18.&4&120&20\_60\_10\_30 \textcolor{red}{\textcjheb{lysk}} KsJL $|$bei einem Toren/(einen) Toren\\
6.&84.&3525.&303.&13746.&22.&3&46&40\_1\_5 \textcolor{red}{\textcjheb{h'm}} MAH $|$hundert(mal)\\
\end{tabular}\medskip \\
Ende des Verses 17.10\\
Verse: 483, Buchstaben: 24, 305, 13748, Totalwerte: 1827, 21956, 989115\\
\\
Ein Verweis dringt bei einem Verst"andigen tiefer ein, als hundert Schl"age bei einem Toren.\\
\newpage 
{\bf -- 17.11}\\
\medskip \\
\begin{tabular}{rrrrrrrrp{120mm}}
WV&WK&WB&ABK&ABB&ABV&AnzB&TW&Zahlencode \textcolor{red}{$\boldsymbol{Grundtext}$} Umschrift $|$"Ubersetzung(en)\\
1.&85.&3526.&306.&13749.&1.&2&21&1\_20 \textcolor{red}{\textcjheb{k'}} AK $|$nur\\
2.&86.&3527.&308.&13751.&3.&3&250&40\_200\_10 \textcolor{red}{\textcjheb{yrm}} MRJ $|$Emp"orung/Widerspenstigkeit\\
3.&87.&3528.&311.&13754.&6.&4&412&10\_2\_100\_300 \textcolor{red}{\textcjheb{+sqby}} JBQS $|$(er (=es)) sucht\\
4.&88.&3529.&315.&13758.&10.&2&270&200\_70 \textcolor{red}{\textcjheb{`r}} Ra $|$(der) B"ose\\
5.&89.&3530.&317.&13760.&12.&5&97&6\_40\_30\_1\_20 \textcolor{red}{\textcjheb{k'lmw}} WMLAK $|$aber ein Bote/und (ein) Bote\\
6.&90.&3531.&322.&13765.&17.&5&238&1\_20\_7\_200\_10 \textcolor{red}{\textcjheb{yrzk'}} AKZRJ $|$grausamer/erbarmungsloser\\
7.&91.&3532.&327.&13770.&22.&4&348&10\_300\_30\_8 \textcolor{red}{\textcjheb{.hl+sy}} JSLC $|$wird gesandt werden/er (=es) wird gesendet\\
8.&92.&3533.&331.&13774.&26.&2&8&2\_6 \textcolor{red}{\textcjheb{wb}} BW $|$gegen ihn\\
\end{tabular}\medskip \\
Ende des Verses 17.11\\
Verse: 484, Buchstaben: 27, 332, 13775, Totalwerte: 1644, 23600, 990759\\
\\
Der B"ose sucht nur Emp"orung; aber ein grausamer Bote wird gegen ihn gesandt werden.\\
\newpage 
{\bf -- 17.12}\\
\medskip \\
\begin{tabular}{rrrrrrrrp{120mm}}
WV&WK&WB&ABK&ABB&ABV&AnzB&TW&Zahlencode \textcolor{red}{$\boldsymbol{Grundtext}$} Umschrift $|$"Ubersetzung(en)\\
1.&93.&3534.&333.&13776.&1.&4&389&80\_3\_6\_300 \textcolor{red}{\textcjheb{+swgp}} PGWS $|$(es) begegne/ein Begegnen\\
2.&94.&3535.&337.&13780.&5.&2&6&4\_2 \textcolor{red}{\textcjheb{bd}} DB $|$eine B"arin\\
3.&95.&3536.&339.&13782.&7.&4&356&300\_20\_6\_30 \textcolor{red}{\textcjheb{lwk+s}} SKWL $|$(die) (der Jungen) beraubt (ist)\\
4.&96.&3537.&343.&13786.&11.&4&313&2\_1\_10\_300 \textcolor{red}{\textcjheb{+sy'b}} BAJS $|$(einem) Mann\\
5.&97.&3538.&347.&13790.&15.&3&37&6\_1\_30 \textcolor{red}{\textcjheb{l'w}} WAL $|$aber nicht/und nicht\\
6.&98.&3539.&350.&13793.&18.&4&120&20\_60\_10\_30 \textcolor{red}{\textcjheb{lysk}} KsJL $|$(ein) Tor\\
7.&99.&3540.&354.&13797.&22.&6&445&2\_1\_6\_30\_400\_6 \textcolor{red}{\textcjheb{wtlw'b}} BAWLTW $|$in seiner Narrheit/seinem Unverstand\\
\end{tabular}\medskip \\
Ende des Verses 17.12\\
Verse: 485, Buchstaben: 27, 359, 13802, Totalwerte: 1666, 25266, 992425\\
\\
Eine B"arin, die der Jungen beraubt ist, begegne einem Manne, aber nicht ein Tor in seiner Narrheit!\\
\newpage 
{\bf -- 17.13}\\
\medskip \\
\begin{tabular}{rrrrrrrrp{120mm}}
WV&WK&WB&ABK&ABB&ABV&AnzB&TW&Zahlencode \textcolor{red}{$\boldsymbol{Grundtext}$} Umschrift $|$"Ubersetzung(en)\\
1.&100.&3541.&360.&13803.&1.&4&352&40\_300\_10\_2 \textcolor{red}{\textcjheb{by+sm}} MSJB $|$wer vergilt/vergeltend\\
2.&101.&3542.&364.&13807.&5.&3&275&200\_70\_5 \textcolor{red}{\textcjheb{h`r}} RaH $|$B"oses\\
3.&102.&3543.&367.&13810.&8.&3&808&400\_8\_400 \textcolor{red}{\textcjheb{t.ht}} TCT $|$f"ur/an Stelle von\\
4.&103.&3544.&370.&13813.&11.&4&22&9\_6\_2\_5 \textcolor{red}{\textcjheb{hbw.t}} tWBH $|$Gutes/Gutem\\
5.&104.&3545.&374.&13817.&15.&2&31&30\_1 \textcolor{red}{\textcjheb{'l}} LA $|$nicht\\
6.&105.&3546.&376.&13819.&17.&4&750&400\_40\_10\_300 \textcolor{red}{\textcjheb{+symt}} TMJS $|$wird weichen/sie (=es) weicht\\
7.&106.&3547.&380.&13823.&21.&3&275&200\_70\_5 \textcolor{red}{\textcjheb{h`r}} RaH $|$das B"ose/(das) Ungl"uck\\
8.&107.&3548.&383.&13826.&24.&5&458&40\_2\_10\_400\_6 \textcolor{red}{\textcjheb{wtybm}} MBJTW $|$von dessen Haus/von seinem Haus\\
\end{tabular}\medskip \\
Ende des Verses 17.13\\
Verse: 486, Buchstaben: 28, 387, 13830, Totalwerte: 2971, 28237, 995396\\
\\
Wer B"oses f"ur Gutes vergilt, von dessen Hause wird das B"ose nicht weichen.\\
\newpage 
{\bf -- 17.14}\\
\medskip \\
\begin{tabular}{rrrrrrrrp{120mm}}
WV&WK&WB&ABK&ABB&ABV&AnzB&TW&Zahlencode \textcolor{red}{$\boldsymbol{Grundtext}$} Umschrift $|$"Ubersetzung(en)\\
1.&108.&3549.&388.&13831.&1.&4&295&80\_6\_9\_200 \textcolor{red}{\textcjheb{r.twp}} PWtR $|$wie wenn einer entfesselt/ein Freilassender\\
2.&109.&3550.&392.&13835.&5.&3&90&40\_10\_40 \textcolor{red}{\textcjheb{mym}} MJM $|$Wasser\\
3.&110.&3551.&395.&13838.&8.&5&911&200\_1\_300\_10\_400 \textcolor{red}{\textcjheb{ty+s'r}} RASJT $|$(ist) (der) Anfang\\
4.&111.&3552.&400.&13843.&13.&4&100&40\_4\_6\_50 \textcolor{red}{\textcjheb{nwdm}} MDWN $|$eines Zanks/des Zanks\\
5.&112.&3553.&404.&13847.&17.&5&176&6\_30\_80\_50\_10 \textcolor{red}{\textcjheb{ynplw}} WLPNJ $|$so ehe/und bevor\\
6.&113.&3554.&409.&13852.&22.&5&508&5\_400\_3\_30\_70 \textcolor{red}{\textcjheb{`lgth}} HTGLa $|$er heftig wird/(er) losbricht\\
7.&114.&3555.&414.&13857.&27.&4&217&5\_200\_10\_2 \textcolor{red}{\textcjheb{byrh}} HRJB $|$den Streit/der Streit\\
8.&115.&3556.&418.&13861.&31.&4&365&50\_9\_6\_300 \textcolor{red}{\textcjheb{+sw.tn}} NtWS $|$lass (ab)\\
\end{tabular}\medskip \\
Ende des Verses 17.14\\
Verse: 487, Buchstaben: 34, 421, 13864, Totalwerte: 2662, 30899, 998058\\
\\
Der Anfang eines Zankes ist, wie wenn einer Wasser entfesselt; so la"s den Streit, ehe er heftig wird.\\
\newpage 
{\bf -- 17.15}\\
\medskip \\
\begin{tabular}{rrrrrrrrp{120mm}}
WV&WK&WB&ABK&ABB&ABV&AnzB&TW&Zahlencode \textcolor{red}{$\boldsymbol{Grundtext}$} Umschrift $|$"Ubersetzung(en)\\
1.&116.&3557.&422.&13865.&1.&5&244&40\_90\_4\_10\_100 \textcolor{red}{\textcjheb{qyd.sm}} M"sDJQ $|$wer rechtfertigt/ein Rechtfertigender\\
2.&117.&3558.&427.&13870.&6.&3&570&200\_300\_70 \textcolor{red}{\textcjheb{`+sr}} RSa $|$den Gesetzlosen/einen Schuldigen\\
3.&118.&3559.&430.&13873.&9.&6&626&6\_40\_200\_300\_10\_70 \textcolor{red}{\textcjheb{`y+srmw}} WMRSJa $|$und wer verdammt/und ein Machender schuldig\\
4.&119.&3560.&436.&13879.&15.&4&204&90\_4\_10\_100 \textcolor{red}{\textcjheb{qyd.s}} "sDJQ $|$(den) Gerechten\\
5.&120.&3561.&440.&13883.&19.&5&878&400\_6\_70\_2\_400 \textcolor{red}{\textcjheb{tb`wt}} TWaBT $|$(ein) Gr"auel\\
6.&121.&3562.&445.&13888.&24.&4&26&10\_5\_6\_5 \textcolor{red}{\textcjheb{hwhy}} JHWH $|$(f"ur) Jahwe\\
7.&122.&3563.&449.&13892.&28.&2&43&3\_40 \textcolor{red}{\textcjheb{mg}} GM $|$sind/(sind) auch\\
8.&123.&3564.&451.&13894.&30.&5&405&300\_50\_10\_5\_40 \textcolor{red}{\textcjheb{mhyn+s}} SNJHM $|$sie alle beide/beide von ihnen\\
\end{tabular}\medskip \\
Ende des Verses 17.15\\
Verse: 488, Buchstaben: 34, 455, 13898, Totalwerte: 2996, 33895, 1001054\\
\\
Wer den Gesetzlosen rechtfertigt, und wer den Gerechten verdammt, sie alle beide sind Jahwe ein Greuel.\\
\newpage 
{\bf -- 17.16}\\
\medskip \\
\begin{tabular}{rrrrrrrrp{120mm}}
WV&WK&WB&ABK&ABB&ABV&AnzB&TW&Zahlencode \textcolor{red}{$\boldsymbol{Grundtext}$} Umschrift $|$"Ubersetzung(en)\\
1.&124.&3565.&456.&13899.&1.&3&75&30\_40\_5 \textcolor{red}{\textcjheb{hml}} LMH $|$wozu (doch)\\
2.&125.&3566.&459.&13902.&4.&2&12&7\_5 \textcolor{red}{\textcjheb{hz}} ZH $|$/dieser\\
3.&126.&3567.&461.&13904.&6.&4&258&40\_8\_10\_200 \textcolor{red}{\textcjheb{ry.hm}} MCJR $|$Geld/Kaufpreis\\
4.&127.&3568.&465.&13908.&10.&3&16&2\_10\_4 \textcolor{red}{\textcjheb{dyb}} BJD $|$in der Hand\\
5.&128.&3569.&468.&13911.&13.&4&120&20\_60\_10\_30 \textcolor{red}{\textcjheb{lysk}} KsJL $|$(eines) Toren\\
6.&129.&3570.&472.&13915.&17.&5&586&30\_100\_50\_6\_400 \textcolor{red}{\textcjheb{twnql}} LQNWT $|$(um) zu kaufen\\
7.&130.&3571.&477.&13920.&22.&4&73&8\_20\_40\_5 \textcolor{red}{\textcjheb{hmk.h}} CKMH $|$Weisheit\\
8.&131.&3572.&481.&13924.&26.&3&38&6\_30\_2 \textcolor{red}{\textcjheb{blw}} WLB $|$da doch der Verstand/und der Verstand\\
9.&132.&3573.&484.&13927.&29.&3&61&1\_10\_50 \textcolor{red}{\textcjheb{ny'}} AJN $|$fehlt/nicht ist\\
\end{tabular}\medskip \\
Ende des Verses 17.16\\
Verse: 489, Buchstaben: 31, 486, 13929, Totalwerte: 1239, 35134, 1002293\\
\\
Wozu doch Geld in der Hand eines Toren, um Weisheit zu kaufen, da ihm doch der Verstand fehlt?\\
\newpage 
{\bf -- 17.17}\\
\medskip \\
\begin{tabular}{rrrrrrrrp{120mm}}
WV&WK&WB&ABK&ABB&ABV&AnzB&TW&Zahlencode \textcolor{red}{$\boldsymbol{Grundtext}$} Umschrift $|$"Ubersetzung(en)\\
1.&133.&3574.&487.&13930.&1.&3&52&2\_20\_30 \textcolor{red}{\textcjheb{lkb}} BKL $|$zu aller/zu jeder\\
2.&134.&3575.&490.&13933.&4.&2&470&70\_400 \textcolor{red}{\textcjheb{t`}} aT $|$Zeit\\
3.&135.&3576.&492.&13935.&6.&3&8&1\_5\_2 \textcolor{red}{\textcjheb{bh'}} AHB $|$liebt/(ist) (ein) Liebender\\
4.&136.&3577.&495.&13938.&9.&3&275&5\_200\_70 \textcolor{red}{\textcjheb{`rh}} HRa $|$der Freund\\
5.&137.&3578.&498.&13941.&12.&3&15&6\_1\_8 \textcolor{red}{\textcjheb{.h'w}} WAC $|$und (als) Bruder\\
6.&138.&3579.&501.&13944.&15.&4&325&30\_90\_200\_5 \textcolor{red}{\textcjheb{hr.sl}} L"sRH $|$f"ur die Drangsal/f"ur die Not\\
7.&139.&3580.&505.&13948.&19.&4&50&10\_6\_30\_4 \textcolor{red}{\textcjheb{dlwy}} JWLD $|$er wird geboren\\
\end{tabular}\medskip \\
Ende des Verses 17.17\\
Verse: 490, Buchstaben: 22, 508, 13951, Totalwerte: 1195, 36329, 1003488\\
\\
Der Freund liebt zu aller Zeit, und als Bruder f"ur die Drangsal wird er geboren.\\
\newpage 
{\bf -- 17.18}\\
\medskip \\
\begin{tabular}{rrrrrrrrp{120mm}}
WV&WK&WB&ABK&ABB&ABV&AnzB&TW&Zahlencode \textcolor{red}{$\boldsymbol{Grundtext}$} Umschrift $|$"Ubersetzung(en)\\
1.&140.&3581.&509.&13952.&1.&3&45&1\_4\_40 \textcolor{red}{\textcjheb{md'}} ADM $|$(ein) Mensch\\
2.&141.&3582.&512.&13955.&4.&3&268&8\_60\_200 \textcolor{red}{\textcjheb{rs.h}} CsR $|$un-/ermangelnd(er)\\
3.&142.&3583.&515.&13958.&7.&2&32&30\_2 \textcolor{red}{\textcjheb{bl}} LB $|$verst"andiger/Herz (Verstand)\\
4.&143.&3584.&517.&13960.&9.&4&576&400\_6\_100\_70 \textcolor{red}{\textcjheb{`qwt}} TWQa $|$ist wer einschl"agt/(ist) schlagend(er)\\
5.&144.&3585.&521.&13964.&13.&2&100&20\_80 \textcolor{red}{\textcjheb{pk}} KP $|$in die Hand\\
6.&145.&3586.&523.&13966.&15.&3&272&70\_200\_2 \textcolor{red}{\textcjheb{br`}} aRB $|$wer leistet/der ist leistendenr\\
7.&146.&3587.&526.&13969.&18.&4&277&70\_200\_2\_5 \textcolor{red}{\textcjheb{hbr`}} aRBH $|$B"urgschaft/mit einem Pfand\\
8.&147.&3588.&530.&13973.&22.&4&170&30\_80\_50\_10 \textcolor{red}{\textcjheb{ynpl}} LPNJ $|$gegen"uber/f"ur\\
9.&148.&3589.&534.&13977.&26.&4&281&200\_70\_5\_6 \textcolor{red}{\textcjheb{wh`r}} RaHW $|$seinem N"achsten/seinen N"achsten\\
\end{tabular}\medskip \\
Ende des Verses 17.18\\
Verse: 491, Buchstaben: 29, 537, 13980, Totalwerte: 2021, 38350, 1005509\\
\\
Ein unverst"andiger Mensch ist, wer in die Hand einschl"agt, wer B"urgschaft leistet gegen"uber seinem N"achsten.\\
\newpage 
{\bf -- 17.19}\\
\medskip \\
\begin{tabular}{rrrrrrrrp{120mm}}
WV&WK&WB&ABK&ABB&ABV&AnzB&TW&Zahlencode \textcolor{red}{$\boldsymbol{Grundtext}$} Umschrift $|$"Ubersetzung(en)\\
1.&149.&3590.&538.&13981.&1.&3&8&1\_5\_2 \textcolor{red}{\textcjheb{bh'}} AHB $|$wer liebt/(ein) Liebender\\
2.&150.&3591.&541.&13984.&4.&3&450&80\_300\_70 \textcolor{red}{\textcjheb{`+sp}} PSa $|$Zank/Missetat\\
3.&151.&3592.&544.&13987.&7.&3&8&1\_5\_2 \textcolor{red}{\textcjheb{bh'}} AHB $|$liebt/(ist) der liebt\\
4.&152.&3593.&547.&13990.&10.&3&135&40\_90\_5 \textcolor{red}{\textcjheb{h.sm}} M"sH $|$"Ubertretung/Zank\\
5.&153.&3594.&550.&13993.&13.&5&60&40\_3\_2\_10\_5 \textcolor{red}{\textcjheb{hybgm}} MGBJH $|$wer hoch macht/ein Hochmachender\\
6.&154.&3595.&555.&13998.&18.&4&494&80\_400\_8\_6 \textcolor{red}{\textcjheb{w.htp}} PTCW $|$seine T"ur/seinen Eingang\\
7.&155.&3596.&559.&14002.&22.&4&442&40\_2\_100\_300 \textcolor{red}{\textcjheb{+sqbm}} MBQS $|$sucht/(ist) suchend\\
8.&156.&3597.&563.&14006.&26.&3&502&300\_2\_200 \textcolor{red}{\textcjheb{rb+s}} SBR $|$Einsturz/(den) Zusammenbruch\\
\end{tabular}\medskip \\
Ende des Verses 17.19\\
Verse: 492, Buchstaben: 28, 565, 14008, Totalwerte: 2099, 40449, 1007608\\
\\
Wer Zank liebt, liebt "Ubertretung; wer seine T"ur hoch macht, sucht Einsturz.\\
\newpage 
{\bf -- 17.20}\\
\medskip \\
\begin{tabular}{rrrrrrrrp{120mm}}
WV&WK&WB&ABK&ABB&ABV&AnzB&TW&Zahlencode \textcolor{red}{$\boldsymbol{Grundtext}$} Umschrift $|$"Ubersetzung(en)\\
1.&157.&3598.&566.&14009.&1.&3&470&70\_100\_300 \textcolor{red}{\textcjheb{+sq`}} aQS $|$wer ist verkehrten/(ein) Verkehrter\\
2.&158.&3599.&569.&14012.&4.&2&32&30\_2 \textcolor{red}{\textcjheb{bl}} LB $|$(im) Herzen(s)\\
3.&159.&3600.&571.&14014.&6.&2&31&30\_1 \textcolor{red}{\textcjheb{'l}} LA $|$nicht\\
4.&160.&3601.&573.&14016.&8.&4&141&10\_40\_90\_1 \textcolor{red}{\textcjheb{'.smy}} JM"sA $|$wird finden/er findet\\
5.&161.&3602.&577.&14020.&12.&3&17&9\_6\_2 \textcolor{red}{\textcjheb{bw.t}} tWB $|$(das) Gute(s)\\
6.&162.&3603.&580.&14023.&15.&5&161&6\_50\_5\_80\_20 \textcolor{red}{\textcjheb{kphnw}} WNHPK $|$und wer sich windet/und wer ist sich drehend(er)\\
7.&163.&3604.&585.&14028.&20.&6&394&2\_30\_300\_6\_50\_6 \textcolor{red}{\textcjheb{wnw+slb}} BLSWNW $|$mit seiner Zunge\\
8.&164.&3605.&591.&14034.&26.&4&126&10\_80\_6\_30 \textcolor{red}{\textcjheb{lwpy}} JPWL $|$wird fallen/(er) f"allt\\
9.&165.&3606.&595.&14038.&30.&4&277&2\_200\_70\_5 \textcolor{red}{\textcjheb{h`rb}} BRaH $|$ins Ungl"uck/in Unheil\\
\end{tabular}\medskip \\
Ende des Verses 17.20\\
Verse: 493, Buchstaben: 33, 598, 14041, Totalwerte: 1649, 42098, 1009257\\
\\
Wer verkehrten Herzens ist, wird das Gute nicht finden; und wer sich mit seiner Zunge windet, wird ins Ungl"uck fallen.\\
\newpage 
{\bf -- 17.21}\\
\medskip \\
\begin{tabular}{rrrrrrrrp{120mm}}
WV&WK&WB&ABK&ABB&ABV&AnzB&TW&Zahlencode \textcolor{red}{$\boldsymbol{Grundtext}$} Umschrift $|$"Ubersetzung(en)\\
1.&166.&3607.&599.&14042.&1.&3&44&10\_30\_4 \textcolor{red}{\textcjheb{dly}} JLD $|$wer zeugt\\
2.&167.&3608.&602.&14045.&4.&4&120&20\_60\_10\_30 \textcolor{red}{\textcjheb{lysk}} KsJL $|$(einen) Toren\\
3.&168.&3609.&606.&14049.&8.&5&444&30\_400\_6\_3\_5 \textcolor{red}{\textcjheb{hgwtl}} LTWGH $|$zum Kummer wird es/zum Kummer (gereicht es)\\
4.&169.&3610.&611.&14054.&13.&2&36&30\_6 \textcolor{red}{\textcjheb{wl}} LW $|$dem/ihm\\
5.&170.&3611.&613.&14056.&15.&3&37&6\_30\_1 \textcolor{red}{\textcjheb{'lw}} WLA $|$und nicht\\
6.&171.&3612.&616.&14059.&18.&4&358&10\_300\_40\_8 \textcolor{red}{\textcjheb{.hm+sy}} JSMC $|$hat Freude/er (=es) freut sich\\
7.&172.&3613.&620.&14063.&22.&3&13&1\_2\_10 \textcolor{red}{\textcjheb{yb'}} ABJ $|$der Vater\\
8.&173.&3614.&623.&14066.&25.&3&82&50\_2\_30 \textcolor{red}{\textcjheb{lbn}} NBL $|$eines Narren/an einem Narren\\
\end{tabular}\medskip \\
Ende des Verses 17.21\\
Verse: 494, Buchstaben: 27, 625, 14068, Totalwerte: 1134, 43232, 1010391\\
\\
Wer einen Toren zeugt, dem wird es zum Kummer, und der Vater eines Narren hat keine Freude.\\
\newpage 
{\bf -- 17.22}\\
\medskip \\
\begin{tabular}{rrrrrrrrp{120mm}}
WV&WK&WB&ABK&ABB&ABV&AnzB&TW&Zahlencode \textcolor{red}{$\boldsymbol{Grundtext}$} Umschrift $|$"Ubersetzung(en)\\
1.&174.&3615.&626.&14069.&1.&2&32&30\_2 \textcolor{red}{\textcjheb{bl}} LB $|$(ein) Herz\\
2.&175.&3616.&628.&14071.&3.&3&348&300\_40\_8 \textcolor{red}{\textcjheb{.hm+s}} SMC $|$fr"ohliches\\
3.&176.&3617.&631.&14074.&6.&4&31&10\_10\_9\_2 \textcolor{red}{\textcjheb{b.tyy}} JJtB $|$bringt/er (=es) bewirkt\\
4.&177.&3618.&635.&14078.&10.&3&13&3\_5\_5 \textcolor{red}{\textcjheb{hhg}} GHH $|$gute Besserung/Heilung\\
5.&178.&3619.&638.&14081.&13.&4&220&6\_200\_6\_8 \textcolor{red}{\textcjheb{.hwrw}} WRWC $|$aber ein Geist/und ein Geist\\
6.&179.&3620.&642.&14085.&17.&4&76&50\_20\_1\_5 \textcolor{red}{\textcjheb{h'kn}} NKAH $|$zergeschlagener/verzagter\\
7.&180.&3621.&646.&14089.&21.&4&712&400\_10\_2\_300 \textcolor{red}{\textcjheb{+sbyt}} TJBS $|$vertrocknet/sie (=es) trocknet aus\\
8.&181.&3622.&650.&14093.&25.&3&243&3\_200\_40 \textcolor{red}{\textcjheb{mrg}} GRM $|$das Gebein\\
\end{tabular}\medskip \\
Ende des Verses 17.22\\
Verse: 495, Buchstaben: 27, 652, 14095, Totalwerte: 1675, 44907, 1012066\\
\\
Ein fr"ohliches Herz bringt gute Besserung, aber ein zerschlagener Geist vertrocknet das Gebein.\\
\newpage 
{\bf -- 17.23}\\
\medskip \\
\begin{tabular}{rrrrrrrrp{120mm}}
WV&WK&WB&ABK&ABB&ABV&AnzB&TW&Zahlencode \textcolor{red}{$\boldsymbol{Grundtext}$} Umschrift $|$"Ubersetzung(en)\\
1.&182.&3623.&653.&14096.&1.&3&312&300\_8\_4 \textcolor{red}{\textcjheb{d.h+s}} SCD $|$ein Geschenk/Bestechung\\
2.&183.&3624.&656.&14099.&4.&4&158&40\_8\_10\_100 \textcolor{red}{\textcjheb{qy.hm}} MCJQ $|$aus dem Busen/aus einem Busen\\
3.&184.&3625.&660.&14103.&8.&3&570&200\_300\_70 \textcolor{red}{\textcjheb{`+sr}} RSa $|$der Gesetzlose/(ein) Frevler\\
4.&185.&3626.&663.&14106.&11.&3&118&10\_100\_8 \textcolor{red}{\textcjheb{.hqy}} JQC $|$(er) nimmt (an)\\
5.&186.&3627.&666.&14109.&14.&5&450&30\_5\_9\_6\_400 \textcolor{red}{\textcjheb{tw.thl}} LHtWT $|$um zu beugen/beugend\\
6.&187.&3628.&671.&14114.&19.&5&615&1\_200\_8\_6\_400 \textcolor{red}{\textcjheb{tw.hr'}} ARCWT $|$(die) Pfade\\
7.&188.&3629.&676.&14119.&24.&4&429&40\_300\_80\_9 \textcolor{red}{\textcjheb{.tp+sm}} MSPt $|$des Rechts\\
\end{tabular}\medskip \\
Ende des Verses 17.23\\
Verse: 496, Buchstaben: 27, 679, 14122, Totalwerte: 2652, 47559, 1014718\\
\\
Der Gesetzlose nimmt ein Geschenk aus dem Busen, um die Pfade des Rechts zu beugen.\\
\newpage 
{\bf -- 17.24}\\
\medskip \\
\begin{tabular}{rrrrrrrrp{120mm}}
WV&WK&WB&ABK&ABB&ABV&AnzB&TW&Zahlencode \textcolor{red}{$\boldsymbol{Grundtext}$} Umschrift $|$"Ubersetzung(en)\\
1.&189.&3630.&680.&14123.&1.&2&401&1\_400 \textcolor{red}{\textcjheb{t'}} AT $|$**\\
2.&190.&3631.&682.&14125.&3.&3&140&80\_50\_10 \textcolor{red}{\textcjheb{ynp}} PNJ $|$vor dem Angesicht/das Antlitz\\
3.&191.&3632.&685.&14128.&6.&4&102&40\_2\_10\_50 \textcolor{red}{\textcjheb{nybm}} MBJN $|$des Verst"andigen/(eines) Verst"andigen\\
4.&192.&3633.&689.&14132.&10.&4&73&8\_20\_40\_5 \textcolor{red}{\textcjheb{hmk.h}} CKMH $|$ist Weisheit/(bekundet) Weisheit\\
5.&193.&3634.&693.&14136.&14.&5&146&6\_70\_10\_50\_10 \textcolor{red}{\textcjheb{yny`w}} WaJNJ $|$aber die Augen/und die Augen\\
6.&194.&3635.&698.&14141.&19.&4&120&20\_60\_10\_30 \textcolor{red}{\textcjheb{lysk}} KsJL $|$des Toren/(von) einem Toren\\
7.&195.&3636.&702.&14145.&23.&4&197&2\_100\_90\_5 \textcolor{red}{\textcjheb{h.sqb}} BQ"sH $|$sind am Ende/(schweifen) an das Ende\\
8.&196.&3637.&706.&14149.&27.&3&291&1\_200\_90 \textcolor{red}{\textcjheb{.sr'}} AR"s $|$der Erde\\
\end{tabular}\medskip \\
Ende des Verses 17.24\\
Verse: 497, Buchstaben: 29, 708, 14151, Totalwerte: 1470, 49029, 1016188\\
\\
Vor dem Angesicht des Verst"andigen ist Weisheit, aber die Augen des Toren sind am Ende der Erde.\\
\newpage 
{\bf -- 17.25}\\
\medskip \\
\begin{tabular}{rrrrrrrrp{120mm}}
WV&WK&WB&ABK&ABB&ABV&AnzB&TW&Zahlencode \textcolor{red}{$\boldsymbol{Grundtext}$} Umschrift $|$"Ubersetzung(en)\\
1.&197.&3638.&709.&14152.&1.&3&150&20\_70\_60 \textcolor{red}{\textcjheb{s`k}} Kas $|$ein Gram/Verdruss\\
2.&198.&3639.&712.&14155.&4.&5&49&30\_1\_2\_10\_6 \textcolor{red}{\textcjheb{wyb'l}} LABJW $|$f"ur seinen Vater\\
3.&199.&3640.&717.&14160.&9.&2&52&2\_50 \textcolor{red}{\textcjheb{nb}} BN $|$(ist) (ein) Sohn\\
4.&200.&3641.&719.&14162.&11.&4&120&20\_60\_10\_30 \textcolor{red}{\textcjheb{lysk}} KsJL $|$t"orichter\\
5.&201.&3642.&723.&14166.&15.&4&286&6\_40\_40\_200 \textcolor{red}{\textcjheb{rmmw}} WMMR $|$und Bitterkeit/und Bitternis\\
6.&202.&3643.&727.&14170.&19.&7&486&30\_10\_6\_30\_4\_400\_6 \textcolor{red}{\textcjheb{wtdlwyl}} LJWLDTW $|$f"ur die welche ihn geboren/f"ur seine Geb"arerin\\
\end{tabular}\medskip \\
Ende des Verses 17.25\\
Verse: 498, Buchstaben: 25, 733, 14176, Totalwerte: 1143, 50172, 1017331\\
\\
Ein t"orichter Sohn ist ein Gram f"ur seinen Vater, und Bitterkeit f"ur die, welche ihn geboren.\\
\newpage 
{\bf -- 17.26}\\
\medskip \\
\begin{tabular}{rrrrrrrrp{120mm}}
WV&WK&WB&ABK&ABB&ABV&AnzB&TW&Zahlencode \textcolor{red}{$\boldsymbol{Grundtext}$} Umschrift $|$"Ubersetzung(en)\\
1.&203.&3644.&734.&14177.&1.&2&43&3\_40 \textcolor{red}{\textcjheb{mg}} GM $|$auch\\
2.&204.&3645.&736.&14179.&3.&4&426&70\_50\_6\_300 \textcolor{red}{\textcjheb{+swn`}} aNWS $|$zu bestrafen/ein Strafen\\
3.&205.&3646.&740.&14183.&7.&5&234&30\_90\_4\_10\_100 \textcolor{red}{\textcjheb{qyd.sl}} L"sDJQ $|$den Gerechten/den Rechtschaffenen\\
4.&206.&3647.&745.&14188.&12.&2&31&30\_1 \textcolor{red}{\textcjheb{'l}} LA $|$nicht\\
5.&207.&3648.&747.&14190.&14.&3&17&9\_6\_2 \textcolor{red}{\textcjheb{bw.t}} tWB $|$gut (ist)\\
6.&208.&3649.&750.&14193.&17.&5&461&30\_5\_20\_6\_400 \textcolor{red}{\textcjheb{twkhl}} LHKWT $|$zu schlagen\\
7.&209.&3650.&755.&14198.&22.&6&116&50\_4\_10\_2\_10\_40 \textcolor{red}{\textcjheb{mybydn}} NDJBJM $|$Edle\\
8.&210.&3651.&761.&14204.&28.&2&100&70\_30 \textcolor{red}{\textcjheb{l`}} aL $|$um willen/gegen\\
9.&211.&3652.&763.&14206.&30.&3&510&10\_300\_200 \textcolor{red}{\textcjheb{r+sy}} JSR $|$der Geradheit/das Rechte\\
\end{tabular}\medskip \\
Ende des Verses 17.26\\
Verse: 499, Buchstaben: 32, 765, 14208, Totalwerte: 1938, 52110, 1019269\\
\\
Auch den Gerechten zu bestrafen, ist nicht gut, Edle zu schlagen um der Geradheit willen.\\
\newpage 
{\bf -- 17.27}\\
\medskip \\
\begin{tabular}{rrrrrrrrp{120mm}}
WV&WK&WB&ABK&ABB&ABV&AnzB&TW&Zahlencode \textcolor{red}{$\boldsymbol{Grundtext}$} Umschrift $|$"Ubersetzung(en)\\
1.&212.&3653.&766.&14209.&1.&4&334&8\_6\_300\_20 \textcolor{red}{\textcjheb{k+sw.h}} CWSK $|$wer zur"uckh"alt/ein Zur"uckhaltender\\
2.&213.&3654.&770.&14213.&5.&5&257&1\_40\_200\_10\_6 \textcolor{red}{\textcjheb{wyrm'}} AMRJW $|$seine Worte\\
3.&214.&3655.&775.&14218.&10.&4&90&10\_6\_4\_70 \textcolor{red}{\textcjheb{`dwy}} JWDa $|$besitzt/bekundet\\
4.&215.&3656.&779.&14222.&14.&3&474&4\_70\_400 \textcolor{red}{\textcjheb{t`d}} DaT $|$Erkenntnis/Einsicht\\
5.&216.&3657.&782.&14225.&17.&3&306&6\_100\_200 \textcolor{red}{\textcjheb{rqw}} WQR $|$und wer ist k"uhlen/und k"uhlen\\
6.&217.&3658.&785.&14228.&20.&3&214&200\_6\_8 \textcolor{red}{\textcjheb{.hwr}} RWC $|$Geist(es)\\
7.&218.&3659.&788.&14231.&23.&3&311&1\_10\_300 \textcolor{red}{\textcjheb{+sy'}} AJS $|$ist ein Mann/(bewahrt ein) Mann\\
8.&219.&3660.&791.&14234.&26.&5&463&400\_2\_6\_50\_5 \textcolor{red}{\textcjheb{hnwbt}} TBWNH $|$verst"andiger/(mit) Vernunft\\
\end{tabular}\medskip \\
Ende des Verses 17.27\\
Verse: 500, Buchstaben: 30, 795, 14238, Totalwerte: 2449, 54559, 1021718\\
\\
Wer seine Worte zur"uckh"alt, besitzt Erkenntnis; und wer k"uhlen Geistes ist, ist ein verst"andiger Mann.\\
\newpage 
{\bf -- 17.28}\\
\medskip \\
\begin{tabular}{rrrrrrrrp{120mm}}
WV&WK&WB&ABK&ABB&ABV&AnzB&TW&Zahlencode \textcolor{red}{$\boldsymbol{Grundtext}$} Umschrift $|$"Ubersetzung(en)\\
1.&220.&3661.&796.&14239.&1.&2&43&3\_40 \textcolor{red}{\textcjheb{mg}} GM $|$auch\\
2.&221.&3662.&798.&14241.&3.&4&47&1\_6\_10\_30 \textcolor{red}{\textcjheb{lyw'}} AWJL $|$(ein) Narr\\
3.&222.&3663.&802.&14245.&7.&5&558&40\_8\_200\_10\_300 \textcolor{red}{\textcjheb{+syr.hm}} MCRJS $|$der schweigt/schweigender\\
4.&223.&3664.&807.&14250.&12.&3&68&8\_20\_40 \textcolor{red}{\textcjheb{mk.h}} CKM $|$f"ur weise/(als) weiser\\
5.&224.&3665.&810.&14253.&15.&4&320&10\_8\_300\_2 \textcolor{red}{\textcjheb{b+s.hy}} JCSB $|$wird gehalten/(er) wird geachtet\\
6.&225.&3666.&814.&14257.&19.&3&50&1\_9\_40 \textcolor{red}{\textcjheb{m.t'}} AtM $|$wer verschlie"st/ein Verschlie"sender\\
7.&226.&3667.&817.&14260.&22.&5&796&300\_80\_400\_10\_6 \textcolor{red}{\textcjheb{wytp+s}} SPTJW $|$seine Lippen\\
8.&227.&3668.&822.&14265.&27.&4&108&50\_2\_6\_50 \textcolor{red}{\textcjheb{nwbn}} NBWN $|$f"ur verst"andig/(gilt als) einsichtig(er)\\
\end{tabular}\medskip \\
Ende des Verses 17.28\\
Verse: 501, Buchstaben: 30, 825, 14268, Totalwerte: 1990, 56549, 1023708\\
\\
Auch ein Narr, der schweigt, wird f"ur weise gehalten, f"ur verst"andig, wer seine Lippen verschlie"st.\\
\\
{\bf Ende des Kapitels 17}\\
\newpage 
{\bf -- 18.1}\\
\medskip \\
\begin{tabular}{rrrrrrrrp{120mm}}
WV&WK&WB&ABK&ABB&ABV&AnzB&TW&Zahlencode \textcolor{red}{$\boldsymbol{Grundtext}$} Umschrift $|$"Ubersetzung(en)\\
1.&1.&3669.&1.&14269.&1.&5&442&30\_400\_1\_6\_5 \textcolor{red}{\textcjheb{hw'tl}} LTAWH $|$nach einem Gel"ust/dem Eigenwillen\\
2.&2.&3670.&6.&14274.&6.&4&412&10\_2\_100\_300 \textcolor{red}{\textcjheb{+sqby}} JBQS $|$trachtet/(er) sucht\\
3.&3.&3671.&10.&14278.&10.&4&334&50\_80\_200\_4 \textcolor{red}{\textcjheb{drpn}} NPRD $|$wer sich absondert/ein sich Absondernder\\
4.&4.&3672.&14.&14282.&14.&3&52&2\_20\_30 \textcolor{red}{\textcjheb{lkb}} BKL $|$gegen alle/mit aller\\
5.&5.&3673.&17.&14285.&17.&5&721&400\_6\_300\_10\_5 \textcolor{red}{\textcjheb{hy+swt}} TWSJH $|$Einsicht/Kraft\\
6.&6.&3674.&22.&14290.&22.&5&513&10\_400\_3\_30\_70 \textcolor{red}{\textcjheb{`lgty}} JTGLa $|$geht er heftig an/er bricht los\\
\end{tabular}\medskip \\
Ende des Verses 18.1\\
Verse: 502, Buchstaben: 26, 26, 14294, Totalwerte: 2474, 2474, 1026182\\
\\
Wer sich absondert, trachtet nach einem Gel"ust; gegen alle Einsicht geht er heftig an.\\
\newpage 
{\bf -- 18.2}\\
\medskip \\
\begin{tabular}{rrrrrrrrp{120mm}}
WV&WK&WB&ABK&ABB&ABV&AnzB&TW&Zahlencode \textcolor{red}{$\boldsymbol{Grundtext}$} Umschrift $|$"Ubersetzung(en)\\
1.&7.&3675.&27.&14295.&1.&2&31&30\_1 \textcolor{red}{\textcjheb{'l}} LA $|$nicht\\
2.&8.&3676.&29.&14297.&3.&4&188&10\_8\_80\_90 \textcolor{red}{\textcjheb{.sp.hy}} JCP"s $|$hat Lust/er (=es) findet Gefallen\\
3.&9.&3677.&33.&14301.&7.&4&120&20\_60\_10\_30 \textcolor{red}{\textcjheb{lysk}} KsJL $|$der Tor/(ein) Tor\\
4.&10.&3678.&37.&14305.&11.&6&465&2\_400\_2\_6\_50\_5 \textcolor{red}{\textcjheb{hnwbtb}} BTBWNH $|$an Verst"andnis/an Einsicht\\
5.&11.&3679.&43.&14311.&17.&2&30&20\_10 \textcolor{red}{\textcjheb{yk}} KJ $|$sondern\\
6.&12.&3680.&45.&14313.&19.&2&41&1\_40 \textcolor{red}{\textcjheb{m'}} AM $|$nur\\
7.&13.&3681.&47.&14315.&21.&7&846&2\_5\_400\_3\_30\_6\_400 \textcolor{red}{\textcjheb{twlgthb}} BHTGLWT $|$daran dass sich offenbare/am Offenbaren\\
8.&14.&3682.&54.&14322.&28.&3&38&30\_2\_6 \textcolor{red}{\textcjheb{wbl}} LBW $|$sein Herz\\
\end{tabular}\medskip \\
Ende des Verses 18.2\\
Verse: 503, Buchstaben: 30, 56, 14324, Totalwerte: 1759, 4233, 1027941\\
\\
Der Tor hat keine Lust an Verst"andnis, sondern nur daran, da"s sein Herz sich offenbare.\\
\newpage 
{\bf -- 18.3}\\
\medskip \\
\begin{tabular}{rrrrrrrrp{120mm}}
WV&WK&WB&ABK&ABB&ABV&AnzB&TW&Zahlencode \textcolor{red}{$\boldsymbol{Grundtext}$} Umschrift $|$"Ubersetzung(en)\\
1.&15.&3683.&57.&14325.&1.&4&11&2\_2\_6\_1 \textcolor{red}{\textcjheb{'wbb}} BBWA $|$wenn kommt\\
2.&16.&3684.&61.&14329.&5.&3&570&200\_300\_70 \textcolor{red}{\textcjheb{`+sr}} RSa $|$ein Gesetzloser/(der) Frevler\\
3.&17.&3685.&64.&14332.&8.&2&3&2\_1 \textcolor{red}{\textcjheb{'b}} BA $|$so kommt/er (=es) kommt\\
4.&18.&3686.&66.&14334.&10.&2&43&3\_40 \textcolor{red}{\textcjheb{mg}} GM $|$auch\\
5.&19.&3687.&68.&14336.&12.&3&15&2\_6\_7 \textcolor{red}{\textcjheb{zwb}} BWZ $|$Verachtung\\
6.&20.&3688.&71.&14339.&15.&3&116&6\_70\_40 \textcolor{red}{\textcjheb{m`w}} WaM $|$und mit\\
7.&21.&3689.&74.&14342.&18.&4&186&100\_30\_6\_50 \textcolor{red}{\textcjheb{nwlq}} QLWN $|$(der) Schande\\
8.&22.&3690.&78.&14346.&22.&4&293&8\_200\_80\_5 \textcolor{red}{\textcjheb{hpr.h}} CRPH $|$kommt Schm"ahung/Schmach\\
\end{tabular}\medskip \\
Ende des Verses 18.3\\
Verse: 504, Buchstaben: 25, 81, 14349, Totalwerte: 1237, 5470, 1029178\\
\\
Wenn ein Gesetzloser kommt, so kommt auch Verachtung; und mit der Schande kommt Schm"ahung.\\
\newpage 
{\bf -- 18.4}\\
\medskip \\
\begin{tabular}{rrrrrrrrp{120mm}}
WV&WK&WB&ABK&ABB&ABV&AnzB&TW&Zahlencode \textcolor{red}{$\boldsymbol{Grundtext}$} Umschrift $|$"Ubersetzung(en)\\
1.&23.&3691.&82.&14350.&1.&3&90&40\_10\_40 \textcolor{red}{\textcjheb{mym}} MJM $|$(wie) Wasser\\
2.&24.&3692.&85.&14353.&4.&5&260&70\_40\_100\_10\_40 \textcolor{red}{\textcjheb{myqm`}} aMQJM $|$tiefe\\
3.&25.&3693.&90.&14358.&9.&4&216&4\_2\_200\_10 \textcolor{red}{\textcjheb{yrbd}} DBRJ $|$(sind) die Worte\\
4.&26.&3694.&94.&14362.&13.&2&90&80\_10 \textcolor{red}{\textcjheb{yp}} PJ $|$aus dem Mund/des Mundes\\
5.&27.&3695.&96.&14364.&15.&3&311&1\_10\_300 \textcolor{red}{\textcjheb{+sy'}} AJS $|$(eines) Mannes\\
6.&28.&3696.&99.&14367.&18.&3&88&50\_8\_30 \textcolor{red}{\textcjheb{l.hn}} NCL $|$(wie) (ein) Bach\\
7.&29.&3697.&102.&14370.&21.&3&122&50\_2\_70 \textcolor{red}{\textcjheb{`bn}} NBa $|$sprudelnder\\
8.&30.&3698.&105.&14373.&24.&4&346&40\_100\_6\_200 \textcolor{red}{\textcjheb{rwqm}} MQWR $|$ein Born/(ist) ein Quell\\
9.&31.&3699.&109.&14377.&28.&4&73&8\_20\_40\_5 \textcolor{red}{\textcjheb{hmk.h}} CKMH $|$der Weisheit\\
\end{tabular}\medskip \\
Ende des Verses 18.4\\
Verse: 505, Buchstaben: 31, 112, 14380, Totalwerte: 1596, 7066, 1030774\\
\\
Die Worte aus dem Munde eines Mannes sind tiefe Wasser, ein sprudelnder Bach, ein Born der Weisheit.\\
\newpage 
{\bf -- 18.5}\\
\medskip \\
\begin{tabular}{rrrrrrrrp{120mm}}
WV&WK&WB&ABK&ABB&ABV&AnzB&TW&Zahlencode \textcolor{red}{$\boldsymbol{Grundtext}$} Umschrift $|$"Ubersetzung(en)\\
1.&32.&3700.&113.&14381.&1.&3&701&300\_1\_400 \textcolor{red}{\textcjheb{t'+s}} SAT $|$die Person/(ein) Erheben\\
2.&33.&3701.&116.&14384.&4.&3&140&80\_50\_10 \textcolor{red}{\textcjheb{ynp}} PNJ $|$anzusehen/Gesichter\\
3.&34.&3702.&119.&14387.&7.&3&570&200\_300\_70 \textcolor{red}{\textcjheb{`+sr}} RSa $|$des Gesetzlosen/(eines) Frevlers\\
4.&35.&3703.&122.&14390.&10.&2&31&30\_1 \textcolor{red}{\textcjheb{'l}} LA $|$nicht\\
5.&36.&3704.&124.&14392.&12.&3&17&9\_6\_2 \textcolor{red}{\textcjheb{bw.t}} tWB $|$(es) (ist) gut\\
6.&37.&3705.&127.&14395.&15.&5&450&30\_5\_9\_6\_400 \textcolor{red}{\textcjheb{tw.thl}} LHtWT $|$(um) zu beugen\\
7.&38.&3706.&132.&14400.&20.&4&204&90\_4\_10\_100 \textcolor{red}{\textcjheb{qyd.s}} "sDJQ $|$den Gerechten/(einen) Rechtschaffenen\\
8.&39.&3707.&136.&14404.&24.&5&431&2\_40\_300\_80\_9 \textcolor{red}{\textcjheb{.tp+smb}} BMSPt $|$im Gericht\\
\end{tabular}\medskip \\
Ende des Verses 18.5\\
Verse: 506, Buchstaben: 28, 140, 14408, Totalwerte: 2544, 9610, 1033318\\
\\
Es ist nicht gut, die Person des Gesetzlosen anzusehen, um den Gerechten zu beugen im Gericht.\\
\newpage 
{\bf -- 18.6}\\
\medskip \\
\begin{tabular}{rrrrrrrrp{120mm}}
WV&WK&WB&ABK&ABB&ABV&AnzB&TW&Zahlencode \textcolor{red}{$\boldsymbol{Grundtext}$} Umschrift $|$"Ubersetzung(en)\\
1.&40.&3708.&141.&14409.&1.&4&790&300\_80\_400\_10 \textcolor{red}{\textcjheb{ytp+s}} SPTJ $|$die Lippen\\
2.&41.&3709.&145.&14413.&5.&4&120&20\_60\_10\_30 \textcolor{red}{\textcjheb{lysk}} KsJL $|$des Toren/(eines) Toren\\
3.&42.&3710.&149.&14417.&9.&4&19&10\_2\_1\_6 \textcolor{red}{\textcjheb{w'by}} JBAW $|$geraten/(sie) kommen\\
4.&43.&3711.&153.&14421.&13.&4&214&2\_200\_10\_2 \textcolor{red}{\textcjheb{byrb}} BRJB $|$in Streit/mit Zank\\
5.&44.&3712.&157.&14425.&17.&4&102&6\_80\_10\_6 \textcolor{red}{\textcjheb{wypw}} WPJW $|$und sein Mund\\
6.&45.&3713.&161.&14429.&21.&7&551&30\_40\_5\_30\_40\_6\_400 \textcolor{red}{\textcjheb{twmlhml}} LMHLMWT $|$nach Schl"agen\\
7.&46.&3714.&168.&14436.&28.&4&311&10\_100\_200\_1 \textcolor{red}{\textcjheb{'rqy}} JQRA $|$(er) ruft\\
\end{tabular}\medskip \\
Ende des Verses 18.6\\
Verse: 507, Buchstaben: 31, 171, 14439, Totalwerte: 2107, 11717, 1035425\\
\\
Die Lippen des Toren geraten in Streit, und sein Mund ruft nach Schl"agen.\\
\newpage 
{\bf -- 18.7}\\
\medskip \\
\begin{tabular}{rrrrrrrrp{120mm}}
WV&WK&WB&ABK&ABB&ABV&AnzB&TW&Zahlencode \textcolor{red}{$\boldsymbol{Grundtext}$} Umschrift $|$"Ubersetzung(en)\\
1.&47.&3715.&172.&14440.&1.&2&90&80\_10 \textcolor{red}{\textcjheb{yp}} PJ $|$der Mund\\
2.&48.&3716.&174.&14442.&3.&4&120&20\_60\_10\_30 \textcolor{red}{\textcjheb{lysk}} KsJL $|$des Toren/(eines) Toren\\
3.&49.&3717.&178.&14446.&7.&4&453&40\_8\_400\_5 \textcolor{red}{\textcjheb{ht.hm}} MCTH $|$wird zum Untergang/(ist) Verderben\\
4.&50.&3718.&182.&14450.&11.&2&36&30\_6 \textcolor{red}{\textcjheb{wl}} LW $|$ihm/f"ur ihn\\
5.&51.&3719.&184.&14452.&13.&6&802&6\_300\_80\_400\_10\_6 \textcolor{red}{\textcjheb{wytp+sw}} WSPTJW $|$und seine Lippen\\
6.&52.&3720.&190.&14458.&19.&4&446&40\_6\_100\_300 \textcolor{red}{\textcjheb{+sqwm}} MWQS $|$sind der Fallstrick/(sind) eine Falle\\
7.&53.&3721.&194.&14462.&23.&4&436&50\_80\_300\_6 \textcolor{red}{\textcjheb{w+spn}} NPSW $|$seiner Seele/(f"ur) seine Seele\\
\end{tabular}\medskip \\
Ende des Verses 18.7\\
Verse: 508, Buchstaben: 26, 197, 14465, Totalwerte: 2383, 14100, 1037808\\
\\
Der Mund des Toren wird ihm zum Untergang, und seine Lippen sind der Fallstrick seiner Seele.\\
\newpage 
{\bf -- 18.8}\\
\medskip \\
\begin{tabular}{rrrrrrrrp{120mm}}
WV&WK&WB&ABK&ABB&ABV&AnzB&TW&Zahlencode \textcolor{red}{$\boldsymbol{Grundtext}$} Umschrift $|$"Ubersetzung(en)\\
1.&54.&3722.&198.&14466.&1.&4&216&4\_2\_200\_10 \textcolor{red}{\textcjheb{yrbd}} DBRJ $|$(die) Worte\\
2.&55.&3723.&202.&14470.&5.&4&303&50\_200\_3\_50 \textcolor{red}{\textcjheb{ngrn}} NRGN $|$des Ohrenbl"asers/eines Verleumders\\
3.&56.&3724.&206.&14474.&9.&8&585&20\_40\_400\_30\_5\_40\_10\_40 \textcolor{red}{\textcjheb{mymhltmk}} KMTLHMJM $|$(sind) wie Leckerbissen\\
4.&57.&3725.&214.&14482.&17.&3&51&6\_5\_40 \textcolor{red}{\textcjheb{mhw}} WHM $|$und sie\\
5.&58.&3726.&217.&14485.&20.&4&220&10\_200\_4\_6 \textcolor{red}{\textcjheb{wdry}} JRDW $|$dringen hinab/(sie) gehen hinab\\
6.&59.&3727.&221.&14489.&24.&4&222&8\_4\_200\_10 \textcolor{red}{\textcjheb{yrd.h}} CDRJ $|$in das Innerste/(zu den) Kammern\\
7.&60.&3728.&225.&14493.&28.&3&61&2\_9\_50 \textcolor{red}{\textcjheb{n.tb}} BtN $|$des Leibes\\
\end{tabular}\medskip \\
Ende des Verses 18.8\\
Verse: 509, Buchstaben: 30, 227, 14495, Totalwerte: 1658, 15758, 1039466\\
\\
Die Worte des Ohrenbl"asers sind wie Leckerbissen, und sie dringen hinab in das Innerste des Leibes.\\
\newpage 
{\bf -- 18.9}\\
\medskip \\
\begin{tabular}{rrrrrrrrp{120mm}}
WV&WK&WB&ABK&ABB&ABV&AnzB&TW&Zahlencode \textcolor{red}{$\boldsymbol{Grundtext}$} Umschrift $|$"Ubersetzung(en)\\
1.&61.&3729.&228.&14496.&1.&2&43&3\_40 \textcolor{red}{\textcjheb{mg}} GM $|$auch\\
2.&62.&3730.&230.&14498.&3.&5&725&40\_400\_200\_80\_5 \textcolor{red}{\textcjheb{hprtm}} MTRPH $|$wer sich l"assig zeigt/ein l"assig sich Zeigender\\
3.&63.&3731.&235.&14503.&8.&7&499&2\_40\_30\_1\_20\_400\_6 \textcolor{red}{\textcjheb{wtk'lmb}} BMLAKTW $|$in seiner Arbeit/bei seiner Arbeit\\
4.&64.&3732.&242.&14510.&15.&2&9&1\_8 \textcolor{red}{\textcjheb{.h'}} AC $|$(ein) Bruder\\
5.&65.&3733.&244.&14512.&17.&3&12&5\_6\_1 \textcolor{red}{\textcjheb{'wh}} HWA $|$ist/(ist) er\\
6.&66.&3734.&247.&14515.&20.&4&132&30\_2\_70\_30 \textcolor{red}{\textcjheb{l`bl}} LBaL $|$des/f"ur einen Herren\\
7.&67.&3735.&251.&14519.&24.&5&758&40\_300\_8\_10\_400 \textcolor{red}{\textcjheb{ty.h+sm}} MSCJT $|$Verderbers/verderbenden\\
\end{tabular}\medskip \\
Ende des Verses 18.9\\
Verse: 510, Buchstaben: 28, 255, 14523, Totalwerte: 2178, 17936, 1041644\\
\\
Auch wer sich l"assig zeigt in seiner Arbeit, ist ein Bruder des Verderbers.\\
\newpage 
{\bf -- 18.10}\\
\medskip \\
\begin{tabular}{rrrrrrrrp{120mm}}
WV&WK&WB&ABK&ABB&ABV&AnzB&TW&Zahlencode \textcolor{red}{$\boldsymbol{Grundtext}$} Umschrift $|$"Ubersetzung(en)\\
1.&68.&3736.&256.&14524.&1.&4&77&40\_3\_4\_30 \textcolor{red}{\textcjheb{ldgm}} MGDL $|$(ein) Turm\\
2.&69.&3737.&260.&14528.&5.&2&77&70\_7 \textcolor{red}{\textcjheb{z`}} aZ $|$starker/(der) St"arke\\
3.&70.&3738.&262.&14530.&7.&2&340&300\_40 \textcolor{red}{\textcjheb{m+s}} SM $|$(ist) der Name\\
4.&71.&3739.&264.&14532.&9.&4&26&10\_5\_6\_5 \textcolor{red}{\textcjheb{hwhy}} JHWH $|$Jahwe(s)\\
5.&72.&3740.&268.&14536.&13.&2&8&2\_6 \textcolor{red}{\textcjheb{wb}} BW $|$dahin/zu ihm\\
6.&73.&3741.&270.&14538.&15.&4&306&10\_200\_6\_90 \textcolor{red}{\textcjheb{.swry}} JRW"s $|$(er) l"auft\\
7.&74.&3742.&274.&14542.&19.&4&204&90\_4\_10\_100 \textcolor{red}{\textcjheb{qyd.s}} "sDJQ $|$der Gerechte/(der) Rechtschaffene\\
8.&75.&3743.&278.&14546.&23.&5&361&6\_50\_300\_3\_2 \textcolor{red}{\textcjheb{bg+snw}} WNSGB $|$und ist in Sicherheit/und er ist geborgen\\
\end{tabular}\medskip \\
Ende des Verses 18.10\\
Verse: 511, Buchstaben: 27, 282, 14550, Totalwerte: 1399, 19335, 1043043\\
\\
Der Name Jahwes ist ein starker Turm; der Gerechte l"auft dahin und ist in Sicherheit.\\
\newpage 
{\bf -- 18.11}\\
\medskip \\
\begin{tabular}{rrrrrrrrp{120mm}}
WV&WK&WB&ABK&ABB&ABV&AnzB&TW&Zahlencode \textcolor{red}{$\boldsymbol{Grundtext}$} Umschrift $|$"Ubersetzung(en)\\
1.&76.&3744.&283.&14551.&1.&3&61&5\_6\_50 \textcolor{red}{\textcjheb{nwh}} HWN $|$(das) Verm"ogen\\
2.&77.&3745.&286.&14554.&4.&4&580&70\_300\_10\_200 \textcolor{red}{\textcjheb{ry+s`}} aSJR $|$des Reichen/(eines) Reichen\\
3.&78.&3746.&290.&14558.&8.&4&710&100\_200\_10\_400 \textcolor{red}{\textcjheb{tyrq}} QRJT $|$ist seine Stadt/(ist) die Burg\\
4.&79.&3747.&294.&14562.&12.&3&83&70\_7\_6 \textcolor{red}{\textcjheb{wz`}} aZW $|$feste/seiner Macht\\
5.&80.&3748.&297.&14565.&15.&6&85&6\_20\_8\_6\_40\_5 \textcolor{red}{\textcjheb{hmw.hkw}} WKCWMH $|$und gleich einer Mauer/und wie eine Mauer\\
6.&81.&3749.&303.&14571.&21.&5&360&50\_300\_3\_2\_5 \textcolor{red}{\textcjheb{hbg+sn}} NSGBH $|$hochragenden/hohe\\
7.&82.&3750.&308.&14576.&26.&7&778&2\_40\_300\_20\_10\_400\_6 \textcolor{red}{\textcjheb{wtyk+smb}} BMSKJTW $|$in seiner Einbildung\\
\end{tabular}\medskip \\
Ende des Verses 18.11\\
Verse: 512, Buchstaben: 32, 314, 14582, Totalwerte: 2657, 21992, 1045700\\
\\
Das Verm"ogen des Reichen ist seine feste Stadt, und in seiner Einbildung gleich einer hochragenden Mauer.\\
\newpage 
{\bf -- 18.12}\\
\medskip \\
\begin{tabular}{rrrrrrrrp{120mm}}
WV&WK&WB&ABK&ABB&ABV&AnzB&TW&Zahlencode \textcolor{red}{$\boldsymbol{Grundtext}$} Umschrift $|$"Ubersetzung(en)\\
1.&83.&3751.&315.&14583.&1.&4&170&30\_80\_50\_10 \textcolor{red}{\textcjheb{ynpl}} LPNJ $|$vor\\
2.&84.&3752.&319.&14587.&5.&3&502&300\_2\_200 \textcolor{red}{\textcjheb{rb+s}} SBR $|$dem Sturz/(dem) Zusammenbruch\\
3.&85.&3753.&322.&14590.&8.&4&20&10\_3\_2\_5 \textcolor{red}{\textcjheb{hbgy}} JGBH $|$wird hoff"artig/er (=es) ist hochm"utig\\
4.&86.&3754.&326.&14594.&12.&2&32&30\_2 \textcolor{red}{\textcjheb{bl}} LB $|$(das) Herz\\
5.&87.&3755.&328.&14596.&14.&3&311&1\_10\_300 \textcolor{red}{\textcjheb{+sy'}} AJS $|$des Mannes/(eines) Mannes\\
6.&88.&3756.&331.&14599.&17.&5&176&6\_30\_80\_50\_10 \textcolor{red}{\textcjheb{ynplw}} WLPNJ $|$und vor(aus)\\
7.&89.&3757.&336.&14604.&22.&4&32&20\_2\_6\_4 \textcolor{red}{\textcjheb{dwbk}} KBWD $|$der Ehre\\
8.&90.&3758.&340.&14608.&26.&4&131&70\_50\_6\_5 \textcolor{red}{\textcjheb{hwn`}} aNWH $|$geht Demut/(kommt) Demut\\
\end{tabular}\medskip \\
Ende des Verses 18.12\\
Verse: 513, Buchstaben: 29, 343, 14611, Totalwerte: 1374, 23366, 1047074\\
\\
Vor dem Sturze wird hoff"artig des Mannes Herz, und der Ehre geht Demut voraus.\\
\newpage 
{\bf -- 18.13}\\
\medskip \\
\begin{tabular}{rrrrrrrrp{120mm}}
WV&WK&WB&ABK&ABB&ABV&AnzB&TW&Zahlencode \textcolor{red}{$\boldsymbol{Grundtext}$} Umschrift $|$"Ubersetzung(en)\\
1.&91.&3759.&344.&14612.&1.&4&352&40\_300\_10\_2 \textcolor{red}{\textcjheb{by+sm}} MSJB $|$wer gibt/ein Gebender\\
2.&92.&3760.&348.&14616.&5.&3&206&4\_2\_200 \textcolor{red}{\textcjheb{rbd}} DBR $|$(Ant)Wort\\
3.&93.&3761.&351.&14619.&8.&4&251&2\_9\_200\_40 \textcolor{red}{\textcjheb{mr.tb}} BtRM $|$bevor\\
4.&94.&3762.&355.&14623.&12.&4&420&10\_300\_40\_70 \textcolor{red}{\textcjheb{`m+sy}} JSMa $|$er (an)h"ort (zu)\\
5.&95.&3763.&359.&14627.&16.&4&437&1\_6\_30\_400 \textcolor{red}{\textcjheb{tlw'}} AWLT $|$Narrheit\\
6.&96.&3764.&363.&14631.&20.&3&16&5\_10\_1 \textcolor{red}{\textcjheb{'yh}} HJA $|$(ist) es\\
7.&97.&3765.&366.&14634.&23.&2&36&30\_6 \textcolor{red}{\textcjheb{wl}} LW $|$dem/f"ur ihn\\
8.&98.&3766.&368.&14636.&25.&5&101&6\_20\_30\_40\_5 \textcolor{red}{\textcjheb{hmlkw}} WKLMH $|$und Schande\\
\end{tabular}\medskip \\
Ende des Verses 18.13\\
Verse: 514, Buchstaben: 29, 372, 14640, Totalwerte: 1819, 25185, 1048893\\
\\
Wer Antwort gibt, bevor er anh"ort, dem ist es Narrheit und Schande.\\
\newpage 
{\bf -- 18.14}\\
\medskip \\
\begin{tabular}{rrrrrrrrp{120mm}}
WV&WK&WB&ABK&ABB&ABV&AnzB&TW&Zahlencode \textcolor{red}{$\boldsymbol{Grundtext}$} Umschrift $|$"Ubersetzung(en)\\
1.&99.&3767.&373.&14641.&1.&3&214&200\_6\_8 \textcolor{red}{\textcjheb{.hwr}} RWC $|$(der) Geist\\
2.&100.&3768.&376.&14644.&4.&3&311&1\_10\_300 \textcolor{red}{\textcjheb{+sy'}} AJS $|$eines Mannes/(des) Mannes\\
3.&101.&3769.&379.&14647.&7.&5&110&10\_20\_30\_20\_30 \textcolor{red}{\textcjheb{lklky}} JKLKL $|$(er) ertr"agt\\
4.&102.&3770.&384.&14652.&12.&5&89&40\_8\_30\_5\_6 \textcolor{red}{\textcjheb{whl.hm}} MCLHW $|$seine Krankheit\\
5.&103.&3771.&389.&14657.&17.&4&220&6\_200\_6\_8 \textcolor{red}{\textcjheb{.hwrw}} WRWC $|$aber ein Geist/und ein Geist\\
6.&104.&3772.&393.&14661.&21.&4&76&50\_20\_1\_5 \textcolor{red}{\textcjheb{h'kn}} NKAH $|$zerschlagener/verzagter\\
7.&105.&3773.&397.&14665.&25.&2&50&40\_10 \textcolor{red}{\textcjheb{ym}} MJ $|$wer\\
8.&106.&3774.&399.&14667.&27.&5&366&10\_300\_1\_50\_5 \textcolor{red}{\textcjheb{hn'+sy}} JSANH $|$richtet ihn auf/er kann sie (=es) ertragen\\
\end{tabular}\medskip \\
Ende des Verses 18.14\\
Verse: 515, Buchstaben: 31, 403, 14671, Totalwerte: 1436, 26621, 1050329\\
\\
Eines Mannes Geist ertr"agt seine Krankheit; aber ein zerschlagener Geist, wer richtet ihn auf?\\
\newpage 
{\bf -- 18.15}\\
\medskip \\
\begin{tabular}{rrrrrrrrp{120mm}}
WV&WK&WB&ABK&ABB&ABV&AnzB&TW&Zahlencode \textcolor{red}{$\boldsymbol{Grundtext}$} Umschrift $|$"Ubersetzung(en)\\
1.&107.&3775.&404.&14672.&1.&2&32&30\_2 \textcolor{red}{\textcjheb{bl}} LB $|$das Herz\\
2.&108.&3776.&406.&14674.&3.&4&108&50\_2\_6\_50 \textcolor{red}{\textcjheb{nwbn}} NBWN $|$des Verst"andigen/(des) Klugen\\
3.&109.&3777.&410.&14678.&7.&4&165&10\_100\_50\_5 \textcolor{red}{\textcjheb{hnqy}} JQNH $|$(er (=es)) erwirbt\\
4.&110.&3778.&414.&14682.&11.&3&474&4\_70\_400 \textcolor{red}{\textcjheb{t`d}} DaT $|$Erkenntnis\\
5.&111.&3779.&417.&14685.&14.&4&64&6\_1\_7\_50 \textcolor{red}{\textcjheb{nz'w}} WAZN $|$und das Ohr\\
6.&112.&3780.&421.&14689.&18.&5&118&8\_20\_40\_10\_40 \textcolor{red}{\textcjheb{mymk.h}} CKMJM $|$der Weisen\\
7.&113.&3781.&426.&14694.&23.&4&802&400\_2\_100\_300 \textcolor{red}{\textcjheb{+sqbt}} TBQS $|$(sie (=es)) sucht (nach)\\
8.&114.&3782.&430.&14698.&27.&3&474&4\_70\_400 \textcolor{red}{\textcjheb{t`d}} DaT $|$Erkenntnis\\
\end{tabular}\medskip \\
Ende des Verses 18.15\\
Verse: 516, Buchstaben: 29, 432, 14700, Totalwerte: 2237, 28858, 1052566\\
\\
Das Herz des Verst"andigen erwirbt Erkenntnis, und das Ohr der Weisen sucht nach Erkenntnis.\\
\newpage 
{\bf -- 18.16}\\
\medskip \\
\begin{tabular}{rrrrrrrrp{120mm}}
WV&WK&WB&ABK&ABB&ABV&AnzB&TW&Zahlencode \textcolor{red}{$\boldsymbol{Grundtext}$} Umschrift $|$"Ubersetzung(en)\\
1.&115.&3783.&433.&14701.&1.&3&490&40\_400\_50 \textcolor{red}{\textcjheb{ntm}} MTN $|$das Geschenk\\
2.&116.&3784.&436.&14704.&4.&3&45&1\_4\_40 \textcolor{red}{\textcjheb{md'}} ADM $|$des Menschen/(eines) Menschen\\
3.&117.&3785.&439.&14707.&7.&5&230&10\_200\_8\_10\_2 \textcolor{red}{\textcjheb{by.hry}} JRCJB $|$macht Raum/er (=es) schafft Raum\\
4.&118.&3786.&444.&14712.&12.&2&36&30\_6 \textcolor{red}{\textcjheb{wl}} LW $|$ihm\\
5.&119.&3787.&446.&14714.&14.&5&176&6\_30\_80\_50\_10 \textcolor{red}{\textcjheb{ynplw}} WLPNJ $|$und zu den/und vor\\
6.&120.&3788.&451.&14719.&19.&5&87&3\_4\_30\_10\_40 \textcolor{red}{\textcjheb{myldg}} GDLJM $|$(die) Gro"sen\\
7.&121.&3789.&456.&14724.&24.&5&124&10\_50\_8\_50\_6 \textcolor{red}{\textcjheb{wn.hny}} JNCNW $|$verschafft ihm Zutritt/er (=es) bringt ihn\\
\end{tabular}\medskip \\
Ende des Verses 18.16\\
Verse: 517, Buchstaben: 28, 460, 14728, Totalwerte: 1188, 30046, 1053754\\
\\
Das Geschenk des Menschen macht ihm Raum und verschafft ihm Zutritt zu den Gro"sen.\\
\newpage 
{\bf -- 18.17}\\
\medskip \\
\begin{tabular}{rrrrrrrrp{120mm}}
WV&WK&WB&ABK&ABB&ABV&AnzB&TW&Zahlencode \textcolor{red}{$\boldsymbol{Grundtext}$} Umschrift $|$"Ubersetzung(en)\\
1.&122.&3790.&461.&14729.&1.&4&204&90\_4\_10\_100 \textcolor{red}{\textcjheb{qyd.s}} "sDJQ $|$Recht hat\\
2.&123.&3791.&465.&14733.&5.&6&562&5\_200\_1\_300\_6\_50 \textcolor{red}{\textcjheb{nw+s'rh}} HRASWN $|$der Erste\\
3.&124.&3792.&471.&14739.&11.&5&220&2\_200\_10\_2\_6 \textcolor{red}{\textcjheb{wbyrb}} BRJBW $|$in seiner Streitsache/in seinem Streitfall\\
4.&125.&3793.&476.&14744.&16.&3&13&10\_2\_1 \textcolor{red}{\textcjheb{'by}} JBA $|$doch es kommt/(und) er (=es) kommt\\
5.&126.&3794.&479.&14747.&19.&4&281&200\_70\_5\_6 \textcolor{red}{\textcjheb{wh`r}} RaHW $|$sein N"achster/sein Gef"ahrte\\
6.&127.&3795.&483.&14751.&23.&5&320&6\_8\_100\_200\_6 \textcolor{red}{\textcjheb{wrq.hw}} WCQRW $|$und forscht ihn aus/und (er) "uberpr"uft ihn\\
\end{tabular}\medskip \\
Ende des Verses 18.17\\
Verse: 518, Buchstaben: 27, 487, 14755, Totalwerte: 1600, 31646, 1055354\\
\\
Der erste in seiner Streitsache hat recht; doch sein N"achster kommt und forscht ihn aus.\\
\newpage 
{\bf -- 18.18}\\
\medskip \\
\begin{tabular}{rrrrrrrrp{120mm}}
WV&WK&WB&ABK&ABB&ABV&AnzB&TW&Zahlencode \textcolor{red}{$\boldsymbol{Grundtext}$} Umschrift $|$"Ubersetzung(en)\\
1.&128.&3796.&488.&14756.&1.&6&154&40\_4\_10\_50\_10\_40 \textcolor{red}{\textcjheb{mynydm}} MDJNJM $|$Zwistigkeiten\\
2.&129.&3797.&494.&14762.&7.&5&722&10\_300\_2\_10\_400 \textcolor{red}{\textcjheb{tyb+sy}} JSBJT $|$(er (=es)) schlichtet\\
3.&130.&3798.&499.&14767.&12.&5&244&5\_3\_6\_200\_30 \textcolor{red}{\textcjheb{lrwgh}} HGWRL $|$das Los\\
4.&131.&3799.&504.&14772.&17.&4&68&6\_2\_10\_50 \textcolor{red}{\textcjheb{nybw}} WBJN $|$und auseinander/und (zwischen)\\
5.&132.&3800.&508.&14776.&21.&6&256&70\_90\_6\_40\_10\_40 \textcolor{red}{\textcjheb{mymw.s`}} a"sWMJM $|$M"achtige(n)\\
6.&133.&3801.&514.&14782.&27.&5&304&10\_80\_200\_10\_4 \textcolor{red}{\textcjheb{dyrpy}} JPRJD $|$bringt (es)/er (=es) trennt\\
\end{tabular}\medskip \\
Ende des Verses 18.18\\
Verse: 519, Buchstaben: 31, 518, 14786, Totalwerte: 1748, 33394, 1057102\\
\\
Das Los schlichtet Zwistigkeiten und bringt M"achtige auseinander.\\
\newpage 
{\bf -- 18.19}\\
\medskip \\
\begin{tabular}{rrrrrrrrp{120mm}}
WV&WK&WB&ABK&ABB&ABV&AnzB&TW&Zahlencode \textcolor{red}{$\boldsymbol{Grundtext}$} Umschrift $|$"Ubersetzung(en)\\
1.&134.&3802.&519.&14787.&1.&2&9&1\_8 \textcolor{red}{\textcjheb{.h'}} AC $|$(ein) Bruder\\
2.&135.&3803.&521.&14789.&3.&4&500&50\_80\_300\_70 \textcolor{red}{\textcjheb{`+spn}} NPSa $|$an dem man treulos gehandelt hat/betrogener\\
3.&136.&3804.&525.&14793.&7.&5&750&40\_100\_200\_10\_400 \textcolor{red}{\textcjheb{tyrqm}} MQRJT $|$widersteht mehr als eine Stadt/(besitzt) mehr als eine Burg\\
4.&137.&3805.&530.&14798.&12.&2&77&70\_7 \textcolor{red}{\textcjheb{z`}} aZ $|$feste/(an) St"arke\\
5.&138.&3806.&532.&14800.&14.&7&156&6\_40\_4\_6\_50\_10\_40 \textcolor{red}{\textcjheb{mynwdmw}} WMDWNJM $|$und Zwistigkeiten\\
6.&139.&3807.&539.&14807.&21.&5&240&20\_2\_200\_10\_8 \textcolor{red}{\textcjheb{.hyrbk}} KBRJC $|$(sind) wie der Riegel\\
7.&140.&3808.&544.&14812.&26.&5&297&1\_200\_40\_6\_50 \textcolor{red}{\textcjheb{nwmr'}} ARMWN $|$einer Burg/eines Wohnturms\\
\end{tabular}\medskip \\
Ende des Verses 18.19\\
Verse: 520, Buchstaben: 30, 548, 14816, Totalwerte: 2029, 35423, 1059131\\
\\
Ein Bruder, an dem man treulos gehandelt hat, widersteht mehr als eine feste Stadt; und Zwistigkeiten sind wie der Riegel einer Burg.\\
\newpage 
{\bf -- 18.20}\\
\medskip \\
\begin{tabular}{rrrrrrrrp{120mm}}
WV&WK&WB&ABK&ABB&ABV&AnzB&TW&Zahlencode \textcolor{red}{$\boldsymbol{Grundtext}$} Umschrift $|$"Ubersetzung(en)\\
1.&141.&3809.&549.&14817.&1.&4&330&40\_80\_200\_10 \textcolor{red}{\textcjheb{yrpm}} MPRJ $|$von der Frucht\\
2.&142.&3810.&553.&14821.&5.&2&90&80\_10 \textcolor{red}{\textcjheb{yp}} PJ $|$des Mundes\\
3.&143.&3811.&555.&14823.&7.&3&311&1\_10\_300 \textcolor{red}{\textcjheb{+sy'}} AJS $|$(eines) Mannes\\
4.&144.&3812.&558.&14826.&10.&4&772&400\_300\_2\_70 \textcolor{red}{\textcjheb{`b+st}} TSBa $|$wird ges"attigt/sie (=es) wird satt\\
5.&145.&3813.&562.&14830.&14.&4&67&2\_9\_50\_6 \textcolor{red}{\textcjheb{wn.tb}} BtNW $|$sein Inneres/sein Leib\\
6.&146.&3814.&566.&14834.&18.&5&809&400\_2\_6\_1\_400 \textcolor{red}{\textcjheb{t'wbt}} TBWAT $|$(vom) Ertrag\\
7.&147.&3815.&571.&14839.&23.&5&796&300\_80\_400\_10\_6 \textcolor{red}{\textcjheb{wytp+s}} SPTJW $|$seiner Lippen\\
8.&148.&3816.&576.&14844.&28.&4&382&10\_300\_2\_70 \textcolor{red}{\textcjheb{`b+sy}} JSBa $|$wird er ges"attigt/er wird satt\\
\end{tabular}\medskip \\
Ende des Verses 18.20\\
Verse: 521, Buchstaben: 31, 579, 14847, Totalwerte: 3557, 38980, 1062688\\
\\
Von der Frucht des Mundes eines Mannes wird sein Inneres ges"attigt, vom Ertrage seiner Lippen wird er ges"attigt.\\
\newpage 
{\bf -- 18.21}\\
\medskip \\
\begin{tabular}{rrrrrrrrp{120mm}}
WV&WK&WB&ABK&ABB&ABV&AnzB&TW&Zahlencode \textcolor{red}{$\boldsymbol{Grundtext}$} Umschrift $|$"Ubersetzung(en)\\
1.&149.&3817.&580.&14848.&1.&3&446&40\_6\_400 \textcolor{red}{\textcjheb{twm}} MWT $|$Tod\\
2.&150.&3818.&583.&14851.&4.&5&74&6\_8\_10\_10\_40 \textcolor{red}{\textcjheb{myy.hw}} WCJJM $|$und Leben\\
3.&151.&3819.&588.&14856.&9.&3&16&2\_10\_4 \textcolor{red}{\textcjheb{dyb}} BJD $|$sind in der Gewalt/(sind) in der Hand\\
4.&152.&3820.&591.&14859.&12.&4&386&30\_300\_6\_50 \textcolor{red}{\textcjheb{nw+sl}} LSWN $|$der Zunge\\
5.&153.&3821.&595.&14863.&16.&6&29&6\_1\_5\_2\_10\_5 \textcolor{red}{\textcjheb{hybh'w}} WAHBJH $|$und wer sie liebt\\
6.&154.&3822.&601.&14869.&22.&4&61&10\_1\_20\_30 \textcolor{red}{\textcjheb{lk'y}} JAKL $|$wird essen/(d)er isst\\
7.&155.&3823.&605.&14873.&26.&4&295&80\_200\_10\_5 \textcolor{red}{\textcjheb{hyrp}} PRJH $|$ihre Frucht\\
\end{tabular}\medskip \\
Ende des Verses 18.21\\
Verse: 522, Buchstaben: 29, 608, 14876, Totalwerte: 1307, 40287, 1063995\\
\\
Tod und Leben sind in der Gewalt der Zunge, und wer sie liebt, wird ihre Frucht essen.\\
\newpage 
{\bf -- 18.22}\\
\medskip \\
\begin{tabular}{rrrrrrrrp{120mm}}
WV&WK&WB&ABK&ABB&ABV&AnzB&TW&Zahlencode \textcolor{red}{$\boldsymbol{Grundtext}$} Umschrift $|$"Ubersetzung(en)\\
1.&156.&3824.&609.&14877.&1.&3&131&40\_90\_1 \textcolor{red}{\textcjheb{'.sm}} M"sA $|$(w)er (hat) gefunden\\
2.&157.&3825.&612.&14880.&4.&3&306&1\_300\_5 \textcolor{red}{\textcjheb{h+s'}} ASH $|$(eine) Frau\\
3.&158.&3826.&615.&14883.&7.&3&131&40\_90\_1 \textcolor{red}{\textcjheb{'.sm}} M"sA $|$((d)er) hat gefunden\\
4.&159.&3827.&618.&14886.&10.&3&17&9\_6\_2 \textcolor{red}{\textcjheb{bw.t}} tWB $|$Gutes\\
5.&160.&3828.&621.&14889.&13.&4&196&6\_10\_80\_100 \textcolor{red}{\textcjheb{qpyw}} WJPQ $|$und (er) erlangt\\
6.&161.&3829.&625.&14893.&17.&4&346&200\_90\_6\_50 \textcolor{red}{\textcjheb{nw.sr}} R"sWN $|$Wohlgefallen/Wohlgef"alliges\\
7.&162.&3830.&629.&14897.&21.&5&66&40\_10\_5\_6\_5 \textcolor{red}{\textcjheb{hwhym}} MJHWH $|$von Jahwe\\
\end{tabular}\medskip \\
Ende des Verses 18.22\\
Verse: 523, Buchstaben: 25, 633, 14901, Totalwerte: 1193, 41480, 1065188\\
\\
Wer ein Weib gefunden, hat Gutes gefunden und hat Wohlgefallen erlangt von Jahwe.\\
\newpage 
{\bf -- 18.23}\\
\medskip \\
\begin{tabular}{rrrrrrrrp{120mm}}
WV&WK&WB&ABK&ABB&ABV&AnzB&TW&Zahlencode \textcolor{red}{$\boldsymbol{Grundtext}$} Umschrift $|$"Ubersetzung(en)\\
1.&163.&3831.&634.&14902.&1.&7&564&400\_8\_50\_6\_50\_10\_40 \textcolor{red}{\textcjheb{mynwn.ht}} TCNWNJM $|$flehentlich/Bitten\\
2.&164.&3832.&641.&14909.&8.&4&216&10\_4\_2\_200 \textcolor{red}{\textcjheb{rbdy}} JDBR $|$bittet/er (=es) spricht\\
3.&165.&3833.&645.&14913.&12.&2&500&200\_300 \textcolor{red}{\textcjheb{+sr}} RS $|$(der) Arme\\
4.&166.&3834.&647.&14915.&14.&5&586&6\_70\_300\_10\_200 \textcolor{red}{\textcjheb{ry+s`w}} WaSJR $|$aber der Reiche/und der Reiche\\
5.&167.&3835.&652.&14920.&19.&4&135&10\_70\_50\_5 \textcolor{red}{\textcjheb{hn`y}} JaNH $|$(er) antwortet\\
6.&168.&3836.&656.&14924.&23.&4&483&70\_7\_6\_400 \textcolor{red}{\textcjheb{twz`}} aZWT $|$Hartes/mit H"arte\\
\end{tabular}\medskip \\
Ende des Verses 18.23\\
Verse: 524, Buchstaben: 26, 659, 14927, Totalwerte: 2484, 43964, 1067672\\
\\
Flehentlich bittet der Arme, aber der Reiche antwortet Hartes.\\
\newpage 
{\bf -- 18.24}\\
\medskip \\
\begin{tabular}{rrrrrrrrp{120mm}}
WV&WK&WB&ABK&ABB&ABV&AnzB&TW&Zahlencode \textcolor{red}{$\boldsymbol{Grundtext}$} Umschrift $|$"Ubersetzung(en)\\
1.&169.&3837.&660.&14928.&1.&3&311&1\_10\_300 \textcolor{red}{\textcjheb{+sy'}} AJS $|$(ein) Mann\\
2.&170.&3838.&663.&14931.&4.&4&320&200\_70\_10\_40 \textcolor{red}{\textcjheb{my`r}} RaJM $|$vieler Freunde/(hat) Freunde\\
3.&171.&3839.&667.&14935.&8.&6&775&30\_5\_400\_200\_70\_70 \textcolor{red}{\textcjheb{``rthl}} LHTRaa $|$wird zu Grunde gehen/zu zertr"ummern\\
4.&172.&3840.&673.&14941.&14.&3&316&6\_10\_300 \textcolor{red}{\textcjheb{+syw}} WJS $|$doch es gibt einen/und es gibt\\
5.&173.&3841.&676.&14944.&17.&3&8&1\_5\_2 \textcolor{red}{\textcjheb{bh'}} AHB $|$der liebt/(einen) Liebenden\\
6.&174.&3842.&679.&14947.&20.&3&106&4\_2\_100 \textcolor{red}{\textcjheb{qbd}} DBQ $|$(und) anh"anglicher ist/(der) anh"anglich (ist)\\
7.&175.&3843.&682.&14950.&23.&3&49&40\_1\_8 \textcolor{red}{\textcjheb{.h'm}} MAC $|$(mehr) als ein Bruder\\
\end{tabular}\medskip \\
Ende des Verses 18.24\\
Verse: 525, Buchstaben: 25, 684, 14952, Totalwerte: 1885, 45849, 1069557\\
\\
Ein Mann vieler Freunde wird zu Grunde gehen; doch es gibt einen, der liebt und anh"anglicher ist als ein Bruder.\\
\\
{\bf Ende des Kapitels 18}\\
\newpage 
{\bf -- 19.1}\\
\medskip \\
\begin{tabular}{rrrrrrrrp{120mm}}
WV&WK&WB&ABK&ABB&ABV&AnzB&TW&Zahlencode \textcolor{red}{$\boldsymbol{Grundtext}$} Umschrift $|$"Ubersetzung(en)\\
1.&1.&3844.&1.&14953.&1.&3&17&9\_6\_2 \textcolor{red}{\textcjheb{bw.t}} tWB $|$besser/gut (ist)\\
2.&2.&3845.&4.&14956.&4.&2&500&200\_300 \textcolor{red}{\textcjheb{+sr}} RS $|$(ein) Armer\\
3.&3.&3846.&6.&14958.&6.&4&61&5\_6\_30\_20 \textcolor{red}{\textcjheb{klwh}} HWLK $|$der wandelt/Gehender\\
4.&4.&3847.&10.&14962.&10.&4&448&2\_400\_40\_6 \textcolor{red}{\textcjheb{wmtb}} BTMW $|$in seiner Vollkommenheit/in seiner Lauterkeit\\
5.&5.&3848.&14.&14966.&14.&4&510&40\_70\_100\_300 \textcolor{red}{\textcjheb{+sq`m}} MaQS $|$als wer verkehrter/als ein Verkehrter\\
6.&6.&3849.&18.&14970.&18.&5&796&300\_80\_400\_10\_6 \textcolor{red}{\textcjheb{wytp+s}} SPTJW $|$(mit) (seinen) Lippen\\
7.&7.&3850.&23.&14975.&23.&4&18&6\_5\_6\_1 \textcolor{red}{\textcjheb{'whw}} WHWA $|$und dabei ist/und er (ist)\\
8.&8.&3851.&27.&14979.&27.&4&120&20\_60\_10\_30 \textcolor{red}{\textcjheb{lysk}} KsJL $|$(ein) Tor\\
\end{tabular}\medskip \\
Ende des Verses 19.1\\
Verse: 526, Buchstaben: 30, 30, 14982, Totalwerte: 2470, 2470, 1072027\\
\\
Besser ein Armer, der in seiner Vollkommenheit wandelt, als wer verkehrter Lippen und dabei ein Tor ist.\\
\newpage 
{\bf -- 19.2}\\
\medskip \\
\begin{tabular}{rrrrrrrrp{120mm}}
WV&WK&WB&ABK&ABB&ABV&AnzB&TW&Zahlencode \textcolor{red}{$\boldsymbol{Grundtext}$} Umschrift $|$"Ubersetzung(en)\\
1.&9.&3852.&31.&14983.&1.&2&43&3\_40 \textcolor{red}{\textcjheb{mg}} GM $|$auch\\
2.&10.&3853.&33.&14985.&3.&3&33&2\_30\_1 \textcolor{red}{\textcjheb{'lb}} BLA $|$Un-/in nicht\\
3.&11.&3854.&36.&14988.&6.&3&474&4\_70\_400 \textcolor{red}{\textcjheb{t`d}} DaT $|$Kenntnis/Wissen\\
4.&12.&3855.&39.&14991.&9.&3&430&50\_80\_300 \textcolor{red}{\textcjheb{+spn}} NPS $|$der Seele/(ist) Seele\\
5.&13.&3856.&42.&14994.&12.&2&31&30\_1 \textcolor{red}{\textcjheb{'l}} LA $|$nicht\\
6.&14.&3857.&44.&14996.&14.&3&17&9\_6\_2 \textcolor{red}{\textcjheb{bw.t}} tWB $|$gut (ist)\\
7.&15.&3858.&47.&14999.&17.&3&97&6\_1\_90 \textcolor{red}{\textcjheb{.s'w}} WA"s $|$und wer hastig ist/und ein Eilender\\
8.&16.&3859.&50.&15002.&20.&6&285&2\_200\_3\_30\_10\_40 \textcolor{red}{\textcjheb{mylgrb}} BRGLJM $|$mit den F"u"sen\\
9.&17.&3860.&56.&15008.&26.&4&24&8\_6\_9\_1 \textcolor{red}{\textcjheb{'.tw.h}} CWtA $|$tritt fehl/(ist) s"undigend\\
\end{tabular}\medskip \\
Ende des Verses 19.2\\
Verse: 527, Buchstaben: 29, 59, 15011, Totalwerte: 1434, 3904, 1073461\\
\\
Auch Unkenntnis der Seele ist nicht gut; und wer mit den F"u"sen hastig ist, tritt fehl.\\
\newpage 
{\bf -- 19.3}\\
\medskip \\
\begin{tabular}{rrrrrrrrp{120mm}}
WV&WK&WB&ABK&ABB&ABV&AnzB&TW&Zahlencode \textcolor{red}{$\boldsymbol{Grundtext}$} Umschrift $|$"Ubersetzung(en)\\
1.&18.&3861.&60.&15012.&1.&4&437&1\_6\_30\_400 \textcolor{red}{\textcjheb{tlw'}} AWLT $|$die Narrheit/die Torheit\\
2.&19.&3862.&64.&15016.&5.&3&45&1\_4\_40 \textcolor{red}{\textcjheb{md'}} ADM $|$des Menschen\\
3.&20.&3863.&67.&15019.&8.&4&570&400\_60\_30\_80 \textcolor{red}{\textcjheb{plst}} TsLP $|$verdirbt/(sie) verdreht\\
4.&21.&3864.&71.&15023.&12.&4&230&4\_200\_20\_6 \textcolor{red}{\textcjheb{wkrd}} DRKW $|$seinen Weg\\
5.&22.&3865.&75.&15027.&16.&3&106&6\_70\_30 \textcolor{red}{\textcjheb{l`w}} WaL $|$und wider\\
6.&23.&3866.&78.&15030.&19.&4&26&10\_5\_6\_5 \textcolor{red}{\textcjheb{hwhy}} JHWH $|$Jahwe\\
7.&24.&3867.&82.&15034.&23.&4&167&10\_7\_70\_80 \textcolor{red}{\textcjheb{p`zy}} JZaP $|$grollt/er (=es) z"urnt\\
8.&25.&3868.&86.&15038.&27.&3&38&30\_2\_6 \textcolor{red}{\textcjheb{wbl}} LBW $|$sein Herz\\
\end{tabular}\medskip \\
Ende des Verses 19.3\\
Verse: 528, Buchstaben: 29, 88, 15040, Totalwerte: 1619, 5523, 1075080\\
\\
Die Narrheit des Menschen verdirbt seinen Weg, und sein Herz grollt wider Jahwe.\\
\newpage 
{\bf -- 19.4}\\
\medskip \\
\begin{tabular}{rrrrrrrrp{120mm}}
WV&WK&WB&ABK&ABB&ABV&AnzB&TW&Zahlencode \textcolor{red}{$\boldsymbol{Grundtext}$} Umschrift $|$"Ubersetzung(en)\\
1.&26.&3869.&89.&15041.&1.&3&61&5\_6\_50 \textcolor{red}{\textcjheb{nwh}} HWN $|$Reichtum/Verm"ogen\\
2.&27.&3870.&92.&15044.&4.&4&160&10\_60\_10\_80 \textcolor{red}{\textcjheb{pysy}} JsJP $|$verschafft/er (=es) mehrt\\
3.&28.&3871.&96.&15048.&8.&4&320&200\_70\_10\_40 \textcolor{red}{\textcjheb{my`r}} RaJM $|$Freunde\\
4.&29.&3872.&100.&15052.&12.&4&252&200\_2\_10\_40 \textcolor{red}{\textcjheb{mybr}} RBJM $|$viele\\
5.&30.&3873.&104.&15056.&16.&3&40&6\_4\_30 \textcolor{red}{\textcjheb{ldw}} WDL $|$aber der Arme/und ein Armer\\
6.&31.&3874.&107.&15059.&19.&5&321&40\_200\_70\_5\_6 \textcolor{red}{\textcjheb{wh`rm}} MRaHW $|$(von) sein(em) Freund\\
7.&32.&3875.&112.&15064.&24.&4&294&10\_80\_200\_4 \textcolor{red}{\textcjheb{drpy}} JPRD $|$trennt sich von ihm/(er) wird getrennt\\
\end{tabular}\medskip \\
Ende des Verses 19.4\\
Verse: 529, Buchstaben: 27, 115, 15067, Totalwerte: 1448, 6971, 1076528\\
\\
Reichtum verschafft viele Freunde; aber der Arme-sein Freund trennt sich von ihm.\\
\newpage 
{\bf -- 19.5}\\
\medskip \\
\begin{tabular}{rrrrrrrrp{120mm}}
WV&WK&WB&ABK&ABB&ABV&AnzB&TW&Zahlencode \textcolor{red}{$\boldsymbol{Grundtext}$} Umschrift $|$"Ubersetzung(en)\\
1.&33.&3876.&116.&15068.&1.&2&74&70\_4 \textcolor{red}{\textcjheb{d`}} aD $|$(ein) Zeuge\\
2.&34.&3877.&118.&15070.&3.&5&650&300\_100\_200\_10\_40 \textcolor{red}{\textcjheb{myrq+s}} SQRJM $|$falscher/(von) L"ugen\\
3.&35.&3878.&123.&15075.&8.&2&31&30\_1 \textcolor{red}{\textcjheb{'l}} LA $|$nicht\\
4.&36.&3879.&125.&15077.&10.&4&165&10\_50\_100\_5 \textcolor{red}{\textcjheb{hqny}} JNQH $|$wird f"ur schuldlos gehalten/(er) bleibt ungestraft\\
5.&37.&3880.&129.&15081.&14.&5&114&6\_10\_80\_10\_8 \textcolor{red}{\textcjheb{.hypyw}} WJPJC $|$und wer ausspricht\\
6.&38.&3881.&134.&15086.&19.&5&79&20\_7\_2\_10\_40 \textcolor{red}{\textcjheb{mybzk}} KZBJM $|$L"ugen\\
7.&39.&3882.&139.&15091.&24.&2&31&30\_1 \textcolor{red}{\textcjheb{'l}} LA $|$nicht\\
8.&40.&3883.&141.&15093.&26.&4&89&10\_40\_30\_9 \textcolor{red}{\textcjheb{.tlmy}} JMLt $|$(er) wird entrinnen\\
\end{tabular}\medskip \\
Ende des Verses 19.5\\
Verse: 530, Buchstaben: 29, 144, 15096, Totalwerte: 1233, 8204, 1077761\\
\\
Ein falscher Zeuge wird nicht f"ur schuldlos gehalten werden; und wer L"ugen ausspricht, wird nicht entrinnen.\\
\newpage 
{\bf -- 19.6}\\
\medskip \\
\begin{tabular}{rrrrrrrrp{120mm}}
WV&WK&WB&ABK&ABB&ABV&AnzB&TW&Zahlencode \textcolor{red}{$\boldsymbol{Grundtext}$} Umschrift $|$"Ubersetzung(en)\\
1.&41.&3884.&145.&15097.&1.&4&252&200\_2\_10\_40 \textcolor{red}{\textcjheb{mybr}} RBJM $|$viele\\
2.&42.&3885.&149.&15101.&5.&4&54&10\_8\_30\_6 \textcolor{red}{\textcjheb{wl.hy}} JCLW $|$schmeicheln/sie bes"anftigen\\
3.&43.&3886.&153.&15105.&9.&3&140&80\_50\_10 \textcolor{red}{\textcjheb{ynp}} PNJ $|$/Gesichter\\
4.&44.&3887.&156.&15108.&12.&4&66&50\_4\_10\_2 \textcolor{red}{\textcjheb{bydn}} NDJB $|$einem Edlen/(einem) Vornehmen\\
5.&45.&3888.&160.&15112.&16.&3&56&6\_20\_30 \textcolor{red}{\textcjheb{lkw}} WKL $|$und alle/und jeder\\
6.&46.&3889.&163.&15115.&19.&3&275&5\_200\_70 \textcolor{red}{\textcjheb{`rh}} HRa $|$sind Freunde/(will sein) der Freund\\
7.&47.&3890.&166.&15118.&22.&4&341&30\_1\_10\_300 \textcolor{red}{\textcjheb{+sy'l}} LAJS $|$des Mannes/von einem Mann\\
8.&48.&3891.&170.&15122.&26.&3&490&40\_400\_50 \textcolor{red}{\textcjheb{ntm}} MTN $|$der Geschenke gibt/des Geschenks\\
\end{tabular}\medskip \\
Ende des Verses 19.6\\
Verse: 531, Buchstaben: 28, 172, 15124, Totalwerte: 1674, 9878, 1079435\\
\\
Viele schmeicheln einem Edlen, und alle sind Freunde des Mannes, der Geschenke gibt.\\
\newpage 
{\bf -- 19.7}\\
\medskip \\
\begin{tabular}{rrrrrrrrp{120mm}}
WV&WK&WB&ABK&ABB&ABV&AnzB&TW&Zahlencode \textcolor{red}{$\boldsymbol{Grundtext}$} Umschrift $|$"Ubersetzung(en)\\
1.&49.&3892.&173.&15125.&1.&2&50&20\_30 \textcolor{red}{\textcjheb{lk}} KL $|$alle\\
2.&50.&3893.&175.&15127.&3.&3&19&1\_8\_10 \textcolor{red}{\textcjheb{y.h'}} ACJ $|$Br"uder\\
3.&51.&3894.&178.&15130.&6.&2&500&200\_300 \textcolor{red}{\textcjheb{+sr}} RS $|$(des) Armen/(von einem) Armen\\
4.&52.&3895.&180.&15132.&8.&5&362&300\_50\_1\_5\_6 \textcolor{red}{\textcjheb{wh'n+s}} SNAHW $|$hassen ihn/sie meiden ihn\\
5.&53.&3896.&185.&15137.&13.&2&81&1\_80 \textcolor{red}{\textcjheb{p'}} AP $|$wie viel/auch\\
6.&54.&3897.&187.&15139.&15.&2&30&20\_10 \textcolor{red}{\textcjheb{yk}} KJ $|$mehr/wenn\\
7.&55.&3898.&189.&15141.&17.&5&321&40\_200\_70\_5\_6 \textcolor{red}{\textcjheb{wh`rm}} MRaHW $|$seine Freunde/seine Gef"ahrten\\
8.&56.&3899.&194.&15146.&22.&4&314&200\_8\_100\_6 \textcolor{red}{\textcjheb{wq.hr}} RCQW $|$entfernen sich/sie halten sich fern\\
9.&57.&3900.&198.&15150.&26.&4&136&40\_40\_50\_6 \textcolor{red}{\textcjheb{wnmm}} MMNW $|$von ihm\\
10.&58.&3901.&202.&15154.&30.&4&324&40\_200\_4\_80 \textcolor{red}{\textcjheb{pdrm}} MRDP $|$er jagt nach/ein Haschender\\
11.&59.&3902.&206.&15158.&34.&5&291&1\_40\_200\_10\_40 \textcolor{red}{\textcjheb{myrm'}} AMRJM $|$(nach) Worten\\
12.&60.&3903.&211.&15163.&39.&2&31&30\_1 \textcolor{red}{\textcjheb{'l}} LA $|$(die) nicht(s)\\
13.&61.&3904.&213.&15165.&41.&3&50&5\_40\_5 \textcolor{red}{\textcjheb{hmh}} HMH $|$(sie) sind\\
\end{tabular}\medskip \\
Ende des Verses 19.7\\
Verse: 532, Buchstaben: 43, 215, 15167, Totalwerte: 2509, 12387, 1081944\\
\\
Alle Br"uder des Armen hassen ihn; wieviel mehr entfernen sich von ihm seine Freunde! Er jagt Worten nach, die nichts sind.\\
\newpage 
{\bf -- 19.8}\\
\medskip \\
\begin{tabular}{rrrrrrrrp{120mm}}
WV&WK&WB&ABK&ABB&ABV&AnzB&TW&Zahlencode \textcolor{red}{$\boldsymbol{Grundtext}$} Umschrift $|$"Ubersetzung(en)\\
1.&62.&3905.&216.&15168.&1.&3&155&100\_50\_5 \textcolor{red}{\textcjheb{hnq}} QNH $|$wer erwirbt/(ein) Erwerbender\\
2.&63.&3906.&219.&15171.&4.&2&32&30\_2 \textcolor{red}{\textcjheb{bl}} LB $|$Herz (=Verstand)\\
3.&64.&3907.&221.&15173.&6.&3&8&1\_5\_2 \textcolor{red}{\textcjheb{bh'}} AHB $|$liebt/(ist) liebend(er)\\
4.&65.&3908.&224.&15176.&9.&4&436&50\_80\_300\_6 \textcolor{red}{\textcjheb{w+spn}} NPSW $|$seine Seele\\
5.&66.&3909.&228.&15180.&13.&3&540&300\_40\_200 \textcolor{red}{\textcjheb{rm+s}} SMR $|$wer achtet/(ein) Wahrender\\
6.&67.&3910.&231.&15183.&16.&5&463&400\_2\_6\_50\_5 \textcolor{red}{\textcjheb{hnwbt}} TBWNH $|$auf Verst"andnis/Einsicht\\
7.&68.&3911.&236.&15188.&21.&4&161&30\_40\_90\_1 \textcolor{red}{\textcjheb{'.sml}} LM"sA $|$wird erlangen/findet\\
8.&69.&3912.&240.&15192.&25.&3&17&9\_6\_2 \textcolor{red}{\textcjheb{bw.t}} tWB $|$Gl"uck/Gutes\\
\end{tabular}\medskip \\
Ende des Verses 19.8\\
Verse: 533, Buchstaben: 27, 242, 15194, Totalwerte: 1812, 14199, 1083756\\
\\
Wer Verstand erwirbt, liebt seine Seele; wer auf Verst"andnis achtet, wird Gl"uck erlangen.\\
\newpage 
{\bf -- 19.9}\\
\medskip \\
\begin{tabular}{rrrrrrrrp{120mm}}
WV&WK&WB&ABK&ABB&ABV&AnzB&TW&Zahlencode \textcolor{red}{$\boldsymbol{Grundtext}$} Umschrift $|$"Ubersetzung(en)\\
1.&70.&3913.&243.&15195.&1.&2&74&70\_4 \textcolor{red}{\textcjheb{d`}} aD $|$(ein) Zeuge\\
2.&71.&3914.&245.&15197.&3.&5&650&300\_100\_200\_10\_40 \textcolor{red}{\textcjheb{myrq+s}} SQRJM $|$falscher/(von) L"ugen\\
3.&72.&3915.&250.&15202.&8.&2&31&30\_1 \textcolor{red}{\textcjheb{'l}} LA $|$nicht\\
4.&73.&3916.&252.&15204.&10.&4&165&10\_50\_100\_5 \textcolor{red}{\textcjheb{hqny}} JNQH $|$wird f"ur schuldlos gehalten werden/(er) bleibt ungestraft\\
5.&74.&3917.&256.&15208.&14.&5&114&6\_10\_80\_10\_8 \textcolor{red}{\textcjheb{.hypyw}} WJPJC $|$und wer ausspricht\\
6.&75.&3918.&261.&15213.&19.&5&79&20\_7\_2\_10\_40 \textcolor{red}{\textcjheb{mybzk}} KZBJM $|$L"ugen\\
7.&76.&3919.&266.&15218.&24.&4&17&10\_1\_2\_4 \textcolor{red}{\textcjheb{db'y}} JABD $|$wird umkommen/(er) kommt um\\
\end{tabular}\medskip \\
Ende des Verses 19.9\\
Verse: 534, Buchstaben: 27, 269, 15221, Totalwerte: 1130, 15329, 1084886\\
\\
Ein falscher Zeuge wird nicht f"ur schuldlos gehalten werden, und wer L"ugen ausspricht, wird umkommen.\\
\newpage 
{\bf -- 19.10}\\
\medskip \\
\begin{tabular}{rrrrrrrrp{120mm}}
WV&WK&WB&ABK&ABB&ABV&AnzB&TW&Zahlencode \textcolor{red}{$\boldsymbol{Grundtext}$} Umschrift $|$"Ubersetzung(en)\\
1.&77.&3920.&270.&15222.&1.&2&31&30\_1 \textcolor{red}{\textcjheb{'l}} LA $|$nicht\\
2.&78.&3921.&272.&15224.&3.&4&62&50\_1\_6\_5 \textcolor{red}{\textcjheb{hw'n}} NAWH $|$geziemt/(ist) geziemend\\
3.&79.&3922.&276.&15228.&7.&5&150&30\_20\_60\_10\_30 \textcolor{red}{\textcjheb{lyskl}} LKsJL $|$einem Toren/dem Toren\\
4.&80.&3923.&281.&15233.&12.&5&529&400\_70\_50\_6\_3 \textcolor{red}{\textcjheb{gwn`t}} TaNWG $|$Wohlleben\\
5.&81.&3924.&286.&15238.&17.&2&81&1\_80 \textcolor{red}{\textcjheb{p'}} AP $|$wie viel/auch\\
6.&82.&3925.&288.&15240.&19.&2&30&20\_10 \textcolor{red}{\textcjheb{yk}} KJ $|$weniger/wenn\\
7.&83.&3926.&290.&15242.&21.&4&106&30\_70\_2\_4 \textcolor{red}{\textcjheb{db`l}} LaBD $|$einem Knecht/dem Sklaven\\
8.&84.&3927.&294.&15246.&25.&3&370&40\_300\_30 \textcolor{red}{\textcjheb{l+sm}} MSL $|$zu herrschen/(ein) Herrschen\\
9.&85.&3928.&297.&15249.&28.&5&552&2\_300\_200\_10\_40 \textcolor{red}{\textcjheb{myr+sb}} BSRJM $|$"uber F"ursten\\
\end{tabular}\medskip \\
Ende des Verses 19.10\\
Verse: 535, Buchstaben: 32, 301, 15253, Totalwerte: 1911, 17240, 1086797\\
\\
Nicht geziemt einem Toren Wohlleben; wieviel weniger einem Knechte, "uber F"ursten zu herrschen!\\
\newpage 
{\bf -- 19.11}\\
\medskip \\
\begin{tabular}{rrrrrrrrp{120mm}}
WV&WK&WB&ABK&ABB&ABV&AnzB&TW&Zahlencode \textcolor{red}{$\boldsymbol{Grundtext}$} Umschrift $|$"Ubersetzung(en)\\
1.&86.&3929.&302.&15254.&1.&3&350&300\_20\_30 \textcolor{red}{\textcjheb{lk+s}} SKL $|$die Einsicht/(die) Klugheit\\
2.&87.&3930.&305.&15257.&4.&3&45&1\_4\_40 \textcolor{red}{\textcjheb{md'}} ADM $|$(eines) Menschen\\
3.&88.&3931.&308.&15260.&7.&5&236&5\_1\_200\_10\_20 \textcolor{red}{\textcjheb{kyr'h}} HARJK $|$macht langm"utig/er (=sie) macht lang\\
4.&89.&3932.&313.&15265.&12.&3&87&1\_80\_6 \textcolor{red}{\textcjheb{wp'}} APW $|$ihn/seine Nase\\
5.&90.&3933.&316.&15268.&15.&7&1093&6\_400\_80\_1\_200\_400\_6 \textcolor{red}{\textcjheb{wtr'ptw}} WTPARTW $|$und sein Ruhm/und seine Auszeichnung\\
6.&91.&3934.&323.&15275.&22.&3&272&70\_2\_200 \textcolor{red}{\textcjheb{rb`}} aBR $|$ist es zu "ubersehen/ist ein Hinweggehen\\
7.&92.&3935.&326.&15278.&25.&2&100&70\_30 \textcolor{red}{\textcjheb{l`}} aL $|$/"uber\\
8.&93.&3936.&328.&15280.&27.&3&450&80\_300\_70 \textcolor{red}{\textcjheb{`+sp}} PSa $|$Vergehung/Verfehlung\\
\end{tabular}\medskip \\
Ende des Verses 19.11\\
Verse: 536, Buchstaben: 29, 330, 15282, Totalwerte: 2633, 19873, 1089430\\
\\
Die Einsicht eines Menschen macht ihn langm"utig, und sein Ruhm ist es, Vergehung zu "ubersehen.\\
\newpage 
{\bf -- 19.12}\\
\medskip \\
\begin{tabular}{rrrrrrrrp{120mm}}
WV&WK&WB&ABK&ABB&ABV&AnzB&TW&Zahlencode \textcolor{red}{$\boldsymbol{Grundtext}$} Umschrift $|$"Ubersetzung(en)\\
1.&94.&3937.&331.&15283.&1.&3&95&50\_5\_40 \textcolor{red}{\textcjheb{mhn}} NHM $|$wie das Knurren/(ein) Knurren\\
2.&95.&3938.&334.&15286.&4.&5&330&20\_20\_80\_10\_200 \textcolor{red}{\textcjheb{rypkk}} KKPJR $|$eines jungen L"owen/wie der Jungl"owe\\
3.&96.&3939.&339.&15291.&9.&3&157&7\_70\_80 \textcolor{red}{\textcjheb{p`z}} ZaP $|$(ist) (der) Zorn\\
4.&97.&3940.&342.&15294.&12.&3&90&40\_30\_20 \textcolor{red}{\textcjheb{klm}} MLK $|$des K"onigs\\
5.&98.&3941.&345.&15297.&15.&4&65&6\_20\_9\_30 \textcolor{red}{\textcjheb{l.tkw}} WKtL $|$aber wie Tau/und wie Tau\\
6.&99.&3942.&349.&15301.&19.&2&100&70\_30 \textcolor{red}{\textcjheb{l`}} aL $|$auf\\
7.&100.&3943.&351.&15303.&21.&3&372&70\_300\_2 \textcolor{red}{\textcjheb{b+s`}} aSB $|$das Gras/dem Gras\\
8.&101.&3944.&354.&15306.&24.&5&352&200\_90\_6\_50\_6 \textcolor{red}{\textcjheb{wnw.sr}} R"sWNW $|$sein Wohlgefallen\\
\end{tabular}\medskip \\
Ende des Verses 19.12\\
Verse: 537, Buchstaben: 28, 358, 15310, Totalwerte: 1561, 21434, 1090991\\
\\
Des K"onigs Zorn ist wie das Knurren eines jungen L"owen, aber sein Wohlgefallen wie Tau auf das Gras.\\
\newpage 
{\bf -- 19.13}\\
\medskip \\
\begin{tabular}{rrrrrrrrp{120mm}}
WV&WK&WB&ABK&ABB&ABV&AnzB&TW&Zahlencode \textcolor{red}{$\boldsymbol{Grundtext}$} Umschrift $|$"Ubersetzung(en)\\
1.&102.&3945.&359.&15311.&1.&3&411&5\_6\_400 \textcolor{red}{\textcjheb{twh}} HWT $|$Verderben\\
2.&103.&3946.&362.&15314.&4.&5&49&30\_1\_2\_10\_6 \textcolor{red}{\textcjheb{wyb'l}} LABJW $|$f"ur seinen Vater\\
3.&104.&3947.&367.&15319.&9.&2&52&2\_50 \textcolor{red}{\textcjheb{nb}} BN $|$(ist) (ein) Sohn\\
4.&105.&3948.&369.&15321.&11.&4&120&20\_60\_10\_30 \textcolor{red}{\textcjheb{lysk}} KsJL $|$t"orichter\\
5.&106.&3949.&373.&15325.&15.&4&120&6\_4\_30\_80 \textcolor{red}{\textcjheb{pldw}} WDLP $|$und eine Traufe\\
6.&107.&3950.&377.&15329.&19.&3&213&9\_200\_4 \textcolor{red}{\textcjheb{dr.t}} tRD $|$best"andige/rinnend\\
7.&108.&3951.&380.&15332.&22.&5&114&40\_4\_10\_50\_10 \textcolor{red}{\textcjheb{ynydm}} MDJNJ $|$sind die Z"ankereien/sind Zwistigkeiten\\
8.&109.&3952.&385.&15337.&27.&3&306&1\_300\_5 \textcolor{red}{\textcjheb{h+s'}} ASH $|$(einer) Frau\\
\end{tabular}\medskip \\
Ende des Verses 19.13\\
Verse: 538, Buchstaben: 29, 387, 15339, Totalwerte: 1385, 22819, 1092376\\
\\
Ein t"orichter Sohn ist Verderben f"ur seinen Vater; und die Z"ankereien eines Weibes sind eine best"andige Traufe.\\
\newpage 
{\bf -- 19.14}\\
\medskip \\
\begin{tabular}{rrrrrrrrp{120mm}}
WV&WK&WB&ABK&ABB&ABV&AnzB&TW&Zahlencode \textcolor{red}{$\boldsymbol{Grundtext}$} Umschrift $|$"Ubersetzung(en)\\
1.&110.&3953.&388.&15340.&1.&3&412&2\_10\_400 \textcolor{red}{\textcjheb{tyb}} BJT $|$Haus\\
2.&111.&3954.&391.&15343.&4.&4&67&6\_5\_6\_50 \textcolor{red}{\textcjheb{nwhw}} WHWN $|$und Gut\\
3.&112.&3955.&395.&15347.&8.&4&488&50\_8\_30\_400 \textcolor{red}{\textcjheb{tl.hn}} NCLT $|$sind ein Erbteil/(ist) Erbe\\
4.&113.&3956.&399.&15351.&12.&4&409&1\_2\_6\_400 \textcolor{red}{\textcjheb{twb'}} ABWT $|$der V"ater\\
5.&114.&3957.&403.&15355.&16.&6&72&6\_40\_10\_5\_6\_5 \textcolor{red}{\textcjheb{hwhymw}} WMJHWH $|$aber von Jahwe/und von Jahwe\\
6.&115.&3958.&409.&15361.&22.&3&306&1\_300\_5 \textcolor{red}{\textcjheb{h+s'}} ASH $|$(kommt) (eine) Frau\\
7.&116.&3959.&412.&15364.&25.&5&790&40\_300\_20\_30\_400 \textcolor{red}{\textcjheb{tlk+sm}} MSKLT $|$einsichtsvolle/verst"andige\\
\end{tabular}\medskip \\
Ende des Verses 19.14\\
Verse: 539, Buchstaben: 29, 416, 15368, Totalwerte: 2544, 25363, 1094920\\
\\
Haus und Gut sind ein Erbteil der V"ater, aber eine einsichtsvolle Frau kommt von Jahwe.\\
\newpage 
{\bf -- 19.15}\\
\medskip \\
\begin{tabular}{rrrrrrrrp{120mm}}
WV&WK&WB&ABK&ABB&ABV&AnzB&TW&Zahlencode \textcolor{red}{$\boldsymbol{Grundtext}$} Umschrift $|$"Ubersetzung(en)\\
1.&117.&3960.&417.&15369.&1.&4&195&70\_90\_30\_5 \textcolor{red}{\textcjheb{hl.s`}} a"sLH $|$Faulheit/Tr"agheit\\
2.&118.&3961.&421.&15373.&5.&4&520&400\_80\_10\_30 \textcolor{red}{\textcjheb{lypt}} TPJL $|$versenkt/(sie) macht fallen\\
3.&119.&3962.&425.&15377.&9.&5&649&400\_200\_4\_40\_5 \textcolor{red}{\textcjheb{hmdrt}} TRDMH $|$in tiefen Schlaf/(in) Tiefschlaf\\
4.&120.&3963.&430.&15382.&14.&4&436&6\_50\_80\_300 \textcolor{red}{\textcjheb{+spnw}} WNPS $|$und eine Seele\\
5.&121.&3964.&434.&15386.&18.&4&255&200\_40\_10\_5 \textcolor{red}{\textcjheb{hymr}} RMJH $|$l"assige/(von) L"assigkeit\\
6.&122.&3965.&438.&15390.&22.&4&672&400\_200\_70\_2 \textcolor{red}{\textcjheb{b`rt}} TRaB $|$wird hungern/(sie) muss hungern\\
\end{tabular}\medskip \\
Ende des Verses 19.15\\
Verse: 540, Buchstaben: 25, 441, 15393, Totalwerte: 2727, 28090, 1097647\\
\\
Faulheit versenkt in tiefen Schlaf, und eine l"assige Seele wird hungern.\\
\newpage 
{\bf -- 19.16}\\
\medskip \\
\begin{tabular}{rrrrrrrrp{120mm}}
WV&WK&WB&ABK&ABB&ABV&AnzB&TW&Zahlencode \textcolor{red}{$\boldsymbol{Grundtext}$} Umschrift $|$"Ubersetzung(en)\\
1.&123.&3966.&442.&15394.&1.&3&540&300\_40\_200 \textcolor{red}{\textcjheb{rm+s}} SMR $|$wer bewahrt/(ein) Wahrender\\
2.&124.&3967.&445.&15397.&4.&4&141&40\_90\_6\_5 \textcolor{red}{\textcjheb{hw.sm}} M"sWH $|$(das) Gebot\\
3.&125.&3968.&449.&15401.&8.&3&540&300\_40\_200 \textcolor{red}{\textcjheb{rm+s}} SMR $|$bewahrt/(ist) bewahrend\\
4.&126.&3969.&452.&15404.&11.&4&436&50\_80\_300\_6 \textcolor{red}{\textcjheb{w+spn}} NPSW $|$seine Seele\\
5.&127.&3970.&456.&15408.&15.&4&20&2\_6\_7\_5 \textcolor{red}{\textcjheb{hzwb}} BWZH $|$wer verachtet/wer nicht achtet auf\\
6.&128.&3971.&460.&15412.&19.&5&240&4\_200\_20\_10\_6 \textcolor{red}{\textcjheb{wykrd}} DRKJW $|$seine Wege\\
7.&129.&3972.&465.&15417.&24.&4&456&10\_6\_40\_400 \textcolor{red}{\textcjheb{tmwy}} JWMT $|$wird sterben/(d)er wird get"otet\\
\end{tabular}\medskip \\
Ende des Verses 19.16\\
Verse: 541, Buchstaben: 27, 468, 15420, Totalwerte: 2373, 30463, 1100020\\
\\
Wer das Gebot bewahrt, bewahrt seine Seele; wer seine Wege verachtet, wird sterben.\\
\newpage 
{\bf -- 19.17}\\
\medskip \\
\begin{tabular}{rrrrrrrrp{120mm}}
WV&WK&WB&ABK&ABB&ABV&AnzB&TW&Zahlencode \textcolor{red}{$\boldsymbol{Grundtext}$} Umschrift $|$"Ubersetzung(en)\\
1.&130.&3973.&469.&15421.&1.&4&81&40\_30\_6\_5 \textcolor{red}{\textcjheb{hwlm}} MLWH $|$(es) leiht/wer leiht\\
2.&131.&3974.&473.&15425.&5.&4&26&10\_5\_6\_5 \textcolor{red}{\textcjheb{hwhy}} JHWH $|$(an) Jahwe\\
3.&132.&3975.&477.&15429.&9.&4&114&8\_6\_50\_50 \textcolor{red}{\textcjheb{nnw.h}} CWNN $|$wer sich erbarmt/(ist) ein sich Erbarmender\\
4.&133.&3976.&481.&15433.&13.&2&34&4\_30 \textcolor{red}{\textcjheb{ld}} DL $|$des Armen/des Bed"urftigen\\
5.&134.&3977.&483.&15435.&15.&5&85&6\_3\_40\_30\_6 \textcolor{red}{\textcjheb{wlmgw}} WGMLW $|$und seine (Wohl)Tat\\
6.&135.&3978.&488.&15440.&20.&4&380&10\_300\_30\_40 \textcolor{red}{\textcjheb{ml+sy}} JSLM $|$wird er vergelten/er vergilt\\
7.&136.&3979.&492.&15444.&24.&2&36&30\_6 \textcolor{red}{\textcjheb{wl}} LW $|$ihm\\
\end{tabular}\medskip \\
Ende des Verses 19.17\\
Verse: 542, Buchstaben: 25, 493, 15445, Totalwerte: 756, 31219, 1100776\\
\\
Wer des Armen sich erbarmt, leiht Jahwe; und er wird ihm seine Wohltat vergelten.\\
\newpage 
{\bf -- 19.18}\\
\medskip \\
\begin{tabular}{rrrrrrrrp{120mm}}
WV&WK&WB&ABK&ABB&ABV&AnzB&TW&Zahlencode \textcolor{red}{$\boldsymbol{Grundtext}$} Umschrift $|$"Ubersetzung(en)\\
1.&137.&3980.&494.&15446.&1.&3&270&10\_60\_200 \textcolor{red}{\textcjheb{rsy}} JsR $|$z"uchtige\\
2.&138.&3981.&497.&15449.&4.&3&72&2\_50\_20 \textcolor{red}{\textcjheb{knb}} BNK $|$deinen Sohn\\
3.&139.&3982.&500.&15452.&7.&2&30&20\_10 \textcolor{red}{\textcjheb{yk}} KJ $|$weil noch/wenn\\
4.&140.&3983.&502.&15454.&9.&2&310&10\_300 \textcolor{red}{\textcjheb{+sy}} JS $|$da ist/es gibt\\
5.&141.&3984.&504.&15456.&11.&4&511&400\_100\_6\_5 \textcolor{red}{\textcjheb{hwqt}} TQWH $|$Hoffnung\\
6.&142.&3985.&508.&15460.&15.&3&37&6\_1\_30 \textcolor{red}{\textcjheb{l'w}} WAL $|$aber zu/und zu\\
7.&143.&3986.&511.&15463.&18.&5&461&5\_40\_10\_400\_6 \textcolor{red}{\textcjheb{wtymh}} HMJTW $|$t"oten ihn\\
8.&144.&3987.&516.&15468.&23.&2&31&1\_30 \textcolor{red}{\textcjheb{l'}} AL $|$nicht\\
9.&145.&3988.&518.&15470.&25.&3&701&400\_300\_1 \textcolor{red}{\textcjheb{'+st}} TSA $|$trachte/er (=es) soll sich erheben\\
10.&146.&3989.&521.&15473.&28.&4&450&50\_80\_300\_20 \textcolor{red}{\textcjheb{k+spn}} NPSK $|$danach/deine Seele\\
\end{tabular}\medskip \\
Ende des Verses 19.18\\
Verse: 543, Buchstaben: 31, 524, 15476, Totalwerte: 2873, 34092, 1103649\\
\\
Z"uchtige deinen Sohn, weil noch Hoffnung da ist; aber trachte nicht danach, ihn zu t"oten.\\
\newpage 
{\bf -- 19.19}\\
\medskip \\
\begin{tabular}{rrrrrrrrp{120mm}}
WV&WK&WB&ABK&ABB&ABV&AnzB&TW&Zahlencode \textcolor{red}{$\boldsymbol{Grundtext}$} Umschrift $|$"Ubersetzung(en)\\
1.&147.&3990.&525.&15477.&1.&3&233&3\_200\_30 \textcolor{red}{\textcjheb{lrg}} GRL $|$wer ist/(ein) Gro"ser\\
2.&148.&3991.&528.&15480.&4.&3&53&8\_40\_5 \textcolor{red}{\textcjheb{hm.h}} CMH $|$j"ahzornig/im Zorn\\
3.&149.&3992.&531.&15483.&7.&3&351&50\_300\_1 \textcolor{red}{\textcjheb{'+sn}} NSA $|$muss/(ist) davontragend\\
4.&150.&3993.&534.&15486.&10.&3&420&70\_50\_300 \textcolor{red}{\textcjheb{+sn`}} aNS $|$b"u"sen daf"ur/(eine) Geldbu"se\\
5.&151.&3994.&537.&15489.&13.&2&30&20\_10 \textcolor{red}{\textcjheb{yk}} KJ $|$denn\\
6.&152.&3995.&539.&15491.&15.&2&41&1\_40 \textcolor{red}{\textcjheb{m'}} AM $|$auch/wenn\\
7.&153.&3996.&541.&15493.&17.&4&530&400\_90\_10\_30 \textcolor{red}{\textcjheb{ly.st}} T"sJL $|$du greifst ein/du willst retten\\
8.&154.&3997.&545.&15497.&21.&4&86&6\_70\_6\_4 \textcolor{red}{\textcjheb{dw`w}} WaWD $|$so noch schlimmer/und noch\\
9.&155.&3998.&549.&15501.&25.&4&546&400\_6\_60\_80 \textcolor{red}{\textcjheb{pswt}} TWsP $|$machst du es/du musst hinzuf"ugen\\
\end{tabular}\medskip \\
Ende des Verses 19.19\\
Verse: 544, Buchstaben: 28, 552, 15504, Totalwerte: 2290, 36382, 1105939\\
\\
Wer j"ahzornig ist, mu"s daf"ur b"u"sen; denn greifst du auch ein, so machst du's nur noch schlimmer.\\
\newpage 
{\bf -- 19.20}\\
\medskip \\
\begin{tabular}{rrrrrrrrp{120mm}}
WV&WK&WB&ABK&ABB&ABV&AnzB&TW&Zahlencode \textcolor{red}{$\boldsymbol{Grundtext}$} Umschrift $|$"Ubersetzung(en)\\
1.&156.&3999.&553.&15505.&1.&3&410&300\_40\_70 \textcolor{red}{\textcjheb{`m+s}} SMa $|$h"ore\\
2.&157.&4000.&556.&15508.&4.&3&165&70\_90\_5 \textcolor{red}{\textcjheb{h.s`}} a"sH $|$(auf) (den) Rat\\
3.&158.&4001.&559.&15511.&7.&4&138&6\_100\_2\_30 \textcolor{red}{\textcjheb{lbqw}} WQBL $|$und nimm an\\
4.&159.&4002.&563.&15515.&11.&4&306&40\_6\_60\_200 \textcolor{red}{\textcjheb{rswm}} MWsR $|$Unterweisung/Zucht\\
5.&160.&4003.&567.&15519.&15.&4&190&30\_40\_70\_50 \textcolor{red}{\textcjheb{n`ml}} LMaN $|$damit\\
6.&161.&4004.&571.&15523.&19.&4&468&400\_8\_20\_40 \textcolor{red}{\textcjheb{mk.ht}} TCKM $|$du weise seiest/du wirst weise\\
7.&162.&4005.&575.&15527.&23.&7&641&2\_1\_8\_200\_10\_400\_20 \textcolor{red}{\textcjheb{ktyr.h'b}} BACRJTK $|$in (der) Zukunft\\
\end{tabular}\medskip \\
Ende des Verses 19.20\\
Verse: 545, Buchstaben: 29, 581, 15533, Totalwerte: 2318, 38700, 1108257\\
\\
H"ore auf Rat und nimm Unterweisung an, damit du weise seiest in der Zukunft.\\
\newpage 
{\bf -- 19.21}\\
\medskip \\
\begin{tabular}{rrrrrrrrp{120mm}}
WV&WK&WB&ABK&ABB&ABV&AnzB&TW&Zahlencode \textcolor{red}{$\boldsymbol{Grundtext}$} Umschrift $|$"Ubersetzung(en)\\
1.&163.&4006.&582.&15534.&1.&4&608&200\_2\_6\_400 \textcolor{red}{\textcjheb{twbr}} RBWT $|$viele\\
2.&164.&4007.&586.&15538.&5.&6&756&40\_8\_300\_2\_6\_400 \textcolor{red}{\textcjheb{twb+s.hm}} MCSBWT $|$Gedanken/Pl"ane\\
3.&165.&4008.&592.&15544.&11.&3&34&2\_30\_2 \textcolor{red}{\textcjheb{blb}} BLB $|$(sind) im Herzen\\
4.&166.&4009.&595.&15547.&14.&3&311&1\_10\_300 \textcolor{red}{\textcjheb{+sy'}} AJS $|$(eines) Mannes\\
5.&167.&4010.&598.&15550.&17.&4&566&6\_70\_90\_400 \textcolor{red}{\textcjheb{t.s`w}} Wa"sT $|$aber der Ratschluss/und der Ratschluss\\
6.&168.&4011.&602.&15554.&21.&4&26&10\_5\_6\_5 \textcolor{red}{\textcjheb{hwhy}} JHWH $|$Jahwe(s)\\
7.&169.&4012.&606.&15558.&25.&3&16&5\_10\_1 \textcolor{red}{\textcjheb{'yh}} HJA $|$er/der\\
8.&170.&4013.&609.&15561.&28.&4&546&400\_100\_6\_40 \textcolor{red}{\textcjheb{mwqt}} TQWM $|$kommt zustande/hat Bestand\\
\end{tabular}\medskip \\
Ende des Verses 19.21\\
Verse: 546, Buchstaben: 31, 612, 15564, Totalwerte: 2863, 41563, 1111120\\
\\
Viele Gedanken sind in dem Herzen eines Mannes; aber der Ratschlu"s Jahwes, er kommt zustande.\\
\newpage 
{\bf -- 19.22}\\
\medskip \\
\begin{tabular}{rrrrrrrrp{120mm}}
WV&WK&WB&ABK&ABB&ABV&AnzB&TW&Zahlencode \textcolor{red}{$\boldsymbol{Grundtext}$} Umschrift $|$"Ubersetzung(en)\\
1.&171.&4014.&613.&15565.&1.&4&807&400\_1\_6\_400 \textcolor{red}{\textcjheb{tw't}} TAWT $|$die Willigkeit/das Verlangen\\
2.&172.&4015.&617.&15569.&5.&3&45&1\_4\_40 \textcolor{red}{\textcjheb{md'}} ADM $|$des Menschen/(beim) Menschen\\
3.&173.&4016.&620.&15572.&8.&4&78&8\_60\_4\_6 \textcolor{red}{\textcjheb{wds.h}} CsDW $|$macht seine Mildt"atigkeit aus/(ist) seine Freundlichkeit\\
4.&174.&4017.&624.&15576.&12.&4&23&6\_9\_6\_2 \textcolor{red}{\textcjheb{bw.tw}} WtWB $|$und besser (ist)\\
5.&175.&4018.&628.&15580.&16.&2&500&200\_300 \textcolor{red}{\textcjheb{+sr}} RS $|$(ein) Armer\\
6.&176.&4019.&630.&15582.&18.&4&351&40\_1\_10\_300 \textcolor{red}{\textcjheb{+sy'm}} MAJS $|$(als ein) Mann\\
7.&177.&4020.&634.&15586.&22.&3&29&20\_7\_2 \textcolor{red}{\textcjheb{bzk}} KZB $|$l"ugnerischer/(der) L"uge\\
\end{tabular}\medskip \\
Ende des Verses 19.22\\
Verse: 547, Buchstaben: 24, 636, 15588, Totalwerte: 1833, 43396, 1112953\\
\\
Die Willigkeit des Menschen macht seine Mildt"atigkeit aus, und besser ein Armer als ein l"ugnerischer Mann.\\
\newpage 
{\bf -- 19.23}\\
\medskip \\
\begin{tabular}{rrrrrrrrp{120mm}}
WV&WK&WB&ABK&ABB&ABV&AnzB&TW&Zahlencode \textcolor{red}{$\boldsymbol{Grundtext}$} Umschrift $|$"Ubersetzung(en)\\
1.&178.&4021.&637.&15589.&1.&4&611&10\_200\_1\_400 \textcolor{red}{\textcjheb{t'ry}} JRAT $|$(die) Furcht\\
2.&179.&4022.&641.&15593.&5.&4&26&10\_5\_6\_5 \textcolor{red}{\textcjheb{hwhy}} JHWH $|$(vor) Jahwe(s)\\
3.&180.&4023.&645.&15597.&9.&5&98&30\_8\_10\_10\_40 \textcolor{red}{\textcjheb{myy.hl}} LCJJM $|$ist zum Leben/(f"uhrt) zum Leben\\
4.&181.&4024.&650.&15602.&14.&4&378&6\_300\_2\_70 \textcolor{red}{\textcjheb{`b+sw}} WSBa $|$und ges"attigt\\
5.&182.&4025.&654.&15606.&18.&4&100&10\_30\_10\_50 \textcolor{red}{\textcjheb{nyly}} JLJN $|$verbringt man die Nacht/er ruht\\
6.&183.&4026.&658.&15610.&22.&2&32&2\_30 \textcolor{red}{\textcjheb{lb}} BL $|$nicht\\
7.&184.&4027.&660.&15612.&24.&4&194&10\_80\_100\_4 \textcolor{red}{\textcjheb{dqpy}} JPQD $|$(er) wird heimgesucht\\
8.&185.&4028.&664.&15616.&28.&2&270&200\_70 \textcolor{red}{\textcjheb{`r}} Ra $|$(von) (dem) "Ubel\\
\end{tabular}\medskip \\
Ende des Verses 19.23\\
Verse: 548, Buchstaben: 29, 665, 15617, Totalwerte: 1709, 45105, 1114662\\
\\
Die Furcht Jahwes ist zum Leben; und ges"attigt verbringt man die Nacht, wird nicht heimgesucht vom "Ubel.\\
\newpage 
{\bf -- 19.24}\\
\medskip \\
\begin{tabular}{rrrrrrrrp{120mm}}
WV&WK&WB&ABK&ABB&ABV&AnzB&TW&Zahlencode \textcolor{red}{$\boldsymbol{Grundtext}$} Umschrift $|$"Ubersetzung(en)\\
1.&186.&4029.&666.&15618.&1.&3&99&9\_40\_50 \textcolor{red}{\textcjheb{nm.t}} tMN $|$hat gesteckt/er (=es) streckte\\
2.&187.&4030.&669.&15621.&4.&3&190&70\_90\_30 \textcolor{red}{\textcjheb{l.s`}} a"sL $|$der Faule/(der) Faulpelz\\
3.&188.&4031.&672.&15624.&7.&3&20&10\_4\_6 \textcolor{red}{\textcjheb{wdy}} JDW $|$seine Hand\\
4.&189.&4032.&675.&15627.&10.&5&530&2\_90\_30\_8\_400 \textcolor{red}{\textcjheb{t.hl.sb}} B"sLCT $|$in die Sch"ussel\\
5.&190.&4033.&680.&15632.&15.&2&43&3\_40 \textcolor{red}{\textcjheb{mg}} GM $|$einmal/doch\\
6.&191.&4034.&682.&15634.&17.&2&31&1\_30 \textcolor{red}{\textcjheb{l'}} AL $|$zu\\
7.&192.&4035.&684.&15636.&19.&4&101&80\_10\_5\_6 \textcolor{red}{\textcjheb{whyp}} PJHW $|$seinem Mund\\
8.&193.&4036.&688.&15640.&23.&2&31&30\_1 \textcolor{red}{\textcjheb{'l}} LA $|$nicht\\
9.&194.&4037.&690.&15642.&25.&6&377&10\_300\_10\_2\_50\_5 \textcolor{red}{\textcjheb{hnby+sy}} JSJBNH $|$bringt er sie zur"uck/er f"uhrt zur"uck sie\\
\end{tabular}\medskip \\
Ende des Verses 19.24\\
Verse: 549, Buchstaben: 30, 695, 15647, Totalwerte: 1422, 46527, 1116084\\
\\
Hat der Faule seine Hand in die Sch"ussel gesteckt, nicht einmal zu seinem Munde bringt er sie zur"uck.\\
\newpage 
{\bf -- 19.25}\\
\medskip \\
\begin{tabular}{rrrrrrrrp{120mm}}
WV&WK&WB&ABK&ABB&ABV&AnzB&TW&Zahlencode \textcolor{red}{$\boldsymbol{Grundtext}$} Umschrift $|$"Ubersetzung(en)\\
1.&195.&4038.&696.&15648.&1.&2&120&30\_90 \textcolor{red}{\textcjheb{.sl}} L"s $|$den Sp"otter\\
2.&196.&4039.&698.&15650.&3.&3&425&400\_20\_5 \textcolor{red}{\textcjheb{hkt}} TKH $|$schl"agst du\\
3.&197.&4040.&701.&15653.&6.&4&496&6\_80\_400\_10 \textcolor{red}{\textcjheb{ytpw}} WPTJ $|$so der Einf"altige/und ein Einf"altiger\\
4.&198.&4041.&705.&15657.&10.&4&320&10\_70\_200\_40 \textcolor{red}{\textcjheb{mr`y}} JaRM $|$(er) wird klug\\
5.&199.&4042.&709.&15661.&14.&6&55&6\_5\_6\_20\_10\_8 \textcolor{red}{\textcjheb{.hykwhw}} WHWKJC $|$und weist man zurecht/und man zurechtwies\\
6.&200.&4043.&715.&15667.&20.&5&138&30\_50\_2\_6\_50 \textcolor{red}{\textcjheb{nwbnl}} LNBWN $|$den Verst"andigen/den Einsichtsvollen\\
7.&201.&4044.&720.&15672.&25.&4&72&10\_2\_10\_50 \textcolor{red}{\textcjheb{nyby}} JBJN $|$so wird er verstehen/er er versteht\\
8.&202.&4045.&724.&15676.&29.&3&474&4\_70\_400 \textcolor{red}{\textcjheb{t`d}} DaT $|$Erkenntnis/Kenntnis\\
\end{tabular}\medskip \\
Ende des Verses 19.25\\
Verse: 550, Buchstaben: 31, 726, 15678, Totalwerte: 2100, 48627, 1118184\\
\\
Schl"agst du den Sp"otter, so wird der Einf"altige klug; und weist man den Verst"andigen zurecht, so wird er Erkenntnis verstehen.\\
\newpage 
{\bf -- 19.26}\\
\medskip \\
\begin{tabular}{rrrrrrrrp{120mm}}
WV&WK&WB&ABK&ABB&ABV&AnzB&TW&Zahlencode \textcolor{red}{$\boldsymbol{Grundtext}$} Umschrift $|$"Ubersetzung(en)\\
1.&203.&4046.&727.&15679.&1.&4&348&40\_300\_4\_4 \textcolor{red}{\textcjheb{dd+sm}} MSDD $|$wer zugrunde richtet/ein Gewalt (An)tuender\\
2.&204.&4047.&731.&15683.&5.&2&3&1\_2 \textcolor{red}{\textcjheb{b'}} AB $|$den Vater/dem Vater\\
3.&205.&4048.&733.&15685.&7.&5&230&10\_2\_200\_10\_8 \textcolor{red}{\textcjheb{.hyrby}} JBRJC $|$verjagt/er macht fliehen\\
4.&206.&4049.&738.&15690.&12.&2&41&1\_40 \textcolor{red}{\textcjheb{m'}} AM $|$die Mutter\\
5.&207.&4050.&740.&15692.&14.&2&52&2\_50 \textcolor{red}{\textcjheb{nb}} BN $|$(ist) (ein) Sohn\\
6.&208.&4051.&742.&15694.&16.&4&352&40\_2\_10\_300 \textcolor{red}{\textcjheb{+sybm}} MBJS $|$der Schande/besch"amender\\
7.&209.&4052.&746.&15698.&20.&6&344&6\_40\_8\_80\_10\_200 \textcolor{red}{\textcjheb{ryp.hmw}} WMCPJR $|$und Schmach bringt/und sch"andlich handelnder\\
\end{tabular}\medskip \\
Ende des Verses 19.26\\
Verse: 551, Buchstaben: 25, 751, 15703, Totalwerte: 1370, 49997, 1119554\\
\\
Wer den Vater zu Grunde richtet, die Mutter verjagt, ist ein Sohn, der Schande und Schmach bringt.\\
\newpage 
{\bf -- 19.27}\\
\medskip \\
\begin{tabular}{rrrrrrrrp{120mm}}
WV&WK&WB&ABK&ABB&ABV&AnzB&TW&Zahlencode \textcolor{red}{$\boldsymbol{Grundtext}$} Umschrift $|$"Ubersetzung(en)\\
1.&210.&4053.&752.&15704.&1.&3&42&8\_4\_30 \textcolor{red}{\textcjheb{ld.h}} CDL $|$lass ab\\
2.&211.&4054.&755.&15707.&4.&3&62&2\_50\_10 \textcolor{red}{\textcjheb{ynb}} BNJ $|$mein Sohn\\
3.&212.&4055.&758.&15710.&7.&4&440&30\_300\_40\_70 \textcolor{red}{\textcjheb{`m+sl}} LSMa $|$zu h"oren\\
4.&213.&4056.&762.&15714.&11.&4&306&40\_6\_60\_200 \textcolor{red}{\textcjheb{rswm}} MWsR $|$auf Unterweisung/(auf) eine Mahnung\\
5.&214.&4057.&766.&15718.&15.&5&739&30\_300\_3\_6\_400 \textcolor{red}{\textcjheb{twg+sl}} LSGWT $|$die abirren macht/zum Abirren\\
6.&215.&4058.&771.&15723.&20.&5&291&40\_1\_40\_200\_10 \textcolor{red}{\textcjheb{yrm'm}} MAMRJ $|$von (den) Worten\\
7.&216.&4059.&776.&15728.&25.&3&474&4\_70\_400 \textcolor{red}{\textcjheb{t`d}} DaT $|$(der) Erkenntnis (f"uhrt es)\\
\end{tabular}\medskip \\
Ende des Verses 19.27\\
Verse: 552, Buchstaben: 27, 778, 15730, Totalwerte: 2354, 52351, 1121908\\
\\
La"s ab, mein Sohn, auf Unterweisung zu h"oren, die abirren macht von den Worten der Erkenntnis.\\
\newpage 
{\bf -- 19.28}\\
\medskip \\
\begin{tabular}{rrrrrrrrp{120mm}}
WV&WK&WB&ABK&ABB&ABV&AnzB&TW&Zahlencode \textcolor{red}{$\boldsymbol{Grundtext}$} Umschrift $|$"Ubersetzung(en)\\
1.&217.&4060.&779.&15731.&1.&2&74&70\_4 \textcolor{red}{\textcjheb{d`}} aD $|$(ein) Zeuge\\
2.&218.&4061.&781.&15733.&3.&5&142&2\_30\_10\_70\_30 \textcolor{red}{\textcjheb{l`ylb}} BLJaL $|$Belials/nichtsnutziger\\
3.&219.&4062.&786.&15738.&8.&4&140&10\_30\_10\_90 \textcolor{red}{\textcjheb{.syly}} JLJ"s $|$(er) spottet\\
4.&220.&4063.&790.&15742.&12.&4&429&40\_300\_80\_9 \textcolor{red}{\textcjheb{.tp+sm}} MSPt $|$des Rechts/das Recht\\
5.&221.&4064.&794.&15746.&16.&3&96&6\_80\_10 \textcolor{red}{\textcjheb{ypw}} WPJ $|$und der Mund\\
6.&222.&4065.&797.&15749.&19.&5&620&200\_300\_70\_10\_40 \textcolor{red}{\textcjheb{my`+sr}} RSaJM $|$der Gesetzlosen/(der) Frevler\\
7.&223.&4066.&802.&15754.&24.&4&112&10\_2\_30\_70 \textcolor{red}{\textcjheb{`lby}} JBLa $|$(er) verschlingt\\
8.&224.&4067.&806.&15758.&28.&3&57&1\_6\_50 \textcolor{red}{\textcjheb{nw'}} AWN $|$Unheil/Unrecht\\
\end{tabular}\medskip \\
Ende des Verses 19.28\\
Verse: 553, Buchstaben: 30, 808, 15760, Totalwerte: 1670, 54021, 1123578\\
\\
Ein Belialszeuge spottet des Rechts, und der Mund der Gesetzlosen verschlingt Unheil.\\
\newpage 
{\bf -- 19.29}\\
\medskip \\
\begin{tabular}{rrrrrrrrp{120mm}}
WV&WK&WB&ABK&ABB&ABV&AnzB&TW&Zahlencode \textcolor{red}{$\boldsymbol{Grundtext}$} Umschrift $|$"Ubersetzung(en)\\
1.&225.&4068.&809.&15761.&1.&5&132&50\_20\_6\_50\_6 \textcolor{red}{\textcjheb{wnwkn}} NKWNW $|$(sie (=es)) sind bereit\\
2.&226.&4069.&814.&15766.&6.&5&200&30\_30\_90\_10\_40 \textcolor{red}{\textcjheb{my.sll}} LL"sJM $|$(f"ur) die Sp"otter\\
3.&227.&4070.&819.&15771.&11.&5&439&300\_80\_9\_10\_40 \textcolor{red}{\textcjheb{my.tp+s}} SPtJM $|$(Straf)Gerichte\\
4.&228.&4071.&824.&15776.&16.&7&527&6\_40\_5\_30\_40\_6\_400 \textcolor{red}{\textcjheb{twmlhmw}} WMHLMWT $|$und Schl"age\\
5.&229.&4072.&831.&15783.&23.&3&39&30\_3\_6 \textcolor{red}{\textcjheb{wgl}} LGW $|$f"ur den R"ucken\\
6.&230.&4073.&834.&15786.&26.&6&170&20\_60\_10\_30\_10\_40 \textcolor{red}{\textcjheb{mylysk}} KsJLJM $|$der Toren\\
\end{tabular}\medskip \\
Ende des Verses 19.29\\
Verse: 554, Buchstaben: 31, 839, 15791, Totalwerte: 1507, 55528, 1125085\\
\\
F"ur die Sp"otter sind Gerichte bereit, und Schl"age f"ur den R"ucken der Toren.\\
\\
{\bf Ende des Kapitels 19}\\
\newpage 
{\bf -- 20.1}\\
\medskip \\
\begin{tabular}{rrrrrrrrp{120mm}}
WV&WK&WB&ABK&ABB&ABV&AnzB&TW&Zahlencode \textcolor{red}{$\boldsymbol{Grundtext}$} Umschrift $|$"Ubersetzung(en)\\
1.&1.&4074.&1.&15792.&1.&2&120&30\_90 \textcolor{red}{\textcjheb{.sl}} L"s $|$ein Sp"otter\\
2.&2.&4075.&3.&15794.&3.&4&75&5\_10\_10\_50 \textcolor{red}{\textcjheb{nyyh}} HJJN $|$(ist) der Wein\\
3.&3.&4076.&7.&15798.&7.&3&50&5\_40\_5 \textcolor{red}{\textcjheb{hmh}} HMH $|$(ein) L"arm(mach)er\\
4.&4.&4077.&10.&15801.&10.&3&520&300\_20\_200 \textcolor{red}{\textcjheb{rk+s}} SKR $|$starkes Getr"ank/(der) Rauschtrank\\
5.&5.&4078.&13.&15804.&13.&3&56&6\_20\_30 \textcolor{red}{\textcjheb{lkw}} WKL $|$und jeder\\
6.&6.&4079.&16.&15807.&16.&3&308&300\_3\_5 \textcolor{red}{\textcjheb{hg+s}} SGH $|$der taumelt/Taumelnde\\
7.&7.&4080.&19.&15810.&19.&2&8&2\_6 \textcolor{red}{\textcjheb{wb}} BW $|$davon\\
8.&8.&4081.&21.&15812.&21.&2&31&30\_1 \textcolor{red}{\textcjheb{'l}} LA $|$nicht\\
9.&9.&4082.&23.&15814.&23.&4&78&10\_8\_20\_40 \textcolor{red}{\textcjheb{mk.hy}} JCKM $|$(er) wird weise\\
\end{tabular}\medskip \\
Ende des Verses 20.1\\
Verse: 555, Buchstaben: 26, 26, 15817, Totalwerte: 1246, 1246, 1126331\\
\\
Der Wein ist ein Sp"otter, starkes Getr"ank ein L"armer; und jeder, der davon taumelt, wird nicht weise.\\
\newpage 
{\bf -- 20.2}\\
\medskip \\
\begin{tabular}{rrrrrrrrp{120mm}}
WV&WK&WB&ABK&ABB&ABV&AnzB&TW&Zahlencode \textcolor{red}{$\boldsymbol{Grundtext}$} Umschrift $|$"Ubersetzung(en)\\
1.&10.&4083.&27.&15818.&1.&3&95&50\_5\_40 \textcolor{red}{\textcjheb{mhn}} NHM $|$(wie das) Knurren\\
2.&11.&4084.&30.&15821.&4.&5&330&20\_20\_80\_10\_200 \textcolor{red}{\textcjheb{rypkk}} KKPJR $|$eines jungen L"owen/wie der Jungl"owe\\
3.&12.&4085.&35.&15826.&9.&4&451&1\_10\_40\_400 \textcolor{red}{\textcjheb{tmy'}} AJMT $|$ist (der) Schrecken/(ist) die Schrecklichkeit\\
4.&13.&4086.&39.&15830.&13.&3&90&40\_30\_20 \textcolor{red}{\textcjheb{klm}} MLK $|$des K"onigs\\
5.&14.&4087.&42.&15833.&16.&6&718&40\_400\_70\_2\_200\_6 \textcolor{red}{\textcjheb{wrb`tm}} MTaBRW $|$wer ihn aufbringt gegen sich/wer ihn erz"urnt\\
6.&15.&4088.&48.&15839.&22.&4&24&8\_6\_9\_1 \textcolor{red}{\textcjheb{'.tw.h}} CWtA $|$verwirkt/(ist) verwirkend\\
7.&16.&4089.&52.&15843.&26.&4&436&50\_80\_300\_6 \textcolor{red}{\textcjheb{w+spn}} NPSW $|$sein Leben/seine Seele\\
\end{tabular}\medskip \\
Ende des Verses 20.2\\
Verse: 556, Buchstaben: 29, 55, 15846, Totalwerte: 2144, 3390, 1128475\\
\\
Des K"onigs Schrecken ist wie das Knurren eines jungen L"owen; wer ihn gegen sich aufbringt, verwirkt sein Leben.\\
\newpage 
{\bf -- 20.3}\\
\medskip \\
\begin{tabular}{rrrrrrrrp{120mm}}
WV&WK&WB&ABK&ABB&ABV&AnzB&TW&Zahlencode \textcolor{red}{$\boldsymbol{Grundtext}$} Umschrift $|$"Ubersetzung(en)\\
1.&17.&4090.&56.&15847.&1.&4&32&20\_2\_6\_4 \textcolor{red}{\textcjheb{dwbk}} KBWD $|$(eine) Ehre (ist)\\
2.&18.&4091.&60.&15851.&5.&4&341&30\_1\_10\_300 \textcolor{red}{\textcjheb{+sy'l}} LAJS $|$es dem Mann/f"ur den Mann\\
3.&19.&4092.&64.&15855.&9.&3&702&300\_2\_400 \textcolor{red}{\textcjheb{tb+s}} SBT $|$abzustehen/ein Ablassen\\
4.&20.&4093.&67.&15858.&12.&4&252&40\_200\_10\_2 \textcolor{red}{\textcjheb{byrm}} MRJB $|$vom Streit\\
5.&21.&4094.&71.&15862.&16.&3&56&6\_20\_30 \textcolor{red}{\textcjheb{lkw}} WKL $|$wer ist/und jeglicher\\
6.&22.&4095.&74.&15865.&19.&4&47&1\_6\_10\_30 \textcolor{red}{\textcjheb{lyw'}} AWJL $|$(ein) Narr\\
7.&23.&4096.&78.&15869.&23.&5&513&10\_400\_3\_30\_70 \textcolor{red}{\textcjheb{`lgty}} JTGLa $|$st"urzt sich hinein/(er) bricht los\\
\end{tabular}\medskip \\
Ende des Verses 20.3\\
Verse: 557, Buchstaben: 27, 82, 15873, Totalwerte: 1943, 5333, 1130418\\
\\
Ehre ist es dem Manne, vom Streite abzustehen; wer ein Narr ist, st"urzt sich hinein.\\
\newpage 
{\bf -- 20.4}\\
\medskip \\
\begin{tabular}{rrrrrrrrp{120mm}}
WV&WK&WB&ABK&ABB&ABV&AnzB&TW&Zahlencode \textcolor{red}{$\boldsymbol{Grundtext}$} Umschrift $|$"Ubersetzung(en)\\
1.&24.&4097.&83.&15874.&1.&4&328&40\_8\_200\_80 \textcolor{red}{\textcjheb{pr.hm}} MCRP $|$wegen des Winters/von Herbst (an)\\
2.&25.&4098.&87.&15878.&5.&3&190&70\_90\_30 \textcolor{red}{\textcjheb{l.s`}} a"sL $|$der Faule/(der) Faulpelz\\
3.&26.&4099.&90.&15881.&8.&2&31&30\_1 \textcolor{red}{\textcjheb{'l}} LA $|$nicht\\
4.&27.&4100.&92.&15883.&10.&4&518&10\_8\_200\_300 \textcolor{red}{\textcjheb{+sr.hy}} JCRS $|$mag pfl"ugen/(er) pfl"ugt\\
5.&28.&4101.&96.&15887.&14.&4&341&10\_300\_1\_30 \textcolor{red}{\textcjheb{l'+sy}} JSAL $|$er wird begehren/er sucht\\
6.&29.&4102.&100.&15891.&18.&5&402&2\_100\_90\_10\_200 \textcolor{red}{\textcjheb{ry.sqb}} BQ"sJR $|$zur Erntezeit/in der Ernte\\
7.&30.&4103.&105.&15896.&23.&4&67&6\_1\_10\_50 \textcolor{red}{\textcjheb{ny'w}} WAJN $|$und nichts (ist) da\\
\end{tabular}\medskip \\
Ende des Verses 20.4\\
Verse: 558, Buchstaben: 26, 108, 15899, Totalwerte: 1877, 7210, 1132295\\
\\
Wegen des Winters mag der Faule nicht pfl"ugen; zur Erntezeit wird er begehren, und nichts ist da.\\
\newpage 
{\bf -- 20.5}\\
\medskip \\
\begin{tabular}{rrrrrrrrp{120mm}}
WV&WK&WB&ABK&ABB&ABV&AnzB&TW&Zahlencode \textcolor{red}{$\boldsymbol{Grundtext}$} Umschrift $|$"Ubersetzung(en)\\
1.&31.&4104.&109.&15900.&1.&3&90&40\_10\_40 \textcolor{red}{\textcjheb{mym}} MJM $|$Wasser\\
2.&32.&4105.&112.&15903.&4.&5&260&70\_40\_100\_10\_40 \textcolor{red}{\textcjheb{myqm`}} aMQJM $|$tiefes\\
3.&33.&4106.&117.&15908.&9.&3&165&70\_90\_5 \textcolor{red}{\textcjheb{h.s`}} a"sH $|$(ist der) Ratschluss\\
4.&34.&4107.&120.&15911.&12.&3&34&2\_30\_2 \textcolor{red}{\textcjheb{blb}} BLB $|$im Herzen\\
5.&35.&4108.&123.&15914.&15.&3&311&1\_10\_300 \textcolor{red}{\textcjheb{+sy'}} AJS $|$des Mannes/(eines) Mannes\\
6.&36.&4109.&126.&15917.&18.&4&317&6\_1\_10\_300 \textcolor{red}{\textcjheb{+sy'w}} WAJS $|$aber ein Mann/und (ein) Mann\\
7.&37.&4110.&130.&15921.&22.&5&463&400\_2\_6\_50\_5 \textcolor{red}{\textcjheb{hnwbt}} TBWNH $|$verst"andiger/(von) Einsicht\\
8.&38.&4111.&135.&15926.&27.&5&99&10\_4\_30\_50\_5 \textcolor{red}{\textcjheb{hnldy}} JDLNH $|$sch"opft ihn heraus/(er) sch"opft herauf sie (=ihn)\\
\end{tabular}\medskip \\
Ende des Verses 20.5\\
Verse: 559, Buchstaben: 31, 139, 15930, Totalwerte: 1739, 8949, 1134034\\
\\
Tiefes Wasser ist der Ratschlu"s im Herzen des Mannes, aber ein verst"andiger Mann sch"opft ihn heraus.\\
\newpage 
{\bf -- 20.6}\\
\medskip \\
\begin{tabular}{rrrrrrrrp{120mm}}
WV&WK&WB&ABK&ABB&ABV&AnzB&TW&Zahlencode \textcolor{red}{$\boldsymbol{Grundtext}$} Umschrift $|$"Ubersetzung(en)\\
1.&39.&4112.&140.&15931.&1.&2&202&200\_2 \textcolor{red}{\textcjheb{br}} RB $|$die meisten/viel(e)\\
2.&40.&4113.&142.&15933.&3.&3&45&1\_4\_40 \textcolor{red}{\textcjheb{md'}} ADM $|$Menschen\\
3.&41.&4114.&145.&15936.&6.&4&311&10\_100\_200\_1 \textcolor{red}{\textcjheb{'rqy}} JQRA $|$(sie) rufen aus\\
4.&42.&4115.&149.&15940.&10.&3&311&1\_10\_300 \textcolor{red}{\textcjheb{+sy'}} AJS $|$(ein) jeder\\
5.&43.&4116.&152.&15943.&13.&4&78&8\_60\_4\_6 \textcolor{red}{\textcjheb{wds.h}} CsDW $|$seine G"ute\\
6.&44.&4117.&156.&15947.&17.&4&317&6\_1\_10\_300 \textcolor{red}{\textcjheb{+sy'w}} WAJS $|$aber einen Mann/und (einen) Mann\\
7.&45.&4118.&160.&15951.&21.&6&147&1\_40\_6\_50\_10\_40 \textcolor{red}{\textcjheb{mynwm'}} AMWNJM $|$zuverl"assigen/treuen\\
8.&46.&4119.&166.&15957.&27.&2&50&40\_10 \textcolor{red}{\textcjheb{ym}} MJ $|$wer\\
9.&47.&4120.&168.&15959.&29.&4&141&10\_40\_90\_1 \textcolor{red}{\textcjheb{'.smy}} JM"sA $|$wird ihn finden/(er) findet (ihn)\\
\end{tabular}\medskip \\
Ende des Verses 20.6\\
Verse: 560, Buchstaben: 32, 171, 15962, Totalwerte: 1602, 10551, 1135636\\
\\
Die meisten Menschen rufen ein jeder seine G"ute aus; aber einen zuverl"assigen Mann, wer wird ihn finden?\\
\newpage 
{\bf -- 20.7}\\
\medskip \\
\begin{tabular}{rrrrrrrrp{120mm}}
WV&WK&WB&ABK&ABB&ABV&AnzB&TW&Zahlencode \textcolor{red}{$\boldsymbol{Grundtext}$} Umschrift $|$"Ubersetzung(en)\\
1.&48.&4121.&172.&15963.&1.&5&495&40\_400\_5\_30\_20 \textcolor{red}{\textcjheb{klhtm}} MTHLK $|$wer wandelt/(ein) Wandelnder\\
2.&49.&4122.&177.&15968.&6.&4&448&2\_400\_40\_6 \textcolor{red}{\textcjheb{wmtb}} BTMW $|$in seiner Vollkommenheit/in seiner Lauterkeit\\
3.&50.&4123.&181.&15972.&10.&4&204&90\_4\_10\_100 \textcolor{red}{\textcjheb{qyd.s}} "sDJQ $|$gerecht/(ist der) Gerechte\\
4.&51.&4124.&185.&15976.&14.&4&511&1\_300\_200\_10 \textcolor{red}{\textcjheb{yr+s'}} ASRJ $|$gl"uckselig sind/Seligkeiten\\
5.&52.&4125.&189.&15980.&18.&4&68&2\_50\_10\_6 \textcolor{red}{\textcjheb{wynb}} BNJW $|$seine Kinder/seinen S"ohnen\\
6.&53.&4126.&193.&15984.&22.&5&225&1\_8\_200\_10\_6 \textcolor{red}{\textcjheb{wyr.h'}} ACRJW $|$nach ihm\\
\end{tabular}\medskip \\
Ende des Verses 20.7\\
Verse: 561, Buchstaben: 26, 197, 15988, Totalwerte: 1951, 12502, 1137587\\
\\
Wer in seiner Vollkommenheit gerecht wandelt, gl"uckselig sind seine Kinder nach ihm!\\
\newpage 
{\bf -- 20.8}\\
\medskip \\
\begin{tabular}{rrrrrrrrp{120mm}}
WV&WK&WB&ABK&ABB&ABV&AnzB&TW&Zahlencode \textcolor{red}{$\boldsymbol{Grundtext}$} Umschrift $|$"Ubersetzung(en)\\
1.&54.&4127.&198.&15989.&1.&3&90&40\_30\_20 \textcolor{red}{\textcjheb{klm}} MLK $|$ein K"onig/der K"onig\\
2.&55.&4128.&201.&15992.&4.&4&318&10\_6\_300\_2 \textcolor{red}{\textcjheb{b+swy}} JWSB $|$der sitzt/sitzend\\
3.&56.&4129.&205.&15996.&8.&2&100&70\_30 \textcolor{red}{\textcjheb{l`}} aL $|$auf\\
4.&57.&4130.&207.&15998.&10.&3&81&20\_60\_1 \textcolor{red}{\textcjheb{'sk}} KsA $|$dem Thron/dem Stuhl\\
5.&58.&4131.&210.&16001.&13.&3&64&4\_10\_50 \textcolor{red}{\textcjheb{nyd}} DJN $|$des Gerichts/des Richtens\\
6.&59.&4132.&213.&16004.&16.&4&252&40\_7\_200\_5 \textcolor{red}{\textcjheb{hrzm}} MZRH $|$zerstreut/sondert aus\\
7.&60.&4133.&217.&16008.&20.&6&148&2\_70\_10\_50\_10\_6 \textcolor{red}{\textcjheb{wyny`b}} BaJNJW $|$mit seinen Augen\\
8.&61.&4134.&223.&16014.&26.&2&50&20\_30 \textcolor{red}{\textcjheb{lk}} KL $|$alles\\
9.&62.&4135.&225.&16016.&28.&2&270&200\_70 \textcolor{red}{\textcjheb{`r}} Ra $|$B"ose\\
\end{tabular}\medskip \\
Ende des Verses 20.8\\
Verse: 562, Buchstaben: 29, 226, 16017, Totalwerte: 1373, 13875, 1138960\\
\\
Ein K"onig, der auf dem Throne des Gerichts sitzt, zerstreut alles B"ose mit seinen Augen.\\
\newpage 
{\bf -- 20.9}\\
\medskip \\
\begin{tabular}{rrrrrrrrp{120mm}}
WV&WK&WB&ABK&ABB&ABV&AnzB&TW&Zahlencode \textcolor{red}{$\boldsymbol{Grundtext}$} Umschrift $|$"Ubersetzung(en)\\
1.&63.&4136.&227.&16018.&1.&2&50&40\_10 \textcolor{red}{\textcjheb{ym}} MJ $|$wer\\
2.&64.&4137.&229.&16020.&3.&4&251&10\_1\_40\_200 \textcolor{red}{\textcjheb{rm'y}} JAMR $|$(er) darf sagen\\
3.&65.&4138.&233.&16024.&7.&5&447&7\_20\_10\_400\_10 \textcolor{red}{\textcjheb{ytykz}} ZKJTJ $|$ich habe gereinigt/ich habe lauter erhalten\\
4.&66.&4139.&238.&16029.&12.&3&42&30\_2\_10 \textcolor{red}{\textcjheb{ybl}} LBJ $|$mein Herz\\
5.&67.&4140.&241.&16032.&15.&5&624&9\_5\_200\_400\_10 \textcolor{red}{\textcjheb{ytrh.t}} tHRTJ $|$ich bin rein (geworden)\\
6.&68.&4141.&246.&16037.&20.&6&468&40\_8\_9\_1\_400\_10 \textcolor{red}{\textcjheb{yt'.t.hm}} MCtATJ $|$von meiner S"unde\\
\end{tabular}\medskip \\
Ende des Verses 20.9\\
Verse: 563, Buchstaben: 25, 251, 16042, Totalwerte: 1882, 15757, 1140842\\
\\
Wer darf sagen: Ich habe mein Herz gereinigt, ich bin rein geworden von meiner S"unde?\\
\newpage 
{\bf -- 20.10}\\
\medskip \\
\begin{tabular}{rrrrrrrrp{120mm}}
WV&WK&WB&ABK&ABB&ABV&AnzB&TW&Zahlencode \textcolor{red}{$\boldsymbol{Grundtext}$} Umschrift $|$"Ubersetzung(en)\\
1.&69.&4142.&252.&16043.&1.&3&53&1\_2\_50 \textcolor{red}{\textcjheb{nb'}} ABN $|$Gewichtssteine/Stein\\
2.&70.&4143.&255.&16046.&4.&4&59&6\_1\_2\_50 \textcolor{red}{\textcjheb{nb'w}} WABN $|$zweierlei/und Stein\\
3.&71.&4144.&259.&16050.&8.&4&96&1\_10\_80\_5 \textcolor{red}{\textcjheb{hpy'}} AJPH $|$Epha/Efa\\
4.&72.&4145.&263.&16054.&12.&5&102&6\_1\_10\_80\_5 \textcolor{red}{\textcjheb{hpy'w}} WAJPH $|$zweierlei/und Efa\\
5.&73.&4146.&268.&16059.&17.&5&878&400\_6\_70\_2\_400 \textcolor{red}{\textcjheb{tb`wt}} TWaBT $|$(ein) Gr"auel\\
6.&74.&4147.&273.&16064.&22.&4&26&10\_5\_6\_5 \textcolor{red}{\textcjheb{hwhy}} JHWH $|$(f"ur) Jahwe\\
7.&75.&4148.&277.&16068.&26.&2&43&3\_40 \textcolor{red}{\textcjheb{mg}} GM $|$sind alle/(sind) auch\\
8.&76.&4149.&279.&16070.&28.&5&405&300\_50\_10\_5\_40 \textcolor{red}{\textcjheb{mhyn+s}} SNJHM $|$sie beide\\
\end{tabular}\medskip \\
Ende des Verses 20.10\\
Verse: 564, Buchstaben: 32, 283, 16074, Totalwerte: 1662, 17419, 1142504\\
\\
Zweierlei Gewichtsteine, zweierlei Epha, sie alle beide sind Jahwe ein Greuel.\\
\newpage 
{\bf -- 20.11}\\
\medskip \\
\begin{tabular}{rrrrrrrrp{120mm}}
WV&WK&WB&ABK&ABB&ABV&AnzB&TW&Zahlencode \textcolor{red}{$\boldsymbol{Grundtext}$} Umschrift $|$"Ubersetzung(en)\\
1.&77.&4150.&284.&16075.&1.&2&43&3\_40 \textcolor{red}{\textcjheb{mg}} GM $|$selbst/auch\\
2.&78.&4151.&286.&16077.&3.&7&188&2\_40\_70\_30\_30\_10\_6 \textcolor{red}{\textcjheb{wyll`mb}} BMaLLJW $|$durch seine Handlungen/an seinen Handlungen\\
3.&79.&4152.&293.&16084.&10.&5&680&10\_400\_50\_20\_200 \textcolor{red}{\textcjheb{rknty}} JTNKR $|$(er (=es)) gibt zu erkennen sich\\
4.&80.&4153.&298.&16089.&15.&3&320&50\_70\_200 \textcolor{red}{\textcjheb{r`n}} NaR $|$(ein) Knabe\\
5.&81.&4154.&301.&16092.&18.&2&41&1\_40 \textcolor{red}{\textcjheb{m'}} AM $|$ob\\
6.&82.&4155.&303.&16094.&20.&2&27&7\_20 \textcolor{red}{\textcjheb{kz}} ZK $|$lauter\\
7.&83.&4156.&305.&16096.&22.&3&47&6\_1\_40 \textcolor{red}{\textcjheb{m'w}} WAM $|$und ob\\
8.&84.&4157.&308.&16099.&25.&3&510&10\_300\_200 \textcolor{red}{\textcjheb{r+sy}} JSR $|$es aufrichtig ist/rechtschaffen\\
9.&85.&4158.&311.&16102.&28.&4&186&80\_70\_30\_6 \textcolor{red}{\textcjheb{wl`p}} PaLW $|$sein Tun\\
\end{tabular}\medskip \\
Ende des Verses 20.11\\
Verse: 565, Buchstaben: 31, 314, 16105, Totalwerte: 2042, 19461, 1144546\\
\\
Selbst ein Knabe gibt sich durch seine Handlungen zu erkennen, ob sein Tun lauter, und ob es aufrichtig ist.\\
\newpage 
{\bf -- 20.12}\\
\medskip \\
\begin{tabular}{rrrrrrrrp{120mm}}
WV&WK&WB&ABK&ABB&ABV&AnzB&TW&Zahlencode \textcolor{red}{$\boldsymbol{Grundtext}$} Umschrift $|$"Ubersetzung(en)\\
1.&86.&4159.&315.&16106.&1.&3&58&1\_7\_50 \textcolor{red}{\textcjheb{nz'}} AZN $|$das Ohr/(ein) Ohr\\
2.&87.&4160.&318.&16109.&4.&4&810&300\_40\_70\_400 \textcolor{red}{\textcjheb{t`m+s}} SMaT $|$h"orend(e(s))\\
3.&88.&4161.&322.&16113.&8.&4&136&6\_70\_10\_50 \textcolor{red}{\textcjheb{ny`w}} WaJN $|$und das Auge/und (ein) Auge\\
4.&89.&4162.&326.&16117.&12.&3&206&200\_1\_5 \textcolor{red}{\textcjheb{h'r}} RAH $|$sehend(e(s))\\
5.&90.&4163.&329.&16120.&15.&4&26&10\_5\_6\_5 \textcolor{red}{\textcjheb{hwhy}} JHWH $|$Jahwe\\
6.&91.&4164.&333.&16124.&19.&3&375&70\_300\_5 \textcolor{red}{\textcjheb{h+s`}} aSH $|$(er) hat gemacht\\
7.&92.&4165.&336.&16127.&22.&2&43&3\_40 \textcolor{red}{\textcjheb{mg}} GM $|$alle/auch\\
8.&93.&4166.&338.&16129.&24.&5&405&300\_50\_10\_5\_40 \textcolor{red}{\textcjheb{mhyn+s}} SNJHM $|$sie beide\\
\end{tabular}\medskip \\
Ende des Verses 20.12\\
Verse: 566, Buchstaben: 28, 342, 16133, Totalwerte: 2059, 21520, 1146605\\
\\
Das h"orende Ohr und das sehende Auge, Jahwe hat sie alle beide gemacht.\\
\newpage 
{\bf -- 20.13}\\
\medskip \\
\begin{tabular}{rrrrrrrrp{120mm}}
WV&WK&WB&ABK&ABB&ABV&AnzB&TW&Zahlencode \textcolor{red}{$\boldsymbol{Grundtext}$} Umschrift $|$"Ubersetzung(en)\\
1.&94.&4167.&343.&16134.&1.&2&31&1\_30 \textcolor{red}{\textcjheb{l'}} AL $|$nicht\\
2.&95.&4168.&345.&16136.&3.&4&408&400\_1\_5\_2 \textcolor{red}{\textcjheb{bh't}} TAHB $|$liebe/du sollst lieben\\
3.&96.&4169.&349.&16140.&7.&3&355&300\_50\_5 \textcolor{red}{\textcjheb{hn+s}} SNH $|$(den) Schlaf\\
4.&97.&4170.&352.&16143.&10.&2&130&80\_50 \textcolor{red}{\textcjheb{np}} PN $|$damit nicht/dass nicht\\
5.&98.&4171.&354.&16145.&12.&4&906&400\_6\_200\_300 \textcolor{red}{\textcjheb{+srwt}} TWRS $|$du verarmst/du arm wirst\\
6.&99.&4172.&358.&16149.&16.&3&188&80\_100\_8 \textcolor{red}{\textcjheb{.hqp}} PQC $|$tue auf/"offne\\
7.&100.&4173.&361.&16152.&19.&5&160&70\_10\_50\_10\_20 \textcolor{red}{\textcjheb{kyny`}} aJNJK $|$deine Augen\\
8.&101.&4174.&366.&16157.&24.&3&372&300\_2\_70 \textcolor{red}{\textcjheb{`b+s}} SBa $|$so wirst du satt haben/(einer) S"attigung\\
9.&102.&4175.&369.&16160.&27.&3&78&30\_8\_40 \textcolor{red}{\textcjheb{m.hl}} LCM $|$(an) Brot\\
\end{tabular}\medskip \\
Ende des Verses 20.13\\
Verse: 567, Buchstaben: 29, 371, 16162, Totalwerte: 2628, 24148, 1149233\\
\\
Liebe nicht den Schlaf, damit du nicht verarmest; tue deine Augen auf, so wirst du satt Brot haben.\\
\newpage 
{\bf -- 20.14}\\
\medskip \\
\begin{tabular}{rrrrrrrrp{120mm}}
WV&WK&WB&ABK&ABB&ABV&AnzB&TW&Zahlencode \textcolor{red}{$\boldsymbol{Grundtext}$} Umschrift $|$"Ubersetzung(en)\\
1.&103.&4176.&372.&16163.&1.&2&270&200\_70 \textcolor{red}{\textcjheb{`r}} Ra $|$schlecht\\
2.&104.&4177.&374.&16165.&3.&2&270&200\_70 \textcolor{red}{\textcjheb{`r}} Ra $|$schlecht\\
3.&105.&4178.&376.&16167.&5.&4&251&10\_1\_40\_200 \textcolor{red}{\textcjheb{rm'y}} JAMR $|$spricht/er (=es) sagt\\
4.&106.&4179.&380.&16171.&9.&5&166&5\_100\_6\_50\_5 \textcolor{red}{\textcjheb{hnwqh}} HQWNH $|$der K"aufer\\
5.&107.&4180.&385.&16176.&14.&4&44&6\_1\_7\_30 \textcolor{red}{\textcjheb{lz'w}} WAZL $|$und wenn er weggeht/und weggehend\\
6.&108.&4181.&389.&16180.&18.&2&36&30\_6 \textcolor{red}{\textcjheb{wl}} LW $|$/zu sich\\
7.&109.&4182.&391.&16182.&20.&2&8&1\_7 \textcolor{red}{\textcjheb{z'}} AZ $|$da(nn)\\
8.&110.&4183.&393.&16184.&22.&5&475&10\_400\_5\_30\_30 \textcolor{red}{\textcjheb{llhty}} JTHLL $|$er r"uhmt sich\\
\end{tabular}\medskip \\
Ende des Verses 20.14\\
Verse: 568, Buchstaben: 26, 397, 16188, Totalwerte: 1520, 25668, 1150753\\
\\
Schlecht, schlecht! spricht der K"aufer; und wenn er weggeht, dann r"uhmt er sich.\\
\newpage 
{\bf -- 20.15}\\
\medskip \\
\begin{tabular}{rrrrrrrrp{120mm}}
WV&WK&WB&ABK&ABB&ABV&AnzB&TW&Zahlencode \textcolor{red}{$\boldsymbol{Grundtext}$} Umschrift $|$"Ubersetzung(en)\\
1.&111.&4184.&398.&16189.&1.&2&310&10\_300 \textcolor{red}{\textcjheb{+sy}} JS $|$es gibt\\
2.&112.&4185.&400.&16191.&3.&3&14&7\_5\_2 \textcolor{red}{\textcjheb{bhz}} ZHB $|$Gold\\
3.&113.&4186.&403.&16194.&6.&3&208&6\_200\_2 \textcolor{red}{\textcjheb{brw}} WRB $|$die Menge/und viel\\
4.&114.&4187.&406.&16197.&9.&6&240&80\_50\_10\_50\_10\_40 \textcolor{red}{\textcjheb{mynynp}} PNJNJM $|$(von) Korallen\\
5.&115.&4188.&412.&16203.&15.&4&66&6\_20\_30\_10 \textcolor{red}{\textcjheb{ylkw}} WKLJ $|$aber ein Ger"at/und ein Gef"a"s\\
6.&116.&4189.&416.&16207.&19.&3&310&10\_100\_200 \textcolor{red}{\textcjheb{rqy}} JQR $|$kostbares/(von) Kostbarkeit\\
7.&117.&4190.&419.&16210.&22.&4&790&300\_80\_400\_10 \textcolor{red}{\textcjheb{ytp+s}} SPTJ $|$(sind) (die) Lippen\\
8.&118.&4191.&423.&16214.&26.&3&474&4\_70\_400 \textcolor{red}{\textcjheb{t`d}} DaT $|$(der) Erkenntnis\\
\end{tabular}\medskip \\
Ende des Verses 20.15\\
Verse: 569, Buchstaben: 28, 425, 16216, Totalwerte: 2412, 28080, 1153165\\
\\
Es gibt Gold und Korallen die Menge; aber ein kostbares Ger"at sind Lippen der Erkenntnis.\\
\newpage 
{\bf -- 20.16}\\
\medskip \\
\begin{tabular}{rrrrrrrrp{120mm}}
WV&WK&WB&ABK&ABB&ABV&AnzB&TW&Zahlencode \textcolor{red}{$\boldsymbol{Grundtext}$} Umschrift $|$"Ubersetzung(en)\\
1.&119.&4192.&426.&16217.&1.&3&138&30\_100\_8 \textcolor{red}{\textcjheb{.hql}} LQC $|$nimm\\
2.&120.&4193.&429.&16220.&4.&4&15&2\_3\_4\_6 \textcolor{red}{\textcjheb{wdgb}} BGDW $|$ihm das Kleid/sein Gewand\\
3.&121.&4194.&433.&16224.&8.&2&30&20\_10 \textcolor{red}{\textcjheb{yk}} KJ $|$denn/weil\\
4.&122.&4195.&435.&16226.&10.&3&272&70\_200\_2 \textcolor{red}{\textcjheb{br`}} aRB $|$er ist B"urge geworden/er geb"urgt\\
5.&123.&4196.&438.&16229.&13.&2&207&7\_200 \textcolor{red}{\textcjheb{rz}} ZR $|$f"ur einen anderen/(f"ur einen) Fremden\\
6.&124.&4197.&440.&16231.&15.&4&82&6\_2\_70\_4 \textcolor{red}{\textcjheb{d`bw}} WBaD $|$und halber/und f"ur\\
7.&125.&4198.&444.&16235.&19.&5&320&50\_20\_200\_10\_40 \textcolor{red}{\textcjheb{myrkn}} NKRJM $|$der Fremden/die fremde (Frau)\\
8.&126.&4199.&449.&16240.&24.&5&51&8\_2\_30\_5\_6 \textcolor{red}{\textcjheb{whlb.h}} CBLHW $|$pf"ande ihn\\
\end{tabular}\medskip \\
Ende des Verses 20.16\\
Verse: 570, Buchstaben: 28, 453, 16244, Totalwerte: 1115, 29195, 1154280\\
\\
Nimm ihm das Kleid, denn er ist f"ur einen anderen B"urge geworden; und der Fremden halber pf"ande ihn.\\
\newpage 
{\bf -- 20.17}\\
\medskip \\
\begin{tabular}{rrrrrrrrp{120mm}}
WV&WK&WB&ABK&ABB&ABV&AnzB&TW&Zahlencode \textcolor{red}{$\boldsymbol{Grundtext}$} Umschrift $|$"Ubersetzung(en)\\
1.&127.&4200.&454.&16245.&1.&3&272&70\_200\_2 \textcolor{red}{\textcjheb{br`}} aRB $|$s"u"s ist/angenehm (ist)\\
2.&128.&4201.&457.&16248.&4.&4&341&30\_1\_10\_300 \textcolor{red}{\textcjheb{+sy'l}} LAJS $|$einem Mann/dem Mann\\
3.&129.&4202.&461.&16252.&8.&3&78&30\_8\_40 \textcolor{red}{\textcjheb{m.hl}} LCM $|$das Brot\\
4.&130.&4203.&464.&16255.&11.&3&600&300\_100\_200 \textcolor{red}{\textcjheb{rq+s}} SQR $|$der Falschheit/des Betrugs\\
5.&131.&4204.&467.&16258.&14.&4&215&6\_1\_8\_200 \textcolor{red}{\textcjheb{r.h'w}} WACR $|$aber hernach/und nachher\\
6.&132.&4205.&471.&16262.&18.&4&81&10\_40\_30\_1 \textcolor{red}{\textcjheb{'lmy}} JMLA $|$wird voll/er (=es) wird gef"ullt\\
7.&133.&4206.&475.&16266.&22.&4&101&80\_10\_5\_6 \textcolor{red}{\textcjheb{whyp}} PJHW $|$sein Mund\\
8.&134.&4207.&479.&16270.&26.&3&188&8\_90\_90 \textcolor{red}{\textcjheb{.s.s.h}} C"s"s $|$(mit) Kies(el)\\
\end{tabular}\medskip \\
Ende des Verses 20.17\\
Verse: 571, Buchstaben: 28, 481, 16272, Totalwerte: 1876, 31071, 1156156\\
\\
Das Brot der Falschheit ist einem Manne s"u"s, aber hernach wird sein Mund voll Kies.\\
\newpage 
{\bf -- 20.18}\\
\medskip \\
\begin{tabular}{rrrrrrrrp{120mm}}
WV&WK&WB&ABK&ABB&ABV&AnzB&TW&Zahlencode \textcolor{red}{$\boldsymbol{Grundtext}$} Umschrift $|$"Ubersetzung(en)\\
1.&135.&4208.&482.&16273.&1.&6&756&40\_8\_300\_2\_6\_400 \textcolor{red}{\textcjheb{twb+s.hm}} MCSBWT $|$Pl"ane\\
2.&136.&4209.&488.&16279.&7.&4&167&2\_70\_90\_5 \textcolor{red}{\textcjheb{h.s`b}} Ba"sH $|$durch Beratung/durch Ratschlag\\
3.&137.&4210.&492.&16283.&11.&4&476&400\_20\_6\_50 \textcolor{red}{\textcjheb{nwkt}} TKWN $|$kommen zustande/du befestigst\\
4.&138.&4211.&496.&16287.&15.&8&854&6\_2\_400\_8\_2\_30\_6\_400 \textcolor{red}{\textcjheb{twlb.htbw}} WBTCBLWT $|$und mit (weiser) "Uberlegung(en)\\
5.&139.&4212.&504.&16295.&23.&3&375&70\_300\_5 \textcolor{red}{\textcjheb{h+s`}} aSH $|$f"uhre/mache\\
6.&140.&4213.&507.&16298.&26.&5&123&40\_30\_8\_40\_5 \textcolor{red}{\textcjheb{hm.hlm}} MLCMH $|$Krieg/den Kampf\\
\end{tabular}\medskip \\
Ende des Verses 20.18\\
Verse: 572, Buchstaben: 30, 511, 16302, Totalwerte: 2751, 33822, 1158907\\
\\
Pl"ane kommen durch Beratung zustande, und mit weiser "Uberlegung f"uhre Krieg.\\
\newpage 
{\bf -- 20.19}\\
\medskip \\
\begin{tabular}{rrrrrrrrp{120mm}}
WV&WK&WB&ABK&ABB&ABV&AnzB&TW&Zahlencode \textcolor{red}{$\boldsymbol{Grundtext}$} Umschrift $|$"Ubersetzung(en)\\
1.&141.&4214.&512.&16303.&1.&4&44&3\_6\_30\_5 \textcolor{red}{\textcjheb{hlwg}} GWLH $|$(es) enth"ullt/wer aufdeckt\\
2.&142.&4215.&516.&16307.&5.&3&70&60\_6\_4 \textcolor{red}{\textcjheb{dws}} sWD $|$das Geheimnis/(ein) Geheimnis\\
3.&143.&4216.&519.&16310.&8.&4&61&5\_6\_30\_20 \textcolor{red}{\textcjheb{klwh}} HWLK $|$wer umhergeht/(ist) umhergehend\\
4.&144.&4217.&523.&16314.&12.&4&260&200\_20\_10\_30 \textcolor{red}{\textcjheb{lykr}} RKJL $|$(als) Verleumder\\
5.&145.&4218.&527.&16318.&16.&5&521&6\_30\_80\_400\_5 \textcolor{red}{\textcjheb{htplw}} WLPTH $|$und mit dem der aufsperrt/mit einem sich verf"uhren Lassenden\\
6.&146.&4219.&532.&16323.&21.&5&796&300\_80\_400\_10\_6 \textcolor{red}{\textcjheb{wytp+s}} SPTJW $|$(durch) seine(n) Lippen\\
7.&147.&4220.&537.&16328.&26.&2&31&30\_1 \textcolor{red}{\textcjheb{'l}} LA $|$nicht\\
8.&148.&4221.&539.&16330.&28.&5&1072&400\_400\_70\_200\_2 \textcolor{red}{\textcjheb{br`tt}} TTaRB $|$lass dich ein/du sollst dich einlassen\\
\end{tabular}\medskip \\
Ende des Verses 20.19\\
Verse: 573, Buchstaben: 32, 543, 16334, Totalwerte: 2855, 36677, 1161762\\
\\
Wer als Verleumder umhergeht, enth"ullt das Geheimnis; und mit dem, der seine Lippen aufsperrt, la"s dich nicht ein.\\
\newpage 
{\bf -- 20.20}\\
\medskip \\
\begin{tabular}{rrrrrrrrp{120mm}}
WV&WK&WB&ABK&ABB&ABV&AnzB&TW&Zahlencode \textcolor{red}{$\boldsymbol{Grundtext}$} Umschrift $|$"Ubersetzung(en)\\
1.&149.&4222.&544.&16335.&1.&4&200&40\_100\_30\_30 \textcolor{red}{\textcjheb{llqm}} MQLL $|$wer flucht\\
2.&150.&4223.&548.&16339.&5.&4&19&1\_2\_10\_6 \textcolor{red}{\textcjheb{wyb'}} ABJW $|$seinem Vater\\
3.&151.&4224.&552.&16343.&9.&4&53&6\_1\_40\_6 \textcolor{red}{\textcjheb{wm'w}} WAMW $|$und seiner Mutter/und seiner Mutter\\
4.&152.&4225.&556.&16347.&13.&4&104&10\_4\_70\_20 \textcolor{red}{\textcjheb{k`dy}} JDaK $|$erl"oschen wird/er (=es) erlischt\\
5.&153.&4226.&560.&16351.&17.&3&256&50\_200\_6 \textcolor{red}{\textcjheb{wrn}} NRW $|$dessen Leuchte/seine Leuchte\\
6.&154.&4227.&563.&16354.&20.&6&369&2\_1\_10\_300\_6\_50 \textcolor{red}{\textcjheb{nw+sy'b}} BAJSWN $|$in (der Zeit)\\
7.&155.&4228.&569.&16360.&26.&3&328&8\_300\_20 \textcolor{red}{\textcjheb{k+s.h}} CSK $|$tiefster Finsternis/(der) Finsternis\\
\end{tabular}\medskip \\
Ende des Verses 20.20\\
Verse: 574, Buchstaben: 28, 571, 16362, Totalwerte: 1329, 38006, 1163091\\
\\
Wer seinem Vater oder seiner Mutter flucht, dessen Leuchte wird erl"oschen in tiefster Finsternis.\\
\newpage 
{\bf -- 20.21}\\
\medskip \\
\begin{tabular}{rrrrrrrrp{120mm}}
WV&WK&WB&ABK&ABB&ABV&AnzB&TW&Zahlencode \textcolor{red}{$\boldsymbol{Grundtext}$} Umschrift $|$"Ubersetzung(en)\\
1.&156.&4229.&572.&16363.&1.&4&93&50\_8\_30\_5 \textcolor{red}{\textcjheb{hl.hn}} NCLH $|$ein Erbe/(ein) Besitz\\
2.&157.&4230.&576.&16367.&5.&5&480&40\_2\_8\_30\_400 \textcolor{red}{\textcjheb{tl.hbm}} MBCLT $|$das wird hastig erlangt/rasch gewonnen(er)\\
3.&158.&4231.&581.&16372.&10.&6&558&2\_200\_1\_300\_50\_5 \textcolor{red}{\textcjheb{hn+s'rb}} BRASNH $|$im Anfang/am Anfang\\
4.&159.&4232.&587.&16378.&16.&7&630&6\_1\_8\_200\_10\_400\_5 \textcolor{red}{\textcjheb{htyr.h'w}} WACRJTH $|$(und) dessen Ende\\
5.&160.&4233.&594.&16385.&23.&2&31&30\_1 \textcolor{red}{\textcjheb{'l}} LA $|$nicht\\
6.&161.&4234.&596.&16387.&25.&4&622&400\_2\_200\_20 \textcolor{red}{\textcjheb{krbt}} TBRK $|$(sie (=es)) wird sein gesegnet\\
\end{tabular}\medskip \\
Ende des Verses 20.21\\
Verse: 575, Buchstaben: 28, 599, 16390, Totalwerte: 2414, 40420, 1165505\\
\\
Ein Erbe, das hastig erlangt wird im Anfang, dessen Ende wird nicht gesegnet sein.\\
\newpage 
{\bf -- 20.22}\\
\medskip \\
\begin{tabular}{rrrrrrrrp{120mm}}
WV&WK&WB&ABK&ABB&ABV&AnzB&TW&Zahlencode \textcolor{red}{$\boldsymbol{Grundtext}$} Umschrift $|$"Ubersetzung(en)\\
1.&162.&4235.&600.&16391.&1.&2&31&1\_30 \textcolor{red}{\textcjheb{l'}} AL $|$nicht\\
2.&163.&4236.&602.&16393.&3.&4&641&400\_1\_40\_200 \textcolor{red}{\textcjheb{rm't}} TAMR $|$sprich/sollst du sagen\\
3.&164.&4237.&606.&16397.&7.&5&376&1\_300\_30\_40\_5 \textcolor{red}{\textcjheb{hml+s'}} ASLMH $|$ich will vergelten\\
4.&165.&4238.&611.&16402.&12.&2&270&200\_70 \textcolor{red}{\textcjheb{`r}} Ra $|$B"oses\\
5.&166.&4239.&613.&16404.&14.&3&111&100\_6\_5 \textcolor{red}{\textcjheb{hwq}} QWH $|$harre/hoffe\\
6.&167.&4240.&616.&16407.&17.&5&56&30\_10\_5\_6\_5 \textcolor{red}{\textcjheb{hwhyl}} LJHWH $|$auf Jahwe\\
7.&168.&4241.&621.&16412.&22.&4&386&6\_10\_300\_70 \textcolor{red}{\textcjheb{`+syw}} WJSa $|$so wird er retten/und er wird helfen\\
8.&169.&4242.&625.&16416.&26.&2&50&30\_20 \textcolor{red}{\textcjheb{kl}} LK $|$dich/dir\\
\end{tabular}\medskip \\
Ende des Verses 20.22\\
Verse: 576, Buchstaben: 27, 626, 16417, Totalwerte: 1921, 42341, 1167426\\
\\
Sprich nicht: Ich will B"oses vergelten. Harre auf Jahwe, so wird er dich retten.\\
\newpage 
{\bf -- 20.23}\\
\medskip \\
\begin{tabular}{rrrrrrrrp{120mm}}
WV&WK&WB&ABK&ABB&ABV&AnzB&TW&Zahlencode \textcolor{red}{$\boldsymbol{Grundtext}$} Umschrift $|$"Ubersetzung(en)\\
1.&170.&4243.&627.&16418.&1.&5&878&400\_6\_70\_2\_400 \textcolor{red}{\textcjheb{tb`wt}} TWaBT $|$(ein) Gr"auel\\
2.&171.&4244.&632.&16423.&6.&4&26&10\_5\_6\_5 \textcolor{red}{\textcjheb{hwhy}} JHWH $|$(f"ur) Jahwe\\
3.&172.&4245.&636.&16427.&10.&3&53&1\_2\_50 \textcolor{red}{\textcjheb{nb'}} ABN $|$sind Gewichtssteine/(sind) Stein\\
4.&173.&4246.&639.&16430.&13.&4&59&6\_1\_2\_50 \textcolor{red}{\textcjheb{nb'w}} WABN $|$zweierlei/und Stein\\
5.&174.&4247.&643.&16434.&17.&6&114&6\_40\_1\_7\_50\_10 \textcolor{red}{\textcjheb{ynz'mw}} WMAZNJ $|$und Waagschalen\\
6.&175.&4248.&649.&16440.&23.&4&285&40\_200\_40\_5 \textcolor{red}{\textcjheb{hmrm}} MRMH $|$tr"ugerische/des Betrugs\\
7.&176.&4249.&653.&16444.&27.&2&31&30\_1 \textcolor{red}{\textcjheb{'l}} LA $|$nicht\\
8.&177.&4250.&655.&16446.&29.&3&17&9\_6\_2 \textcolor{red}{\textcjheb{bw.t}} tWB $|$(sind) gut\\
\end{tabular}\medskip \\
Ende des Verses 20.23\\
Verse: 577, Buchstaben: 31, 657, 16448, Totalwerte: 1463, 43804, 1168889\\
\\
Zweierlei Gewichtsteine sind Jahwe ein Greuel, und tr"ugerische Waagschalen sind nicht gut.\\
\newpage 
{\bf -- 20.24}\\
\medskip \\
\begin{tabular}{rrrrrrrrp{120mm}}
WV&WK&WB&ABK&ABB&ABV&AnzB&TW&Zahlencode \textcolor{red}{$\boldsymbol{Grundtext}$} Umschrift $|$"Ubersetzung(en)\\
1.&178.&4251.&658.&16449.&1.&5&66&40\_10\_5\_6\_5 \textcolor{red}{\textcjheb{hwhym}} MJHWH $|$von Jahwe\\
2.&179.&4252.&663.&16454.&6.&5&214&40\_90\_70\_4\_10 \textcolor{red}{\textcjheb{yd`.sm}} M"saDJ $|$h"angen ab die Schritte/(werden gelenkt) die Schritte\\
3.&180.&4253.&668.&16459.&11.&3&205&3\_2\_200 \textcolor{red}{\textcjheb{rbg}} GBR $|$des Mannes/eines Mannes\\
4.&181.&4254.&671.&16462.&14.&4&51&6\_1\_4\_40 \textcolor{red}{\textcjheb{md'w}} WADM $|$und der Mensch\\
5.&182.&4255.&675.&16466.&18.&2&45&40\_5 \textcolor{red}{\textcjheb{hm}} MH $|$wie/was\\
6.&183.&4256.&677.&16468.&20.&4&72&10\_2\_10\_50 \textcolor{red}{\textcjheb{nyby}} JBJN $|$sollte er verstehen/er k"onnte begreifen\\
7.&184.&4257.&681.&16472.&24.&4&230&4\_200\_20\_6 \textcolor{red}{\textcjheb{wkrd}} DRKW $|$seinen Weg\\
\end{tabular}\medskip \\
Ende des Verses 20.24\\
Verse: 578, Buchstaben: 27, 684, 16475, Totalwerte: 883, 44687, 1169772\\
\\
Des Mannes Schritte h"angen ab von Jahwe; und der Mensch, wie sollte er seinen Weg verstehen?\\
\newpage 
{\bf -- 20.25}\\
\medskip \\
\begin{tabular}{rrrrrrrrp{120mm}}
WV&WK&WB&ABK&ABB&ABV&AnzB&TW&Zahlencode \textcolor{red}{$\boldsymbol{Grundtext}$} Umschrift $|$"Ubersetzung(en)\\
1.&185.&4258.&685.&16476.&1.&4&446&40\_6\_100\_300 \textcolor{red}{\textcjheb{+sqwm}} MWQS $|$(es ist) (ein) Fallstrick\\
2.&186.&4259.&689.&16480.&5.&3&45&1\_4\_40 \textcolor{red}{\textcjheb{md'}} ADM $|$(dem) Menschen\\
3.&187.&4260.&692.&16483.&8.&3&110&10\_30\_70 \textcolor{red}{\textcjheb{`ly}} JLa $|$vorschnell zu sprechen/er sagt unbedacht\\
4.&188.&4261.&695.&16486.&11.&3&404&100\_4\_300 \textcolor{red}{\textcjheb{+sdq}} QDS $|$geheiligt/Heiligkeit\\
5.&189.&4262.&698.&16489.&14.&4&215&6\_1\_8\_200 \textcolor{red}{\textcjheb{r.h'w}} WACR $|$und (her)nach\\
6.&190.&4263.&702.&16493.&18.&5&304&50\_4\_200\_10\_40 \textcolor{red}{\textcjheb{myrdn}} NDRJM $|$den Gel"ubden/den Gel"obnissen\\
7.&191.&4264.&707.&16498.&23.&4&332&30\_2\_100\_200 \textcolor{red}{\textcjheb{rqbl}} LBQR $|$zu "uberlegen/Bedenken zu haben\\
\end{tabular}\medskip \\
Ende des Verses 20.25\\
Verse: 579, Buchstaben: 26, 710, 16501, Totalwerte: 1856, 46543, 1171628\\
\\
Ein Fallstrick des Menschen ist es, vorschnell zu sprechen: Geheiligt! -und nach den Gel"ubden zu "uberlegen.\\
\newpage 
{\bf -- 20.26}\\
\medskip \\
\begin{tabular}{rrrrrrrrp{120mm}}
WV&WK&WB&ABK&ABB&ABV&AnzB&TW&Zahlencode \textcolor{red}{$\boldsymbol{Grundtext}$} Umschrift $|$"Ubersetzung(en)\\
1.&192.&4265.&711.&16502.&1.&4&252&40\_7\_200\_5 \textcolor{red}{\textcjheb{hrzm}} MZRH $|$(es) zerstreut/es sondert aus\\
2.&193.&4266.&715.&16506.&5.&5&620&200\_300\_70\_10\_40 \textcolor{red}{\textcjheb{my`+sr}} RSaJM $|$die Gesetzlosen/(die) Frevler\\
3.&194.&4267.&720.&16511.&10.&3&90&40\_30\_20 \textcolor{red}{\textcjheb{klm}} MLK $|$(ein) K"onig\\
4.&195.&4268.&723.&16514.&13.&3&68&8\_20\_40 \textcolor{red}{\textcjheb{mk.h}} CKM $|$weiser\\
5.&196.&4269.&726.&16517.&16.&4&318&6\_10\_300\_2 \textcolor{red}{\textcjheb{b+syw}} WJSB $|$und f"uhrt hin/und er macht drehen\\
6.&197.&4270.&730.&16521.&20.&5&155&70\_30\_10\_5\_40 \textcolor{red}{\textcjheb{mhyl`}} aLJHM $|$"uber sie/"uber ihnen\\
7.&198.&4271.&735.&16526.&25.&4&137&1\_6\_80\_50 \textcolor{red}{\textcjheb{npw'}} AWPN $|$das Dreschrad/die Dreschwalze\\
\end{tabular}\medskip \\
Ende des Verses 20.26\\
Verse: 580, Buchstaben: 28, 738, 16529, Totalwerte: 1640, 48183, 1173268\\
\\
Ein weiser K"onig zerstreut die Gesetzlosen und f"uhrt das Dreschrad "uber sie hin.\\
\newpage 
{\bf -- 20.27}\\
\medskip \\
\begin{tabular}{rrrrrrrrp{120mm}}
WV&WK&WB&ABK&ABB&ABV&AnzB&TW&Zahlencode \textcolor{red}{$\boldsymbol{Grundtext}$} Umschrift $|$"Ubersetzung(en)\\
1.&199.&4272.&739.&16530.&1.&2&250&50\_200 \textcolor{red}{\textcjheb{rn}} NR $|$(eine) Leuchte\\
2.&200.&4273.&741.&16532.&3.&4&26&10\_5\_6\_5 \textcolor{red}{\textcjheb{hwhy}} JHWH $|$Jahwe(s)\\
3.&201.&4274.&745.&16536.&7.&4&790&50\_300\_40\_400 \textcolor{red}{\textcjheb{tm+sn}} NSMT $|$ist der Geist/(ist der) Odem\\
4.&202.&4275.&749.&16540.&11.&3&45&1\_4\_40 \textcolor{red}{\textcjheb{md'}} ADM $|$des Menschen\\
5.&203.&4276.&752.&16543.&14.&3&388&8\_80\_300 \textcolor{red}{\textcjheb{+sp.h}} CPS $|$durchforschend(er)\\
6.&204.&4277.&755.&16546.&17.&2&50&20\_30 \textcolor{red}{\textcjheb{lk}} KL $|$alle\\
7.&205.&4278.&757.&16548.&19.&4&222&8\_4\_200\_10 \textcolor{red}{\textcjheb{yrd.h}} CDRJ $|$Kammern\\
8.&206.&4279.&761.&16552.&23.&3&61&2\_9\_50 \textcolor{red}{\textcjheb{n.tb}} BtN $|$des Leibes\\
\end{tabular}\medskip \\
Ende des Verses 20.27\\
Verse: 581, Buchstaben: 25, 763, 16554, Totalwerte: 1832, 50015, 1175100\\
\\
Der Geist des Menschen ist eine Leuchte Jahwes, durchforschend alle Kammern des Leibes.\\
\newpage 
{\bf -- 20.28}\\
\medskip \\
\begin{tabular}{rrrrrrrrp{120mm}}
WV&WK&WB&ABK&ABB&ABV&AnzB&TW&Zahlencode \textcolor{red}{$\boldsymbol{Grundtext}$} Umschrift $|$"Ubersetzung(en)\\
1.&207.&4280.&764.&16555.&1.&3&72&8\_60\_4 \textcolor{red}{\textcjheb{ds.h}} CsD $|$G"ute\\
2.&208.&4281.&767.&16558.&4.&4&447&6\_1\_40\_400 \textcolor{red}{\textcjheb{tm'w}} WAMT $|$und Wahrheit/und Treue\\
3.&209.&4282.&771.&16562.&8.&4&306&10\_90\_200\_6 \textcolor{red}{\textcjheb{wr.sy}} J"sRW $|$(sie) beh"uten\\
4.&210.&4283.&775.&16566.&12.&3&90&40\_30\_20 \textcolor{red}{\textcjheb{klm}} MLK $|$den K"onig\\
5.&211.&4284.&778.&16569.&15.&4&140&6\_60\_70\_4 \textcolor{red}{\textcjheb{d`sw}} WsaD $|$und er st"utzt/und er befestigt\\
6.&212.&4285.&782.&16573.&19.&4&74&2\_8\_60\_4 \textcolor{red}{\textcjheb{ds.hb}} BCsD $|$(durch) (die) G"ute\\
7.&213.&4286.&786.&16577.&23.&4&87&20\_60\_1\_6 \textcolor{red}{\textcjheb{w'sk}} KsAW $|$seinen Thron\\
\end{tabular}\medskip \\
Ende des Verses 20.28\\
Verse: 582, Buchstaben: 26, 789, 16580, Totalwerte: 1216, 51231, 1176316\\
\\
G"ute und Wahrheit beh"uten den K"onig, und durch G"ute st"utzt er seinen Thron.\\
\newpage 
{\bf -- 20.29}\\
\medskip \\
\begin{tabular}{rrrrrrrrp{120mm}}
WV&WK&WB&ABK&ABB&ABV&AnzB&TW&Zahlencode \textcolor{red}{$\boldsymbol{Grundtext}$} Umschrift $|$"Ubersetzung(en)\\
1.&214.&4287.&790.&16581.&1.&5&1081&400\_80\_1\_200\_400 \textcolor{red}{\textcjheb{tr'pt}} TPART $|$der Schmuck/der Ruhm\\
2.&215.&4288.&795.&16586.&6.&6&266&2\_8\_6\_200\_10\_40 \textcolor{red}{\textcjheb{myrw.hb}} BCWRJM $|$(der) J"unglinge\\
3.&216.&4289.&801.&16592.&12.&3&68&20\_8\_40 \textcolor{red}{\textcjheb{m.hk}} KCM $|$(ist) ihre Kraft\\
4.&217.&4290.&804.&16595.&15.&4&215&6\_5\_4\_200 \textcolor{red}{\textcjheb{rdhw}} WHDR $|$und die Zierde/und der Schmuck\\
5.&218.&4291.&808.&16599.&19.&5&207&7\_100\_50\_10\_40 \textcolor{red}{\textcjheb{mynqz}} ZQNJM $|$der Alten\\
6.&219.&4292.&813.&16604.&24.&4&317&300\_10\_2\_5 \textcolor{red}{\textcjheb{hby+s}} SJBH $|$(ist) graues Haar\\
\end{tabular}\medskip \\
Ende des Verses 20.29\\
Verse: 583, Buchstaben: 27, 816, 16607, Totalwerte: 2154, 53385, 1178470\\
\\
Der Schmuck der J"unglinge ist ihre Kraft, und graues Haar die Zierde der Alten.\\
\newpage 
{\bf -- 20.30}\\
\medskip \\
\begin{tabular}{rrrrrrrrp{120mm}}
WV&WK&WB&ABK&ABB&ABV&AnzB&TW&Zahlencode \textcolor{red}{$\boldsymbol{Grundtext}$} Umschrift $|$"Ubersetzung(en)\\
1.&220.&4293.&817.&16608.&1.&5&616&8\_2\_200\_6\_400 \textcolor{red}{\textcjheb{twrb.h}} CBRWT $|$Striemen\\
2.&221.&4294.&822.&16613.&6.&3&240&80\_90\_70 \textcolor{red}{\textcjheb{`.sp}} P"sa $|$Wund-/(einer) Wunde\\
3.&222.&4295.&825.&16616.&9.&5&750&400\_40\_200\_10\_100 \textcolor{red}{\textcjheb{qyrmt}} TMRJQ $|$scheuern weg/sind S"auberung\\
4.&223.&4296.&830.&16621.&14.&3&272&2\_200\_70 \textcolor{red}{\textcjheb{`rb}} BRa $|$das B"ose/gegen das "Ubel\\
5.&224.&4297.&833.&16624.&17.&5&472&6\_40\_20\_6\_400 \textcolor{red}{\textcjheb{twkmw}} WMKWT $|$und Schl"age\\
6.&225.&4298.&838.&16629.&22.&4&222&8\_4\_200\_10 \textcolor{red}{\textcjheb{yrd.h}} CDRJ $|$scheuern die Kammern/(f"ur die) Kammern\\
7.&226.&4299.&842.&16633.&26.&3&61&2\_9\_50 \textcolor{red}{\textcjheb{n.tb}} BtN $|$des Leibes\\
\end{tabular}\medskip \\
Ende des Verses 20.30\\
Verse: 584, Buchstaben: 28, 844, 16635, Totalwerte: 2633, 56018, 1181103\\
\\
Wundstriemen scheuern das B"ose weg, und Schl"age scheuern die Kammern des Leibes.\\
\\
{\bf Ende des Kapitels 20}\\
\newpage 
{\bf -- 21.1}\\
\medskip \\
\begin{tabular}{rrrrrrrrp{120mm}}
WV&WK&WB&ABK&ABB&ABV&AnzB&TW&Zahlencode \textcolor{red}{$\boldsymbol{Grundtext}$} Umschrift $|$"Ubersetzung(en)\\
1.&1.&4300.&1.&16636.&1.&4&123&80\_30\_3\_10 \textcolor{red}{\textcjheb{yglp}} PLGJ $|$(gleich) B"ache(n)\\
2.&2.&4301.&5.&16640.&5.&3&90&40\_10\_40 \textcolor{red}{\textcjheb{mym}} MJM $|$(von) Wasser\\
3.&3.&4302.&8.&16643.&8.&2&32&30\_2 \textcolor{red}{\textcjheb{bl}} LB $|$ist das Herz/(sind das) Herz\\
4.&4.&4303.&10.&16645.&10.&3&90&40\_30\_20 \textcolor{red}{\textcjheb{klm}} MLK $|$eines K"onigs/des K"onigs\\
5.&5.&4304.&13.&16648.&13.&3&16&2\_10\_4 \textcolor{red}{\textcjheb{dyb}} BJD $|$in der Hand\\
6.&6.&4305.&16.&16651.&16.&4&26&10\_5\_6\_5 \textcolor{red}{\textcjheb{hwhy}} JHWH $|$Jahwe(s)\\
7.&7.&4306.&20.&16655.&20.&2&100&70\_30 \textcolor{red}{\textcjheb{l`}} aL $|$wo/zu\\
8.&8.&4307.&22.&16657.&22.&2&50&20\_30 \textcolor{red}{\textcjheb{lk}} KL $|$hin immer/allem\\
9.&9.&4308.&24.&16659.&24.&3&501&1\_300\_200 \textcolor{red}{\textcjheb{r+s'}} ASR $|$/was\\
10.&10.&4309.&27.&16662.&27.&4&188&10\_8\_80\_90 \textcolor{red}{\textcjheb{.sp.hy}} JCP"s $|$er will/er (=es) begehrt\\
11.&11.&4310.&31.&16666.&31.&4&75&10\_9\_50\_6 \textcolor{red}{\textcjheb{wn.ty}} JtNW $|$neigt er es/er leitet es\\
\end{tabular}\medskip \\
Ende des Verses 21.1\\
Verse: 585, Buchstaben: 34, 34, 16669, Totalwerte: 1291, 1291, 1182394\\
\\
Gleich Wasserb"achen ist eines K"onigs Herz in der Hand Jahwes; wohin immer er will, neigt er es.\\
\newpage 
{\bf -- 21.2}\\
\medskip \\
\begin{tabular}{rrrrrrrrp{120mm}}
WV&WK&WB&ABK&ABB&ABV&AnzB&TW&Zahlencode \textcolor{red}{$\boldsymbol{Grundtext}$} Umschrift $|$"Ubersetzung(en)\\
1.&12.&4311.&35.&16670.&1.&2&50&20\_30 \textcolor{red}{\textcjheb{lk}} KL $|$jeder\\
2.&13.&4312.&37.&16672.&3.&3&224&4\_200\_20 \textcolor{red}{\textcjheb{krd}} DRK $|$Weg\\
3.&14.&4313.&40.&16675.&6.&3&311&1\_10\_300 \textcolor{red}{\textcjheb{+sy'}} AJS $|$(eines) Mannes\\
4.&15.&4314.&43.&16678.&9.&3&510&10\_300\_200 \textcolor{red}{\textcjheb{r+sy}} JSR $|$ist gerade\\
5.&16.&4315.&46.&16681.&12.&6&148&2\_70\_10\_50\_10\_6 \textcolor{red}{\textcjheb{wyny`b}} BaJNJW $|$in seinen Augen\\
6.&17.&4316.&52.&16687.&18.&4&476&6\_400\_20\_50 \textcolor{red}{\textcjheb{nktw}} WTKN $|$aber (es) w"agt/und pr"ufend\\
7.&18.&4317.&56.&16691.&22.&4&438&30\_2\_6\_400 \textcolor{red}{\textcjheb{twbl}} LBWT $|$(die) Herzen\\
8.&19.&4318.&60.&16695.&26.&4&26&10\_5\_6\_5 \textcolor{red}{\textcjheb{hwhy}} JHWH $|$(ist) Jahwe\\
\end{tabular}\medskip \\
Ende des Verses 21.2\\
Verse: 586, Buchstaben: 29, 63, 16698, Totalwerte: 2183, 3474, 1184577\\
\\
Jeder Weg eines Mannes ist gerade in seinen Augen, aber Jahwe w"agt die Herzen.\\
\newpage 
{\bf -- 21.3}\\
\medskip \\
\begin{tabular}{rrrrrrrrp{120mm}}
WV&WK&WB&ABK&ABB&ABV&AnzB&TW&Zahlencode \textcolor{red}{$\boldsymbol{Grundtext}$} Umschrift $|$"Ubersetzung(en)\\
1.&20.&4319.&64.&16699.&1.&3&375&70\_300\_5 \textcolor{red}{\textcjheb{h+s`}} aSH $|$"uben\\
2.&21.&4320.&67.&16702.&4.&4&199&90\_4\_100\_5 \textcolor{red}{\textcjheb{hqd.s}} "sDQH $|$Gerechtigkeit\\
3.&22.&4321.&71.&16706.&8.&5&435&6\_40\_300\_80\_9 \textcolor{red}{\textcjheb{.tp+smw}} WMSPt $|$und Recht\\
4.&23.&4322.&76.&16711.&13.&4&260&50\_2\_8\_200 \textcolor{red}{\textcjheb{r.hbn}} NBCR $|$ist angenehmer/er (=es) wird vorgezogen\\
5.&24.&4323.&80.&16715.&17.&5&56&30\_10\_5\_6\_5 \textcolor{red}{\textcjheb{hwhyl}} LJHWH $|$(von) Jahwe\\
6.&25.&4324.&85.&16720.&22.&4&57&40\_7\_2\_8 \textcolor{red}{\textcjheb{.hbzm}} MZBC $|$(mehr) als (ein) (Schlacht)Opfer\\
\end{tabular}\medskip \\
Ende des Verses 21.3\\
Verse: 587, Buchstaben: 25, 88, 16723, Totalwerte: 1382, 4856, 1185959\\
\\
Gerechtigkeit und Recht "uben ist Jahwe angenehmer als Opfer.\\
\newpage 
{\bf -- 21.4}\\
\medskip \\
\begin{tabular}{rrrrrrrrp{120mm}}
WV&WK&WB&ABK&ABB&ABV&AnzB&TW&Zahlencode \textcolor{red}{$\boldsymbol{Grundtext}$} Umschrift $|$"Ubersetzung(en)\\
1.&26.&4325.&89.&16724.&1.&3&246&200\_6\_40 \textcolor{red}{\textcjheb{mwr}} RWM $|$Stolz/H"ohe\\
2.&27.&4326.&92.&16727.&4.&5&180&70\_10\_50\_10\_40 \textcolor{red}{\textcjheb{myny`}} aJNJM $|$(der) Augen\\
3.&28.&4327.&97.&16732.&9.&4&216&6\_200\_8\_2 \textcolor{red}{\textcjheb{b.hrw}} WRCB $|$und Hochmut/und Weite\\
4.&29.&4328.&101.&16736.&13.&2&32&30\_2 \textcolor{red}{\textcjheb{bl}} LB $|$des Herzens\\
5.&30.&4329.&103.&16738.&15.&2&250&50\_200 \textcolor{red}{\textcjheb{rn}} NR $|$die Leuchte\\
6.&31.&4330.&105.&16740.&17.&5&620&200\_300\_70\_10\_40 \textcolor{red}{\textcjheb{my`+sr}} RSaJM $|$der Gesetzlosen/(der) Frevler\\
7.&32.&4331.&110.&16745.&22.&4&418&8\_9\_1\_400 \textcolor{red}{\textcjheb{t'.t.h}} CtAT $|$sind S"unde/(ist) S"unde\\
\end{tabular}\medskip \\
Ende des Verses 21.4\\
Verse: 588, Buchstaben: 25, 113, 16748, Totalwerte: 1962, 6818, 1187921\\
\\
Stolz der Augen und Hochmut des Herzens, die Leuchte der Gesetzlosen, sind S"unde.\\
\newpage 
{\bf -- 21.5}\\
\medskip \\
\begin{tabular}{rrrrrrrrp{120mm}}
WV&WK&WB&ABK&ABB&ABV&AnzB&TW&Zahlencode \textcolor{red}{$\boldsymbol{Grundtext}$} Umschrift $|$"Ubersetzung(en)\\
1.&33.&4332.&114.&16749.&1.&6&756&40\_8\_300\_2\_6\_400 \textcolor{red}{\textcjheb{twb+s.hm}} MCSBWT $|$die Gedanken/(die) Pl"ane\\
2.&34.&4333.&120.&16755.&7.&4&304&8\_200\_6\_90 \textcolor{red}{\textcjheb{.swr.h}} CRW"s $|$das Flei"sigen/(eines) Flei"sigen\\
3.&35.&4334.&124.&16759.&11.&2&21&1\_20 \textcolor{red}{\textcjheb{k'}} AK $|$f"uhren nur/(f"uhren) gewiss\\
4.&36.&4335.&126.&16761.&13.&5&676&30\_40\_6\_400\_200 \textcolor{red}{\textcjheb{rtwml}} LMWTR $|$zum "Uberfluss/zu Gewinn\\
5.&37.&4336.&131.&16766.&18.&3&56&6\_20\_30 \textcolor{red}{\textcjheb{lkw}} WKL $|$und jeder\\
6.&38.&4337.&134.&16769.&21.&2&91&1\_90 \textcolor{red}{\textcjheb{.s'}} A"s $|$der hastig ist/Eilfertige\\
7.&39.&4338.&136.&16771.&23.&2&21&1\_20 \textcolor{red}{\textcjheb{k'}} AK $|$es ist nur/(gelangt) nur\\
8.&40.&4339.&138.&16773.&25.&6&344&30\_40\_8\_60\_6\_200 \textcolor{red}{\textcjheb{rws.hml}} LMCsWR $|$(zu) (dem) Mangel\\
\end{tabular}\medskip \\
Ende des Verses 21.5\\
Verse: 589, Buchstaben: 30, 143, 16778, Totalwerte: 2269, 9087, 1190190\\
\\
Die Gedanken des Flei"sigen f"uhren nur zum "Uberflu"s; und jeder, der hastig ist-es ist nur zum Mangel.\\
\newpage 
{\bf -- 21.6}\\
\medskip \\
\begin{tabular}{rrrrrrrrp{120mm}}
WV&WK&WB&ABK&ABB&ABV&AnzB&TW&Zahlencode \textcolor{red}{$\boldsymbol{Grundtext}$} Umschrift $|$"Ubersetzung(en)\\
1.&41.&4340.&144.&16779.&1.&3&180&80\_70\_30 \textcolor{red}{\textcjheb{l`p}} PaL $|$Erwerb\\
2.&42.&4341.&147.&16782.&4.&6&703&1\_6\_90\_200\_6\_400 \textcolor{red}{\textcjheb{twr.sw'}} AW"sRWT $|$(von) Sch"atzen\\
3.&43.&4342.&153.&16788.&10.&5&388&2\_30\_300\_6\_50 \textcolor{red}{\textcjheb{nw+slb}} BLSWN $|$durch (eine) Zunge\\
4.&44.&4343.&158.&16793.&15.&3&600&300\_100\_200 \textcolor{red}{\textcjheb{rq+s}} SQR $|$(der) L"uge/der Falschheit\\
5.&45.&4344.&161.&16796.&18.&3&37&5\_2\_30 \textcolor{red}{\textcjheb{lbh}} HBL $|$ist Dunst/(ist wie) Windhauch\\
6.&46.&4345.&164.&16799.&21.&3&134&50\_4\_80 \textcolor{red}{\textcjheb{pdn}} NDP $|$verwehender/verweht\\
7.&47.&4346.&167.&16802.&24.&5&452&40\_2\_100\_300\_10 \textcolor{red}{\textcjheb{y+sqbm}} MBQSJ $|$solche suchen/(wie) Suchende\\
8.&48.&4347.&172.&16807.&29.&3&446&40\_6\_400 \textcolor{red}{\textcjheb{twm}} MWT $|$den Tod\\
\end{tabular}\medskip \\
Ende des Verses 21.6\\
Verse: 590, Buchstaben: 31, 174, 16809, Totalwerte: 2940, 12027, 1193130\\
\\
Erwerb von Sch"atzen durch L"ugenzunge ist verwehender Dunst; solche suchen den Tod.\\
\newpage 
{\bf -- 21.7}\\
\medskip \\
\begin{tabular}{rrrrrrrrp{120mm}}
WV&WK&WB&ABK&ABB&ABV&AnzB&TW&Zahlencode \textcolor{red}{$\boldsymbol{Grundtext}$} Umschrift $|$"Ubersetzung(en)\\
1.&49.&4348.&175.&16810.&1.&2&304&300\_4 \textcolor{red}{\textcjheb{d+s}} SD $|$die Gewaltt"atigkeit/die Gewalttat\\
2.&50.&4349.&177.&16812.&3.&5&620&200\_300\_70\_10\_40 \textcolor{red}{\textcjheb{my`+sr}} RSaJM $|$der Gesetzlosen/(der) Frevler\\
3.&51.&4350.&182.&16817.&8.&5&259&10\_3\_6\_200\_40 \textcolor{red}{\textcjheb{mrwgy}} JGWRM $|$rafft sie hinweg/er (=sie) rei"st sie fort\\
4.&52.&4351.&187.&16822.&13.&2&30&20\_10 \textcolor{red}{\textcjheb{yk}} KJ $|$denn\\
5.&53.&4352.&189.&16824.&15.&4&97&40\_1\_50\_6 \textcolor{red}{\textcjheb{wn'm}} MANW $|$sie weiger(t)en sich\\
6.&54.&4353.&193.&16828.&19.&5&806&30\_70\_300\_6\_400 \textcolor{red}{\textcjheb{tw+s`l}} LaSWT $|$zu "uben/zu tun\\
7.&55.&4354.&198.&16833.&24.&4&429&40\_300\_80\_9 \textcolor{red}{\textcjheb{.tp+sm}} MSPt $|$(das) Recht(e)\\
\end{tabular}\medskip \\
Ende des Verses 21.7\\
Verse: 591, Buchstaben: 27, 201, 16836, Totalwerte: 2545, 14572, 1195675\\
\\
Die Gewaltt"atigkeit der Gesetzlosen rafft sie hinweg, denn Recht zu "uben weigern sie sich.\\
\newpage 
{\bf -- 21.8}\\
\medskip \\
\begin{tabular}{rrrrrrrrp{120mm}}
WV&WK&WB&ABK&ABB&ABV&AnzB&TW&Zahlencode \textcolor{red}{$\boldsymbol{Grundtext}$} Umschrift $|$"Ubersetzung(en)\\
1.&56.&4355.&202.&16837.&1.&5&205&5\_80\_20\_80\_20 \textcolor{red}{\textcjheb{kpkph}} HPKPK $|$(viel)gewunden\\
2.&57.&4356.&207.&16842.&6.&3&224&4\_200\_20 \textcolor{red}{\textcjheb{krd}} DRK $|$(ist) der Weg\\
3.&58.&4357.&210.&16845.&9.&3&311&1\_10\_300 \textcolor{red}{\textcjheb{+sy'}} AJS $|$(des) Mannes\\
4.&59.&4358.&213.&16848.&12.&3&213&6\_7\_200 \textcolor{red}{\textcjheb{rzw}} WZR $|$(und) schuldbeladen(en)\\
5.&60.&4359.&216.&16851.&15.&3&33&6\_7\_20 \textcolor{red}{\textcjheb{kzw}} WZK $|$der Lautere aber/und der Reine\\
6.&61.&4360.&219.&16854.&18.&3&510&10\_300\_200 \textcolor{red}{\textcjheb{r+sy}} JSR $|$gerade/rechtschaffen\\
7.&62.&4361.&222.&16857.&21.&4&186&80\_70\_30\_6 \textcolor{red}{\textcjheb{wl`p}} PaLW $|$(ist) sein Tun\\
\end{tabular}\medskip \\
Ende des Verses 21.8\\
Verse: 592, Buchstaben: 24, 225, 16860, Totalwerte: 1682, 16254, 1197357\\
\\
Vielgewunden ist der Weg des schuldbeladenen Mannes; der Lautere aber, sein Tun ist gerade.\\
\newpage 
{\bf -- 21.9}\\
\medskip \\
\begin{tabular}{rrrrrrrrp{120mm}}
WV&WK&WB&ABK&ABB&ABV&AnzB&TW&Zahlencode \textcolor{red}{$\boldsymbol{Grundtext}$} Umschrift $|$"Ubersetzung(en)\\
1.&63.&4362.&226.&16861.&1.&3&17&9\_6\_2 \textcolor{red}{\textcjheb{bw.t}} tWB $|$besser ist es/gut (ist es)\\
2.&64.&4363.&229.&16864.&4.&4&732&30\_300\_2\_400 \textcolor{red}{\textcjheb{tb+sl}} LSBT $|$zu wohnen\\
3.&65.&4364.&233.&16868.&8.&2&100&70\_30 \textcolor{red}{\textcjheb{l`}} aL $|$auf\\
4.&66.&4365.&235.&16870.&10.&3&530&80\_50\_400 \textcolor{red}{\textcjheb{tnp}} PNT $|$einer Ecke/einer Zinne\\
5.&67.&4366.&238.&16873.&13.&2&6&3\_3 \textcolor{red}{\textcjheb{gg}} GG $|$(des) Dach(es)\\
6.&68.&4367.&240.&16875.&15.&4&741&40\_1\_300\_400 \textcolor{red}{\textcjheb{t+s'm}} MAST $|$als eine Frau\\
7.&69.&4368.&244.&16879.&19.&6&154&40\_4\_10\_50\_10\_40 \textcolor{red}{\textcjheb{mynydm}} MDJNJM $|$z"ankische/streits"uchtige\\
8.&70.&4369.&250.&16885.&25.&4&418&6\_2\_10\_400 \textcolor{red}{\textcjheb{tybw}} WBJT $|$und ein Haus\\
9.&71.&4370.&254.&16889.&29.&3&210&8\_2\_200 \textcolor{red}{\textcjheb{rb.h}} CBR $|$gemeinsames\\
\end{tabular}\medskip \\
Ende des Verses 21.9\\
Verse: 593, Buchstaben: 31, 256, 16891, Totalwerte: 2908, 19162, 1200265\\
\\
Besser ist es, auf einer Dachecke zu wohnen, als ein z"ankisches Weib und ein gemeinsames Haus.\\
\newpage 
{\bf -- 21.10}\\
\medskip \\
\begin{tabular}{rrrrrrrrp{120mm}}
WV&WK&WB&ABK&ABB&ABV&AnzB&TW&Zahlencode \textcolor{red}{$\boldsymbol{Grundtext}$} Umschrift $|$"Ubersetzung(en)\\
1.&72.&4371.&257.&16892.&1.&3&430&50\_80\_300 \textcolor{red}{\textcjheb{+spn}} NPS $|$die Seele\\
2.&73.&4372.&260.&16895.&4.&3&570&200\_300\_70 \textcolor{red}{\textcjheb{`+sr}} RSa $|$des Gesetzlosen/(eines) Frevlers\\
3.&74.&4373.&263.&16898.&7.&4&412&1\_6\_400\_5 \textcolor{red}{\textcjheb{htw'}} AWTH $|$(sie) begehrt\\
4.&75.&4374.&267.&16902.&11.&2&270&200\_70 \textcolor{red}{\textcjheb{`r}} Ra $|$(das) B"ose(s)\\
5.&76.&4375.&269.&16904.&13.&2&31&30\_1 \textcolor{red}{\textcjheb{'l}} LA $|$nicht\\
6.&77.&4376.&271.&16906.&15.&3&68&10\_8\_50 \textcolor{red}{\textcjheb{n.hy}} JCN $|$findet Gnade/er (=es) wird bemitleidet\\
7.&78.&4377.&274.&16909.&18.&6&148&2\_70\_10\_50\_10\_6 \textcolor{red}{\textcjheb{wyny`b}} BaJNJW $|$in seinen Augen\\
8.&79.&4378.&280.&16915.&24.&4&281&200\_70\_5\_6 \textcolor{red}{\textcjheb{wh`r}} RaHW $|$sein N"achster/sein Gef"ahrte\\
\end{tabular}\medskip \\
Ende des Verses 21.10\\
Verse: 594, Buchstaben: 27, 283, 16918, Totalwerte: 2210, 21372, 1202475\\
\\
Die Seele des Gesetzlosen begehrt das B"ose: sein N"achster findet keine Gnade in seinen Augen.\\
\newpage 
{\bf -- 21.11}\\
\medskip \\
\begin{tabular}{rrrrrrrrp{120mm}}
WV&WK&WB&ABK&ABB&ABV&AnzB&TW&Zahlencode \textcolor{red}{$\boldsymbol{Grundtext}$} Umschrift $|$"Ubersetzung(en)\\
1.&80.&4379.&284.&16919.&1.&4&422&2\_70\_50\_300 \textcolor{red}{\textcjheb{+sn`b}} BaNS $|$wenn man bestraft/durch Strafen\\
2.&81.&4380.&288.&16923.&5.&2&120&30\_90 \textcolor{red}{\textcjheb{.sl}} L"s $|$den Sp"otter/einen Sp"otter\\
3.&82.&4381.&290.&16925.&7.&4&78&10\_8\_20\_40 \textcolor{red}{\textcjheb{mk.hy}} JCKM $|$so wird weise/er (=es) wird weise\\
4.&83.&4382.&294.&16929.&11.&3&490&80\_400\_10 \textcolor{red}{\textcjheb{ytp}} PTJ $|$der Einf"altige/(der) Unerfahrene\\
5.&84.&4383.&297.&16932.&14.&7&373&6\_2\_5\_300\_20\_10\_30 \textcolor{red}{\textcjheb{lyk+shbw}} WBHSKJL $|$und wenn man belehrt/und durch Einsichtigmachen\\
6.&85.&4384.&304.&16939.&21.&4&98&30\_8\_20\_40 \textcolor{red}{\textcjheb{mk.hl}} LCKM $|$den Weisen\\
7.&86.&4385.&308.&16943.&25.&3&118&10\_100\_8 \textcolor{red}{\textcjheb{.hqy}} JQC $|$(so) er nimmt (an)\\
8.&87.&4386.&311.&16946.&28.&3&474&4\_70\_400 \textcolor{red}{\textcjheb{t`d}} DaT $|$Erkenntnis\\
\end{tabular}\medskip \\
Ende des Verses 21.11\\
Verse: 595, Buchstaben: 30, 313, 16948, Totalwerte: 2173, 23545, 1204648\\
\\
Wenn man den Sp"otter bestraft, so wird der Einf"altige weise; und wenn man den Weisen belehrt, so nimmt er Erkenntnis an.\\
\newpage 
{\bf -- 21.12}\\
\medskip \\
\begin{tabular}{rrrrrrrrp{120mm}}
WV&WK&WB&ABK&ABB&ABV&AnzB&TW&Zahlencode \textcolor{red}{$\boldsymbol{Grundtext}$} Umschrift $|$"Ubersetzung(en)\\
1.&88.&4387.&314.&16949.&1.&5&400&40\_300\_20\_10\_30 \textcolor{red}{\textcjheb{lyk+sm}} MSKJL $|$(es) hat acht/Einsicht habend\\
2.&89.&4388.&319.&16954.&6.&4&204&90\_4\_10\_100 \textcolor{red}{\textcjheb{qyd.s}} "sDJQ $|$(ist) (ein) Gerechter\\
3.&90.&4389.&323.&16958.&10.&4&442&30\_2\_10\_400 \textcolor{red}{\textcjheb{tybl}} LBJT $|$auf das Haus/bez"uglich des Hauses\\
4.&91.&4390.&327.&16962.&14.&3&570&200\_300\_70 \textcolor{red}{\textcjheb{`+sr}} RSa $|$des Gesetzlosen/(eines) Frevlers\\
5.&92.&4391.&330.&16965.&17.&4&210&40\_60\_30\_80 \textcolor{red}{\textcjheb{plsm}} MsLP $|$er st"urzt/ein zu Fall Bringender\\
6.&93.&4392.&334.&16969.&21.&5&620&200\_300\_70\_10\_40 \textcolor{red}{\textcjheb{my`+sr}} RSaJM $|$die Gesetzlosen/(die) Frevler\\
7.&94.&4393.&339.&16974.&26.&3&300&30\_200\_70 \textcolor{red}{\textcjheb{`rl}} LRa $|$ins Ungl"uck/(in) Unheil\\
\end{tabular}\medskip \\
Ende des Verses 21.12\\
Verse: 596, Buchstaben: 28, 341, 16976, Totalwerte: 2746, 26291, 1207394\\
\\
Ein Gerechter hat acht auf das Haus des Gesetzlosen, er st"urzt die Gesetzlosen ins Ungl"uck.\\
\newpage 
{\bf -- 21.13}\\
\medskip \\
\begin{tabular}{rrrrrrrrp{120mm}}
WV&WK&WB&ABK&ABB&ABV&AnzB&TW&Zahlencode \textcolor{red}{$\boldsymbol{Grundtext}$} Umschrift $|$"Ubersetzung(en)\\
1.&95.&4394.&342.&16977.&1.&3&50&1\_9\_40 \textcolor{red}{\textcjheb{m.t'}} AtM $|$wer verstopft/ein Verschlie"sender\\
2.&96.&4395.&345.&16980.&4.&4&64&1\_7\_50\_6 \textcolor{red}{\textcjheb{wnz'}} AZNW $|$sein Ohr\\
3.&97.&4396.&349.&16984.&8.&5&617&40\_7\_70\_100\_400 \textcolor{red}{\textcjheb{tq`zm}} MZaQT $|$vor dem Schrei\\
4.&98.&4397.&354.&16989.&13.&2&34&4\_30 \textcolor{red}{\textcjheb{ld}} DL $|$(des) Armen\\
5.&99.&4398.&356.&16991.&15.&2&43&3\_40 \textcolor{red}{\textcjheb{mg}} GM $|$auch\\
6.&100.&4399.&358.&16993.&17.&3&12&5\_6\_1 \textcolor{red}{\textcjheb{'wh}} HWA $|$(d)er\\
7.&101.&4400.&361.&16996.&20.&4&311&10\_100\_200\_1 \textcolor{red}{\textcjheb{'rqy}} JQRA $|$(er) wird rufen\\
8.&102.&4401.&365.&17000.&24.&3&37&6\_30\_1 \textcolor{red}{\textcjheb{'lw}} WLA $|$und nicht\\
9.&103.&4402.&368.&17003.&27.&4&135&10\_70\_50\_5 \textcolor{red}{\textcjheb{hn`y}} JaNH $|$erh"ort werden/er wird beantwortet\\
\end{tabular}\medskip \\
Ende des Verses 21.13\\
Verse: 597, Buchstaben: 30, 371, 17006, Totalwerte: 1303, 27594, 1208697\\
\\
Wer sein Ohr verstopft vor dem Schrei des Armen, auch er wird rufen und nicht erh"ort werden.\\
\newpage 
{\bf -- 21.14}\\
\medskip \\
\begin{tabular}{rrrrrrrrp{120mm}}
WV&WK&WB&ABK&ABB&ABV&AnzB&TW&Zahlencode \textcolor{red}{$\boldsymbol{Grundtext}$} Umschrift $|$"Ubersetzung(en)\\
1.&104.&4403.&372.&17007.&1.&3&490&40\_400\_50 \textcolor{red}{\textcjheb{ntm}} MTN $|$eine Gabe\\
2.&105.&4404.&375.&17010.&4.&4&662&2\_60\_400\_200 \textcolor{red}{\textcjheb{rtsb}} BsTR $|$im Verborgenen/im geheimen\\
3.&106.&4405.&379.&17014.&8.&4&115&10\_20\_80\_5 \textcolor{red}{\textcjheb{hpky}} JKPH $|$(er (=sie)) wendet ab\\
4.&107.&4406.&383.&17018.&12.&2&81&1\_80 \textcolor{red}{\textcjheb{p'}} AP $|$(den) Zorn\\
5.&108.&4407.&385.&17020.&14.&4&318&6\_300\_8\_4 \textcolor{red}{\textcjheb{d.h+sw}} WSCD $|$und (ein) Geschenk\\
6.&109.&4408.&389.&17024.&18.&3&110&2\_8\_100 \textcolor{red}{\textcjheb{q.hb}} BCQ $|$im Busen\\
7.&110.&4409.&392.&17027.&21.&3&53&8\_40\_5 \textcolor{red}{\textcjheb{hm.h}} CMH $|$den Grimm/Glut\\
8.&111.&4410.&395.&17030.&24.&3&82&70\_7\_5 \textcolor{red}{\textcjheb{hz`}} aZH $|$heftige(n)\\
\end{tabular}\medskip \\
Ende des Verses 21.14\\
Verse: 598, Buchstaben: 26, 397, 17032, Totalwerte: 1911, 29505, 1210608\\
\\
Eine Gabe im Verborgenen wendet den Zorn ab, und ein Geschenk im Busen den heftigen Grimm.\\
\newpage 
{\bf -- 21.15}\\
\medskip \\
\begin{tabular}{rrrrrrrrp{120mm}}
WV&WK&WB&ABK&ABB&ABV&AnzB&TW&Zahlencode \textcolor{red}{$\boldsymbol{Grundtext}$} Umschrift $|$"Ubersetzung(en)\\
1.&112.&4411.&398.&17033.&1.&4&353&300\_40\_8\_5 \textcolor{red}{\textcjheb{h.hm+s}} SMCH $|$Freude\\
2.&113.&4412.&402.&17037.&5.&5&234&30\_90\_4\_10\_100 \textcolor{red}{\textcjheb{qyd.sl}} L"sDJQ $|$ist es dem Gerechten/hat der Gerechte\\
3.&114.&4413.&407.&17042.&10.&4&776&70\_300\_6\_400 \textcolor{red}{\textcjheb{tw+s`}} aSWT $|$zu "uben/(am) Machen\\
4.&115.&4414.&411.&17046.&14.&4&429&40\_300\_80\_9 \textcolor{red}{\textcjheb{.tp+sm}} MSPt $|$(das) Recht\\
5.&116.&4415.&415.&17050.&18.&5&459&6\_40\_8\_400\_5 \textcolor{red}{\textcjheb{ht.hmw}} WMCTH $|$aber ein Schrecken/und Schrecken\\
6.&117.&4416.&420.&17055.&23.&5&220&30\_80\_70\_30\_10 \textcolor{red}{\textcjheb{yl`pl}} LPaLJ $|$denen die tun/(ist beschieden) den Tuenden\\
7.&118.&4417.&425.&17060.&28.&3&57&1\_6\_50 \textcolor{red}{\textcjheb{nw'}} AWN $|$Frevel/Unrecht\\
\end{tabular}\medskip \\
Ende des Verses 21.15\\
Verse: 599, Buchstaben: 30, 427, 17062, Totalwerte: 2528, 32033, 1213136\\
\\
Dem Gerechten ist es Freude, Recht zu "uben; aber denen, die Frevel tun, ein Schrecken.\\
\newpage 
{\bf -- 21.16}\\
\medskip \\
\begin{tabular}{rrrrrrrrp{120mm}}
WV&WK&WB&ABK&ABB&ABV&AnzB&TW&Zahlencode \textcolor{red}{$\boldsymbol{Grundtext}$} Umschrift $|$"Ubersetzung(en)\\
1.&119.&4418.&428.&17063.&1.&3&45&1\_4\_40 \textcolor{red}{\textcjheb{md'}} ADM $|$(ein) Mensch\\
2.&120.&4419.&431.&17066.&4.&4&481&400\_6\_70\_5 \textcolor{red}{\textcjheb{h`wt}} TWaH $|$der abirrt/abirrend(er)\\
3.&121.&4420.&435.&17070.&8.&4&264&40\_4\_200\_20 \textcolor{red}{\textcjheb{krdm}} MDRK $|$vom Weg\\
4.&122.&4421.&439.&17074.&12.&4&355&5\_300\_20\_30 \textcolor{red}{\textcjheb{lk+sh}} HSKL $|$der Einsicht/des Einsehens\\
5.&123.&4422.&443.&17078.&16.&4&137&2\_100\_5\_30 \textcolor{red}{\textcjheb{lhqb}} BQHL $|$in der Versammlung\\
6.&124.&4423.&447.&17082.&20.&5&331&200\_80\_1\_10\_40 \textcolor{red}{\textcjheb{my'pr}} RPAJM $|$der Schatten/der Verstorbenen\\
7.&125.&4424.&452.&17087.&25.&4&74&10\_50\_6\_8 \textcolor{red}{\textcjheb{.hwny}} JNWC $|$(er) wird ruhen\\
\end{tabular}\medskip \\
Ende des Verses 21.16\\
Verse: 600, Buchstaben: 28, 455, 17090, Totalwerte: 1687, 33720, 1214823\\
\\
Ein Mensch, der von dem Wege der Einsicht abirrt, wird ruhen in der Versammlung der Schatten.\\
\newpage 
{\bf -- 21.17}\\
\medskip \\
\begin{tabular}{rrrrrrrrp{120mm}}
WV&WK&WB&ABK&ABB&ABV&AnzB&TW&Zahlencode \textcolor{red}{$\boldsymbol{Grundtext}$} Umschrift $|$"Ubersetzung(en)\\
1.&126.&4425.&456.&17091.&1.&3&311&1\_10\_300 \textcolor{red}{\textcjheb{+sy'}} AJS $|$(ein) Mann\\
2.&127.&4426.&459.&17094.&4.&5&314&40\_8\_60\_6\_200 \textcolor{red}{\textcjheb{rws.hm}} MCsWR $|$des Mangels\\
3.&128.&4427.&464.&17099.&9.&3&8&1\_5\_2 \textcolor{red}{\textcjheb{bh'}} AHB $|$(wird) wird werden wer liebt/(ein) Liebender\\
4.&129.&4428.&467.&17102.&12.&4&353&300\_40\_8\_5 \textcolor{red}{\textcjheb{h.hm+s}} SMCH $|$Freude/Fr"ohlichkeit\\
5.&130.&4429.&471.&17106.&16.&3&8&1\_5\_2 \textcolor{red}{\textcjheb{bh'}} AHB $|$wer liebt/(ein) Liebender\\
6.&131.&4430.&474.&17109.&19.&3&70&10\_10\_50 \textcolor{red}{\textcjheb{nyy}} JJN $|$Wein\\
7.&132.&4431.&477.&17112.&22.&4&396&6\_300\_40\_50 \textcolor{red}{\textcjheb{nm+sw}} WSMN $|$und "Ol\\
8.&133.&4432.&481.&17116.&26.&2&31&30\_1 \textcolor{red}{\textcjheb{'l}} LA $|$nicht\\
9.&134.&4433.&483.&17118.&28.&5&590&10\_70\_300\_10\_200 \textcolor{red}{\textcjheb{ry+s`y}} JaSJR $|$(er) wird reich (machen) (sich)\\
\end{tabular}\medskip \\
Ende des Verses 21.17\\
Verse: 601, Buchstaben: 32, 487, 17122, Totalwerte: 2081, 35801, 1216904\\
\\
Wer Freude liebt, wird ein Mann des Mangels werden; wer Wein und "Ol liebt, wird nicht reich.\\
\newpage 
{\bf -- 21.18}\\
\medskip \\
\begin{tabular}{rrrrrrrrp{120mm}}
WV&WK&WB&ABK&ABB&ABV&AnzB&TW&Zahlencode \textcolor{red}{$\boldsymbol{Grundtext}$} Umschrift $|$"Ubersetzung(en)\\
1.&135.&4434.&488.&17123.&1.&3&300&20\_80\_200 \textcolor{red}{\textcjheb{rpk}} KPR $|$ein L"osegeld/S"uhne\\
2.&136.&4435.&491.&17126.&4.&5&234&30\_90\_4\_10\_100 \textcolor{red}{\textcjheb{qyd.sl}} L"sDJQ $|$f"ur den Gerechten\\
3.&137.&4436.&496.&17131.&9.&3&570&200\_300\_70 \textcolor{red}{\textcjheb{`+sr}} RSa $|$ist der Gesetzlose/(ist) (der) Frevler\\
4.&138.&4437.&499.&17134.&12.&4&814&6\_400\_8\_400 \textcolor{red}{\textcjheb{t.htw}} WTCT $|$und an die Stelle\\
5.&139.&4438.&503.&17138.&16.&5&560&10\_300\_200\_10\_40 \textcolor{red}{\textcjheb{myr+sy}} JSRJM $|$der Aufrichtigen/der Geraden\\
6.&140.&4439.&508.&17143.&21.&4&15&2\_6\_3\_4 \textcolor{red}{\textcjheb{dgwb}} BWGD $|$tritt der Treulose/(tritt) der Abtr"unnige\\
\end{tabular}\medskip \\
Ende des Verses 21.18\\
Verse: 602, Buchstaben: 24, 511, 17146, Totalwerte: 2493, 38294, 1219397\\
\\
Der Gesetzlose ist ein L"osegeld f"ur den Gerechten, und der Treulose tritt an die Stelle der Aufrichtigen.\\
\newpage 
{\bf -- 21.19}\\
\medskip \\
\begin{tabular}{rrrrrrrrp{120mm}}
WV&WK&WB&ABK&ABB&ABV&AnzB&TW&Zahlencode \textcolor{red}{$\boldsymbol{Grundtext}$} Umschrift $|$"Ubersetzung(en)\\
1.&141.&4440.&512.&17147.&1.&3&17&9\_6\_2 \textcolor{red}{\textcjheb{bw.t}} tWB $|$besser ist es/gut (ist)\\
2.&142.&4441.&515.&17150.&4.&3&702&300\_2\_400 \textcolor{red}{\textcjheb{tb+s}} SBT $|$zu wohnen/ein Weilen\\
3.&143.&4442.&518.&17153.&7.&4&293&2\_1\_200\_90 \textcolor{red}{\textcjheb{.sr'b}} BAR"s $|$in einem Lande\\
4.&144.&4443.&522.&17157.&11.&4&246&40\_4\_2\_200 \textcolor{red}{\textcjheb{rbdm}} MDBR $|$w"usten/der W"uste\\
5.&145.&4444.&526.&17161.&15.&4&741&40\_1\_300\_400 \textcolor{red}{\textcjheb{t+s'm}} MAST $|$als eine Frau\\
6.&146.&4445.&530.&17165.&19.&6&150&40\_4\_6\_50\_10\_40 \textcolor{red}{\textcjheb{mynwdm}} MDWNJM $|$z"ankische/streits"uchtige\\
7.&147.&4446.&536.&17171.&25.&4&156&6\_20\_70\_60 \textcolor{red}{\textcjheb{s`kw}} WKas $|$und "Arger/und des Verdrusses\\
\end{tabular}\medskip \\
Ende des Verses 21.19\\
Verse: 603, Buchstaben: 28, 539, 17174, Totalwerte: 2305, 40599, 1221702\\
\\
Besser ist es, in einem w"usten Lande zu wohnen, als ein z"ankisches Weib und "Arger.\\
\newpage 
{\bf -- 21.20}\\
\medskip \\
\begin{tabular}{rrrrrrrrp{120mm}}
WV&WK&WB&ABK&ABB&ABV&AnzB&TW&Zahlencode \textcolor{red}{$\boldsymbol{Grundtext}$} Umschrift $|$"Ubersetzung(en)\\
1.&148.&4447.&540.&17175.&1.&4&297&1\_6\_90\_200 \textcolor{red}{\textcjheb{r.sw'}} AW"sR $|$(ein) Schatz\\
2.&149.&4448.&544.&17179.&5.&4&102&50\_8\_40\_4 \textcolor{red}{\textcjheb{dm.hn}} NCMD $|$kostbarer/(ist) angenehm\\
3.&150.&4449.&548.&17183.&9.&4&396&6\_300\_40\_50 \textcolor{red}{\textcjheb{nm+sw}} WSMN $|$und "Ol (ist)\\
4.&151.&4450.&552.&17187.&13.&4&63&2\_50\_6\_5 \textcolor{red}{\textcjheb{hwnb}} BNWH $|$in der Wohnung\\
5.&152.&4451.&556.&17191.&17.&3&68&8\_20\_40 \textcolor{red}{\textcjheb{mk.h}} CKM $|$des Weisen/(eines) Weisen\\
6.&153.&4452.&559.&17194.&20.&5&126&6\_20\_60\_10\_30 \textcolor{red}{\textcjheb{lyskw}} WKsJL $|$aber ein t"orichter/und ein t"orichter\\
7.&154.&4453.&564.&17199.&25.&3&45&1\_4\_40 \textcolor{red}{\textcjheb{md'}} ADM $|$Mensch\\
8.&155.&4454.&567.&17202.&28.&6&168&10\_2\_30\_70\_50\_6 \textcolor{red}{\textcjheb{wn`lby}} JBLaNW $|$verschlingt es/(er) vergeudet ihn\\
\end{tabular}\medskip \\
Ende des Verses 21.20\\
Verse: 604, Buchstaben: 33, 572, 17207, Totalwerte: 1265, 41864, 1222967\\
\\
Ein kostbarer Schatz und "Ol ist in der Wohnung des Weisen, aber ein t"orichter Mensch verschlingt es.\\
\newpage 
{\bf -- 21.21}\\
\medskip \\
\begin{tabular}{rrrrrrrrp{120mm}}
WV&WK&WB&ABK&ABB&ABV&AnzB&TW&Zahlencode \textcolor{red}{$\boldsymbol{Grundtext}$} Umschrift $|$"Ubersetzung(en)\\
1.&156.&4455.&573.&17208.&1.&3&284&200\_4\_80 \textcolor{red}{\textcjheb{pdr}} RDP $|$wer nachjagt/ein Strebender (nach)\\
2.&157.&4456.&576.&17211.&4.&4&199&90\_4\_100\_5 \textcolor{red}{\textcjheb{hqd.s}} "sDQH $|$(der) Gerechtigkeit\\
3.&158.&4457.&580.&17215.&8.&4&78&6\_8\_60\_4 \textcolor{red}{\textcjheb{ds.hw}} WCsD $|$und der G"ute/und Liebe\\
4.&159.&4458.&584.&17219.&12.&4&141&10\_40\_90\_1 \textcolor{red}{\textcjheb{'.smy}} JM"sA $|$wird finden/(d)er findet\\
5.&160.&4459.&588.&17223.&16.&4&68&8\_10\_10\_40 \textcolor{red}{\textcjheb{myy.h}} CJJM $|$Leben\\
6.&161.&4460.&592.&17227.&20.&4&199&90\_4\_100\_5 \textcolor{red}{\textcjheb{hqd.s}} "sDQH $|$Gerechtigkeit\\
7.&162.&4461.&596.&17231.&24.&5&38&6\_20\_2\_6\_4 \textcolor{red}{\textcjheb{dwbkw}} WKBWD $|$und Ehre\\
\end{tabular}\medskip \\
Ende des Verses 21.21\\
Verse: 605, Buchstaben: 28, 600, 17235, Totalwerte: 1007, 42871, 1223974\\
\\
Wer der Gerechtigkeit und der G"ute nachjagt, wird Leben finden, Gerechtigkeit und Ehre.\\
\newpage 
{\bf -- 21.22}\\
\medskip \\
\begin{tabular}{rrrrrrrrp{120mm}}
WV&WK&WB&ABK&ABB&ABV&AnzB&TW&Zahlencode \textcolor{red}{$\boldsymbol{Grundtext}$} Umschrift $|$"Ubersetzung(en)\\
1.&163.&4462.&601.&17236.&1.&3&280&70\_10\_200 \textcolor{red}{\textcjheb{ry`}} aJR $|$die Stadt/(eine) Stadt\\
2.&164.&4463.&604.&17239.&4.&5&255&3\_2\_200\_10\_40 \textcolor{red}{\textcjheb{myrbg}} GBRJM $|$der Helden/(von) Helden\\
3.&165.&4464.&609.&17244.&9.&3&105&70\_30\_5 \textcolor{red}{\textcjheb{hl`}} aLH $|$(er (=es)) ersteigt\\
4.&166.&4465.&612.&17247.&12.&3&68&8\_20\_40 \textcolor{red}{\textcjheb{mk.h}} CKM $|$(der) Weise\\
5.&167.&4466.&615.&17250.&15.&4&220&6\_10\_200\_4 \textcolor{red}{\textcjheb{dryw}} WJRD $|$und (er) st"urzt (nieder)\\
6.&168.&4467.&619.&17254.&19.&2&77&70\_7 \textcolor{red}{\textcjheb{z`}} aZ $|$die Feste/(die) Macht\\
7.&169.&4468.&621.&17256.&21.&5&64&40\_2\_9\_8\_5 \textcolor{red}{\textcjheb{h.h.tbm}} MBtCH $|$ihres Vertrauens\\
\end{tabular}\medskip \\
Ende des Verses 21.22\\
Verse: 606, Buchstaben: 25, 625, 17260, Totalwerte: 1069, 43940, 1225043\\
\\
Der Weise ersteigt die Stadt der Helden und st"urzt nieder die Feste ihres Vertrauens.\\
\newpage 
{\bf -- 21.23}\\
\medskip \\
\begin{tabular}{rrrrrrrrp{120mm}}
WV&WK&WB&ABK&ABB&ABV&AnzB&TW&Zahlencode \textcolor{red}{$\boldsymbol{Grundtext}$} Umschrift $|$"Ubersetzung(en)\\
1.&170.&4469.&626.&17261.&1.&3&540&300\_40\_200 \textcolor{red}{\textcjheb{rm+s}} SMR $|$wer bewahrt/(ein) H"utender\\
2.&171.&4470.&629.&17264.&4.&3&96&80\_10\_6 \textcolor{red}{\textcjheb{wyp}} PJW $|$seinen Mund\\
3.&172.&4471.&632.&17267.&7.&6&398&6\_30\_300\_6\_50\_6 \textcolor{red}{\textcjheb{wnw+slw}} WLSWNW $|$und seine Zunge\\
4.&173.&4472.&638.&17273.&13.&3&540&300\_40\_200 \textcolor{red}{\textcjheb{rm+s}} SMR $|$bewahrt/(ist) bewahrend\\
5.&174.&4473.&641.&17276.&16.&5&736&40\_90\_200\_6\_400 \textcolor{red}{\textcjheb{twr.sm}} M"sRWT $|$vor Drangsalen/vor Bedr"angnissen\\
6.&175.&4474.&646.&17281.&21.&4&436&50\_80\_300\_6 \textcolor{red}{\textcjheb{w+spn}} NPSW $|$seine Seele\\
\end{tabular}\medskip \\
Ende des Verses 21.23\\
Verse: 607, Buchstaben: 24, 649, 17284, Totalwerte: 2746, 46686, 1227789\\
\\
Wer seinen Mund und seine Zunge bewahrt, bewahrt vor Drangsalen seine Seele.\\
\newpage 
{\bf -- 21.24}\\
\medskip \\
\begin{tabular}{rrrrrrrrp{120mm}}
WV&WK&WB&ABK&ABB&ABV&AnzB&TW&Zahlencode \textcolor{red}{$\boldsymbol{Grundtext}$} Umschrift $|$"Ubersetzung(en)\\
1.&176.&4475.&650.&17285.&1.&2&11&7\_4 \textcolor{red}{\textcjheb{dz}} ZD $|$der "Uberm"utige/ein "Uberm"utiger\\
2.&177.&4476.&652.&17287.&3.&4&225&10\_5\_10\_200 \textcolor{red}{\textcjheb{ryhy}} JHJR $|$Stolze(r)\\
3.&178.&4477.&656.&17291.&7.&2&120&30\_90 \textcolor{red}{\textcjheb{.sl}} L"s $|$Sp"otter\\
4.&179.&4478.&658.&17293.&9.&3&346&300\_40\_6 \textcolor{red}{\textcjheb{wm+s}} SMW $|$(ist) sein Name\\
5.&180.&4479.&661.&17296.&12.&4&381&70\_6\_300\_5 \textcolor{red}{\textcjheb{h+sw`}} aWSH $|$handelt/(ist) handelnd\\
6.&181.&4480.&665.&17300.&16.&5&674&2\_70\_2\_200\_400 \textcolor{red}{\textcjheb{trb`b}} BaBRT $|$mit "Ubermut/in "Ubermut\\
7.&182.&4481.&670.&17305.&21.&4&67&7\_4\_6\_50 \textcolor{red}{\textcjheb{nwdz}} ZDWN $|$vermessenem/der Frechheit\\
\end{tabular}\medskip \\
Ende des Verses 21.24\\
Verse: 608, Buchstaben: 24, 673, 17308, Totalwerte: 1824, 48510, 1229613\\
\\
Der "Uberm"utige, Stolze-Sp"otter ist sein Name-handelt mit vermessenem "Ubermut.\\
\newpage 
{\bf -- 21.25}\\
\medskip \\
\begin{tabular}{rrrrrrrrp{120mm}}
WV&WK&WB&ABK&ABB&ABV&AnzB&TW&Zahlencode \textcolor{red}{$\boldsymbol{Grundtext}$} Umschrift $|$"Ubersetzung(en)\\
1.&183.&4482.&674.&17309.&1.&4&807&400\_1\_6\_400 \textcolor{red}{\textcjheb{tw't}} TAWT $|$die Begierde\\
2.&184.&4483.&678.&17313.&5.&3&190&70\_90\_30 \textcolor{red}{\textcjheb{l.s`}} a"sL $|$des Faulen/(des) Tr"agen\\
3.&185.&4484.&681.&17316.&8.&6&906&400\_40\_10\_400\_50\_6 \textcolor{red}{\textcjheb{wntymt}} TMJTNW $|$(sie) t"otet ihn\\
4.&186.&4485.&687.&17322.&14.&2&30&20\_10 \textcolor{red}{\textcjheb{yk}} KJ $|$denn\\
5.&187.&4486.&689.&17324.&16.&4&97&40\_1\_50\_6 \textcolor{red}{\textcjheb{wn'm}} MANW $|$(sie (=es)) weiger(t)en sich\\
6.&188.&4487.&693.&17328.&20.&4&30&10\_4\_10\_6 \textcolor{red}{\textcjheb{wydy}} JDJW $|$seine H"ande\\
7.&189.&4488.&697.&17332.&24.&5&806&30\_70\_300\_6\_400 \textcolor{red}{\textcjheb{tw+s`l}} LaSWT $|$zu arbeiten\\
\end{tabular}\medskip \\
Ende des Verses 21.25\\
Verse: 609, Buchstaben: 28, 701, 17336, Totalwerte: 2866, 51376, 1232479\\
\\
Die Begierde des Faulen t"otet ihn, denn seine H"ande weigern sich zu arbeiten.\\
\newpage 
{\bf -- 21.26}\\
\medskip \\
\begin{tabular}{rrrrrrrrp{120mm}}
WV&WK&WB&ABK&ABB&ABV&AnzB&TW&Zahlencode \textcolor{red}{$\boldsymbol{Grundtext}$} Umschrift $|$"Ubersetzung(en)\\
1.&190.&4489.&702.&17337.&1.&2&50&20\_30 \textcolor{red}{\textcjheb{lk}} KL $|$den ganzen/all\\
2.&191.&4490.&704.&17339.&3.&4&61&5\_10\_6\_40 \textcolor{red}{\textcjheb{mwyh}} HJWM $|$(der) Tag\\
3.&192.&4491.&708.&17343.&7.&5&417&5\_400\_1\_6\_5 \textcolor{red}{\textcjheb{hw'th}} HTAWH $|$begehrt/er zeig(t)e sich begierig\\
4.&193.&4492.&713.&17348.&12.&4&412&400\_1\_6\_5 \textcolor{red}{\textcjheb{hw't}} TAWH $|$(und) begehrt man/(im) Verlangen\\
5.&194.&4493.&717.&17352.&16.&5&210&6\_90\_4\_10\_100 \textcolor{red}{\textcjheb{qyd.sw}} W"sDJQ $|$aber der Gerechte/und ein Gerechter\\
6.&195.&4494.&722.&17357.&21.&3&460&10\_400\_50 \textcolor{red}{\textcjheb{nty}} JTN $|$(er) gibt\\
7.&196.&4495.&725.&17360.&24.&3&37&6\_30\_1 \textcolor{red}{\textcjheb{'lw}} WLA $|$und nicht\\
8.&197.&4496.&728.&17363.&27.&4&338&10\_8\_300\_20 \textcolor{red}{\textcjheb{k+s.hy}} JCSK $|$(er) h"alt zur"uck\\
\end{tabular}\medskip \\
Ende des Verses 21.26\\
Verse: 610, Buchstaben: 30, 731, 17366, Totalwerte: 1985, 53361, 1234464\\
\\
Den ganzen Tag begehrt und begehrt man, aber der Gerechte gibt und h"alt nicht zur"uck.\\
\newpage 
{\bf -- 21.27}\\
\medskip \\
\begin{tabular}{rrrrrrrrp{120mm}}
WV&WK&WB&ABK&ABB&ABV&AnzB&TW&Zahlencode \textcolor{red}{$\boldsymbol{Grundtext}$} Umschrift $|$"Ubersetzung(en)\\
1.&198.&4497.&732.&17367.&1.&3&17&7\_2\_8 \textcolor{red}{\textcjheb{.hbz}} ZBC $|$das Opfer/(ein) (Schlacht)Opfer\\
2.&199.&4498.&735.&17370.&4.&5&620&200\_300\_70\_10\_40 \textcolor{red}{\textcjheb{my`+sr}} RSaJM $|$der Gesetzlosen/(der) Frevler\\
3.&200.&4499.&740.&17375.&9.&5&483&400\_6\_70\_2\_5 \textcolor{red}{\textcjheb{hb`wt}} TWaBH $|$(ist) (ein) Gr"auel\\
4.&201.&4500.&745.&17380.&14.&2&81&1\_80 \textcolor{red}{\textcjheb{p'}} AP $|$wiewiel mehr/auch\\
5.&202.&4501.&747.&17382.&16.&2&30&20\_10 \textcolor{red}{\textcjheb{yk}} KJ $|$wenn\\
6.&203.&4502.&749.&17384.&18.&4&54&2\_7\_40\_5 \textcolor{red}{\textcjheb{hmzb}} BZMH $|$in b"oser Absicht/durch eine Schandtat\\
7.&204.&4503.&753.&17388.&22.&6&79&10\_2\_10\_1\_50\_6 \textcolor{red}{\textcjheb{wn'yby}} JBJANW $|$er bringt (dar) ihn (=es)\\
\end{tabular}\medskip \\
Ende des Verses 21.27\\
Verse: 611, Buchstaben: 27, 758, 17393, Totalwerte: 1364, 54725, 1235828\\
\\
Das Opfer der Gesetzlosen ist ein Greuel; wieviel mehr, wenn er es in b"oser Absicht bringt.\\
\newpage 
{\bf -- 21.28}\\
\medskip \\
\begin{tabular}{rrrrrrrrp{120mm}}
WV&WK&WB&ABK&ABB&ABV&AnzB&TW&Zahlencode \textcolor{red}{$\boldsymbol{Grundtext}$} Umschrift $|$"Ubersetzung(en)\\
1.&205.&4504.&759.&17394.&1.&2&74&70\_4 \textcolor{red}{\textcjheb{d`}} aD $|$eine Zunge/(ein) Zeuge\\
2.&206.&4505.&761.&17396.&3.&5&79&20\_7\_2\_10\_40 \textcolor{red}{\textcjheb{mybzk}} KZBJM $|$(der) L"ugen\\
3.&207.&4506.&766.&17401.&8.&4&17&10\_1\_2\_4 \textcolor{red}{\textcjheb{db'y}} JABD $|$wird umkommen/(er) geht zugrunde\\
4.&208.&4507.&770.&17405.&12.&4&317&6\_1\_10\_300 \textcolor{red}{\textcjheb{+sy'w}} WAJS $|$ein Mann aber/und (eines) Mann\\
5.&209.&4508.&774.&17409.&16.&4&416&300\_6\_40\_70 \textcolor{red}{\textcjheb{`mw+s}} SWMa $|$welcher h"ort/h"orend(en)\\
6.&210.&4509.&778.&17413.&20.&4&178&30\_50\_90\_8 \textcolor{red}{\textcjheb{.h.snl}} LN"sC $|$immerdar/hat Bestand\\
7.&211.&4510.&782.&17417.&24.&4&216&10\_4\_2\_200 \textcolor{red}{\textcjheb{rbdy}} JDBR $|$darf reden/(was) er redet\\
\end{tabular}\medskip \\
Ende des Verses 21.28\\
Verse: 612, Buchstaben: 27, 785, 17420, Totalwerte: 1297, 56022, 1237125\\
\\
Ein L"ugenzeuge wird umkommen; ein Mann aber, welcher h"ort, darf immerdar reden.\\
\newpage 
{\bf -- 21.29}\\
\medskip \\
\begin{tabular}{rrrrrrrrp{120mm}}
WV&WK&WB&ABK&ABB&ABV&AnzB&TW&Zahlencode \textcolor{red}{$\boldsymbol{Grundtext}$} Umschrift $|$"Ubersetzung(en)\\
1.&212.&4511.&786.&17421.&1.&3&82&5\_70\_7 \textcolor{red}{\textcjheb{z`h}} HaZ $|$(es) zeigt ein trotziges/er (=es) zeigt(e) Trotz\\
2.&213.&4512.&789.&17424.&4.&3&311&1\_10\_300 \textcolor{red}{\textcjheb{+sy'}} AJS $|$(ein) Mann\\
3.&214.&4513.&792.&17427.&7.&3&570&200\_300\_70 \textcolor{red}{\textcjheb{`+sr}} RSa $|$gesetzloser/b"oser\\
4.&215.&4514.&795.&17430.&10.&5&148&2\_80\_50\_10\_6 \textcolor{red}{\textcjheb{wynpb}} BPNJW $|$(auf seinem) Gesicht\\
5.&216.&4515.&800.&17435.&15.&4&516&6\_10\_300\_200 \textcolor{red}{\textcjheb{r+syw}} WJSR $|$aber der Aufrichtige/und ein Gerader\\
6.&217.&4516.&804.&17439.&19.&3&12&5\_6\_1 \textcolor{red}{\textcjheb{'wh}} HWA $|$er\\
7.&218.&4517.&807.&17442.&22.&4&90&10\_20\_10\_50 \textcolor{red}{\textcjheb{nyky}} JKJN $|$(er) merkt auf\\
8.&219.&4518.&811.&17446.&26.&5&240&4\_200\_20\_10\_6 \textcolor{red}{\textcjheb{wykrd}} DRKJW $|$seinen Weg\\
\end{tabular}\medskip \\
Ende des Verses 21.29\\
Verse: 613, Buchstaben: 30, 815, 17450, Totalwerte: 1969, 57991, 1239094\\
\\
Ein gesetzloser Mann zeigt ein trotziges Gesicht; aber der Aufrichtige, er merkt auf seinen Weg.\\
\newpage 
{\bf -- 21.30}\\
\medskip \\
\begin{tabular}{rrrrrrrrp{120mm}}
WV&WK&WB&ABK&ABB&ABV&AnzB&TW&Zahlencode \textcolor{red}{$\boldsymbol{Grundtext}$} Umschrift $|$"Ubersetzung(en)\\
1.&220.&4519.&816.&17451.&1.&3&61&1\_10\_50 \textcolor{red}{\textcjheb{ny'}} AJN $|$da ist keine/nicht gibt es\\
2.&221.&4520.&819.&17454.&4.&4&73&8\_20\_40\_5 \textcolor{red}{\textcjheb{hmk.h}} CKMH $|$Weisheit\\
3.&222.&4521.&823.&17458.&8.&4&67&6\_1\_10\_50 \textcolor{red}{\textcjheb{ny'w}} WAJN $|$und keine/und nicht gibt es\\
4.&223.&4522.&827.&17462.&12.&5&463&400\_2\_6\_50\_5 \textcolor{red}{\textcjheb{hnwbt}} TBWNH $|$Einsicht\\
5.&224.&4523.&832.&17467.&17.&4&67&6\_1\_10\_50 \textcolor{red}{\textcjheb{ny'w}} WAJN $|$und kein/und nicht gibt es\\
6.&225.&4524.&836.&17471.&21.&3&165&70\_90\_5 \textcolor{red}{\textcjheb{h.s`}} a"sH $|$Rat(schlag)\\
7.&226.&4525.&839.&17474.&24.&4&87&30\_50\_3\_4 \textcolor{red}{\textcjheb{dgnl}} LNGD $|$gegen"uber/vor\\
8.&227.&4526.&843.&17478.&28.&4&26&10\_5\_6\_5 \textcolor{red}{\textcjheb{hwhy}} JHWH $|$Jahwe\\
\end{tabular}\medskip \\
Ende des Verses 21.30\\
Verse: 614, Buchstaben: 31, 846, 17481, Totalwerte: 1009, 59000, 1240103\\
\\
Da ist keine Weisheit und keine Einsicht und kein Rat gegen"uber Jahwe.\\
\newpage 
{\bf -- 21.31}\\
\medskip \\
\begin{tabular}{rrrrrrrrp{120mm}}
WV&WK&WB&ABK&ABB&ABV&AnzB&TW&Zahlencode \textcolor{red}{$\boldsymbol{Grundtext}$} Umschrift $|$"Ubersetzung(en)\\
1.&228.&4527.&847.&17482.&1.&3&126&60\_6\_60 \textcolor{red}{\textcjheb{sws}} sWs $|$das Ross/(ein) Ross\\
2.&229.&4528.&850.&17485.&4.&4&116&40\_6\_20\_50 \textcolor{red}{\textcjheb{nkwm}} MWKN $|$(er (=es)) wird ger"ustet\\
3.&230.&4529.&854.&17489.&8.&4&86&30\_10\_6\_40 \textcolor{red}{\textcjheb{mwyl}} LJWM $|$f"ur den Tag\\
4.&231.&4530.&858.&17493.&12.&5&123&40\_30\_8\_40\_5 \textcolor{red}{\textcjheb{hm.hlm}} MLCMH $|$des Streites/(der) Schlacht\\
5.&232.&4531.&863.&17498.&17.&6&62&6\_30\_10\_5\_6\_5 \textcolor{red}{\textcjheb{hwhylw}} WLJHWH $|$aber Jahwes/und bei Jahwe\\
6.&233.&4532.&869.&17504.&23.&6&786&5\_400\_300\_6\_70\_5 \textcolor{red}{\textcjheb{h`w+sth}} HTSWaH $|$(ist) die Rettung\\
\end{tabular}\medskip \\
Ende des Verses 21.31\\
Verse: 615, Buchstaben: 28, 874, 17509, Totalwerte: 1299, 60299, 1241402\\
\\
Das Ro"s wird ger"ustet f"ur den Tag des Streites, aber die Rettung ist Jahwes.\\
\\
{\bf Ende des Kapitels 21}\\
\newpage 
{\bf -- 22.1}\\
\medskip \\
\begin{tabular}{rrrrrrrrp{120mm}}
WV&WK&WB&ABK&ABB&ABV&AnzB&TW&Zahlencode \textcolor{red}{$\boldsymbol{Grundtext}$} Umschrift $|$"Ubersetzung(en)\\
1.&1.&4533.&1.&17510.&1.&4&260&50\_2\_8\_200 \textcolor{red}{\textcjheb{r.hbn}} NBCR $|$vorz"uglicher ist/er (=es) ist wertvoll\\
2.&2.&4534.&5.&17514.&5.&2&340&300\_40 \textcolor{red}{\textcjheb{m+s}} SM $|$ein guter Name/der (gute) Name\\
3.&3.&4535.&7.&17516.&7.&4&610&40\_70\_300\_200 \textcolor{red}{\textcjheb{r+s`m}} MaSR $|$(mehr) als Reichtum\\
4.&4.&4536.&11.&17520.&11.&2&202&200\_2 \textcolor{red}{\textcjheb{br}} RB $|$gro"ser\\
5.&5.&4537.&13.&17522.&13.&4&200&40\_20\_60\_80 \textcolor{red}{\textcjheb{pskm}} MKsP $|$(mehr) als Silber\\
6.&6.&4538.&17.&17526.&17.&5&60&6\_40\_7\_5\_2 \textcolor{red}{\textcjheb{bhzmw}} WMZHB $|$und (mehr als) Gold\\
7.&7.&4539.&22.&17531.&22.&2&58&8\_50 \textcolor{red}{\textcjheb{n.h}} CN $|$Anmut\\
8.&8.&4540.&24.&17533.&24.&3&17&9\_6\_2 \textcolor{red}{\textcjheb{bw.t}} tWB $|$besser/gute\\
\end{tabular}\medskip \\
Ende des Verses 22.1\\
Verse: 616, Buchstaben: 26, 26, 17535, Totalwerte: 1747, 1747, 1243149\\
\\
Ein guter Name ist vorz"uglicher als gro"ser Reichtum, Anmut besser als Silber und Gold.\\
\newpage 
{\bf -- 22.2}\\
\medskip \\
\begin{tabular}{rrrrrrrrp{120mm}}
WV&WK&WB&ABK&ABB&ABV&AnzB&TW&Zahlencode \textcolor{red}{$\boldsymbol{Grundtext}$} Umschrift $|$"Ubersetzung(en)\\
1.&9.&4541.&27.&17536.&1.&4&580&70\_300\_10\_200 \textcolor{red}{\textcjheb{ry+s`}} aSJR $|$Reiche(r)\\
2.&10.&4542.&31.&17540.&5.&3&506&6\_200\_300 \textcolor{red}{\textcjheb{+srw}} WRS $|$und Arme(r)\\
3.&11.&4543.&34.&17543.&8.&5&439&50\_80\_3\_300\_6 \textcolor{red}{\textcjheb{w+sgpn}} NPGSW $|$(sie) begegne(te)n sich\\
4.&12.&4544.&39.&17548.&13.&3&375&70\_300\_5 \textcolor{red}{\textcjheb{h+s`}} aSH $|$gemacht hat/(ein) Machender (war)\\
5.&13.&4545.&42.&17551.&16.&3&90&20\_30\_40 \textcolor{red}{\textcjheb{mlk}} KLM $|$sie alle\\
6.&14.&4546.&45.&17554.&19.&4&26&10\_5\_6\_5 \textcolor{red}{\textcjheb{hwhy}} JHWH $|$Jahwe\\
\end{tabular}\medskip \\
Ende des Verses 22.2\\
Verse: 617, Buchstaben: 22, 48, 17557, Totalwerte: 2016, 3763, 1245165\\
\\
Reiche und Arme begegnen sich: Jahwe hat sie alle gemacht.\\
\newpage 
{\bf -- 22.3}\\
\medskip \\
\begin{tabular}{rrrrrrrrp{120mm}}
WV&WK&WB&ABK&ABB&ABV&AnzB&TW&Zahlencode \textcolor{red}{$\boldsymbol{Grundtext}$} Umschrift $|$"Ubersetzung(en)\\
1.&15.&4547.&49.&17558.&1.&4&316&70\_200\_6\_40 \textcolor{red}{\textcjheb{mwr`}} aRWM $|$der Kluge/(ein) Kluger\\
2.&16.&4548.&53.&17562.&5.&3&206&200\_1\_5 \textcolor{red}{\textcjheb{h'r}} RAH $|$sieht\\
3.&17.&4549.&56.&17565.&8.&3&275&200\_70\_5 \textcolor{red}{\textcjheb{h`r}} RaH $|$(das) Ungl"uck\\
4.&18.&4550.&59.&17568.&11.&5&676&6\_10\_60\_400\_200 \textcolor{red}{\textcjheb{rtsyw}} WJsTR $|$und (er) verbirgt sich\\
5.&19.&4551.&64.&17573.&16.&6&546&6\_80\_400\_10\_10\_40 \textcolor{red}{\textcjheb{myytpw}} WPTJJM $|$die Einf"altigen aber/und Einf"altige\\
6.&20.&4552.&70.&17579.&22.&4&278&70\_2\_200\_6 \textcolor{red}{\textcjheb{wrb`}} aBRW $|$gehen weiter/(sie) gehen vorbei\\
7.&21.&4553.&74.&17583.&26.&6&482&6\_50\_70\_50\_300\_6 \textcolor{red}{\textcjheb{w+sn`nw}} WNaNSW $|$und leiden Strafe/und (sie) werden bestraft\\
\end{tabular}\medskip \\
Ende des Verses 22.3\\
Verse: 618, Buchstaben: 31, 79, 17588, Totalwerte: 2779, 6542, 1247944\\
\\
Der Kluge sieht das Ungl"uck und verbirgt sich; die Einf"altigen aber gehen weiter und leiden Strafe.\\
\newpage 
{\bf -- 22.4}\\
\medskip \\
\begin{tabular}{rrrrrrrrp{120mm}}
WV&WK&WB&ABK&ABB&ABV&AnzB&TW&Zahlencode \textcolor{red}{$\boldsymbol{Grundtext}$} Umschrift $|$"Ubersetzung(en)\\
1.&22.&4554.&80.&17589.&1.&3&172&70\_100\_2 \textcolor{red}{\textcjheb{bq`}} aQB $|$die Folge/Lohn\\
2.&23.&4555.&83.&17592.&4.&4&131&70\_50\_6\_5 \textcolor{red}{\textcjheb{hwn`}} aNWH $|$(der) Demut\\
3.&24.&4556.&87.&17596.&8.&4&611&10\_200\_1\_400 \textcolor{red}{\textcjheb{t'ry}} JRAT $|$(n"amlich) (der) Furcht\\
4.&25.&4557.&91.&17600.&12.&4&26&10\_5\_6\_5 \textcolor{red}{\textcjheb{hwhy}} JHWH $|$(vor) Jahwe(s)\\
5.&26.&4558.&95.&17604.&16.&3&570&70\_300\_200 \textcolor{red}{\textcjheb{r+s`}} aSR $|$(ist) Reichtum\\
6.&27.&4559.&98.&17607.&19.&5&38&6\_20\_2\_6\_4 \textcolor{red}{\textcjheb{dwbkw}} WKBWD $|$und Ehre\\
7.&28.&4560.&103.&17612.&24.&5&74&6\_8\_10\_10\_40 \textcolor{red}{\textcjheb{myy.hw}} WCJJM $|$und Leben\\
\end{tabular}\medskip \\
Ende des Verses 22.4\\
Verse: 619, Buchstaben: 28, 107, 17616, Totalwerte: 1622, 8164, 1249566\\
\\
Die Folge der Demut, der Furcht Jahwes, ist Reichtum und Ehre und Leben.\\
\newpage 
{\bf -- 22.5}\\
\medskip \\
\begin{tabular}{rrrrrrrrp{120mm}}
WV&WK&WB&ABK&ABB&ABV&AnzB&TW&Zahlencode \textcolor{red}{$\boldsymbol{Grundtext}$} Umschrift $|$"Ubersetzung(en)\\
1.&29.&4561.&108.&17617.&1.&4&190&90\_50\_10\_40 \textcolor{red}{\textcjheb{myn.s}} "sNJM $|$Dornen\\
2.&30.&4562.&112.&17621.&5.&4&138&80\_8\_10\_40 \textcolor{red}{\textcjheb{my.hp}} PCJM $|$(und) Schlingen\\
3.&31.&4563.&116.&17625.&9.&4&226&2\_4\_200\_20 \textcolor{red}{\textcjheb{krdb}} BDRK $|$(sind) auf dem Weg\\
4.&32.&4564.&120.&17629.&13.&3&470&70\_100\_300 \textcolor{red}{\textcjheb{+sq`}} aQS $|$des Verkehrten/verkehrten\\
5.&33.&4565.&123.&17632.&16.&4&546&300\_6\_40\_200 \textcolor{red}{\textcjheb{rmw+s}} SWMR $|$wer bewahrt/ein Beachtender\\
6.&34.&4566.&127.&17636.&20.&4&436&50\_80\_300\_6 \textcolor{red}{\textcjheb{w+spn}} NPSW $|$seine Seele\\
7.&35.&4567.&131.&17640.&24.&4&318&10\_200\_8\_100 \textcolor{red}{\textcjheb{q.hry}} JRCQ $|$h"alt sich fern/(d)er bleibt fern\\
8.&36.&4568.&135.&17644.&28.&3&85&40\_5\_40 \textcolor{red}{\textcjheb{mhm}} MHM $|$von ihnen\\
\end{tabular}\medskip \\
Ende des Verses 22.5\\
Verse: 620, Buchstaben: 30, 137, 17646, Totalwerte: 2409, 10573, 1251975\\
\\
Dornen, Schlingen sind auf dem Wege des Verkehrten; wer seine Seele bewahrt, h"alt sich fern von ihnen.\\
\newpage 
{\bf -- 22.6}\\
\medskip \\
\begin{tabular}{rrrrrrrrp{120mm}}
WV&WK&WB&ABK&ABB&ABV&AnzB&TW&Zahlencode \textcolor{red}{$\boldsymbol{Grundtext}$} Umschrift $|$"Ubersetzung(en)\\
1.&37.&4569.&138.&17647.&1.&3&78&8\_50\_20 \textcolor{red}{\textcjheb{kn.h}} CNK $|$erziehe/leite an\\
2.&38.&4570.&141.&17650.&4.&4&350&30\_50\_70\_200 \textcolor{red}{\textcjheb{r`nl}} LNaR $|$den Knaben\\
3.&39.&4571.&145.&17654.&8.&2&100&70\_30 \textcolor{red}{\textcjheb{l`}} aL $|$gem"a"s/auf\\
4.&40.&4572.&147.&17656.&10.&2&90&80\_10 \textcolor{red}{\textcjheb{yp}} PJ $|$/Mund\\
5.&41.&4573.&149.&17658.&12.&4&230&4\_200\_20\_6 \textcolor{red}{\textcjheb{wkrd}} DRKW $|$seinem Weg\\
6.&42.&4574.&153.&17662.&16.&2&43&3\_40 \textcolor{red}{\textcjheb{mg}} GM $|$auch\\
7.&43.&4575.&155.&17664.&18.&2&30&20\_10 \textcolor{red}{\textcjheb{yk}} KJ $|$wenn\\
8.&44.&4576.&157.&17666.&20.&5&177&10\_7\_100\_10\_50 \textcolor{red}{\textcjheb{nyqzy}} JZQJN $|$er wird alt\\
9.&45.&4577.&162.&17671.&25.&2&31&30\_1 \textcolor{red}{\textcjheb{'l}} LA $|$nicht\\
10.&46.&4578.&164.&17673.&27.&4&276&10\_60\_6\_200 \textcolor{red}{\textcjheb{rwsy}} JsWR $|$wird er weichen/er weicht ab\\
11.&47.&4579.&168.&17677.&31.&4&135&40\_40\_50\_5 \textcolor{red}{\textcjheb{hnmm}} MMNH $|$davon\\
\end{tabular}\medskip \\
Ende des Verses 22.6\\
Verse: 621, Buchstaben: 34, 171, 17680, Totalwerte: 1540, 12113, 1253515\\
\\
Erziehe den Knaben seinem Wege gem"a"s; er wird nicht davon weichen, auch wenn er alt wird.\\
\newpage 
{\bf -- 22.7}\\
\medskip \\
\begin{tabular}{rrrrrrrrp{120mm}}
WV&WK&WB&ABK&ABB&ABV&AnzB&TW&Zahlencode \textcolor{red}{$\boldsymbol{Grundtext}$} Umschrift $|$"Ubersetzung(en)\\
1.&48.&4580.&172.&17681.&1.&4&580&70\_300\_10\_200 \textcolor{red}{\textcjheb{ry+s`}} aSJR $|$der Reiche/(ein) Reicher\\
2.&49.&4581.&176.&17685.&5.&5&552&2\_200\_300\_10\_40 \textcolor{red}{\textcjheb{my+srb}} BRSJM $|$"uber Arme\\
3.&50.&4582.&181.&17690.&10.&5&386&10\_40\_300\_6\_30 \textcolor{red}{\textcjheb{lw+smy}} JMSWL $|$(er) herrscht\\
4.&51.&4583.&186.&17695.&15.&4&82&6\_70\_2\_4 \textcolor{red}{\textcjheb{db`w}} WaBD $|$und (ein) Knecht\\
5.&52.&4584.&190.&17699.&19.&3&41&30\_6\_5 \textcolor{red}{\textcjheb{hwl}} LWH $|$ist der Borgende/(ist) Borgender\\
6.&53.&4585.&193.&17702.&22.&4&341&30\_1\_10\_300 \textcolor{red}{\textcjheb{+sy'l}} LAJS $|$des/einem Mann\\
7.&54.&4586.&197.&17706.&26.&4&81&40\_30\_6\_5 \textcolor{red}{\textcjheb{hwlm}} MLWH $|$Leihenden/Leihendem\\
\end{tabular}\medskip \\
Ende des Verses 22.7\\
Verse: 622, Buchstaben: 29, 200, 17709, Totalwerte: 2063, 14176, 1255578\\
\\
Der Reiche herrscht "uber den Armen, und der Borgende ist ein Knecht des Leihenden.\\
\newpage 
{\bf -- 22.8}\\
\medskip \\
\begin{tabular}{rrrrrrrrp{120mm}}
WV&WK&WB&ABK&ABB&ABV&AnzB&TW&Zahlencode \textcolor{red}{$\boldsymbol{Grundtext}$} Umschrift $|$"Ubersetzung(en)\\
1.&55.&4587.&201.&17710.&1.&4&283&7\_6\_200\_70 \textcolor{red}{\textcjheb{`rwz}} ZWRa $|$wer s"at/(ein) S"aender\\
2.&56.&4588.&205.&17714.&5.&4&111&70\_6\_30\_5 \textcolor{red}{\textcjheb{hlw`}} aWLH $|$Unrecht\\
3.&57.&4589.&209.&17718.&9.&5&406&10\_100\_90\_6\_200 \textcolor{red}{\textcjheb{rw.sqy}} JQ"sWR $|$wird ernten/(d)er erntet\\
4.&58.&4590.&214.&17723.&14.&3&57&1\_6\_50 \textcolor{red}{\textcjheb{nw'}} AWN $|$Unheil\\
5.&59.&4591.&217.&17726.&17.&4&317&6\_300\_2\_9 \textcolor{red}{\textcjheb{.tb+sw}} WSBt $|$und (die) Rute/und mit dem Stock\\
6.&60.&4592.&221.&17730.&21.&5&678&70\_2\_200\_400\_6 \textcolor{red}{\textcjheb{wtrb`}} aBRTW $|$seines Zornes/seiner "Uberheblichkeit\\
7.&61.&4593.&226.&17735.&26.&4&65&10\_20\_30\_5 \textcolor{red}{\textcjheb{hlky}} JKLH $|$wird ein Ende nehmen/er (=es) ist zu Ende\\
\end{tabular}\medskip \\
Ende des Verses 22.8\\
Verse: 623, Buchstaben: 29, 229, 17738, Totalwerte: 1917, 16093, 1257495\\
\\
Wer Unrecht s"at, wird Unheil ernten, und seines Zornes Rute wird ein Ende nehmen.\\
\newpage 
{\bf -- 22.9}\\
\medskip \\
\begin{tabular}{rrrrrrrrp{120mm}}
WV&WK&WB&ABK&ABB&ABV&AnzB&TW&Zahlencode \textcolor{red}{$\boldsymbol{Grundtext}$} Umschrift $|$"Ubersetzung(en)\\
1.&62.&4594.&230.&17739.&1.&3&17&9\_6\_2 \textcolor{red}{\textcjheb{bw.t}} tWB $|$wer g"utigen/wer guten\\
2.&63.&4595.&233.&17742.&4.&3&130&70\_10\_50 \textcolor{red}{\textcjheb{ny`}} aJN $|$Auges (ist)\\
3.&64.&4596.&236.&17745.&7.&3&12&5\_6\_1 \textcolor{red}{\textcjheb{'wh}} HWA $|$(d)er\\
4.&65.&4597.&239.&17748.&10.&4&232&10\_2\_200\_20 \textcolor{red}{\textcjheb{krby}} JBRK $|$wird gesegnet werden/(er) wird gesegnet\\
5.&66.&4598.&243.&17752.&14.&2&30&20\_10 \textcolor{red}{\textcjheb{yk}} KJ $|$denn/weil\\
6.&67.&4599.&245.&17754.&16.&3&500&50\_400\_50 \textcolor{red}{\textcjheb{ntn}} NTN $|$er gibt\\
7.&68.&4600.&248.&17757.&19.&5&124&40\_30\_8\_40\_6 \textcolor{red}{\textcjheb{wm.hlm}} MLCMW $|$von seinem Brot\\
8.&69.&4601.&253.&17762.&24.&3&64&30\_4\_30 \textcolor{red}{\textcjheb{ldl}} LDL $|$dem Armen/dem Geringen\\
\end{tabular}\medskip \\
Ende des Verses 22.9\\
Verse: 624, Buchstaben: 26, 255, 17764, Totalwerte: 1109, 17202, 1258604\\
\\
Wer g"utigen Auges ist, der wird gesegnet werden; denn er gibt von seinem Brote dem Armen.\\
\newpage 
{\bf -- 22.10}\\
\medskip \\
\begin{tabular}{rrrrrrrrp{120mm}}
WV&WK&WB&ABK&ABB&ABV&AnzB&TW&Zahlencode \textcolor{red}{$\boldsymbol{Grundtext}$} Umschrift $|$"Ubersetzung(en)\\
1.&70.&4602.&256.&17765.&1.&3&503&3\_200\_300 \textcolor{red}{\textcjheb{+srg}} GRS $|$treibe fort/vertreibe\\
2.&71.&4603.&259.&17768.&4.&2&120&30\_90 \textcolor{red}{\textcjheb{.sl}} L"s $|$den Sp"otter\\
3.&72.&4604.&261.&17770.&6.&4&107&6\_10\_90\_1 \textcolor{red}{\textcjheb{'.syw}} WJ"sA $|$so geht hinaus/und er (=es) schwindet\\
4.&73.&4605.&265.&17774.&10.&4&100&40\_4\_6\_50 \textcolor{red}{\textcjheb{nwdm}} MDWN $|$der Zank\\
5.&74.&4606.&269.&17778.&14.&5&718&6\_10\_300\_2\_400 \textcolor{red}{\textcjheb{tb+syw}} WJSBT $|$und (es) h"oren auf/und er (=es) h"ort auf\\
6.&75.&4607.&274.&17783.&19.&3&64&4\_10\_50 \textcolor{red}{\textcjheb{nyd}} DJN $|$Streit\\
7.&76.&4608.&277.&17786.&22.&5&192&6\_100\_30\_6\_50 \textcolor{red}{\textcjheb{nwlqw}} WQLWN $|$und Schande/und Schimpf\\
\end{tabular}\medskip \\
Ende des Verses 22.10\\
Verse: 625, Buchstaben: 26, 281, 17790, Totalwerte: 1804, 19006, 1260408\\
\\
Treibe den Sp"otter fort, so geht der Zank hinaus, und Streit und Schande h"oren auf.\\
\newpage 
{\bf -- 22.11}\\
\medskip \\
\begin{tabular}{rrrrrrrrp{120mm}}
WV&WK&WB&ABK&ABB&ABV&AnzB&TW&Zahlencode \textcolor{red}{$\boldsymbol{Grundtext}$} Umschrift $|$"Ubersetzung(en)\\
1.&77.&4609.&282.&17791.&1.&3&8&1\_5\_2 \textcolor{red}{\textcjheb{bh'}} AHB $|$wer liebt/(ein) Liebender\\
2.&78.&4610.&285.&17794.&4.&4&220&9\_5\_6\_200 \textcolor{red}{\textcjheb{rwh.t}} tHWR $|$Reinheit/reines\\
3.&79.&4611.&289.&17798.&8.&2&32&30\_2 \textcolor{red}{\textcjheb{bl}} LB $|$des Herzens/Herz (=Verstand)\\
4.&80.&4612.&291.&17800.&10.&2&58&8\_50 \textcolor{red}{\textcjheb{n.h}} CN $|$wessen Anmut/(bei der) Anmut\\
5.&81.&4613.&293.&17802.&12.&5&796&300\_80\_400\_10\_6 \textcolor{red}{\textcjheb{wytp+s}} SPTJW $|$(seiner) Lippen (sind)\\
6.&82.&4614.&298.&17807.&17.&4&281&200\_70\_5\_6 \textcolor{red}{\textcjheb{wh`r}} RaHW $|$dessen Freund/sein Freund\\
7.&83.&4615.&302.&17811.&21.&3&90&40\_30\_20 \textcolor{red}{\textcjheb{klm}} MLK $|$(ist der) K"onig\\
\end{tabular}\medskip \\
Ende des Verses 22.11\\
Verse: 626, Buchstaben: 23, 304, 17813, Totalwerte: 1485, 20491, 1261893\\
\\
Wer Reinheit des Herzens liebt, wessen Lippen Anmut sind, dessen Freund ist der K"onig.\\
\newpage 
{\bf -- 22.12}\\
\medskip \\
\begin{tabular}{rrrrrrrrp{120mm}}
WV&WK&WB&ABK&ABB&ABV&AnzB&TW&Zahlencode \textcolor{red}{$\boldsymbol{Grundtext}$} Umschrift $|$"Ubersetzung(en)\\
1.&84.&4616.&305.&17814.&1.&4&140&70\_10\_50\_10 \textcolor{red}{\textcjheb{yny`}} aJNJ $|$die Augen\\
2.&85.&4617.&309.&17818.&5.&4&26&10\_5\_6\_5 \textcolor{red}{\textcjheb{hwhy}} JHWH $|$(von) Jahwe(s)\\
3.&86.&4618.&313.&17822.&9.&4&346&50\_90\_200\_6 \textcolor{red}{\textcjheb{wr.sn}} N"sRW $|$(sie) beh"ute(te)n\\
4.&87.&4619.&317.&17826.&13.&3&474&4\_70\_400 \textcolor{red}{\textcjheb{t`d}} DaT $|$(die) Erkenntnis\\
5.&88.&4620.&320.&17829.&16.&5&186&6\_10\_60\_30\_80 \textcolor{red}{\textcjheb{plsyw}} WJsLP $|$und er vereitelt\\
6.&89.&4621.&325.&17834.&21.&4&216&4\_2\_200\_10 \textcolor{red}{\textcjheb{yrbd}} DBRJ $|$die Worte\\
7.&90.&4622.&329.&17838.&25.&3&9&2\_3\_4 \textcolor{red}{\textcjheb{dgb}} BGD $|$des Treulosen\\
\end{tabular}\medskip \\
Ende des Verses 22.12\\
Verse: 627, Buchstaben: 27, 331, 17840, Totalwerte: 1397, 21888, 1263290\\
\\
Die Augen Jahwes beh"uten die Erkenntnis, und er vereitelt die Worte des Treulosen.\\
\newpage 
{\bf -- 22.13}\\
\medskip \\
\begin{tabular}{rrrrrrrrp{120mm}}
WV&WK&WB&ABK&ABB&ABV&AnzB&TW&Zahlencode \textcolor{red}{$\boldsymbol{Grundtext}$} Umschrift $|$"Ubersetzung(en)\\
1.&91.&4623.&332.&17841.&1.&3&241&1\_40\_200 \textcolor{red}{\textcjheb{rm'}} AMR $|$(er (=es)) spricht\\
2.&92.&4624.&335.&17844.&4.&3&190&70\_90\_30 \textcolor{red}{\textcjheb{l.s`}} a"sL $|$der Faule/(der) Faulpelz\\
3.&93.&4625.&338.&17847.&7.&3&211&1\_200\_10 \textcolor{red}{\textcjheb{yr'}} ARJ $|$(ein) L"owe\\
4.&94.&4626.&341.&17850.&10.&4&106&2\_8\_6\_90 \textcolor{red}{\textcjheb{.sw.hb}} BCW"s $|$ist drau"sen/(ist) auf der Gasse\\
5.&95.&4627.&345.&17854.&14.&4&428&2\_400\_6\_20 \textcolor{red}{\textcjheb{kwtb}} BTWK $|$mitten auf/inmitten\\
6.&96.&4628.&349.&17858.&18.&5&616&200\_8\_2\_6\_400 \textcolor{red}{\textcjheb{twb.hr}} RCBWT $|$den Stra"sen/der Pl"atze\\
7.&97.&4629.&354.&17863.&23.&4&299&1\_200\_90\_8 \textcolor{red}{\textcjheb{.h.sr'}} AR"sC $|$ich m"ochte ermordet werden/ich werde get"otet\\
\end{tabular}\medskip \\
Ende des Verses 22.13\\
Verse: 628, Buchstaben: 26, 357, 17866, Totalwerte: 2091, 23979, 1265381\\
\\
Der Faule spricht: Ein L"owe ist drau"sen; ich m"ochte ermordet werden mitten auf den Stra"sen!\\
\newpage 
{\bf -- 22.14}\\
\medskip \\
\begin{tabular}{rrrrrrrrp{120mm}}
WV&WK&WB&ABK&ABB&ABV&AnzB&TW&Zahlencode \textcolor{red}{$\boldsymbol{Grundtext}$} Umschrift $|$"Ubersetzung(en)\\
1.&98.&4630.&358.&17867.&1.&4&319&300\_6\_8\_5 \textcolor{red}{\textcjheb{h.hw+s}} SWCH $|$(eine) Grube\\
2.&99.&4631.&362.&17871.&5.&4&215&70\_40\_100\_5 \textcolor{red}{\textcjheb{hqm`}} aMQH $|$tiefe\\
3.&100.&4632.&366.&17875.&9.&2&90&80\_10 \textcolor{red}{\textcjheb{yp}} PJ $|$(ist) der Mund\\
4.&101.&4633.&368.&17877.&11.&4&613&7\_200\_6\_400 \textcolor{red}{\textcjheb{twrz}} ZRWT $|$fremder Frauen/von fremden (Frauen)\\
5.&102.&4634.&372.&17881.&15.&4&123&7\_70\_6\_40 \textcolor{red}{\textcjheb{mw`z}} ZaWM $|$wem z"urnt/ein Verfluchter\\
6.&103.&4635.&376.&17885.&19.&4&26&10\_5\_6\_5 \textcolor{red}{\textcjheb{hwhy}} JHWH $|$Jahwe(s)\\
7.&104.&4636.&380.&17889.&23.&4&126&10\_80\_6\_30 \textcolor{red}{\textcjheb{lwpy}} JPWL $|$((d)er) f"allt\\
8.&105.&4637.&384.&17893.&27.&2&340&300\_40 \textcolor{red}{\textcjheb{m+s}} SM $|$hinein\\
\end{tabular}\medskip \\
Ende des Verses 22.14\\
Verse: 629, Buchstaben: 28, 385, 17894, Totalwerte: 1852, 25831, 1267233\\
\\
Der Mund fremder Weiber ist eine tiefe Grube; wem Jahwe z"urnt, der f"allt hinein.\\
\newpage 
{\bf -- 22.15}\\
\medskip \\
\begin{tabular}{rrrrrrrrp{120mm}}
WV&WK&WB&ABK&ABB&ABV&AnzB&TW&Zahlencode \textcolor{red}{$\boldsymbol{Grundtext}$} Umschrift $|$"Ubersetzung(en)\\
1.&106.&4638.&386.&17895.&1.&4&437&1\_6\_30\_400 \textcolor{red}{\textcjheb{tlw'}} AWLT $|$Narrheit/(ist) Torheit\\
2.&107.&4639.&390.&17899.&5.&5&611&100\_300\_6\_200\_5 \textcolor{red}{\textcjheb{hrw+sq}} QSWRH $|$ist gekettet/gebunden\\
3.&108.&4640.&395.&17904.&10.&3&34&2\_30\_2 \textcolor{red}{\textcjheb{blb}} BLB $|$an das Herz/am Herzen\\
4.&109.&4641.&398.&17907.&13.&3&320&50\_70\_200 \textcolor{red}{\textcjheb{r`n}} NaR $|$des Knaben/(eines) Knaben\\
5.&110.&4642.&401.&17910.&16.&3&311&300\_2\_9 \textcolor{red}{\textcjheb{.tb+s}} SBt $|$die Rute/(eine) Rute\\
6.&111.&4643.&404.&17913.&19.&4&306&40\_6\_60\_200 \textcolor{red}{\textcjheb{rswm}} MWsR $|$(der) Zucht\\
7.&112.&4644.&408.&17917.&23.&7&383&10\_200\_8\_10\_100\_50\_5 \textcolor{red}{\textcjheb{hnqy.hry}} JRCJQNH $|$wird entfernen sie/er (=sie) wird fernhalten sie\\
8.&113.&4645.&415.&17924.&30.&4&136&40\_40\_50\_6 \textcolor{red}{\textcjheb{wnmm}} MMNW $|$davon/von ihm\\
\end{tabular}\medskip \\
Ende des Verses 22.15\\
Verse: 630, Buchstaben: 33, 418, 17927, Totalwerte: 2538, 28369, 1269771\\
\\
Narrheit ist gekettet an das Herz des Knaben; die Rute der Zucht wird sie davon entfernen.\\
\newpage 
{\bf -- 22.16}\\
\medskip \\
\begin{tabular}{rrrrrrrrp{120mm}}
WV&WK&WB&ABK&ABB&ABV&AnzB&TW&Zahlencode \textcolor{red}{$\boldsymbol{Grundtext}$} Umschrift $|$"Ubersetzung(en)\\
1.&114.&4646.&419.&17928.&1.&3&470&70\_300\_100 \textcolor{red}{\textcjheb{q+s`}} aSQ $|$wer bedr"uckt/(ein) Bedr"uckender\\
2.&115.&4647.&422.&17931.&4.&2&34&4\_30 \textcolor{red}{\textcjheb{ld}} DL $|$den Armen/(einen) Geringen\\
3.&116.&4648.&424.&17933.&6.&6&643&30\_5\_200\_2\_6\_400 \textcolor{red}{\textcjheb{twbrhl}} LHRBWT $|$zur Bereicherung ist es/verschafft mehr\\
4.&117.&4649.&430.&17939.&12.&2&36&30\_6 \textcolor{red}{\textcjheb{wl}} LW $|$ihm/sich\\
5.&118.&4650.&432.&17941.&14.&3&500&50\_400\_50 \textcolor{red}{\textcjheb{ntn}} NTN $|$wer gibt\\
6.&119.&4651.&435.&17944.&17.&5&610&30\_70\_300\_10\_200 \textcolor{red}{\textcjheb{ry+s`l}} LaSJR $|$(dem) Reichen\\
7.&120.&4652.&440.&17949.&22.&2&21&1\_20 \textcolor{red}{\textcjheb{k'}} AK $|$nur\\
8.&121.&4653.&442.&17951.&24.&6&344&30\_40\_8\_60\_6\_200 \textcolor{red}{\textcjheb{rws.hml}} LMCsWR $|$(ist es) (zu) (dem) Mangel\\
\end{tabular}\medskip \\
Ende des Verses 22.16\\
Verse: 631, Buchstaben: 29, 447, 17956, Totalwerte: 2658, 31027, 1272429\\
\\
Wer den Armen bedr"uckt, ihm zur Bereicherung ist es; wer dem Reichen gibt, es ist nur zum Mangel.\\
\newpage 
{\bf -- 22.17}\\
\medskip \\
\begin{tabular}{rrrrrrrrp{120mm}}
WV&WK&WB&ABK&ABB&ABV&AnzB&TW&Zahlencode \textcolor{red}{$\boldsymbol{Grundtext}$} Umschrift $|$"Ubersetzung(en)\\
1.&122.&4654.&448.&17957.&1.&2&14&5\_9 \textcolor{red}{\textcjheb{.th}} Ht $|$neige\\
2.&123.&4655.&450.&17959.&3.&4&78&1\_7\_50\_20 \textcolor{red}{\textcjheb{knz'}} AZNK $|$dein Ohr\\
3.&124.&4656.&454.&17963.&7.&4&416&6\_300\_40\_70 \textcolor{red}{\textcjheb{`m+sw}} WSMa $|$und h"ore\\
4.&125.&4657.&458.&17967.&11.&4&216&4\_2\_200\_10 \textcolor{red}{\textcjheb{yrbd}} DBRJ $|$die Worte\\
5.&126.&4658.&462.&17971.&15.&5&118&8\_20\_40\_10\_40 \textcolor{red}{\textcjheb{mymk.h}} CKMJM $|$der Weisen/von Weisen\\
6.&127.&4659.&467.&17976.&20.&4&58&6\_30\_2\_20 \textcolor{red}{\textcjheb{kblw}} WLBK $|$und dein Herz\\
7.&128.&4660.&471.&17980.&24.&4&1110&400\_300\_10\_400 \textcolor{red}{\textcjheb{ty+st}} TSJT $|$(du sollst) richte(n)\\
8.&129.&4661.&475.&17984.&28.&5&514&30\_4\_70\_400\_10 \textcolor{red}{\textcjheb{yt`dl}} LDaTJ $|$auf mein Wissen/auf meine Belehrung\\
\end{tabular}\medskip \\
Ende des Verses 22.17\\
Verse: 632, Buchstaben: 32, 479, 17988, Totalwerte: 2524, 33551, 1274953\\
\\
Neige dein Ohr und h"ore die Worte der Weisen, und richte dein Herz auf mein Wissen!\\
\newpage 
{\bf -- 22.18}\\
\medskip \\
\begin{tabular}{rrrrrrrrp{120mm}}
WV&WK&WB&ABK&ABB&ABV&AnzB&TW&Zahlencode \textcolor{red}{$\boldsymbol{Grundtext}$} Umschrift $|$"Ubersetzung(en)\\
1.&130.&4662.&480.&17989.&1.&2&30&20\_10 \textcolor{red}{\textcjheb{yk}} KJ $|$denn\\
2.&131.&4663.&482.&17991.&3.&4&170&50\_70\_10\_40 \textcolor{red}{\textcjheb{my`n}} NaJM $|$lieblich ist es/(es ist) angenehm\\
3.&132.&4664.&486.&17995.&7.&2&30&20\_10 \textcolor{red}{\textcjheb{yk}} KJ $|$wenn\\
4.&133.&4665.&488.&17997.&9.&5&980&400\_300\_40\_200\_40 \textcolor{red}{\textcjheb{mrm+st}} TSMRM $|$du bewahrst sie\\
5.&134.&4666.&493.&18002.&14.&5&83&2\_2\_9\_50\_20 \textcolor{red}{\textcjheb{kn.tbb}} BBtNK $|$in deinem Inneren/in deinem Leib\\
6.&135.&4667.&498.&18007.&19.&4&86&10\_20\_50\_6 \textcolor{red}{\textcjheb{wnky}} JKNW $|$m"ogen sie haben Bestand/sie werden haften\\
7.&136.&4668.&502.&18011.&23.&4&28&10\_8\_4\_6 \textcolor{red}{\textcjheb{wd.hy}} JCDW $|$allzumal/zusammen\\
8.&137.&4669.&506.&18015.&27.&2&100&70\_30 \textcolor{red}{\textcjheb{l`}} aL $|$auf/an\\
9.&138.&4670.&508.&18017.&29.&5&810&300\_80\_400\_10\_20 \textcolor{red}{\textcjheb{kytp+s}} SPTJK $|$deinen Lippen\\
\end{tabular}\medskip \\
Ende des Verses 22.18\\
Verse: 633, Buchstaben: 33, 512, 18021, Totalwerte: 2317, 35868, 1277270\\
\\
Denn lieblich ist es, wenn du sie in deinem Innern bewahrst; m"ochten sie allzumal auf deinen Lippen Bestand haben!\\
\newpage 
{\bf -- 22.19}\\
\medskip \\
\begin{tabular}{rrrrrrrrp{120mm}}
WV&WK&WB&ABK&ABB&ABV&AnzB&TW&Zahlencode \textcolor{red}{$\boldsymbol{Grundtext}$} Umschrift $|$"Ubersetzung(en)\\
1.&139.&4671.&513.&18022.&1.&5&451&30\_5\_10\_6\_400 \textcolor{red}{\textcjheb{twyhl}} LHJWT $|$damit sei/dass sei\\
2.&140.&4672.&518.&18027.&6.&5&28&2\_10\_5\_6\_5 \textcolor{red}{\textcjheb{hwhyb}} BJHWH $|$auf Jahwe/bei Jahwe\\
3.&141.&4673.&523.&18032.&11.&5&79&40\_2\_9\_8\_20 \textcolor{red}{\textcjheb{k.h.tbm}} MBtCK $|$dein Vertrauen\\
4.&142.&4674.&528.&18037.&16.&7&515&5\_6\_4\_70\_400\_10\_20 \textcolor{red}{\textcjheb{kyt`dwh}} HWDaTJK $|$belehrt habe ich dich/ich belehre dich\\
5.&143.&4675.&535.&18044.&23.&4&61&5\_10\_6\_40 \textcolor{red}{\textcjheb{mwyh}} HJWM $|$heute\\
6.&144.&4676.&539.&18048.&27.&2&81&1\_80 \textcolor{red}{\textcjheb{p'}} AP $|$ja/auch\\
7.&145.&4677.&541.&18050.&29.&3&406&1\_400\_5 \textcolor{red}{\textcjheb{ht'}} ATH $|$dich\\
\end{tabular}\medskip \\
Ende des Verses 22.19\\
Verse: 634, Buchstaben: 31, 543, 18052, Totalwerte: 1621, 37489, 1278891\\
\\
Damit dein Vertrauen auf Jahwe sei, habe ich heute dich, ja dich, belehrt.\\
\newpage 
{\bf -- 22.20}\\
\medskip \\
\begin{tabular}{rrrrrrrrp{120mm}}
WV&WK&WB&ABK&ABB&ABV&AnzB&TW&Zahlencode \textcolor{red}{$\boldsymbol{Grundtext}$} Umschrift $|$"Ubersetzung(en)\\
1.&146.&4678.&544.&18053.&1.&3&36&5\_30\_1 \textcolor{red}{\textcjheb{'lh}} HLA $|$(etwa) nicht\\
2.&147.&4679.&547.&18056.&4.&5&832&20\_400\_2\_400\_10 \textcolor{red}{\textcjheb{ytbtk}} KTBTJ $|$ich habe (auf)geschrieben\\
3.&148.&4680.&552.&18061.&9.&2&50&30\_20 \textcolor{red}{\textcjheb{kl}} LK $|$dir/f"ur dich\\
4.&149.&4681.&554.&18063.&11.&5&676&300\_30\_300\_6\_40 \textcolor{red}{\textcjheb{mw+sl+s}} SLSWM $|$Vortreffliches/Kernspr"uche\\
5.&150.&4682.&559.&18068.&16.&7&614&2\_40\_6\_70\_90\_6\_400 \textcolor{red}{\textcjheb{tw.s`wmb}} BMWa"sWT $|$an Ratschl"agen/mit Ratschl"agen\\
6.&151.&4683.&566.&18075.&23.&4&480&6\_4\_70\_400 \textcolor{red}{\textcjheb{t`dw}} WDaT $|$und Erkenntnis\\
\end{tabular}\medskip \\
Ende des Verses 22.20\\
Verse: 635, Buchstaben: 26, 569, 18078, Totalwerte: 2688, 40177, 1281579\\
\\
Habe ich dir nicht Vortreffliches aufgeschrieben an Ratschl"agen und Erkenntnis,\\
\newpage 
{\bf -- 22.21}\\
\medskip \\
\begin{tabular}{rrrrrrrrp{120mm}}
WV&WK&WB&ABK&ABB&ABV&AnzB&TW&Zahlencode \textcolor{red}{$\boldsymbol{Grundtext}$} Umschrift $|$"Ubersetzung(en)\\
1.&152.&4684.&570.&18079.&1.&7&145&30\_5\_6\_4\_10\_70\_20 \textcolor{red}{\textcjheb{k`ydwhl}} LHWDJaK $|$um dir kundzutun\\
2.&153.&4685.&577.&18086.&8.&3&409&100\_300\_9 \textcolor{red}{\textcjheb{.t+sq}} QSt $|$die Richtschnur/(der) Wahrheit\\
3.&154.&4686.&580.&18089.&11.&4&251&1\_40\_200\_10 \textcolor{red}{\textcjheb{yrm'}} AMRJ $|$(der) Worte\\
4.&155.&4687.&584.&18093.&15.&3&441&1\_40\_400 \textcolor{red}{\textcjheb{tm'}} AMT $|$(der) Wahrheit\\
5.&156.&4688.&587.&18096.&18.&5&347&30\_5\_300\_10\_2 \textcolor{red}{\textcjheb{by+shl}} LHSJB $|$damit du denen zur"uckbringst/zu antworten\\
6.&157.&4689.&592.&18101.&23.&5&291&1\_40\_200\_10\_40 \textcolor{red}{\textcjheb{myrm'}} AMRJM $|$Worte\\
7.&158.&4690.&597.&18106.&28.&3&441&1\_40\_400 \textcolor{red}{\textcjheb{tm'}} AMT $|$welche Wahrheit sind/(der) Wahrheit\\
8.&159.&4691.&600.&18109.&31.&6&398&30\_300\_30\_8\_10\_20 \textcolor{red}{\textcjheb{ky.hl+sl}} LSLCJK $|$denen die dich senden/denen die dich gesandt\\
\end{tabular}\medskip \\
Ende des Verses 22.21\\
Verse: 636, Buchstaben: 36, 605, 18114, Totalwerte: 2723, 42900, 1284302\\
\\
um dir kundzutun die Richtschnur der Worte der Wahrheit: damit du denen, die dich senden, Worte zur"uckbringest, welche Wahrheit sind?\\
\newpage 
{\bf -- 22.22}\\
\medskip \\
\begin{tabular}{rrrrrrrrp{120mm}}
WV&WK&WB&ABK&ABB&ABV&AnzB&TW&Zahlencode \textcolor{red}{$\boldsymbol{Grundtext}$} Umschrift $|$"Ubersetzung(en)\\
1.&160.&4692.&606.&18115.&1.&2&31&1\_30 \textcolor{red}{\textcjheb{l'}} AL $|$nicht\\
2.&161.&4693.&608.&18117.&3.&4&440&400\_3\_7\_30 \textcolor{red}{\textcjheb{lzgt}} TGZL $|$beraube/sollst du berauben\\
3.&162.&4694.&612.&18121.&7.&2&34&4\_30 \textcolor{red}{\textcjheb{ld}} DL $|$den Armen/(einen) Geringen\\
4.&163.&4695.&614.&18123.&9.&2&30&20\_10 \textcolor{red}{\textcjheb{yk}} KJ $|$weil\\
5.&164.&4696.&616.&18125.&11.&2&34&4\_30 \textcolor{red}{\textcjheb{ld}} DL $|$arm/gering\\
6.&165.&4697.&618.&18127.&13.&3&12&5\_6\_1 \textcolor{red}{\textcjheb{'wh}} HWA $|$(ist) er\\
7.&166.&4698.&621.&18130.&16.&3&37&6\_1\_30 \textcolor{red}{\textcjheb{l'w}} WAL $|$und nicht\\
8.&167.&4699.&624.&18133.&19.&4&425&400\_4\_20\_1 \textcolor{red}{\textcjheb{'kdt}} TDKA $|$zertritt/du sollst unterdr"ucken\\
9.&168.&4700.&628.&18137.&23.&3&130&70\_50\_10 \textcolor{red}{\textcjheb{yn`}} aNJ $|$den Elenden/(einen) Armen\\
10.&169.&4701.&631.&18140.&26.&4&572&2\_300\_70\_200 \textcolor{red}{\textcjheb{r`+sb}} BSaR $|$im Tor\\
\end{tabular}\medskip \\
Ende des Verses 22.22\\
Verse: 637, Buchstaben: 29, 634, 18143, Totalwerte: 1745, 44645, 1286047\\
\\
Beraube nicht den Armen, weil er arm ist, und zertritt nicht den Elenden im Tore.\\
\newpage 
{\bf -- 22.23}\\
\medskip \\
\begin{tabular}{rrrrrrrrp{120mm}}
WV&WK&WB&ABK&ABB&ABV&AnzB&TW&Zahlencode \textcolor{red}{$\boldsymbol{Grundtext}$} Umschrift $|$"Ubersetzung(en)\\
1.&170.&4702.&635.&18144.&1.&2&30&20\_10 \textcolor{red}{\textcjheb{yk}} KJ $|$denn\\
2.&171.&4703.&637.&18146.&3.&4&26&10\_5\_6\_5 \textcolor{red}{\textcjheb{hwhy}} JHWH $|$Jahwe\\
3.&172.&4704.&641.&18150.&7.&4&222&10\_200\_10\_2 \textcolor{red}{\textcjheb{byry}} JRJB $|$wird f"uhren/er f"uhrt\\
4.&173.&4705.&645.&18154.&11.&4&252&200\_10\_2\_40 \textcolor{red}{\textcjheb{mbyr}} RJBM $|$ihre Rechtssache/ihren Streitfall\\
5.&174.&4706.&649.&18158.&15.&4&178&6\_100\_2\_70 \textcolor{red}{\textcjheb{`bqw}} WQBa $|$und berauben/und er beraubt\\
6.&175.&4707.&653.&18162.&19.&2&401&1\_400 \textcolor{red}{\textcjheb{t'}} AT $|$**\\
7.&176.&4708.&655.&18164.&21.&6&227&100\_2\_70\_10\_5\_40 \textcolor{red}{\textcjheb{mhy`bq}} QBaJHM $|$ihre Berauber/ihre R"auber\\
8.&177.&4709.&661.&18170.&27.&3&430&50\_80\_300 \textcolor{red}{\textcjheb{+spn}} NPS $|$des Lebens\\
\end{tabular}\medskip \\
Ende des Verses 22.23\\
Verse: 638, Buchstaben: 29, 663, 18172, Totalwerte: 1766, 46411, 1287813\\
\\
Denn Jahwe wird ihre Rechtssache f"uhren, und ihre Berauber des Lebens berauben.\\
\newpage 
{\bf -- 22.24}\\
\medskip \\
\begin{tabular}{rrrrrrrrp{120mm}}
WV&WK&WB&ABK&ABB&ABV&AnzB&TW&Zahlencode \textcolor{red}{$\boldsymbol{Grundtext}$} Umschrift $|$"Ubersetzung(en)\\
1.&178.&4710.&664.&18173.&1.&2&31&1\_30 \textcolor{red}{\textcjheb{l'}} AL $|$nicht\\
2.&179.&4711.&666.&18175.&3.&4&1070&400\_400\_200\_70 \textcolor{red}{\textcjheb{`rtt}} TTRa $|$geselle dich/du sollst dich einlassen\\
3.&180.&4712.&670.&18179.&7.&2&401&1\_400 \textcolor{red}{\textcjheb{t'}} AT $|$zu/mit//**\\
4.&181.&4713.&672.&18181.&9.&3&102&2\_70\_30 \textcolor{red}{\textcjheb{l`b}} BaL $|$einem/(einem) Mann\\
5.&182.&4714.&675.&18184.&12.&2&81&1\_80 \textcolor{red}{\textcjheb{p'}} AP $|$Zornigen/des Zorns\\
6.&183.&4715.&677.&18186.&14.&3&407&6\_1\_400 \textcolor{red}{\textcjheb{t'w}} WAT $|$und mit///und **\\
7.&184.&4716.&680.&18189.&17.&3&311&1\_10\_300 \textcolor{red}{\textcjheb{+sy'}} AJS $|$(einem) Mann\\
8.&185.&4717.&683.&18192.&20.&4&454&8\_40\_6\_400 \textcolor{red}{\textcjheb{twm.h}} CMWT $|$hitzigen/(von) Erregungen\\
9.&186.&4718.&687.&18196.&24.&2&31&30\_1 \textcolor{red}{\textcjheb{'l}} LA $|$nicht\\
10.&187.&4719.&689.&18198.&26.&4&409&400\_2\_6\_1 \textcolor{red}{\textcjheb{'wbt}} TBWA $|$geh um/du sollst umgehen\\
\end{tabular}\medskip \\
Ende des Verses 22.24\\
Verse: 639, Buchstaben: 29, 692, 18201, Totalwerte: 3297, 49708, 1291110\\
\\
Geselle dich nicht zu einem Zornigen, und geh nicht um mit einem hitzigen Manne,\\
\newpage 
{\bf -- 22.25}\\
\medskip \\
\begin{tabular}{rrrrrrrrp{120mm}}
WV&WK&WB&ABK&ABB&ABV&AnzB&TW&Zahlencode \textcolor{red}{$\boldsymbol{Grundtext}$} Umschrift $|$"Ubersetzung(en)\\
1.&188.&4720.&693.&18202.&1.&2&130&80\_50 \textcolor{red}{\textcjheb{np}} PN $|$damit nicht/dass nicht\\
2.&189.&4721.&695.&18204.&3.&4&511&400\_1\_30\_80 \textcolor{red}{\textcjheb{pl't}} TALP $|$du lernst/du wirst vertraut\\
3.&190.&4722.&699.&18208.&7.&5&615&1\_200\_8\_400\_6 \textcolor{red}{\textcjheb{wt.hr'}} ARCTW $|$(mit) seine(n) Pfade(n)\\
4.&191.&4723.&704.&18213.&12.&5&544&6\_30\_100\_8\_400 \textcolor{red}{\textcjheb{t.hqlw}} WLQCT $|$und davontragest/und du schaffst\\
5.&192.&4724.&709.&18218.&17.&4&446&40\_6\_100\_300 \textcolor{red}{\textcjheb{+sqwm}} MWQS $|$einen Fallstrick/(eine) Falle\\
6.&193.&4725.&713.&18222.&21.&5&480&30\_50\_80\_300\_20 \textcolor{red}{\textcjheb{k+spnl}} LNPSK $|$f"ur deine Seele\\
\end{tabular}\medskip \\
Ende des Verses 22.25\\
Verse: 640, Buchstaben: 25, 717, 18226, Totalwerte: 2726, 52434, 1293836\\
\\
damit du seine Pfade nicht lernest und einen Fallstrick davontragest f"ur deine Seele.\\
\newpage 
{\bf -- 22.26}\\
\medskip \\
\begin{tabular}{rrrrrrrrp{120mm}}
WV&WK&WB&ABK&ABB&ABV&AnzB&TW&Zahlencode \textcolor{red}{$\boldsymbol{Grundtext}$} Umschrift $|$"Ubersetzung(en)\\
1.&194.&4726.&718.&18227.&1.&2&31&1\_30 \textcolor{red}{\textcjheb{l'}} AL $|$nicht\\
2.&195.&4727.&720.&18229.&3.&3&415&400\_5\_10 \textcolor{red}{\textcjheb{yht}} THJ $|$sei/du sollst sein\\
3.&196.&4728.&723.&18232.&6.&5&582&2\_400\_100\_70\_10 \textcolor{red}{\textcjheb{y`qtb}} BTQaJ $|$unter denen die einschlagen in/bei Schlagenden\\
4.&197.&4729.&728.&18237.&11.&2&100&20\_80 \textcolor{red}{\textcjheb{pk}} KP $|$die Hand\\
5.&198.&4730.&730.&18239.&13.&6&324&2\_70\_200\_2\_10\_40 \textcolor{red}{\textcjheb{mybr`b}} BaRBJM $|$unter denen die B"urgschaft leisten/bei (den) B"urgenden\\
6.&199.&4731.&736.&18245.&19.&5&747&40\_300\_1\_6\_400 \textcolor{red}{\textcjheb{tw'+sm}} MSAWT $|$f"ur Darlehn/(f"ur) Schulden\\
\end{tabular}\medskip \\
Ende des Verses 22.26\\
Verse: 641, Buchstaben: 23, 740, 18249, Totalwerte: 2199, 54633, 1296035\\
\\
Sei nicht unter denen, die in die Hand einschlagen, unter denen, welche f"ur Darlehn B"urgschaft leisten.\\
\newpage 
{\bf -- 22.27}\\
\medskip \\
\begin{tabular}{rrrrrrrrp{120mm}}
WV&WK&WB&ABK&ABB&ABV&AnzB&TW&Zahlencode \textcolor{red}{$\boldsymbol{Grundtext}$} Umschrift $|$"Ubersetzung(en)\\
1.&200.&4732.&741.&18250.&1.&2&41&1\_40 \textcolor{red}{\textcjheb{m'}} AM $|$wenn\\
2.&201.&4733.&743.&18252.&3.&3&61&1\_10\_50 \textcolor{red}{\textcjheb{ny'}} AJN $|$nicht(s)\\
3.&202.&4734.&746.&18255.&6.&2&50&30\_20 \textcolor{red}{\textcjheb{kl}} LK $|$du hast\\
4.&203.&4735.&748.&18257.&8.&4&400&30\_300\_30\_40 \textcolor{red}{\textcjheb{ml+sl}} LSLM $|$um zu bezahlen\\
5.&204.&4736.&752.&18261.&12.&3&75&30\_40\_5 \textcolor{red}{\textcjheb{hml}} LMH $|$warum\\
6.&205.&4737.&755.&18264.&15.&3&118&10\_100\_8 \textcolor{red}{\textcjheb{.hqy}} JQC $|$er soll (weg)nehmen\\
7.&206.&4738.&758.&18267.&18.&5&382&40\_300\_20\_2\_20 \textcolor{red}{\textcjheb{kbk+sm}} MSKBK $|$dein Bett/dein Lager\\
8.&207.&4739.&763.&18272.&23.&6&878&40\_400\_8\_400\_10\_20 \textcolor{red}{\textcjheb{kyt.htm}} MTCTJK $|$(weg) (von) unter dir\\
\end{tabular}\medskip \\
Ende des Verses 22.27\\
Verse: 642, Buchstaben: 28, 768, 18277, Totalwerte: 2005, 56638, 1298040\\
\\
Wenn du nicht hast, um zu bezahlen, warum soll er dein Bett unter dir wegnehmen?\\
\newpage 
{\bf -- 22.28}\\
\medskip \\
\begin{tabular}{rrrrrrrrp{120mm}}
WV&WK&WB&ABK&ABB&ABV&AnzB&TW&Zahlencode \textcolor{red}{$\boldsymbol{Grundtext}$} Umschrift $|$"Ubersetzung(en)\\
1.&208.&4740.&769.&18278.&1.&2&31&1\_30 \textcolor{red}{\textcjheb{l'}} AL $|$nicht\\
2.&209.&4741.&771.&18280.&3.&3&463&400\_60\_3 \textcolor{red}{\textcjheb{gst}} TsG $|$(du sollst) verr"ucke(n)\\
3.&210.&4742.&774.&18283.&6.&4&41&3\_2\_6\_30 \textcolor{red}{\textcjheb{lwbg}} GBWL $|$die Grenze/(eine) Grenze\\
4.&211.&4743.&778.&18287.&10.&4&146&70\_6\_30\_40 \textcolor{red}{\textcjheb{mlw`}} aWLM $|$(ur)alte\\
5.&212.&4744.&782.&18291.&14.&3&501&1\_300\_200 \textcolor{red}{\textcjheb{r+s'}} ASR $|$welche\\
6.&213.&4745.&785.&18294.&17.&3&376&70\_300\_6 \textcolor{red}{\textcjheb{w+s`}} aSW $|$(sie) haben gemacht\\
7.&214.&4746.&788.&18297.&20.&6&439&1\_2\_6\_400\_10\_20 \textcolor{red}{\textcjheb{kytwb'}} ABWTJK $|$deine V"ater\\
\end{tabular}\medskip \\
Ende des Verses 22.28\\
Verse: 643, Buchstaben: 25, 793, 18302, Totalwerte: 1997, 58635, 1300037\\
\\
Verr"ucke nicht die alte Grenze, welche deine V"ater gemacht haben.\\
\newpage 
{\bf -- 22.29}\\
\medskip \\
\begin{tabular}{rrrrrrrrp{120mm}}
WV&WK&WB&ABK&ABB&ABV&AnzB&TW&Zahlencode \textcolor{red}{$\boldsymbol{Grundtext}$} Umschrift $|$"Ubersetzung(en)\\
1.&215.&4747.&794.&18303.&1.&4&425&8\_7\_10\_400 \textcolor{red}{\textcjheb{tyz.h}} CZJT $|$siehst du\\
2.&216.&4748.&798.&18307.&5.&3&311&1\_10\_300 \textcolor{red}{\textcjheb{+sy'}} AJS $|$(einen) Mann\\
3.&217.&4749.&801.&18310.&8.&4&255&40\_5\_10\_200 \textcolor{red}{\textcjheb{ryhm}} MHJR $|$der gewandt ist/behenden\\
4.&218.&4750.&805.&18314.&12.&7&499&2\_40\_30\_1\_20\_400\_6 \textcolor{red}{\textcjheb{wtk'lmb}} BMLAKTW $|$in seinem Gesch"aft/bei seiner Arbeit\\
5.&219.&4751.&812.&18321.&19.&4&170&30\_80\_50\_10 \textcolor{red}{\textcjheb{ynpl}} LPNJ $|$vor\\
6.&220.&4752.&816.&18325.&23.&5&140&40\_30\_20\_10\_40 \textcolor{red}{\textcjheb{myklm}} MLKJM $|$K"onige(n)\\
7.&221.&4753.&821.&18330.&28.&5&512&10\_400\_10\_90\_2 \textcolor{red}{\textcjheb{b.syty}} JTJ"sB $|$wird er stehen/er stellt sich hin\\
8.&222.&4754.&826.&18335.&33.&2&32&2\_30 \textcolor{red}{\textcjheb{lb}} BL $|$nicht\\
9.&223.&4755.&828.&18337.&35.&5&512&10\_400\_10\_90\_2 \textcolor{red}{\textcjheb{b.syty}} JTJ"sB $|$wird er stehen/er stellt sich hin\\
10.&224.&4756.&833.&18342.&40.&4&170&30\_80\_50\_10 \textcolor{red}{\textcjheb{ynpl}} LPNJ $|$vor\\
11.&225.&4757.&837.&18346.&44.&5&378&8\_300\_20\_10\_40 \textcolor{red}{\textcjheb{myk+s.h}} CSKJM $|$Niedrige(n)\\
\end{tabular}\medskip \\
Ende des Verses 22.29\\
Verse: 644, Buchstaben: 48, 841, 18350, Totalwerte: 3404, 62039, 1303441\\
\\
Siehst du einen Mann, der gewandt ist in seinem Gesch"aft-vor K"onigen wird er stehen, er wird nicht vor Niedrigen stehen.\\
\\
{\bf Ende des Kapitels 22}\\
\newpage 
{\bf -- 23.1}\\
\medskip \\
\begin{tabular}{rrrrrrrrp{120mm}}
WV&WK&WB&ABK&ABB&ABV&AnzB&TW&Zahlencode \textcolor{red}{$\boldsymbol{Grundtext}$} Umschrift $|$"Ubersetzung(en)\\
1.&1.&4758.&1.&18351.&1.&2&30&20\_10 \textcolor{red}{\textcjheb{yk}} KJ $|$wenn\\
2.&2.&4759.&3.&18353.&3.&3&702&400\_300\_2 \textcolor{red}{\textcjheb{b+st}} TSB $|$du setzt dich (hin)\\
3.&3.&4760.&6.&18356.&6.&5&114&30\_30\_8\_6\_40 \textcolor{red}{\textcjheb{mw.hll}} LLCWM $|$um zu speisen\\
4.&4.&4761.&11.&18361.&11.&2&401&1\_400 \textcolor{red}{\textcjheb{t'}} AT $|$mit\\
5.&5.&4762.&13.&18363.&13.&4&376&40\_6\_300\_30 \textcolor{red}{\textcjheb{l+swm}} MWSL $|$einem Herrscher/(einem) Herrschenden\\
6.&6.&4763.&17.&18367.&17.&3&62&2\_10\_50 \textcolor{red}{\textcjheb{nyb}} BJN $|$(so) (be)achte \\
7.&7.&4764.&20.&18370.&20.&4&462&400\_2\_10\_50 \textcolor{red}{\textcjheb{nybt}} TBJN $|$wohl/du sollst merken\\
8.&8.&4765.&24.&18374.&24.&2&401&1\_400 \textcolor{red}{\textcjheb{t'}} AT $|$**\\
9.&9.&4766.&26.&18376.&26.&3&501&1\_300\_200 \textcolor{red}{\textcjheb{r+s'}} ASR $|$wen du/was\\
10.&10.&4767.&29.&18379.&29.&5&190&30\_80\_50\_10\_20 \textcolor{red}{\textcjheb{kynpl}} LPNJK $|$vor dir hast/vor deinem (An)Gesicht\\
\end{tabular}\medskip \\
Ende des Verses 23.1\\
Verse: 645, Buchstaben: 33, 33, 18383, Totalwerte: 3239, 3239, 1306680\\
\\
Wenn du dich hinsetzest, um mit einem Herrscher zu speisen, so beachte wohl, wen du vor dir hast;\\
\newpage 
{\bf -- 23.2}\\
\medskip \\
\begin{tabular}{rrrrrrrrp{120mm}}
WV&WK&WB&ABK&ABB&ABV&AnzB&TW&Zahlencode \textcolor{red}{$\boldsymbol{Grundtext}$} Umschrift $|$"Ubersetzung(en)\\
1.&11.&4768.&34.&18384.&1.&4&746&6\_300\_40\_400 \textcolor{red}{\textcjheb{tm+sw}} WSMT $|$und setze\\
2.&12.&4769.&38.&18388.&5.&4&380&300\_20\_10\_50 \textcolor{red}{\textcjheb{nyk+s}} SKJN $|$(ein) Messer\\
3.&13.&4770.&42.&18392.&9.&4&122&2\_30\_70\_20 \textcolor{red}{\textcjheb{k`lb}} BLaK $|$an deine Kehle\\
4.&14.&4771.&46.&18396.&13.&2&41&1\_40 \textcolor{red}{\textcjheb{m'}} AM $|$wenn\\
5.&15.&4772.&48.&18398.&15.&3&102&2\_70\_30 \textcolor{red}{\textcjheb{l`b}} BaL $|$Besitzer\\
6.&16.&4773.&51.&18401.&18.&3&430&50\_80\_300 \textcolor{red}{\textcjheb{+spn}} NPS $|$gierig/(von) Gier\\
7.&17.&4774.&54.&18404.&21.&3&406&1\_400\_5 \textcolor{red}{\textcjheb{ht'}} ATH $|$du (bist)\\
\end{tabular}\medskip \\
Ende des Verses 23.2\\
Verse: 646, Buchstaben: 23, 56, 18406, Totalwerte: 2227, 5466, 1308907\\
\\
und setze ein Messer an deine Kehle, wenn du gierig bist.\\
\newpage 
{\bf -- 23.3}\\
\medskip \\
\begin{tabular}{rrrrrrrrp{120mm}}
WV&WK&WB&ABK&ABB&ABV&AnzB&TW&Zahlencode \textcolor{red}{$\boldsymbol{Grundtext}$} Umschrift $|$"Ubersetzung(en)\\
1.&18.&4775.&57.&18407.&1.&2&31&1\_30 \textcolor{red}{\textcjheb{l'}} AL $|$nicht\\
2.&19.&4776.&59.&18409.&3.&4&807&400\_400\_1\_6 \textcolor{red}{\textcjheb{w'tt}} TTAW $|$lass dich gel"usten/du sollst begehren\\
3.&20.&4777.&63.&18413.&7.&9&611&30\_40\_9\_70\_40\_6\_400\_10\_6 \textcolor{red}{\textcjheb{wytwm`.tml}} LMtaMWTJW $|$nach seinen Leckerbissen\\
4.&21.&4778.&72.&18422.&16.&4&18&6\_5\_6\_1 \textcolor{red}{\textcjheb{'whw}} WHWA $|$denn sie sind/und er (=es ist)\\
5.&22.&4779.&76.&18426.&20.&3&78&30\_8\_40 \textcolor{red}{\textcjheb{m.hl}} LCM $|$eine Speise/Brot\\
6.&23.&4780.&79.&18429.&23.&5&79&20\_7\_2\_10\_40 \textcolor{red}{\textcjheb{mybzk}} KZBJM $|$tr"ugerische/(von) L"ugen\\
\end{tabular}\medskip \\
Ende des Verses 23.3\\
Verse: 647, Buchstaben: 27, 83, 18433, Totalwerte: 1624, 7090, 1310531\\
\\
La"s dich nicht gel"usten nach seinen Leckerbissen, denn sie sind eine tr"ugliche Speise.\\
\newpage 
{\bf -- 23.4}\\
\medskip \\
\begin{tabular}{rrrrrrrrp{120mm}}
WV&WK&WB&ABK&ABB&ABV&AnzB&TW&Zahlencode \textcolor{red}{$\boldsymbol{Grundtext}$} Umschrift $|$"Ubersetzung(en)\\
1.&24.&4781.&84.&18434.&1.&2&31&1\_30 \textcolor{red}{\textcjheb{l'}} AL $|$nicht\\
2.&25.&4782.&86.&18436.&3.&4&483&400\_10\_3\_70 \textcolor{red}{\textcjheb{`gyt}} TJGa $|$bem"uhe dich/sollst du dich abm"uhen\\
3.&26.&4783.&90.&18440.&7.&6&615&30\_5\_70\_300\_10\_200 \textcolor{red}{\textcjheb{ry+s`hl}} LHaSJR $|$(um) reich zu werden\\
4.&27.&4784.&96.&18446.&13.&6&522&40\_2\_10\_50\_400\_20 \textcolor{red}{\textcjheb{ktnybm}} MBJNTK $|$von deiner Klugheit/ob deiner Einsicht\\
5.&28.&4785.&102.&18452.&19.&3&42&8\_4\_30 \textcolor{red}{\textcjheb{ld.h}} CDL $|$lass ab (davon)\\
\end{tabular}\medskip \\
Ende des Verses 23.4\\
Verse: 648, Buchstaben: 21, 104, 18454, Totalwerte: 1693, 8783, 1312224\\
\\
Bem"uhe dich nicht, reich zu werden, la"s ab von deiner Klugheit.\\
\newpage 
{\bf -- 23.5}\\
\medskip \\
\begin{tabular}{rrrrrrrrp{120mm}}
WV&WK&WB&ABK&ABB&ABV&AnzB&TW&Zahlencode \textcolor{red}{$\boldsymbol{Grundtext}$} Umschrift $|$"Ubersetzung(en)\\
1.&29.&4786.&105.&18455.&1.&5&561&5\_400\_70\_6\_80 \textcolor{red}{\textcjheb{pw`th}} HTaWP $|$willst du hinfliegen lassen/etwa du kannst fliegen machen\\
2.&30.&4787.&110.&18460.&6.&5&160&70\_10\_50\_10\_20 \textcolor{red}{\textcjheb{kyny`}} aJNJK $|$deine Augen\\
3.&31.&4788.&115.&18465.&11.&2&8&2\_6 \textcolor{red}{\textcjheb{wb}} BW $|$darauf/nach ihm\\
4.&32.&4789.&117.&18467.&13.&6&123&6\_1\_10\_50\_50\_6 \textcolor{red}{\textcjheb{wnny'w}} WAJNNW $|$und siehe fort ist es/und er ist nicht da\\
5.&33.&4790.&123.&18473.&19.&2&30&20\_10 \textcolor{red}{\textcjheb{yk}} KJ $|$denn\\
6.&34.&4791.&125.&18475.&21.&3&375&70\_300\_5 \textcolor{red}{\textcjheb{h+s`}} aSH $|$sicherlich/(ein) Machen\\
7.&35.&4792.&128.&18478.&24.&4&385&10\_70\_300\_5 \textcolor{red}{\textcjheb{h+s`y}} JaSH $|$schafft es/er machte\\
8.&36.&4793.&132.&18482.&28.&2&36&30\_6 \textcolor{red}{\textcjheb{wl}} LW $|$sich\\
9.&37.&4794.&134.&18484.&30.&5&200&20\_50\_80\_10\_40 \textcolor{red}{\textcjheb{mypnk}} KNPJM $|$(die) Fl"ugel\\
10.&38.&4795.&139.&18489.&35.&4&570&20\_50\_300\_200 \textcolor{red}{\textcjheb{r+snk}} KNSR $|$gleich dem Adler/wie (ein) Adler\\
11.&39.&4796.&143.&18493.&39.&4&166&6\_70\_10\_80 \textcolor{red}{\textcjheb{py`w}} WaJP $|$der fliegt/und er fliegt\\
12.&40.&4797.&147.&18497.&43.&5&395&5\_300\_40\_10\_40 \textcolor{red}{\textcjheb{mym+sh}} HSMJM $|$gen Himmel/(zum) Himmel\\
\end{tabular}\medskip \\
Ende des Verses 23.5\\
Verse: 649, Buchstaben: 47, 151, 18501, Totalwerte: 3009, 11792, 1315233\\
\\
Willst du deine Augen darauf hinfliegen lassen, und siehe, fort ist es? Denn sicherlich schafft es sich Fl"ugel gleich dem Adler, der gen Himmel fliegt.\\
\newpage 
{\bf -- 23.6}\\
\medskip \\
\begin{tabular}{rrrrrrrrp{120mm}}
WV&WK&WB&ABK&ABB&ABV&AnzB&TW&Zahlencode \textcolor{red}{$\boldsymbol{Grundtext}$} Umschrift $|$"Ubersetzung(en)\\
1.&41.&4798.&152.&18502.&1.&2&31&1\_30 \textcolor{red}{\textcjheb{l'}} AL $|$nicht\\
2.&42.&4799.&154.&18504.&3.&4&478&400\_30\_8\_40 \textcolor{red}{\textcjheb{m.hlt}} TLCM $|$iss/du sollst essen\\
3.&43.&4800.&158.&18508.&7.&2&401&1\_400 \textcolor{red}{\textcjheb{t'}} AT $|$**\\
4.&44.&4801.&160.&18510.&9.&3&78&30\_8\_40 \textcolor{red}{\textcjheb{m.hl}} LCM $|$das Brot\\
5.&45.&4802.&163.&18513.&12.&2&270&200\_70 \textcolor{red}{\textcjheb{`r}} Ra $|$des scheel/(eines) B"osen\\
6.&46.&4803.&165.&18515.&14.&3&130&70\_10\_50 \textcolor{red}{\textcjheb{ny`}} aJN $|$Sehenden/(von) Auge\\
7.&47.&4804.&168.&18518.&17.&3&37&6\_1\_30 \textcolor{red}{\textcjheb{l'w}} WAL $|$und nicht\\
8.&48.&4805.&171.&18521.&20.&4&807&400\_400\_1\_6 \textcolor{red}{\textcjheb{w'tt}} TTAW $|$lass dich gel"usten/du sollst begehren\\
9.&49.&4806.&175.&18525.&24.&8&605&30\_40\_9\_70\_40\_400\_10\_6 \textcolor{red}{\textcjheb{wytm`.tml}} LMtaMTJW $|$nach seinen Leckerbissen\\
\end{tabular}\medskip \\
Ende des Verses 23.6\\
Verse: 650, Buchstaben: 31, 182, 18532, Totalwerte: 2837, 14629, 1318070\\
\\
I"s nicht das Brot des Scheelsehenden, und la"s dich nicht gel"usten nach seinen Leckerbissen.\\
\newpage 
{\bf -- 23.7}\\
\medskip \\
\begin{tabular}{rrrrrrrrp{120mm}}
WV&WK&WB&ABK&ABB&ABV&AnzB&TW&Zahlencode \textcolor{red}{$\boldsymbol{Grundtext}$} Umschrift $|$"Ubersetzung(en)\\
1.&50.&4807.&183.&18533.&1.&2&30&20\_10 \textcolor{red}{\textcjheb{yk}} KJ $|$denn\\
2.&51.&4808.&185.&18535.&3.&3&66&20\_40\_6 \textcolor{red}{\textcjheb{wmk}} KMW $|$wie\\
3.&52.&4809.&188.&18538.&6.&3&570&300\_70\_200 \textcolor{red}{\textcjheb{r`+s}} SaR $|$er abmisst/Schauder\\
4.&53.&4810.&191.&18541.&9.&5&438&2\_50\_80\_300\_6 \textcolor{red}{\textcjheb{w+spnb}} BNPSW $|$in seiner Seele/f"ur seine Seele\\
5.&54.&4811.&196.&18546.&14.&2&70&20\_50 \textcolor{red}{\textcjheb{nk}} KN $|$so\\
6.&55.&4812.&198.&18548.&16.&3&12&5\_6\_1 \textcolor{red}{\textcjheb{'wh}} HWA $|$ist er/(w"are) er (=es)\\
7.&56.&4813.&201.&18551.&19.&3&51&1\_20\_30 \textcolor{red}{\textcjheb{lk'}} AKL $|$iss\\
8.&57.&4814.&204.&18554.&22.&4&711&6\_300\_400\_5 \textcolor{red}{\textcjheb{ht+sw}} WSTH $|$und trink\\
9.&58.&4815.&208.&18558.&26.&4&251&10\_1\_40\_200 \textcolor{red}{\textcjheb{rm'y}} JAMR $|$spricht er/er sagt\\
10.&59.&4816.&212.&18562.&30.&2&50&30\_20 \textcolor{red}{\textcjheb{kl}} LK $|$zu dir\\
11.&60.&4817.&214.&18564.&32.&4&44&6\_30\_2\_6 \textcolor{red}{\textcjheb{wblw}} WLBW $|$aber sein Herz/und sein Herz\\
12.&61.&4818.&218.&18568.&36.&2&32&2\_30 \textcolor{red}{\textcjheb{lb}} BL $|$nicht\\
13.&62.&4819.&220.&18570.&38.&3&130&70\_40\_20 \textcolor{red}{\textcjheb{km`}} aMK $|$ist mit dir/(ist) bei dir\\
\end{tabular}\medskip \\
Ende des Verses 23.7\\
Verse: 651, Buchstaben: 40, 222, 18572, Totalwerte: 2455, 17084, 1320525\\
\\
Denn wie er es abmi"st in seiner Seele, so ist er. 'I"s und trink!', spricht er zu dir, aber sein Herz ist nicht mit dir.\\
\newpage 
{\bf -- 23.8}\\
\medskip \\
\begin{tabular}{rrrrrrrrp{120mm}}
WV&WK&WB&ABK&ABB&ABV&AnzB&TW&Zahlencode \textcolor{red}{$\boldsymbol{Grundtext}$} Umschrift $|$"Ubersetzung(en)\\
1.&63.&4820.&223.&18573.&1.&3&500&80\_400\_20 \textcolor{red}{\textcjheb{ktp}} PTK $|$deinen Bissen\\
2.&64.&4821.&226.&18576.&4.&4&451&1\_20\_30\_400 \textcolor{red}{\textcjheb{tlk'}} AKLT $|$den du gegessen (hast)\\
3.&65.&4822.&230.&18580.&8.&6&566&400\_100\_10\_1\_50\_5 \textcolor{red}{\textcjheb{hn'yqt}} TQJANH $|$du musst ausspeien (sie (=ihn))\\
4.&66.&4823.&236.&18586.&14.&4&714&6\_300\_8\_400 \textcolor{red}{\textcjheb{t.h+sw}} WSCT $|$und du wirst verlieren/und du hast verschwendet\\
5.&67.&4824.&240.&18590.&18.&5&236&4\_2\_200\_10\_20 \textcolor{red}{\textcjheb{kyrbd}} DBRJK $|$deine Worte\\
6.&68.&4825.&245.&18595.&23.&7&225&5\_50\_70\_10\_40\_10\_40 \textcolor{red}{\textcjheb{mymy`nh}} HNaJMJM $|$freundlichen/die lieblichen\\
\end{tabular}\medskip \\
Ende des Verses 23.8\\
Verse: 652, Buchstaben: 29, 251, 18601, Totalwerte: 2692, 19776, 1323217\\
\\
Deinen Bissen, den du gegessen hast, mu"st du ausspeien, und deine freundlichen Worte wirst du verlieren.\\
\newpage 
{\bf -- 23.9}\\
\medskip \\
\begin{tabular}{rrrrrrrrp{120mm}}
WV&WK&WB&ABK&ABB&ABV&AnzB&TW&Zahlencode \textcolor{red}{$\boldsymbol{Grundtext}$} Umschrift $|$"Ubersetzung(en)\\
1.&69.&4826.&252.&18602.&1.&5&70&2\_1\_7\_50\_10 \textcolor{red}{\textcjheb{ynz'b}} BAZNJ $|$zu den Ohren/vor (zwei) Ohren\\
2.&70.&4827.&257.&18607.&6.&4&120&20\_60\_10\_30 \textcolor{red}{\textcjheb{lysk}} KsJL $|$eines Toren/(eines) Narren\\
3.&71.&4828.&261.&18611.&10.&2&31&1\_30 \textcolor{red}{\textcjheb{l'}} AL $|$nicht\\
4.&72.&4829.&263.&18613.&12.&4&606&400\_4\_2\_200 \textcolor{red}{\textcjheb{rbdt}} TDBR $|$(du sollst) rede(n)\\
5.&73.&4830.&267.&18617.&16.&2&30&20\_10 \textcolor{red}{\textcjheb{yk}} KJ $|$denn\\
6.&74.&4831.&269.&18619.&18.&4&25&10\_2\_6\_7 \textcolor{red}{\textcjheb{zwby}} JBWZ $|$er wird verachten/er verachtet\\
7.&75.&4832.&273.&18623.&22.&4&380&30\_300\_20\_30 \textcolor{red}{\textcjheb{lk+sl}} LSKL $|$die Einsicht\\
8.&76.&4833.&277.&18627.&26.&4&100&40\_30\_10\_20 \textcolor{red}{\textcjheb{kylm}} MLJK $|$deiner Worte\\
\end{tabular}\medskip \\
Ende des Verses 23.9\\
Verse: 653, Buchstaben: 29, 280, 18630, Totalwerte: 1362, 21138, 1324579\\
\\
Rede nicht zu den Ohren eines Toren, denn er wird die Einsicht deiner Worte verachten.\\
\newpage 
{\bf -- 23.10}\\
\medskip \\
\begin{tabular}{rrrrrrrrp{120mm}}
WV&WK&WB&ABK&ABB&ABV&AnzB&TW&Zahlencode \textcolor{red}{$\boldsymbol{Grundtext}$} Umschrift $|$"Ubersetzung(en)\\
1.&77.&4834.&281.&18631.&1.&2&31&1\_30 \textcolor{red}{\textcjheb{l'}} AL $|$nicht\\
2.&78.&4835.&283.&18633.&3.&3&463&400\_60\_3 \textcolor{red}{\textcjheb{gst}} TsG $|$(du sollst) verr"ucke(n)\\
3.&79.&4836.&286.&18636.&6.&4&41&3\_2\_6\_30 \textcolor{red}{\textcjheb{lwbg}} GBWL $|$die Grenze/(eine) Grenze\\
4.&80.&4837.&290.&18640.&10.&4&146&70\_6\_30\_40 \textcolor{red}{\textcjheb{mlw`}} aWLM $|$(ur)alte\\
5.&81.&4838.&294.&18644.&14.&5&322&6\_2\_300\_4\_10 \textcolor{red}{\textcjheb{yd+sbw}} WBSDJ $|$und in (die) Felder\\
6.&82.&4839.&299.&18649.&19.&6&506&10\_400\_6\_40\_10\_40 \textcolor{red}{\textcjheb{mymwty}} JTWMJM $|$der Waisen/(der) Verwaisten\\
7.&83.&4840.&305.&18655.&25.&2&31&1\_30 \textcolor{red}{\textcjheb{l'}} AL $|$nicht\\
8.&84.&4841.&307.&18657.&27.&3&403&400\_2\_1 \textcolor{red}{\textcjheb{'bt}} TBA $|$dringe ein/du sollst kommen\\
\end{tabular}\medskip \\
Ende des Verses 23.10\\
Verse: 654, Buchstaben: 29, 309, 18659, Totalwerte: 1943, 23081, 1326522\\
\\
Verr"ucke nicht die alte Grenze, und dringe nicht ein in die Felder der Waisen.\\
\newpage 
{\bf -- 23.11}\\
\medskip \\
\begin{tabular}{rrrrrrrrp{120mm}}
WV&WK&WB&ABK&ABB&ABV&AnzB&TW&Zahlencode \textcolor{red}{$\boldsymbol{Grundtext}$} Umschrift $|$"Ubersetzung(en)\\
1.&85.&4842.&310.&18660.&1.&2&30&20\_10 \textcolor{red}{\textcjheb{yk}} KJ $|$denn\\
2.&86.&4843.&312.&18662.&3.&4&74&3\_1\_30\_40 \textcolor{red}{\textcjheb{ml'g}} GALM $|$ihr Erl"oser/ihr Ausl"oser\\
3.&87.&4844.&316.&18666.&7.&3&115&8\_7\_100 \textcolor{red}{\textcjheb{qz.h}} CZQ $|$(er) ist stark\\
4.&88.&4845.&319.&18669.&10.&3&12&5\_6\_1 \textcolor{red}{\textcjheb{'wh}} HWA $|$(d)er\\
5.&89.&4846.&322.&18672.&13.&4&222&10\_200\_10\_2 \textcolor{red}{\textcjheb{byry}} JRJB $|$(er) wird f"uhren\\
6.&90.&4847.&326.&18676.&17.&2&401&1\_400 \textcolor{red}{\textcjheb{t'}} AT $|$**\\
7.&91.&4848.&328.&18678.&19.&4&252&200\_10\_2\_40 \textcolor{red}{\textcjheb{mbyr}} RJBM $|$ihren Rechtsstreit/ihren Streitfall\\
8.&92.&4849.&332.&18682.&23.&3&421&1\_400\_20 \textcolor{red}{\textcjheb{kt'}} ATK $|$wider dich/mit dir\\
\end{tabular}\medskip \\
Ende des Verses 23.11\\
Verse: 655, Buchstaben: 25, 334, 18684, Totalwerte: 1527, 24608, 1328049\\
\\
Denn ihr Erl"oser ist stark; er wird ihren Rechtsstreit wider dich f"uhren.\\
\newpage 
{\bf -- 23.12}\\
\medskip \\
\begin{tabular}{rrrrrrrrp{120mm}}
WV&WK&WB&ABK&ABB&ABV&AnzB&TW&Zahlencode \textcolor{red}{$\boldsymbol{Grundtext}$} Umschrift $|$"Ubersetzung(en)\\
1.&93.&4850.&335.&18685.&1.&5&23&5\_2\_10\_1\_5 \textcolor{red}{\textcjheb{h'ybh}} HBJAH $|$bringe her/mache kommen\\
2.&94.&4851.&340.&18690.&6.&5&336&30\_40\_6\_60\_200 \textcolor{red}{\textcjheb{rswml}} LMWsR $|$zur Unterweisung/zur Zucht\\
3.&95.&4852.&345.&18695.&11.&3&52&30\_2\_20 \textcolor{red}{\textcjheb{kbl}} LBK $|$dein Herz\\
4.&96.&4853.&348.&18698.&14.&5&84&6\_1\_7\_50\_20 \textcolor{red}{\textcjheb{knz'w}} WAZNK $|$und dein(e) Ohr(en)\\
5.&97.&4854.&353.&18703.&19.&5&281&30\_1\_40\_200\_10 \textcolor{red}{\textcjheb{yrm'l}} LAMRJ $|$zu den Worten/f"ur Worte\\
6.&98.&4855.&358.&18708.&24.&3&474&4\_70\_400 \textcolor{red}{\textcjheb{t`d}} DaT $|$(der) Erkenntnis\\
\end{tabular}\medskip \\
Ende des Verses 23.12\\
Verse: 656, Buchstaben: 26, 360, 18710, Totalwerte: 1250, 25858, 1329299\\
\\
Bringe dein Herz her zur Unterweisung, und deine Ohren zu den Worten der Erkenntnis.\\
\newpage 
{\bf -- 23.13}\\
\medskip \\
\begin{tabular}{rrrrrrrrp{120mm}}
WV&WK&WB&ABK&ABB&ABV&AnzB&TW&Zahlencode \textcolor{red}{$\boldsymbol{Grundtext}$} Umschrift $|$"Ubersetzung(en)\\
1.&99.&4856.&361.&18711.&1.&2&31&1\_30 \textcolor{red}{\textcjheb{l'}} AL $|$nicht\\
2.&100.&4857.&363.&18713.&3.&4&560&400\_40\_50\_70 \textcolor{red}{\textcjheb{`nmt}} TMNa $|$entziehe/du sollst zur"uckhalten \\
3.&101.&4858.&367.&18717.&7.&4&360&40\_50\_70\_200 \textcolor{red}{\textcjheb{r`nm}} MNaR $|$dem Knaben\\
4.&102.&4859.&371.&18721.&11.&4&306&40\_6\_60\_200 \textcolor{red}{\textcjheb{rswm}} MWsR $|$die Z"uchtigung/Zucht\\
5.&103.&4860.&375.&18725.&15.&2&30&20\_10 \textcolor{red}{\textcjheb{yk}} KJ $|$wenn\\
6.&104.&4861.&377.&18727.&17.&4&476&400\_20\_50\_6 \textcolor{red}{\textcjheb{wnkt}} TKNW $|$du schl"agst ihn\\
7.&105.&4862.&381.&18731.&21.&4&313&2\_300\_2\_9 \textcolor{red}{\textcjheb{.tb+sb}} BSBt $|$mit der Rute\\
8.&106.&4863.&385.&18735.&25.&2&31&30\_1 \textcolor{red}{\textcjheb{'l}} LA $|$nicht\\
9.&107.&4864.&387.&18737.&27.&4&456&10\_40\_6\_400 \textcolor{red}{\textcjheb{twmy}} JMWT $|$er wird sterben\\
\end{tabular}\medskip \\
Ende des Verses 23.13\\
Verse: 657, Buchstaben: 30, 390, 18740, Totalwerte: 2563, 28421, 1331862\\
\\
Entziehe dem Knaben nicht die Z"uchtigung; wenn du ihn mit der Rute schl"agst, wird er nicht sterben.\\
\newpage 
{\bf -- 23.14}\\
\medskip \\
\begin{tabular}{rrrrrrrrp{120mm}}
WV&WK&WB&ABK&ABB&ABV&AnzB&TW&Zahlencode \textcolor{red}{$\boldsymbol{Grundtext}$} Umschrift $|$"Ubersetzung(en)\\
1.&108.&4865.&391.&18741.&1.&3&406&1\_400\_5 \textcolor{red}{\textcjheb{ht'}} ATH $|$du\\
2.&109.&4866.&394.&18744.&4.&4&313&2\_300\_2\_9 \textcolor{red}{\textcjheb{.tb+sb}} BSBt $|$mit der Rute\\
3.&110.&4867.&398.&18748.&8.&4&476&400\_20\_50\_6 \textcolor{red}{\textcjheb{wnkt}} TKNW $|$(du) schl"agst ihn\\
4.&111.&4868.&402.&18752.&12.&5&442&6\_50\_80\_300\_6 \textcolor{red}{\textcjheb{w+spnw}} WNPSW $|$und seine Seele\\
5.&112.&4869.&407.&18757.&17.&5&377&40\_300\_1\_6\_30 \textcolor{red}{\textcjheb{lw'+sm}} MSAWL $|$vor dem Scheol/vor der Unterwelt\\
6.&113.&4870.&412.&18762.&22.&4&530&400\_90\_10\_30 \textcolor{red}{\textcjheb{ly.st}} T"sJL $|$du (er)rettest\\
\end{tabular}\medskip \\
Ende des Verses 23.14\\
Verse: 658, Buchstaben: 25, 415, 18765, Totalwerte: 2544, 30965, 1334406\\
\\
Du schl"agst ihn mit der Rute, und du errettest seine Seele von dem Scheol.\\
\newpage 
{\bf -- 23.15}\\
\medskip \\
\begin{tabular}{rrrrrrrrp{120mm}}
WV&WK&WB&ABK&ABB&ABV&AnzB&TW&Zahlencode \textcolor{red}{$\boldsymbol{Grundtext}$} Umschrift $|$"Ubersetzung(en)\\
1.&114.&4871.&416.&18766.&1.&3&62&2\_50\_10 \textcolor{red}{\textcjheb{ynb}} BNJ $|$mein Sohn\\
2.&115.&4872.&419.&18769.&4.&2&41&1\_40 \textcolor{red}{\textcjheb{m'}} AM $|$wenn\\
3.&116.&4873.&421.&18771.&6.&3&68&8\_20\_40 \textcolor{red}{\textcjheb{mk.h}} CKM $|$(er (=es)) ist weise\\
4.&117.&4874.&424.&18774.&9.&3&52&30\_2\_20 \textcolor{red}{\textcjheb{kbl}} LBK $|$dein Herz\\
5.&118.&4875.&427.&18777.&12.&4&358&10\_300\_40\_8 \textcolor{red}{\textcjheb{.hm+sy}} JSMC $|$so wird sich freuen/er (=es) freut sich\\
6.&119.&4876.&431.&18781.&16.&3&42&30\_2\_10 \textcolor{red}{\textcjheb{ybl}} LBJ $|$mein Herz\\
7.&120.&4877.&434.&18784.&19.&2&43&3\_40 \textcolor{red}{\textcjheb{mg}} GM $|$auch\\
8.&121.&4878.&436.&18786.&21.&3&61&1\_50\_10 \textcolor{red}{\textcjheb{yn'}} ANJ $|$/ich\\
\end{tabular}\medskip \\
Ende des Verses 23.15\\
Verse: 659, Buchstaben: 23, 438, 18788, Totalwerte: 727, 31692, 1335133\\
\\
Mein Sohn, wenn dein Herz weise ist, so wird auch mein Herz sich freuen;\\
\newpage 
{\bf -- 23.16}\\
\medskip \\
\begin{tabular}{rrrrrrrrp{120mm}}
WV&WK&WB&ABK&ABB&ABV&AnzB&TW&Zahlencode \textcolor{red}{$\boldsymbol{Grundtext}$} Umschrift $|$"Ubersetzung(en)\\
1.&122.&4879.&439.&18789.&1.&7&568&6\_400\_70\_30\_7\_50\_5 \textcolor{red}{\textcjheb{hnzl`tw}} WTaLZNH $|$und (es) werden frohlocken/und sie (=es) werden jauchzen\\
2.&123.&4880.&446.&18796.&8.&6&476&20\_30\_10\_6\_400\_10 \textcolor{red}{\textcjheb{ytwylk}} KLJWTJ $|$meine Nieren\\
3.&124.&4881.&452.&18802.&14.&4&208&2\_4\_2\_200 \textcolor{red}{\textcjheb{rbdb}} BDBR $|$wenn reden/im Sagen\\
4.&125.&4882.&456.&18806.&18.&5&810&300\_80\_400\_10\_20 \textcolor{red}{\textcjheb{kytp+s}} SPTJK $|$deine Lippen\\
5.&126.&4883.&461.&18811.&23.&6&600&40\_10\_300\_200\_10\_40 \textcolor{red}{\textcjheb{myr+sym}} MJSRJM $|$Geradheit(en)\\
\end{tabular}\medskip \\
Ende des Verses 23.16\\
Verse: 660, Buchstaben: 28, 466, 18816, Totalwerte: 2662, 34354, 1337795\\
\\
und meine Nieren werden frohlocken, wenn deine Lippen Geradheit reden.\\
\newpage 
{\bf -- 23.17}\\
\medskip \\
\begin{tabular}{rrrrrrrrp{120mm}}
WV&WK&WB&ABK&ABB&ABV&AnzB&TW&Zahlencode \textcolor{red}{$\boldsymbol{Grundtext}$} Umschrift $|$"Ubersetzung(en)\\
1.&127.&4884.&467.&18817.&1.&2&31&1\_30 \textcolor{red}{\textcjheb{l'}} AL $|$nicht\\
2.&128.&4885.&469.&18819.&3.&4&161&10\_100\_50\_1 \textcolor{red}{\textcjheb{'nqy}} JQNA $|$beneide/er (=es) soll eifern\\
3.&129.&4886.&473.&18823.&7.&3&52&30\_2\_20 \textcolor{red}{\textcjheb{kbl}} LBK $|$dein Herz\\
4.&130.&4887.&476.&18826.&10.&6&70&2\_8\_9\_1\_10\_40 \textcolor{red}{\textcjheb{my'.t.hb}} BCtAJM $|$("uber) die S"under\\
5.&131.&4888.&482.&18832.&16.&2&30&20\_10 \textcolor{red}{\textcjheb{yk}} KJ $|$sondern/denn\\
6.&132.&4889.&484.&18834.&18.&2&41&1\_40 \textcolor{red}{\textcjheb{m'}} AM $|$/wenn\\
7.&133.&4890.&486.&18836.&20.&5&613&2\_10\_200\_1\_400 \textcolor{red}{\textcjheb{t'ryb}} BJRAT $|$(beeifere sich) um die Furcht/in der Furcht\\
8.&134.&4891.&491.&18841.&25.&4&26&10\_5\_6\_5 \textcolor{red}{\textcjheb{hwhy}} JHWH $|$(vor) Jahwe(s)\\
9.&135.&4892.&495.&18845.&29.&2&50&20\_30 \textcolor{red}{\textcjheb{lk}} KL $|$jeden/all\\
10.&136.&4893.&497.&18847.&31.&4&61&5\_10\_6\_40 \textcolor{red}{\textcjheb{mwyh}} HJWM $|$(der) Tag\\
\end{tabular}\medskip \\
Ende des Verses 23.17\\
Verse: 661, Buchstaben: 34, 500, 18850, Totalwerte: 1135, 35489, 1338930\\
\\
Dein Herz beneide nicht die S"under, sondern beeifere sich jeden Tag um die Furcht Jahwes.\\
\newpage 
{\bf -- 23.18}\\
\medskip \\
\begin{tabular}{rrrrrrrrp{120mm}}
WV&WK&WB&ABK&ABB&ABV&AnzB&TW&Zahlencode \textcolor{red}{$\boldsymbol{Grundtext}$} Umschrift $|$"Ubersetzung(en)\\
1.&137.&4894.&501.&18851.&1.&2&30&20\_10 \textcolor{red}{\textcjheb{yk}} KJ $|$wahrlich/denn\\
2.&138.&4895.&503.&18853.&3.&2&41&1\_40 \textcolor{red}{\textcjheb{m'}} AM $|$/wenn\\
3.&139.&4896.&505.&18855.&5.&2&310&10\_300 \textcolor{red}{\textcjheb{+sy}} JS $|$es gibt\\
4.&140.&4897.&507.&18857.&7.&5&619&1\_8\_200\_10\_400 \textcolor{red}{\textcjheb{tyr.h'}} ACRJT $|$ein (gutes) Ende\\
5.&141.&4898.&512.&18862.&12.&6&932&6\_400\_100\_6\_400\_20 \textcolor{red}{\textcjheb{ktwqtw}} WTQWTK $|$und deine Hoffnung\\
6.&142.&4899.&518.&18868.&18.&2&31&30\_1 \textcolor{red}{\textcjheb{'l}} LA $|$nicht\\
7.&143.&4900.&520.&18870.&20.&4&1020&400\_20\_200\_400 \textcolor{red}{\textcjheb{trkt}} TKRT $|$wird vernichtet werden/sie wird zerst"ort\\
\end{tabular}\medskip \\
Ende des Verses 23.18\\
Verse: 662, Buchstaben: 23, 523, 18873, Totalwerte: 2983, 38472, 1341913\\
\\
Wahrlich, es gibt ein Ende, und deine Hoffnung wird nicht vernichtet werden.\\
\newpage 
{\bf -- 23.19}\\
\medskip \\
\begin{tabular}{rrrrrrrrp{120mm}}
WV&WK&WB&ABK&ABB&ABV&AnzB&TW&Zahlencode \textcolor{red}{$\boldsymbol{Grundtext}$} Umschrift $|$"Ubersetzung(en)\\
1.&144.&4901.&524.&18874.&1.&3&410&300\_40\_70 \textcolor{red}{\textcjheb{`m+s}} SMa $|$h"ore\\
2.&145.&4902.&527.&18877.&4.&3&406&1\_400\_5 \textcolor{red}{\textcjheb{ht'}} ATH $|$du\\
3.&146.&4903.&530.&18880.&7.&3&62&2\_50\_10 \textcolor{red}{\textcjheb{ynb}} BNJ $|$mein Sohn\\
4.&147.&4904.&533.&18883.&10.&4&74&6\_8\_20\_40 \textcolor{red}{\textcjheb{mk.hw}} WCKM $|$und werde weise\\
5.&148.&4905.&537.&18887.&14.&4&507&6\_1\_300\_200 \textcolor{red}{\textcjheb{r+s'w}} WASR $|$und leite geradeaus/und f"uhre (gerade)\\
6.&149.&4906.&541.&18891.&18.&4&226&2\_4\_200\_20 \textcolor{red}{\textcjheb{krdb}} BDRK $|$auf dem Weg\\
7.&150.&4907.&545.&18895.&22.&3&52&30\_2\_20 \textcolor{red}{\textcjheb{kbl}} LBK $|$dein Herz\\
\end{tabular}\medskip \\
Ende des Verses 23.19\\
Verse: 663, Buchstaben: 24, 547, 18897, Totalwerte: 1737, 40209, 1343650\\
\\
H"ore du, mein Sohn, und werde weise, und leite dein Herz geradeaus auf dem Wege.\\
\newpage 
{\bf -- 23.20}\\
\medskip \\
\begin{tabular}{rrrrrrrrp{120mm}}
WV&WK&WB&ABK&ABB&ABV&AnzB&TW&Zahlencode \textcolor{red}{$\boldsymbol{Grundtext}$} Umschrift $|$"Ubersetzung(en)\\
1.&151.&4908.&548.&18898.&1.&2&31&1\_30 \textcolor{red}{\textcjheb{l'}} AL $|$nicht\\
2.&152.&4909.&550.&18900.&3.&3&415&400\_5\_10 \textcolor{red}{\textcjheb{yht}} THJ $|$sei/du sollst sein\\
3.&153.&4910.&553.&18903.&6.&5&75&2\_60\_2\_1\_10 \textcolor{red}{\textcjheb{y'bsb}} BsBAJ $|$unter S"aufern/bei Zechenden\\
4.&154.&4911.&558.&18908.&11.&3&70&10\_10\_50 \textcolor{red}{\textcjheb{nyy}} JJN $|$Wein\\
5.&155.&4912.&561.&18911.&14.&5&79&2\_7\_30\_30\_10 \textcolor{red}{\textcjheb{yllzb}} BZLLJ $|$noch unter denen die verprassen/Anfressenden\\
6.&156.&4913.&566.&18916.&19.&3&502&2\_300\_200 \textcolor{red}{\textcjheb{r+sb}} BSR $|$(mit) Fleisch\\
7.&157.&4914.&569.&18919.&22.&3&76&30\_40\_6 \textcolor{red}{\textcjheb{wml}} LMW $|$/sich\\
\end{tabular}\medskip \\
Ende des Verses 23.20\\
Verse: 664, Buchstaben: 24, 571, 18921, Totalwerte: 1248, 41457, 1344898\\
\\
Sei nicht unter Weins"aufern, noch unter denen, die Fleisch verprassen;\\
\newpage 
{\bf -- 23.21}\\
\medskip \\
\begin{tabular}{rrrrrrrrp{120mm}}
WV&WK&WB&ABK&ABB&ABV&AnzB&TW&Zahlencode \textcolor{red}{$\boldsymbol{Grundtext}$} Umschrift $|$"Ubersetzung(en)\\
1.&158.&4915.&572.&18922.&1.&2&30&20\_10 \textcolor{red}{\textcjheb{yk}} KJ $|$denn\\
2.&159.&4916.&574.&18924.&3.&3&63&60\_2\_1 \textcolor{red}{\textcjheb{'bs}} sBA $|$ein S"aufer/ein Zechender\\
3.&160.&4917.&577.&18927.&6.&5&79&6\_7\_6\_30\_30 \textcolor{red}{\textcjheb{llwzw}} WZWLL $|$und (ein) Schlemmer\\
4.&161.&4918.&582.&18932.&11.&4&516&10\_6\_200\_300 \textcolor{red}{\textcjheb{+srwy}} JWRS $|$verarmen/(er) wird arm\\
5.&162.&4919.&586.&18936.&15.&6&426&6\_100\_200\_70\_10\_40 \textcolor{red}{\textcjheb{my`rqw}} WQRaJM $|$und in Lumpen/und mit Lumpen\\
6.&163.&4920.&592.&18942.&21.&5&742&400\_30\_2\_10\_300 \textcolor{red}{\textcjheb{+syblt}} TLBJS $|$(sie (=es)) kleidet\\
7.&164.&4921.&597.&18947.&26.&4&101&50\_6\_40\_5 \textcolor{red}{\textcjheb{hmwn}} NWMH $|$Schlummer\\
\end{tabular}\medskip \\
Ende des Verses 23.21\\
Verse: 665, Buchstaben: 29, 600, 18950, Totalwerte: 1957, 43414, 1346855\\
\\
denn ein S"aufer und ein Schlemmer verarmen, und Schlummer kleidet in Lumpen.\\
\newpage 
{\bf -- 23.22}\\
\medskip \\
\begin{tabular}{rrrrrrrrp{120mm}}
WV&WK&WB&ABK&ABB&ABV&AnzB&TW&Zahlencode \textcolor{red}{$\boldsymbol{Grundtext}$} Umschrift $|$"Ubersetzung(en)\\
1.&165.&4922.&601.&18951.&1.&3&410&300\_40\_70 \textcolor{red}{\textcjheb{`m+s}} SMa $|$h"ore\\
2.&166.&4923.&604.&18954.&4.&5&63&30\_1\_2\_10\_20 \textcolor{red}{\textcjheb{kyb'l}} LABJK $|$auf deinen Vater\\
3.&167.&4924.&609.&18959.&9.&2&12&7\_5 \textcolor{red}{\textcjheb{hz}} ZH $|$der\\
4.&168.&4925.&611.&18961.&11.&4&64&10\_30\_4\_20 \textcolor{red}{\textcjheb{kdly}} JLDK $|$gezeugt hat dich/(er) zeugte dich\\
5.&169.&4926.&615.&18965.&15.&3&37&6\_1\_30 \textcolor{red}{\textcjheb{l'w}} WAL $|$und nicht\\
6.&170.&4927.&618.&18968.&18.&4&415&400\_2\_6\_7 \textcolor{red}{\textcjheb{zwbt}} TBWZ $|$(du darfst) verachte(n)\\
7.&171.&4928.&622.&18972.&22.&2&30&20\_10 \textcolor{red}{\textcjheb{yk}} KJ $|$wenn\\
8.&172.&4929.&624.&18974.&24.&4&162&7\_100\_50\_5 \textcolor{red}{\textcjheb{hnqz}} ZQNH $|$sie alt geworden ist/sie gealtert\\
9.&173.&4930.&628.&18978.&28.&3&61&1\_40\_20 \textcolor{red}{\textcjheb{km'}} AMK $|$deine Mutter\\
\end{tabular}\medskip \\
Ende des Verses 23.22\\
Verse: 666, Buchstaben: 30, 630, 18980, Totalwerte: 1254, 44668, 1348109\\
\\
H"ore auf deinen Vater, der dich gezeugt hat, und verachte deine Mutter nicht, wenn sie alt geworden ist.\\
\newpage 
{\bf -- 23.23}\\
\medskip \\
\begin{tabular}{rrrrrrrrp{120mm}}
WV&WK&WB&ABK&ABB&ABV&AnzB&TW&Zahlencode \textcolor{red}{$\boldsymbol{Grundtext}$} Umschrift $|$"Ubersetzung(en)\\
1.&174.&4931.&631.&18981.&1.&3&441&1\_40\_400 \textcolor{red}{\textcjheb{tm'}} AMT $|$Wahrheit\\
2.&175.&4932.&634.&18984.&4.&3&155&100\_50\_5 \textcolor{red}{\textcjheb{hnq}} QNH $|$kaufe/erwirb\\
3.&176.&4933.&637.&18987.&7.&3&37&6\_1\_30 \textcolor{red}{\textcjheb{l'w}} WAL $|$und nicht\\
4.&177.&4934.&640.&18990.&10.&4&660&400\_40\_20\_200 \textcolor{red}{\textcjheb{rkmt}} TMKR $|$(du sollst) verkaufe(n) (sie)\\
5.&178.&4935.&644.&18994.&14.&4&73&8\_20\_40\_5 \textcolor{red}{\textcjheb{hmk.h}} CKMH $|$Weisheit\\
6.&179.&4936.&648.&18998.&18.&5&312&6\_40\_6\_60\_200 \textcolor{red}{\textcjheb{rswmw}} WMWsR $|$und Unterweisung/und Zucht\\
7.&180.&4937.&653.&19003.&23.&5&73&6\_2\_10\_50\_5 \textcolor{red}{\textcjheb{hnybw}} WBJNH $|$und Verstand/und Einsicht\\
\end{tabular}\medskip \\
Ende des Verses 23.23\\
Verse: 667, Buchstaben: 27, 657, 19007, Totalwerte: 1751, 46419, 1349860\\
\\
Kaufe Wahrheit und verkaufe sie nicht, Weisheit und Unterweisung und Verstand.\\
\newpage 
{\bf -- 23.24}\\
\medskip \\
\begin{tabular}{rrrrrrrrp{120mm}}
WV&WK&WB&ABK&ABB&ABV&AnzB&TW&Zahlencode \textcolor{red}{$\boldsymbol{Grundtext}$} Umschrift $|$"Ubersetzung(en)\\
1.&181.&4938.&658.&19008.&1.&3&39&3\_6\_30 \textcolor{red}{\textcjheb{lwg}} GWL $|$hoch/ein Jauchzen\\
2.&182.&4939.&661.&19011.&4.&4&49&10\_3\_6\_30 \textcolor{red}{\textcjheb{lwgy}} JGWL $|$frohlockt/er (=es) wird jauchzen\\
3.&183.&4940.&665.&19015.&8.&3&13&1\_2\_10 \textcolor{red}{\textcjheb{yb'}} ABJ $|$der Vater\\
4.&184.&4941.&668.&19018.&11.&4&204&90\_4\_10\_100 \textcolor{red}{\textcjheb{qyd.s}} "sDJQ $|$eines Gerechten/(des) Gerechten\\
5.&185.&4942.&672.&19022.&15.&4&50&10\_6\_30\_4 \textcolor{red}{\textcjheb{dlwy}} JWLD $|$(und) wer gezeugt hat/der Zeugende\\
6.&186.&4943.&676.&19026.&19.&3&68&8\_20\_40 \textcolor{red}{\textcjheb{mk.h}} CKM $|$einen Weisen/(des) Weisen\\
7.&187.&4944.&679.&19029.&22.&5&364&6\_10\_300\_40\_8 \textcolor{red}{\textcjheb{.hm+syw}} WJSMC $|$der freut sich/und er kann sich freuen\\
8.&188.&4945.&684.&19034.&27.&2&8&2\_6 \textcolor{red}{\textcjheb{wb}} BW $|$seiner/an ihm\\
\end{tabular}\medskip \\
Ende des Verses 23.24\\
Verse: 668, Buchstaben: 28, 685, 19035, Totalwerte: 795, 47214, 1350655\\
\\
Hoch frohlockt der Vater eines Gerechten; und wer einen Weisen gezeugt hat, der freut sich seiner.\\
\newpage 
{\bf -- 23.25}\\
\medskip \\
\begin{tabular}{rrrrrrrrp{120mm}}
WV&WK&WB&ABK&ABB&ABV&AnzB&TW&Zahlencode \textcolor{red}{$\boldsymbol{Grundtext}$} Umschrift $|$"Ubersetzung(en)\\
1.&189.&4946.&686.&19036.&1.&4&358&10\_300\_40\_8 \textcolor{red}{\textcjheb{.hm+sy}} JSMC $|$(er (=es)) m"oge(n) sich freuen\\
2.&190.&4947.&690.&19040.&5.&4&33&1\_2\_10\_20 \textcolor{red}{\textcjheb{kyb'}} ABJK $|$dein Vater\\
3.&191.&4948.&694.&19044.&9.&4&67&6\_1\_40\_20 \textcolor{red}{\textcjheb{km'w}} WAMK $|$und deine Mutter\\
4.&192.&4949.&698.&19048.&13.&4&439&6\_400\_3\_30 \textcolor{red}{\textcjheb{lgtw}} WTGL $|$und (sie (=es) m"oge) frohlocken\\
5.&193.&4950.&702.&19052.&17.&6&470&10\_6\_30\_4\_400\_20 \textcolor{red}{\textcjheb{ktdlwy}} JWLDTK $|$die dich geboren (hat)\\
\end{tabular}\medskip \\
Ende des Verses 23.25\\
Verse: 669, Buchstaben: 22, 707, 19057, Totalwerte: 1367, 48581, 1352022\\
\\
Freuen m"ogen sich dein Vater und deine Mutter, und frohlocken, die dich geboren!\\
\newpage 
{\bf -- 23.26}\\
\medskip \\
\begin{tabular}{rrrrrrrrp{120mm}}
WV&WK&WB&ABK&ABB&ABV&AnzB&TW&Zahlencode \textcolor{red}{$\boldsymbol{Grundtext}$} Umschrift $|$"Ubersetzung(en)\\
1.&194.&4951.&708.&19058.&1.&3&455&400\_50\_5 \textcolor{red}{\textcjheb{hnt}} TNH $|$gib\\
2.&195.&4952.&711.&19061.&4.&3&62&2\_50\_10 \textcolor{red}{\textcjheb{ynb}} BNJ $|$mein Sohn\\
3.&196.&4953.&714.&19064.&7.&3&52&30\_2\_20 \textcolor{red}{\textcjheb{kbl}} LBK $|$dein Herz\\
4.&197.&4954.&717.&19067.&10.&2&40&30\_10 \textcolor{red}{\textcjheb{yl}} LJ $|$mir\\
5.&198.&4955.&719.&19069.&12.&6&166&6\_70\_10\_50\_10\_20 \textcolor{red}{\textcjheb{kyny`w}} WaJNJK $|$und deine Augen\\
6.&199.&4956.&725.&19075.&18.&4&234&4\_200\_20\_10 \textcolor{red}{\textcjheb{ykrd}} DRKJ $|$an meinen Wegen/(meine) Wege\\
7.&200.&4957.&729.&19079.&22.&5&745&400\_200\_90\_50\_5 \textcolor{red}{\textcjheb{hn.srt}} TR"sNH $|$lass Gefallen haben/sie sollen beobachten\\
\end{tabular}\medskip \\
Ende des Verses 23.26\\
Verse: 670, Buchstaben: 26, 733, 19083, Totalwerte: 1754, 50335, 1353776\\
\\
Gib mir, mein Sohn, dein Herz, und la"s deine Augen Gefallen haben an meinen Wegen!\\
\newpage 
{\bf -- 23.27}\\
\medskip \\
\begin{tabular}{rrrrrrrrp{120mm}}
WV&WK&WB&ABK&ABB&ABV&AnzB&TW&Zahlencode \textcolor{red}{$\boldsymbol{Grundtext}$} Umschrift $|$"Ubersetzung(en)\\
1.&201.&4958.&734.&19084.&1.&2&30&20\_10 \textcolor{red}{\textcjheb{yk}} KJ $|$denn\\
2.&202.&4959.&736.&19086.&3.&4&319&300\_6\_8\_5 \textcolor{red}{\textcjheb{h.hw+s}} SWCH $|$(eine) Grube\\
3.&203.&4960.&740.&19090.&7.&4&215&70\_40\_100\_5 \textcolor{red}{\textcjheb{hqm`}} aMQH $|$tiefe\\
4.&204.&4961.&744.&19094.&11.&4&68&7\_6\_50\_5 \textcolor{red}{\textcjheb{hnwz}} ZWNH $|$ist die Hure/(ist die) Buhlerin\\
5.&205.&4962.&748.&19098.&15.&4&209&6\_2\_1\_200 \textcolor{red}{\textcjheb{r'bw}} WBAR $|$und (ein) Brunnen(schacht)\\
6.&206.&4963.&752.&19102.&19.&3&295&90\_200\_5 \textcolor{red}{\textcjheb{hr.s}} "sRH $|$enger\\
7.&207.&4964.&755.&19105.&22.&5&285&50\_20\_200\_10\_5 \textcolor{red}{\textcjheb{hyrkn}} NKRJH $|$die Fremde/die Ausl"anderin\\
\end{tabular}\medskip \\
Ende des Verses 23.27\\
Verse: 671, Buchstaben: 26, 759, 19109, Totalwerte: 1421, 51756, 1355197\\
\\
Denn die Hure ist eine tiefe Grube und die Fremde ein enger Brunnen;\\
\newpage 
{\bf -- 23.28}\\
\medskip \\
\begin{tabular}{rrrrrrrrp{120mm}}
WV&WK&WB&ABK&ABB&ABV&AnzB&TW&Zahlencode \textcolor{red}{$\boldsymbol{Grundtext}$} Umschrift $|$"Ubersetzung(en)\\
1.&208.&4965.&760.&19110.&1.&2&81&1\_80 \textcolor{red}{\textcjheb{p'}} AP $|$ja/auch\\
2.&209.&4966.&762.&19112.&3.&3&16&5\_10\_1 \textcolor{red}{\textcjheb{'yh}} HJA $|$sie\\
3.&210.&4967.&765.&19115.&6.&4&508&20\_8\_400\_80 \textcolor{red}{\textcjheb{pt.hk}} KCTP $|$wie ein R"auber\\
4.&211.&4968.&769.&19119.&10.&4&603&400\_1\_200\_2 \textcolor{red}{\textcjheb{br't}} TARB $|$(sie) lauert (auf)\\
5.&212.&4969.&773.&19123.&14.&7&71&6\_2\_6\_3\_4\_10\_40 \textcolor{red}{\textcjheb{mydgwbw}} WBWGDJM $|$und (die) Treulose(n)\\
6.&213.&4970.&780.&19130.&21.&4&47&2\_1\_4\_40 \textcolor{red}{\textcjheb{md'b}} BADM $|$unter den Menschen\\
7.&214.&4971.&784.&19134.&25.&4&546&400\_6\_60\_80 \textcolor{red}{\textcjheb{pswt}} TWsP $|$sie mehrt\\
\end{tabular}\medskip \\
Ende des Verses 23.28\\
Verse: 672, Buchstaben: 28, 787, 19137, Totalwerte: 1872, 53628, 1357069\\
\\
ja, sie lauert auf wie ein R"auber, und sie mehrt die Treulosen unter den Menschen.\\
\newpage 
{\bf -- 23.29}\\
\medskip \\
\begin{tabular}{rrrrrrrrp{120mm}}
WV&WK&WB&ABK&ABB&ABV&AnzB&TW&Zahlencode \textcolor{red}{$\boldsymbol{Grundtext}$} Umschrift $|$"Ubersetzung(en)\\
1.&215.&4972.&788.&19138.&1.&3&80&30\_40\_10 \textcolor{red}{\textcjheb{yml}} LMJ $|$wer hat\\
2.&216.&4973.&791.&19141.&4.&3&17&1\_6\_10 \textcolor{red}{\textcjheb{yw'}} AWJ $|$Ach/(ein) Weh\\
3.&217.&4974.&794.&19144.&7.&3&80&30\_40\_10 \textcolor{red}{\textcjheb{yml}} LMJ $|$wer (hat)\\
4.&218.&4975.&797.&19147.&10.&4&19&1\_2\_6\_10 \textcolor{red}{\textcjheb{ywb'}} ABWJ $|$Weh/Verlangen\\
5.&219.&4976.&801.&19151.&14.&3&80&30\_40\_10 \textcolor{red}{\textcjheb{yml}} LMJ $|$wer\\
6.&220.&4977.&804.&19154.&17.&6&150&40\_4\_6\_50\_10\_40 \textcolor{red}{\textcjheb{mynwdm}} MDWNJM $|$Z"ankereien\\
7.&221.&4978.&810.&19160.&23.&3&80&30\_40\_10 \textcolor{red}{\textcjheb{yml}} LMJ $|$wer\\
8.&222.&4979.&813.&19163.&26.&3&318&300\_10\_8 \textcolor{red}{\textcjheb{.hy+s}} SJC $|$Klage/Sorge\\
9.&223.&4980.&816.&19166.&29.&3&80&30\_40\_10 \textcolor{red}{\textcjheb{yml}} LMJ $|$wer\\
10.&224.&4981.&819.&19169.&32.&5&290&80\_90\_70\_10\_40 \textcolor{red}{\textcjheb{my`.sp}} P"saJM $|$Wunden\\
11.&225.&4982.&824.&19174.&37.&3&98&8\_50\_40 \textcolor{red}{\textcjheb{mn.h}} CNM $|$ohne Ursache/grundlos\\
12.&226.&4983.&827.&19177.&40.&3&80&30\_40\_10 \textcolor{red}{\textcjheb{yml}} LMJ $|$wer (hat)\\
13.&227.&4984.&830.&19180.&43.&6&494&8\_20\_30\_30\_6\_400 \textcolor{red}{\textcjheb{twllk.h}} CKLLWT $|$Tr"ubung/Tr"ubheit\\
14.&228.&4985.&836.&19186.&49.&5&180&70\_10\_50\_10\_40 \textcolor{red}{\textcjheb{myny`}} aJNJM $|$(bei)der Augen\\
\end{tabular}\medskip \\
Ende des Verses 23.29\\
Verse: 673, Buchstaben: 53, 840, 19190, Totalwerte: 2046, 55674, 1359115\\
\\
Wer hat Ach, wer hat Weh, wer Z"ankereien, wer Klage, wer Wunden ohne Ursache? Wer Tr"ubung der Augen?\\
\newpage 
{\bf -- 23.30}\\
\medskip \\
\begin{tabular}{rrrrrrrrp{120mm}}
WV&WK&WB&ABK&ABB&ABV&AnzB&TW&Zahlencode \textcolor{red}{$\boldsymbol{Grundtext}$} Umschrift $|$"Ubersetzung(en)\\
1.&229.&4986.&841.&19191.&1.&7&329&30\_40\_1\_8\_200\_10\_40 \textcolor{red}{\textcjheb{myr.h'ml}} LMACRJM $|$die (bis) sp"at Sitzen(den)\\
2.&230.&4987.&848.&19198.&8.&2&100&70\_30 \textcolor{red}{\textcjheb{l`}} aL $|$bei\\
3.&231.&4988.&850.&19200.&10.&4&75&5\_10\_10\_50 \textcolor{red}{\textcjheb{nyyh}} HJJN $|$dem Wein\\
4.&232.&4989.&854.&19204.&14.&5&83&30\_2\_1\_10\_40 \textcolor{red}{\textcjheb{my'bl}} LBAJM $|$die einkehren/die Kommenden\\
5.&233.&4990.&859.&19209.&19.&4&338&30\_8\_100\_200 \textcolor{red}{\textcjheb{rq.hl}} LCQR $|$zu kosten\\
6.&234.&4991.&863.&19213.&23.&4&160&40\_40\_60\_20 \textcolor{red}{\textcjheb{ksmm}} MMsK $|$um Mischtrank/von dem Mischtrank\\
\end{tabular}\medskip \\
Ende des Verses 23.30\\
Verse: 674, Buchstaben: 26, 866, 19216, Totalwerte: 1085, 56759, 1360200\\
\\
Die sp"at beim Weine sitzen, die einkehren, um Mischtrank zu kosten.\\
\newpage 
{\bf -- 23.31}\\
\medskip \\
\begin{tabular}{rrrrrrrrp{120mm}}
WV&WK&WB&ABK&ABB&ABV&AnzB&TW&Zahlencode \textcolor{red}{$\boldsymbol{Grundtext}$} Umschrift $|$"Ubersetzung(en)\\
1.&235.&4992.&867.&19217.&1.&2&31&1\_30 \textcolor{red}{\textcjheb{l'}} AL $|$nicht\\
2.&236.&4993.&869.&19219.&3.&3&601&400\_200\_1 \textcolor{red}{\textcjheb{'rt}} TRA $|$sieh an/du sollst schauen (nach)\\
3.&237.&4994.&872.&19222.&6.&3&70&10\_10\_50 \textcolor{red}{\textcjheb{nyy}} JJN $|$den Wein\\
4.&238.&4995.&875.&19225.&9.&2&30&20\_10 \textcolor{red}{\textcjheb{yk}} KJ $|$wenn/wie\\
5.&239.&4996.&877.&19227.&11.&5&455&10\_400\_1\_4\_40 \textcolor{red}{\textcjheb{md'ty}} JTADM $|$er sich rot zeigt/er rot schimmert\\
6.&240.&4997.&882.&19232.&16.&2&30&20\_10 \textcolor{red}{\textcjheb{yk}} KJ $|$wenn/wie\\
7.&241.&4998.&884.&19234.&18.&3&460&10\_400\_50 \textcolor{red}{\textcjheb{nty}} JTN $|$er blinkt/er (=es) wird gegeben\\
8.&242.&4999.&887.&19237.&21.&4&92&2\_20\_10\_60 \textcolor{red}{\textcjheb{sykb}} BKJs $|$im Becher/in den Becher\\
9.&243.&5000.&891.&19241.&25.&4&136&70\_10\_50\_6 \textcolor{red}{\textcjheb{wny`}} aJNW $|$/sein Auge\\
10.&244.&5001.&895.&19245.&29.&5&465&10\_400\_5\_30\_20 \textcolor{red}{\textcjheb{klhty}} JTHLK $|$hinuntergleitet/er ergeht sich\\
11.&245.&5002.&900.&19250.&34.&7&602&2\_40\_10\_300\_200\_10\_40 \textcolor{red}{\textcjheb{myr+symb}} BMJSRJM $|$leicht/in Geradheiten\\
\end{tabular}\medskip \\
Ende des Verses 23.31\\
Verse: 675, Buchstaben: 40, 906, 19256, Totalwerte: 2972, 59731, 1363172\\
\\
Sieh den Wein nicht an, wenn er sich rot zeigt, wenn er im Becher blinkt, leicht hinuntergleitet.\\
\newpage 
{\bf -- 23.32}\\
\medskip \\
\begin{tabular}{rrrrrrrrp{120mm}}
WV&WK&WB&ABK&ABB&ABV&AnzB&TW&Zahlencode \textcolor{red}{$\boldsymbol{Grundtext}$} Umschrift $|$"Ubersetzung(en)\\
1.&246.&5003.&907.&19257.&1.&6&625&1\_8\_200\_10\_400\_6 \textcolor{red}{\textcjheb{wtyr.h'}} ACRJTW $|$sein Ende ist/am Ende\\
2.&247.&5004.&913.&19263.&7.&4&378&20\_50\_8\_300 \textcolor{red}{\textcjheb{+s.hnk}} KNCS $|$(dass) wie eine Schlange\\
3.&248.&5005.&917.&19267.&11.&3&330&10\_300\_20 \textcolor{red}{\textcjheb{k+sy}} JSK $|$er bei"st\\
4.&249.&5006.&920.&19270.&14.&7&326&6\_20\_90\_80\_70\_50\_10 \textcolor{red}{\textcjheb{yn`p.skw}} WK"sPaNJ $|$und wie ein Basilisk/und wie eine Otter\\
5.&250.&5007.&927.&19277.&21.&4&590&10\_80\_200\_300 \textcolor{red}{\textcjheb{+srpy}} JPRS $|$sticht/er sondert Gift ab\\
\end{tabular}\medskip \\
Ende des Verses 23.32\\
Verse: 676, Buchstaben: 24, 930, 19280, Totalwerte: 2249, 61980, 1365421\\
\\
Sein Ende ist, da"s er bei"st wie eine Schlange und sticht wie ein Basilisk.\\
\newpage 
{\bf -- 23.33}\\
\medskip \\
\begin{tabular}{rrrrrrrrp{120mm}}
WV&WK&WB&ABK&ABB&ABV&AnzB&TW&Zahlencode \textcolor{red}{$\boldsymbol{Grundtext}$} Umschrift $|$"Ubersetzung(en)\\
1.&251.&5008.&931.&19281.&1.&5&160&70\_10\_50\_10\_20 \textcolor{red}{\textcjheb{kyny`}} aJNJK $|$deine Augen\\
2.&252.&5009.&936.&19286.&6.&4&217&10\_200\_1\_6 \textcolor{red}{\textcjheb{w'ry}} JRAW $|$werden sehen/sie sehen\\
3.&253.&5010.&940.&19290.&10.&4&613&7\_200\_6\_400 \textcolor{red}{\textcjheb{twrz}} ZRWT $|$Seltsames/befremdliche (Dinge)\\
4.&254.&5011.&944.&19294.&14.&4&58&6\_30\_2\_20 \textcolor{red}{\textcjheb{kblw}} WLBK $|$und dein Herz\\
5.&255.&5012.&948.&19298.&18.&4&216&10\_4\_2\_200 \textcolor{red}{\textcjheb{rbdy}} JDBR $|$wird reden/er (=es) redet\\
6.&256.&5013.&952.&19302.&22.&6&911&400\_5\_80\_20\_6\_400 \textcolor{red}{\textcjheb{twkpht}} THPKWT $|$verkehrte (Dinge)\\
\end{tabular}\medskip \\
Ende des Verses 23.33\\
Verse: 677, Buchstaben: 27, 957, 19307, Totalwerte: 2175, 64155, 1367596\\
\\
Deine Augen werden Seltsames sehen, und dein Herz wird verkehrte Dinge reden.\\
\newpage 
{\bf -- 23.34}\\
\medskip \\
\begin{tabular}{rrrrrrrrp{120mm}}
WV&WK&WB&ABK&ABB&ABV&AnzB&TW&Zahlencode \textcolor{red}{$\boldsymbol{Grundtext}$} Umschrift $|$"Ubersetzung(en)\\
1.&257.&5014.&958.&19308.&1.&5&431&6\_5\_10\_10\_400 \textcolor{red}{\textcjheb{tyyhw}} WHJJT $|$und du wirst sein/und du bist\\
2.&258.&5015.&963.&19313.&6.&4&342&20\_300\_20\_2 \textcolor{red}{\textcjheb{bk+sk}} KSKB $|$wie einer der liegt/wie ein Liegender\\
3.&259.&5016.&967.&19317.&10.&3&34&2\_30\_2 \textcolor{red}{\textcjheb{blb}} BLB $|$im Herzen\\
4.&260.&5017.&970.&19320.&13.&2&50&10\_40 \textcolor{red}{\textcjheb{my}} JM $|$des Meeres\\
5.&261.&5018.&972.&19322.&15.&5&348&6\_20\_300\_20\_2 \textcolor{red}{\textcjheb{bk+skw}} WKSKB $|$und wie einer der da liegt/und wie ein Liegender\\
6.&262.&5019.&977.&19327.&20.&4&503&2\_200\_1\_300 \textcolor{red}{\textcjheb{+s'rb}} BRAS $|$auf der Spitze/an der Spitze\\
7.&263.&5020.&981.&19331.&24.&3&40&8\_2\_30 \textcolor{red}{\textcjheb{lb.h}} CBL $|$eines Mastes/des Mastes\\
\end{tabular}\medskip \\
Ende des Verses 23.34\\
Verse: 678, Buchstaben: 26, 983, 19333, Totalwerte: 1748, 65903, 1369344\\
\\
Und du wirst sein wie einer, der im Herzen des Meeres liegt, und wie einer, der da liegt auf der Spitze eines Mastes.\\
\newpage 
{\bf -- 23.35}\\
\medskip \\
\begin{tabular}{rrrrrrrrp{120mm}}
WV&WK&WB&ABK&ABB&ABV&AnzB&TW&Zahlencode \textcolor{red}{$\boldsymbol{Grundtext}$} Umschrift $|$"Ubersetzung(en)\\
1.&264.&5021.&984.&19334.&1.&5&91&5\_20\_6\_50\_10 \textcolor{red}{\textcjheb{ynwkh}} HKWNJ $|$man hat mich geschlagen/sie schlugen mich\\
2.&265.&5022.&989.&19339.&6.&2&32&2\_30 \textcolor{red}{\textcjheb{lb}} BL $|$nicht\\
3.&266.&5023.&991.&19341.&8.&5&458&8\_30\_10\_400\_10 \textcolor{red}{\textcjheb{ytyl.h}} CLJTJ $|$schmerzte es mich/ich wurde krank\\
4.&267.&5024.&996.&19346.&13.&6&141&5\_30\_40\_6\_50\_10 \textcolor{red}{\textcjheb{ynwmlh}} HLMWNJ $|$man hat mich gepr"ugelt/sie pr"ugelten mich\\
5.&268.&5025.&1002.&19352.&19.&2&32&2\_30 \textcolor{red}{\textcjheb{lb}} BL $|$nicht(s)\\
6.&269.&5026.&1004.&19354.&21.&5&494&10\_4\_70\_400\_10 \textcolor{red}{\textcjheb{yt`dy}} JDaTJ $|$f"uhlte ich es/ich merkte\\
7.&270.&5027.&1009.&19359.&26.&3&450&40\_400\_10 \textcolor{red}{\textcjheb{ytm}} MTJ $|$wann\\
8.&271.&5028.&1012.&19362.&29.&4&201&1\_100\_10\_90 \textcolor{red}{\textcjheb{.syq'}} AQJ"s $|$werde ich aufwachen/ich erwache\\
9.&272.&5029.&1016.&19366.&33.&5&157&1\_6\_60\_10\_80 \textcolor{red}{\textcjheb{pysw'}} AWsJP $|$ich will es wieder tun/ich will hinzuf"ugen\\
10.&273.&5030.&1021.&19371.&38.&6&459&1\_2\_100\_300\_50\_6 \textcolor{red}{\textcjheb{wn+sqb'}} ABQSNW $|$will aufsuchen ihn/ich will verlangen ihn\\
11.&274.&5031.&1027.&19377.&44.&3&80&70\_6\_4 \textcolor{red}{\textcjheb{dw`}} aWD $|$abermals/noch (mehr)\\
\end{tabular}\medskip \\
Ende des Verses 23.35\\
Verse: 679, Buchstaben: 46, 1029, 19379, Totalwerte: 2595, 68498, 1371939\\
\\
"Man hat mich geschlagen, es schmerzte mich nicht; man hat mich gepr"ugelt, ich f"uhlte es nicht. Wann werde ich aufwachen? Ich will es wieder tun, will ihn abermals aufsuchen."\\
\\
{\bf Ende des Kapitels 23}\\
\newpage 
{\bf -- 24.1}\\
\medskip \\
\begin{tabular}{rrrrrrrrp{120mm}}
WV&WK&WB&ABK&ABB&ABV&AnzB&TW&Zahlencode \textcolor{red}{$\boldsymbol{Grundtext}$} Umschrift $|$"Ubersetzung(en)\\
1.&1.&5032.&1.&19380.&1.&2&31&1\_30 \textcolor{red}{\textcjheb{l'}} AL $|$nicht\\
2.&2.&5033.&3.&19382.&3.&4&551&400\_100\_50\_1 \textcolor{red}{\textcjheb{'nqt}} TQNA $|$(du sollst) beneide(n)\\
3.&3.&5034.&7.&19386.&7.&5&363&2\_1\_50\_300\_10 \textcolor{red}{\textcjheb{y+sn'b}} BANSJ $|$(die) Menschen\\
4.&4.&5035.&12.&19391.&12.&3&275&200\_70\_5 \textcolor{red}{\textcjheb{h`r}} RaH $|$b"ose/(der) Bosheit\\
5.&5.&5036.&15.&19394.&15.&3&37&6\_1\_30 \textcolor{red}{\textcjheb{l'w}} WAL $|$und nicht\\
6.&6.&5037.&18.&19397.&18.&4&807&400\_400\_1\_6 \textcolor{red}{\textcjheb{w'tt}} TTAW $|$lass dich gel"usten/du sollst dich gel"usten lassen\\
7.&7.&5038.&22.&19401.&22.&5&451&30\_5\_10\_6\_400 \textcolor{red}{\textcjheb{twyhl}} LHJWT $|$zu sein\\
8.&8.&5039.&27.&19406.&27.&3&441&1\_400\_40 \textcolor{red}{\textcjheb{mt'}} ATM $|$mit ihnen\\
\end{tabular}\medskip \\
Ende des Verses 24.1\\
Verse: 680, Buchstaben: 29, 29, 19408, Totalwerte: 2956, 2956, 1374895\\
\\
Beneide nicht b"ose Menschen, und la"s dich nicht gel"usten, mit ihnen zu sein;\\
\newpage 
{\bf -- 24.2}\\
\medskip \\
\begin{tabular}{rrrrrrrrp{120mm}}
WV&WK&WB&ABK&ABB&ABV&AnzB&TW&Zahlencode \textcolor{red}{$\boldsymbol{Grundtext}$} Umschrift $|$"Ubersetzung(en)\\
1.&9.&5040.&30.&19409.&1.&2&30&20\_10 \textcolor{red}{\textcjheb{yk}} KJ $|$denn\\
2.&10.&5041.&32.&19411.&3.&2&304&300\_4 \textcolor{red}{\textcjheb{d+s}} SD $|$(auf) Gewalttat\\
3.&11.&5042.&34.&19413.&5.&4&23&10\_5\_3\_5 \textcolor{red}{\textcjheb{hghy}} JHGH $|$(er (=es)) (er)sinnt\\
4.&12.&5043.&38.&19417.&9.&3&72&30\_2\_40 \textcolor{red}{\textcjheb{mbl}} LBM $|$ihr Herz\\
5.&13.&5044.&41.&19420.&12.&4&146&6\_70\_40\_30 \textcolor{red}{\textcjheb{lm`w}} WaML $|$und M"uhsal/und Unheil\\
6.&14.&5045.&45.&19424.&16.&6&835&300\_80\_400\_10\_5\_40 \textcolor{red}{\textcjheb{mhytp+s}} SPTJHM $|$ihre Lippen\\
7.&15.&5046.&51.&19430.&22.&6&661&400\_4\_2\_200\_50\_5 \textcolor{red}{\textcjheb{hnrbdt}} TDBRNH $|$(sie) reden\\
\end{tabular}\medskip \\
Ende des Verses 24.2\\
Verse: 681, Buchstaben: 27, 56, 19435, Totalwerte: 2071, 5027, 1376966\\
\\
denn ihr Herz sinnt auf Gewalttat, und ihre Lippen reden M"uhsal.\\
\newpage 
{\bf -- 24.3}\\
\medskip \\
\begin{tabular}{rrrrrrrrp{120mm}}
WV&WK&WB&ABK&ABB&ABV&AnzB&TW&Zahlencode \textcolor{red}{$\boldsymbol{Grundtext}$} Umschrift $|$"Ubersetzung(en)\\
1.&16.&5047.&57.&19436.&1.&5&75&2\_8\_20\_40\_5 \textcolor{red}{\textcjheb{hmk.hb}} BCKMH $|$durch Weisheit\\
2.&17.&5048.&62.&19441.&6.&4&67&10\_2\_50\_5 \textcolor{red}{\textcjheb{hnby}} JBNH $|$(er (=es)) wird gebaut\\
3.&18.&5049.&66.&19445.&10.&3&412&2\_10\_400 \textcolor{red}{\textcjheb{tyb}} BJT $|$(ein) Haus\\
4.&19.&5050.&69.&19448.&13.&7&471&6\_2\_400\_2\_6\_50\_5 \textcolor{red}{\textcjheb{hnwbtbw}} WBTBWNH $|$und durch Verstand/und durch Einsicht\\
5.&20.&5051.&76.&19455.&20.&6&536&10\_400\_20\_6\_50\_50 \textcolor{red}{\textcjheb{nnwkty}} JTKWNN $|$wird es befestigt/er (=es) wird fest gegr"undet\\
\end{tabular}\medskip \\
Ende des Verses 24.3\\
Verse: 682, Buchstaben: 25, 81, 19460, Totalwerte: 1561, 6588, 1378527\\
\\
Durch Weisheit wird ein Haus gebaut, und durch Verstand wird es befestigt;\\
\newpage 
{\bf -- 24.4}\\
\medskip \\
\begin{tabular}{rrrrrrrrp{120mm}}
WV&WK&WB&ABK&ABB&ABV&AnzB&TW&Zahlencode \textcolor{red}{$\boldsymbol{Grundtext}$} Umschrift $|$"Ubersetzung(en)\\
1.&21.&5052.&82.&19461.&1.&5&482&6\_2\_4\_70\_400 \textcolor{red}{\textcjheb{t`dbw}} WBDaT $|$und durch Erkenntnis\\
2.&22.&5053.&87.&19466.&6.&5&262&8\_4\_200\_10\_40 \textcolor{red}{\textcjheb{myrd.h}} CDRJM $|$Kammern\\
3.&23.&5054.&92.&19471.&11.&5&87&10\_40\_30\_1\_6 \textcolor{red}{\textcjheb{w'lmy}} JMLAW $|$(sie) (werden) sich f"ullen\\
4.&24.&5055.&97.&19476.&16.&2&50&20\_30 \textcolor{red}{\textcjheb{lk}} KL $|$(mit) allerlei\\
5.&25.&5056.&99.&19478.&18.&3&61&5\_6\_50 \textcolor{red}{\textcjheb{nwh}} HWN $|$Gut\\
6.&26.&5057.&102.&19481.&21.&3&310&10\_100\_200 \textcolor{red}{\textcjheb{rqy}} JQR $|$kostbarem\\
7.&27.&5058.&105.&19484.&24.&5&176&6\_50\_70\_10\_40 \textcolor{red}{\textcjheb{my`nw}} WNaJM $|$und lieblichem\\
\end{tabular}\medskip \\
Ende des Verses 24.4\\
Verse: 683, Buchstaben: 28, 109, 19488, Totalwerte: 1428, 8016, 1379955\\
\\
und durch Erkenntnis f"ullen sich die Kammern mit allerlei kostbarem und lieblichem Gut.\\
\newpage 
{\bf -- 24.5}\\
\medskip \\
\begin{tabular}{rrrrrrrrp{120mm}}
WV&WK&WB&ABK&ABB&ABV&AnzB&TW&Zahlencode \textcolor{red}{$\boldsymbol{Grundtext}$} Umschrift $|$"Ubersetzung(en)\\
1.&28.&5059.&110.&19489.&1.&3&205&3\_2\_200 \textcolor{red}{\textcjheb{rbg}} GBR $|$ein Mann\\
2.&29.&5060.&113.&19492.&4.&3&68&8\_20\_40 \textcolor{red}{\textcjheb{mk.h}} CKM $|$weiser\\
3.&30.&5061.&116.&19495.&7.&4&85&2\_70\_6\_7 \textcolor{red}{\textcjheb{zw`b}} BaWZ $|$ist stark/(besteht) in der Kraft\\
4.&31.&5062.&120.&19499.&11.&4&317&6\_1\_10\_300 \textcolor{red}{\textcjheb{+sy'w}} WAJS $|$und (ein) Mann\\
5.&32.&5063.&124.&19503.&15.&3&474&4\_70\_400 \textcolor{red}{\textcjheb{t`d}} DaT $|$von Erkenntnis/(von) Verstand\\
6.&33.&5064.&127.&19506.&18.&4&171&40\_1\_40\_90 \textcolor{red}{\textcjheb{.sm'm}} MAM"s $|$befestigt/entfaltet\\
7.&34.&5065.&131.&19510.&22.&2&28&20\_8 \textcolor{red}{\textcjheb{.hk}} KC $|$seine Kraft/St"arke\\
\end{tabular}\medskip \\
Ende des Verses 24.5\\
Verse: 684, Buchstaben: 23, 132, 19511, Totalwerte: 1348, 9364, 1381303\\
\\
Ein weiser Mann ist stark, und ein Mann von Erkenntnis befestigt seine Kraft.\\
\newpage 
{\bf -- 24.6}\\
\medskip \\
\begin{tabular}{rrrrrrrrp{120mm}}
WV&WK&WB&ABK&ABB&ABV&AnzB&TW&Zahlencode \textcolor{red}{$\boldsymbol{Grundtext}$} Umschrift $|$"Ubersetzung(en)\\
1.&35.&5066.&133.&19512.&1.&2&30&20\_10 \textcolor{red}{\textcjheb{yk}} KJ $|$denn\\
2.&36.&5067.&135.&19514.&3.&7&848&2\_400\_8\_2\_30\_6\_400 \textcolor{red}{\textcjheb{twlb.htb}} BTCBLWT $|$mit weiser "Uberlegung/durch (kluge) Gedanken\\
3.&37.&5068.&142.&19521.&10.&4&775&400\_70\_300\_5 \textcolor{red}{\textcjheb{h+s`t}} TaSH $|$wirst du f"uhren/du musst f"uhren\\
4.&38.&5069.&146.&19525.&14.&2&50&30\_20 \textcolor{red}{\textcjheb{kl}} LK $|$gl"ucklichen/zu deinen Gunsten\\
5.&39.&5070.&148.&19527.&16.&5&123&40\_30\_8\_40\_5 \textcolor{red}{\textcjheb{hm.hlm}} MLCMH $|$Krieg/den Kampf\\
6.&40.&5071.&153.&19532.&21.&6&787&6\_400\_300\_6\_70\_5 \textcolor{red}{\textcjheb{h`w+stw}} WTSWaH $|$und (es) ist Heil/und Erfolg (besteht)\\
7.&41.&5072.&159.&19538.&27.&3&204&2\_200\_2 \textcolor{red}{\textcjheb{brb}} BRB $|$bei der Menge/in einer Vielzahl\\
8.&42.&5073.&162.&19541.&30.&4&176&10\_6\_70\_90 \textcolor{red}{\textcjheb{.s`wy}} JWa"s $|$Ratgeber/(von) Ratgebend(em)\\
\end{tabular}\medskip \\
Ende des Verses 24.6\\
Verse: 685, Buchstaben: 33, 165, 19544, Totalwerte: 2993, 12357, 1384296\\
\\
Denn mit weiser "Uberlegung wirst du gl"ucklich Krieg f"uhren, und bei der Ratgeber Menge ist Heil.\\
\newpage 
{\bf -- 24.7}\\
\medskip \\
\begin{tabular}{rrrrrrrrp{120mm}}
WV&WK&WB&ABK&ABB&ABV&AnzB&TW&Zahlencode \textcolor{red}{$\boldsymbol{Grundtext}$} Umschrift $|$"Ubersetzung(en)\\
1.&43.&5074.&166.&19545.&1.&5&647&200\_1\_40\_6\_400 \textcolor{red}{\textcjheb{twm'r}} RAMWT $|$zu hoch ist/zu hoch sind\\
2.&44.&5075.&171.&19550.&6.&5&77&30\_1\_6\_10\_30 \textcolor{red}{\textcjheb{lyw'l}} LAWJL $|$dem Toren\\
3.&45.&5076.&176.&19555.&11.&5&474&8\_20\_40\_6\_400 \textcolor{red}{\textcjheb{twmk.h}} CKMWT $|$Weisheit(en)\\
4.&46.&5077.&181.&19560.&16.&4&572&2\_300\_70\_200 \textcolor{red}{\textcjheb{r`+sb}} BSaR $|$im Tor\\
5.&47.&5078.&185.&19564.&20.&2&31&30\_1 \textcolor{red}{\textcjheb{'l}} LA $|$nicht\\
6.&48.&5079.&187.&19566.&22.&4&498&10\_80\_400\_8 \textcolor{red}{\textcjheb{.htpy}} JPTC $|$er tut auf\\
7.&49.&5080.&191.&19570.&26.&4&101&80\_10\_5\_6 \textcolor{red}{\textcjheb{whyp}} PJHW $|$seinen Mund\\
\end{tabular}\medskip \\
Ende des Verses 24.7\\
Verse: 686, Buchstaben: 29, 194, 19573, Totalwerte: 2400, 14757, 1386696\\
\\
Weisheit ist dem Narren zu hoch, im Tore tut er seinen Mund nicht auf.\\
\newpage 
{\bf -- 24.8}\\
\medskip \\
\begin{tabular}{rrrrrrrrp{120mm}}
WV&WK&WB&ABK&ABB&ABV&AnzB&TW&Zahlencode \textcolor{red}{$\boldsymbol{Grundtext}$} Umschrift $|$"Ubersetzung(en)\\
1.&50.&5081.&195.&19574.&1.&4&350&40\_8\_300\_2 \textcolor{red}{\textcjheb{b+s.hm}} MCSB $|$wer darauf sinnt/(einen) Sinnenden\\
2.&51.&5082.&199.&19578.&5.&4&305&30\_5\_200\_70 \textcolor{red}{\textcjheb{`rhl}} LHRa $|$B"oses zu tun/auf B"osestun\\
3.&52.&5083.&203.&19582.&9.&2&36&30\_6 \textcolor{red}{\textcjheb{wl}} LW $|$den\\
4.&53.&5084.&205.&19584.&11.&3&102&2\_70\_30 \textcolor{red}{\textcjheb{l`b}} BaL $|$einen Schmied/Herr\\
5.&54.&5085.&208.&19587.&14.&5&493&40\_7\_40\_6\_400 \textcolor{red}{\textcjheb{twmzm}} MZMWT $|$(der) R"anke\\
6.&55.&5086.&213.&19592.&19.&5&317&10\_100\_200\_1\_6 \textcolor{red}{\textcjheb{w'rqy}} JQRAW $|$man nennt\\
\end{tabular}\medskip \\
Ende des Verses 24.8\\
Verse: 687, Buchstaben: 23, 217, 19596, Totalwerte: 1603, 16360, 1388299\\
\\
Wer darauf sinnt, B"oses zu tun, den nennt man einen R"ankeschmied.\\
\newpage 
{\bf -- 24.9}\\
\medskip \\
\begin{tabular}{rrrrrrrrp{120mm}}
WV&WK&WB&ABK&ABB&ABV&AnzB&TW&Zahlencode \textcolor{red}{$\boldsymbol{Grundtext}$} Umschrift $|$"Ubersetzung(en)\\
1.&56.&5087.&218.&19597.&1.&3&447&7\_40\_400 \textcolor{red}{\textcjheb{tmz}} ZMT $|$das Vorhaben/die Schandtat\\
2.&57.&5088.&221.&19600.&4.&4&437&1\_6\_30\_400 \textcolor{red}{\textcjheb{tlw'}} AWLT $|$der Narrheit/der Torheit\\
3.&58.&5089.&225.&19604.&8.&4&418&8\_9\_1\_400 \textcolor{red}{\textcjheb{t'.t.h}} CtAT $|$(ist) (die) S"unde\\
4.&59.&5090.&229.&19608.&12.&6&884&6\_400\_6\_70\_2\_400 \textcolor{red}{\textcjheb{tb`wtw}} WTWaBT $|$und ein Gr"auel\\
5.&60.&5091.&235.&19614.&18.&4&75&30\_1\_4\_40 \textcolor{red}{\textcjheb{md'l}} LADM $|$(f"ur) den Menschen\\
6.&61.&5092.&239.&19618.&22.&2&120&30\_90 \textcolor{red}{\textcjheb{.sl}} L"s $|$(ist) der Sp"otter\\
\end{tabular}\medskip \\
Ende des Verses 24.9\\
Verse: 688, Buchstaben: 23, 240, 19619, Totalwerte: 2381, 18741, 1390680\\
\\
Das Vorhaben der Narrheit ist die S"unde, und der Sp"otter ist den Menschen ein Greuel.\\
\newpage 
{\bf -- 24.10}\\
\medskip \\
\begin{tabular}{rrrrrrrrp{120mm}}
WV&WK&WB&ABK&ABB&ABV&AnzB&TW&Zahlencode \textcolor{red}{$\boldsymbol{Grundtext}$} Umschrift $|$"Ubersetzung(en)\\
1.&62.&5093.&241.&19620.&1.&6&1095&5\_400\_200\_80\_10\_400 \textcolor{red}{\textcjheb{typrth}} HTRPJT $|$zeigst du dich schlaff\\
2.&63.&5094.&247.&19626.&7.&4&58&2\_10\_6\_40 \textcolor{red}{\textcjheb{mwyb}} BJWM $|$am Tag\\
3.&64.&5095.&251.&19630.&11.&3&295&90\_200\_5 \textcolor{red}{\textcjheb{hr.s}} "sRH $|$der Drangsal/(der) Bedr"angnis\\
4.&65.&5096.&254.&19633.&14.&2&290&90\_200 \textcolor{red}{\textcjheb{r.s}} "sR $|$so ist gering/versagt\\
5.&66.&5097.&256.&19635.&16.&4&53&20\_8\_20\_5 \textcolor{red}{\textcjheb{hk.hk}} KCKH $|$deine Kraft\\
\end{tabular}\medskip \\
Ende des Verses 24.10\\
Verse: 689, Buchstaben: 19, 259, 19638, Totalwerte: 1791, 20532, 1392471\\
\\
Zeigst du dich schlaff am Tage der Drangsal, so ist deine Kraft gering.\\
\newpage 
{\bf -- 24.11}\\
\medskip \\
\begin{tabular}{rrrrrrrrp{120mm}}
WV&WK&WB&ABK&ABB&ABV&AnzB&TW&Zahlencode \textcolor{red}{$\boldsymbol{Grundtext}$} Umschrift $|$"Ubersetzung(en)\\
1.&67.&5098.&260.&19639.&1.&3&125&5\_90\_30 \textcolor{red}{\textcjheb{l.sh}} H"sL $|$(er)rette\\
2.&68.&5099.&263.&19642.&4.&5&188&30\_100\_8\_10\_40 \textcolor{red}{\textcjheb{my.hql}} LQCJM $|$die geschleppt werden/die Gef"uhrten\\
3.&69.&5100.&268.&19647.&9.&4&476&30\_40\_6\_400 \textcolor{red}{\textcjheb{twml}} LMWT $|$zum Tod\\
4.&70.&5101.&272.&19651.&13.&5&105&6\_40\_9\_10\_40 \textcolor{red}{\textcjheb{my.tmw}} WMtJM $|$und die hinwanken/und Wankenden\\
5.&71.&5102.&277.&19656.&18.&4&238&30\_5\_200\_3 \textcolor{red}{\textcjheb{grhl}} LHRG $|$zur W"urgung/zur Hinmordung\\
6.&72.&5103.&281.&19660.&22.&2&41&1\_40 \textcolor{red}{\textcjheb{m'}} AM $|$/nicht\\
7.&73.&5104.&283.&19662.&24.&5&734&400\_8\_300\_6\_20 \textcolor{red}{\textcjheb{kw+s.ht}} TCSWK $|$o halte sie zur"uck/du sollst zur"uckhalten (Hilfe)\\
\end{tabular}\medskip \\
Ende des Verses 24.11\\
Verse: 690, Buchstaben: 28, 287, 19666, Totalwerte: 1907, 22439, 1394378\\
\\
Errette, die zum Tode geschleppt werden, und die zur W"urgung hinwanken, o halte sie zur"uck!\\
\newpage 
{\bf -- 24.12}\\
\medskip \\
\begin{tabular}{rrrrrrrrp{120mm}}
WV&WK&WB&ABK&ABB&ABV&AnzB&TW&Zahlencode \textcolor{red}{$\boldsymbol{Grundtext}$} Umschrift $|$"Ubersetzung(en)\\
1.&74.&5105.&288.&19667.&1.&2&30&20\_10 \textcolor{red}{\textcjheb{yk}} KJ $|$wenn\\
2.&75.&5106.&290.&19669.&3.&4&641&400\_1\_40\_200 \textcolor{red}{\textcjheb{rm't}} TAMR $|$du sprichst/du sagst\\
3.&76.&5107.&294.&19673.&7.&2&55&5\_50 \textcolor{red}{\textcjheb{nh}} HN $|$siehe\\
4.&77.&5108.&296.&19675.&9.&2&31&30\_1 \textcolor{red}{\textcjheb{'l}} LA $|$nichts\\
5.&78.&5109.&298.&19677.&11.&5&140&10\_4\_70\_50\_6 \textcolor{red}{\textcjheb{wn`dy}} JDaNW $|$wir wussten\\
6.&79.&5110.&303.&19682.&16.&2&12&7\_5 \textcolor{red}{\textcjheb{hz}} ZH $|$davon/(von) diesem\\
7.&80.&5111.&305.&19684.&18.&3&36&5\_30\_1 \textcolor{red}{\textcjheb{'lh}} HLA $|$(etwa) nicht\\
8.&81.&5112.&308.&19687.&21.&3&470&400\_20\_50 \textcolor{red}{\textcjheb{nkt}} TKN $|$wird der w"agt/der Pr"ufende\\
9.&82.&5113.&311.&19690.&24.&4&438&30\_2\_6\_400 \textcolor{red}{\textcjheb{twbl}} LBWT $|$(die) Herzen\\
10.&83.&5114.&315.&19694.&28.&3&12&5\_6\_1 \textcolor{red}{\textcjheb{'wh}} HWA $|$er\\
11.&84.&5115.&318.&19697.&31.&4&72&10\_2\_10\_50 \textcolor{red}{\textcjheb{nyby}} JBJN $|$es merken/(er) merkt (es)\\
12.&85.&5116.&322.&19701.&35.&4&346&6\_50\_90\_200 \textcolor{red}{\textcjheb{r.snw}} WN"sR $|$und der achthat/und der Beh"utende\\
13.&86.&5117.&326.&19705.&39.&4&450&50\_80\_300\_20 \textcolor{red}{\textcjheb{k+spn}} NPSK $|$(auf) deine Seele\\
14.&87.&5118.&330.&19709.&43.&3&12&5\_6\_1 \textcolor{red}{\textcjheb{'wh}} HWA $|$er\\
15.&88.&5119.&333.&19712.&46.&3&84&10\_4\_70 \textcolor{red}{\textcjheb{`dy}} JDa $|$es wissen/(es) wei"s (es)\\
16.&89.&5120.&336.&19715.&49.&5&323&6\_5\_300\_10\_2 \textcolor{red}{\textcjheb{by+shw}} WHSJB $|$und er wird vergelten/und er vergilt\\
17.&90.&5121.&341.&19720.&54.&4&75&30\_1\_4\_40 \textcolor{red}{\textcjheb{md'l}} LADM $|$dem Menschen\\
18.&91.&5122.&345.&19724.&58.&5&206&20\_80\_70\_30\_6 \textcolor{red}{\textcjheb{wl`pk}} KPaLW $|$nach seinem Tun/gem"a"s seinem Tun\\
\end{tabular}\medskip \\
Ende des Verses 24.12\\
Verse: 691, Buchstaben: 62, 349, 19728, Totalwerte: 3433, 25872, 1397811\\
\\
Wenn du sprichst: Siehe, wir wu"sten nichts davon-wird nicht er, der die Herzen w"agt, es merken? Und er, der auf deine Seele achthat, es wissen? Und er wird dem Menschen vergelten nach seinem Tun.\\
\newpage 
{\bf -- 24.13}\\
\medskip \\
\begin{tabular}{rrrrrrrrp{120mm}}
WV&WK&WB&ABK&ABB&ABV&AnzB&TW&Zahlencode \textcolor{red}{$\boldsymbol{Grundtext}$} Umschrift $|$"Ubersetzung(en)\\
1.&92.&5123.&350.&19729.&1.&3&51&1\_20\_30 \textcolor{red}{\textcjheb{lk'}} AKL $|$iss\\
2.&93.&5124.&353.&19732.&4.&3&62&2\_50\_10 \textcolor{red}{\textcjheb{ynb}} BNJ $|$mein Sohn\\
3.&94.&5125.&356.&19735.&7.&3&306&4\_2\_300 \textcolor{red}{\textcjheb{+sbd}} DBS $|$Honig\\
4.&95.&5126.&359.&19738.&10.&2&30&20\_10 \textcolor{red}{\textcjheb{yk}} KJ $|$denn\\
5.&96.&5127.&361.&19740.&12.&3&17&9\_6\_2 \textcolor{red}{\textcjheb{bw.t}} tWB $|$er ist gut\\
6.&97.&5128.&364.&19743.&15.&4&536&6\_50\_80\_400 \textcolor{red}{\textcjheb{tpnw}} WNPT $|$und (Honig)Seim\\
7.&98.&5129.&368.&19747.&19.&4&546&40\_400\_6\_100 \textcolor{red}{\textcjheb{qwtm}} MTWQ $|$(ist) s"u"s(en)\\
8.&99.&5130.&372.&19751.&23.&2&100&70\_30 \textcolor{red}{\textcjheb{l`}} aL $|$/f"ur\\
9.&100.&5131.&374.&19753.&25.&3&48&8\_20\_20 \textcolor{red}{\textcjheb{kk.h}} CKK $|$deinem Gaumen/deinen Gaumen\\
\end{tabular}\medskip \\
Ende des Verses 24.13\\
Verse: 692, Buchstaben: 27, 376, 19755, Totalwerte: 1696, 27568, 1399507\\
\\
I"s Honig, mein Sohn, denn er ist gut, und Honigseim ist deinem Gaumen s"u"s.\\
\newpage 
{\bf -- 24.14}\\
\medskip \\
\begin{tabular}{rrrrrrrrp{120mm}}
WV&WK&WB&ABK&ABB&ABV&AnzB&TW&Zahlencode \textcolor{red}{$\boldsymbol{Grundtext}$} Umschrift $|$"Ubersetzung(en)\\
1.&101.&5132.&377.&19756.&1.&2&70&20\_50 \textcolor{red}{\textcjheb{nk}} KN $|$(eben)so\\
2.&102.&5133.&379.&19758.&3.&3&79&4\_70\_5 \textcolor{red}{\textcjheb{h`d}} DaH $|$betrachte/erlerne\\
3.&103.&5134.&382.&19761.&6.&4&73&8\_20\_40\_5 \textcolor{red}{\textcjheb{hmk.h}} CKMH $|$(die) Weisheit\\
4.&104.&5135.&386.&19765.&10.&5&480&30\_50\_80\_300\_20 \textcolor{red}{\textcjheb{k+spnl}} LNPSK $|$f"ur deine Seele\\
5.&105.&5136.&391.&19770.&15.&2&41&1\_40 \textcolor{red}{\textcjheb{m'}} AM $|$wenn\\
6.&106.&5137.&393.&19772.&17.&4&531&40\_90\_1\_400 \textcolor{red}{\textcjheb{t'.sm}} M"sAT $|$du sie gefunden hast/du sie erworben\\
7.&107.&5138.&397.&19776.&21.&3&316&6\_10\_300 \textcolor{red}{\textcjheb{+syw}} WJS $|$so gibt es/und es gibt\\
8.&108.&5139.&400.&19779.&24.&5&619&1\_8\_200\_10\_400 \textcolor{red}{\textcjheb{tyr.h'}} ACRJT $|$eine Zukunft/ein (gutes) Ende\\
9.&109.&5140.&405.&19784.&29.&6&932&6\_400\_100\_6\_400\_20 \textcolor{red}{\textcjheb{ktwqtw}} WTQWTK $|$und deine Hoffnung\\
10.&110.&5141.&411.&19790.&35.&2&31&30\_1 \textcolor{red}{\textcjheb{'l}} LA $|$nicht\\
11.&111.&5142.&413.&19792.&37.&4&1020&400\_20\_200\_400 \textcolor{red}{\textcjheb{trkt}} TKRT $|$wird vernichtet werden/(sie) wird zerst"ort\\
\end{tabular}\medskip \\
Ende des Verses 24.14\\
Verse: 693, Buchstaben: 40, 416, 19795, Totalwerte: 4192, 31760, 1403699\\
\\
Ebenso betrachte die Weisheit f"ur deine Seele: wenn du sie gefunden hast, so gibt es eine Zukunft, und deine Hoffnung wird nicht vernichtet werden.\\
\newpage 
{\bf -- 24.15}\\
\medskip \\
\begin{tabular}{rrrrrrrrp{120mm}}
WV&WK&WB&ABK&ABB&ABV&AnzB&TW&Zahlencode \textcolor{red}{$\boldsymbol{Grundtext}$} Umschrift $|$"Ubersetzung(en)\\
1.&112.&5143.&417.&19796.&1.&2&31&1\_30 \textcolor{red}{\textcjheb{l'}} AL $|$nicht\\
2.&113.&5144.&419.&19798.&3.&4&603&400\_1\_200\_2 \textcolor{red}{\textcjheb{br't}} TARB $|$laure/du sollst belauern\\
3.&114.&5145.&423.&19802.&7.&3&570&200\_300\_70 \textcolor{red}{\textcjheb{`+sr}} RSa $|$Gesetzloser/(als) Frevler\\
4.&115.&5146.&426.&19805.&10.&4&91&30\_50\_6\_5 \textcolor{red}{\textcjheb{hwnl}} LNWH $|$auf die Wohnung/die Wohnstatt\\
5.&116.&5147.&430.&19809.&14.&4&204&90\_4\_10\_100 \textcolor{red}{\textcjheb{qyd.s}} "sDJQ $|$(des) Gerechten\\
6.&117.&5148.&434.&19813.&18.&2&31&1\_30 \textcolor{red}{\textcjheb{l'}} AL $|$nicht\\
7.&118.&5149.&436.&19815.&20.&4&708&400\_300\_4\_4 \textcolor{red}{\textcjheb{dd+st}} TSDD $|$zerst"ore/du sollst verw"usten\\
8.&119.&5150.&440.&19819.&24.&4&298&200\_2\_90\_6 \textcolor{red}{\textcjheb{w.sbr}} RB"sW $|$seine Lagerst"atte/seinen Lagerplatz\\
\end{tabular}\medskip \\
Ende des Verses 24.15\\
Verse: 694, Buchstaben: 27, 443, 19822, Totalwerte: 2536, 34296, 1406235\\
\\
Laure nicht, Gesetzloser, auf die Wohnung des Gerechten, zerst"ore nicht seine Lagerst"atte.\\
\newpage 
{\bf -- 24.16}\\
\medskip \\
\begin{tabular}{rrrrrrrrp{120mm}}
WV&WK&WB&ABK&ABB&ABV&AnzB&TW&Zahlencode \textcolor{red}{$\boldsymbol{Grundtext}$} Umschrift $|$"Ubersetzung(en)\\
1.&120.&5151.&444.&19823.&1.&2&30&20\_10 \textcolor{red}{\textcjheb{yk}} KJ $|$denn/wenn\\
2.&121.&5152.&446.&19825.&3.&3&372&300\_2\_70 \textcolor{red}{\textcjheb{`b+s}} SBa $|$sieben(mal)\\
3.&122.&5153.&449.&19828.&6.&4&126&10\_80\_6\_30 \textcolor{red}{\textcjheb{lwpy}} JPWL $|$(er) f"allt\\
4.&123.&5154.&453.&19832.&10.&4&204&90\_4\_10\_100 \textcolor{red}{\textcjheb{qyd.s}} "sDJQ $|$der Gerechte/(ein) Gerechter\\
5.&124.&5155.&457.&19836.&14.&3&146&6\_100\_40 \textcolor{red}{\textcjheb{mqw}} WQM $|$und (er) steht (wieder) auf\\
6.&125.&5156.&460.&19839.&17.&6&626&6\_200\_300\_70\_10\_40 \textcolor{red}{\textcjheb{my`+srw}} WRSaJM $|$aber die Gesetzlosen/und Frevler\\
7.&126.&5157.&466.&19845.&23.&5&366&10\_20\_300\_30\_6 \textcolor{red}{\textcjheb{wl+sky}} JKSLW $|$st"urzen nieder/(sie) straucheln\\
8.&127.&5158.&471.&19850.&28.&4&277&2\_200\_70\_5 \textcolor{red}{\textcjheb{h`rb}} BRaH $|$im Ungl"uck/ins Ungl"uck\\
\end{tabular}\medskip \\
Ende des Verses 24.16\\
Verse: 695, Buchstaben: 31, 474, 19853, Totalwerte: 2147, 36443, 1408382\\
\\
Denn der Gerechte f"allt siebenmal und steht wieder auf, aber die Gesetzlosen st"urzen nieder im Ungl"uck.\\
\newpage 
{\bf -- 24.17}\\
\medskip \\
\begin{tabular}{rrrrrrrrp{120mm}}
WV&WK&WB&ABK&ABB&ABV&AnzB&TW&Zahlencode \textcolor{red}{$\boldsymbol{Grundtext}$} Umschrift $|$"Ubersetzung(en)\\
1.&128.&5159.&475.&19854.&1.&4&162&2\_50\_80\_30 \textcolor{red}{\textcjheb{lpnb}} BNPL $|$"uber den Fall/wenn f"allt\\
2.&129.&5160.&479.&19858.&5.&6&49&1\_6\_10\_2\_10\_20 \textcolor{red}{\textcjheb{kybyw'}} AWJBJK $|$dein(es) Feind(es)\\
3.&130.&5161.&485.&19864.&11.&2&31&1\_30 \textcolor{red}{\textcjheb{l'}} AL $|$nicht\\
4.&131.&5162.&487.&19866.&13.&4&748&400\_300\_40\_8 \textcolor{red}{\textcjheb{.hm+st}} TSMC $|$(du sollst) dich freue(n)\\
5.&132.&5163.&491.&19870.&17.&6&364&6\_2\_20\_300\_30\_6 \textcolor{red}{\textcjheb{wl+skbw}} WBKSLW $|$und "uber seinen Sturz/und bei seinem Straucheln\\
6.&133.&5164.&497.&19876.&23.&2&31&1\_30 \textcolor{red}{\textcjheb{l'}} AL $|$nicht\\
7.&134.&5165.&499.&19878.&25.&3&43&10\_3\_30 \textcolor{red}{\textcjheb{lgy}} JGL $|$frohlocke/er (=es) soll jubeln\\
8.&135.&5166.&502.&19881.&28.&3&52&30\_2\_20 \textcolor{red}{\textcjheb{kbl}} LBK $|$dein Herz\\
\end{tabular}\medskip \\
Ende des Verses 24.17\\
Verse: 696, Buchstaben: 30, 504, 19883, Totalwerte: 1480, 37923, 1409862\\
\\
Freue dich nicht "uber den Fall deines Feindes, und dein Herz frohlocke nicht "uber seinen Sturz:\\
\newpage 
{\bf -- 24.18}\\
\medskip \\
\begin{tabular}{rrrrrrrrp{120mm}}
WV&WK&WB&ABK&ABB&ABV&AnzB&TW&Zahlencode \textcolor{red}{$\boldsymbol{Grundtext}$} Umschrift $|$"Ubersetzung(en)\\
1.&136.&5167.&505.&19884.&1.&2&130&80\_50 \textcolor{red}{\textcjheb{np}} PN $|$damit nicht/dass nicht\\
2.&137.&5168.&507.&19886.&3.&4&216&10\_200\_1\_5 \textcolor{red}{\textcjheb{h'ry}} JRAH $|$er (=es) sehe\\
3.&138.&5169.&511.&19890.&7.&4&26&10\_5\_6\_5 \textcolor{red}{\textcjheb{hwhy}} JHWH $|$Jahwe\\
4.&139.&5170.&515.&19894.&11.&3&276&6\_200\_70 \textcolor{red}{\textcjheb{`rw}} WRa $|$und es b"ose sei/und (es ist) schlecht\\
5.&140.&5171.&518.&19897.&14.&6&148&2\_70\_10\_50\_10\_6 \textcolor{red}{\textcjheb{wyny`b}} BaJNJW $|$in seinen Augen\\
6.&141.&5172.&524.&19903.&20.&5&323&6\_5\_300\_10\_2 \textcolor{red}{\textcjheb{by+shw}} WHSJB $|$und er abwende/und es macht abkehren\\
7.&142.&5173.&529.&19908.&25.&5&156&40\_70\_30\_10\_6 \textcolor{red}{\textcjheb{wyl`m}} MaLJW $|$von ihm\\
8.&143.&5174.&534.&19913.&30.&3&87&1\_80\_6 \textcolor{red}{\textcjheb{wp'}} APW $|$seinen Zorn\\
\end{tabular}\medskip \\
Ende des Verses 24.18\\
Verse: 697, Buchstaben: 32, 536, 19915, Totalwerte: 1362, 39285, 1411224\\
\\
damit Jahwe es nicht sehe, und es b"ose sei in seinen Augen, und er seinen Zorn von ihm abwende.\\
\newpage 
{\bf -- 24.19}\\
\medskip \\
\begin{tabular}{rrrrrrrrp{120mm}}
WV&WK&WB&ABK&ABB&ABV&AnzB&TW&Zahlencode \textcolor{red}{$\boldsymbol{Grundtext}$} Umschrift $|$"Ubersetzung(en)\\
1.&144.&5175.&537.&19916.&1.&2&31&1\_30 \textcolor{red}{\textcjheb{l'}} AL $|$nicht\\
2.&145.&5176.&539.&19918.&3.&4&1008&400\_400\_8\_200 \textcolor{red}{\textcjheb{r.htt}} TTCR $|$erz"urne dich/du sollst dich erhitzen\\
3.&146.&5177.&543.&19922.&7.&6&362&2\_40\_200\_70\_10\_40 \textcolor{red}{\textcjheb{my`rmb}} BMRaJM $|$"uber die "Ubelt"ater\\
4.&147.&5178.&549.&19928.&13.&2&31&1\_30 \textcolor{red}{\textcjheb{l'}} AL $|$nicht\\
5.&148.&5179.&551.&19930.&15.&4&551&400\_100\_50\_1 \textcolor{red}{\textcjheb{'nqt}} TQNA $|$beneide/du sollst dich ereifern\\
6.&149.&5180.&555.&19934.&19.&6&622&2\_200\_300\_70\_10\_40 \textcolor{red}{\textcjheb{my`+srb}} BRSaJM $|$die Gesetzlosen/"uber die Frevler\\
\end{tabular}\medskip \\
Ende des Verses 24.19\\
Verse: 698, Buchstaben: 24, 560, 19939, Totalwerte: 2605, 41890, 1413829\\
\\
Erz"urne dich nicht "uber die "Ubelt"ater, beneide nicht die Gesetzlosen;\\
\newpage 
{\bf -- 24.20}\\
\medskip \\
\begin{tabular}{rrrrrrrrp{120mm}}
WV&WK&WB&ABK&ABB&ABV&AnzB&TW&Zahlencode \textcolor{red}{$\boldsymbol{Grundtext}$} Umschrift $|$"Ubersetzung(en)\\
1.&150.&5181.&561.&19940.&1.&2&30&20\_10 \textcolor{red}{\textcjheb{yk}} KJ $|$denn\\
2.&151.&5182.&563.&19942.&3.&2&31&30\_1 \textcolor{red}{\textcjheb{'l}} LA $|$nicht\\
3.&152.&5183.&565.&19944.&5.&4&420&400\_5\_10\_5 \textcolor{red}{\textcjheb{hyht}} THJH $|$(sie (=es)) wird sein\\
4.&153.&5184.&569.&19948.&9.&5&619&1\_8\_200\_10\_400 \textcolor{red}{\textcjheb{tyr.h'}} ACRJT $|$eine Zukunft/ein (gutes) Ende\\
5.&154.&5185.&574.&19953.&14.&3&300&30\_200\_70 \textcolor{red}{\textcjheb{`rl}} LRa $|$f"ur den B"osen/f"ur den Frevler\\
6.&155.&5186.&577.&19956.&17.&2&250&50\_200 \textcolor{red}{\textcjheb{rn}} NR $|$die Leuchte\\
7.&156.&5187.&579.&19958.&19.&5&620&200\_300\_70\_10\_40 \textcolor{red}{\textcjheb{my`+sr}} RSaJM $|$der Gesetzlosen/(der) Frevler\\
8.&157.&5188.&584.&19963.&24.&4&104&10\_4\_70\_20 \textcolor{red}{\textcjheb{k`dy}} JDaK $|$wird erl"oschen/er (=sie) erlischt\\
\end{tabular}\medskip \\
Ende des Verses 24.20\\
Verse: 699, Buchstaben: 27, 587, 19966, Totalwerte: 2374, 44264, 1416203\\
\\
denn f"ur den B"osen wird keine Zukunft sein, die Leuchte der Gesetzlosen wird erl"oschen.\\
\newpage 
{\bf -- 24.21}\\
\medskip \\
\begin{tabular}{rrrrrrrrp{120mm}}
WV&WK&WB&ABK&ABB&ABV&AnzB&TW&Zahlencode \textcolor{red}{$\boldsymbol{Grundtext}$} Umschrift $|$"Ubersetzung(en)\\
1.&158.&5189.&588.&19967.&1.&3&211&10\_200\_1 \textcolor{red}{\textcjheb{'ry}} JRA $|$f"urchte\\
2.&159.&5190.&591.&19970.&4.&2&401&1\_400 \textcolor{red}{\textcjheb{t'}} AT $|$**\\
3.&160.&5191.&593.&19972.&6.&4&26&10\_5\_6\_5 \textcolor{red}{\textcjheb{hwhy}} JHWH $|$Jahwe\\
4.&161.&5192.&597.&19976.&10.&3&62&2\_50\_10 \textcolor{red}{\textcjheb{ynb}} BNJ $|$mein Sohn\\
5.&162.&5193.&600.&19979.&13.&4&96&6\_40\_30\_20 \textcolor{red}{\textcjheb{klmw}} WMLK $|$und den K"onig\\
6.&163.&5194.&604.&19983.&17.&2&110&70\_40 \textcolor{red}{\textcjheb{m`}} aM $|$mit\\
7.&164.&5195.&606.&19985.&19.&5&406&300\_6\_50\_10\_40 \textcolor{red}{\textcjheb{mynw+s}} SWNJM $|$Aufr"uhrern/Andersdenkenden\\
8.&165.&5196.&611.&19990.&24.&2&31&1\_30 \textcolor{red}{\textcjheb{l'}} AL $|$nicht\\
9.&166.&5197.&613.&19992.&26.&5&1072&400\_400\_70\_200\_2 \textcolor{red}{\textcjheb{br`tt}} TTaRB $|$lass dich ein/du sollst dich einlassen\\
\end{tabular}\medskip \\
Ende des Verses 24.21\\
Verse: 700, Buchstaben: 30, 617, 19996, Totalwerte: 2415, 46679, 1418618\\
\\
Mein Sohn, f"urchte Jahwe und den K"onig; mit Aufr"uhrern la"s dich nicht ein.\\
\newpage 
{\bf -- 24.22}\\
\medskip \\
\begin{tabular}{rrrrrrrrp{120mm}}
WV&WK&WB&ABK&ABB&ABV&AnzB&TW&Zahlencode \textcolor{red}{$\boldsymbol{Grundtext}$} Umschrift $|$"Ubersetzung(en)\\
1.&167.&5198.&618.&19997.&1.&2&30&20\_10 \textcolor{red}{\textcjheb{yk}} KJ $|$denn\\
2.&168.&5199.&620.&19999.&3.&4&521&80\_400\_1\_40 \textcolor{red}{\textcjheb{m'tp}} PTAM $|$pl"otzlich\\
3.&169.&5200.&624.&20003.&7.&4&156&10\_100\_6\_40 \textcolor{red}{\textcjheb{mwqy}} JQWM $|$(er (=es)) erhebt sich\\
4.&170.&5201.&628.&20007.&11.&4&55&1\_10\_4\_40 \textcolor{red}{\textcjheb{mdy'}} AJDM $|$ihr Verderben\\
5.&171.&5202.&632.&20011.&15.&4&100&6\_80\_10\_4 \textcolor{red}{\textcjheb{dypw}} WPJD $|$und (der) Untergang/und Unheil\\
6.&172.&5203.&636.&20015.&19.&5&405&300\_50\_10\_5\_40 \textcolor{red}{\textcjheb{mhyn+s}} SNJHM $|$ihrer beiden/von ihnen beiden\\
7.&173.&5204.&641.&20020.&24.&2&50&40\_10 \textcolor{red}{\textcjheb{ym}} MJ $|$wer\\
8.&174.&5205.&643.&20022.&26.&4&90&10\_6\_4\_70 \textcolor{red}{\textcjheb{`dwy}} JWDa $|$wei"s ihn/(ist) wissend\\
\end{tabular}\medskip \\
Ende des Verses 24.22\\
Verse: 701, Buchstaben: 29, 646, 20025, Totalwerte: 1407, 48086, 1420025\\
\\
Denn pl"otzlich erhebt sich ihr Verderben; und ihrer beider Untergang, wer wei"s ihn?\\
\newpage 
{\bf -- 24.23}\\
\medskip \\
\begin{tabular}{rrrrrrrrp{120mm}}
WV&WK&WB&ABK&ABB&ABV&AnzB&TW&Zahlencode \textcolor{red}{$\boldsymbol{Grundtext}$} Umschrift $|$"Ubersetzung(en)\\
1.&175.&5206.&647.&20026.&1.&2&43&3\_40 \textcolor{red}{\textcjheb{mg}} GM $|$auch\\
2.&176.&5207.&649.&20028.&3.&3&36&1\_30\_5 \textcolor{red}{\textcjheb{hl'}} ALH $|$diese sind/dies ist\\
3.&177.&5208.&652.&20031.&6.&6&148&30\_8\_20\_40\_10\_40 \textcolor{red}{\textcjheb{mymk.hl}} LCKMJM $|$von (den) Weisen\\
4.&178.&5209.&658.&20037.&12.&3&225&5\_20\_200 \textcolor{red}{\textcjheb{rkh}} HKR $|$ansehen/betrachten\\
5.&179.&5210.&661.&20040.&15.&4&180&80\_50\_10\_40 \textcolor{red}{\textcjheb{mynp}} PNJM $|$die Person/Gesichter\\
6.&180.&5211.&665.&20044.&19.&5&431&2\_40\_300\_80\_9 \textcolor{red}{\textcjheb{.tp+smb}} BMSPt $|$im Gericht\\
7.&181.&5212.&670.&20049.&24.&2&32&2\_30 \textcolor{red}{\textcjheb{lb}} BL $|$nicht\\
8.&182.&5213.&672.&20051.&26.&3&17&9\_6\_2 \textcolor{red}{\textcjheb{bw.t}} tWB $|$gut (ist)\\
\end{tabular}\medskip \\
Ende des Verses 24.23\\
Verse: 702, Buchstaben: 28, 674, 20053, Totalwerte: 1112, 49198, 1421137\\
\\
Auch diese sind von den Weisen: Die Person ansehen im Gericht ist nicht gut.\\
\newpage 
{\bf -- 24.24}\\
\medskip \\
\begin{tabular}{rrrrrrrrp{120mm}}
WV&WK&WB&ABK&ABB&ABV&AnzB&TW&Zahlencode \textcolor{red}{$\boldsymbol{Grundtext}$} Umschrift $|$"Ubersetzung(en)\\
1.&183.&5214.&675.&20054.&1.&3&241&1\_40\_200 \textcolor{red}{\textcjheb{rm'}} AMR $|$wer spricht\\
2.&184.&5215.&678.&20057.&4.&4&600&30\_200\_300\_70 \textcolor{red}{\textcjheb{`+srl}} LRSa $|$zu dem Gesetzlosen/zu dem Schuldigen\\
3.&185.&5216.&682.&20061.&8.&4&204&90\_4\_10\_100 \textcolor{red}{\textcjheb{qyd.s}} "sDJQ $|$gerecht\\
4.&186.&5217.&686.&20065.&12.&3&406&1\_400\_5 \textcolor{red}{\textcjheb{ht'}} ATH $|$(bist) du\\
5.&187.&5218.&689.&20068.&15.&5&123&10\_100\_2\_5\_6 \textcolor{red}{\textcjheb{whbqy}} JQBHW $|$den verfluchen/sie (=es) sollen verfluchen ihn\\
6.&188.&5219.&694.&20073.&20.&4&160&70\_40\_10\_40 \textcolor{red}{\textcjheb{mym`}} aMJM $|$die V"olker\\
7.&189.&5220.&698.&20077.&24.&7&144&10\_7\_70\_40\_6\_5\_6 \textcolor{red}{\textcjheb{whwm`zy}} JZaMWHW $|$den verw"unschen/sie (=es) verw"unschen ihn\\
8.&190.&5221.&705.&20084.&31.&5&121&30\_1\_40\_10\_40 \textcolor{red}{\textcjheb{mym'l}} LAMJM $|$die V"olkerschaften/Nationen\\
\end{tabular}\medskip \\
Ende des Verses 24.24\\
Verse: 703, Buchstaben: 35, 709, 20088, Totalwerte: 1999, 51197, 1423136\\
\\
Wer zu dem Gesetzlosen spricht: Du bist gerecht, den verfluchen die V"olker, den verw"unschen die V"olkerschaften;\\
\newpage 
{\bf -- 24.25}\\
\medskip \\
\begin{tabular}{rrrrrrrrp{120mm}}
WV&WK&WB&ABK&ABB&ABV&AnzB&TW&Zahlencode \textcolor{red}{$\boldsymbol{Grundtext}$} Umschrift $|$"Ubersetzung(en)\\
1.&191.&5222.&710.&20089.&1.&9&170&6\_30\_40\_6\_20\_10\_8\_10\_40 \textcolor{red}{\textcjheb{my.hykwmlw}} WLMWKJCJM $|$denen aber welche gerecht entscheiden/und den Zurechtweisenden\\
2.&192.&5223.&719.&20098.&10.&4&170&10\_50\_70\_40 \textcolor{red}{\textcjheb{m`ny}} JNaM $|$er (=es) geht wohl\\
3.&193.&5224.&723.&20102.&14.&6&161&6\_70\_30\_10\_5\_40 \textcolor{red}{\textcjheb{mhyl`w}} WaLJHM $|$und "uber sie\\
4.&194.&5225.&729.&20108.&20.&4&409&400\_2\_6\_1 \textcolor{red}{\textcjheb{'wbt}} TBWA $|$(sie (=es)) kommt\\
5.&195.&5226.&733.&20112.&24.&4&622&2\_200\_20\_400 \textcolor{red}{\textcjheb{tkrb}} BRKT $|$(die) Segnung\\
6.&196.&5227.&737.&20116.&28.&3&17&9\_6\_2 \textcolor{red}{\textcjheb{bw.t}} tWB $|$des Guten\\
\end{tabular}\medskip \\
Ende des Verses 24.25\\
Verse: 704, Buchstaben: 30, 739, 20118, Totalwerte: 1549, 52746, 1424685\\
\\
denen aber, welche gerecht entscheiden, geht es wohl, und "uber sie kommt Segnung des Guten.\\
\newpage 
{\bf -- 24.26}\\
\medskip \\
\begin{tabular}{rrrrrrrrp{120mm}}
WV&WK&WB&ABK&ABB&ABV&AnzB&TW&Zahlencode \textcolor{red}{$\boldsymbol{Grundtext}$} Umschrift $|$"Ubersetzung(en)\\
1.&197.&5228.&740.&20119.&1.&5&830&300\_80\_400\_10\_40 \textcolor{red}{\textcjheb{mytp+s}} SPTJM $|$(die) Lippen\\
2.&198.&5229.&745.&20124.&6.&3&410&10\_300\_100 \textcolor{red}{\textcjheb{q+sy}} JSQ $|$(er) k"usst\\
3.&199.&5230.&748.&20127.&9.&4&352&40\_300\_10\_2 \textcolor{red}{\textcjheb{by+sm}} MSJB $|$wer Antwort gibt/einem Antwortenden\\
4.&200.&5231.&752.&20131.&13.&5&256&4\_2\_200\_10\_40 \textcolor{red}{\textcjheb{myrbd}} DBRJM $|$/(mit) Worte(n)\\
5.&201.&5232.&757.&20136.&18.&5&128&50\_20\_8\_10\_40 \textcolor{red}{\textcjheb{my.hkn}} NKCJM $|$richtige/geraden\\
\end{tabular}\medskip \\
Ende des Verses 24.26\\
Verse: 705, Buchstaben: 22, 761, 20140, Totalwerte: 1976, 54722, 1426661\\
\\
Die Lippen k"u"st, wer richtige Antwort gibt.\\
\newpage 
{\bf -- 24.27}\\
\medskip \\
\begin{tabular}{rrrrrrrrp{120mm}}
WV&WK&WB&ABK&ABB&ABV&AnzB&TW&Zahlencode \textcolor{red}{$\boldsymbol{Grundtext}$} Umschrift $|$"Ubersetzung(en)\\
1.&202.&5233.&762.&20141.&1.&3&75&5\_20\_50 \textcolor{red}{\textcjheb{nkh}} HKN $|$besorge/verrichte\\
2.&203.&5234.&765.&20144.&4.&4&106&2\_8\_6\_90 \textcolor{red}{\textcjheb{.sw.hb}} BCW"s $|$drau"sen/im Drau"sen\\
3.&204.&5235.&769.&20148.&8.&6&511&40\_30\_1\_20\_400\_20 \textcolor{red}{\textcjheb{ktk'lm}} MLAKTK $|$deine Arbeit\\
4.&205.&5236.&775.&20154.&14.&5&485&6\_70\_400\_4\_5 \textcolor{red}{\textcjheb{hdt`w}} WaTDH $|$und bestelle sie\\
5.&206.&5237.&780.&20159.&19.&4&311&2\_300\_4\_5 \textcolor{red}{\textcjheb{hd+sb}} BSDH $|$auf dem Feld\\
6.&207.&5238.&784.&20163.&23.&2&50&30\_20 \textcolor{red}{\textcjheb{kl}} LK $|$dir/f"ur dich\\
7.&208.&5239.&786.&20165.&25.&3&209&1\_8\_200 \textcolor{red}{\textcjheb{r.h'}} ACR $|$hernach dann/nachher\\
8.&209.&5240.&789.&20168.&28.&5&468&6\_2\_50\_10\_400 \textcolor{red}{\textcjheb{tynbw}} WBNJT $|$(und) du magst bauen\\
9.&210.&5241.&794.&20173.&33.&4&432&2\_10\_400\_20 \textcolor{red}{\textcjheb{ktyb}} BJTK $|$dein Haus\\
\end{tabular}\medskip \\
Ende des Verses 24.27\\
Verse: 706, Buchstaben: 36, 797, 20176, Totalwerte: 2647, 57369, 1429308\\
\\
Besorge drau"sen deine Arbeit und bestelle sie dir auf dem Felde; hernach magst du dann dein Haus bauen.\\
\newpage 
{\bf -- 24.28}\\
\medskip \\
\begin{tabular}{rrrrrrrrp{120mm}}
WV&WK&WB&ABK&ABB&ABV&AnzB&TW&Zahlencode \textcolor{red}{$\boldsymbol{Grundtext}$} Umschrift $|$"Ubersetzung(en)\\
1.&211.&5242.&798.&20177.&1.&2&31&1\_30 \textcolor{red}{\textcjheb{l'}} AL $|$nicht\\
2.&212.&5243.&800.&20179.&3.&3&415&400\_5\_10 \textcolor{red}{\textcjheb{yht}} THJ $|$werde/du sollst sein\\
3.&213.&5244.&803.&20182.&6.&2&74&70\_4 \textcolor{red}{\textcjheb{d`}} aD $|$(ein) Zeuge\\
4.&214.&5245.&805.&20184.&8.&3&98&8\_50\_40 \textcolor{red}{\textcjheb{mn.h}} CNM $|$ohne Ursache/grundlos\\
5.&215.&5246.&808.&20187.&11.&4&292&2\_200\_70\_20 \textcolor{red}{\textcjheb{k`rb}} BRaK $|$wider deinen N"achsten/gegen deinen N"achsten\\
6.&216.&5247.&812.&20191.&15.&6&901&6\_5\_80\_400\_10\_400 \textcolor{red}{\textcjheb{tytphw}} WHPTJT $|$wolltest du denn t"auschen/und etwa du t"auschst\\
7.&217.&5248.&818.&20197.&21.&6&812&2\_300\_80\_400\_10\_20 \textcolor{red}{\textcjheb{kytp+sb}} BSPTJK $|$mit deinen Lippen\\
\end{tabular}\medskip \\
Ende des Verses 24.28\\
Verse: 707, Buchstaben: 26, 823, 20202, Totalwerte: 2623, 59992, 1431931\\
\\
Werde nicht ohne Ursache Zeuge wider deinen N"achsten; wolltest du denn t"auschen mit deinen Lippen?\\
\newpage 
{\bf -- 24.29}\\
\medskip \\
\begin{tabular}{rrrrrrrrp{120mm}}
WV&WK&WB&ABK&ABB&ABV&AnzB&TW&Zahlencode \textcolor{red}{$\boldsymbol{Grundtext}$} Umschrift $|$"Ubersetzung(en)\\
1.&218.&5249.&824.&20203.&1.&2&31&1\_30 \textcolor{red}{\textcjheb{l'}} AL $|$nicht\\
2.&219.&5250.&826.&20205.&3.&4&641&400\_1\_40\_200 \textcolor{red}{\textcjheb{rm't}} TAMR $|$sprich/sollst du sagen\\
3.&220.&5251.&830.&20209.&7.&4&521&20\_1\_300\_200 \textcolor{red}{\textcjheb{r+s'k}} KASR $|$(so) wie\\
4.&221.&5252.&834.&20213.&11.&3&375&70\_300\_5 \textcolor{red}{\textcjheb{h+s`}} aSH $|$er getan (hat)\\
5.&222.&5253.&837.&20216.&14.&2&40&30\_10 \textcolor{red}{\textcjheb{yl}} LJ $|$mir\\
6.&223.&5254.&839.&20218.&16.&2&70&20\_50 \textcolor{red}{\textcjheb{nk}} KN $|$so\\
7.&224.&5255.&841.&20220.&18.&4&376&1\_70\_300\_5 \textcolor{red}{\textcjheb{h+s`'}} AaSH $|$will ich tun\\
8.&225.&5256.&845.&20224.&22.&2&36&30\_6 \textcolor{red}{\textcjheb{wl}} LW $|$ihm\\
9.&226.&5257.&847.&20226.&24.&4&313&1\_300\_10\_2 \textcolor{red}{\textcjheb{by+s'}} ASJB $|$will vergelten/ich werde vergelten\\
10.&227.&5258.&851.&20230.&28.&4&341&30\_1\_10\_300 \textcolor{red}{\textcjheb{+sy'l}} LAJS $|$dem Mann\\
11.&228.&5259.&855.&20234.&32.&5&206&20\_80\_70\_30\_6 \textcolor{red}{\textcjheb{wl`pk}} KPaLW $|$nach seinem Werk/gem"a"s seinem Tun\\
\end{tabular}\medskip \\
Ende des Verses 24.29\\
Verse: 708, Buchstaben: 36, 859, 20238, Totalwerte: 2950, 62942, 1434881\\
\\
Sprich nicht: Wie er mir getan hat, so will ich ihm tun, will dem Manne vergelten nach seinem Werke.\\
\newpage 
{\bf -- 24.30}\\
\medskip \\
\begin{tabular}{rrrrrrrrp{120mm}}
WV&WK&WB&ABK&ABB&ABV&AnzB&TW&Zahlencode \textcolor{red}{$\boldsymbol{Grundtext}$} Umschrift $|$"Ubersetzung(en)\\
1.&229.&5260.&860.&20239.&1.&2&100&70\_30 \textcolor{red}{\textcjheb{l`}} aL $|$an\\
2.&230.&5261.&862.&20241.&3.&3&309&300\_4\_5 \textcolor{red}{\textcjheb{hd+s}} SDH $|$dem Acker/dem Feld\\
3.&231.&5262.&865.&20244.&6.&3&311&1\_10\_300 \textcolor{red}{\textcjheb{+sy'}} AJS $|$(eines) Mannes\\
4.&232.&5263.&868.&20247.&9.&3&190&70\_90\_30 \textcolor{red}{\textcjheb{l.s`}} a"sL $|$faulen/tr"agen\\
5.&233.&5264.&871.&20250.&12.&5&682&70\_2\_200\_400\_10 \textcolor{red}{\textcjheb{ytrb`}} aBRTJ $|$kam ich vor"uber/ich ging vorbei\\
6.&234.&5265.&876.&20255.&17.&3&106&6\_70\_30 \textcolor{red}{\textcjheb{l`w}} WaL $|$und an\\
7.&235.&5266.&879.&20258.&20.&3&260&20\_200\_40 \textcolor{red}{\textcjheb{mrk}} KRM $|$dem Weinberg/(dem) Weingarten\\
8.&236.&5267.&882.&20261.&23.&3&45&1\_4\_40 \textcolor{red}{\textcjheb{md'}} ADM $|$(eines) Menschen\\
9.&237.&5268.&885.&20264.&26.&3&268&8\_60\_200 \textcolor{red}{\textcjheb{rs.h}} CsR $|$un-/entbehrend\\
10.&238.&5269.&888.&20267.&29.&2&32&30\_2 \textcolor{red}{\textcjheb{bl}} LB $|$verst"andigen/Herz (=Verstand)\\
\end{tabular}\medskip \\
Ende des Verses 24.30\\
Verse: 709, Buchstaben: 30, 889, 20268, Totalwerte: 2303, 65245, 1437184\\
\\
An dem Acker eines faulen Mannes kam ich vor"uber, und an dem Weinberge eines unverst"andigen Menschen.\\
\newpage 
{\bf -- 24.31}\\
\medskip \\
\begin{tabular}{rrrrrrrrp{120mm}}
WV&WK&WB&ABK&ABB&ABV&AnzB&TW&Zahlencode \textcolor{red}{$\boldsymbol{Grundtext}$} Umschrift $|$"Ubersetzung(en)\\
1.&239.&5270.&890.&20269.&1.&4&66&6\_5\_50\_5 \textcolor{red}{\textcjheb{hnhw}} WHNH $|$und siehe\\
2.&240.&5271.&894.&20273.&5.&3&105&70\_30\_5 \textcolor{red}{\textcjheb{hl`}} aLH $|$er war "uberwachsen/er ging auf\\
3.&241.&5272.&897.&20276.&8.&3&56&20\_30\_6 \textcolor{red}{\textcjheb{wlk}} KLW $|$ganz/alles an ihm (war)\\
4.&242.&5273.&900.&20279.&11.&6&540&100\_40\_300\_50\_10\_40 \textcolor{red}{\textcjheb{myn+smq}} QMSNJM $|$mit Disteln/(mit) Unkr"autern\\
5.&243.&5274.&906.&20285.&17.&3&86&20\_60\_6 \textcolor{red}{\textcjheb{wsk}} KsW $|$bedeckt war/sie (=es) bedeckten\\
6.&244.&5275.&909.&20288.&20.&4&146&80\_50\_10\_6 \textcolor{red}{\textcjheb{wynp}} PNJW $|$seine Fl"ache/seine Gesichter\\
7.&245.&5276.&913.&20292.&24.&5&288&8\_200\_30\_10\_40 \textcolor{red}{\textcjheb{mylr.h}} CRLJM $|$mit Brennnesseln/(mit) Disteln\\
8.&246.&5277.&918.&20297.&29.&4&213&6\_3\_4\_200 \textcolor{red}{\textcjheb{rdgw}} WGDR $|$und seine Mauer/und die Mauer\\
9.&247.&5278.&922.&20301.&33.&5&69&1\_2\_50\_10\_6 \textcolor{red}{\textcjheb{wynb'}} ABNJW $|$steinerne/seiner Steine\\
10.&248.&5279.&927.&20306.&38.&5&320&50\_5\_200\_60\_5 \textcolor{red}{\textcjheb{hsrhn}} NHRsH $|$eingerissen/(sie) war niedergerissen\\
\end{tabular}\medskip \\
Ende des Verses 24.31\\
Verse: 710, Buchstaben: 42, 931, 20310, Totalwerte: 1889, 67134, 1439073\\
\\
Und siehe, er war ganz mit Disteln "uberwachsen, seine Fl"ache war mit Brennesseln bedeckt, und seine steinerne Mauer eingerissen.\\
\newpage 
{\bf -- 24.32}\\
\medskip \\
\begin{tabular}{rrrrrrrrp{120mm}}
WV&WK&WB&ABK&ABB&ABV&AnzB&TW&Zahlencode \textcolor{red}{$\boldsymbol{Grundtext}$} Umschrift $|$"Ubersetzung(en)\\
1.&249.&5280.&932.&20311.&1.&5&27&6\_1\_8\_7\_5 \textcolor{red}{\textcjheb{hz.h'w}} WACZH $|$und ich schaute (es)\\
2.&250.&5281.&937.&20316.&6.&4&81&1\_50\_20\_10 \textcolor{red}{\textcjheb{ykn'}} ANKJ $|$ich\\
3.&251.&5282.&941.&20320.&10.&4&711&1\_300\_10\_400 \textcolor{red}{\textcjheb{ty+s'}} ASJT $|$richtete darauf/ich setzte\\
4.&252.&5283.&945.&20324.&14.&3&42&30\_2\_10 \textcolor{red}{\textcjheb{ybl}} LBJ $|$(in) mein Herz\\
5.&253.&5284.&948.&20327.&17.&5&621&200\_1\_10\_400\_10 \textcolor{red}{\textcjheb{yty'r}} RAJTJ $|$ich sah es/ich betrachtete es\\
6.&254.&5285.&953.&20332.&22.&5&548&30\_100\_8\_400\_10 \textcolor{red}{\textcjheb{yt.hql}} LQCTJ $|$empfing/ich nahm (mir)\\
7.&255.&5286.&958.&20337.&27.&4&306&40\_6\_60\_200 \textcolor{red}{\textcjheb{rswm}} MWsR $|$Unterweisung/eine Lehre daraus\\
\end{tabular}\medskip \\
Ende des Verses 24.32\\
Verse: 711, Buchstaben: 30, 961, 20340, Totalwerte: 2336, 69470, 1441409\\
\\
Und ich schaute es, ich richtete mein Herz darauf; ich sah es, empfing Unterweisung:\\
\newpage 
{\bf -- 24.33}\\
\medskip \\
\begin{tabular}{rrrrrrrrp{120mm}}
WV&WK&WB&ABK&ABB&ABV&AnzB&TW&Zahlencode \textcolor{red}{$\boldsymbol{Grundtext}$} Umschrift $|$"Ubersetzung(en)\\
1.&256.&5287.&962.&20341.&1.&3&119&40\_70\_9 \textcolor{red}{\textcjheb{.t`m}} Mat $|$(ein) wenig\\
2.&257.&5288.&965.&20344.&4.&4&756&300\_50\_6\_400 \textcolor{red}{\textcjheb{twn+s}} SNWT $|$Schlaf\\
3.&258.&5289.&969.&20348.&8.&3&119&40\_70\_9 \textcolor{red}{\textcjheb{.t`m}} Mat $|$(ein) wenig\\
4.&259.&5290.&972.&20351.&11.&6&902&400\_50\_6\_40\_6\_400 \textcolor{red}{\textcjheb{twmwnt}} TNWMWT $|$Schlummer\\
5.&260.&5291.&978.&20357.&17.&3&119&40\_70\_9 \textcolor{red}{\textcjheb{.t`m}} Mat $|$(ein) wenig\\
6.&261.&5292.&981.&20360.&20.&3&110&8\_2\_100 \textcolor{red}{\textcjheb{qb.h}} CBQ $|$Falten/Verschr"anken\\
7.&262.&5293.&984.&20363.&23.&4&64&10\_4\_10\_40 \textcolor{red}{\textcjheb{mydy}} JDJM $|$(beider) H"ande\\
8.&263.&5294.&988.&20367.&27.&4&352&30\_300\_20\_2 \textcolor{red}{\textcjheb{bk+sl}} LSKB $|$um auszuruhen/zum Ruhen\\
\end{tabular}\medskip \\
Ende des Verses 24.33\\
Verse: 712, Buchstaben: 30, 991, 20370, Totalwerte: 2541, 72011, 1443950\\
\\
Ein wenig Schlaf, ein wenig Schlummer, ein wenig H"andefalten, um auszuruhen-\\
\newpage 
{\bf -- 24.34}\\
\medskip \\
\begin{tabular}{rrrrrrrrp{120mm}}
WV&WK&WB&ABK&ABB&ABV&AnzB&TW&Zahlencode \textcolor{red}{$\boldsymbol{Grundtext}$} Umschrift $|$"Ubersetzung(en)\\
1.&264.&5295.&992.&20371.&1.&3&9&6\_2\_1 \textcolor{red}{\textcjheb{'bw}} WBA $|$und (er (=es)) kommt\\
2.&265.&5296.&995.&20374.&4.&5&495&40\_400\_5\_30\_20 \textcolor{red}{\textcjheb{klhtm}} MTHLK $|$herangeschrittten/einhergehend\\
3.&266.&5297.&1000.&20379.&9.&4&530&200\_10\_300\_20 \textcolor{red}{\textcjheb{k+syr}} RJSK $|$deine Armut\\
4.&267.&5298.&1004.&20383.&13.&7&344&6\_40\_8\_60\_200\_10\_20 \textcolor{red}{\textcjheb{kyrs.hmw}} WMCsRJK $|$und deine Not/und deine Verluste\\
5.&268.&5299.&1011.&20390.&20.&4&331&20\_1\_10\_300 \textcolor{red}{\textcjheb{+sy'k}} KAJS $|$wie (ein) Mann\\
6.&269.&5300.&1015.&20394.&24.&3&93&40\_3\_50 \textcolor{red}{\textcjheb{ngm}} MGN $|$gewappneter/(mit) Schild\\
\end{tabular}\medskip \\
Ende des Verses 24.34\\
Verse: 713, Buchstaben: 26, 1017, 20396, Totalwerte: 1802, 73813, 1445752\\
\\
und deine Armut kommt herangeschritten, und deine Not wie ein gewappneter Mann.\\
\\
{\bf Ende des Kapitels 24}\\
\newpage 
{\bf -- 25.1}\\
\medskip \\
\begin{tabular}{rrrrrrrrp{120mm}}
WV&WK&WB&ABK&ABB&ABV&AnzB&TW&Zahlencode \textcolor{red}{$\boldsymbol{Grundtext}$} Umschrift $|$"Ubersetzung(en)\\
1.&1.&5301.&1.&20397.&1.&2&43&3\_40 \textcolor{red}{\textcjheb{mg}} GM $|$auch\\
2.&2.&5302.&3.&20399.&3.&3&36&1\_30\_5 \textcolor{red}{\textcjheb{hl'}} ALH $|$diese (sind)\\
3.&3.&5303.&6.&20402.&6.&4&380&40\_300\_30\_10 \textcolor{red}{\textcjheb{yl+sm}} MSLJ $|$Spr"uche\\
4.&4.&5304.&10.&20406.&10.&4&375&300\_30\_40\_5 \textcolor{red}{\textcjheb{hml+s}} SLMH $|$(von) Salomo(s)\\
5.&5.&5305.&14.&20410.&14.&3&501&1\_300\_200 \textcolor{red}{\textcjheb{r+s'}} ASR $|$welche\\
6.&6.&5306.&17.&20413.&17.&6&591&5\_70\_400\_10\_100\_6 \textcolor{red}{\textcjheb{wqyt`h}} HaTJQW $|$zusammengetragen haben/(sie) sammelten\\
7.&7.&5307.&23.&20419.&23.&4&361&1\_50\_300\_10 \textcolor{red}{\textcjheb{y+sn'}} ANSJ $|$die M"anner\\
8.&8.&5308.&27.&20423.&27.&5&130&8\_7\_100\_10\_5 \textcolor{red}{\textcjheb{hyqz.h}} CZQJH $|$Hiskia(s)/(von) Chiskija//$<$meine St"arke ist Jah$>$\\
9.&9.&5309.&32.&20428.&32.&3&90&40\_30\_20 \textcolor{red}{\textcjheb{klm}} MLK $|$des K"onigs\\
10.&10.&5310.&35.&20431.&35.&5&30&10\_5\_6\_4\_5 \textcolor{red}{\textcjheb{hdwhy}} JHWDH $|$(von) Juda///$<$Lobpreis$>$\\
\end{tabular}\medskip \\
Ende des Verses 25.1\\
Verse: 714, Buchstaben: 39, 39, 20435, Totalwerte: 2537, 2537, 1448289\\
\\
Auch diese sind Spr"uche Salomos, welche die M"anner Hiskias, des K"onigs von Juda, zusammengetragen haben:\\
\newpage 
{\bf -- 25.2}\\
\medskip \\
\begin{tabular}{rrrrrrrrp{120mm}}
WV&WK&WB&ABK&ABB&ABV&AnzB&TW&Zahlencode \textcolor{red}{$\boldsymbol{Grundtext}$} Umschrift $|$"Ubersetzung(en)\\
1.&11.&5311.&40.&20436.&1.&3&26&20\_2\_4 \textcolor{red}{\textcjheb{dbk}} KBD $|$Ehre\\
2.&12.&5312.&43.&20439.&4.&5&86&1\_30\_5\_10\_40 \textcolor{red}{\textcjheb{myhl'}} ALHJM $|$Gottes\\
3.&13.&5313.&48.&20444.&9.&4&665&5\_60\_400\_200 \textcolor{red}{\textcjheb{rtsh}} HsTR $|$(ist) (es zu) verbergen\\
4.&14.&5314.&52.&20448.&13.&3&206&4\_2\_200 \textcolor{red}{\textcjheb{rbd}} DBR $|$(eine) Sache\\
5.&15.&5315.&55.&20451.&16.&4&32&6\_20\_2\_4 \textcolor{red}{\textcjheb{dbkw}} WKBD $|$aber Ehre/und Ehre\\
6.&16.&5316.&59.&20455.&20.&5&140&40\_30\_20\_10\_40 \textcolor{red}{\textcjheb{myklm}} MLKJM $|$der K"onige\\
7.&17.&5317.&64.&20460.&25.&3&308&8\_100\_200 \textcolor{red}{\textcjheb{rq.h}} CQR $|$(ist) (zu) erforschen\\
8.&18.&5318.&67.&20463.&28.&3&206&4\_2\_200 \textcolor{red}{\textcjheb{rbd}} DBR $|$(eine) Sache\\
\end{tabular}\medskip \\
Ende des Verses 25.2\\
Verse: 715, Buchstaben: 30, 69, 20465, Totalwerte: 1669, 4206, 1449958\\
\\
Gottes Ehre ist es, eine Sache zu verbergen, aber der K"onige Ehre, eine Sache zu erforschen.\\
\newpage 
{\bf -- 25.3}\\
\medskip \\
\begin{tabular}{rrrrrrrrp{120mm}}
WV&WK&WB&ABK&ABB&ABV&AnzB&TW&Zahlencode \textcolor{red}{$\boldsymbol{Grundtext}$} Umschrift $|$"Ubersetzung(en)\\
1.&19.&5319.&70.&20466.&1.&4&390&300\_40\_10\_40 \textcolor{red}{\textcjheb{mym+s}} SMJM $|$der Himmel/die Himmel\\
2.&20.&5320.&74.&20470.&5.&4&276&30\_200\_6\_40 \textcolor{red}{\textcjheb{mwrl}} LRWM $|$an H"ohe\\
3.&21.&5321.&78.&20474.&9.&4&297&6\_1\_200\_90 \textcolor{red}{\textcjheb{.sr'w}} WAR"s $|$und die Erde\\
4.&22.&5322.&82.&20478.&13.&4&240&30\_70\_40\_100 \textcolor{red}{\textcjheb{qm`l}} LaMQ $|$an Tiefe\\
5.&23.&5323.&86.&20482.&17.&3&38&6\_30\_2 \textcolor{red}{\textcjheb{blw}} WLB $|$und das Herz\\
6.&24.&5324.&89.&20485.&20.&5&140&40\_30\_20\_10\_40 \textcolor{red}{\textcjheb{myklm}} MLKJM $|$der K"onige\\
7.&25.&5325.&94.&20490.&25.&3&61&1\_10\_50 \textcolor{red}{\textcjheb{ny'}} AJN $|$sind un-/nicht ist\\
8.&26.&5326.&97.&20493.&28.&3&308&8\_100\_200 \textcolor{red}{\textcjheb{rq.h}} CQR $|$erforschlich/Erforschung\\
\end{tabular}\medskip \\
Ende des Verses 25.3\\
Verse: 716, Buchstaben: 30, 99, 20495, Totalwerte: 1750, 5956, 1451708\\
\\
Der Himmel an H"ohe, und die Erde an Tiefe, und das Herz der K"onige sind unerforschlich.\\
\newpage 
{\bf -- 25.4}\\
\medskip \\
\begin{tabular}{rrrrrrrrp{120mm}}
WV&WK&WB&ABK&ABB&ABV&AnzB&TW&Zahlencode \textcolor{red}{$\boldsymbol{Grundtext}$} Umschrift $|$"Ubersetzung(en)\\
1.&27.&5327.&100.&20496.&1.&3&14&5\_3\_6 \textcolor{red}{\textcjheb{wgh}} HGW $|$man entferne/schaffe weg\\
2.&28.&5328.&103.&20499.&4.&5&123&60\_10\_3\_10\_40 \textcolor{red}{\textcjheb{mygys}} sJGJM $|$die Schlacken\\
3.&29.&5329.&108.&20504.&9.&4&200&40\_20\_60\_80 \textcolor{red}{\textcjheb{pskm}} MKsP $|$vom Silber\\
4.&30.&5330.&112.&20508.&13.&4&107&6\_10\_90\_1 \textcolor{red}{\textcjheb{'.syw}} WJ"sA $|$so geht hervor/und er (=es) kommt hervor\\
5.&31.&5331.&116.&20512.&17.&4&400&30\_90\_200\_80 \textcolor{red}{\textcjheb{pr.sl}} L"sRP $|$f"ur den Goldschmied/dem Schmelzenden\\
6.&32.&5332.&120.&20516.&21.&3&60&20\_30\_10 \textcolor{red}{\textcjheb{ylk}} KLJ $|$ein Ger"at/(ein) Gef"a"s\\
\end{tabular}\medskip \\
Ende des Verses 25.4\\
Verse: 717, Buchstaben: 23, 122, 20518, Totalwerte: 904, 6860, 1452612\\
\\
Man entferne die Schlacken von dem Silber, so geht f"ur den Goldschmied ein Ger"at hervor.\\
\newpage 
{\bf -- 25.5}\\
\medskip \\
\begin{tabular}{rrrrrrrrp{120mm}}
WV&WK&WB&ABK&ABB&ABV&AnzB&TW&Zahlencode \textcolor{red}{$\boldsymbol{Grundtext}$} Umschrift $|$"Ubersetzung(en)\\
1.&33.&5333.&123.&20519.&1.&3&14&5\_3\_6 \textcolor{red}{\textcjheb{wgh}} HGW $|$man entferne/schaffe weg\\
2.&34.&5334.&126.&20522.&4.&3&570&200\_300\_70 \textcolor{red}{\textcjheb{`+sr}} RSa $|$den Gesetzlosen/(einen) Frevler\\
3.&35.&5335.&129.&20525.&7.&4&170&30\_80\_50\_10 \textcolor{red}{\textcjheb{ynpl}} LPNJ $|$vor\\
4.&36.&5336.&133.&20529.&11.&3&90&40\_30\_20 \textcolor{red}{\textcjheb{klm}} MLK $|$dem K"onig\\
5.&37.&5337.&136.&20532.&14.&5&92&6\_10\_20\_6\_50 \textcolor{red}{\textcjheb{nwkyw}} WJKWN $|$so wird feststehen/und er (=es) hat Bestand\\
6.&38.&5338.&141.&20537.&19.&4&196&2\_90\_4\_100 \textcolor{red}{\textcjheb{qd.sb}} B"sDQ $|$durch Gerechtigkeit\\
7.&39.&5339.&145.&20541.&23.&4&87&20\_60\_1\_6 \textcolor{red}{\textcjheb{w'sk}} KsAW $|$sein Thron\\
\end{tabular}\medskip \\
Ende des Verses 25.5\\
Verse: 718, Buchstaben: 26, 148, 20544, Totalwerte: 1219, 8079, 1453831\\
\\
Man entferne den Gesetzlosen vor dem K"onig, so wird sein Thron feststehen durch Gerechtigkeit.\\
\newpage 
{\bf -- 25.6}\\
\medskip \\
\begin{tabular}{rrrrrrrrp{120mm}}
WV&WK&WB&ABK&ABB&ABV&AnzB&TW&Zahlencode \textcolor{red}{$\boldsymbol{Grundtext}$} Umschrift $|$"Ubersetzung(en)\\
1.&40.&5340.&149.&20545.&1.&2&31&1\_30 \textcolor{red}{\textcjheb{l'}} AL $|$nicht\\
2.&41.&5341.&151.&20547.&3.&5&1009&400\_400\_5\_4\_200 \textcolor{red}{\textcjheb{rdhtt}} TTHDR $|$(du sollst) br"uste(n) dich\\
3.&42.&5342.&156.&20552.&8.&4&170&30\_80\_50\_10 \textcolor{red}{\textcjheb{ynpl}} LPNJ $|$vor\\
4.&43.&5343.&160.&20556.&12.&3&90&40\_30\_20 \textcolor{red}{\textcjheb{klm}} MLK $|$dem K"onig\\
5.&44.&5344.&163.&20559.&15.&6&194&6\_2\_40\_100\_6\_40 \textcolor{red}{\textcjheb{mwqmbw}} WBMQWM $|$und an den Platz/und an der Stelle\\
6.&45.&5345.&169.&20565.&21.&5&87&3\_4\_30\_10\_40 \textcolor{red}{\textcjheb{myldg}} GDLJM $|$der Gro"sen\\
7.&46.&5346.&174.&20570.&26.&2&31&1\_30 \textcolor{red}{\textcjheb{l'}} AL $|$nicht\\
8.&47.&5347.&176.&20572.&28.&4&514&400\_70\_40\_4 \textcolor{red}{\textcjheb{dm`t}} TaMD $|$stelle dich/sollst du stehen\\
\end{tabular}\medskip \\
Ende des Verses 25.6\\
Verse: 719, Buchstaben: 31, 179, 20575, Totalwerte: 2126, 10205, 1455957\\
\\
Br"uste dich nicht vor dem K"onig, und stelle dich nicht an den Platz der Gro"sen.\\
\newpage 
{\bf -- 25.7}\\
\medskip \\
\begin{tabular}{rrrrrrrrp{120mm}}
WV&WK&WB&ABK&ABB&ABV&AnzB&TW&Zahlencode \textcolor{red}{$\boldsymbol{Grundtext}$} Umschrift $|$"Ubersetzung(en)\\
1.&48.&5348.&180.&20576.&1.&2&30&20\_10 \textcolor{red}{\textcjheb{yk}} KJ $|$denn\\
2.&49.&5349.&182.&20578.&3.&3&17&9\_6\_2 \textcolor{red}{\textcjheb{bw.t}} tWB $|$besser ist es/gut (ist es)\\
3.&50.&5350.&185.&20581.&6.&3&241&1\_40\_200 \textcolor{red}{\textcjheb{rm'}} AMR $|$dass man sage/(ein) Sagen\\
4.&51.&5351.&188.&20584.&9.&2&50&30\_20 \textcolor{red}{\textcjheb{kl}} LK $|$(zu) dir\\
5.&52.&5352.&190.&20586.&11.&3&105&70\_30\_5 \textcolor{red}{\textcjheb{hl`}} aLH $|$komm herauf\\
6.&53.&5353.&193.&20589.&14.&3&60&5\_50\_5 \textcolor{red}{\textcjheb{hnh}} HNH $|$hier(her)\\
7.&54.&5354.&196.&20592.&17.&7&485&40\_5\_300\_80\_10\_30\_20 \textcolor{red}{\textcjheb{klyp+shm}} MHSPJLK $|$als dass man dich erniedrige/als ein Erniedrigen dich\\
8.&55.&5355.&203.&20599.&24.&4&170&30\_80\_50\_10 \textcolor{red}{\textcjheb{ynpl}} LPNJ $|$vor\\
9.&56.&5356.&207.&20603.&28.&4&66&50\_4\_10\_2 \textcolor{red}{\textcjheb{bydn}} NDJB $|$dem Edlen/(einem) Edlen\\
10.&57.&5357.&211.&20607.&32.&3&501&1\_300\_200 \textcolor{red}{\textcjheb{r+s'}} ASR $|$den doch/was\\
11.&58.&5358.&214.&20610.&35.&3&207&200\_1\_6 \textcolor{red}{\textcjheb{w'r}} RAW $|$(sie) haben gesehen\\
12.&59.&5359.&217.&20613.&38.&5&160&70\_10\_50\_10\_20 \textcolor{red}{\textcjheb{kyny`}} aJNJK $|$deine Augen\\
\end{tabular}\medskip \\
Ende des Verses 25.7\\
Verse: 720, Buchstaben: 42, 221, 20617, Totalwerte: 2092, 12297, 1458049\\
\\
Denn besser ist es, da"s man dir sage: Komm hier herauf, als da"s man dich erniedrige vor dem Edlen, den deine Augen doch gesehen haben.\\
\newpage 
{\bf -- 25.8}\\
\medskip \\
\begin{tabular}{rrrrrrrrp{120mm}}
WV&WK&WB&ABK&ABB&ABV&AnzB&TW&Zahlencode \textcolor{red}{$\boldsymbol{Grundtext}$} Umschrift $|$"Ubersetzung(en)\\
1.&60.&5360.&222.&20618.&1.&2&31&1\_30 \textcolor{red}{\textcjheb{l'}} AL $|$nicht\\
2.&61.&5361.&224.&20620.&3.&3&491&400\_90\_1 \textcolor{red}{\textcjheb{'.st}} T"sA $|$geh aus/du sollst es vorbringen\\
3.&62.&5362.&227.&20623.&6.&3&232&30\_200\_2 \textcolor{red}{\textcjheb{brl}} LRB $|$zu einem Streithandel/zum Rechtsstreit\\
4.&63.&5363.&230.&20626.&9.&3&245&40\_5\_200 \textcolor{red}{\textcjheb{rhm}} MHR $|$eilig/vorschnell\\
5.&64.&5364.&233.&20629.&12.&2&130&80\_50 \textcolor{red}{\textcjheb{np}} PN $|$damit nicht\\
6.&65.&5365.&235.&20631.&14.&2&45&40\_5 \textcolor{red}{\textcjheb{hm}} MH $|$was\\
7.&66.&5366.&237.&20633.&16.&4&775&400\_70\_300\_5 \textcolor{red}{\textcjheb{h+s`t}} TaSH $|$du zu tun hast/du wirst machen\\
8.&67.&5367.&241.&20637.&20.&7&626&2\_1\_8\_200\_10\_400\_5 \textcolor{red}{\textcjheb{htyr.h'b}} BACRJTH $|$am Ende (davon nicht fraglich werde)\\
9.&68.&5368.&248.&20644.&27.&6&107&2\_5\_20\_30\_10\_40 \textcolor{red}{\textcjheb{mylkhb}} BHKLJM $|$wenn besch"amt\\
10.&69.&5369.&254.&20650.&33.&3&421&1\_400\_20 \textcolor{red}{\textcjheb{kt'}} ATK $|$dich\\
11.&70.&5370.&257.&20653.&36.&3&290&200\_70\_20 \textcolor{red}{\textcjheb{k`r}} RaK $|$dein N"achster/dein Gef"ahrte\\
\end{tabular}\medskip \\
Ende des Verses 25.8\\
Verse: 721, Buchstaben: 38, 259, 20655, Totalwerte: 3393, 15690, 1461442\\
\\
Geh nicht eilig aus zu einem Streithandel, damit am Ende davon nicht fraglich werde, was du zu tun hast, wenn dein N"achster dich besch"amt. -\\
\newpage 
{\bf -- 25.9}\\
\medskip \\
\begin{tabular}{rrrrrrrrp{120mm}}
WV&WK&WB&ABK&ABB&ABV&AnzB&TW&Zahlencode \textcolor{red}{$\boldsymbol{Grundtext}$} Umschrift $|$"Ubersetzung(en)\\
1.&71.&5371.&260.&20656.&1.&4&232&200\_10\_2\_20 \textcolor{red}{\textcjheb{kbyr}} RJBK $|$deinen Streithandel/deinen Rechtsstreit\\
2.&72.&5372.&264.&20660.&5.&3&212&200\_10\_2 \textcolor{red}{\textcjheb{byr}} RJB $|$f"uhre/streite\\
3.&73.&5373.&267.&20663.&8.&2&401&1\_400 \textcolor{red}{\textcjheb{t'}} AT $|$mit\\
4.&74.&5374.&269.&20665.&10.&3&290&200\_70\_20 \textcolor{red}{\textcjheb{k`r}} RaK $|$deinem N"achsten/deinem Gef"ahrten\\
5.&75.&5375.&272.&20668.&13.&4&76&6\_60\_6\_4 \textcolor{red}{\textcjheb{dwsw}} WsWD $|$aber das Geheimnis/und ein Geheimnis\\
6.&76.&5376.&276.&20672.&17.&3&209&1\_8\_200 \textcolor{red}{\textcjheb{r.h'}} ACR $|$(eines) anderen\\
7.&77.&5377.&279.&20675.&20.&2&31&1\_30 \textcolor{red}{\textcjheb{l'}} AL $|$nicht\\
8.&78.&5378.&281.&20677.&22.&3&433&400\_3\_30 \textcolor{red}{\textcjheb{lgt}} TGL $|$enth"ulle/du sollst aufdecken\\
\end{tabular}\medskip \\
Ende des Verses 25.9\\
Verse: 722, Buchstaben: 24, 283, 20679, Totalwerte: 1884, 17574, 1463326\\
\\
F"uhre deinen Streithandel mit deinem N"achsten, aber enth"ulle nicht das Geheimnis eines anderen;\\
\newpage 
{\bf -- 25.10}\\
\medskip \\
\begin{tabular}{rrrrrrrrp{120mm}}
WV&WK&WB&ABK&ABB&ABV&AnzB&TW&Zahlencode \textcolor{red}{$\boldsymbol{Grundtext}$} Umschrift $|$"Ubersetzung(en)\\
1.&79.&5379.&284.&20680.&1.&2&130&80\_50 \textcolor{red}{\textcjheb{np}} PN $|$damit nicht/dass nicht\\
2.&80.&5380.&286.&20682.&3.&5&102&10\_8\_60\_4\_20 \textcolor{red}{\textcjheb{kds.hy}} JCsDK $|$dich schm"ahe/er (=es) tadelt dich\\
3.&81.&5381.&291.&20687.&8.&3&410&300\_40\_70 \textcolor{red}{\textcjheb{`m+s}} SMa $|$wer es h"ort/(ein) H"orender\\
4.&82.&5382.&294.&20690.&11.&5&432&6\_4\_2\_400\_20 \textcolor{red}{\textcjheb{ktbdw}} WDBTK $|$und dein "ubler Ruf/deine ("uble) Nachrede\\
5.&83.&5383.&299.&20695.&16.&2&31&30\_1 \textcolor{red}{\textcjheb{'l}} LA $|$nicht\\
6.&84.&5384.&301.&20697.&18.&4&708&400\_300\_6\_2 \textcolor{red}{\textcjheb{bw+st}} TSWB $|$mehr weiche/sie (=es) kehre zur"uck\\
\end{tabular}\medskip \\
Ende des Verses 25.10\\
Verse: 723, Buchstaben: 21, 304, 20700, Totalwerte: 1813, 19387, 1465139\\
\\
damit dich nicht schm"ahe, wer es h"ort, und dein "ubler Ruf nicht mehr weiche.\\
\newpage 
{\bf -- 25.11}\\
\medskip \\
\begin{tabular}{rrrrrrrrp{120mm}}
WV&WK&WB&ABK&ABB&ABV&AnzB&TW&Zahlencode \textcolor{red}{$\boldsymbol{Grundtext}$} Umschrift $|$"Ubersetzung(en)\\
1.&85.&5385.&305.&20701.&1.&5&504&400\_80\_6\_8\_10 \textcolor{red}{\textcjheb{y.hwpt}} TPWCJ $|$"Apfel\\
2.&86.&5386.&310.&20706.&6.&3&14&7\_5\_2 \textcolor{red}{\textcjheb{bhz}} ZHB $|$goldene/(aus) Gold\\
3.&87.&5387.&313.&20709.&9.&7&778&2\_40\_300\_20\_10\_6\_400 \textcolor{red}{\textcjheb{twyk+smb}} BMSKJWT $|$in Prunkger"aten/in Gef"a"sen\\
4.&88.&5388.&320.&20716.&16.&3&160&20\_60\_80 \textcolor{red}{\textcjheb{psk}} KsP $|$silbernen/(von) Silber\\
5.&89.&5389.&323.&20719.&19.&3&206&4\_2\_200 \textcolor{red}{\textcjheb{rbd}} DBR $|$(so) (ist ein) Wort\\
6.&90.&5390.&326.&20722.&22.&3&206&4\_2\_200 \textcolor{red}{\textcjheb{rbd}} DBR $|$geredet/gesprochenes\\
7.&91.&5391.&329.&20725.&25.&2&100&70\_30 \textcolor{red}{\textcjheb{l`}} aL $|$zu/an\\
8.&92.&5392.&331.&20727.&27.&5&147&1\_80\_50\_10\_6 \textcolor{red}{\textcjheb{wynp'}} APNJW $|$seiner Zeit/seiner Gelegenheit\\
\end{tabular}\medskip \\
Ende des Verses 25.11\\
Verse: 724, Buchstaben: 31, 335, 20731, Totalwerte: 2115, 21502, 1467254\\
\\
Goldene "Apfel in silbernen Prunkger"aten: so ist ein Wort, geredet zu seiner Zeit.\\
\newpage 
{\bf -- 25.12}\\
\medskip \\
\begin{tabular}{rrrrrrrrp{120mm}}
WV&WK&WB&ABK&ABB&ABV&AnzB&TW&Zahlencode \textcolor{red}{$\boldsymbol{Grundtext}$} Umschrift $|$"Ubersetzung(en)\\
1.&93.&5393.&336.&20732.&1.&3&97&50\_7\_40 \textcolor{red}{\textcjheb{mzn}} NZM $|$(wie) (ein) (Ohr)Ring\\
2.&94.&5394.&339.&20735.&4.&3&14&7\_5\_2 \textcolor{red}{\textcjheb{bhz}} ZHB $|$goldener/(von) Gold\\
3.&95.&5395.&342.&20738.&7.&4&54&6\_8\_30\_10 \textcolor{red}{\textcjheb{yl.hw}} WCLJ $|$und (ein) (Hals)Geschmeide\\
4.&96.&5396.&346.&20742.&11.&3&460&20\_400\_40 \textcolor{red}{\textcjheb{mtk}} KTM $|$von feinem Gold/aus Feingold\\
5.&97.&5397.&349.&20745.&14.&5&84&40\_6\_20\_10\_8 \textcolor{red}{\textcjheb{.hykwm}} MWKJC $|$so ist ein Tadler/(ist) (ein) Mahnender\\
6.&98.&5398.&354.&20750.&19.&3&68&8\_20\_40 \textcolor{red}{\textcjheb{mk.h}} CKM $|$weiser\\
7.&99.&5399.&357.&20753.&22.&2&100&70\_30 \textcolor{red}{\textcjheb{l`}} aL $|$f"ur\\
8.&100.&5400.&359.&20755.&24.&3&58&1\_7\_50 \textcolor{red}{\textcjheb{nz'}} AZN $|$ein Ohr/das Ohr\\
9.&101.&5401.&362.&20758.&27.&4&810&300\_40\_70\_400 \textcolor{red}{\textcjheb{t`m+s}} SMaT $|$h"orende(s)\\
\end{tabular}\medskip \\
Ende des Verses 25.12\\
Verse: 725, Buchstaben: 30, 365, 20761, Totalwerte: 1745, 23247, 1468999\\
\\
Ein goldener Ohrring und ein Halsgeschmeide von feinem Golde: so ist ein weiser Tadler f"ur ein h"orendes Ohr.\\
\newpage 
{\bf -- 25.13}\\
\medskip \\
\begin{tabular}{rrrrrrrrp{120mm}}
WV&WK&WB&ABK&ABB&ABV&AnzB&TW&Zahlencode \textcolor{red}{$\boldsymbol{Grundtext}$} Umschrift $|$"Ubersetzung(en)\\
1.&102.&5402.&366.&20762.&1.&4&560&20\_90\_50\_400 \textcolor{red}{\textcjheb{tn.sk}} K"sNT $|$wie K"uhlung\\
2.&103.&5403.&370.&20766.&5.&3&333&300\_30\_3 \textcolor{red}{\textcjheb{gl+s}} SLG $|$des Schnees/(von) Schnee\\
3.&104.&5404.&373.&20769.&8.&4&58&2\_10\_6\_40 \textcolor{red}{\textcjheb{mwyb}} BJWM $|$an einem Tag\\
4.&105.&5405.&377.&20773.&12.&4&400&100\_90\_10\_200 \textcolor{red}{\textcjheb{ry.sq}} Q"sJR $|$(der) Ernte\\
5.&106.&5406.&381.&20777.&16.&3&300&90\_10\_200 \textcolor{red}{\textcjheb{ry.s}} "sJR $|$(ist ein) Bote\\
6.&107.&5407.&384.&20780.&19.&4&141&50\_1\_40\_50 \textcolor{red}{\textcjheb{nm'n}} NAMN $|$treuer/zuverl"assiger\\
7.&108.&5408.&388.&20784.&23.&6&384&30\_300\_30\_8\_10\_6 \textcolor{red}{\textcjheb{wy.hl+sl}} LSLCJW $|$denen die ihn senden/f"ur seine Auftraggeber\\
8.&109.&5409.&394.&20790.&29.&4&436&6\_50\_80\_300 \textcolor{red}{\textcjheb{+spnw}} WNPS $|$und die Seele\\
9.&110.&5410.&398.&20794.&33.&5&71&1\_4\_50\_10\_6 \textcolor{red}{\textcjheb{wynd'}} ADNJW $|$seines Herrn\\
10.&111.&5411.&403.&20799.&38.&4&322&10\_300\_10\_2 \textcolor{red}{\textcjheb{by+sy}} JSJB $|$er erquickt\\
\end{tabular}\medskip \\
Ende des Verses 25.13\\
Verse: 726, Buchstaben: 41, 406, 20802, Totalwerte: 3005, 26252, 1472004\\
\\
Wie K"uhlung des Schnees an einem Erntetage ist ein treuer Bote denen, die ihn senden: er erquickt die Seele seines Herrn.\\
\newpage 
{\bf -- 25.14}\\
\medskip \\
\begin{tabular}{rrrrrrrrp{120mm}}
WV&WK&WB&ABK&ABB&ABV&AnzB&TW&Zahlencode \textcolor{red}{$\boldsymbol{Grundtext}$} Umschrift $|$"Ubersetzung(en)\\
1.&112.&5412.&407.&20803.&1.&6&411&50\_300\_10\_1\_10\_40 \textcolor{red}{\textcjheb{my'y+sn}} NSJAJM $|$Wolken\\
2.&113.&5413.&413.&20809.&7.&4&220&6\_200\_6\_8 \textcolor{red}{\textcjheb{.hwrw}} WRWC $|$und Wind\\
3.&114.&5414.&417.&20813.&11.&4&349&6\_3\_300\_40 \textcolor{red}{\textcjheb{m+sgw}} WGSM $|$und (kein) Regen\\
4.&115.&5415.&421.&20817.&15.&3&61&1\_10\_50 \textcolor{red}{\textcjheb{ny'}} AJN $|$/nicht (gibt es)\\
5.&116.&5416.&424.&20820.&18.&3&311&1\_10\_300 \textcolor{red}{\textcjheb{+sy'}} AJS $|$so ist ein Mann/(so) (ist) jemand\\
6.&117.&5417.&427.&20823.&21.&5&505&40\_400\_5\_30\_30 \textcolor{red}{\textcjheb{llhtm}} MTHLL $|$welcher prahlt/der sich r"uhmt\\
7.&118.&5418.&432.&20828.&26.&4&842&2\_40\_400\_400 \textcolor{red}{\textcjheb{ttmb}} BMTT $|$mit Geschenk/mit einer Gabe\\
8.&119.&5419.&436.&20832.&30.&3&600&300\_100\_200 \textcolor{red}{\textcjheb{rq+s}} SQR $|$tr"ugerischem/des Betrugs\\
\end{tabular}\medskip \\
Ende des Verses 25.14\\
Verse: 727, Buchstaben: 32, 438, 20834, Totalwerte: 3299, 29551, 1475303\\
\\
Wolken und Wind, und kein Regen: so ist ein Mann, welcher mit tr"ugerischem Geschenke prahlt.\\
\newpage 
{\bf -- 25.15}\\
\medskip \\
\begin{tabular}{rrrrrrrrp{120mm}}
WV&WK&WB&ABK&ABB&ABV&AnzB&TW&Zahlencode \textcolor{red}{$\boldsymbol{Grundtext}$} Umschrift $|$"Ubersetzung(en)\\
1.&120.&5420.&439.&20835.&1.&4&223&2\_1\_200\_20 \textcolor{red}{\textcjheb{kr'b}} BARK $|$durch Lang-/in der L"ange\\
2.&121.&5421.&443.&20839.&5.&4&131&1\_80\_10\_40 \textcolor{red}{\textcjheb{myp'}} APJM $|$mut/(zweier) Nasenl"ocher\\
3.&122.&5422.&447.&20843.&9.&4&495&10\_80\_400\_5 \textcolor{red}{\textcjheb{htpy}} JPTH $|$wird "uberredet/er (=es) werden umgestimmt\\
4.&123.&5423.&451.&20847.&13.&4&250&100\_90\_10\_50 \textcolor{red}{\textcjheb{ny.sq}} Q"sJN $|$(ein) Richter\\
5.&124.&5424.&455.&20851.&17.&5&392&6\_30\_300\_6\_50 \textcolor{red}{\textcjheb{nw+slw}} WLSWN $|$und eine Zunge\\
6.&125.&5425.&460.&20856.&22.&3&225&200\_20\_5 \textcolor{red}{\textcjheb{hkr}} RKH $|$gelinde/zarte\\
7.&126.&5426.&463.&20859.&25.&4&902&400\_300\_2\_200 \textcolor{red}{\textcjheb{rb+st}} TSBR $|$zerbricht/(sie) kann brechen\\
8.&127.&5427.&467.&20863.&29.&3&243&3\_200\_40 \textcolor{red}{\textcjheb{mrg}} GRM $|$Knochen/(das) Gebein\\
\end{tabular}\medskip \\
Ende des Verses 25.15\\
Verse: 728, Buchstaben: 31, 469, 20865, Totalwerte: 2861, 32412, 1478164\\
\\
Ein Richter wird "uberredet durch Langmut, und eine gelinde Zunge zerbricht Knochen.\\
\newpage 
{\bf -- 25.16}\\
\medskip \\
\begin{tabular}{rrrrrrrrp{120mm}}
WV&WK&WB&ABK&ABB&ABV&AnzB&TW&Zahlencode \textcolor{red}{$\boldsymbol{Grundtext}$} Umschrift $|$"Ubersetzung(en)\\
1.&128.&5428.&470.&20866.&1.&3&306&4\_2\_300 \textcolor{red}{\textcjheb{+sbd}} DBS $|$Honig\\
2.&129.&5429.&473.&20869.&4.&4&531&40\_90\_1\_400 \textcolor{red}{\textcjheb{t'.sm}} M"sAT $|$hast du gefunden\\
3.&130.&5430.&477.&20873.&8.&3&51&1\_20\_30 \textcolor{red}{\textcjheb{lk'}} AKL $|$(so) iss\\
4.&131.&5431.&480.&20876.&11.&3&34&4\_10\_20 \textcolor{red}{\textcjheb{kyd}} DJK $|$dein Gen"uge/nach Bedarf\\
5.&132.&5432.&483.&20879.&14.&2&130&80\_50 \textcolor{red}{\textcjheb{np}} PN $|$damit nicht/dass nicht\\
6.&133.&5433.&485.&20881.&16.&6&828&400\_300\_2\_70\_50\_6 \textcolor{red}{\textcjheb{wn`b+st}} TSBaNW $|$du seiner satt werdest/du satt bekommst ihn\\
7.&134.&5434.&491.&20887.&22.&6&518&6\_5\_100\_1\_400\_6 \textcolor{red}{\textcjheb{wt'qhw}} WHQATW $|$und ihn ausspeist/und du erbrichst ihn\\
\end{tabular}\medskip \\
Ende des Verses 25.16\\
Verse: 729, Buchstaben: 27, 496, 20892, Totalwerte: 2398, 34810, 1480562\\
\\
Hast du Honig gefunden, so i"s dein Gen"uge, damit du seiner nicht satt werdest und ihn ausspeiest.\\
\newpage 
{\bf -- 25.17}\\
\medskip \\
\begin{tabular}{rrrrrrrrp{120mm}}
WV&WK&WB&ABK&ABB&ABV&AnzB&TW&Zahlencode \textcolor{red}{$\boldsymbol{Grundtext}$} Umschrift $|$"Ubersetzung(en)\\
1.&135.&5435.&497.&20893.&1.&3&305&5\_100\_200 \textcolor{red}{\textcjheb{rqh}} HQR $|$mache selten\\
2.&136.&5436.&500.&20896.&4.&4&253&200\_3\_30\_20 \textcolor{red}{\textcjheb{klgr}} RGLK $|$deinen Fu"s\\
3.&137.&5437.&504.&20900.&8.&4&452&40\_2\_10\_400 \textcolor{red}{\textcjheb{tybm}} MBJT $|$im Haus/in das Haus\\
4.&138.&5438.&508.&20904.&12.&3&290&200\_70\_20 \textcolor{red}{\textcjheb{k`r}} RaK $|$deines N"achsten/deines Nachbarn\\
5.&139.&5439.&511.&20907.&15.&2&130&80\_50 \textcolor{red}{\textcjheb{np}} PN $|$damit nicht/dass nicht\\
6.&140.&5440.&513.&20909.&17.&5&402&10\_300\_2\_70\_20 \textcolor{red}{\textcjheb{k`b+sy}} JSBaK $|$er deiner satt werde/er habe dich satt\\
7.&141.&5441.&518.&20914.&22.&5&377&6\_300\_50\_1\_20 \textcolor{red}{\textcjheb{k'n+sw}} WSNAK $|$und (er) hass(t)e dich\\
\end{tabular}\medskip \\
Ende des Verses 25.17\\
Verse: 730, Buchstaben: 26, 522, 20918, Totalwerte: 2209, 37019, 1482771\\
\\
Mache deinen Fu"s selten im Hause deines N"achsten, damit er deiner nicht satt werde und dich hasse.\\
\newpage 
{\bf -- 25.18}\\
\medskip \\
\begin{tabular}{rrrrrrrrp{120mm}}
WV&WK&WB&ABK&ABB&ABV&AnzB&TW&Zahlencode \textcolor{red}{$\boldsymbol{Grundtext}$} Umschrift $|$"Ubersetzung(en)\\
1.&142.&5442.&523.&20919.&1.&4&220&40\_80\_10\_90 \textcolor{red}{\textcjheb{.sypm}} MPJ"s $|$(wie ein) (Streit)Hammer\\
2.&143.&5443.&527.&20923.&5.&4&216&6\_8\_200\_2 \textcolor{red}{\textcjheb{br.hw}} WCRB $|$und (ein) Schwert\\
3.&144.&5444.&531.&20927.&9.&3&104&6\_8\_90 \textcolor{red}{\textcjheb{.s.hw}} WC"s $|$und (ein) Pfeil\\
4.&145.&5445.&534.&20930.&12.&4&406&300\_50\_6\_50 \textcolor{red}{\textcjheb{nwn+s}} SNWN $|$gesch"arfter/spitzer\\
5.&146.&5446.&538.&20934.&16.&3&311&1\_10\_300 \textcolor{red}{\textcjheb{+sy'}} AJS $|$so ist ein Mann/(ist) jemand\\
6.&147.&5447.&541.&20937.&19.&3&125&70\_50\_5 \textcolor{red}{\textcjheb{hn`}} aNH $|$der ablegt/der antwortend ist\\
7.&148.&5448.&544.&20940.&22.&5&283&2\_200\_70\_5\_6 \textcolor{red}{\textcjheb{wh`rb}} BRaHW $|$wider seinen N"achsten/gegen seinen Gef"ahrten\\
8.&149.&5449.&549.&20945.&27.&2&74&70\_4 \textcolor{red}{\textcjheb{d`}} aD $|$Zeugnis/(als) Zeuge\\
9.&150.&5450.&551.&20947.&29.&3&600&300\_100\_200 \textcolor{red}{\textcjheb{rq+s}} SQR $|$falsches/der L"uge\\
\end{tabular}\medskip \\
Ende des Verses 25.18\\
Verse: 731, Buchstaben: 31, 553, 20949, Totalwerte: 2339, 39358, 1485110\\
\\
Hammer und Schwert und gesch"arfter Pfeil: so ist ein Mann, der wider seinen N"achsten falsches Zeugnis ablegt.\\
\newpage 
{\bf -- 25.19}\\
\medskip \\
\begin{tabular}{rrrrrrrrp{120mm}}
WV&WK&WB&ABK&ABB&ABV&AnzB&TW&Zahlencode \textcolor{red}{$\boldsymbol{Grundtext}$} Umschrift $|$"Ubersetzung(en)\\
1.&151.&5451.&554.&20950.&1.&2&350&300\_50 \textcolor{red}{\textcjheb{n+s}} SN $|$(wie) (ein) Zahn\\
2.&152.&5452.&556.&20952.&3.&3&275&200\_70\_5 \textcolor{red}{\textcjheb{h`r}} RaH $|$zerbrochener/zerbr"ockelnder\\
3.&153.&5453.&559.&20955.&6.&4&239&6\_200\_3\_30 \textcolor{red}{\textcjheb{lgrw}} WRGL $|$und ein Fu"s\\
4.&154.&5454.&563.&20959.&10.&5&520&40\_6\_70\_4\_400 \textcolor{red}{\textcjheb{td`wm}} MWaDT $|$wankender\\
5.&155.&5455.&568.&20964.&15.&4&59&40\_2\_9\_8 \textcolor{red}{\textcjheb{.h.tbm}} MBtC $|$so ist das Vertrauen/ist der Verlass\\
6.&156.&5456.&572.&20968.&19.&4&15&2\_6\_3\_4 \textcolor{red}{\textcjheb{dgwb}} BWGD $|$auf einen Treulosen\\
7.&157.&5457.&576.&20972.&23.&4&58&2\_10\_6\_40 \textcolor{red}{\textcjheb{mwyb}} BJWM $|$am Tag\\
8.&158.&5458.&580.&20976.&27.&3&295&90\_200\_5 \textcolor{red}{\textcjheb{hr.s}} "sRH $|$der Drangsal/(der) Bedr"angnis\\
\end{tabular}\medskip \\
Ende des Verses 25.19\\
Verse: 732, Buchstaben: 29, 582, 20978, Totalwerte: 1811, 41169, 1486921\\
\\
Ein zerbrochener Zahn und ein wankender Fu"s: so ist das Vertrauen auf einen Treulosen am Tage der Drangsal.\\
\newpage 
{\bf -- 25.20}\\
\medskip \\
\begin{tabular}{rrrrrrrrp{120mm}}
WV&WK&WB&ABK&ABB&ABV&AnzB&TW&Zahlencode \textcolor{red}{$\boldsymbol{Grundtext}$} Umschrift $|$"Ubersetzung(en)\\
1.&159.&5459.&583.&20979.&1.&4&119&40\_70\_4\_5 \textcolor{red}{\textcjheb{hd`m}} MaDH $|$einer der ablegt/wie ein Abstreifender\\
2.&160.&5460.&587.&20983.&5.&3&9&2\_3\_4 \textcolor{red}{\textcjheb{dgb}} BGD $|$das Oberkleid/(ein) Gewand\\
3.&161.&5461.&590.&20986.&8.&4&58&2\_10\_6\_40 \textcolor{red}{\textcjheb{mwyb}} BJWM $|$am Tag\\
4.&162.&5462.&594.&20990.&12.&3&305&100\_200\_5 \textcolor{red}{\textcjheb{hrq}} QRH $|$(der) K"alte\\
5.&163.&5463.&597.&20993.&15.&3&138&8\_40\_90 \textcolor{red}{\textcjheb{.sm.h}} CM"s $|$Essig/wie S"aure\\
6.&164.&5464.&600.&20996.&18.&2&100&70\_30 \textcolor{red}{\textcjheb{l`}} aL $|$auf\\
7.&165.&5465.&602.&20998.&20.&3&650&50\_400\_200 \textcolor{red}{\textcjheb{rtn}} NTR $|$Natron/Laugensalz\\
8.&166.&5466.&605.&21001.&23.&3&506&6\_300\_200 \textcolor{red}{\textcjheb{r+sw}} WSR $|$so wer singt/und so ist ein Singender\\
9.&167.&5467.&608.&21004.&26.&5&552&2\_300\_200\_10\_40 \textcolor{red}{\textcjheb{myr+sb}} BSRJM $|$(mit) Lieder(n)\\
10.&168.&5468.&613.&21009.&31.&2&100&70\_30 \textcolor{red}{\textcjheb{l`}} aL $|$/f"ur\\
11.&169.&5469.&615.&21011.&33.&2&32&30\_2 \textcolor{red}{\textcjheb{bl}} LB $|$(ein(em)) Herz(en)\\
12.&170.&5470.&617.&21013.&35.&2&270&200\_70 \textcolor{red}{\textcjheb{`r}} Ra $|$traurigen/vergr"amtes\\
\end{tabular}\medskip \\
Ende des Verses 25.20\\
Verse: 733, Buchstaben: 36, 618, 21014, Totalwerte: 2839, 44008, 1489760\\
\\
Einer, der das Oberkleid ablegt am Tage der K"alte, Essig auf Natron: so, wer einem traurigen Herzen Lieder singt.\\
\newpage 
{\bf -- 25.21}\\
\medskip \\
\begin{tabular}{rrrrrrrrp{120mm}}
WV&WK&WB&ABK&ABB&ABV&AnzB&TW&Zahlencode \textcolor{red}{$\boldsymbol{Grundtext}$} Umschrift $|$"Ubersetzung(en)\\
1.&171.&5471.&619.&21015.&1.&2&41&1\_40 \textcolor{red}{\textcjheb{m'}} AM $|$wenn\\
2.&172.&5472.&621.&21017.&3.&3&272&200\_70\_2 \textcolor{red}{\textcjheb{b`r}} RaB $|$hungert/ist hungrig\\
3.&173.&5473.&624.&21020.&6.&4&371&300\_50\_1\_20 \textcolor{red}{\textcjheb{k'n+s}} SNAK $|$deinen Hasser/dein Feind\\
4.&174.&5474.&628.&21024.&10.&6&67&5\_1\_20\_30\_5\_6 \textcolor{red}{\textcjheb{whlk'h}} HAKLHW $|$speise ihn mit/mache essen ihn\\
5.&175.&5475.&634.&21030.&16.&3&78&30\_8\_40 \textcolor{red}{\textcjheb{m.hl}} LCM $|$Brot\\
6.&176.&5476.&637.&21033.&19.&3&47&6\_1\_40 \textcolor{red}{\textcjheb{m'w}} WAM $|$und wenn\\
7.&177.&5477.&640.&21036.&22.&3&131&90\_40\_1 \textcolor{red}{\textcjheb{'m.s}} "sMA $|$ihn d"urstet/er durstig (ist)\\
8.&178.&5478.&643.&21039.&25.&5&416&5\_300\_100\_5\_6 \textcolor{red}{\textcjheb{whq+sh}} HSQHW $|$tr"anke ihn mit/mache trinken ihn\\
9.&179.&5479.&648.&21044.&30.&3&90&40\_10\_40 \textcolor{red}{\textcjheb{mym}} MJM $|$Wasser\\
\end{tabular}\medskip \\
Ende des Verses 25.21\\
Verse: 734, Buchstaben: 32, 650, 21046, Totalwerte: 1513, 45521, 1491273\\
\\
Wenn deinen Hasser hungert, speise ihn mit Brot, und wenn ihn d"urstet, tr"anke ihn mit Wasser;\\
\newpage 
{\bf -- 25.22}\\
\medskip \\
\begin{tabular}{rrrrrrrrp{120mm}}
WV&WK&WB&ABK&ABB&ABV&AnzB&TW&Zahlencode \textcolor{red}{$\boldsymbol{Grundtext}$} Umschrift $|$"Ubersetzung(en)\\
1.&180.&5480.&651.&21047.&1.&2&30&20\_10 \textcolor{red}{\textcjheb{yk}} KJ $|$denn\\
2.&181.&5481.&653.&21049.&3.&5&91&3\_8\_30\_10\_40 \textcolor{red}{\textcjheb{myl.hg}} GCLJM $|$gl"uhende Kohlen/(Gl"uh)Kohlen\\
3.&182.&5482.&658.&21054.&8.&3&406&1\_400\_5 \textcolor{red}{\textcjheb{ht'}} ATH $|$du wirst/du (bist)\\
4.&183.&5483.&661.&21057.&11.&3&413&8\_400\_5 \textcolor{red}{\textcjheb{ht.h}} CTH $|$h"aufen(d)\\
5.&184.&5484.&664.&21060.&14.&2&100&70\_30 \textcolor{red}{\textcjheb{l`}} aL $|$auf\\
6.&185.&5485.&666.&21062.&16.&4&507&200\_1\_300\_6 \textcolor{red}{\textcjheb{w+s'r}} RASW $|$sein Haupt\\
7.&186.&5486.&670.&21066.&20.&5&32&6\_10\_5\_6\_5 \textcolor{red}{\textcjheb{hwhyw}} WJHWH $|$und Jahwe\\
8.&187.&5487.&675.&21071.&25.&4&380&10\_300\_30\_40 \textcolor{red}{\textcjheb{ml+sy}} JSLM $|$wird vergelten/(er) vergilt (es)\\
9.&188.&5488.&679.&21075.&29.&2&50&30\_20 \textcolor{red}{\textcjheb{kl}} LK $|$dir\\
\end{tabular}\medskip \\
Ende des Verses 25.22\\
Verse: 735, Buchstaben: 30, 680, 21076, Totalwerte: 2009, 47530, 1493282\\
\\
denn gl"uhende Kohlen wirst du auf sein Haupt h"aufen, und Jahwe wird dir vergelten.\\
\newpage 
{\bf -- 25.23}\\
\medskip \\
\begin{tabular}{rrrrrrrrp{120mm}}
WV&WK&WB&ABK&ABB&ABV&AnzB&TW&Zahlencode \textcolor{red}{$\boldsymbol{Grundtext}$} Umschrift $|$"Ubersetzung(en)\\
1.&189.&5489.&681.&21077.&1.&3&214&200\_6\_8 \textcolor{red}{\textcjheb{.hwr}} RWC $|$(der) Wind\\
2.&190.&5490.&684.&21080.&4.&4&226&90\_80\_6\_50 \textcolor{red}{\textcjheb{nwp.s}} "sPWN $|$(aus dem) Nord(en)\\
3.&191.&5491.&688.&21084.&8.&5&474&400\_8\_6\_30\_30 \textcolor{red}{\textcjheb{llw.ht}} TCWLL $|$gebiert/sie (=er) erzeugt\\
4.&192.&5492.&693.&21089.&13.&3&343&3\_300\_40 \textcolor{red}{\textcjheb{m+sg}} GSM $|$(einen) Regen(guss)\\
5.&193.&5493.&696.&21092.&16.&5&186&6\_80\_50\_10\_40 \textcolor{red}{\textcjheb{mynpw}} WPNJM $|$und Gesichter/und Mienen\\
6.&194.&5494.&701.&21097.&21.&6&217&50\_7\_70\_40\_10\_40 \textcolor{red}{\textcjheb{mym`zn}} NZaMJM $|$verdrie"sliche/"argerliche\\
7.&195.&5495.&707.&21103.&27.&4&386&30\_300\_6\_50 \textcolor{red}{\textcjheb{nw+sl}} LSWN $|$(erzeugen) (eine) Zunge\\
8.&196.&5496.&711.&21107.&31.&3&660&60\_400\_200 \textcolor{red}{\textcjheb{rts}} sTR $|$heimliche/des Verstecks\\
\end{tabular}\medskip \\
Ende des Verses 25.23\\
Verse: 736, Buchstaben: 33, 713, 21109, Totalwerte: 2706, 50236, 1495988\\
\\
Nordwind gebiert Regen, und eine heimliche Zunge verdrie"sliche Gesichter.\\
\newpage 
{\bf -- 25.24}\\
\medskip \\
\begin{tabular}{rrrrrrrrp{120mm}}
WV&WK&WB&ABK&ABB&ABV&AnzB&TW&Zahlencode \textcolor{red}{$\boldsymbol{Grundtext}$} Umschrift $|$"Ubersetzung(en)\\
1.&197.&5497.&714.&21110.&1.&3&17&9\_6\_2 \textcolor{red}{\textcjheb{bw.t}} tWB $|$besser/er (=es) ist gut\\
2.&198.&5498.&717.&21113.&4.&3&702&300\_2\_400 \textcolor{red}{\textcjheb{tb+s}} SBT $|$(zu) wohnen\\
3.&199.&5499.&720.&21116.&7.&2&100&70\_30 \textcolor{red}{\textcjheb{l`}} aL $|$auf\\
4.&200.&5500.&722.&21118.&9.&3&530&80\_50\_400 \textcolor{red}{\textcjheb{tnp}} PNT $|$einer Ecke/der Zinne\\
5.&201.&5501.&725.&21121.&12.&2&6&3\_3 \textcolor{red}{\textcjheb{gg}} GG $|$(des) Dach(es)\\
6.&202.&5502.&727.&21123.&14.&4&741&40\_1\_300\_400 \textcolor{red}{\textcjheb{t+s'm}} MAST $|$als eine Frau\\
7.&203.&5503.&731.&21127.&18.&6&150&40\_4\_6\_50\_10\_40 \textcolor{red}{\textcjheb{mynwdm}} MDWNJM $|$z"ankische/von Streitigkeiten\\
8.&204.&5504.&737.&21133.&24.&4&418&6\_2\_10\_400 \textcolor{red}{\textcjheb{tybw}} WBJT $|$und ein Haus\\
9.&205.&5505.&741.&21137.&28.&3&210&8\_2\_200 \textcolor{red}{\textcjheb{rb.h}} CBR $|$gemeinsames\\
\end{tabular}\medskip \\
Ende des Verses 25.24\\
Verse: 737, Buchstaben: 30, 743, 21139, Totalwerte: 2874, 53110, 1498862\\
\\
Besser auf einer Dachecke wohnen, als ein z"ankisches Weib und ein gemeinsames Haus.\\
\newpage 
{\bf -- 25.25}\\
\medskip \\
\begin{tabular}{rrrrrrrrp{120mm}}
WV&WK&WB&ABK&ABB&ABV&AnzB&TW&Zahlencode \textcolor{red}{$\boldsymbol{Grundtext}$} Umschrift $|$"Ubersetzung(en)\\
1.&206.&5506.&744.&21140.&1.&3&90&40\_10\_40 \textcolor{red}{\textcjheb{mym}} MJM $|$(wie) Wasser\\
2.&207.&5507.&747.&21143.&4.&4&350&100\_200\_10\_40 \textcolor{red}{\textcjheb{myrq}} QRJM $|$frisches/k"uhles\\
3.&208.&5508.&751.&21147.&8.&2&100&70\_30 \textcolor{red}{\textcjheb{l`}} aL $|$auf/f"ur\\
4.&209.&5509.&753.&21149.&10.&3&430&50\_80\_300 \textcolor{red}{\textcjheb{+spn}} NPS $|$(eine) Seele\\
5.&210.&5510.&756.&21152.&13.&4&165&70\_10\_80\_5 \textcolor{red}{\textcjheb{hpy`}} aJPH $|$lechzende/m"ude\\
6.&211.&5511.&760.&21156.&17.&6&427&6\_300\_40\_6\_70\_5 \textcolor{red}{\textcjheb{h`wm+sw}} WSMWaH $|$so eine Nachricht/und eine Kunde\\
7.&212.&5512.&766.&21162.&23.&4&22&9\_6\_2\_5 \textcolor{red}{\textcjheb{hbw.t}} tWBH $|$gute\\
8.&213.&5513.&770.&21166.&27.&4&331&40\_1\_200\_90 \textcolor{red}{\textcjheb{.sr'm}} MAR"s $|$aus (dem) Lande\\
9.&214.&5514.&774.&21170.&31.&4&348&40\_200\_8\_100 \textcolor{red}{\textcjheb{q.hrm}} MRCQ $|$fernem/der Ferne\\
\end{tabular}\medskip \\
Ende des Verses 25.25\\
Verse: 738, Buchstaben: 34, 777, 21173, Totalwerte: 2263, 55373, 1501125\\
\\
Frisches Wasser auf eine lechzende Seele: so eine gute Nachricht aus fernem Lande.\\
\newpage 
{\bf -- 25.26}\\
\medskip \\
\begin{tabular}{rrrrrrrrp{120mm}}
WV&WK&WB&ABK&ABB&ABV&AnzB&TW&Zahlencode \textcolor{red}{$\boldsymbol{Grundtext}$} Umschrift $|$"Ubersetzung(en)\\
1.&215.&5515.&778.&21174.&1.&4&170&40\_70\_10\_50 \textcolor{red}{\textcjheb{ny`m}} MaJN $|$(wie ein) Quell\\
2.&216.&5516.&782.&21178.&5.&4&630&50\_200\_80\_300 \textcolor{red}{\textcjheb{+sprn}} NRPS $|$getr"ubter\\
3.&217.&5517.&786.&21182.&9.&5&352&6\_40\_100\_6\_200 \textcolor{red}{\textcjheb{rwqmw}} WMQWR $|$und Brunnen/und (ein) Born\\
4.&218.&5518.&791.&21187.&14.&4&748&40\_300\_8\_400 \textcolor{red}{\textcjheb{t.h+sm}} MSCT $|$verderbter/verdorbener\\
5.&219.&5519.&795.&21191.&18.&4&204&90\_4\_10\_100 \textcolor{red}{\textcjheb{qyd.s}} "sDJQ $|$so ist der Gerechte/(ist) (ein) Gerechter\\
6.&220.&5520.&799.&21195.&22.&2&49&40\_9 \textcolor{red}{\textcjheb{.tm}} Mt $|$der wankt/wankend\\
7.&221.&5521.&801.&21197.&24.&4&170&30\_80\_50\_10 \textcolor{red}{\textcjheb{ynpl}} LPNJ $|$vor\\
8.&222.&5522.&805.&21201.&28.&3&570&200\_300\_70 \textcolor{red}{\textcjheb{`+sr}} RSa $|$dem Gesetzlosen/(einem) Frevler\\
\end{tabular}\medskip \\
Ende des Verses 25.26\\
Verse: 739, Buchstaben: 30, 807, 21203, Totalwerte: 2893, 58266, 1504018\\
\\
Getr"ubter Quell und verderbter Brunnen: so ist der Gerechte, der vor dem Gesetzlosen wankt.\\
\newpage 
{\bf -- 25.27}\\
\medskip \\
\begin{tabular}{rrrrrrrrp{120mm}}
WV&WK&WB&ABK&ABB&ABV&AnzB&TW&Zahlencode \textcolor{red}{$\boldsymbol{Grundtext}$} Umschrift $|$"Ubersetzung(en)\\
1.&223.&5523.&808.&21204.&1.&3&51&1\_20\_30 \textcolor{red}{\textcjheb{lk'}} AKL $|$essen\\
2.&224.&5524.&811.&21207.&4.&3&306&4\_2\_300 \textcolor{red}{\textcjheb{+sbd}} DBS $|$Honig\\
3.&225.&5525.&814.&21210.&7.&5&613&5\_200\_2\_6\_400 \textcolor{red}{\textcjheb{twbrh}} HRBWT $|$(zu) viel\\
4.&226.&5526.&819.&21215.&12.&2&31&30\_1 \textcolor{red}{\textcjheb{'l}} LA $|$nicht\\
5.&227.&5527.&821.&21217.&14.&3&17&9\_6\_2 \textcolor{red}{\textcjheb{bw.t}} tWB $|$ist gut\\
6.&228.&5528.&824.&21220.&17.&4&314&6\_8\_100\_200 \textcolor{red}{\textcjheb{rq.hw}} WCQR $|$aber erforschen/und "Uberpr"ufung\\
7.&229.&5529.&828.&21224.&21.&4&66&20\_2\_4\_40 \textcolor{red}{\textcjheb{mdbk}} KBDM $|$schwere Dinge/ihrer Wertigkeit\\
8.&230.&5530.&832.&21228.&25.&4&32&20\_2\_6\_4 \textcolor{red}{\textcjheb{dwbk}} KBWD $|$(ist) (eine) Ehre\\
\end{tabular}\medskip \\
Ende des Verses 25.27\\
Verse: 740, Buchstaben: 28, 835, 21231, Totalwerte: 1430, 59696, 1505448\\
\\
Viel Honig essen ist nicht gut, aber schwere Dinge erforschen ist Ehre.\\
\newpage 
{\bf -- 25.28}\\
\medskip \\
\begin{tabular}{rrrrrrrrp{120mm}}
WV&WK&WB&ABK&ABB&ABV&AnzB&TW&Zahlencode \textcolor{red}{$\boldsymbol{Grundtext}$} Umschrift $|$"Ubersetzung(en)\\
1.&231.&5531.&836.&21232.&1.&3&280&70\_10\_200 \textcolor{red}{\textcjheb{ry`}} aJR $|$(wie) (eine) Stadt\\
2.&232.&5532.&839.&21235.&4.&5&381&80\_200\_6\_90\_5 \textcolor{red}{\textcjheb{h.swrp}} PRW"sH $|$erbrochene/eingerissene\\
3.&233.&5533.&844.&21240.&9.&3&61&1\_10\_50 \textcolor{red}{\textcjheb{ny'}} AJN $|$ohne\\
4.&234.&5534.&847.&21243.&12.&4&59&8\_6\_40\_5 \textcolor{red}{\textcjheb{hmw.h}} CWMH $|$Mauer\\
5.&235.&5535.&851.&21247.&16.&3&311&1\_10\_300 \textcolor{red}{\textcjheb{+sy'}} AJS $|$so ist der Mann/(so) (ist) jemand\\
6.&236.&5536.&854.&21250.&19.&3&501&1\_300\_200 \textcolor{red}{\textcjheb{r+s'}} ASR $|$dessen/der (ist)\\
7.&237.&5537.&857.&21253.&22.&3&61&1\_10\_50 \textcolor{red}{\textcjheb{ny'}} AJN $|$mangelt/ohne\\
8.&238.&5538.&860.&21256.&25.&4&400&40\_70\_90\_200 \textcolor{red}{\textcjheb{r.s`m}} Ma"sR $|$Beherrschung\\
9.&239.&5539.&864.&21260.&29.&5&250&30\_200\_6\_8\_6 \textcolor{red}{\textcjheb{w.hwrl}} LRWCW $|$(seines) Geist(es)\\
\end{tabular}\medskip \\
Ende des Verses 25.28\\
Verse: 741, Buchstaben: 33, 868, 21264, Totalwerte: 2304, 62000, 1507752\\
\\
Eine erbrochene Stadt ohne Mauer: so ist ein Mann, dessen Geist Beherrschung mangelt.\\
\\
{\bf Ende des Kapitels 25}\\
\newpage 
{\bf -- 26.1}\\
\medskip \\
\begin{tabular}{rrrrrrrrp{120mm}}
WV&WK&WB&ABK&ABB&ABV&AnzB&TW&Zahlencode \textcolor{red}{$\boldsymbol{Grundtext}$} Umschrift $|$"Ubersetzung(en)\\
1.&1.&5540.&1.&21265.&1.&4&353&20\_300\_30\_3 \textcolor{red}{\textcjheb{gl+sk}} KSLG $|$wie (der) Schnee\\
2.&2.&5541.&5.&21269.&5.&4&202&2\_100\_10\_90 \textcolor{red}{\textcjheb{.syqb}} BQJ"s $|$im Sommer\\
3.&3.&5542.&9.&21273.&9.&5&275&6\_20\_40\_9\_200 \textcolor{red}{\textcjheb{r.tmkw}} WKMtR $|$und wie (der) Regen\\
4.&4.&5543.&14.&21278.&14.&5&402&2\_100\_90\_10\_200 \textcolor{red}{\textcjheb{ry.sqb}} BQ"sJR $|$in der Ernte\\
5.&5.&5544.&19.&21283.&19.&2&70&20\_50 \textcolor{red}{\textcjheb{nk}} KN $|$so\\
6.&6.&5545.&21.&21285.&21.&2&31&30\_1 \textcolor{red}{\textcjheb{'l}} LA $|$nicht\\
7.&7.&5546.&23.&21287.&23.&4&62&50\_1\_6\_5 \textcolor{red}{\textcjheb{hw'n}} NAWH $|$(ist) (er (=es)) geziemend\\
8.&8.&5547.&27.&21291.&27.&5&150&30\_20\_60\_10\_30 \textcolor{red}{\textcjheb{lyskl}} LKsJL $|$dem Toren/f"ur den Toren\\
9.&9.&5548.&32.&21296.&32.&4&32&20\_2\_6\_4 \textcolor{red}{\textcjheb{dwbk}} KBWD $|$Ehre\\
\end{tabular}\medskip \\
Ende des Verses 26.1\\
Verse: 742, Buchstaben: 35, 35, 21299, Totalwerte: 1577, 1577, 1509329\\
\\
Wie Schnee im Sommer und wie Regen in der Ernte, so ist Ehre dem Toren nicht geziemend.\\
\newpage 
{\bf -- 26.2}\\
\medskip \\
\begin{tabular}{rrrrrrrrp{120mm}}
WV&WK&WB&ABK&ABB&ABV&AnzB&TW&Zahlencode \textcolor{red}{$\boldsymbol{Grundtext}$} Umschrift $|$"Ubersetzung(en)\\
1.&10.&5549.&36.&21300.&1.&5&396&20\_90\_80\_6\_200 \textcolor{red}{\textcjheb{rwp.sk}} K"sPWR $|$wie der Sperling/wie das V"ogelchen\\
2.&11.&5550.&41.&21305.&6.&4&90&30\_50\_6\_4 \textcolor{red}{\textcjheb{dwnl}} LNWD $|$hin und her flattert/im Flattern\\
3.&12.&5551.&45.&21309.&10.&5&430&20\_4\_200\_6\_200 \textcolor{red}{\textcjheb{rwrdk}} KDRWR $|$wie die Schwalbe\\
4.&13.&5552.&50.&21314.&15.&4&186&30\_70\_6\_80 \textcolor{red}{\textcjheb{pw`l}} LaWP $|$wegfliegt/im Fliegen\\
5.&14.&5553.&54.&21318.&19.&2&70&20\_50 \textcolor{red}{\textcjheb{nk}} KN $|$so (ist)\\
6.&15.&5554.&56.&21320.&21.&4&560&100\_30\_30\_400 \textcolor{red}{\textcjheb{tllq}} QLLT $|$(ein) Fluch\\
7.&16.&5555.&60.&21324.&25.&3&98&8\_50\_40 \textcolor{red}{\textcjheb{mn.h}} CNM $|$unverdienter/ohne Grund\\
8.&17.&5556.&63.&21327.&28.&2&31&30\_1 \textcolor{red}{\textcjheb{'l}} LA $|$nicht\\
9.&18.&5557.&65.&21329.&30.&3&403&400\_2\_1 \textcolor{red}{\textcjheb{'bt}} TBA $|$er trifft ein/sie (=es) trifft ein\\
\end{tabular}\medskip \\
Ende des Verses 26.2\\
Verse: 743, Buchstaben: 32, 67, 21331, Totalwerte: 2264, 3841, 1511593\\
\\
Wie der Sperling hin und her flattert, wie die Schwalbe wegfliegt, so ein unverdienter Fluch: er trifft nicht ein.\\
\newpage 
{\bf -- 26.3}\\
\medskip \\
\begin{tabular}{rrrrrrrrp{120mm}}
WV&WK&WB&ABK&ABB&ABV&AnzB&TW&Zahlencode \textcolor{red}{$\boldsymbol{Grundtext}$} Umschrift $|$"Ubersetzung(en)\\
1.&19.&5558.&68.&21332.&1.&3&315&300\_6\_9 \textcolor{red}{\textcjheb{.tw+s}} SWt $|$die Peitsche\\
2.&20.&5559.&71.&21335.&4.&4&156&30\_60\_6\_60 \textcolor{red}{\textcjheb{swsl}} LsWs $|$dem Pferd/f"ur das Ross\\
3.&21.&5560.&75.&21339.&8.&3&443&40\_400\_3 \textcolor{red}{\textcjheb{gtm}} MTG $|$(der) Zaum\\
4.&22.&5561.&78.&21342.&11.&5&284&30\_8\_40\_6\_200 \textcolor{red}{\textcjheb{rwm.hl}} LCMWR $|$dem Esel/f"ur den Esel\\
5.&23.&5562.&83.&21347.&16.&4&317&6\_300\_2\_9 \textcolor{red}{\textcjheb{.tb+sw}} WSBt $|$und der Stock/und die Rute\\
6.&24.&5563.&87.&21351.&20.&3&39&30\_3\_6 \textcolor{red}{\textcjheb{wgl}} LGW $|$dem R"ucken/f"ur den R"ucken\\
7.&25.&5564.&90.&21354.&23.&6&170&20\_60\_10\_30\_10\_40 \textcolor{red}{\textcjheb{mylysk}} KsJLJM $|$der Toren\\
\end{tabular}\medskip \\
Ende des Verses 26.3\\
Verse: 744, Buchstaben: 28, 95, 21359, Totalwerte: 1724, 5565, 1513317\\
\\
Die Peitsche dem Pferde, der Zaum dem Esel, und der Stock dem R"ucken der Toren.\\
\newpage 
{\bf -- 26.4}\\
\medskip \\
\begin{tabular}{rrrrrrrrp{120mm}}
WV&WK&WB&ABK&ABB&ABV&AnzB&TW&Zahlencode \textcolor{red}{$\boldsymbol{Grundtext}$} Umschrift $|$"Ubersetzung(en)\\
1.&26.&5565.&96.&21360.&1.&2&31&1\_30 \textcolor{red}{\textcjheb{l'}} AL $|$nicht\\
2.&27.&5566.&98.&21362.&3.&3&520&400\_70\_50 \textcolor{red}{\textcjheb{n`t}} TaN $|$antworte/du sollst erwidern\\
3.&28.&5567.&101.&21365.&6.&4&120&20\_60\_10\_30 \textcolor{red}{\textcjheb{lysk}} KsJL $|$dem Toren/(einem) Toren\\
4.&29.&5568.&105.&21369.&10.&6&463&20\_1\_6\_30\_400\_6 \textcolor{red}{\textcjheb{wtlw'k}} KAWLTW $|$nach seiner Narrheit/gem"a"s seiner Narrheit\\
5.&30.&5569.&111.&21375.&16.&2&130&80\_50 \textcolor{red}{\textcjheb{np}} PN $|$damit nicht/dass nicht\\
6.&31.&5570.&113.&21377.&18.&4&711&400\_300\_6\_5 \textcolor{red}{\textcjheb{hw+st}} TSWH $|$du gleich(st) (werdest)\\
7.&32.&5571.&117.&21381.&22.&2&36&30\_6 \textcolor{red}{\textcjheb{wl}} LW $|$ihm\\
8.&33.&5572.&119.&21383.&24.&2&43&3\_40 \textcolor{red}{\textcjheb{mg}} GM $|$auch\\
9.&34.&5573.&121.&21385.&26.&3&406&1\_400\_5 \textcolor{red}{\textcjheb{ht'}} ATH $|$du (selber)\\
\end{tabular}\medskip \\
Ende des Verses 26.4\\
Verse: 745, Buchstaben: 28, 123, 21387, Totalwerte: 2460, 8025, 1515777\\
\\
Antworte dem Toren nicht nach seiner Narrheit, damit nicht auch du ihm gleich werdest.\\
\newpage 
{\bf -- 26.5}\\
\medskip \\
\begin{tabular}{rrrrrrrrp{120mm}}
WV&WK&WB&ABK&ABB&ABV&AnzB&TW&Zahlencode \textcolor{red}{$\boldsymbol{Grundtext}$} Umschrift $|$"Ubersetzung(en)\\
1.&35.&5574.&124.&21388.&1.&3&125&70\_50\_5 \textcolor{red}{\textcjheb{hn`}} aNH $|$antworte/erwidere\\
2.&36.&5575.&127.&21391.&4.&4&120&20\_60\_10\_30 \textcolor{red}{\textcjheb{lysk}} KsJL $|$dem Toren/(einem) Toren\\
3.&37.&5576.&131.&21395.&8.&6&463&20\_1\_6\_30\_400\_6 \textcolor{red}{\textcjheb{wtlw'k}} KAWLTW $|$nach seiner Narrheit/gem"a"s seiner Narrheit\\
4.&38.&5577.&137.&21401.&14.&2&130&80\_50 \textcolor{red}{\textcjheb{np}} PN $|$damit nicht/dass nicht\\
5.&39.&5578.&139.&21403.&16.&4&30&10\_5\_10\_5 \textcolor{red}{\textcjheb{hyhy}} JHJH $|$er sei\\
6.&40.&5579.&143.&21407.&20.&3&68&8\_20\_40 \textcolor{red}{\textcjheb{mk.h}} CKM $|$weise\\
7.&41.&5580.&146.&21410.&23.&6&148&2\_70\_10\_50\_10\_6 \textcolor{red}{\textcjheb{wyny`b}} BaJNJW $|$in seinen Augen\\
\end{tabular}\medskip \\
Ende des Verses 26.5\\
Verse: 746, Buchstaben: 28, 151, 21415, Totalwerte: 1084, 9109, 1516861\\
\\
Antworte dem Toren nach seiner Narrheit, damit er nicht weise sei in seinen Augen.\\
\newpage 
{\bf -- 26.6}\\
\medskip \\
\begin{tabular}{rrrrrrrrp{120mm}}
WV&WK&WB&ABK&ABB&ABV&AnzB&TW&Zahlencode \textcolor{red}{$\boldsymbol{Grundtext}$} Umschrift $|$"Ubersetzung(en)\\
1.&42.&5581.&152.&21416.&1.&4&235&40\_100\_90\_5 \textcolor{red}{\textcjheb{h.sqm}} MQ"sH $|$(es) haut sich ab/abhauend\\
2.&43.&5582.&156.&21420.&5.&5&283&200\_3\_30\_10\_40 \textcolor{red}{\textcjheb{mylgr}} RGLJM $|$(die) (beide) F"u"se\\
3.&44.&5583.&161.&21425.&10.&3&108&8\_40\_60 \textcolor{red}{\textcjheb{sm.h}} CMs $|$Unbill/Unrecht\\
4.&45.&5584.&164.&21428.&13.&3&705&300\_400\_5 \textcolor{red}{\textcjheb{ht+s}} STH $|$trinkt/(ist) trinkend(er)\\
5.&46.&5585.&167.&21431.&16.&3&338&300\_30\_8 \textcolor{red}{\textcjheb{.hl+s}} SLC $|$wer ausrichten l"asst/(ein) Sendender\\
6.&47.&5586.&170.&21434.&19.&5&256&4\_2\_200\_10\_40 \textcolor{red}{\textcjheb{myrbd}} DBRJM $|$Bestellungen/Worte\\
7.&48.&5587.&175.&21439.&24.&3&16&2\_10\_4 \textcolor{red}{\textcjheb{dyb}} BJD $|$durch\\
8.&49.&5588.&178.&21442.&27.&4&120&20\_60\_10\_30 \textcolor{red}{\textcjheb{lysk}} KsJL $|$(einen) Toren\\
\end{tabular}\medskip \\
Ende des Verses 26.6\\
Verse: 747, Buchstaben: 30, 181, 21445, Totalwerte: 2061, 11170, 1518922\\
\\
Die F"u"se haut sich ab, Unbill trinkt, wer Bestellungen ausrichten l"a"st durch einen Toren.\\
\newpage 
{\bf -- 26.7}\\
\medskip \\
\begin{tabular}{rrrrrrrrp{120mm}}
WV&WK&WB&ABK&ABB&ABV&AnzB&TW&Zahlencode \textcolor{red}{$\boldsymbol{Grundtext}$} Umschrift $|$"Ubersetzung(en)\\
1.&50.&5589.&182.&21446.&1.&4&50&4\_30\_10\_6 \textcolor{red}{\textcjheb{wyld}} DLJW $|$schlaff h"angen (herab)\\
2.&51.&5590.&186.&21450.&5.&4&450&300\_100\_10\_40 \textcolor{red}{\textcjheb{myq+s}} SQJM $|$die Beine/(beide) Schenkel\\
3.&52.&5591.&190.&21454.&9.&4&188&40\_80\_60\_8 \textcolor{red}{\textcjheb{.hspm}} MPsC $|$des Lahmen/von einem Lahmen\\
4.&53.&5592.&194.&21458.&13.&4&376&6\_40\_300\_30 \textcolor{red}{\textcjheb{l+smw}} WMSL $|$so ein Spruch/und (ein) Spruch\\
5.&54.&5593.&198.&21462.&17.&3&92&2\_80\_10 \textcolor{red}{\textcjheb{ypb}} BPJ $|$im Mund\\
6.&55.&5594.&201.&21465.&20.&6&170&20\_60\_10\_30\_10\_40 \textcolor{red}{\textcjheb{mylysk}} KsJLJM $|$der Toren/(von) Toren\\
\end{tabular}\medskip \\
Ende des Verses 26.7\\
Verse: 748, Buchstaben: 25, 206, 21470, Totalwerte: 1326, 12496, 1520248\\
\\
Schlaff h"angen die Beine des Lahmen herab: so ein Spruch im Munde der Toren.\\
\newpage 
{\bf -- 26.8}\\
\medskip \\
\begin{tabular}{rrrrrrrrp{120mm}}
WV&WK&WB&ABK&ABB&ABV&AnzB&TW&Zahlencode \textcolor{red}{$\boldsymbol{Grundtext}$} Umschrift $|$"Ubersetzung(en)\\
1.&56.&5595.&207.&21471.&1.&5&516&20\_90\_200\_6\_200 \textcolor{red}{\textcjheb{rwr.sk}} K"sRWR $|$wie das Binden/wie (ein) Kiesel\\
2.&57.&5596.&212.&21476.&6.&3&53&1\_2\_50 \textcolor{red}{\textcjheb{nb'}} ABN $|$(eines) Stein(es)\\
3.&58.&5597.&215.&21479.&9.&6&290&2\_40\_200\_3\_40\_5 \textcolor{red}{\textcjheb{hmgrmb}} BMRGMH $|$in eine Schleuder/an einer Schleuder\\
4.&59.&5598.&221.&21485.&15.&2&70&20\_50 \textcolor{red}{\textcjheb{nk}} KN $|$so (ist)\\
5.&60.&5599.&223.&21487.&17.&4&506&50\_6\_400\_50 \textcolor{red}{\textcjheb{ntwn}} NWTN $|$wer erweist/ein Gebender\\
6.&61.&5600.&227.&21491.&21.&5&150&30\_20\_60\_10\_30 \textcolor{red}{\textcjheb{lyskl}} LKsJL $|$einem Toren/dem Toren\\
7.&62.&5601.&232.&21496.&26.&4&32&20\_2\_6\_4 \textcolor{red}{\textcjheb{dwbk}} KBWD $|$Ehre\\
\end{tabular}\medskip \\
Ende des Verses 26.8\\
Verse: 749, Buchstaben: 29, 235, 21499, Totalwerte: 1617, 14113, 1521865\\
\\
Wie das Binden eines Steines in eine Schleuder: so, wer einem Toren Ehre erweist.\\
\newpage 
{\bf -- 26.9}\\
\medskip \\
\begin{tabular}{rrrrrrrrp{120mm}}
WV&WK&WB&ABK&ABB&ABV&AnzB&TW&Zahlencode \textcolor{red}{$\boldsymbol{Grundtext}$} Umschrift $|$"Ubersetzung(en)\\
1.&63.&5602.&236.&21500.&1.&3&22&8\_6\_8 \textcolor{red}{\textcjheb{.hw.h}} CWC $|$(wie) ein Dorn\\
2.&64.&5603.&239.&21503.&4.&3&105&70\_30\_5 \textcolor{red}{\textcjheb{hl`}} aLH $|$(d)er ger"at\\
3.&65.&5604.&242.&21506.&7.&3&16&2\_10\_4 \textcolor{red}{\textcjheb{dyb}} BJD $|$in die Hand\\
4.&66.&5605.&245.&21509.&10.&4&526&300\_20\_6\_200 \textcolor{red}{\textcjheb{rwk+s}} SKWR $|$eines Trunkenen/(eines) Betrunkenen\\
5.&67.&5606.&249.&21513.&14.&4&376&6\_40\_300\_30 \textcolor{red}{\textcjheb{l+smw}} WMSL $|$so ein Spruch/und (der) Spruch\\
6.&68.&5607.&253.&21517.&18.&3&92&2\_80\_10 \textcolor{red}{\textcjheb{ypb}} BPJ $|$im Munde/in den Mund\\
7.&69.&5608.&256.&21520.&21.&6&170&20\_60\_10\_30\_10\_40 \textcolor{red}{\textcjheb{mylysk}} KsJLJM $|$der Toren\\
\end{tabular}\medskip \\
Ende des Verses 26.9\\
Verse: 750, Buchstaben: 26, 261, 21525, Totalwerte: 1307, 15420, 1523172\\
\\
Ein Dorn, der in die Hand eines Trunkenen ger"at: so ein Spruch im Munde der Toren.\\
\newpage 
{\bf -- 26.10}\\
\medskip \\
\begin{tabular}{rrrrrrrrp{120mm}}
WV&WK&WB&ABK&ABB&ABV&AnzB&TW&Zahlencode \textcolor{red}{$\boldsymbol{Grundtext}$} Umschrift $|$"Ubersetzung(en)\\
1.&70.&5609.&262.&21526.&1.&2&202&200\_2 \textcolor{red}{\textcjheb{br}} RB $|$(wie) ein Sch"utze\\
2.&71.&5610.&264.&21528.&3.&5&114&40\_8\_6\_30\_30 \textcolor{red}{\textcjheb{llw.hm}} MCWLL $|$der verwundet/(ist) beben machend\\
3.&72.&5611.&269.&21533.&8.&2&50&20\_30 \textcolor{red}{\textcjheb{lk}} KL $|$alles\\
4.&73.&5612.&271.&21535.&10.&4&526&6\_300\_20\_200 \textcolor{red}{\textcjheb{rk+sw}} WSKR $|$so wer dingt/und ein Dingender\\
5.&74.&5613.&275.&21539.&14.&4&120&20\_60\_10\_30 \textcolor{red}{\textcjheb{lysk}} KsJL $|$den Toren/(einen) Toren\\
6.&75.&5614.&279.&21543.&18.&4&526&6\_300\_20\_200 \textcolor{red}{\textcjheb{rk+sw}} WSKR $|$und dingt/und ein Dingender\\
7.&76.&5615.&283.&21547.&22.&5&322&70\_2\_200\_10\_40 \textcolor{red}{\textcjheb{myrb`}} aBRJM $|$die Vor"ubergehenden/Herumziehende\\
\end{tabular}\medskip \\
Ende des Verses 26.10\\
Verse: 751, Buchstaben: 26, 287, 21551, Totalwerte: 1860, 17280, 1525032\\
\\
Ein Sch"utze, der alles verwundet: so, wer den Toren dingt und die Vor"ubergehenden dingt.\\
\newpage 
{\bf -- 26.11}\\
\medskip \\
\begin{tabular}{rrrrrrrrp{120mm}}
WV&WK&WB&ABK&ABB&ABV&AnzB&TW&Zahlencode \textcolor{red}{$\boldsymbol{Grundtext}$} Umschrift $|$"Ubersetzung(en)\\
1.&77.&5616.&288.&21552.&1.&4&72&20\_20\_30\_2 \textcolor{red}{\textcjheb{blkk}} KKLB $|$wie ein Hund\\
2.&78.&5617.&292.&21556.&5.&2&302&300\_2 \textcolor{red}{\textcjheb{b+s}} SB $|$der zur"uckkehrt/(ist) zur"uckkehrend\\
3.&79.&5618.&294.&21558.&7.&2&100&70\_30 \textcolor{red}{\textcjheb{l`}} aL $|$zu\\
4.&80.&5619.&296.&21560.&9.&3&107&100\_1\_6 \textcolor{red}{\textcjheb{w'q}} QAW $|$seinem Gespei\\
5.&81.&5620.&299.&21563.&12.&4&120&20\_60\_10\_30 \textcolor{red}{\textcjheb{lysk}} KsJL $|$(so ist) (ein) Tor\\
6.&82.&5621.&303.&21567.&16.&4&361&300\_6\_50\_5 \textcolor{red}{\textcjheb{hnw+s}} SWNH $|$der wiederholt/wiederholend\\
7.&83.&5622.&307.&21571.&20.&6&445&2\_1\_6\_30\_400\_6 \textcolor{red}{\textcjheb{wtlw'b}} BAWLTW $|$seine Narrheit/in seiner Narredei\\
\end{tabular}\medskip \\
Ende des Verses 26.11\\
Verse: 752, Buchstaben: 25, 312, 21576, Totalwerte: 1507, 18787, 1526539\\
\\
Wie ein Hund, der zur"uckkehrt zu seinem Gespei: so ist ein Tor, der seine Narrheit wiederholt.\\
\newpage 
{\bf -- 26.12}\\
\medskip \\
\begin{tabular}{rrrrrrrrp{120mm}}
WV&WK&WB&ABK&ABB&ABV&AnzB&TW&Zahlencode \textcolor{red}{$\boldsymbol{Grundtext}$} Umschrift $|$"Ubersetzung(en)\\
1.&84.&5623.&313.&21577.&1.&4&611&200\_1\_10\_400 \textcolor{red}{\textcjheb{ty'r}} RAJT $|$du siehst\\
2.&85.&5624.&317.&21581.&5.&3&311&1\_10\_300 \textcolor{red}{\textcjheb{+sy'}} AJS $|$(einen) Mann\\
3.&86.&5625.&320.&21584.&8.&3&68&8\_20\_40 \textcolor{red}{\textcjheb{mk.h}} CKM $|$der weise ist/weisen\\
4.&87.&5626.&323.&21587.&11.&6&148&2\_70\_10\_50\_10\_6 \textcolor{red}{\textcjheb{wyny`b}} BaJNJW $|$in seinen Augen\\
5.&88.&5627.&329.&21593.&17.&4&511&400\_100\_6\_5 \textcolor{red}{\textcjheb{hwqt}} TQWH $|$Hoffnung\\
6.&89.&5628.&333.&21597.&21.&5&150&30\_20\_60\_10\_30 \textcolor{red}{\textcjheb{lyskl}} LKsJL $|$f"ur einen Toren ist/(ist) f"ur den Toren\\
7.&90.&5629.&338.&21602.&26.&4&136&40\_40\_50\_6 \textcolor{red}{\textcjheb{wnmm}} MMNW $|$(mehr) als f"ur ihn\\
\end{tabular}\medskip \\
Ende des Verses 26.12\\
Verse: 753, Buchstaben: 29, 341, 21605, Totalwerte: 1935, 20722, 1528474\\
\\
Siehst du einen Mann, der in seinen Augen weise ist-f"ur einen Toren ist mehr Hoffnung als f"ur ihn.\\
\newpage 
{\bf -- 26.13}\\
\medskip \\
\begin{tabular}{rrrrrrrrp{120mm}}
WV&WK&WB&ABK&ABB&ABV&AnzB&TW&Zahlencode \textcolor{red}{$\boldsymbol{Grundtext}$} Umschrift $|$"Ubersetzung(en)\\
1.&91.&5630.&342.&21606.&1.&3&241&1\_40\_200 \textcolor{red}{\textcjheb{rm'}} AMR $|$(er (=es)) spricht\\
2.&92.&5631.&345.&21609.&4.&3&190&70\_90\_30 \textcolor{red}{\textcjheb{l.s`}} a"sL $|$der Faule/(der) Faulpelz\\
3.&93.&5632.&348.&21612.&7.&3&338&300\_8\_30 \textcolor{red}{\textcjheb{l.h+s}} SCL $|$der Br"uller/(ein) Jungleu\\
4.&94.&5633.&351.&21615.&10.&4&226&2\_4\_200\_20 \textcolor{red}{\textcjheb{krdb}} BDRK $|$(ist) auf dem Weg\\
5.&95.&5634.&355.&21619.&14.&3&211&1\_200\_10 \textcolor{red}{\textcjheb{yr'}} ARJ $|$(ein) L"owe\\
6.&96.&5635.&358.&21622.&17.&3&62&2\_10\_50 \textcolor{red}{\textcjheb{nyb}} BJN $|$inmitten/mitten auf\\
7.&97.&5636.&361.&21625.&20.&6&621&5\_200\_8\_2\_6\_400 \textcolor{red}{\textcjheb{twb.hrh}} HRCBWT $|$der Stra"sen/den Pl"atzen\\
\end{tabular}\medskip \\
Ende des Verses 26.13\\
Verse: 754, Buchstaben: 25, 366, 21630, Totalwerte: 1889, 22611, 1530363\\
\\
Der Faule spricht: Der Br"uller ist auf dem Wege, ein L"owe inmitten der Stra"sen.\\
\newpage 
{\bf -- 26.14}\\
\medskip \\
\begin{tabular}{rrrrrrrrp{120mm}}
WV&WK&WB&ABK&ABB&ABV&AnzB&TW&Zahlencode \textcolor{red}{$\boldsymbol{Grundtext}$} Umschrift $|$"Ubersetzung(en)\\
1.&98.&5637.&367.&21631.&1.&4&439&5\_4\_30\_400 \textcolor{red}{\textcjheb{tldh}} HDLT $|$die T"ure\\
2.&99.&5638.&371.&21635.&5.&4&468&400\_60\_6\_2 \textcolor{red}{\textcjheb{bwst}} TsWB $|$(sie) dreht sich\\
3.&100.&5639.&375.&21639.&9.&2&100&70\_30 \textcolor{red}{\textcjheb{l`}} aL $|$in/um\\
4.&101.&5640.&377.&21641.&11.&4&305&90\_10\_200\_5 \textcolor{red}{\textcjheb{hry.s}} "sJRH $|$ihre(r) Angel\\
5.&102.&5641.&381.&21645.&15.&4&196&6\_70\_90\_30 \textcolor{red}{\textcjheb{l.s`w}} Wa"sL $|$so der Faule/und ein Fauler\\
6.&103.&5642.&385.&21649.&19.&2&100&70\_30 \textcolor{red}{\textcjheb{l`}} aL $|$auf\\
7.&104.&5643.&387.&21651.&21.&4&455&40\_9\_400\_6 \textcolor{red}{\textcjheb{wt.tm}} MtTW $|$seinem Bett\\
\end{tabular}\medskip \\
Ende des Verses 26.14\\
Verse: 755, Buchstaben: 24, 390, 21654, Totalwerte: 2063, 24674, 1532426\\
\\
Die T"ur dreht sich in ihrer Angel: so der Faule auf seinem Bette.\\
\newpage 
{\bf -- 26.15}\\
\medskip \\
\begin{tabular}{rrrrrrrrp{120mm}}
WV&WK&WB&ABK&ABB&ABV&AnzB&TW&Zahlencode \textcolor{red}{$\boldsymbol{Grundtext}$} Umschrift $|$"Ubersetzung(en)\\
1.&105.&5644.&391.&21655.&1.&3&99&9\_40\_50 \textcolor{red}{\textcjheb{nm.t}} tMN $|$hat gesteckt/er (=es) steckte\\
2.&106.&5645.&394.&21658.&4.&3&190&70\_90\_30 \textcolor{red}{\textcjheb{l.s`}} a"sL $|$der Faule/(ein) Fauler\\
3.&107.&5646.&397.&21661.&7.&3&20&10\_4\_6 \textcolor{red}{\textcjheb{wdy}} JDW $|$seine Hand\\
4.&108.&5647.&400.&21664.&10.&5&530&2\_90\_30\_8\_400 \textcolor{red}{\textcjheb{t.hl.sb}} B"sLCT $|$in die Sch"ussel\\
5.&109.&5648.&405.&21669.&15.&4&86&50\_30\_1\_5 \textcolor{red}{\textcjheb{h'ln}} NLAH $|$beschwerlich wird es ihm/er f"uhlte sich m"ude\\
6.&110.&5649.&409.&21673.&19.&6&352&30\_5\_300\_10\_2\_5 \textcolor{red}{\textcjheb{hby+shl}} LHSJBH $|$zur"uckzubringen sie/zu f"uhren sie zur"uck\\
7.&111.&5650.&415.&21679.&25.&2&31&1\_30 \textcolor{red}{\textcjheb{l'}} AL $|$zu\\
8.&112.&5651.&417.&21681.&27.&3&96&80\_10\_6 \textcolor{red}{\textcjheb{wyp}} PJW $|$seinem Mund\\
\end{tabular}\medskip \\
Ende des Verses 26.15\\
Verse: 756, Buchstaben: 29, 419, 21683, Totalwerte: 1404, 26078, 1533830\\
\\
Hat der Faule seine Hand in die Sch"ussel gesteckt, beschwerlich wird es ihm, sie zu seinem Munde zur"uckzubringen.\\
\newpage 
{\bf -- 26.16}\\
\medskip \\
\begin{tabular}{rrrrrrrrp{120mm}}
WV&WK&WB&ABK&ABB&ABV&AnzB&TW&Zahlencode \textcolor{red}{$\boldsymbol{Grundtext}$} Umschrift $|$"Ubersetzung(en)\\
1.&113.&5652.&420.&21684.&1.&3&68&8\_20\_40 \textcolor{red}{\textcjheb{mk.h}} CKM $|$weise(r) ist\\
2.&114.&5653.&423.&21687.&4.&3&190&70\_90\_30 \textcolor{red}{\textcjheb{l.s`}} a"sL $|$der Faule/(ein) Fauler\\
3.&115.&5654.&426.&21690.&7.&6&148&2\_70\_10\_50\_10\_6 \textcolor{red}{\textcjheb{wyny`b}} BaJNJW $|$in seinen Augen\\
4.&116.&5655.&432.&21696.&13.&5&417&40\_300\_2\_70\_5 \textcolor{red}{\textcjheb{h`b+sm}} MSBaH $|$(mehr) als sieben\\
5.&117.&5656.&437.&21701.&18.&5&362&40\_300\_10\_2\_10 \textcolor{red}{\textcjheb{yby+sm}} MSJBJ $|$die antworten/Erwidernde\\
6.&118.&5657.&442.&21706.&23.&3&119&9\_70\_40 \textcolor{red}{\textcjheb{m`.t}} taM $|$verst"andig/(mit) Verstand\\
\end{tabular}\medskip \\
Ende des Verses 26.16\\
Verse: 757, Buchstaben: 25, 444, 21708, Totalwerte: 1304, 27382, 1535134\\
\\
Der Faule ist weiser in seinen Augen als sieben, die verst"andig antworten.\\
\newpage 
{\bf -- 26.17}\\
\medskip \\
\begin{tabular}{rrrrrrrrp{120mm}}
WV&WK&WB&ABK&ABB&ABV&AnzB&TW&Zahlencode \textcolor{red}{$\boldsymbol{Grundtext}$} Umschrift $|$"Ubersetzung(en)\\
1.&119.&5658.&445.&21709.&1.&5&165&40\_8\_7\_10\_100 \textcolor{red}{\textcjheb{qyz.hm}} MCZJQ $|$der ergreift/ein Greifender\\
2.&120.&5659.&450.&21714.&6.&5&70&2\_1\_7\_50\_10 \textcolor{red}{\textcjheb{ynz'b}} BAZNJ $|$bei den Ohren/an die Ohren\\
3.&121.&5660.&455.&21719.&11.&3&52&20\_30\_2 \textcolor{red}{\textcjheb{blk}} KLB $|$einen Hund/(des) Hundes\\
4.&122.&5661.&458.&21722.&14.&3&272&70\_2\_200 \textcolor{red}{\textcjheb{rb`}} aBR $|$wer vorbeigehend/(ist ein) Vorbeigehender\\
5.&123.&5662.&461.&21725.&17.&5&712&40\_400\_70\_2\_200 \textcolor{red}{\textcjheb{rb`tm}} MTaBR $|$sich ereifert/sich Ereifender\\
6.&124.&5663.&466.&21730.&22.&2&100&70\_30 \textcolor{red}{\textcjheb{l`}} aL $|$"uber/wegen\\
7.&125.&5664.&468.&21732.&24.&3&212&200\_10\_2 \textcolor{red}{\textcjheb{byr}} RJB $|$einen Streit/(eines) Streits\\
8.&126.&5665.&471.&21735.&27.&2&31&30\_1 \textcolor{red}{\textcjheb{'l}} LA $|$(der) nicht(s)\\
9.&127.&5666.&473.&21737.&29.&2&36&30\_6 \textcolor{red}{\textcjheb{wl}} LW $|$ihn angeht/zu ihn\\
\end{tabular}\medskip \\
Ende des Verses 26.17\\
Verse: 758, Buchstaben: 30, 474, 21738, Totalwerte: 1650, 29032, 1536784\\
\\
Der ergreift einen Hund bei den Ohren, wer vorbeigehend sich "uber einen Streit ereifert, der ihn nichts angeht.\\
\newpage 
{\bf -- 26.18}\\
\medskip \\
\begin{tabular}{rrrrrrrrp{120mm}}
WV&WK&WB&ABK&ABB&ABV&AnzB&TW&Zahlencode \textcolor{red}{$\boldsymbol{Grundtext}$} Umschrift $|$"Ubersetzung(en)\\
1.&128.&5667.&475.&21739.&1.&7&530&20\_40\_400\_30\_5\_30\_5 \textcolor{red}{\textcjheb{hlhltmk}} KMTLHLH $|$wie ein Wahnsinniger/wie ein sich unsinnig Geb"ardender\\
2.&129.&5668.&482.&21746.&8.&4&220&5\_10\_200\_5 \textcolor{red}{\textcjheb{hryh}} HJRH $|$der schleudert/der schie"send (ist)\\
3.&130.&5669.&486.&21750.&12.&4&157&7\_100\_10\_40 \textcolor{red}{\textcjheb{myqz}} ZQJM $|$(mit) Brandgeschosse(n)\\
4.&131.&5670.&490.&21754.&16.&4&148&8\_90\_10\_40 \textcolor{red}{\textcjheb{my.s.h}} C"sJM $|$Pfeile(n)\\
5.&132.&5671.&494.&21758.&20.&4&452&6\_40\_6\_400 \textcolor{red}{\textcjheb{twmw}} WMWT $|$und Tod\\
\end{tabular}\medskip \\
Ende des Verses 26.18\\
Verse: 759, Buchstaben: 23, 497, 21761, Totalwerte: 1507, 30539, 1538291\\
\\
Wie ein Wahnsinniger, der Brandgeschosse, Pfeile und Tod schleudert:\\
\newpage 
{\bf -- 26.19}\\
\medskip \\
\begin{tabular}{rrrrrrrrp{120mm}}
WV&WK&WB&ABK&ABB&ABV&AnzB&TW&Zahlencode \textcolor{red}{$\boldsymbol{Grundtext}$} Umschrift $|$"Ubersetzung(en)\\
1.&133.&5672.&498.&21762.&1.&2&70&20\_50 \textcolor{red}{\textcjheb{nk}} KN $|$so (ist)\\
2.&134.&5673.&500.&21764.&3.&3&311&1\_10\_300 \textcolor{red}{\textcjheb{+sy'}} AJS $|$ein Mann/jemand\\
3.&135.&5674.&503.&21767.&6.&3&245&200\_40\_5 \textcolor{red}{\textcjheb{hmr}} RMH $|$der betr"ugt\\
4.&136.&5675.&506.&21770.&9.&2&401&1\_400 \textcolor{red}{\textcjheb{t'}} AT $|$**\\
5.&137.&5676.&508.&21772.&11.&4&281&200\_70\_5\_6 \textcolor{red}{\textcjheb{wh`r}} RaHW $|$seinen Freund\\
6.&138.&5677.&512.&21776.&15.&4&247&6\_1\_40\_200 \textcolor{red}{\textcjheb{rm'w}} WAMR $|$und (er) spricht\\
7.&139.&5678.&516.&21780.&19.&3&36&5\_30\_1 \textcolor{red}{\textcjheb{'lh}} HLA $|$(etwa) nicht\\
8.&140.&5679.&519.&21783.&22.&4&448&40\_300\_8\_100 \textcolor{red}{\textcjheb{q.h+sm}} MSCQ $|$habe Scherz getrieben/scherzend\\
9.&141.&5680.&523.&21787.&26.&3&61&1\_50\_10 \textcolor{red}{\textcjheb{yn'}} ANJ $|$ich (war)\\
\end{tabular}\medskip \\
Ende des Verses 26.19\\
Verse: 760, Buchstaben: 28, 525, 21789, Totalwerte: 2100, 32639, 1540391\\
\\
so ein Mann, der seinen N"achsten betr"ugt und spricht: Habe ich nicht Scherz getrieben?\\
\newpage 
{\bf -- 26.20}\\
\medskip \\
\begin{tabular}{rrrrrrrrp{120mm}}
WV&WK&WB&ABK&ABB&ABV&AnzB&TW&Zahlencode \textcolor{red}{$\boldsymbol{Grundtext}$} Umschrift $|$"Ubersetzung(en)\\
1.&142.&5681.&526.&21790.&1.&4&143&2\_1\_80\_60 \textcolor{red}{\textcjheb{sp'b}} BAPs $|$wo es fehlt an/aus Mangel an\\
2.&143.&5682.&530.&21794.&5.&4&210&70\_90\_10\_40 \textcolor{red}{\textcjheb{my.s`}} a"sJM $|$Holz\\
3.&144.&5683.&534.&21798.&9.&4&427&400\_20\_2\_5 \textcolor{red}{\textcjheb{hbkt}} TKBH $|$(er (=es)) erlischt\\
4.&145.&5684.&538.&21802.&13.&2&301&1\_300 \textcolor{red}{\textcjheb{+s'}} AS $|$das Feuer/(ein) Feuer\\
5.&146.&5685.&540.&21804.&15.&5&69&6\_2\_1\_10\_50 \textcolor{red}{\textcjheb{ny'bw}} WBAJN $|$und wo ist kein/und ohne\\
6.&147.&5686.&545.&21809.&20.&4&303&50\_200\_3\_50 \textcolor{red}{\textcjheb{ngrn}} NRGN $|$Ohrenbl"aser/Verleumder\\
7.&148.&5687.&549.&21813.&24.&4&810&10\_300\_400\_100 \textcolor{red}{\textcjheb{qt+sy}} JSTQ $|$h"ort auf/er (=es) legt sich\\
8.&149.&5688.&553.&21817.&28.&4&100&40\_4\_6\_50 \textcolor{red}{\textcjheb{nwdm}} MDWN $|$der Zank/der Streit\\
\end{tabular}\medskip \\
Ende des Verses 26.20\\
Verse: 761, Buchstaben: 31, 556, 21820, Totalwerte: 2363, 35002, 1542754\\
\\
Wo es an Holz fehlt, erlischt das Feuer; und wo kein Ohrenbl"aser ist, h"ort der Zank auf.\\
\newpage 
{\bf -- 26.21}\\
\medskip \\
\begin{tabular}{rrrrrrrrp{120mm}}
WV&WK&WB&ABK&ABB&ABV&AnzB&TW&Zahlencode \textcolor{red}{$\boldsymbol{Grundtext}$} Umschrift $|$"Ubersetzung(en)\\
1.&150.&5689.&557.&21821.&1.&3&128&80\_8\_40 \textcolor{red}{\textcjheb{m.hp}} PCM $|$(wie) Kohle\\
2.&151.&5690.&560.&21824.&4.&6&121&30\_3\_8\_30\_10\_40 \textcolor{red}{\textcjheb{myl.hgl}} LGCLJM $|$zur Glut/f"ur Gluten\\
3.&152.&5691.&566.&21830.&10.&5&216&6\_70\_90\_10\_40 \textcolor{red}{\textcjheb{my.s`w}} Wa"sJM $|$und Holz/und Holzscheite\\
4.&153.&5692.&571.&21835.&15.&3&331&30\_1\_300 \textcolor{red}{\textcjheb{+s'l}} LAS $|$zum Feuer/f"ur Feuer\\
5.&154.&5693.&574.&21838.&18.&4&317&6\_1\_10\_300 \textcolor{red}{\textcjheb{+sy'w}} WAJS $|$und (ein) Mann\\
6.&155.&5694.&578.&21842.&22.&6&150&40\_4\_6\_50\_10\_40 \textcolor{red}{\textcjheb{mynwdm}} MDWNJM $|$z"ankischer/(von) Streitigkeiten\\
7.&156.&5695.&584.&21848.&28.&5&446&30\_8\_200\_8\_200 \textcolor{red}{\textcjheb{r.hr.hl}} LCRCR $|$zum Sch"uren/f"ur das Sch"uren\\
8.&157.&5696.&589.&21853.&33.&3&212&200\_10\_2 \textcolor{red}{\textcjheb{byr}} RJB $|$des Streits/(von) Streit\\
\end{tabular}\medskip \\
Ende des Verses 26.21\\
Verse: 762, Buchstaben: 35, 591, 21855, Totalwerte: 1921, 36923, 1544675\\
\\
Kohle zur Glut und Holz zum Feuer, und ein z"ankischer Mann zum Sch"uren des Streites.\\
\newpage 
{\bf -- 26.22}\\
\medskip \\
\begin{tabular}{rrrrrrrrp{120mm}}
WV&WK&WB&ABK&ABB&ABV&AnzB&TW&Zahlencode \textcolor{red}{$\boldsymbol{Grundtext}$} Umschrift $|$"Ubersetzung(en)\\
1.&158.&5697.&592.&21856.&1.&4&216&4\_2\_200\_10 \textcolor{red}{\textcjheb{yrbd}} DBRJ $|$die Worte\\
2.&159.&5698.&596.&21860.&5.&4&303&50\_200\_3\_50 \textcolor{red}{\textcjheb{ngrn}} NRGN $|$des Ohrenbl"asers/des Verleumders\\
3.&160.&5699.&600.&21864.&9.&8&585&20\_40\_400\_30\_5\_40\_10\_40 \textcolor{red}{\textcjheb{mymhltmk}} KMTLHMJM $|$(sind) wie Leckerbissen\\
4.&161.&5700.&608.&21872.&17.&3&51&6\_5\_40 \textcolor{red}{\textcjheb{mhw}} WHM $|$und sie\\
5.&162.&5701.&611.&21875.&20.&4&220&10\_200\_4\_6 \textcolor{red}{\textcjheb{wdry}} JRDW $|$dringen hinab/(sie) gehen hinab\\
6.&163.&5702.&615.&21879.&24.&4&222&8\_4\_200\_10 \textcolor{red}{\textcjheb{yrd.h}} CDRJ $|$in das Innerste/(in die) Kammern\\
7.&164.&5703.&619.&21883.&28.&3&61&2\_9\_50 \textcolor{red}{\textcjheb{n.tb}} BtN $|$des Leibes\\
\end{tabular}\medskip \\
Ende des Verses 26.22\\
Verse: 763, Buchstaben: 30, 621, 21885, Totalwerte: 1658, 38581, 1546333\\
\\
Die Worte des Ohrenbl"asers sind wie Leckerbissen, und sie dringen hinab in das Innerste des Leibes.\\
\newpage 
{\bf -- 26.23}\\
\medskip \\
\begin{tabular}{rrrrrrrrp{120mm}}
WV&WK&WB&ABK&ABB&ABV&AnzB&TW&Zahlencode \textcolor{red}{$\boldsymbol{Grundtext}$} Umschrift $|$"Ubersetzung(en)\\
1.&165.&5704.&622.&21886.&1.&3&160&20\_60\_80 \textcolor{red}{\textcjheb{psk}} KsP $|$mit Silber/(wie) Silber\\
2.&166.&5705.&625.&21889.&4.&5&123&60\_10\_3\_10\_40 \textcolor{red}{\textcjheb{mygys}} sJGJM $|$(von) Schlacken\\
3.&167.&5706.&630.&21894.&9.&4&215&40\_90\_80\_5 \textcolor{red}{\textcjheb{hp.sm}} M"sPH $|$"uberzogen\\
4.&168.&5707.&634.&21898.&13.&2&100&70\_30 \textcolor{red}{\textcjheb{l`}} aL $|$ein/auf\\
5.&169.&5708.&636.&21900.&15.&3&508&8\_200\_300 \textcolor{red}{\textcjheb{+sr.h}} CRS $|$(irdenes) Ton(geschirr)\\
6.&170.&5709.&639.&21903.&18.&5&830&300\_80\_400\_10\_40 \textcolor{red}{\textcjheb{mytp+s}} SPTJM $|$(so) (sind) (zwei) Lippen\\
7.&171.&5710.&644.&21908.&23.&5&184&4\_30\_100\_10\_40 \textcolor{red}{\textcjheb{myqld}} DLQJM $|$feurige/brennende\\
8.&172.&5711.&649.&21913.&28.&3&38&6\_30\_2 \textcolor{red}{\textcjheb{blw}} WLB $|$und ein Herz\\
9.&173.&5712.&652.&21916.&31.&2&270&200\_70 \textcolor{red}{\textcjheb{`r}} Ra $|$b"oses\\
\end{tabular}\medskip \\
Ende des Verses 26.23\\
Verse: 764, Buchstaben: 32, 653, 21917, Totalwerte: 2428, 41009, 1548761\\
\\
Ein irdenes Geschirr, mit Schlackensilber "uberzogen: so sind feurige Lippen und ein b"oses Herz.\\
\newpage 
{\bf -- 26.24}\\
\medskip \\
\begin{tabular}{rrrrrrrrp{120mm}}
WV&WK&WB&ABK&ABB&ABV&AnzB&TW&Zahlencode \textcolor{red}{$\boldsymbol{Grundtext}$} Umschrift $|$"Ubersetzung(en)\\
1.&174.&5713.&654.&21918.&1.&5&788&2\_300\_80\_400\_6 \textcolor{red}{\textcjheb{wtp+sb}} BSPTW $|$mit seinen Lippen\\
2.&175.&5714.&659.&21923.&6.&4&280&10\_50\_20\_200 \textcolor{red}{\textcjheb{rkny}} JNKR $|$(er (=es)) verstellt sich\\
3.&176.&5715.&663.&21927.&10.&4&357&300\_6\_50\_1 \textcolor{red}{\textcjheb{'nw+s}} SWNA $|$der Hasser/(ein) Hassender\\
4.&177.&5716.&667.&21931.&14.&6&316&6\_2\_100\_200\_2\_6 \textcolor{red}{\textcjheb{wbrqbw}} WBQRBW $|$aber in seinem Inneren/und in seinem Inneren\\
5.&178.&5717.&673.&21937.&20.&4&720&10\_300\_10\_400 \textcolor{red}{\textcjheb{ty+sy}} JSJT $|$er hegt\\
6.&179.&5718.&677.&21941.&24.&4&285&40\_200\_40\_5 \textcolor{red}{\textcjheb{hmrm}} MRMH $|$Trug\\
\end{tabular}\medskip \\
Ende des Verses 26.24\\
Verse: 765, Buchstaben: 27, 680, 21944, Totalwerte: 2746, 43755, 1551507\\
\\
Der Hasser verstellt sich mit seinen Lippen, aber in seinem Innern hegt er Trug.\\
\newpage 
{\bf -- 26.25}\\
\medskip \\
\begin{tabular}{rrrrrrrrp{120mm}}
WV&WK&WB&ABK&ABB&ABV&AnzB&TW&Zahlencode \textcolor{red}{$\boldsymbol{Grundtext}$} Umschrift $|$"Ubersetzung(en)\\
1.&180.&5719.&681.&21945.&1.&2&30&20\_10 \textcolor{red}{\textcjheb{yk}} KJ $|$wenn\\
2.&181.&5720.&683.&21947.&3.&4&118&10\_8\_50\_50 \textcolor{red}{\textcjheb{nn.hy}} JCNN $|$er holdselig macht/(ein)er macht lieblich\\
3.&182.&5721.&687.&21951.&7.&4&142&100\_6\_30\_6 \textcolor{red}{\textcjheb{wlwq}} QWLW $|$seine Stimme\\
4.&183.&5722.&691.&21955.&11.&2&31&1\_30 \textcolor{red}{\textcjheb{l'}} AL $|$nicht\\
5.&184.&5723.&693.&21957.&13.&4&491&400\_1\_40\_50 \textcolor{red}{\textcjheb{nm't}} TAMN $|$traue/du sollst vertrauen\\
6.&185.&5724.&697.&21961.&17.&2&8&2\_6 \textcolor{red}{\textcjheb{wb}} BW $|$ihm/auf ihn\\
7.&186.&5725.&699.&21963.&19.&2&30&20\_10 \textcolor{red}{\textcjheb{yk}} KJ $|$denn\\
8.&187.&5726.&701.&21965.&21.&3&372&300\_2\_70 \textcolor{red}{\textcjheb{`b+s}} SBa $|$sieben\\
9.&188.&5727.&704.&21968.&24.&6&884&400\_6\_70\_2\_6\_400 \textcolor{red}{\textcjheb{twb`wt}} TWaBWT $|$Gr"auel\\
10.&189.&5728.&710.&21974.&30.&4&40&2\_30\_2\_6 \textcolor{red}{\textcjheb{wblb}} BLBW $|$(sind) in seinem Herzen\\
\end{tabular}\medskip \\
Ende des Verses 26.25\\
Verse: 766, Buchstaben: 33, 713, 21977, Totalwerte: 2146, 45901, 1553653\\
\\
Wenn er seine Stimme holdselig macht, traue ihm nicht; denn sieben Greuel sind in seinem Herzen.\\
\newpage 
{\bf -- 26.26}\\
\medskip \\
\begin{tabular}{rrrrrrrrp{120mm}}
WV&WK&WB&ABK&ABB&ABV&AnzB&TW&Zahlencode \textcolor{red}{$\boldsymbol{Grundtext}$} Umschrift $|$"Ubersetzung(en)\\
1.&190.&5729.&714.&21978.&1.&4&485&400\_20\_60\_5 \textcolor{red}{\textcjheb{hskt}} TKsH $|$versteckt sich/er (=es) verbirgt sich\\
2.&191.&5730.&718.&21982.&5.&4&356&300\_50\_1\_5 \textcolor{red}{\textcjheb{h'n+s}} SNAH $|$(der) Hass\\
3.&192.&5731.&722.&21986.&9.&6&399&2\_40\_300\_1\_6\_50 \textcolor{red}{\textcjheb{nw'+smb}} BMSAWN $|$in Trug/unter T"auschung\\
4.&193.&5732.&728.&21992.&15.&4&438&400\_3\_30\_5 \textcolor{red}{\textcjheb{hlgt}} TGLH $|$wird sich enth"ullen/sie (=es) wird enth"ullt\\
5.&194.&5733.&732.&21996.&19.&4&676&200\_70\_400\_6 \textcolor{red}{\textcjheb{wt`r}} RaTW $|$seine Bosheit\\
6.&195.&5734.&736.&22000.&23.&4&137&2\_100\_5\_30 \textcolor{red}{\textcjheb{lhqb}} BQHL $|$in der Versammlung\\
\end{tabular}\medskip \\
Ende des Verses 26.26\\
Verse: 767, Buchstaben: 26, 739, 22003, Totalwerte: 2491, 48392, 1556144\\
\\
Versteckt sich der Ha"s in Trug, seine Bosheit wird sich in der Versammlung enth"ullen.\\
\newpage 
{\bf -- 26.27}\\
\medskip \\
\begin{tabular}{rrrrrrrrp{120mm}}
WV&WK&WB&ABK&ABB&ABV&AnzB&TW&Zahlencode \textcolor{red}{$\boldsymbol{Grundtext}$} Umschrift $|$"Ubersetzung(en)\\
1.&196.&5735.&740.&22004.&1.&3&225&20\_200\_5 \textcolor{red}{\textcjheb{hrk}} KRH $|$wer gr"abt/(ein) Grabender\\
2.&197.&5736.&743.&22007.&4.&3&708&300\_8\_400 \textcolor{red}{\textcjheb{t.h+s}} SCT $|$eine Grube\\
3.&198.&5737.&746.&22010.&7.&2&7&2\_5 \textcolor{red}{\textcjheb{hb}} BH $|$hinein/in sie\\
4.&199.&5738.&748.&22012.&9.&3&120&10\_80\_30 \textcolor{red}{\textcjheb{lpy}} JPL $|$(er (=es)) f"allt\\
5.&200.&5739.&751.&22015.&12.&4&69&6\_3\_30\_30 \textcolor{red}{\textcjheb{llgw}} WGLL $|$und wer w"alzt/und ein Rollender\\
6.&201.&5740.&755.&22019.&16.&3&53&1\_2\_50 \textcolor{red}{\textcjheb{nb'}} ABN $|$(einen) Stein\\
7.&202.&5741.&758.&22022.&19.&4&47&1\_30\_10\_6 \textcolor{red}{\textcjheb{wyl'}} ALJW $|$auf den/zu ihm\\
8.&203.&5742.&762.&22026.&23.&4&708&400\_300\_6\_2 \textcolor{red}{\textcjheb{bw+st}} TSWB $|$er (=es) kehrt zur"uck\\
\end{tabular}\medskip \\
Ende des Verses 26.27\\
Verse: 768, Buchstaben: 26, 765, 22029, Totalwerte: 1937, 50329, 1558081\\
\\
Wer eine Grube gr"abt, f"allt hinein; und wer einen Stein w"alzt, auf den kehrt er zur"uck.\\
\newpage 
{\bf -- 26.28}\\
\medskip \\
\begin{tabular}{rrrrrrrrp{120mm}}
WV&WK&WB&ABK&ABB&ABV&AnzB&TW&Zahlencode \textcolor{red}{$\boldsymbol{Grundtext}$} Umschrift $|$"Ubersetzung(en)\\
1.&204.&5743.&766.&22030.&1.&4&386&30\_300\_6\_50 \textcolor{red}{\textcjheb{nw+sl}} LSWN $|$eine Zunge/(die) Zunge\\
2.&205.&5744.&770.&22034.&5.&3&600&300\_100\_200 \textcolor{red}{\textcjheb{rq+s}} SQR $|$(der) L"uge(n)\\
3.&206.&5745.&773.&22037.&8.&4&361&10\_300\_50\_1 \textcolor{red}{\textcjheb{'n+sy}} JSNA $|$(er (=sie)) hasst\\
4.&207.&5746.&777.&22041.&12.&4&40&4\_20\_10\_6 \textcolor{red}{\textcjheb{wykd}} DKJW $|$diejenigen welche sie zermalmt/ihre Zermalmten\\
5.&208.&5747.&781.&22045.&16.&3&91&6\_80\_5 \textcolor{red}{\textcjheb{hpw}} WPH $|$und ein Mund\\
6.&209.&5748.&784.&22048.&19.&3&138&8\_30\_100 \textcolor{red}{\textcjheb{ql.h}} CLQ $|$glatter\\
7.&210.&5749.&787.&22051.&22.&4&385&10\_70\_300\_5 \textcolor{red}{\textcjheb{h+s`y}} JaSH $|$(er) bereitet\\
8.&211.&5750.&791.&22055.&26.&4&57&40\_4\_8\_5 \textcolor{red}{\textcjheb{h.hdm}} MDCH $|$Sturz\\
\end{tabular}\medskip \\
Ende des Verses 26.28\\
Verse: 769, Buchstaben: 29, 794, 22058, Totalwerte: 2058, 52387, 1560139\\
\\
Eine L"ugenzunge ha"st diejenigen, welche sie zermalmt; und ein glatter Mund bereitet Sturz.\\
\\
{\bf Ende des Kapitels 26}\\
\newpage 
{\bf -- 27.1}\\
\medskip \\
\begin{tabular}{rrrrrrrrp{120mm}}
WV&WK&WB&ABK&ABB&ABV&AnzB&TW&Zahlencode \textcolor{red}{$\boldsymbol{Grundtext}$} Umschrift $|$"Ubersetzung(en)\\
1.&1.&5751.&1.&22059.&1.&2&31&1\_30 \textcolor{red}{\textcjheb{l'}} AL $|$nicht\\
2.&2.&5752.&3.&22061.&3.&5&865&400\_400\_5\_30\_30 \textcolor{red}{\textcjheb{llhtt}} TTHLL $|$r"uhme dich/du sollst dich r"uhmen\\
3.&3.&5753.&8.&22066.&8.&4&58&2\_10\_6\_40 \textcolor{red}{\textcjheb{mwyb}} BJWM $|$des Tages/mit (einem) Tag\\
4.&4.&5754.&12.&22070.&12.&3&248&40\_8\_200 \textcolor{red}{\textcjheb{r.hm}} MCR $|$(von) morgen(den)\\
5.&5.&5755.&15.&22073.&15.&2&30&20\_10 \textcolor{red}{\textcjheb{yk}} KJ $|$denn\\
6.&6.&5756.&17.&22075.&17.&2&31&30\_1 \textcolor{red}{\textcjheb{'l}} LA $|$nicht\\
7.&7.&5757.&19.&22077.&19.&3&474&400\_4\_70 \textcolor{red}{\textcjheb{`dt}} TDa $|$du wei"st\\
8.&8.&5758.&22.&22080.&22.&2&45&40\_5 \textcolor{red}{\textcjheb{hm}} MH $|$was\\
9.&9.&5759.&24.&22082.&24.&3&44&10\_30\_4 \textcolor{red}{\textcjheb{dly}} JLD $|$gebiert/(ein) Kind\\
10.&10.&5760.&27.&22085.&27.&3&56&10\_6\_40 \textcolor{red}{\textcjheb{mwy}} JWM $|$ein Tag/der Tag (ist)\\
\end{tabular}\medskip \\
Ende des Verses 27.1\\
Verse: 770, Buchstaben: 29, 29, 22087, Totalwerte: 1882, 1882, 1562021\\
\\
R"uhme dich nicht des morgenden Tages, denn du wei"st nicht, was ein Tag gebiert.\\
\newpage 
{\bf -- 27.2}\\
\medskip \\
\begin{tabular}{rrrrrrrrp{120mm}}
WV&WK&WB&ABK&ABB&ABV&AnzB&TW&Zahlencode \textcolor{red}{$\boldsymbol{Grundtext}$} Umschrift $|$"Ubersetzung(en)\\
1.&11.&5761.&30.&22088.&1.&5&95&10\_5\_30\_30\_20 \textcolor{red}{\textcjheb{kllhy}} JHLLK $|$es r"uhme dich/er (=es) soll dich r"uhmen\\
2.&12.&5762.&35.&22093.&6.&2&207&7\_200 \textcolor{red}{\textcjheb{rz}} ZR $|$ein anderer/(ein) fremder\\
3.&13.&5763.&37.&22095.&8.&3&37&6\_30\_1 \textcolor{red}{\textcjheb{'lw}} WLA $|$und nicht\\
4.&14.&5764.&40.&22098.&11.&3&110&80\_10\_20 \textcolor{red}{\textcjheb{kyp}} PJK $|$dein Mund\\
5.&15.&5765.&43.&22101.&14.&4&280&50\_20\_200\_10 \textcolor{red}{\textcjheb{yrkn}} NKRJ $|$ein Fremder/(der) Fremde\\
6.&16.&5766.&47.&22105.&18.&3&37&6\_1\_30 \textcolor{red}{\textcjheb{l'w}} WAL $|$und nicht\\
7.&17.&5767.&50.&22108.&21.&5&810&300\_80\_400\_10\_20 \textcolor{red}{\textcjheb{kytp+s}} SPTJK $|$deine Lippen\\
\end{tabular}\medskip \\
Ende des Verses 27.2\\
Verse: 771, Buchstaben: 25, 54, 22112, Totalwerte: 1576, 3458, 1563597\\
\\
Es r"uhme dich ein anderer und nicht dein Mund, ein Fremder und nicht deine Lippen.\\
\newpage 
{\bf -- 27.3}\\
\medskip \\
\begin{tabular}{rrrrrrrrp{120mm}}
WV&WK&WB&ABK&ABB&ABV&AnzB&TW&Zahlencode \textcolor{red}{$\boldsymbol{Grundtext}$} Umschrift $|$"Ubersetzung(en)\\
1.&18.&5768.&55.&22113.&1.&3&26&20\_2\_4 \textcolor{red}{\textcjheb{dbk}} KBD $|$schwer ist/die Schwere\\
2.&19.&5769.&58.&22116.&4.&3&53&1\_2\_50 \textcolor{red}{\textcjheb{nb'}} ABN $|$der Stein/(des) Steins\\
3.&20.&5770.&61.&22119.&7.&4&95&6\_50\_9\_30 \textcolor{red}{\textcjheb{l.tnw}} WNtL $|$und eine Last/und die Last\\
4.&21.&5771.&65.&22123.&11.&4&49&5\_8\_6\_30 \textcolor{red}{\textcjheb{lw.hh}} HCWL $|$der Sand/des Sandes\\
5.&22.&5772.&69.&22127.&15.&4&156&6\_20\_70\_60 \textcolor{red}{\textcjheb{s`kw}} WKas $|$aber der Unmut/und Verdruss\\
6.&23.&5773.&73.&22131.&19.&4&47&1\_6\_10\_30 \textcolor{red}{\textcjheb{lyw'}} AWJL $|$des Narren/"uber (einen) Toren\\
7.&24.&5774.&77.&22135.&23.&3&26&20\_2\_4 \textcolor{red}{\textcjheb{dbk}} KBD $|$(ist) schwerer\\
8.&25.&5775.&80.&22138.&26.&6&445&40\_300\_50\_10\_5\_40 \textcolor{red}{\textcjheb{mhyn+sm}} MSNJHM $|$als (sie) beide\\
\end{tabular}\medskip \\
Ende des Verses 27.3\\
Verse: 772, Buchstaben: 31, 85, 22143, Totalwerte: 897, 4355, 1564494\\
\\
Schwer ist der Stein, und der Sand eine Last; aber der Unmut des Narren ist schwerer als beide.\\
\newpage 
{\bf -- 27.4}\\
\medskip \\
\begin{tabular}{rrrrrrrrp{120mm}}
WV&WK&WB&ABK&ABB&ABV&AnzB&TW&Zahlencode \textcolor{red}{$\boldsymbol{Grundtext}$} Umschrift $|$"Ubersetzung(en)\\
1.&26.&5776.&86.&22144.&1.&7&644&1\_20\_7\_200\_10\_6\_400 \textcolor{red}{\textcjheb{twyrzk'}} AKZRJWT $|$grausam ist/die Grausamkeit\\
2.&27.&5777.&93.&22151.&8.&3&53&8\_40\_5 \textcolor{red}{\textcjheb{hm.h}} CMH $|$Grimm/(der) Glut\\
3.&28.&5778.&96.&22154.&11.&4&395&6\_300\_9\_80 \textcolor{red}{\textcjheb{p.t+sw}} WStP $|$und eine "uberstr"omende Flut/und das "Uberfluten\\
4.&29.&5779.&100.&22158.&15.&2&81&1\_80 \textcolor{red}{\textcjheb{p'}} AP $|$(des) Zorn(s)\\
5.&30.&5780.&102.&22160.&17.&3&56&6\_40\_10 \textcolor{red}{\textcjheb{ymw}} WMJ $|$wer aber/und wer\\
6.&31.&5781.&105.&22163.&20.&4&124&10\_70\_40\_4 \textcolor{red}{\textcjheb{dm`y}} JaMD $|$kann bestehen/h"alt stand\\
7.&32.&5782.&109.&22167.&24.&4&170&30\_80\_50\_10 \textcolor{red}{\textcjheb{ynpl}} LPNJ $|$vor\\
8.&33.&5783.&113.&22171.&28.&4&156&100\_50\_1\_5 \textcolor{red}{\textcjheb{h'nq}} QNAH $|$der Eifersucht\\
\end{tabular}\medskip \\
Ende des Verses 27.4\\
Verse: 773, Buchstaben: 31, 116, 22174, Totalwerte: 1679, 6034, 1566173\\
\\
Grimm ist grausam, und Zorn eine "uberstr"omende Flut; wer aber kann bestehen vor der Eifersucht!\\
\newpage 
{\bf -- 27.5}\\
\medskip \\
\begin{tabular}{rrrrrrrrp{120mm}}
WV&WK&WB&ABK&ABB&ABV&AnzB&TW&Zahlencode \textcolor{red}{$\boldsymbol{Grundtext}$} Umschrift $|$"Ubersetzung(en)\\
1.&34.&5784.&117.&22175.&1.&4&22&9\_6\_2\_5 \textcolor{red}{\textcjheb{hbw.t}} tWBH $|$besser/gut (ist)\\
2.&35.&5785.&121.&22179.&5.&5&834&400\_6\_20\_8\_400 \textcolor{red}{\textcjheb{t.hkwt}} TWKCT $|$Tadel/(eine) Zurechtweisung\\
3.&36.&5786.&126.&22184.&10.&4&78&40\_3\_30\_5 \textcolor{red}{\textcjheb{hlgm}} MGLH $|$offene(r)\\
4.&37.&5787.&130.&22188.&14.&5&53&40\_1\_5\_2\_5 \textcolor{red}{\textcjheb{hbh'm}} MAHBH $|$als (die) Liebe\\
5.&38.&5788.&135.&22193.&19.&5&1100&40\_60\_400\_200\_400 \textcolor{red}{\textcjheb{trtsm}} MsTRT $|$verhehlte/verheimlichte\\
\end{tabular}\medskip \\
Ende des Verses 27.5\\
Verse: 774, Buchstaben: 23, 139, 22197, Totalwerte: 2087, 8121, 1568260\\
\\
Besser offener Tadel als verhehlte Liebe.\\
\newpage 
{\bf -- 27.6}\\
\medskip \\
\begin{tabular}{rrrrrrrrp{120mm}}
WV&WK&WB&ABK&ABB&ABV&AnzB&TW&Zahlencode \textcolor{red}{$\boldsymbol{Grundtext}$} Umschrift $|$"Ubersetzung(en)\\
1.&39.&5789.&140.&22198.&1.&6&191&50\_1\_40\_50\_10\_40 \textcolor{red}{\textcjheb{mynm'n}} NAMNJM $|$treugemeint sind/zuverl"assig (sind)\\
2.&40.&5790.&146.&22204.&7.&4&250&80\_90\_70\_10 \textcolor{red}{\textcjheb{y`.sp}} P"saJ $|$(die) Wunden\\
3.&41.&5791.&150.&22208.&11.&4&14&1\_6\_5\_2 \textcolor{red}{\textcjheb{bhw'}} AWHB $|$dessen der liebt/eines Liebenden\\
4.&42.&5792.&154.&22212.&15.&7&1132&6\_50\_70\_400\_200\_6\_400 \textcolor{red}{\textcjheb{twrt`nw}} WNaTRWT $|$und "uberreichlich/und sich erbitten lassend\\
5.&43.&5793.&161.&22219.&22.&6&866&50\_300\_10\_100\_6\_400 \textcolor{red}{\textcjheb{twqy+sn}} NSJQWT $|$K"usse\\
6.&44.&5794.&167.&22225.&28.&4&357&300\_6\_50\_1 \textcolor{red}{\textcjheb{'nw+s}} SWNA $|$des Hassers/(eines) Hassenden\\
\end{tabular}\medskip \\
Ende des Verses 27.6\\
Verse: 775, Buchstaben: 31, 170, 22228, Totalwerte: 2810, 10931, 1571070\\
\\
Treugemeint sind die Wunden dessen, der liebt, und "uberreichlich des Hassers K"usse.\\
\newpage 
{\bf -- 27.7}\\
\medskip \\
\begin{tabular}{rrrrrrrrp{120mm}}
WV&WK&WB&ABK&ABB&ABV&AnzB&TW&Zahlencode \textcolor{red}{$\boldsymbol{Grundtext}$} Umschrift $|$"Ubersetzung(en)\\
1.&45.&5795.&171.&22229.&1.&3&430&50\_80\_300 \textcolor{red}{\textcjheb{+spn}} NPS $|$(eine) Seele\\
2.&46.&5796.&174.&22232.&4.&4&377&300\_2\_70\_5 \textcolor{red}{\textcjheb{h`b+s}} SBaH $|$satte\\
3.&47.&5797.&178.&22236.&8.&4&468&400\_2\_6\_60 \textcolor{red}{\textcjheb{swbt}} TBWs $|$zertritt/(sie) tritt nieder\\
4.&48.&5798.&182.&22240.&12.&3&530&50\_80\_400 \textcolor{red}{\textcjheb{tpn}} NPT $|$Honig(seim)\\
5.&49.&5799.&185.&22243.&15.&4&436&6\_50\_80\_300 \textcolor{red}{\textcjheb{+spnw}} WNPS $|$aber einer Seele/und einer Seele\\
6.&50.&5800.&189.&22247.&19.&4&277&200\_70\_2\_5 \textcolor{red}{\textcjheb{hb`r}} RaBH $|$hungrige(n)\\
7.&51.&5801.&193.&22251.&23.&2&50&20\_30 \textcolor{red}{\textcjheb{lk}} KL $|$alles\\
8.&52.&5802.&195.&22253.&25.&2&240&40\_200 \textcolor{red}{\textcjheb{rm}} MR $|$Bittere\\
9.&53.&5803.&197.&22255.&27.&4&546&40\_400\_6\_100 \textcolor{red}{\textcjheb{qwtm}} MTWQ $|$(ist) s"u"s\\
\end{tabular}\medskip \\
Ende des Verses 27.7\\
Verse: 776, Buchstaben: 30, 200, 22258, Totalwerte: 3354, 14285, 1574424\\
\\
Eine satte Seele zertritt Honigseim; aber einer hungrigen Seele ist alles Bittere s"u"s.\\
\newpage 
{\bf -- 27.8}\\
\medskip \\
\begin{tabular}{rrrrrrrrp{120mm}}
WV&WK&WB&ABK&ABB&ABV&AnzB&TW&Zahlencode \textcolor{red}{$\boldsymbol{Grundtext}$} Umschrift $|$"Ubersetzung(en)\\
1.&54.&5804.&201.&22259.&1.&5&396&20\_90\_80\_6\_200 \textcolor{red}{\textcjheb{rwp.sk}} K"sPWR $|$wie ein Vogel/wie ein V"oglein\\
2.&55.&5805.&206.&22264.&6.&5&464&50\_6\_4\_4\_400 \textcolor{red}{\textcjheb{tddwn}} NWDDT $|$der schweift fern/entfliehend (ist)\\
3.&56.&5806.&211.&22269.&11.&2&90&40\_50 \textcolor{red}{\textcjheb{nm}} MN $|$von\\
4.&57.&5807.&213.&22271.&13.&3&155&100\_50\_5 \textcolor{red}{\textcjheb{hnq}} QNH $|$seinem Nest\\
5.&58.&5808.&216.&22274.&16.&2&70&20\_50 \textcolor{red}{\textcjheb{nk}} KN $|$so (ist)\\
6.&59.&5809.&218.&22276.&18.&3&311&1\_10\_300 \textcolor{red}{\textcjheb{+sy'}} AJS $|$(ein) Mann\\
7.&60.&5810.&221.&22279.&21.&4&64&50\_6\_4\_4 \textcolor{red}{\textcjheb{ddwn}} NWDD $|$der schweift fern/fliehend\\
8.&61.&5811.&225.&22283.&25.&6&232&40\_40\_100\_6\_40\_6 \textcolor{red}{\textcjheb{wmwqmm}} MMQWMW $|$von seinem (Wohn)Ort\\
\end{tabular}\medskip \\
Ende des Verses 27.8\\
Verse: 777, Buchstaben: 30, 230, 22288, Totalwerte: 1782, 16067, 1576206\\
\\
Wie ein Vogel, der fern von seinem Neste schweift: so ein Mann, der fern von seinem Wohnorte schweift.\\
\newpage 
{\bf -- 27.9}\\
\medskip \\
\begin{tabular}{rrrrrrrrp{120mm}}
WV&WK&WB&ABK&ABB&ABV&AnzB&TW&Zahlencode \textcolor{red}{$\boldsymbol{Grundtext}$} Umschrift $|$"Ubersetzung(en)\\
1.&62.&5812.&231.&22289.&1.&3&390&300\_40\_50 \textcolor{red}{\textcjheb{nm+s}} SMN $|$(wie) "Ol\\
2.&63.&5813.&234.&22292.&4.&5&715&6\_100\_9\_200\_400 \textcolor{red}{\textcjheb{tr.tqw}} WQtRT $|$und R"aucherwerk\\
3.&64.&5814.&239.&22297.&9.&4&358&10\_300\_40\_8 \textcolor{red}{\textcjheb{.hm+sy}} JSMC $|$erfreuen/er (=es) erfreut\\
4.&65.&5815.&243.&22301.&13.&2&32&30\_2 \textcolor{red}{\textcjheb{bl}} LB $|$das Herz\\
5.&66.&5816.&245.&22303.&15.&4&546&6\_40\_400\_100 \textcolor{red}{\textcjheb{qtmw}} WMTQ $|$und die S"u"sigkeit/und die Annehmlichkeit\\
6.&67.&5817.&249.&22307.&19.&4&281&200\_70\_5\_6 \textcolor{red}{\textcjheb{wh`r}} RaHW $|$(s)eines Freundes\\
7.&68.&5818.&253.&22311.&23.&4&600&40\_70\_90\_400 \textcolor{red}{\textcjheb{t.s`m}} Ma"sT $|$kommt aus dem Rat/mehr (als der) Rat\\
8.&69.&5819.&257.&22315.&27.&3&430&50\_80\_300 \textcolor{red}{\textcjheb{+spn}} NPS $|$der Seele\\
\end{tabular}\medskip \\
Ende des Verses 27.9\\
Verse: 778, Buchstaben: 29, 259, 22317, Totalwerte: 3352, 19419, 1579558\\
\\
"Ol und R"aucherwerk erfreuen das Herz, und die S"u"sigkeit eines Freundes kommt aus dem Rate der Seele.\\
\newpage 
{\bf -- 27.10}\\
\medskip \\
\begin{tabular}{rrrrrrrrp{120mm}}
WV&WK&WB&ABK&ABB&ABV&AnzB&TW&Zahlencode \textcolor{red}{$\boldsymbol{Grundtext}$} Umschrift $|$"Ubersetzung(en)\\
1.&70.&5820.&260.&22318.&1.&3&290&200\_70\_20 \textcolor{red}{\textcjheb{k`r}} RaK $|$deinen Freund\\
2.&71.&5821.&263.&22321.&4.&4&281&6\_200\_70\_5 \textcolor{red}{\textcjheb{h`rw}} WRaH $|$und (einen) Freund\\
3.&72.&5822.&267.&22325.&8.&4&33&1\_2\_10\_20 \textcolor{red}{\textcjheb{kyb'}} ABJK $|$deines Vaters\\
4.&73.&5823.&271.&22329.&12.&2&31&1\_30 \textcolor{red}{\textcjheb{l'}} AL $|$nicht\\
5.&74.&5824.&273.&22331.&14.&4&479&400\_70\_7\_2 \textcolor{red}{\textcjheb{bz`t}} TaZB $|$(sollst du) verlass(en)\\
6.&75.&5825.&277.&22335.&18.&4&418&6\_2\_10\_400 \textcolor{red}{\textcjheb{tybw}} WBJT $|$und (in) das Haus\\
7.&76.&5826.&281.&22339.&22.&4&39&1\_8\_10\_20 \textcolor{red}{\textcjheb{ky.h'}} ACJK $|$deines Bruders\\
8.&77.&5827.&285.&22343.&26.&2&31&1\_30 \textcolor{red}{\textcjheb{l'}} AL $|$nicht\\
9.&78.&5828.&287.&22345.&28.&4&409&400\_2\_6\_1 \textcolor{red}{\textcjheb{'wbt}} TBWA $|$geh/du sollst betreten\\
10.&79.&5829.&291.&22349.&32.&4&58&2\_10\_6\_40 \textcolor{red}{\textcjheb{mwyb}} BJWM $|$am Tag\\
11.&80.&5830.&295.&22353.&36.&4&35&1\_10\_4\_20 \textcolor{red}{\textcjheb{kdy'}} AJDK $|$deiner Not/deines Ungl"ucks\\
12.&81.&5831.&299.&22357.&40.&3&17&9\_6\_2 \textcolor{red}{\textcjheb{bw.t}} tWB $|$besser/gut (ist)\\
13.&82.&5832.&302.&22360.&43.&3&370&300\_20\_50 \textcolor{red}{\textcjheb{nk+s}} SKN $|$ein Nachbar\\
14.&83.&5833.&305.&22363.&46.&4&308&100\_200\_6\_2 \textcolor{red}{\textcjheb{bwrq}} QRWB $|$naher\\
15.&84.&5834.&309.&22367.&50.&3&49&40\_1\_8 \textcolor{red}{\textcjheb{.h'm}} MAC $|$als ein Bruder\\
16.&85.&5835.&312.&22370.&53.&4&314&200\_8\_6\_100 \textcolor{red}{\textcjheb{qw.hr}} RCWQ $|$ferner\\
\end{tabular}\medskip \\
Ende des Verses 27.10\\
Verse: 779, Buchstaben: 56, 315, 22373, Totalwerte: 3162, 22581, 1582720\\
\\
Verla"s nicht deinen Freund und deines Vaters Freund, und geh nicht am Tage deiner Not in deines Bruders Haus: besser ein naher Nachbar als ein ferner Bruder.\\
\newpage 
{\bf -- 27.11}\\
\medskip \\
\begin{tabular}{rrrrrrrrp{120mm}}
WV&WK&WB&ABK&ABB&ABV&AnzB&TW&Zahlencode \textcolor{red}{$\boldsymbol{Grundtext}$} Umschrift $|$"Ubersetzung(en)\\
1.&86.&5836.&316.&22374.&1.&3&68&8\_20\_40 \textcolor{red}{\textcjheb{mk.h}} CKM $|$(sei) weise\\
2.&87.&5837.&319.&22377.&4.&3&62&2\_50\_10 \textcolor{red}{\textcjheb{ynb}} BNJ $|$mein Sohn\\
3.&88.&5838.&322.&22380.&7.&4&354&6\_300\_40\_8 \textcolor{red}{\textcjheb{.hm+sw}} WSMC $|$und erfreue\\
4.&89.&5839.&326.&22384.&11.&3&42&30\_2\_10 \textcolor{red}{\textcjheb{ybl}} LBJ $|$mein Herz\\
5.&90.&5840.&329.&22387.&14.&6&324&6\_1\_300\_10\_2\_5 \textcolor{red}{\textcjheb{hby+s'w}} WASJBH $|$damit ich geben k"onne/und ich kann erwidern\\
6.&91.&5841.&335.&22393.&20.&4&298&8\_200\_80\_10 \textcolor{red}{\textcjheb{ypr.h}} CRPJ $|$meinem Schm"aher/meinem Schm"ahenden mich\\
7.&92.&5842.&339.&22397.&24.&3&206&4\_2\_200 \textcolor{red}{\textcjheb{rbd}} DBR $|$Antwort/(ein) Wort\\
\end{tabular}\medskip \\
Ende des Verses 27.11\\
Verse: 780, Buchstaben: 26, 341, 22399, Totalwerte: 1354, 23935, 1584074\\
\\
Sei weise, mein Sohn, und erfreue mein Herz, damit ich Antwort geben k"onne meinem Schm"aher.\\
\newpage 
{\bf -- 27.12}\\
\medskip \\
\begin{tabular}{rrrrrrrrp{120mm}}
WV&WK&WB&ABK&ABB&ABV&AnzB&TW&Zahlencode \textcolor{red}{$\boldsymbol{Grundtext}$} Umschrift $|$"Ubersetzung(en)\\
1.&93.&5843.&342.&22400.&1.&4&316&70\_200\_6\_40 \textcolor{red}{\textcjheb{mwr`}} aRWM $|$der Kluge/(ein) Kluger\\
2.&94.&5844.&346.&22404.&5.&3&206&200\_1\_5 \textcolor{red}{\textcjheb{h'r}} RAH $|$(er) sieht\\
3.&95.&5845.&349.&22407.&8.&3&275&200\_70\_5 \textcolor{red}{\textcjheb{h`r}} RaH $|$(das) Ungl"uck\\
4.&96.&5846.&352.&22410.&11.&4&710&50\_60\_400\_200 \textcolor{red}{\textcjheb{rtsn}} NsTR $|$(und) (er) verbirgt sich\\
5.&97.&5847.&356.&22414.&15.&5&531&80\_400\_1\_10\_40 \textcolor{red}{\textcjheb{my'tp}} PTAJM $|$(die) Einf"altige(n)\\
6.&98.&5848.&361.&22419.&20.&4&278&70\_2\_200\_6 \textcolor{red}{\textcjheb{wrb`}} aBRW $|$(sie) gehen weiter\\
7.&99.&5849.&365.&22423.&24.&5&476&50\_70\_50\_300\_6 \textcolor{red}{\textcjheb{w+sn`n}} NaNSW $|$und leiden Strafe/(und) sie m"ussen b"u"sen (es)\\
\end{tabular}\medskip \\
Ende des Verses 27.12\\
Verse: 781, Buchstaben: 28, 369, 22427, Totalwerte: 2792, 26727, 1586866\\
\\
Der Kluge sieht das Ungl"uck und verbirgt sich; die Einf"altigen gehen weiter und leiden Strafe.\\
\newpage 
{\bf -- 27.13}\\
\medskip \\
\begin{tabular}{rrrrrrrrp{120mm}}
WV&WK&WB&ABK&ABB&ABV&AnzB&TW&Zahlencode \textcolor{red}{$\boldsymbol{Grundtext}$} Umschrift $|$"Ubersetzung(en)\\
1.&100.&5850.&370.&22428.&1.&2&108&100\_8 \textcolor{red}{\textcjheb{.hq}} QC $|$nimm\\
2.&101.&5851.&372.&22430.&3.&4&15&2\_3\_4\_6 \textcolor{red}{\textcjheb{wdgb}} BGDW $|$ihm das Kleid/sein Gewand\\
3.&102.&5852.&376.&22434.&7.&2&30&20\_10 \textcolor{red}{\textcjheb{yk}} KJ $|$denn/weil\\
4.&103.&5853.&378.&22436.&9.&3&272&70\_200\_2 \textcolor{red}{\textcjheb{br`}} aRB $|$er ist B"urge geworden/er hat geb"urgt\\
5.&104.&5854.&381.&22439.&12.&2&207&7\_200 \textcolor{red}{\textcjheb{rz}} ZR $|$f"ur einen anderen/(f"ur einen) Fremden\\
6.&105.&5855.&383.&22441.&14.&4&82&6\_2\_70\_4 \textcolor{red}{\textcjheb{d`bw}} WBaD $|$und halber/und wegen\\
7.&106.&5856.&387.&22445.&18.&5&285&50\_20\_200\_10\_5 \textcolor{red}{\textcjheb{hyrkn}} NKRJH $|$der Fremden\\
8.&107.&5857.&392.&22450.&23.&5&51&8\_2\_30\_5\_6 \textcolor{red}{\textcjheb{whlb.h}} CBLHW $|$pf"ande ihn\\
\end{tabular}\medskip \\
Ende des Verses 27.13\\
Verse: 782, Buchstaben: 27, 396, 22454, Totalwerte: 1050, 27777, 1587916\\
\\
Nimm ihm das Kleid, denn er ist f"ur einen anderen B"urge geworden; und der Fremden halber pf"ande ihn.\\
\newpage 
{\bf -- 27.14}\\
\medskip \\
\begin{tabular}{rrrrrrrrp{120mm}}
WV&WK&WB&ABK&ABB&ABV&AnzB&TW&Zahlencode \textcolor{red}{$\boldsymbol{Grundtext}$} Umschrift $|$"Ubersetzung(en)\\
1.&108.&5858.&397.&22455.&1.&4&262&40\_2\_200\_20 \textcolor{red}{\textcjheb{krbm}} MBRK $|$wer Gl"uck w"unscht/(ein) Segnender\\
2.&109.&5859.&401.&22459.&5.&4&281&200\_70\_5\_6 \textcolor{red}{\textcjheb{wh`r}} RaHW $|$(und) seinem N"achsten/seinen Gef"ahrten\\
3.&110.&5860.&405.&22463.&9.&4&138&2\_100\_6\_30 \textcolor{red}{\textcjheb{lwqb}} BQWL $|$mit Stimme\\
4.&111.&5861.&409.&22467.&13.&4&43&3\_4\_6\_30 \textcolor{red}{\textcjheb{lwdg}} GDWL $|$lauter\\
5.&112.&5862.&413.&22471.&17.&4&304&2\_2\_100\_200 \textcolor{red}{\textcjheb{rqbb}} BBQR $|$fr"uhmorgens/am Morgen\\
6.&113.&5863.&417.&22475.&21.&5&375&5\_300\_20\_10\_40 \textcolor{red}{\textcjheb{myk+sh}} HSKJM $|$aufsteht/ein Aufstehen\\
7.&114.&5864.&422.&22480.&26.&4&165&100\_30\_30\_5 \textcolor{red}{\textcjheb{hllq}} QLLH $|$als Verw"unschung/(als) Fluch\\
8.&115.&5865.&426.&22484.&30.&4&710&400\_8\_300\_2 \textcolor{red}{\textcjheb{b+s.ht}} TCSB $|$wird es angerechnet/wird sie (=es) wird zugerechnet werden\\
9.&116.&5866.&430.&22488.&34.&2&36&30\_6 \textcolor{red}{\textcjheb{wl}} LW $|$ihm\\
\end{tabular}\medskip \\
Ende des Verses 27.14\\
Verse: 783, Buchstaben: 35, 431, 22489, Totalwerte: 2314, 30091, 1590230\\
\\
Wer fr"uhmorgens aufsteht und seinem N"achsten mit lauter Stimme Gl"uck w"unscht, als Verw"unschung wird es ihm angerechnet.\\
\newpage 
{\bf -- 27.15}\\
\medskip \\
\begin{tabular}{rrrrrrrrp{120mm}}
WV&WK&WB&ABK&ABB&ABV&AnzB&TW&Zahlencode \textcolor{red}{$\boldsymbol{Grundtext}$} Umschrift $|$"Ubersetzung(en)\\
1.&117.&5867.&432.&22490.&1.&3&114&4\_30\_80 \textcolor{red}{\textcjheb{pld}} DLP $|$eine (Dach)Traufe\\
2.&118.&5868.&435.&22493.&4.&4&219&9\_6\_200\_4 \textcolor{red}{\textcjheb{drw.t}} tWRD $|$best"andige/rinnend\\
3.&119.&5869.&439.&22497.&8.&4&58&2\_10\_6\_40 \textcolor{red}{\textcjheb{mwyb}} BJWM $|$am Tag\\
4.&120.&5870.&443.&22501.&12.&5&473&60\_3\_200\_10\_200 \textcolor{red}{\textcjheb{ryrgs}} sGRJR $|$des str"omenden Regens/heftigen Regens\\
5.&121.&5871.&448.&22506.&17.&4&707&6\_1\_300\_400 \textcolor{red}{\textcjheb{t+s'w}} WAST $|$und eine Frau\\
6.&122.&5872.&452.&22510.&21.&6&150&40\_4\_6\_50\_10\_40 \textcolor{red}{\textcjheb{mynwdm}} MDWNJM $|$z"ankische/(von) Streitigkeiten\\
7.&123.&5873.&458.&22516.&27.&5&761&50\_300\_400\_6\_5 \textcolor{red}{\textcjheb{hwt+sn}} NSTWH $|$(sie) gleichen sich\\
\end{tabular}\medskip \\
Ende des Verses 27.15\\
Verse: 784, Buchstaben: 31, 462, 22520, Totalwerte: 2482, 32573, 1592712\\
\\
Eine best"andige Traufe am Tage des str"omenden Regens und ein z"ankisches Weib gleichen sich.\\
\newpage 
{\bf -- 27.16}\\
\medskip \\
\begin{tabular}{rrrrrrrrp{120mm}}
WV&WK&WB&ABK&ABB&ABV&AnzB&TW&Zahlencode \textcolor{red}{$\boldsymbol{Grundtext}$} Umschrift $|$"Ubersetzung(en)\\
1.&124.&5874.&463.&22521.&1.&5&235&90\_80\_50\_10\_5 \textcolor{red}{\textcjheb{hynp.s}} "sPNJH $|$wer dies zur"uckh"alt/ein Bergender sie\\
2.&125.&5875.&468.&22526.&6.&3&220&90\_80\_50 \textcolor{red}{\textcjheb{np.s}} "sPN $|$h"alt zur"uck/(d)er birgt\\
3.&126.&5876.&471.&22529.&9.&3&214&200\_6\_8 \textcolor{red}{\textcjheb{.hwr}} RWC $|$(den) Wind\\
4.&127.&5877.&474.&22532.&12.&4&396&6\_300\_40\_50 \textcolor{red}{\textcjheb{nm+sw}} WSMN $|$und in "Ol/und (auf) "Ol\\
5.&128.&5878.&478.&22536.&16.&5&116&10\_40\_10\_50\_6 \textcolor{red}{\textcjheb{wnymy}} JMJNW $|$seine Rechte\\
6.&129.&5879.&483.&22541.&21.&4&311&10\_100\_200\_1 \textcolor{red}{\textcjheb{'rqy}} JQRA $|$greift/er (=sie) trifft\\
\end{tabular}\medskip \\
Ende des Verses 27.16\\
Verse: 785, Buchstaben: 24, 486, 22544, Totalwerte: 1492, 34065, 1594204\\
\\
Wer dieses zur"uckh"alt, h"alt den Wind zur"uck und seine Rechte greift in "Ol.\\
\newpage 
{\bf -- 27.17}\\
\medskip \\
\begin{tabular}{rrrrrrrrp{120mm}}
WV&WK&WB&ABK&ABB&ABV&AnzB&TW&Zahlencode \textcolor{red}{$\boldsymbol{Grundtext}$} Umschrift $|$"Ubersetzung(en)\\
1.&130.&5880.&487.&22545.&1.&4&239&2\_200\_7\_30 \textcolor{red}{\textcjheb{lzrb}} BRZL $|$Eisen\\
2.&131.&5881.&491.&22549.&5.&5&241&2\_2\_200\_7\_30 \textcolor{red}{\textcjheb{lzrbb}} BBRZL $|$durch Eisen/mit Eisen\\
3.&132.&5882.&496.&22554.&10.&3&22&10\_8\_4 \textcolor{red}{\textcjheb{d.hy}} JCD $|$wird scharf/er (=es) wird gesch"arft\\
4.&133.&5883.&499.&22557.&13.&4&317&6\_1\_10\_300 \textcolor{red}{\textcjheb{+sy'w}} WAJS $|$und (ein) Mann\\
5.&134.&5884.&503.&22561.&17.&3&22&10\_8\_4 \textcolor{red}{\textcjheb{d.hy}} JCD $|$sch"arft/(er) macht sch"arfen\\
6.&135.&5885.&506.&22564.&20.&3&140&80\_50\_10 \textcolor{red}{\textcjheb{ynp}} PNJ $|$das Gesicht/die Gesichter\\
7.&136.&5886.&509.&22567.&23.&4&281&200\_70\_5\_6 \textcolor{red}{\textcjheb{wh`r}} RaHW $|$des anderen/seines Gef"ahrten\\
\end{tabular}\medskip \\
Ende des Verses 27.17\\
Verse: 786, Buchstaben: 26, 512, 22570, Totalwerte: 1262, 35327, 1595466\\
\\
Eisen wird scharf durch Eisen, und ein Mann sch"arft das Angesicht des anderen.\\
\newpage 
{\bf -- 27.18}\\
\medskip \\
\begin{tabular}{rrrrrrrrp{120mm}}
WV&WK&WB&ABK&ABB&ABV&AnzB&TW&Zahlencode \textcolor{red}{$\boldsymbol{Grundtext}$} Umschrift $|$"Ubersetzung(en)\\
1.&137.&5887.&513.&22571.&1.&3&340&50\_90\_200 \textcolor{red}{\textcjheb{r.sn}} N"sR $|$wer wartet/(ein) Pflegender\\
2.&138.&5888.&516.&22574.&4.&4&456&400\_1\_50\_5 \textcolor{red}{\textcjheb{hn't}} TANH $|$des Feigenbaums/den Feigenbaum\\
3.&139.&5889.&520.&22578.&8.&4&61&10\_1\_20\_30 \textcolor{red}{\textcjheb{lk'y}} JAKL $|$((d)er) wird essen\\
4.&140.&5890.&524.&22582.&12.&4&295&80\_200\_10\_5 \textcolor{red}{\textcjheb{hyrp}} PRJH $|$seine Frucht/dessen Frucht\\
5.&141.&5891.&528.&22586.&16.&4&546&6\_300\_40\_200 \textcolor{red}{\textcjheb{rm+sw}} WSMR $|$und wer wacht "uber/und ein Achtender\\
6.&142.&5892.&532.&22590.&20.&5&71&1\_4\_50\_10\_6 \textcolor{red}{\textcjheb{wynd'}} ADNJW $|$seinen Herrn\\
7.&143.&5893.&537.&22595.&25.&4&36&10\_20\_2\_4 \textcolor{red}{\textcjheb{dbky}} JKBD $|$wird geehrt werden/(d)er wird geehrt\\
\end{tabular}\medskip \\
Ende des Verses 27.18\\
Verse: 787, Buchstaben: 28, 540, 22598, Totalwerte: 1805, 37132, 1597271\\
\\
Wer des Feigenbaumes wartet, wird seine Frucht essen; und wer "uber seinen Herrn wacht, wird geehrt werden.\\
\newpage 
{\bf -- 27.19}\\
\medskip \\
\begin{tabular}{rrrrrrrrp{120mm}}
WV&WK&WB&ABK&ABB&ABV&AnzB&TW&Zahlencode \textcolor{red}{$\boldsymbol{Grundtext}$} Umschrift $|$"Ubersetzung(en)\\
1.&144.&5894.&541.&22599.&1.&4&110&20\_40\_10\_40 \textcolor{red}{\textcjheb{mymk}} KMJM $|$wie (im) Wasser\\
2.&145.&5895.&545.&22603.&5.&5&185&5\_80\_50\_10\_40 \textcolor{red}{\textcjheb{mynph}} HPNJM $|$das Angesicht/das Antlitz\\
3.&146.&5896.&550.&22608.&10.&5&210&30\_80\_50\_10\_40 \textcolor{red}{\textcjheb{mynpl}} LPNJM $|$dem Angesicht entspricht/dem Antlitz (zeigt)\\
4.&147.&5897.&555.&22613.&15.&2&70&20\_50 \textcolor{red}{\textcjheb{nk}} KN $|$so\\
5.&148.&5898.&557.&22615.&17.&2&32&30\_2 \textcolor{red}{\textcjheb{bl}} LB $|$das Herz\\
6.&149.&5899.&559.&22617.&19.&4&50&5\_1\_4\_40 \textcolor{red}{\textcjheb{md'h}} HADM $|$des Menschen\\
7.&150.&5900.&563.&22621.&23.&4&75&30\_1\_4\_40 \textcolor{red}{\textcjheb{md'l}} LADM $|$dem Menschen\\
\end{tabular}\medskip \\
Ende des Verses 27.19\\
Verse: 788, Buchstaben: 26, 566, 22624, Totalwerte: 732, 37864, 1598003\\
\\
Wie im Wasser das Angesicht dem Angesicht entspricht, so das Herz des Menschen dem Menschen.\\
\newpage 
{\bf -- 27.20}\\
\medskip \\
\begin{tabular}{rrrrrrrrp{120mm}}
WV&WK&WB&ABK&ABB&ABV&AnzB&TW&Zahlencode \textcolor{red}{$\boldsymbol{Grundtext}$} Umschrift $|$"Ubersetzung(en)\\
1.&151.&5901.&567.&22625.&1.&4&337&300\_1\_6\_30 \textcolor{red}{\textcjheb{lw'+s}} SAWL $|$Scheol/Totenreich\\
2.&152.&5902.&571.&22629.&5.&5&18&6\_1\_2\_4\_5 \textcolor{red}{\textcjheb{hdb'w}} WABDH $|$und Abgrund/und Unterwelt\\
3.&153.&5903.&576.&22634.&10.&2&31&30\_1 \textcolor{red}{\textcjheb{'l}} LA $|$sind un-/nicht\\
4.&154.&5904.&578.&22636.&12.&6&827&400\_300\_2\_70\_50\_5 \textcolor{red}{\textcjheb{hn`b+st}} TSBaNH $|$ers"attlich/(sie) werden satt\\
5.&155.&5905.&584.&22642.&18.&5&146&6\_70\_10\_50\_10 \textcolor{red}{\textcjheb{yny`w}} WaJNJ $|$so die Augen/und die Augen\\
6.&156.&5906.&589.&22647.&23.&4&50&5\_1\_4\_40 \textcolor{red}{\textcjheb{md'h}} HADM $|$des Menschen\\
7.&157.&5907.&593.&22651.&27.&2&31&30\_1 \textcolor{red}{\textcjheb{'l}} LA $|$sind un-/nicht\\
8.&158.&5908.&595.&22653.&29.&6&827&400\_300\_2\_70\_50\_5 \textcolor{red}{\textcjheb{hn`b+st}} TSBaNH $|$ers"attlich/(sie) werden satt\\
\end{tabular}\medskip \\
Ende des Verses 27.20\\
Verse: 789, Buchstaben: 34, 600, 22658, Totalwerte: 2267, 40131, 1600270\\
\\
Scheol und Abgrund sind uners"attlich: so sind uners"attlich die Augen des Menschen.\\
\newpage 
{\bf -- 27.21}\\
\medskip \\
\begin{tabular}{rrrrrrrrp{120mm}}
WV&WK&WB&ABK&ABB&ABV&AnzB&TW&Zahlencode \textcolor{red}{$\boldsymbol{Grundtext}$} Umschrift $|$"Ubersetzung(en)\\
1.&159.&5909.&601.&22659.&1.&4&410&40\_90\_200\_80 \textcolor{red}{\textcjheb{pr.sm}} M"sRP $|$der Schmelztiegel/wie ein Schmelztiegel\\
2.&160.&5910.&605.&22663.&5.&4&190&30\_20\_60\_80 \textcolor{red}{\textcjheb{pskl}} LKsP $|$f"ur das Silber\\
3.&161.&5911.&609.&22667.&9.&4&232&6\_20\_6\_200 \textcolor{red}{\textcjheb{rwkw}} WKWR $|$und der Ofen/und ein Ofen\\
4.&162.&5912.&613.&22671.&13.&4&44&30\_7\_5\_2 \textcolor{red}{\textcjheb{bhzl}} LZHB $|$f"ur das Gold\\
5.&163.&5913.&617.&22675.&17.&4&317&6\_1\_10\_300 \textcolor{red}{\textcjheb{+sy'w}} WAJS $|$und (ein) Mann\\
6.&164.&5914.&621.&22679.&21.&3&120&30\_80\_10 \textcolor{red}{\textcjheb{ypl}} LPJ $|$nach Ma"sgabe/entsprechend\\
7.&165.&5915.&624.&22682.&24.&5&111&40\_5\_30\_30\_6 \textcolor{red}{\textcjheb{wllhm}} MHLLW $|$seines Lobes/seiner Anerkennung\\
\end{tabular}\medskip \\
Ende des Verses 27.21\\
Verse: 790, Buchstaben: 28, 628, 22686, Totalwerte: 1424, 41555, 1601694\\
\\
Der Schmelztiegel f"ur das Silber, und der Ofen f"ur das Gold; und ein Mann nach Ma"sgabe seines Lobes.\\
\newpage 
{\bf -- 27.22}\\
\medskip \\
\begin{tabular}{rrrrrrrrp{120mm}}
WV&WK&WB&ABK&ABB&ABV&AnzB&TW&Zahlencode \textcolor{red}{$\boldsymbol{Grundtext}$} Umschrift $|$"Ubersetzung(en)\\
1.&166.&5916.&629.&22687.&1.&2&41&1\_40 \textcolor{red}{\textcjheb{m'}} AM $|$wenn\\
2.&167.&5917.&631.&22689.&3.&5&1126&400\_20\_400\_6\_300 \textcolor{red}{\textcjheb{+swtkt}} TKTWS $|$du zerstie"sest\\
3.&168.&5918.&636.&22694.&8.&2&401&1\_400 \textcolor{red}{\textcjheb{t'}} AT $|$**\\
4.&169.&5919.&638.&22696.&10.&5&52&5\_1\_6\_10\_30 \textcolor{red}{\textcjheb{lyw'h}} HAWJL $|$den Narren/den Toren\\
5.&170.&5920.&643.&22701.&15.&5&762&2\_40\_20\_400\_300 \textcolor{red}{\textcjheb{+stkmb}} BMKTS $|$im M"orser\\
6.&171.&5921.&648.&22706.&20.&4&428&2\_400\_6\_20 \textcolor{red}{\textcjheb{kwtb}} BTWK $|$mitten unter/in Mitte\\
7.&172.&5922.&652.&22710.&24.&6&701&5\_200\_10\_80\_6\_400 \textcolor{red}{\textcjheb{twpyrh}} HRJPWT $|$der Gr"utze/der K"orner\\
8.&173.&5923.&658.&22716.&30.&4&112&2\_70\_30\_10 \textcolor{red}{\textcjheb{yl`b}} BaLJ $|$mit der Keule/mit dem St"o"sel\\
9.&174.&5924.&662.&22720.&34.&2&31&30\_1 \textcolor{red}{\textcjheb{'l}} LA $|$(so) nicht (doch)\\
10.&175.&5925.&664.&22722.&36.&4&666&400\_60\_6\_200 \textcolor{red}{\textcjheb{rwst}} TsWR $|$w"urde weichen/sie (=es) weicht\\
11.&176.&5926.&668.&22726.&40.&5&156&40\_70\_30\_10\_6 \textcolor{red}{\textcjheb{wyl`m}} MaLJW $|$(weg) von ihm\\
12.&177.&5927.&673.&22731.&45.&5&443&1\_6\_30\_400\_6 \textcolor{red}{\textcjheb{wtlw'}} AWLTW $|$seine Narrheit/seine Torheit\\
\end{tabular}\medskip \\
Ende des Verses 27.22\\
Verse: 791, Buchstaben: 49, 677, 22735, Totalwerte: 4919, 46474, 1606613\\
\\
Wenn du den Narren mit der Keule im M"orser zerstie"sest, mitten unter der Gr"utze, so w"urde seine Narrheit doch nicht von ihm weichen.\\
\newpage 
{\bf -- 27.23}\\
\medskip \\
\begin{tabular}{rrrrrrrrp{120mm}}
WV&WK&WB&ABK&ABB&ABV&AnzB&TW&Zahlencode \textcolor{red}{$\boldsymbol{Grundtext}$} Umschrift $|$"Ubersetzung(en)\\
1.&178.&5928.&678.&22736.&1.&3&84&10\_4\_70 \textcolor{red}{\textcjheb{`dy}} JDa $|$bek"ummere dich/achte\\
2.&179.&5929.&681.&22739.&4.&3&474&400\_4\_70 \textcolor{red}{\textcjheb{`dt}} TDa $|$wohl um/du sollst achten auf\\
3.&180.&5930.&684.&22742.&7.&3&140&80\_50\_10 \textcolor{red}{\textcjheb{ynp}} PNJ $|$das Aussehen/die Gesichter\\
4.&181.&5931.&687.&22745.&10.&4&161&90\_1\_50\_20 \textcolor{red}{\textcjheb{kn'.s}} "sANK $|$deines Kleinviehs\\
5.&182.&5932.&691.&22749.&14.&3&710&300\_10\_400 \textcolor{red}{\textcjheb{ty+s}} SJT $|$richte\\
6.&183.&5933.&694.&22752.&17.&3&52&30\_2\_20 \textcolor{red}{\textcjheb{kbl}} LBK $|$deine Aufmerksamkeit/dein Herz\\
7.&184.&5934.&697.&22755.&20.&6&354&30\_70\_4\_200\_10\_40 \textcolor{red}{\textcjheb{myrd`l}} LaDRJM $|$auf die Herden\\
\end{tabular}\medskip \\
Ende des Verses 27.23\\
Verse: 792, Buchstaben: 25, 702, 22760, Totalwerte: 1975, 48449, 1608588\\
\\
Bek"ummere dich wohl um das Aussehen deines Kleinviehes, richte deine Aufmerksamkeit auf die Herden.\\
\newpage 
{\bf -- 27.24}\\
\medskip \\
\begin{tabular}{rrrrrrrrp{120mm}}
WV&WK&WB&ABK&ABB&ABV&AnzB&TW&Zahlencode \textcolor{red}{$\boldsymbol{Grundtext}$} Umschrift $|$"Ubersetzung(en)\\
1.&185.&5935.&703.&22761.&1.&2&30&20\_10 \textcolor{red}{\textcjheb{yk}} KJ $|$denn\\
2.&186.&5936.&705.&22763.&3.&2&31&30\_1 \textcolor{red}{\textcjheb{'l}} LA $|$nicht\\
3.&187.&5937.&707.&22765.&5.&5&176&30\_70\_6\_30\_40 \textcolor{red}{\textcjheb{mlw`l}} LaWLM $|$(auf) ewig\\
4.&188.&5938.&712.&22770.&10.&3&118&8\_60\_50 \textcolor{red}{\textcjheb{ns.h}} CsN $|$ist Wohlstand/(bleibt) Reichtum\\
5.&189.&5939.&715.&22773.&13.&3&47&6\_1\_40 \textcolor{red}{\textcjheb{m'w}} WAM $|$und (nicht)\\
6.&190.&5940.&718.&22776.&16.&3&257&50\_7\_200 \textcolor{red}{\textcjheb{rzn}} NZR $|$w"ahrt eine Krone/Diadem (=Verm"ogen)\\
7.&191.&5941.&721.&22779.&19.&4&240&30\_4\_6\_200 \textcolor{red}{\textcjheb{rwdl}} LDWR $|$von Geschlecht\\
8.&192.&5942.&725.&22783.&23.&3&210&4\_6\_200 \textcolor{red}{\textcjheb{rwd}} DWR $|$zu Geschlecht\\
\end{tabular}\medskip \\
Ende des Verses 27.24\\
Verse: 793, Buchstaben: 25, 727, 22785, Totalwerte: 1109, 49558, 1609697\\
\\
Denn Wohlstand ist nicht ewig; und w"ahrt eine Krone von Geschlecht zu Geschlecht?\\
\newpage 
{\bf -- 27.25}\\
\medskip \\
\begin{tabular}{rrrrrrrrp{120mm}}
WV&WK&WB&ABK&ABB&ABV&AnzB&TW&Zahlencode \textcolor{red}{$\boldsymbol{Grundtext}$} Umschrift $|$"Ubersetzung(en)\\
1.&193.&5943.&728.&22786.&1.&3&38&3\_30\_5 \textcolor{red}{\textcjheb{hlg}} GLH $|$ist geschwunden/er deckte auf\\
2.&194.&5944.&731.&22789.&4.&4&308&8\_90\_10\_200 \textcolor{red}{\textcjheb{ry.s.h}} C"sJR $|$das Heu/Gras\\
3.&195.&5945.&735.&22793.&8.&5&262&6\_50\_200\_1\_5 \textcolor{red}{\textcjheb{h'rnw}} WNRAH $|$und erscheint/und er (=es) zeig(t)e sich\\
4.&196.&5946.&740.&22798.&13.&3&305&4\_300\_1 \textcolor{red}{\textcjheb{'+sd}} DSA $|$das junge Gras/Frischgr"un\\
5.&197.&5947.&743.&22801.&16.&6&203&6\_50\_1\_60\_80\_6 \textcolor{red}{\textcjheb{wps'nw}} WNAsPW $|$und sind eingesammelt/und sie (=es) m"ussen gesammelt werden\\
6.&198.&5948.&749.&22807.&22.&5&778&70\_300\_2\_6\_400 \textcolor{red}{\textcjheb{twb+s`}} aSBWT $|$die Kr"auter\\
7.&199.&5949.&754.&22812.&27.&4&255&5\_200\_10\_40 \textcolor{red}{\textcjheb{myrh}} HRJM $|$der Berge\\
\end{tabular}\medskip \\
Ende des Verses 27.25\\
Verse: 794, Buchstaben: 30, 757, 22815, Totalwerte: 2149, 51707, 1611846\\
\\
Ist geschwunden das Heu, und erscheint das junge Gras, und sind eingesammelt die Kr"auter der Berge,\\
\newpage 
{\bf -- 27.26}\\
\medskip \\
\begin{tabular}{rrrrrrrrp{120mm}}
WV&WK&WB&ABK&ABB&ABV&AnzB&TW&Zahlencode \textcolor{red}{$\boldsymbol{Grundtext}$} Umschrift $|$"Ubersetzung(en)\\
1.&200.&5950.&758.&22816.&1.&5&372&20\_2\_300\_10\_40 \textcolor{red}{\textcjheb{my+sbk}} KBSJM $|$so Schafe/L"ammer\\
2.&201.&5951.&763.&22821.&6.&6&388&30\_30\_2\_6\_300\_20 \textcolor{red}{\textcjheb{k+swbll}} LLBWSK $|$(dienen) zu deiner (Be)Kleidung\\
3.&202.&5952.&769.&22827.&12.&5&264&6\_40\_8\_10\_200 \textcolor{red}{\textcjheb{ry.hmw}} WMCJR $|$und der Kaufpreis\\
4.&203.&5953.&774.&22832.&17.&3&309&300\_4\_5 \textcolor{red}{\textcjheb{hd+s}} SDH $|$f"ur ein Feld/eines Feldes\\
5.&204.&5954.&777.&22835.&20.&6&530&70\_400\_6\_4\_10\_40 \textcolor{red}{\textcjheb{mydwt`}} aTWDJM $|$(sind) B"ocke\\
\end{tabular}\medskip \\
Ende des Verses 27.26\\
Verse: 795, Buchstaben: 25, 782, 22840, Totalwerte: 1863, 53570, 1613709\\
\\
so dienen Schafe zu deiner Kleidung, und der Kaufpreis f"ur ein Feld sind B"ocke;\\
\newpage 
{\bf -- 27.27}\\
\medskip \\
\begin{tabular}{rrrrrrrrp{120mm}}
WV&WK&WB&ABK&ABB&ABV&AnzB&TW&Zahlencode \textcolor{red}{$\boldsymbol{Grundtext}$} Umschrift $|$"Ubersetzung(en)\\
1.&205.&5955.&783.&22841.&1.&3&20&6\_4\_10 \textcolor{red}{\textcjheb{ydw}} WDJ $|$und genug/und ausreichend\\
2.&206.&5956.&786.&22844.&4.&3&40&8\_30\_2 \textcolor{red}{\textcjheb{bl.h}} CLB $|$Milch\\
3.&207.&5957.&789.&22847.&7.&4&127&70\_7\_10\_40 \textcolor{red}{\textcjheb{myz`}} aZJM $|$(von) Ziegen\\
4.&208.&5958.&793.&22851.&11.&5&128&30\_30\_8\_40\_20 \textcolor{red}{\textcjheb{km.hll}} LLCMK $|$ist da zu deiner Nahrung/zu deiner Kost\\
5.&209.&5959.&798.&22856.&16.&4&108&30\_30\_8\_40 \textcolor{red}{\textcjheb{m.hll}} LLCM $|$zur Nahrung/zur Kost\\
6.&210.&5960.&802.&22860.&20.&4&432&2\_10\_400\_20 \textcolor{red}{\textcjheb{ktyb}} BJTK $|$deines Hauses\\
7.&211.&5961.&806.&22864.&24.&5&74&6\_8\_10\_10\_40 \textcolor{red}{\textcjheb{myy.hw}} WCJJM $|$und (zum) Leben(sunterhalt)\\
8.&212.&5962.&811.&22869.&29.&8&786&30\_50\_70\_200\_6\_400\_10\_20 \textcolor{red}{\textcjheb{kytwr`nl}} LNaRWTJK $|$f"ur deine M"agde\\
\end{tabular}\medskip \\
Ende des Verses 27.27\\
Verse: 796, Buchstaben: 36, 818, 22876, Totalwerte: 1715, 55285, 1615424\\
\\
und genug Ziegenmilch ist da zu deiner Nahrung, zur Nahrung deines Hauses, und Lebensunterhalt f"ur deine M"agde.\\
\\
{\bf Ende des Kapitels 27}\\
\newpage 
{\bf -- 28.1}\\
\medskip \\
\begin{tabular}{rrrrrrrrp{120mm}}
WV&WK&WB&ABK&ABB&ABV&AnzB&TW&Zahlencode \textcolor{red}{$\boldsymbol{Grundtext}$} Umschrift $|$"Ubersetzung(en)\\
1.&1.&5963.&1.&22877.&1.&3&116&50\_60\_6 \textcolor{red}{\textcjheb{wsn}} NsW $|$(sie) (=es) fliehen\\
2.&2.&5964.&4.&22880.&4.&4&67&6\_1\_10\_50 \textcolor{red}{\textcjheb{ny'w}} WAJN $|$obgleich da ist kein/und nicht (ist)\\
3.&3.&5965.&8.&22884.&8.&3&284&200\_4\_80 \textcolor{red}{\textcjheb{pdr}} RDP $|$Verfolger/(ein) Nachjagend(er)\\
4.&4.&5966.&11.&22887.&11.&3&570&200\_300\_70 \textcolor{red}{\textcjheb{`+sr}} RSa $|$die Gesetzlosen/dem Frevler\\
5.&5.&5967.&14.&22890.&14.&7&260&6\_90\_4\_10\_100\_10\_40 \textcolor{red}{\textcjheb{myqyd.sw}} W"sDJQJM $|$aber die Gerechten/und die Gerechten\\
6.&6.&5968.&21.&22897.&21.&5&330&20\_20\_80\_10\_200 \textcolor{red}{\textcjheb{rypkk}} KKPJR $|$gleich einem jungen L"owen/(sind) wie (ein) Jungleu\\
7.&7.&5969.&26.&22902.&26.&4&29&10\_2\_9\_8 \textcolor{red}{\textcjheb{.h.tby}} JBtC $|$sind getrost/unerschrocken\\
\end{tabular}\medskip \\
Ende des Verses 28.1\\
Verse: 797, Buchstaben: 29, 29, 22905, Totalwerte: 1656, 1656, 1617080\\
\\
Die Gesetzlosen fliehen, obgleich kein Verfolger da ist; die Gerechten aber sind getrost gleich einem jungen L"owen.\\
\newpage 
{\bf -- 28.2}\\
\medskip \\
\begin{tabular}{rrrrrrrrp{120mm}}
WV&WK&WB&ABK&ABB&ABV&AnzB&TW&Zahlencode \textcolor{red}{$\boldsymbol{Grundtext}$} Umschrift $|$"Ubersetzung(en)\\
1.&8.&5970.&30.&22906.&1.&4&452&2\_80\_300\_70 \textcolor{red}{\textcjheb{`+spb}} BPSa $|$durch die Frevelhaftigkeit/durch (die) S"unde\\
2.&9.&5971.&34.&22910.&5.&3&291&1\_200\_90 \textcolor{red}{\textcjheb{.sr'}} AR"s $|$eines Landes/des Landes \\
3.&10.&5972.&37.&22913.&8.&4&252&200\_2\_10\_40 \textcolor{red}{\textcjheb{mybr}} RBJM $|$werden viele/(sind) viele\\
4.&11.&5973.&41.&22917.&12.&4&515&300\_200\_10\_5 \textcolor{red}{\textcjheb{hyr+s}} SRJH $|$seiner F"ursten/ihre F"ursten\\
5.&12.&5974.&45.&22921.&16.&5&53&6\_2\_1\_4\_40 \textcolor{red}{\textcjheb{md'bw}} WBADM $|$aber durch einen Mann/und durch einen Menschen\\
6.&13.&5975.&50.&22926.&21.&4&102&40\_2\_10\_50 \textcolor{red}{\textcjheb{nybm}} MBJN $|$verst"andigen\\
7.&14.&5976.&54.&22930.&25.&3&84&10\_4\_70 \textcolor{red}{\textcjheb{`dy}} JDa $|$/der kennt\\
8.&15.&5977.&57.&22933.&28.&2&70&20\_50 \textcolor{red}{\textcjheb{nk}} KN $|$einsichtigen/(das) Recht\\
9.&16.&5978.&59.&22935.&30.&5&241&10\_1\_200\_10\_20 \textcolor{red}{\textcjheb{kyr'y}} JARJK $|$wird sein Bestand verl"angert/er besteht dauerhaft\\
\end{tabular}\medskip \\
Ende des Verses 28.2\\
Verse: 798, Buchstaben: 34, 63, 22939, Totalwerte: 2060, 3716, 1619140\\
\\
Durch die Frevelhaftigkeit eines Landes werden seiner F"ursten viele; aber durch einen verst"andigen, einsichtigen Mann wird sein Bestand verl"angert.\\
\newpage 
{\bf -- 28.3}\\
\medskip \\
\begin{tabular}{rrrrrrrrp{120mm}}
WV&WK&WB&ABK&ABB&ABV&AnzB&TW&Zahlencode \textcolor{red}{$\boldsymbol{Grundtext}$} Umschrift $|$"Ubersetzung(en)\\
1.&17.&5979.&64.&22940.&1.&3&205&3\_2\_200 \textcolor{red}{\textcjheb{rbg}} GBR $|$ein Mann\\
2.&18.&5980.&67.&22943.&4.&2&500&200\_300 \textcolor{red}{\textcjheb{+sr}} RS $|$armer\\
3.&19.&5981.&69.&22945.&6.&4&476&6\_70\_300\_100 \textcolor{red}{\textcjheb{q+s`w}} WaSQ $|$der bedr"uckt/und bedr"uckend(er)\\
4.&20.&5982.&73.&22949.&10.&4&84&4\_30\_10\_40 \textcolor{red}{\textcjheb{myld}} DLJM $|$(die) Geringe(n)\\
5.&21.&5983.&77.&22953.&14.&3&249&40\_9\_200 \textcolor{red}{\textcjheb{r.tm}} MtR $|$ist ein Regen/(gleicht dem) Regen\\
6.&22.&5984.&80.&22956.&17.&3&148&60\_8\_80 \textcolor{red}{\textcjheb{p.hs}} sCP $|$der hinwegschwemmt/(der) wegschwemmend (ist)\\
7.&23.&5985.&83.&22959.&20.&4&67&6\_1\_10\_50 \textcolor{red}{\textcjheb{ny'w}} WAJN $|$und bringt kein/und nicht gibt es\\
8.&24.&5986.&87.&22963.&24.&3&78&30\_8\_40 \textcolor{red}{\textcjheb{m.hl}} LCM $|$Brot\\
\end{tabular}\medskip \\
Ende des Verses 28.3\\
Verse: 799, Buchstaben: 26, 89, 22965, Totalwerte: 1807, 5523, 1620947\\
\\
Ein armer Mann, der Geringe bedr"uckt, ist ein Regen, der hinwegschwemmt und kein Brot bringt.\\
\newpage 
{\bf -- 28.4}\\
\medskip \\
\begin{tabular}{rrrrrrrrp{120mm}}
WV&WK&WB&ABK&ABB&ABV&AnzB&TW&Zahlencode \textcolor{red}{$\boldsymbol{Grundtext}$} Umschrift $|$"Ubersetzung(en)\\
1.&25.&5987.&90.&22966.&1.&4&89&70\_7\_2\_10 \textcolor{red}{\textcjheb{ybz`}} aZBJ $|$die verlassen/Verlassende\\
2.&26.&5988.&94.&22970.&5.&4&611&400\_6\_200\_5 \textcolor{red}{\textcjheb{hrwt}} TWRH $|$das Gesetz/(die) Weisung\\
3.&27.&5989.&98.&22974.&9.&5&81&10\_5\_30\_30\_6 \textcolor{red}{\textcjheb{wllhy}} JHLLW $|$r"uhmen/(sie) preisen\\
4.&28.&5990.&103.&22979.&14.&3&570&200\_300\_70 \textcolor{red}{\textcjheb{`+sr}} RSa $|$die Gesetzlosen/(den) Frevler\\
5.&29.&5991.&106.&22982.&17.&5&556&6\_300\_40\_200\_10 \textcolor{red}{\textcjheb{yrm+sw}} WSMRJ $|$die aber beobachten/und Bewahrende\\
6.&30.&5992.&111.&22987.&22.&4&611&400\_6\_200\_5 \textcolor{red}{\textcjheb{hrwt}} TWRH $|$das Gesetz/(die) Weisung\\
7.&31.&5993.&115.&22991.&26.&5&619&10\_400\_3\_200\_6 \textcolor{red}{\textcjheb{wrgty}} JTGRW $|$entr"usten sich/(sie) erregen sich\\
8.&32.&5994.&120.&22996.&31.&2&42&2\_40 \textcolor{red}{\textcjheb{mb}} BM $|$"uber sie/gegen sie\\
\end{tabular}\medskip \\
Ende des Verses 28.4\\
Verse: 800, Buchstaben: 32, 121, 22997, Totalwerte: 3179, 8702, 1624126\\
\\
Die das Gesetz verlassen, r"uhmen die Gesetzlosen; die aber das Gesetz beobachten, entr"usten sich "uber sie.\\
\newpage 
{\bf -- 28.5}\\
\medskip \\
\begin{tabular}{rrrrrrrrp{120mm}}
WV&WK&WB&ABK&ABB&ABV&AnzB&TW&Zahlencode \textcolor{red}{$\boldsymbol{Grundtext}$} Umschrift $|$"Ubersetzung(en)\\
1.&33.&5995.&122.&22998.&1.&4&361&1\_50\_300\_10 \textcolor{red}{\textcjheb{y+sn'}} ANSJ $|$Menschen\\
2.&34.&5996.&126.&23002.&5.&2&270&200\_70 \textcolor{red}{\textcjheb{`r}} Ra $|$b"ose/(von) Bosheit\\
3.&35.&5997.&128.&23004.&7.&2&31&30\_1 \textcolor{red}{\textcjheb{'l}} LA $|$nicht\\
4.&36.&5998.&130.&23006.&9.&5&78&10\_2\_10\_50\_6 \textcolor{red}{\textcjheb{wnyby}} JBJNW $|$(sie) verstehen\\
5.&37.&5999.&135.&23011.&14.&4&429&40\_300\_80\_9 \textcolor{red}{\textcjheb{.tp+sm}} MSPt $|$das Recht\\
6.&38.&6000.&139.&23015.&18.&6&458&6\_40\_2\_100\_300\_10 \textcolor{red}{\textcjheb{y+sqbmw}} WMBQSJ $|$die aber suchen/und Suchende\\
7.&39.&6001.&145.&23021.&24.&4&26&10\_5\_6\_5 \textcolor{red}{\textcjheb{hwhy}} JHWH $|$Jahwe\\
8.&40.&6002.&149.&23025.&28.&5&78&10\_2\_10\_50\_6 \textcolor{red}{\textcjheb{wnyby}} JBJNW $|$(sie) verstehen\\
9.&41.&6003.&154.&23030.&33.&2&50&20\_30 \textcolor{red}{\textcjheb{lk}} KL $|$alles\\
\end{tabular}\medskip \\
Ende des Verses 28.5\\
Verse: 801, Buchstaben: 34, 155, 23031, Totalwerte: 1781, 10483, 1625907\\
\\
B"ose Menschen verstehen das Recht nicht; die aber Jahwe suchen, verstehen alles.\\
\newpage 
{\bf -- 28.6}\\
\medskip \\
\begin{tabular}{rrrrrrrrp{120mm}}
WV&WK&WB&ABK&ABB&ABV&AnzB&TW&Zahlencode \textcolor{red}{$\boldsymbol{Grundtext}$} Umschrift $|$"Ubersetzung(en)\\
1.&42.&6004.&156.&23032.&1.&3&17&9\_6\_2 \textcolor{red}{\textcjheb{bw.t}} tWB $|$besser/gut (ist)\\
2.&43.&6005.&159.&23035.&4.&2&500&200\_300 \textcolor{red}{\textcjheb{+sr}} RS $|$(ein) Armer\\
3.&44.&6006.&161.&23037.&6.&4&61&5\_6\_30\_20 \textcolor{red}{\textcjheb{klwh}} HWLK $|$der wandelt/Gehender\\
4.&45.&6007.&165.&23041.&10.&4&448&2\_400\_40\_6 \textcolor{red}{\textcjheb{wmtb}} BTMW $|$in seiner Vollkommenheit/in seiner Lauterkeit\\
5.&46.&6008.&169.&23045.&14.&4&510&40\_70\_100\_300 \textcolor{red}{\textcjheb{+sq`m}} MaQS $|$als ein Verkehrter\\
6.&47.&6009.&173.&23049.&18.&5&274&4\_200\_20\_10\_40 \textcolor{red}{\textcjheb{mykrd}} DRKJM $|$der geht auf zwei Wegen/mit einem zwiesp"altigen Wandel\\
7.&48.&6010.&178.&23054.&23.&4&18&6\_5\_6\_1 \textcolor{red}{\textcjheb{'whw}} WHWA $|$und dabei ist/und (d)er (ist)\\
8.&49.&6011.&182.&23058.&27.&4&580&70\_300\_10\_200 \textcolor{red}{\textcjheb{ry+s`}} aSJR $|$reich/(ein) Reicher\\
\end{tabular}\medskip \\
Ende des Verses 28.6\\
Verse: 802, Buchstaben: 30, 185, 23061, Totalwerte: 2408, 12891, 1628315\\
\\
Besser ein Armer, der in seiner Vollkommenheit wandelt, als ein Verkehrter, der auf zwei Wegen geht und dabei reich ist.\\
\newpage 
{\bf -- 28.7}\\
\medskip \\
\begin{tabular}{rrrrrrrrp{120mm}}
WV&WK&WB&ABK&ABB&ABV&AnzB&TW&Zahlencode \textcolor{red}{$\boldsymbol{Grundtext}$} Umschrift $|$"Ubersetzung(en)\\
1.&50.&6012.&186.&23062.&1.&4&346&50\_6\_90\_200 \textcolor{red}{\textcjheb{r.swn}} NW"sR $|$(wer) bewahrt\\
2.&51.&6013.&190.&23066.&5.&4&611&400\_6\_200\_5 \textcolor{red}{\textcjheb{hrwt}} TWRH $|$das Gesetz/(die) Weisung\\
3.&52.&6014.&194.&23070.&9.&2&52&2\_50 \textcolor{red}{\textcjheb{nb}} BN $|$(ist) (ein) Sohn\\
4.&53.&6015.&196.&23072.&11.&4&102&40\_2\_10\_50 \textcolor{red}{\textcjheb{nybm}} MBJN $|$verst"andiger/einsichtiger\\
5.&54.&6016.&200.&23076.&15.&4&281&6\_200\_70\_5 \textcolor{red}{\textcjheb{h`rw}} WRaH $|$wer sich aber gesellt/und ein Verkehrender\\
6.&55.&6017.&204.&23080.&19.&6&123&7\_6\_30\_30\_10\_40 \textcolor{red}{\textcjheb{myllwz}} ZWLLJM $|$zu Schlemmern/mit Ausscheifenden\\
7.&56.&6018.&210.&23086.&25.&5&110&10\_20\_30\_10\_40 \textcolor{red}{\textcjheb{mylky}} JKLJM $|$((d)er) macht Schande\\
8.&57.&6019.&215.&23091.&30.&4&19&1\_2\_10\_6 \textcolor{red}{\textcjheb{wyb'}} ABJW $|$seinem Vater\\
\end{tabular}\medskip \\
Ende des Verses 28.7\\
Verse: 803, Buchstaben: 33, 218, 23094, Totalwerte: 1644, 14535, 1629959\\
\\
Ein verst"andiger Sohn bewahrt das Gesetz; wer sich aber zu Schlemmern gesellt, macht seinem Vater Schande.\\
\newpage 
{\bf -- 28.8}\\
\medskip \\
\begin{tabular}{rrrrrrrrp{120mm}}
WV&WK&WB&ABK&ABB&ABV&AnzB&TW&Zahlencode \textcolor{red}{$\boldsymbol{Grundtext}$} Umschrift $|$"Ubersetzung(en)\\
1.&58.&6020.&219.&23095.&1.&4&247&40\_200\_2\_5 \textcolor{red}{\textcjheb{hbrm}} MRBH $|$wer mehrt/ein Vermehrender\\
2.&59.&6021.&223.&23099.&5.&4&67&5\_6\_50\_6 \textcolor{red}{\textcjheb{wnwh}} HWNW $|$sein Verm"ogen\\
3.&60.&6022.&227.&23103.&9.&4&372&2\_50\_300\_20 \textcolor{red}{\textcjheb{k+snb}} BNSK $|$durch Zins\\
4.&61.&6023.&231.&23107.&13.&7&1020&6\_2\_400\_200\_2\_10\_400 \textcolor{red}{\textcjheb{tybrtbw}} WBTRBJT $|$und (durch) Wucher\\
5.&62.&6024.&238.&23114.&20.&5&144&30\_8\_6\_50\_50 \textcolor{red}{\textcjheb{nnw.hl}} LCWNN $|$f"ur den der sich erbarmt/f"ur einen sich Erbarmenden\\
6.&63.&6025.&243.&23119.&25.&4&84&4\_30\_10\_40 \textcolor{red}{\textcjheb{myld}} DLJM $|$der Armen\\
7.&64.&6026.&247.&23123.&29.&6&258&10\_100\_2\_90\_50\_6 \textcolor{red}{\textcjheb{wn.sbqy}} JQB"sNW $|$(er) sammelt ihn (=es)\\
\end{tabular}\medskip \\
Ende des Verses 28.8\\
Verse: 804, Buchstaben: 34, 252, 23128, Totalwerte: 2192, 16727, 1632151\\
\\
Wer sein Verm"ogen durch Zins und durch Wucher mehrt, sammelt es f"ur den, der sich der Armen erbarmt.\\
\newpage 
{\bf -- 28.9}\\
\medskip \\
\begin{tabular}{rrrrrrrrp{120mm}}
WV&WK&WB&ABK&ABB&ABV&AnzB&TW&Zahlencode \textcolor{red}{$\boldsymbol{Grundtext}$} Umschrift $|$"Ubersetzung(en)\\
1.&65.&6027.&253.&23129.&1.&4&310&40\_60\_10\_200 \textcolor{red}{\textcjheb{rysm}} MsJR $|$wer abwendet/ein Abwendender\\
2.&66.&6028.&257.&23133.&5.&4&64&1\_7\_50\_6 \textcolor{red}{\textcjheb{wnz'}} AZNW $|$sein Ohr\\
3.&67.&6029.&261.&23137.&9.&4&450&40\_300\_40\_70 \textcolor{red}{\textcjheb{`m+sm}} MSMa $|$vom H"oren\\
4.&68.&6030.&265.&23141.&13.&4&611&400\_6\_200\_5 \textcolor{red}{\textcjheb{hrwt}} TWRH $|$des Gesetzes/(auf) Weisung\\
5.&69.&6031.&269.&23145.&17.&2&43&3\_40 \textcolor{red}{\textcjheb{mg}} GM $|$selbst/auch\\
6.&70.&6032.&271.&23147.&19.&5&916&400\_80\_30\_400\_6 \textcolor{red}{\textcjheb{wtlpt}} TPLTW $|$sein Gebet\\
7.&71.&6033.&276.&23152.&24.&5&483&400\_6\_70\_2\_5 \textcolor{red}{\textcjheb{hb`wt}} TWaBH $|$(ist) (ein) Gr"auel\\
\end{tabular}\medskip \\
Ende des Verses 28.9\\
Verse: 805, Buchstaben: 28, 280, 23156, Totalwerte: 2877, 19604, 1635028\\
\\
Wer sein Ohr abwendet vom H"oren des Gesetzes: selbst sein Gebet ist ein Greuel.\\
\newpage 
{\bf -- 28.10}\\
\medskip \\
\begin{tabular}{rrrrrrrrp{120mm}}
WV&WK&WB&ABK&ABB&ABV&AnzB&TW&Zahlencode \textcolor{red}{$\boldsymbol{Grundtext}$} Umschrift $|$"Ubersetzung(en)\\
1.&72.&6034.&281.&23157.&1.&4&348&40\_300\_3\_5 \textcolor{red}{\textcjheb{hg+sm}} MSGH $|$wer irref"uhrt/(ein) Irref"uhrender\\
2.&73.&6035.&285.&23161.&5.&5&560&10\_300\_200\_10\_40 \textcolor{red}{\textcjheb{myr+sy}} JSRJM $|$Aufrichtige/Gerade\\
3.&74.&6036.&290.&23166.&10.&4&226&2\_4\_200\_20 \textcolor{red}{\textcjheb{krdb}} BDRK $|$auf (einen) Weg\\
4.&75.&6037.&294.&23170.&14.&2&270&200\_70 \textcolor{red}{\textcjheb{`r}} Ra $|$b"osen/schlechten\\
5.&76.&6038.&296.&23172.&16.&6&722&2\_300\_8\_6\_400\_6 \textcolor{red}{\textcjheb{wtw.h+sb}} BSCWTW $|$in seine Grube\\
6.&77.&6039.&302.&23178.&22.&3&12&5\_6\_1 \textcolor{red}{\textcjheb{'wh}} HWA $|$(er) (selbst)\\
7.&78.&6040.&305.&23181.&25.&4&126&10\_80\_6\_30 \textcolor{red}{\textcjheb{lwpy}} JPWL $|$wird fallen/(er) f"allt\\
8.&79.&6041.&309.&23185.&29.&7&546&6\_400\_40\_10\_40\_10\_40 \textcolor{red}{\textcjheb{mymymtw}} WTMJMJM $|$aber die Vollkommenen/und die Lauteren\\
9.&80.&6042.&316.&23192.&36.&5&104&10\_50\_8\_30\_6 \textcolor{red}{\textcjheb{wl.hny}} JNCLW $|$(sie) (werden) erben\\
10.&81.&6043.&321.&23197.&41.&3&17&9\_6\_2 \textcolor{red}{\textcjheb{bw.t}} tWB $|$Gutes\\
\end{tabular}\medskip \\
Ende des Verses 28.10\\
Verse: 806, Buchstaben: 43, 323, 23199, Totalwerte: 2931, 22535, 1637959\\
\\
Wer Aufrichtige irref"uhrt auf b"osen Weg, wird selbst in seine Grube fallen; aber die Vollkommenen werden Gutes erben.\\
\newpage 
{\bf -- 28.11}\\
\medskip \\
\begin{tabular}{rrrrrrrrp{120mm}}
WV&WK&WB&ABK&ABB&ABV&AnzB&TW&Zahlencode \textcolor{red}{$\boldsymbol{Grundtext}$} Umschrift $|$"Ubersetzung(en)\\
1.&82.&6044.&324.&23200.&1.&3&68&8\_20\_40 \textcolor{red}{\textcjheb{mk.h}} CKM $|$weise (ist)\\
2.&83.&6045.&327.&23203.&4.&6&148&2\_70\_10\_50\_10\_6 \textcolor{red}{\textcjheb{wyny`b}} BaJNJW $|$in seinen Augen\\
3.&84.&6046.&333.&23209.&10.&3&311&1\_10\_300 \textcolor{red}{\textcjheb{+sy'}} AJS $|$(ein) Mann\\
4.&85.&6047.&336.&23212.&13.&4&580&70\_300\_10\_200 \textcolor{red}{\textcjheb{ry+s`}} aSJR $|$reicher\\
5.&86.&6048.&340.&23216.&17.&3&40&6\_4\_30 \textcolor{red}{\textcjheb{ldw}} WDL $|$aber ein Armer/und ein Armer\\
6.&87.&6049.&343.&23219.&20.&4&102&40\_2\_10\_50 \textcolor{red}{\textcjheb{nybm}} MBJN $|$verst"andiger\\
7.&88.&6050.&347.&23223.&24.&6&374&10\_8\_100\_200\_50\_6 \textcolor{red}{\textcjheb{wnrq.hy}} JCQRNW $|$((d)er) durchschaut ihn\\
\end{tabular}\medskip \\
Ende des Verses 28.11\\
Verse: 807, Buchstaben: 29, 352, 23228, Totalwerte: 1623, 24158, 1639582\\
\\
Ein reicher Mann ist weise in seinen Augen, aber ein verst"andiger Armer durchschaut ihn.\\
\newpage 
{\bf -- 28.12}\\
\medskip \\
\begin{tabular}{rrrrrrrrp{120mm}}
WV&WK&WB&ABK&ABB&ABV&AnzB&TW&Zahlencode \textcolor{red}{$\boldsymbol{Grundtext}$} Umschrift $|$"Ubersetzung(en)\\
1.&89.&6051.&353.&23229.&1.&4&192&2\_70\_30\_90 \textcolor{red}{\textcjheb{.sl`b}} BaL"s $|$wenn frohlocken/beim Frohlocken\\
2.&90.&6052.&357.&23233.&5.&6&254&90\_4\_10\_100\_10\_40 \textcolor{red}{\textcjheb{myqyd.s}} "sDJQJM $|$die Gerechten/der Gerechten\\
3.&91.&6053.&363.&23239.&11.&3&207&200\_2\_5 \textcolor{red}{\textcjheb{hbr}} RBH $|$ist gro"s/(ist) reichlich(e)\\
4.&92.&6054.&366.&23242.&14.&5&1081&400\_80\_1\_200\_400 \textcolor{red}{\textcjheb{tr'pt}} TPART $|$die Pracht/Ehre\\
5.&93.&6055.&371.&23247.&19.&5&154&6\_2\_100\_6\_40 \textcolor{red}{\textcjheb{mwqbw}} WBQWM $|$wenn aber emporkommen/und beim Aufstehen\\
6.&94.&6056.&376.&23252.&24.&5&620&200\_300\_70\_10\_40 \textcolor{red}{\textcjheb{my`+sr}} RSaJM $|$die Gesetzlosen/(der) Frevler\\
7.&95.&6057.&381.&23257.&29.&4&398&10\_8\_80\_300 \textcolor{red}{\textcjheb{+sp.hy}} JCPS $|$verstecken sich/er (=es) l"asst sich suchen\\
8.&96.&6058.&385.&23261.&33.&3&45&1\_4\_40 \textcolor{red}{\textcjheb{md'}} ADM $|$die Menschen/(ein) Mensch\\
\end{tabular}\medskip \\
Ende des Verses 28.12\\
Verse: 808, Buchstaben: 35, 387, 23263, Totalwerte: 2951, 27109, 1642533\\
\\
Wenn die Gerechten frohlocken, ist die Pracht gro"s; wenn aber die Gesetzlosen emporkommen, verstecken sich die Menschen.\\
\newpage 
{\bf -- 28.13}\\
\medskip \\
\begin{tabular}{rrrrrrrrp{120mm}}
WV&WK&WB&ABK&ABB&ABV&AnzB&TW&Zahlencode \textcolor{red}{$\boldsymbol{Grundtext}$} Umschrift $|$"Ubersetzung(en)\\
1.&97.&6059.&388.&23264.&1.&4&125&40\_20\_60\_5 \textcolor{red}{\textcjheb{hskm}} MKsH $|$wer verbirgt/Bedeckender\\
2.&98.&6060.&392.&23268.&5.&5&466&80\_300\_70\_10\_6 \textcolor{red}{\textcjheb{wy`+sp}} PSaJW $|$seine "Ubertretungen/seine Missetaten\\
3.&99.&6061.&397.&23273.&10.&2&31&30\_1 \textcolor{red}{\textcjheb{'l}} LA $|$nicht\\
4.&100.&6062.&399.&23275.&12.&5&148&10\_90\_30\_10\_8 \textcolor{red}{\textcjheb{.hyl.sy}} J"sLJC $|$wird haben Gelingen/(er) hat Gedeihen\\
5.&101.&6063.&404.&23280.&17.&5&61&6\_40\_6\_4\_5 \textcolor{red}{\textcjheb{hdwmw}} WMWDH $|$wer sie aber bekennt/und ein Bekennender\\
6.&102.&6064.&409.&23285.&22.&4&85&6\_70\_7\_2 \textcolor{red}{\textcjheb{bz`w}} WaZB $|$und l"asst/und Ablassender\\
7.&103.&6065.&413.&23289.&26.&4&258&10\_200\_8\_40 \textcolor{red}{\textcjheb{m.hry}} JRCM $|$wird Barmherzigkeit erlangen/(d)er findet Erbarmen\\
\end{tabular}\medskip \\
Ende des Verses 28.13\\
Verse: 809, Buchstaben: 29, 416, 23292, Totalwerte: 1174, 28283, 1643707\\
\\
Wer seine "Ubertretungen verbirgt, wird kein Gelingen haben; wer sie aber bekennt und l"a"st, wird Barmherzigkeit erlangen.\\
\newpage 
{\bf -- 28.14}\\
\medskip \\
\begin{tabular}{rrrrrrrrp{120mm}}
WV&WK&WB&ABK&ABB&ABV&AnzB&TW&Zahlencode \textcolor{red}{$\boldsymbol{Grundtext}$} Umschrift $|$"Ubersetzung(en)\\
1.&104.&6066.&417.&23293.&1.&4&511&1\_300\_200\_10 \textcolor{red}{\textcjheb{yr+s'}} ASRJ $|$gl"uckselig/Seligkeiten\\
2.&105.&6067.&421.&23297.&5.&3&45&1\_4\_40 \textcolor{red}{\textcjheb{md'}} ADM $|$der Mensch/(dem) Menschen\\
3.&106.&6068.&424.&23300.&8.&4&132&40\_80\_8\_4 \textcolor{red}{\textcjheb{d.hpm}} MPCD $|$der sich f"urchtet/der f"urchtend (ist)\\
4.&107.&6069.&428.&23304.&12.&4&454&400\_40\_10\_4 \textcolor{red}{\textcjheb{dymt}} TMJD $|$best"andig/stets\\
5.&108.&6070.&432.&23308.&16.&5&451&6\_40\_100\_300\_5 \textcolor{red}{\textcjheb{h+sqmw}} WMQSH $|$wer aber verh"artet/und ein Verh"artender\\
6.&109.&6071.&437.&23313.&21.&3&38&30\_2\_6 \textcolor{red}{\textcjheb{wbl}} LBW $|$sein Herz\\
7.&110.&6072.&440.&23316.&24.&4&126&10\_80\_6\_30 \textcolor{red}{\textcjheb{lwpy}} JPWL $|$wird fallen/(d)er st"urzt\\
8.&111.&6073.&444.&23320.&28.&4&277&2\_200\_70\_5 \textcolor{red}{\textcjheb{h`rb}} BRaH $|$ins Ungl"uck\\
\end{tabular}\medskip \\
Ende des Verses 28.14\\
Verse: 810, Buchstaben: 31, 447, 23323, Totalwerte: 2034, 30317, 1645741\\
\\
Gl"uckselig der Mensch, der sich best"andig f"urchtet; wer aber sein Herz verh"artet, wird ins Ungl"uck fallen.\\
\newpage 
{\bf -- 28.15}\\
\medskip \\
\begin{tabular}{rrrrrrrrp{120mm}}
WV&WK&WB&ABK&ABB&ABV&AnzB&TW&Zahlencode \textcolor{red}{$\boldsymbol{Grundtext}$} Umschrift $|$"Ubersetzung(en)\\
1.&112.&6074.&448.&23324.&1.&3&211&1\_200\_10 \textcolor{red}{\textcjheb{yr'}} ARJ $|$(wie) (ein) L"owe\\
2.&113.&6075.&451.&23327.&4.&3&95&50\_5\_40 \textcolor{red}{\textcjheb{mhn}} NHM $|$br"ullender/knurrender\\
3.&114.&6076.&454.&23330.&7.&3&12&6\_4\_2 \textcolor{red}{\textcjheb{bdw}} WDB $|$und (ein) B"ar\\
4.&115.&6077.&457.&23333.&10.&4&506&300\_6\_100\_100 \textcolor{red}{\textcjheb{qqw+s}} SWQQ $|$gieriger/"uberfallender\\
5.&116.&6078.&461.&23337.&14.&3&370&40\_300\_30 \textcolor{red}{\textcjheb{l+sm}} MSL $|$so ist ein Herrscher/(ist ein) Herrschender\\
6.&117.&6079.&464.&23340.&17.&3&570&200\_300\_70 \textcolor{red}{\textcjheb{`+sr}} RSa $|$gesetzloser/ruchloser\\
7.&118.&6080.&467.&23343.&20.&2&100&70\_30 \textcolor{red}{\textcjheb{l`}} aL $|$"uber\\
8.&119.&6081.&469.&23345.&22.&2&110&70\_40 \textcolor{red}{\textcjheb{m`}} aM $|$(ein) Volk\\
9.&120.&6082.&471.&23347.&24.&2&34&4\_30 \textcolor{red}{\textcjheb{ld}} DL $|$armes\\
\end{tabular}\medskip \\
Ende des Verses 28.15\\
Verse: 811, Buchstaben: 25, 472, 23348, Totalwerte: 2008, 32325, 1647749\\
\\
Ein br"ullender L"owe und ein gieriger B"ar: so ist ein gesetzloser Herrscher "uber ein armes Volk.\\
\newpage 
{\bf -- 28.16}\\
\medskip \\
\begin{tabular}{rrrrrrrrp{120mm}}
WV&WK&WB&ABK&ABB&ABV&AnzB&TW&Zahlencode \textcolor{red}{$\boldsymbol{Grundtext}$} Umschrift $|$"Ubersetzung(en)\\
1.&121.&6083.&473.&23349.&1.&4&67&50\_3\_10\_4 \textcolor{red}{\textcjheb{dygn}} NGJD $|$du F"urst/(einem) F"urst(en)\\
2.&122.&6084.&477.&23353.&5.&3&268&8\_60\_200 \textcolor{red}{\textcjheb{rs.h}} CsR $|$ohne/(dem) er (=es) mangelt (an)\\
3.&123.&6085.&480.&23356.&8.&6&864&400\_2\_6\_50\_6\_400 \textcolor{red}{\textcjheb{twnwbt}} TBWNWT $|$Verstand/Einsichten\\
4.&124.&6086.&486.&23362.&14.&3&208&6\_200\_2 \textcolor{red}{\textcjheb{brw}} WRB $|$und reich/und (dem) reichlich\\
5.&125.&6087.&489.&23365.&17.&6&916&40\_70\_300\_100\_6\_400 \textcolor{red}{\textcjheb{twq+s`m}} MaSQWT $|$an Erpressungen/(widerfahren) Erpressungen\\
6.&126.&6088.&495.&23371.&23.&4&361&300\_50\_1\_10 \textcolor{red}{\textcjheb{y'n+s}} SNAJ $|$wer hasst\\
7.&127.&6089.&499.&23375.&27.&3&162&2\_90\_70 \textcolor{red}{\textcjheb{`.sb}} B"sa $|$(unrechtm"a"sigen) Gewinn\\
8.&128.&6090.&502.&23378.&30.&5&241&10\_1\_200\_10\_20 \textcolor{red}{\textcjheb{kyr'y}} JARJK $|$wird verl"angern/(d)er macht lang\\
9.&129.&6091.&507.&23383.&35.&4&100&10\_40\_10\_40 \textcolor{red}{\textcjheb{mymy}} JMJM $|$seine Tage/die Tage\\
\end{tabular}\medskip \\
Ende des Verses 28.16\\
Verse: 812, Buchstaben: 38, 510, 23386, Totalwerte: 3187, 35512, 1650936\\
\\
Du F"urst, ohne Verstand und reich an Erpressungen! Wer unrechtm"a"sigen Gewinn ha"st, wird seine Tage verl"angern.\\
\newpage 
{\bf -- 28.17}\\
\medskip \\
\begin{tabular}{rrrrrrrrp{120mm}}
WV&WK&WB&ABK&ABB&ABV&AnzB&TW&Zahlencode \textcolor{red}{$\boldsymbol{Grundtext}$} Umschrift $|$"Ubersetzung(en)\\
1.&130.&6092.&511.&23387.&1.&3&45&1\_4\_40 \textcolor{red}{\textcjheb{md'}} ADM $|$(ein) Mensch\\
2.&131.&6093.&514.&23390.&4.&3&470&70\_300\_100 \textcolor{red}{\textcjheb{q+s`}} aSQ $|$belastet/gedr"uckt(er)\\
3.&132.&6094.&517.&23393.&7.&3&46&2\_4\_40 \textcolor{red}{\textcjheb{mdb}} BDM $|$mit dem Blut/von dem Blut\\
4.&133.&6095.&520.&23396.&10.&3&430&50\_80\_300 \textcolor{red}{\textcjheb{+spn}} NPS $|$(einer) Seele\\
5.&134.&6096.&523.&23399.&13.&2&74&70\_4 \textcolor{red}{\textcjheb{d`}} aD $|$bis zur\\
6.&135.&6097.&525.&23401.&15.&3&208&2\_6\_200 \textcolor{red}{\textcjheb{rwb}} BWR $|$Grube\\
7.&136.&6098.&528.&23404.&18.&4&126&10\_50\_6\_60 \textcolor{red}{\textcjheb{swny}} JNWs $|$(er) flieht\\
8.&137.&6099.&532.&23408.&22.&2&31&1\_30 \textcolor{red}{\textcjheb{l'}} AL $|$nicht\\
9.&138.&6100.&534.&23410.&24.&5&476&10\_400\_40\_20\_6 \textcolor{red}{\textcjheb{wkmty}} JTMKW $|$man unterst"utze/sie sollen festhalten\\
10.&139.&6101.&539.&23415.&29.&2&8&2\_6 \textcolor{red}{\textcjheb{wb}} BW $|$ihn\\
\end{tabular}\medskip \\
Ende des Verses 28.17\\
Verse: 813, Buchstaben: 30, 540, 23416, Totalwerte: 1914, 37426, 1652850\\
\\
Ein Mensch, belastet mit dem Blute einer Seele, flieht bis zur Grube: man unterst"utze ihn nicht!\\
\newpage 
{\bf -- 28.18}\\
\medskip \\
\begin{tabular}{rrrrrrrrp{120mm}}
WV&WK&WB&ABK&ABB&ABV&AnzB&TW&Zahlencode \textcolor{red}{$\boldsymbol{Grundtext}$} Umschrift $|$"Ubersetzung(en)\\
1.&140.&6102.&541.&23417.&1.&4&61&5\_6\_30\_20 \textcolor{red}{\textcjheb{klwh}} HWLK $|$wer wandelt/(ein) Gehender\\
2.&141.&6103.&545.&23421.&5.&4&490&400\_40\_10\_40 \textcolor{red}{\textcjheb{mymt}} TMJM $|$vollkommen/unstr"aflich\\
3.&142.&6104.&549.&23425.&9.&4&386&10\_6\_300\_70 \textcolor{red}{\textcjheb{`+swy}} JWSa $|$wird gerettet werden/dem wird geholfen\\
4.&143.&6105.&553.&23429.&13.&5&526&6\_50\_70\_100\_300 \textcolor{red}{\textcjheb{+sq`nw}} WNaQS $|$wer aber verkehrt/und ein Verkehrter\\
5.&144.&6106.&558.&23434.&18.&5&274&4\_200\_20\_10\_40 \textcolor{red}{\textcjheb{mykrd}} DRKJM $|$auf zwei Wegen/zweier Wege\\
6.&145.&6107.&563.&23439.&23.&4&126&10\_80\_6\_30 \textcolor{red}{\textcjheb{lwpy}} JPWL $|$wird fallen/(d)er f"allt\\
7.&146.&6108.&567.&23443.&27.&4&411&2\_1\_8\_400 \textcolor{red}{\textcjheb{t.h'b}} BACT $|$auf einmal/in eine (davon)\\
\end{tabular}\medskip \\
Ende des Verses 28.18\\
Verse: 814, Buchstaben: 30, 570, 23446, Totalwerte: 2274, 39700, 1655124\\
\\
Wer vollkommen wandelt, wird gerettet werden; wer aber verkehrt auf zwei Wegen geht, wird auf einmal fallen.\\
\newpage 
{\bf -- 28.19}\\
\medskip \\
\begin{tabular}{rrrrrrrrp{120mm}}
WV&WK&WB&ABK&ABB&ABV&AnzB&TW&Zahlencode \textcolor{red}{$\boldsymbol{Grundtext}$} Umschrift $|$"Ubersetzung(en)\\
1.&147.&6109.&571.&23447.&1.&3&76&70\_2\_4 \textcolor{red}{\textcjheb{db`}} aBD $|$wer bebaut/(ein) Bestellender\\
2.&148.&6110.&574.&23450.&4.&5&451&1\_4\_40\_400\_6 \textcolor{red}{\textcjheb{wtmd'}} ADMTW $|$sein (Acker)Land\\
3.&149.&6111.&579.&23455.&9.&4&382&10\_300\_2\_70 \textcolor{red}{\textcjheb{`b+sy}} JSBa $|$wird ges"attigt werden/(d)er wird satt\\
4.&150.&6112.&583.&23459.&13.&3&78&30\_8\_40 \textcolor{red}{\textcjheb{m.hl}} LCM $|$mit Brot/(an) Brot\\
5.&151.&6113.&586.&23462.&16.&5&330&6\_40\_200\_4\_80 \textcolor{red}{\textcjheb{pdrmw}} WMRDP $|$wer aber nachjagt/und ein Nachjagender\\
6.&152.&6114.&591.&23467.&21.&4&350&200\_100\_10\_40 \textcolor{red}{\textcjheb{myqr}} RQJM $|$nichtigen Dingen/leeren (Dingen)\\
7.&153.&6115.&595.&23471.&25.&4&382&10\_300\_2\_70 \textcolor{red}{\textcjheb{`b+sy}} JSBa $|$wird ges"attigt werden/(d)er wird satt\\
8.&154.&6116.&599.&23475.&29.&3&510&200\_10\_300 \textcolor{red}{\textcjheb{+syr}} RJS $|$mit Armut/(an) Armut\\
\end{tabular}\medskip \\
Ende des Verses 28.19\\
Verse: 815, Buchstaben: 31, 601, 23477, Totalwerte: 2559, 42259, 1657683\\
\\
Wer sein Land bebaut, wird mit Brot ges"attigt werden; wer aber nichtigen Dingen nachjagt, wird mit Armut ges"attigt werden.\\
\newpage 
{\bf -- 28.20}\\
\medskip \\
\begin{tabular}{rrrrrrrrp{120mm}}
WV&WK&WB&ABK&ABB&ABV&AnzB&TW&Zahlencode \textcolor{red}{$\boldsymbol{Grundtext}$} Umschrift $|$"Ubersetzung(en)\\
1.&155.&6117.&602.&23478.&1.&3&311&1\_10\_300 \textcolor{red}{\textcjheb{+sy'}} AJS $|$(ein) Mann\\
2.&156.&6118.&605.&23481.&4.&6&503&1\_40\_6\_50\_6\_400 \textcolor{red}{\textcjheb{twnwm'}} AMWNWT $|$treuer/der Zuverl"assigkeit\\
3.&157.&6119.&611.&23487.&10.&2&202&200\_2 \textcolor{red}{\textcjheb{br}} RB $|$hat viel/(ist) viel\\
4.&158.&6120.&613.&23489.&12.&5&628&2\_200\_20\_6\_400 \textcolor{red}{\textcjheb{twkrb}} BRKWT $|$Seg(nung)en\\
5.&159.&6121.&618.&23494.&17.&3&97&6\_1\_90 \textcolor{red}{\textcjheb{.s'w}} WA"s $|$wer aber hastig ist/und ein Eilender\\
6.&160.&6122.&621.&23497.&20.&6&615&30\_5\_70\_300\_10\_200 \textcolor{red}{\textcjheb{ry+s`hl}} LHaSJR $|$reich zu werden/um bereichern sich\\
7.&161.&6123.&627.&23503.&26.&2&31&30\_1 \textcolor{red}{\textcjheb{'l}} LA $|$nicht\\
8.&162.&6124.&629.&23505.&28.&4&165&10\_50\_100\_5 \textcolor{red}{\textcjheb{hqny}} JNQH $|$wird schuldlos sein/(er) bleibt ungestraft\\
\end{tabular}\medskip \\
Ende des Verses 28.20\\
Verse: 816, Buchstaben: 31, 632, 23508, Totalwerte: 2552, 44811, 1660235\\
\\
Ein treuer Mann hat viel Segen; wer aber hastig ist, reich zu werden, wird nicht schuldlos sein.\\
\newpage 
{\bf -- 28.21}\\
\medskip \\
\begin{tabular}{rrrrrrrrp{120mm}}
WV&WK&WB&ABK&ABB&ABV&AnzB&TW&Zahlencode \textcolor{red}{$\boldsymbol{Grundtext}$} Umschrift $|$"Ubersetzung(en)\\
1.&163.&6125.&633.&23509.&1.&3&225&5\_20\_200 \textcolor{red}{\textcjheb{rkh}} HKR $|$ansehen/erkennen\\
2.&164.&6126.&636.&23512.&4.&4&180&80\_50\_10\_40 \textcolor{red}{\textcjheb{mynp}} PNJM $|$die Person/Gesichter\\
3.&165.&6127.&640.&23516.&8.&2&31&30\_1 \textcolor{red}{\textcjheb{'l}} LA $|$nicht\\
4.&166.&6128.&642.&23518.&10.&3&17&9\_6\_2 \textcolor{red}{\textcjheb{bw.t}} tWB $|$ist gut\\
5.&167.&6129.&645.&23521.&13.&3&106&6\_70\_30 \textcolor{red}{\textcjheb{l`w}} WaL $|$und um\\
6.&168.&6130.&648.&23524.&16.&2&480&80\_400 \textcolor{red}{\textcjheb{tp}} PT $|$(einen) Bissen\\
7.&169.&6131.&650.&23526.&18.&3&78&30\_8\_40 \textcolor{red}{\textcjheb{m.hl}} LCM $|$Brot\\
8.&170.&6132.&653.&23529.&21.&4&460&10\_80\_300\_70 \textcolor{red}{\textcjheb{`+spy}} JPSa $|$kann "ubertreten/er (=es) kann sich vergehen\\
9.&171.&6133.&657.&23533.&25.&3&205&3\_2\_200 \textcolor{red}{\textcjheb{rbg}} GBR $|$ein Mann\\
\end{tabular}\medskip \\
Ende des Verses 28.21\\
Verse: 817, Buchstaben: 27, 659, 23535, Totalwerte: 1782, 46593, 1662017\\
\\
Die Person ansehen ist nicht gut, und um einen Bissen Brot kann ein Mann "ubertreten.\\
\newpage 
{\bf -- 28.22}\\
\medskip \\
\begin{tabular}{rrrrrrrrp{120mm}}
WV&WK&WB&ABK&ABB&ABV&AnzB&TW&Zahlencode \textcolor{red}{$\boldsymbol{Grundtext}$} Umschrift $|$"Ubersetzung(en)\\
1.&172.&6134.&660.&23536.&1.&4&87&50\_2\_5\_30 \textcolor{red}{\textcjheb{lhbn}} NBHL $|$(es) hascht/Hastender (ist)\\
2.&173.&6135.&664.&23540.&5.&4&91&30\_5\_6\_50 \textcolor{red}{\textcjheb{nwhl}} LHWN $|$nach Reichtum\\
3.&174.&6136.&668.&23544.&9.&3&311&1\_10\_300 \textcolor{red}{\textcjheb{+sy'}} AJS $|$(ein) Mann\\
4.&175.&6137.&671.&23547.&12.&2&270&200\_70 \textcolor{red}{\textcjheb{`r}} Ra $|$scheel-/(mit) b"osem\\
5.&176.&6138.&673.&23549.&14.&3&130&70\_10\_50 \textcolor{red}{\textcjheb{ny`}} aJN $|$sehender/Auge\\
6.&177.&6139.&676.&23552.&17.&3&37&6\_30\_1 \textcolor{red}{\textcjheb{'lw}} WLA $|$und nicht\\
7.&178.&6140.&679.&23555.&20.&3&84&10\_4\_70 \textcolor{red}{\textcjheb{`dy}} JDa $|$er (er)kennt\\
8.&179.&6141.&682.&23558.&23.&2&30&20\_10 \textcolor{red}{\textcjheb{yk}} KJ $|$dass\\
9.&180.&6142.&684.&23560.&25.&3&268&8\_60\_200 \textcolor{red}{\textcjheb{rs.h}} CsR $|$Mangel\\
10.&181.&6143.&687.&23563.&28.&5&69&10\_2\_1\_50\_6 \textcolor{red}{\textcjheb{wn'by}} JBANW $|$"uber ihn kommen wird/(er) wird treffen ihn\\
\end{tabular}\medskip \\
Ende des Verses 28.22\\
Verse: 818, Buchstaben: 32, 691, 23567, Totalwerte: 1377, 47970, 1663394\\
\\
Ein scheelsehender Mann hascht nach Reichtum, und er erkennt nicht, da"s Mangel "uber ihn kommen wird.\\
\newpage 
{\bf -- 28.23}\\
\medskip \\
\begin{tabular}{rrrrrrrrp{120mm}}
WV&WK&WB&ABK&ABB&ABV&AnzB&TW&Zahlencode \textcolor{red}{$\boldsymbol{Grundtext}$} Umschrift $|$"Ubersetzung(en)\\
1.&182.&6144.&692.&23568.&1.&5&84&40\_6\_20\_10\_8 \textcolor{red}{\textcjheb{.hykwm}} MWKJC $|$wer straft/(ein) Zurechtweisender\\
2.&183.&6145.&697.&23573.&6.&3&45&1\_4\_40 \textcolor{red}{\textcjheb{md'}} ADM $|$(einen) Menschen\\
3.&184.&6146.&700.&23576.&9.&4&219&1\_8\_200\_10 \textcolor{red}{\textcjheb{yr.h'}} ACRJ $|$hernach\\
4.&185.&6147.&704.&23580.&13.&2&58&8\_50 \textcolor{red}{\textcjheb{n.h}} CN $|$mehr Gunst\\
5.&186.&6148.&706.&23582.&15.&4&141&10\_40\_90\_1 \textcolor{red}{\textcjheb{'.smy}} JM"sA $|$wird finden/er findet\\
6.&187.&6149.&710.&23586.&19.&6&228&40\_40\_8\_30\_10\_100 \textcolor{red}{\textcjheb{qyl.hmm}} MMCLJQ $|$als wer schmeichelt/mehr als ein Schmeichelnder\\
7.&188.&6150.&716.&23592.&25.&4&386&30\_300\_6\_50 \textcolor{red}{\textcjheb{nw+sl}} LSWN $|$(mit) der Zunge\\
\end{tabular}\medskip \\
Ende des Verses 28.23\\
Verse: 819, Buchstaben: 28, 719, 23595, Totalwerte: 1161, 49131, 1664555\\
\\
Wer einen Menschen straft, wird hernach mehr Gunst finden, als wer mit der Zunge schmeichelt.\\
\newpage 
{\bf -- 28.24}\\
\medskip \\
\begin{tabular}{rrrrrrrrp{120mm}}
WV&WK&WB&ABK&ABB&ABV&AnzB&TW&Zahlencode \textcolor{red}{$\boldsymbol{Grundtext}$} Umschrift $|$"Ubersetzung(en)\\
1.&189.&6151.&720.&23596.&1.&4&46&3\_6\_7\_30 \textcolor{red}{\textcjheb{lzwg}} GWZL $|$wer beraubt\\
2.&190.&6152.&724.&23600.&5.&4&19&1\_2\_10\_6 \textcolor{red}{\textcjheb{wyb'}} ABJW $|$seinen Vater\\
3.&191.&6153.&728.&23604.&9.&4&53&6\_1\_40\_6 \textcolor{red}{\textcjheb{wm'w}} WAMW $|$und seine Mutter\\
4.&192.&6154.&732.&23608.&13.&4&247&6\_1\_40\_200 \textcolor{red}{\textcjheb{rm'w}} WAMR $|$und spricht/und sagt\\
5.&193.&6155.&736.&23612.&17.&3&61&1\_10\_50 \textcolor{red}{\textcjheb{ny'}} AJN $|$nicht ist es\\
6.&194.&6156.&739.&23615.&20.&3&450&80\_300\_70 \textcolor{red}{\textcjheb{`+sp}} PSa $|$ein Frevel/S"unde\\
7.&195.&6157.&742.&23618.&23.&3&210&8\_2\_200 \textcolor{red}{\textcjheb{rb.h}} CBR $|$ein Genosse\\
8.&196.&6158.&745.&23621.&26.&3&12&5\_6\_1 \textcolor{red}{\textcjheb{'wh}} HWA $|$(ist) (d)er\\
9.&197.&6159.&748.&23624.&29.&4&341&30\_1\_10\_300 \textcolor{red}{\textcjheb{+sy'l}} LAJS $|$des/eines Mannes\\
10.&198.&6160.&752.&23628.&33.&5&758&40\_300\_8\_10\_400 \textcolor{red}{\textcjheb{ty.h+sm}} MSCJT $|$Verderbers/verderbend(en)\\
\end{tabular}\medskip \\
Ende des Verses 28.24\\
Verse: 820, Buchstaben: 37, 756, 23632, Totalwerte: 2197, 51328, 1666752\\
\\
Wer seinen Vater und seine Mutter beraubt und spricht: Kein Frevel ist es! der ist ein Genosse des Verderbers.\\
\newpage 
{\bf -- 28.25}\\
\medskip \\
\begin{tabular}{rrrrrrrrp{120mm}}
WV&WK&WB&ABK&ABB&ABV&AnzB&TW&Zahlencode \textcolor{red}{$\boldsymbol{Grundtext}$} Umschrift $|$"Ubersetzung(en)\\
1.&199.&6161.&757.&23633.&1.&3&210&200\_8\_2 \textcolor{red}{\textcjheb{b.hr}} RCB $|$der Hab-/ein Weiter\\
2.&200.&6162.&760.&23636.&4.&3&430&50\_80\_300 \textcolor{red}{\textcjheb{+spn}} NPS $|$gierige/von Verlangen\\
3.&201.&6163.&763.&23639.&7.&4&218&10\_3\_200\_5 \textcolor{red}{\textcjheb{hrgy}} JGRH $|$(er) erregt\\
4.&202.&6164.&767.&23643.&11.&4&100&40\_4\_6\_50 \textcolor{red}{\textcjheb{nwdm}} MDWN $|$Zank\\
5.&203.&6165.&771.&23647.&15.&5&31&6\_2\_6\_9\_8 \textcolor{red}{\textcjheb{.h.twbw}} WBWtC $|$wer aber vertraut/und ein Vertrauender\\
6.&204.&6166.&776.&23652.&20.&2&100&70\_30 \textcolor{red}{\textcjheb{l`}} aL $|$auf\\
7.&205.&6167.&778.&23654.&22.&4&26&10\_5\_6\_5 \textcolor{red}{\textcjheb{hwhy}} JHWH $|$Jahwe\\
8.&206.&6168.&782.&23658.&26.&4&364&10\_4\_300\_50 \textcolor{red}{\textcjheb{n+sdy}} JDSN $|$wird reichlich ges"attigt/(er) wird gelabt\\
\end{tabular}\medskip \\
Ende des Verses 28.25\\
Verse: 821, Buchstaben: 29, 785, 23661, Totalwerte: 1479, 52807, 1668231\\
\\
Der Habgierige erregt Zank; wer aber auf Jahwe vertraut, wird reichlich ges"attigt.\\
\newpage 
{\bf -- 28.26}\\
\medskip \\
\begin{tabular}{rrrrrrrrp{120mm}}
WV&WK&WB&ABK&ABB&ABV&AnzB&TW&Zahlencode \textcolor{red}{$\boldsymbol{Grundtext}$} Umschrift $|$"Ubersetzung(en)\\
1.&207.&6169.&786.&23662.&1.&4&25&2\_6\_9\_8 \textcolor{red}{\textcjheb{.h.twb}} BWtC $|$wer vertraut/(ein) Vertrauender\\
2.&208.&6170.&790.&23666.&5.&4&40&2\_30\_2\_6 \textcolor{red}{\textcjheb{wblb}} BLBW $|$auf sein Herz/in sein Herz\\
3.&209.&6171.&794.&23670.&9.&3&12&5\_6\_1 \textcolor{red}{\textcjheb{'wh}} HWA $|$(d)er (ist)\\
4.&210.&6172.&797.&23673.&12.&4&120&20\_60\_10\_30 \textcolor{red}{\textcjheb{lysk}} KsJL $|$(ein) Tor\\
5.&211.&6173.&801.&23677.&16.&5&67&6\_5\_6\_30\_20 \textcolor{red}{\textcjheb{klwhw}} WHWLK $|$wer aber wandelt/und ein Wandelnder\\
6.&212.&6174.&806.&23682.&21.&5&75&2\_8\_20\_40\_5 \textcolor{red}{\textcjheb{hmk.hb}} BCKMH $|$in Weisheit/mit Weisheit\\
7.&213.&6175.&811.&23687.&26.&3&12&5\_6\_1 \textcolor{red}{\textcjheb{'wh}} HWA $|$(d)er\\
8.&214.&6176.&814.&23690.&29.&4&89&10\_40\_30\_9 \textcolor{red}{\textcjheb{.tlmy}} JMLt $|$wird entrinnen/(er) wird gerettet\\
\end{tabular}\medskip \\
Ende des Verses 28.26\\
Verse: 822, Buchstaben: 32, 817, 23693, Totalwerte: 440, 53247, 1668671\\
\\
Wer auf sein Herz vertraut, der ist ein Tor; wer aber in Weisheit wandelt, der wird entrinnen.\\
\newpage 
{\bf -- 28.27}\\
\medskip \\
\begin{tabular}{rrrrrrrrp{120mm}}
WV&WK&WB&ABK&ABB&ABV&AnzB&TW&Zahlencode \textcolor{red}{$\boldsymbol{Grundtext}$} Umschrift $|$"Ubersetzung(en)\\
1.&215.&6177.&818.&23694.&1.&4&506&50\_6\_400\_50 \textcolor{red}{\textcjheb{ntwn}} NWTN $|$wer gibt/(ein) Gebender\\
2.&216.&6178.&822.&23698.&5.&3&530&30\_200\_300 \textcolor{red}{\textcjheb{+srl}} LRS $|$dem Armen\\
3.&217.&6179.&825.&23701.&8.&3&61&1\_10\_50 \textcolor{red}{\textcjheb{ny'}} AJN $|$wird haben keinen/hat keinen\\
4.&218.&6180.&828.&23704.&11.&5&314&40\_8\_60\_6\_200 \textcolor{red}{\textcjheb{rws.hm}} MCsWR $|$Mangel\\
5.&219.&6181.&833.&23709.&16.&6&196&6\_40\_70\_30\_10\_40 \textcolor{red}{\textcjheb{myl`mw}} WMaLJM $|$wer aber verh"ullt/und Verh"ullende\\
6.&220.&6182.&839.&23715.&22.&5&146&70\_10\_50\_10\_6 \textcolor{red}{\textcjheb{wyny`}} aJNJW $|$seine Augen\\
7.&221.&6183.&844.&23720.&27.&2&202&200\_2 \textcolor{red}{\textcjheb{br}} RB $|$wird "uberh"auft werden/(erhalten) reichlich\\
8.&222.&6184.&846.&23722.&29.&5&647&40\_1\_200\_6\_400 \textcolor{red}{\textcjheb{twr'm}} MARWT $|$(mit) Fl"uche(n)\\
\end{tabular}\medskip \\
Ende des Verses 28.27\\
Verse: 823, Buchstaben: 33, 850, 23726, Totalwerte: 2602, 55849, 1671273\\
\\
Wer dem Armen gibt, wird keinen Mangel haben; wer aber seine Augen verh"ullt, wird mit Fl"uchen "uberh"auft werden.\\
\newpage 
{\bf -- 28.28}\\
\medskip \\
\begin{tabular}{rrrrrrrrp{120mm}}
WV&WK&WB&ABK&ABB&ABV&AnzB&TW&Zahlencode \textcolor{red}{$\boldsymbol{Grundtext}$} Umschrift $|$"Ubersetzung(en)\\
1.&223.&6185.&851.&23727.&1.&4&148&2\_100\_6\_40 \textcolor{red}{\textcjheb{mwqb}} BQWM $|$wenn emporkommen/im Erheben\\
2.&224.&6186.&855.&23731.&5.&5&620&200\_300\_70\_10\_40 \textcolor{red}{\textcjheb{my`+sr}} RSaJM $|$die Gesetzlosen/Frevler\\
3.&225.&6187.&860.&23736.&10.&4&670&10\_60\_400\_200 \textcolor{red}{\textcjheb{rtsy}} JsTR $|$verbergen sich/er (=es) verbirgt sich\\
4.&226.&6188.&864.&23740.&14.&3&45&1\_4\_40 \textcolor{red}{\textcjheb{md'}} ADM $|$die Menschen/(der) Mensch\\
5.&227.&6189.&867.&23743.&17.&6&55&6\_2\_1\_2\_4\_40 \textcolor{red}{\textcjheb{mdb'bw}} WBABDM $|$und wenn sie umkommen/und bei ihrem Umkommen\\
6.&228.&6190.&873.&23749.&23.&4&218&10\_200\_2\_6 \textcolor{red}{\textcjheb{wbry}} JRBW $|$mehren sich/sie (=es) werden viel sein\\
7.&229.&6191.&877.&23753.&27.&6&254&90\_4\_10\_100\_10\_40 \textcolor{red}{\textcjheb{myqyd.s}} "sDJQJM $|$(die) Gerechten\\
\end{tabular}\medskip \\
Ende des Verses 28.28\\
Verse: 824, Buchstaben: 32, 882, 23758, Totalwerte: 2010, 57859, 1673283\\
\\
Wenn die Gesetzlosen emporkommen, verbergen sich die Menschen; und wenn sie umkommen, mehren sich die Gerechten.\\
\\
{\bf Ende des Kapitels 28}\\
\newpage 
{\bf -- 29.1}\\
\medskip \\
\begin{tabular}{rrrrrrrrp{120mm}}
WV&WK&WB&ABK&ABB&ABV&AnzB&TW&Zahlencode \textcolor{red}{$\boldsymbol{Grundtext}$} Umschrift $|$"Ubersetzung(en)\\
1.&1.&6192.&1.&23759.&1.&3&311&1\_10\_300 \textcolor{red}{\textcjheb{+sy'}} AJS $|$(ein) Mann\\
2.&2.&6193.&4.&23762.&4.&6&840&400\_6\_20\_8\_6\_400 \textcolor{red}{\textcjheb{tw.hkwt}} TWKCWT $|$der oft zurechtgewiesen/(der) Widerreden\\
3.&3.&6194.&10.&23768.&10.&4&445&40\_100\_300\_5 \textcolor{red}{\textcjheb{h+sqm}} MQSH $|$verh"artet/(ist) hart machend\\
4.&4.&6195.&14.&23772.&14.&3&350&70\_200\_80 \textcolor{red}{\textcjheb{pr`}} aRP $|$den Nacken\\
5.&5.&6196.&17.&23775.&17.&3&550&80\_400\_70 \textcolor{red}{\textcjheb{`tp}} PTa $|$pl"otzlich/augenblicklich\\
6.&6.&6197.&20.&23778.&20.&4&512&10\_300\_2\_200 \textcolor{red}{\textcjheb{rb+sy}} JSBR $|$wird zerschmettert werden/er wird zerbrochen\\
7.&7.&6198.&24.&23782.&24.&4&67&6\_1\_10\_50 \textcolor{red}{\textcjheb{ny'w}} WAJN $|$ohne/und nicht gibt es\\
8.&8.&6199.&28.&23786.&28.&4&321&40\_200\_80\_1 \textcolor{red}{\textcjheb{'prm}} MRPA $|$(eine) Heilung\\
\end{tabular}\medskip \\
Ende des Verses 29.1\\
Verse: 825, Buchstaben: 31, 31, 23789, Totalwerte: 3396, 3396, 1676679\\
\\
Ein Mann, der, oft zurechtgewiesen, den Nacken verh"artet, wird pl"otzlich zerschmettert werden ohne Heilung.\\
\newpage 
{\bf -- 29.2}\\
\medskip \\
\begin{tabular}{rrrrrrrrp{120mm}}
WV&WK&WB&ABK&ABB&ABV&AnzB&TW&Zahlencode \textcolor{red}{$\boldsymbol{Grundtext}$} Umschrift $|$"Ubersetzung(en)\\
1.&9.&6200.&32.&23790.&1.&5&610&2\_200\_2\_6\_400 \textcolor{red}{\textcjheb{twbrb}} BRBWT $|$wenn sich mehren\\
2.&10.&6201.&37.&23795.&6.&6&254&90\_4\_10\_100\_10\_40 \textcolor{red}{\textcjheb{myqyd.s}} "sDJQJM $|$(die) Gerechte(n)\\
3.&11.&6202.&43.&23801.&12.&4&358&10\_300\_40\_8 \textcolor{red}{\textcjheb{.hm+sy}} JSMC $|$(er (=es)) freut sich\\
4.&12.&6203.&47.&23805.&16.&3&115&5\_70\_40 \textcolor{red}{\textcjheb{m`h}} HaM $|$das Volk\\
5.&13.&6204.&50.&23808.&19.&5&378&6\_2\_40\_300\_30 \textcolor{red}{\textcjheb{l+smbw}} WBMSL $|$wenn aber herrscht/und wenn herrscht\\
6.&14.&6205.&55.&23813.&24.&3&570&200\_300\_70 \textcolor{red}{\textcjheb{`+sr}} RSa $|$ein Gesetzloser/(ein) Frevler\\
7.&15.&6206.&58.&23816.&27.&4&69&10\_1\_50\_8 \textcolor{red}{\textcjheb{.hn'y}} JANC $|$(er (=es)) seufzt\\
8.&16.&6207.&62.&23820.&31.&2&110&70\_40 \textcolor{red}{\textcjheb{m`}} aM $|$ein Volk/das Volk\\
\end{tabular}\medskip \\
Ende des Verses 29.2\\
Verse: 826, Buchstaben: 32, 63, 23821, Totalwerte: 2464, 5860, 1679143\\
\\
Wenn die Gerechten sich mehren, freut sich das Volk; wenn aber ein Gesetzloser herrscht, seufzt ein Volk.\\
\newpage 
{\bf -- 29.3}\\
\medskip \\
\begin{tabular}{rrrrrrrrp{120mm}}
WV&WK&WB&ABK&ABB&ABV&AnzB&TW&Zahlencode \textcolor{red}{$\boldsymbol{Grundtext}$} Umschrift $|$"Ubersetzung(en)\\
1.&17.&6208.&64.&23822.&1.&3&311&1\_10\_300 \textcolor{red}{\textcjheb{+sy'}} AJS $|$(ein) Mann\\
2.&18.&6209.&67.&23825.&4.&3&8&1\_5\_2 \textcolor{red}{\textcjheb{bh'}} AHB $|$der liebt/liebend\\
3.&19.&6210.&70.&23828.&7.&4&73&8\_20\_40\_5 \textcolor{red}{\textcjheb{hmk.h}} CKMH $|$Weisheit\\
4.&20.&6211.&74.&23832.&11.&4&358&10\_300\_40\_8 \textcolor{red}{\textcjheb{.hm+sy}} JSMC $|$er (er)freut\\
5.&21.&6212.&78.&23836.&15.&4&19&1\_2\_10\_6 \textcolor{red}{\textcjheb{wyb'}} ABJW $|$seinen Vater\\
6.&22.&6213.&82.&23840.&19.&4&281&6\_200\_70\_5 \textcolor{red}{\textcjheb{h`rw}} WRaH $|$wer sich aber gesellt/und ein Verkehrender\\
7.&23.&6214.&86.&23844.&23.&5&469&7\_6\_50\_6\_400 \textcolor{red}{\textcjheb{twnwz}} ZWNWT $|$zu Huren/mit Dirnen\\
8.&24.&6215.&91.&23849.&28.&4&17&10\_1\_2\_4 \textcolor{red}{\textcjheb{db'y}} JABD $|$richtet zu Grunde/(er) vernichtet\\
9.&25.&6216.&95.&23853.&32.&3&61&5\_6\_50 \textcolor{red}{\textcjheb{nwh}} HWN $|$(das) Verm"ogen\\
\end{tabular}\medskip \\
Ende des Verses 29.3\\
Verse: 827, Buchstaben: 34, 97, 23855, Totalwerte: 1597, 7457, 1680740\\
\\
Ein Mann, der Weisheit liebt, erfreut seinen Vater; wer sich aber zu Huren gesellt, richtet das Verm"ogen zu Grunde.\\
\newpage 
{\bf -- 29.4}\\
\medskip \\
\begin{tabular}{rrrrrrrrp{120mm}}
WV&WK&WB&ABK&ABB&ABV&AnzB&TW&Zahlencode \textcolor{red}{$\boldsymbol{Grundtext}$} Umschrift $|$"Ubersetzung(en)\\
1.&26.&6217.&98.&23856.&1.&3&90&40\_30\_20 \textcolor{red}{\textcjheb{klm}} MLK $|$(ein) K"onig\\
2.&27.&6218.&101.&23859.&4.&5&431&2\_40\_300\_80\_9 \textcolor{red}{\textcjheb{.tp+smb}} BMSPt $|$durch Recht/mit dem Recht\\
3.&28.&6219.&106.&23864.&9.&5&134&10\_70\_40\_10\_4 \textcolor{red}{\textcjheb{dym`y}} JaMJD $|$gibt Bestand/(er) richtet auf\\
4.&29.&6220.&111.&23869.&14.&3&291&1\_200\_90 \textcolor{red}{\textcjheb{.sr'}} AR"s $|$dem Land/(ein) Land\\
5.&30.&6221.&114.&23872.&17.&4&317&6\_1\_10\_300 \textcolor{red}{\textcjheb{+sy'w}} WAJS $|$aber ein Mann/und (ein) Mann\\
6.&31.&6222.&118.&23876.&21.&6&1052&400\_200\_6\_40\_6\_400 \textcolor{red}{\textcjheb{twmwrt}} TRWMWT $|$der Geschenke liebt/(h"aufend) Abgaben\\
7.&32.&6223.&124.&23882.&27.&6&330&10\_5\_200\_60\_50\_5 \textcolor{red}{\textcjheb{hnsrhy}} JHRsNH $|$bringt es herunter/(er) zerst"ort sie (=es)\\
\end{tabular}\medskip \\
Ende des Verses 29.4\\
Verse: 828, Buchstaben: 32, 129, 23887, Totalwerte: 2645, 10102, 1683385\\
\\
Ein K"onig gibt durch Recht dem Lande Bestand; aber ein Mann, der Geschenke liebt, bringt es herunter.\\
\newpage 
{\bf -- 29.5}\\
\medskip \\
\begin{tabular}{rrrrrrrrp{120mm}}
WV&WK&WB&ABK&ABB&ABV&AnzB&TW&Zahlencode \textcolor{red}{$\boldsymbol{Grundtext}$} Umschrift $|$"Ubersetzung(en)\\
1.&33.&6224.&130.&23888.&1.&3&205&3\_2\_200 \textcolor{red}{\textcjheb{rbg}} GBR $|$ein Mann\\
2.&34.&6225.&133.&23891.&4.&5&188&40\_8\_30\_10\_100 \textcolor{red}{\textcjheb{qyl.hm}} MCLJQ $|$der schmeichelt/schmeichelnder\\
3.&35.&6226.&138.&23896.&9.&2&100&70\_30 \textcolor{red}{\textcjheb{l`}} aL $|$/bei\\
4.&36.&6227.&140.&23898.&11.&4&281&200\_70\_5\_6 \textcolor{red}{\textcjheb{wh`r}} RaHW $|$seinem N"achsten\\
5.&37.&6228.&144.&23902.&15.&3&900&200\_300\_400 \textcolor{red}{\textcjheb{t+sr}} RST $|$(ein) Netz\\
6.&38.&6229.&147.&23905.&18.&4&586&80\_6\_200\_300 \textcolor{red}{\textcjheb{+srwp}} PWRS $|$breitet aus/(er ist) ausbreitend\\
7.&39.&6230.&151.&23909.&22.&2&100&70\_30 \textcolor{red}{\textcjheb{l`}} aL $|$vor/f"ur\\
8.&40.&6231.&153.&23911.&24.&5&206&80\_70\_40\_10\_6 \textcolor{red}{\textcjheb{wym`p}} PaMJW $|$seine Tritte/seine Schritte\\
\end{tabular}\medskip \\
Ende des Verses 29.5\\
Verse: 829, Buchstaben: 28, 157, 23915, Totalwerte: 2566, 12668, 1685951\\
\\
Ein Mann, der seinem N"achsten schmeichelt, breitet ein Netz aus vor seine Tritte.\\
\newpage 
{\bf -- 29.6}\\
\medskip \\
\begin{tabular}{rrrrrrrrp{120mm}}
WV&WK&WB&ABK&ABB&ABV&AnzB&TW&Zahlencode \textcolor{red}{$\boldsymbol{Grundtext}$} Umschrift $|$"Ubersetzung(en)\\
1.&41.&6232.&158.&23916.&1.&4&452&2\_80\_300\_70 \textcolor{red}{\textcjheb{`+spb}} BPSa $|$in der "Ubertretung/in der Missetat\\
2.&42.&6233.&162.&23920.&5.&3&311&1\_10\_300 \textcolor{red}{\textcjheb{+sy'}} AJS $|$des Mannes/(eines) Mannes\\
3.&43.&6234.&165.&23923.&8.&2&270&200\_70 \textcolor{red}{\textcjheb{`r}} Ra $|$b"osen/(des) B"osen\\
4.&44.&6235.&167.&23925.&10.&4&446&40\_6\_100\_300 \textcolor{red}{\textcjheb{+sqwm}} MWQS $|$ist ein Fallstrick/(liegt der) Fallstrick\\
5.&45.&6236.&171.&23929.&14.&5&210&6\_90\_4\_10\_100 \textcolor{red}{\textcjheb{qyd.sw}} W"sDJQ $|$aber der Gerechte/und (ein) Gerechter\\
6.&46.&6237.&176.&23934.&19.&4&266&10\_200\_6\_50 \textcolor{red}{\textcjheb{nwry}} JRWN $|$(er) jubelt\\
7.&47.&6238.&180.&23938.&23.&4&354&6\_300\_40\_8 \textcolor{red}{\textcjheb{.hm+sw}} WSMC $|$und (ist) fr"ohlich\\
\end{tabular}\medskip \\
Ende des Verses 29.6\\
Verse: 830, Buchstaben: 26, 183, 23941, Totalwerte: 2309, 14977, 1688260\\
\\
In der "Ubertretung des b"osen Mannes ist ein Fallstrick; aber der Gerechte jubelt und ist fr"ohlich.\\
\newpage 
{\bf -- 29.7}\\
\medskip \\
\begin{tabular}{rrrrrrrrp{120mm}}
WV&WK&WB&ABK&ABB&ABV&AnzB&TW&Zahlencode \textcolor{red}{$\boldsymbol{Grundtext}$} Umschrift $|$"Ubersetzung(en)\\
1.&48.&6239.&184.&23942.&1.&3&84&10\_4\_70 \textcolor{red}{\textcjheb{`dy}} JDa $|$(es) erkennt/Kennend(er) (ist)\\
2.&49.&6240.&187.&23945.&4.&4&204&90\_4\_10\_100 \textcolor{red}{\textcjheb{qyd.s}} "sDJQ $|$der Gerechte/(ein) Gerechter\\
3.&50.&6241.&191.&23949.&8.&3&64&4\_10\_50 \textcolor{red}{\textcjheb{nyd}} DJN $|$das Recht/die Rechtslage\\
4.&51.&6242.&194.&23952.&11.&4&84&4\_30\_10\_40 \textcolor{red}{\textcjheb{myld}} DLJM $|$der Armen/Hilfsbed"urftiger\\
5.&52.&6243.&198.&23956.&15.&3&570&200\_300\_70 \textcolor{red}{\textcjheb{`+sr}} RSa $|$der Gesetzlose/(ein) Frevler\\
6.&53.&6244.&201.&23959.&18.&2&31&30\_1 \textcolor{red}{\textcjheb{'l}} LA $|$nicht\\
7.&54.&6245.&203.&23961.&20.&4&72&10\_2\_10\_50 \textcolor{red}{\textcjheb{nyby}} JBJN $|$versteht/er wird Einsicht haben\\
8.&55.&6246.&207.&23965.&24.&3&474&4\_70\_400 \textcolor{red}{\textcjheb{t`d}} DaT $|$Erkenntnis/(und) Verst"andnis\\
\end{tabular}\medskip \\
Ende des Verses 29.7\\
Verse: 831, Buchstaben: 26, 209, 23967, Totalwerte: 1583, 16560, 1689843\\
\\
Der Gerechte erkennt das Recht der Armen; der Gesetzlose versteht keine Erkenntnis.\\
\newpage 
{\bf -- 29.8}\\
\medskip \\
\begin{tabular}{rrrrrrrrp{120mm}}
WV&WK&WB&ABK&ABB&ABV&AnzB&TW&Zahlencode \textcolor{red}{$\boldsymbol{Grundtext}$} Umschrift $|$"Ubersetzung(en)\\
1.&56.&6247.&210.&23968.&1.&4&361&1\_50\_300\_10 \textcolor{red}{\textcjheb{y+sn'}} ANSJ $|$/M"anner\\
2.&57.&6248.&214.&23972.&5.&4&176&30\_90\_6\_50 \textcolor{red}{\textcjheb{nw.sl}} L"sWN $|$Sp"otter/des Spotts\\
3.&58.&6249.&218.&23976.&9.&5&114&10\_80\_10\_8\_6 \textcolor{red}{\textcjheb{w.hypy}} JPJCW $|$versetzen in Aufruhr/(sie) erregen\\
4.&59.&6250.&223.&23981.&14.&4&315&100\_200\_10\_5 \textcolor{red}{\textcjheb{hyrq}} QRJH $|$eine Stadt\\
5.&60.&6251.&227.&23985.&18.&6&124&6\_8\_20\_40\_10\_40 \textcolor{red}{\textcjheb{mymk.hw}} WCKMJM $|$Weise aber/und Weise\\
6.&61.&6252.&233.&23991.&24.&5&328&10\_300\_10\_2\_6 \textcolor{red}{\textcjheb{wby+sy}} JSJBW $|$wenden ab/(sie) beschwichtigen\\
7.&62.&6253.&238.&23996.&29.&2&81&1\_80 \textcolor{red}{\textcjheb{p'}} AP $|$(den) Zorn\\
\end{tabular}\medskip \\
Ende des Verses 29.8\\
Verse: 832, Buchstaben: 30, 239, 23997, Totalwerte: 1499, 18059, 1691342\\
\\
Sp"otter versetzen eine Stadt in Aufruhr, Weise aber wenden den Zorn ab.\\
\newpage 
{\bf -- 29.9}\\
\medskip \\
\begin{tabular}{rrrrrrrrp{120mm}}
WV&WK&WB&ABK&ABB&ABV&AnzB&TW&Zahlencode \textcolor{red}{$\boldsymbol{Grundtext}$} Umschrift $|$"Ubersetzung(en)\\
1.&63.&6254.&240.&23998.&1.&3&311&1\_10\_300 \textcolor{red}{\textcjheb{+sy'}} AJS $|$(wenn) (ein) Mann\\
2.&64.&6255.&243.&24001.&4.&3&68&8\_20\_40 \textcolor{red}{\textcjheb{mk.h}} CKM $|$weiser\\
3.&65.&6256.&246.&24004.&7.&4&439&50\_300\_80\_9 \textcolor{red}{\textcjheb{.tp+sn}} NSPt $|$(er (=es)) rechtet\\
4.&66.&6257.&250.&24008.&11.&2&401&1\_400 \textcolor{red}{\textcjheb{t'}} AT $|$mit\\
5.&67.&6258.&252.&24010.&13.&3&311&1\_10\_300 \textcolor{red}{\textcjheb{+sy'}} AJS $|$(einem) (Mann)\\
6.&68.&6259.&255.&24013.&16.&4&47&1\_6\_10\_30 \textcolor{red}{\textcjheb{lyw'}} AWJL $|$n"arrischen/(einem) Narren\\
7.&69.&6260.&259.&24017.&20.&4&216&6\_200\_3\_7 \textcolor{red}{\textcjheb{zgrw}} WRGZ $|$mag er sich erz"urnen/und (d)er tobt(e)\\
8.&70.&6261.&263.&24021.&24.&4&414&6\_300\_8\_100 \textcolor{red}{\textcjheb{q.h+sw}} WSCQ $|$oder lachen/und lacht(e)\\
9.&71.&6262.&267.&24025.&28.&4&67&6\_1\_10\_50 \textcolor{red}{\textcjheb{ny'w}} WAJN $|$(und) nicht\\
10.&72.&6263.&271.&24029.&32.&3&458&50\_8\_400 \textcolor{red}{\textcjheb{t.hn}} NCT $|$hat er Ruhe/(ist) Ruhe\\
\end{tabular}\medskip \\
Ende des Verses 29.9\\
Verse: 833, Buchstaben: 34, 273, 24031, Totalwerte: 2732, 20791, 1694074\\
\\
Wenn ein weiser Mann mit einem n"arrischen Manne rechtet-mag er sich erz"urnen oder lachen, er hat keine Ruhe.\\
\newpage 
{\bf -- 29.10}\\
\medskip \\
\begin{tabular}{rrrrrrrrp{120mm}}
WV&WK&WB&ABK&ABB&ABV&AnzB&TW&Zahlencode \textcolor{red}{$\boldsymbol{Grundtext}$} Umschrift $|$"Ubersetzung(en)\\
1.&73.&6264.&274.&24032.&1.&4&361&1\_50\_300\_10 \textcolor{red}{\textcjheb{y+sn'}} ANSJ $|$Menschen/M"anner\\
2.&74.&6265.&278.&24036.&5.&4&94&4\_40\_10\_40 \textcolor{red}{\textcjheb{mymd}} DMJM $|$(von) Blut(taten)\\
3.&75.&6266.&282.&24040.&9.&5&367&10\_300\_50\_1\_6 \textcolor{red}{\textcjheb{w'n+sy}} JSNAW $|$(sie) hassen\\
4.&76.&6267.&287.&24045.&14.&2&440&400\_40 \textcolor{red}{\textcjheb{mt}} TM $|$den Unstr"aflichen/(einen) Unbescholtenen\\
5.&77.&6268.&289.&24047.&16.&6&566&6\_10\_300\_200\_10\_40 \textcolor{red}{\textcjheb{myr+syw}} WJSRJM $|$aber die Aufrichtigen/und Gerade\\
6.&78.&6269.&295.&24053.&22.&5&418&10\_2\_100\_300\_6 \textcolor{red}{\textcjheb{w+sqby}} JBQSW $|$(sie) (be)k"ummern sich\\
7.&79.&6270.&300.&24058.&27.&4&436&50\_80\_300\_6 \textcolor{red}{\textcjheb{w+spn}} NPSW $|$(um) seine Seele\\
\end{tabular}\medskip \\
Ende des Verses 29.10\\
Verse: 834, Buchstaben: 30, 303, 24061, Totalwerte: 2682, 23473, 1696756\\
\\
Blutmenschen hassen den Unstr"aflichen, aber die Aufrichtigen bek"ummern sich um seine Seele.\\
\newpage 
{\bf -- 29.11}\\
\medskip \\
\begin{tabular}{rrrrrrrrp{120mm}}
WV&WK&WB&ABK&ABB&ABV&AnzB&TW&Zahlencode \textcolor{red}{$\boldsymbol{Grundtext}$} Umschrift $|$"Ubersetzung(en)\\
1.&80.&6271.&304.&24062.&1.&2&50&20\_30 \textcolor{red}{\textcjheb{lk}} KL $|$seinen ganzen/all\\
2.&81.&6272.&306.&24064.&3.&4&220&200\_6\_8\_6 \textcolor{red}{\textcjheb{w.hwr}} RWCW $|$(seinem) Unmut\\
3.&82.&6273.&310.&24068.&7.&5&117&10\_6\_90\_10\_1 \textcolor{red}{\textcjheb{'y.swy}} JW"sJA $|$l"asst herausfahren/er (=es) l"asst freien Lauf\\
4.&83.&6274.&315.&24073.&12.&4&120&20\_60\_10\_30 \textcolor{red}{\textcjheb{lysk}} KsJL $|$(der) Tor\\
5.&84.&6275.&319.&24077.&16.&4&74&6\_8\_20\_40 \textcolor{red}{\textcjheb{mk.hw}} WCKM $|$aber der Weise/und der Weise\\
6.&85.&6276.&323.&24081.&20.&5&217&2\_1\_8\_6\_200 \textcolor{red}{\textcjheb{rw.h'b}} BACWR $|$h"alt ihn zur"uck/zuletzt\\
7.&86.&6277.&328.&24086.&25.&6&375&10\_300\_2\_8\_50\_5 \textcolor{red}{\textcjheb{hn.hb+sy}} JSBCNH $|$beschwichtigend/(er) beschwichtigt\\
\end{tabular}\medskip \\
Ende des Verses 29.11\\
Verse: 835, Buchstaben: 30, 333, 24091, Totalwerte: 1173, 24646, 1697929\\
\\
Der Tor l"a"st seinen ganzen Unmut herausfahren, aber der Weise h"alt ihn beschwichtigend zur"uck.\\
\newpage 
{\bf -- 29.12}\\
\medskip \\
\begin{tabular}{rrrrrrrrp{120mm}}
WV&WK&WB&ABK&ABB&ABV&AnzB&TW&Zahlencode \textcolor{red}{$\boldsymbol{Grundtext}$} Umschrift $|$"Ubersetzung(en)\\
1.&87.&6278.&334.&24092.&1.&3&370&40\_300\_30 \textcolor{red}{\textcjheb{l+sm}} MSL $|$ein Herrscher/(ein) Herrschender\\
2.&88.&6279.&337.&24095.&4.&5&452&40\_100\_300\_10\_2 \textcolor{red}{\textcjheb{by+sqm}} MQSJB $|$der horcht/(der) h"orend (ist)\\
3.&89.&6280.&342.&24100.&9.&2&100&70\_30 \textcolor{red}{\textcjheb{l`}} aL $|$auf\\
4.&90.&6281.&344.&24102.&11.&3&206&4\_2\_200 \textcolor{red}{\textcjheb{rbd}} DBR $|$Rede/(ein) Wort\\
5.&91.&6282.&347.&24105.&14.&3&600&300\_100\_200 \textcolor{red}{\textcjheb{rq+s}} SQR $|$(der) L"uge(n)\\
6.&92.&6283.&350.&24108.&17.&2&50&20\_30 \textcolor{red}{\textcjheb{lk}} KL $|$alle\\
7.&93.&6284.&352.&24110.&19.&6&956&40\_300\_200\_400\_10\_6 \textcolor{red}{\textcjheb{wytr+sm}} MSRTJW $|$dessen Diener/seine Bediensteten\\
8.&94.&6285.&358.&24116.&25.&5&620&200\_300\_70\_10\_40 \textcolor{red}{\textcjheb{my`+sr}} RSaJM $|$sind gesetzlos/(werden) Frevler\\
\end{tabular}\medskip \\
Ende des Verses 29.12\\
Verse: 836, Buchstaben: 29, 362, 24120, Totalwerte: 3354, 28000, 1701283\\
\\
Ein Herrscher, der auf L"ugenrede horcht, dessen Diener sind alle gesetzlos.\\
\newpage 
{\bf -- 29.13}\\
\medskip \\
\begin{tabular}{rrrrrrrrp{120mm}}
WV&WK&WB&ABK&ABB&ABV&AnzB&TW&Zahlencode \textcolor{red}{$\boldsymbol{Grundtext}$} Umschrift $|$"Ubersetzung(en)\\
1.&95.&6286.&363.&24121.&1.&2&500&200\_300 \textcolor{red}{\textcjheb{+sr}} RS $|$der Arme/(ein) Armer\\
2.&96.&6287.&365.&24123.&3.&4&317&6\_1\_10\_300 \textcolor{red}{\textcjheb{+sy'w}} WAJS $|$und der/und (ein) Mann\\
3.&97.&6288.&369.&24127.&7.&5&490&400\_20\_20\_10\_40 \textcolor{red}{\textcjheb{mykkt}} TKKJM $|$Bedr"ucker/der Bedr"uckungen\\
4.&98.&6289.&374.&24132.&12.&5&439&50\_80\_3\_300\_6 \textcolor{red}{\textcjheb{w+sgpn}} NPGSW $|$(sie) begegne(te)n sich\\
5.&99.&6290.&379.&24137.&17.&4&251&40\_1\_10\_200 \textcolor{red}{\textcjheb{ry'm}} MAJR $|$(es) erleuchtet/leuchten machend\\
6.&100.&6291.&383.&24141.&21.&4&140&70\_10\_50\_10 \textcolor{red}{\textcjheb{yny`}} aJNJ $|$(die) Augen\\
7.&101.&6292.&387.&24145.&25.&5&405&300\_50\_10\_5\_40 \textcolor{red}{\textcjheb{mhyn+s}} SNJHM $|$ihrer beider/beider von ihnen\\
8.&102.&6293.&392.&24150.&30.&4&26&10\_5\_6\_5 \textcolor{red}{\textcjheb{hwhy}} JHWH $|$(ist) Jahwe\\
\end{tabular}\medskip \\
Ende des Verses 29.13\\
Verse: 837, Buchstaben: 33, 395, 24153, Totalwerte: 2568, 30568, 1703851\\
\\
Der Arme und der Bedr"ucker begegnen sich: Jahwe erleuchtet ihrer beider Augen.\\
\newpage 
{\bf -- 29.14}\\
\medskip \\
\begin{tabular}{rrrrrrrrp{120mm}}
WV&WK&WB&ABK&ABB&ABV&AnzB&TW&Zahlencode \textcolor{red}{$\boldsymbol{Grundtext}$} Umschrift $|$"Ubersetzung(en)\\
1.&103.&6294.&396.&24154.&1.&3&90&40\_30\_20 \textcolor{red}{\textcjheb{klm}} MLK $|$(wenn) (ein) K"onig\\
2.&104.&6295.&399.&24157.&4.&4&395&300\_6\_80\_9 \textcolor{red}{\textcjheb{.tpw+s}} SWPt $|$der richtet/(ist) rechtsprechend\\
3.&105.&6296.&403.&24161.&8.&4&443&2\_1\_40\_400 \textcolor{red}{\textcjheb{tm'b}} BAMT $|$in Wahrheit\\
4.&106.&6297.&407.&24165.&12.&4&84&4\_30\_10\_40 \textcolor{red}{\textcjheb{myld}} DLJM $|$die Geringen/Geringe\\
5.&107.&6298.&411.&24169.&16.&4&87&20\_60\_1\_6 \textcolor{red}{\textcjheb{w'sk}} KsAW $|$dessen Thron/sein Thron\\
6.&108.&6299.&415.&24173.&20.&3&104&30\_70\_4 \textcolor{red}{\textcjheb{d`l}} LaD $|$immerdar/f"ur immer\\
7.&109.&6300.&418.&24176.&23.&4&86&10\_20\_6\_50 \textcolor{red}{\textcjheb{nwky}} JKWN $|$wird feststehen/hat Bestand\\
\end{tabular}\medskip \\
Ende des Verses 29.14\\
Verse: 838, Buchstaben: 26, 421, 24179, Totalwerte: 1289, 31857, 1705140\\
\\
Ein K"onig, der die Geringen in Wahrheit richtet, dessen Thron wird feststehen immerdar.\\
\newpage 
{\bf -- 29.15}\\
\medskip \\
\begin{tabular}{rrrrrrrrp{120mm}}
WV&WK&WB&ABK&ABB&ABV&AnzB&TW&Zahlencode \textcolor{red}{$\boldsymbol{Grundtext}$} Umschrift $|$"Ubersetzung(en)\\
1.&110.&6301.&422.&24180.&1.&3&311&300\_2\_9 \textcolor{red}{\textcjheb{.tb+s}} SBt $|$Rute\\
2.&111.&6302.&425.&24183.&4.&6&840&6\_400\_6\_20\_8\_400 \textcolor{red}{\textcjheb{t.hkwtw}} WTWKCT $|$und Zucht/und Zurechtweisung\\
3.&112.&6303.&431.&24189.&10.&3&460&10\_400\_50 \textcolor{red}{\textcjheb{nty}} JTN $|$geben/(er) gibt\\
4.&113.&6304.&434.&24192.&13.&4&73&8\_20\_40\_5 \textcolor{red}{\textcjheb{hmk.h}} CKMH $|$Weisheit\\
5.&114.&6305.&438.&24196.&17.&4&326&6\_50\_70\_200 \textcolor{red}{\textcjheb{r`nw}} WNaR $|$aber ein Knabe/und (ein) Knabe\\
6.&115.&6306.&442.&24200.&21.&4&378&40\_300\_30\_8 \textcolor{red}{\textcjheb{.hl+sm}} MSLC $|$sich selbst "uberlassener/zuchtloser\\
7.&116.&6307.&446.&24204.&25.&4&352&40\_2\_10\_300 \textcolor{red}{\textcjheb{+sybm}} MBJS $|$macht Schande/(ist) Schande machend\\
8.&117.&6308.&450.&24208.&29.&3&47&1\_40\_6 \textcolor{red}{\textcjheb{wm'}} AMW $|$seiner Mutter\\
\end{tabular}\medskip \\
Ende des Verses 29.15\\
Verse: 839, Buchstaben: 31, 452, 24210, Totalwerte: 2787, 34644, 1707927\\
\\
Rute und Zucht geben Weisheit; aber ein sich selbst "uberlassener Knabe macht seiner Mutter Schande.\\
\newpage 
{\bf -- 29.16}\\
\medskip \\
\begin{tabular}{rrrrrrrrp{120mm}}
WV&WK&WB&ABK&ABB&ABV&AnzB&TW&Zahlencode \textcolor{red}{$\boldsymbol{Grundtext}$} Umschrift $|$"Ubersetzung(en)\\
1.&118.&6309.&453.&24211.&1.&5&610&2\_200\_2\_6\_400 \textcolor{red}{\textcjheb{twbrb}} BRBWT $|$wenn sich mehren\\
2.&119.&6310.&458.&24216.&6.&5&620&200\_300\_70\_10\_40 \textcolor{red}{\textcjheb{my`+sr}} RSaJM $|$die Gesetzlosen/(die) Frevler\\
3.&120.&6311.&463.&24221.&11.&4&217&10\_200\_2\_5 \textcolor{red}{\textcjheb{hbry}} JRBH $|$(er (=es)) mehrt sich\\
4.&121.&6312.&467.&24225.&15.&3&450&80\_300\_70 \textcolor{red}{\textcjheb{`+sp}} PSa $|$die "Ubertretung/Missetat\\
5.&122.&6313.&470.&24228.&18.&7&260&6\_90\_4\_10\_100\_10\_40 \textcolor{red}{\textcjheb{myqyd.sw}} W"sDJQJM $|$aber die Gerechten/und die Gerechten\\
6.&123.&6314.&477.&24235.&25.&6&592&2\_40\_80\_30\_400\_40 \textcolor{red}{\textcjheb{mtlpmb}} BMPLTM $|$ihrem Fall/bei ihrem Sturz\\
7.&124.&6315.&483.&24241.&31.&4&217&10\_200\_1\_6 \textcolor{red}{\textcjheb{w'ry}} JRAW $|$(sie) werden (zu)sehen\\
\end{tabular}\medskip \\
Ende des Verses 29.16\\
Verse: 840, Buchstaben: 34, 486, 24244, Totalwerte: 2966, 37610, 1710893\\
\\
Wenn die Gesetzlosen sich mehren, mehrt sich die "Ubertretung; aber die Gerechten werden ihrem Falle zusehen.\\
\newpage 
{\bf -- 29.17}\\
\medskip \\
\begin{tabular}{rrrrrrrrp{120mm}}
WV&WK&WB&ABK&ABB&ABV&AnzB&TW&Zahlencode \textcolor{red}{$\boldsymbol{Grundtext}$} Umschrift $|$"Ubersetzung(en)\\
1.&125.&6316.&487.&24245.&1.&3&270&10\_60\_200 \textcolor{red}{\textcjheb{rsy}} JsR $|$z"uchtige\\
2.&126.&6317.&490.&24248.&4.&3&72&2\_50\_20 \textcolor{red}{\textcjheb{knb}} BNK $|$deinen Sohn\\
3.&127.&6318.&493.&24251.&7.&6&104&6\_10\_50\_10\_8\_20 \textcolor{red}{\textcjheb{k.hynyw}} WJNJCK $|$so wird er dir Ruhe verschaffen/und er wird ruhen machen dich\\
4.&128.&6319.&499.&24257.&13.&4&466&6\_10\_400\_50 \textcolor{red}{\textcjheb{ntyw}} WJTN $|$und gew"ahren/und er bereitet\\
5.&129.&6320.&503.&24261.&17.&6&214&40\_70\_4\_50\_10\_40 \textcolor{red}{\textcjheb{mynd`m}} MaDNJM $|$Wonne\\
6.&130.&6321.&509.&24267.&23.&5&480&30\_50\_80\_300\_20 \textcolor{red}{\textcjheb{k+spnl}} LNPSK $|$deiner Seele\\
\end{tabular}\medskip \\
Ende des Verses 29.17\\
Verse: 841, Buchstaben: 27, 513, 24271, Totalwerte: 1606, 39216, 1712499\\
\\
Z"uchtige deinen Sohn, so wird er dir Ruhe verschaffen und Wonne gew"ahren deiner Seele.\\
\newpage 
{\bf -- 29.18}\\
\medskip \\
\begin{tabular}{rrrrrrrrp{120mm}}
WV&WK&WB&ABK&ABB&ABV&AnzB&TW&Zahlencode \textcolor{red}{$\boldsymbol{Grundtext}$} Umschrift $|$"Ubersetzung(en)\\
1.&131.&6322.&514.&24272.&1.&4&63&2\_1\_10\_50 \textcolor{red}{\textcjheb{ny'b}} BAJN $|$wenn nicht da ist/ohne\\
2.&132.&6323.&518.&24276.&5.&4&71&8\_7\_6\_50 \textcolor{red}{\textcjheb{nwz.h}} CZWN $|$ein Gesicht/Schau (=Offenbarung)\\
3.&133.&6324.&522.&24280.&9.&4&360&10\_80\_200\_70 \textcolor{red}{\textcjheb{`rpy}} JPRa $|$(er (=es)) wird z"ugellos\\
4.&134.&6325.&526.&24284.&13.&2&110&70\_40 \textcolor{red}{\textcjheb{m`}} aM $|$(ein) Volk\\
5.&135.&6326.&528.&24286.&15.&4&546&6\_300\_40\_200 \textcolor{red}{\textcjheb{rm+sw}} WSMR $|$aber wenn es beobachtet/und ein Beachtender\\
6.&136.&6327.&532.&24290.&19.&4&611&400\_6\_200\_5 \textcolor{red}{\textcjheb{hrwt}} TWRH $|$das Gesetz/(die) Weisung\\
7.&137.&6328.&536.&24294.&23.&5&512&1\_300\_200\_5\_6 \textcolor{red}{\textcjheb{whr+s'}} ASRHW $|$gl"uckselig ist es/seine Gl"uckseligkeiten (=Heil ihm)\\
\end{tabular}\medskip \\
Ende des Verses 29.18\\
Verse: 842, Buchstaben: 27, 540, 24298, Totalwerte: 2273, 41489, 1714772\\
\\
Wenn kein Gesicht da ist, wird ein Volk z"ugellos; aber gl"uckselig ist es, wenn es das Gesetz beobachtet.\\
\newpage 
{\bf -- 29.19}\\
\medskip \\
\begin{tabular}{rrrrrrrrp{120mm}}
WV&WK&WB&ABK&ABB&ABV&AnzB&TW&Zahlencode \textcolor{red}{$\boldsymbol{Grundtext}$} Umschrift $|$"Ubersetzung(en)\\
1.&138.&6329.&541.&24299.&1.&6&258&2\_4\_2\_200\_10\_40 \textcolor{red}{\textcjheb{myrbdb}} BDBRJM $|$durch Worte\\
2.&139.&6330.&547.&24305.&7.&2&31&30\_1 \textcolor{red}{\textcjheb{'l}} LA $|$nicht\\
3.&140.&6331.&549.&24307.&9.&4&276&10\_6\_60\_200 \textcolor{red}{\textcjheb{rswy}} JWsR $|$wird zurechtgewiesen/er (=es) wird belehrt\\
4.&141.&6332.&553.&24311.&13.&3&76&70\_2\_4 \textcolor{red}{\textcjheb{db`}} aBD $|$(ein) Knecht\\
5.&142.&6333.&556.&24314.&16.&2&30&20\_10 \textcolor{red}{\textcjheb{yk}} KJ $|$denn\\
6.&143.&6334.&558.&24316.&18.&4&72&10\_2\_10\_50 \textcolor{red}{\textcjheb{nyby}} JBJN $|$er versteht\\
7.&144.&6335.&562.&24320.&22.&4&67&6\_1\_10\_50 \textcolor{red}{\textcjheb{ny'w}} WAJN $|$aber nicht/und nicht\\
8.&145.&6336.&566.&24324.&26.&4&165&40\_70\_50\_5 \textcolor{red}{\textcjheb{hn`m}} MaNH $|$er folgt/er handelnd entsprechend\\
\end{tabular}\medskip \\
Ende des Verses 29.19\\
Verse: 843, Buchstaben: 29, 569, 24327, Totalwerte: 975, 42464, 1715747\\
\\
Durch Worte wird ein Knecht nicht zurechtgewiesen; denn er versteht, aber er folgt nicht.\\
\newpage 
{\bf -- 29.20}\\
\medskip \\
\begin{tabular}{rrrrrrrrp{120mm}}
WV&WK&WB&ABK&ABB&ABV&AnzB&TW&Zahlencode \textcolor{red}{$\boldsymbol{Grundtext}$} Umschrift $|$"Ubersetzung(en)\\
1.&146.&6337.&570.&24328.&1.&4&425&8\_7\_10\_400 \textcolor{red}{\textcjheb{tyz.h}} CZJT $|$siehst du\\
2.&147.&6338.&574.&24332.&5.&3&311&1\_10\_300 \textcolor{red}{\textcjheb{+sy'}} AJS $|$(einen) Mann\\
3.&148.&6339.&577.&24335.&8.&2&91&1\_90 \textcolor{red}{\textcjheb{.s'}} A"s $|$der hastig ist/hastend\\
4.&149.&6340.&579.&24337.&10.&6&224&2\_4\_2\_200\_10\_6 \textcolor{red}{\textcjheb{wyrbdb}} BDBRJW $|$in seinen Worten/mit seinen Worten\\
5.&150.&6341.&585.&24343.&16.&4&511&400\_100\_6\_5 \textcolor{red}{\textcjheb{hwqt}} TQWH $|$mehr Hoffnung/(so ist) Hoffnung\\
6.&151.&6342.&589.&24347.&20.&5&150&30\_20\_60\_10\_30 \textcolor{red}{\textcjheb{lyskl}} LKsJL $|$ist f"ur einen Toren/f"ur den Toren\\
7.&152.&6343.&594.&24352.&25.&4&136&40\_40\_50\_6 \textcolor{red}{\textcjheb{wnmm}} MMNW $|$(mehr) als f"ur ihn\\
\end{tabular}\medskip \\
Ende des Verses 29.20\\
Verse: 844, Buchstaben: 28, 597, 24355, Totalwerte: 1848, 44312, 1717595\\
\\
Siehst du einen Mann, der hastig ist in seinen Worten-f"ur einen Toren ist mehr Hoffnung als f"ur ihn.\\
\newpage 
{\bf -- 29.21}\\
\medskip \\
\begin{tabular}{rrrrrrrrp{120mm}}
WV&WK&WB&ABK&ABB&ABV&AnzB&TW&Zahlencode \textcolor{red}{$\boldsymbol{Grundtext}$} Umschrift $|$"Ubersetzung(en)\\
1.&153.&6344.&598.&24356.&1.&4&270&40\_80\_50\_100 \textcolor{red}{\textcjheb{qnpm}} MPNQ $|$wenn einer verh"atschelt/verz"artelnd\\
2.&154.&6345.&602.&24360.&5.&4&360&40\_50\_70\_200 \textcolor{red}{\textcjheb{r`nm}} MNaR $|$von Jugend auf/von Jugend an\\
3.&155.&6346.&606.&24364.&9.&4&82&70\_2\_4\_6 \textcolor{red}{\textcjheb{wdb`}} aBDW $|$seinen Knecht\\
4.&156.&6347.&610.&24368.&13.&7&631&6\_1\_8\_200\_10\_400\_6 \textcolor{red}{\textcjheb{wtyr.h'w}} WACRJTW $|$so am Ende/und zuletzt\\
5.&157.&6348.&617.&24375.&20.&4&30&10\_5\_10\_5 \textcolor{red}{\textcjheb{hyhy}} JHJH $|$wird dieser werden/er wird sein\\
6.&158.&6349.&621.&24379.&24.&4&146&40\_50\_6\_50 \textcolor{red}{\textcjheb{nwnm}} MNWN $|$zum Sohn/ein Ver"achter\\
\end{tabular}\medskip \\
Ende des Verses 29.21\\
Verse: 845, Buchstaben: 27, 624, 24382, Totalwerte: 1519, 45831, 1719114\\
\\
Wenn einer seinen Knecht von Jugend auf verh"atschelt, so wird dieser am Ende zum Sohne werden.\\
\newpage 
{\bf -- 29.22}\\
\medskip \\
\begin{tabular}{rrrrrrrrp{120mm}}
WV&WK&WB&ABK&ABB&ABV&AnzB&TW&Zahlencode \textcolor{red}{$\boldsymbol{Grundtext}$} Umschrift $|$"Ubersetzung(en)\\
1.&159.&6350.&625.&24383.&1.&3&311&1\_10\_300 \textcolor{red}{\textcjheb{+sy'}} AJS $|$(ein) Mann\\
2.&160.&6351.&628.&24386.&4.&2&81&1\_80 \textcolor{red}{\textcjheb{p'}} AP $|$zorniger/des Zorns\\
3.&161.&6352.&630.&24388.&6.&4&218&10\_3\_200\_5 \textcolor{red}{\textcjheb{hrgy}} JGRH $|$(er) erregt\\
4.&162.&6353.&634.&24392.&10.&4&100&40\_4\_6\_50 \textcolor{red}{\textcjheb{nwdm}} MDWN $|$Zank/Streit\\
5.&163.&6354.&638.&24396.&14.&4&108&6\_2\_70\_30 \textcolor{red}{\textcjheb{l`bw}} WBaL $|$und ein (Mann)/und der Besitzer\\
6.&164.&6355.&642.&24400.&18.&3&53&8\_40\_5 \textcolor{red}{\textcjheb{hm.h}} CMH $|$Hitziger/(von) Glut\\
7.&165.&6356.&645.&24403.&21.&2&202&200\_2 \textcolor{red}{\textcjheb{br}} RB $|$ist reich/(begeht) viel\\
8.&166.&6357.&647.&24405.&23.&3&450&80\_300\_70 \textcolor{red}{\textcjheb{`+sp}} PSa $|$an "Ubertretung/Verfehlung\\
\end{tabular}\medskip \\
Ende des Verses 29.22\\
Verse: 846, Buchstaben: 25, 649, 24407, Totalwerte: 1523, 47354, 1720637\\
\\
Ein zorniger Mann erregt Zank, und ein Hitziger ist reich an "Ubertretung.\\
\newpage 
{\bf -- 29.23}\\
\medskip \\
\begin{tabular}{rrrrrrrrp{120mm}}
WV&WK&WB&ABK&ABB&ABV&AnzB&TW&Zahlencode \textcolor{red}{$\boldsymbol{Grundtext}$} Umschrift $|$"Ubersetzung(en)\\
1.&167.&6358.&650.&24408.&1.&4&410&3\_1\_6\_400 \textcolor{red}{\textcjheb{tw'g}} GAWT $|$Hoffart/der Hochmut\\
2.&168.&6359.&654.&24412.&5.&3&45&1\_4\_40 \textcolor{red}{\textcjheb{md'}} ADM $|$des Menschen\\
3.&169.&6360.&657.&24415.&8.&7&876&400\_300\_80\_10\_30\_50\_6 \textcolor{red}{\textcjheb{wnlyp+st}} TSPJLNW $|$wird ihn erniedrigen/er (=sie) erniedrigt ihn\\
4.&170.&6361.&664.&24422.&15.&4&416&6\_300\_80\_30 \textcolor{red}{\textcjheb{lp+sw}} WSPL $|$wer aber ist niedrigen/und ein Niedriger\\
5.&171.&6362.&668.&24426.&19.&3&214&200\_6\_8 \textcolor{red}{\textcjheb{.hwr}} RWC $|$(im) Geist(es)\\
6.&172.&6363.&671.&24429.&22.&4&470&10\_400\_40\_20 \textcolor{red}{\textcjheb{kmty}} JTMK $|$wird erlangen/(er) erlangt\\
7.&173.&6364.&675.&24433.&26.&4&32&20\_2\_6\_4 \textcolor{red}{\textcjheb{dwbk}} KBWD $|$Ehre\\
\end{tabular}\medskip \\
Ende des Verses 29.23\\
Verse: 847, Buchstaben: 29, 678, 24436, Totalwerte: 2463, 49817, 1723100\\
\\
Des Menschen Hoffart wird ihn erniedrigen; wer aber niedrigen Geistes ist, wird Ehre erlangen.\\
\newpage 
{\bf -- 29.24}\\
\medskip \\
\begin{tabular}{rrrrrrrrp{120mm}}
WV&WK&WB&ABK&ABB&ABV&AnzB&TW&Zahlencode \textcolor{red}{$\boldsymbol{Grundtext}$} Umschrift $|$"Ubersetzung(en)\\
1.&174.&6365.&679.&24437.&1.&4&144&8\_6\_30\_100 \textcolor{red}{\textcjheb{qlw.h}} CWLQ $|$wer teilt\\
2.&175.&6366.&683.&24441.&5.&2&110&70\_40 \textcolor{red}{\textcjheb{m`}} aM $|$mit\\
3.&176.&6367.&685.&24443.&7.&3&55&3\_50\_2 \textcolor{red}{\textcjheb{bng}} GNB $|$(einem) Dieb\\
4.&177.&6368.&688.&24446.&10.&4&357&300\_6\_50\_1 \textcolor{red}{\textcjheb{'nw+s}} SWNA $|$hasst/(ist) hassender\\
5.&178.&6369.&692.&24450.&14.&4&436&50\_80\_300\_6 \textcolor{red}{\textcjheb{w+spn}} NPSW $|$seine (eigene) Seele\\
6.&179.&6370.&696.&24454.&18.&3&36&1\_30\_5 \textcolor{red}{\textcjheb{hl'}} ALH $|$den Fluch\\
7.&180.&6371.&699.&24457.&21.&4&420&10\_300\_40\_70 \textcolor{red}{\textcjheb{`m+sy}} JSMa $|$er h"ort\\
8.&181.&6372.&703.&24461.&25.&3&37&6\_30\_1 \textcolor{red}{\textcjheb{'lw}} WLA $|$und nicht\\
9.&182.&6373.&706.&24464.&28.&4&27&10\_3\_10\_4 \textcolor{red}{\textcjheb{dygy}} JGJD $|$zeigt es an/er darf (es) anzeigen\\
\end{tabular}\medskip \\
Ende des Verses 29.24\\
Verse: 848, Buchstaben: 31, 709, 24467, Totalwerte: 1622, 51439, 1724722\\
\\
Wer mit einem Diebe teilt, ha"st seine eigene Seele: er h"ort den Fluch und zeigt es nicht an.\\
\newpage 
{\bf -- 29.25}\\
\medskip \\
\begin{tabular}{rrrrrrrrp{120mm}}
WV&WK&WB&ABK&ABB&ABV&AnzB&TW&Zahlencode \textcolor{red}{$\boldsymbol{Grundtext}$} Umschrift $|$"Ubersetzung(en)\\
1.&183.&6374.&710.&24468.&1.&4&612&8\_200\_4\_400 \textcolor{red}{\textcjheb{tdr.h}} CRDT $|$Furcht/der Schrecken\\
2.&184.&6375.&714.&24472.&5.&3&45&1\_4\_40 \textcolor{red}{\textcjheb{md'}} ADM $|$(des) Menschen\\
3.&185.&6376.&717.&24475.&8.&3&460&10\_400\_50 \textcolor{red}{\textcjheb{nty}} JTN $|$(er) legt\\
4.&186.&6377.&720.&24478.&11.&4&446&40\_6\_100\_300 \textcolor{red}{\textcjheb{+sqwm}} MWQS $|$(einen) Fallstrick\\
5.&187.&6378.&724.&24482.&15.&5&31&6\_2\_6\_9\_8 \textcolor{red}{\textcjheb{.h.twbw}} WBWtC $|$wer aber vertraut/und ein Vertrauender\\
6.&188.&6379.&729.&24487.&20.&5&28&2\_10\_5\_6\_5 \textcolor{red}{\textcjheb{hwhyb}} BJHWH $|$auf Jahwe\\
7.&189.&6380.&734.&24492.&25.&4&315&10\_300\_3\_2 \textcolor{red}{\textcjheb{bg+sy}} JSGB $|$wird in Sicherheit gesetzt/(er) wird gesch"utzt\\
\end{tabular}\medskip \\
Ende des Verses 29.25\\
Verse: 849, Buchstaben: 28, 737, 24495, Totalwerte: 1937, 53376, 1726659\\
\\
Menschenfurcht legt einen Fallstrick; wer aber auf Jahwe vertraut, wird in Sicherheit gesetzt.\\
\newpage 
{\bf -- 29.26}\\
\medskip \\
\begin{tabular}{rrrrrrrrp{120mm}}
WV&WK&WB&ABK&ABB&ABV&AnzB&TW&Zahlencode \textcolor{red}{$\boldsymbol{Grundtext}$} Umschrift $|$"Ubersetzung(en)\\
1.&190.&6381.&738.&24496.&1.&4&252&200\_2\_10\_40 \textcolor{red}{\textcjheb{mybr}} RBJM $|$viele\\
2.&191.&6382.&742.&24500.&5.&6&492&40\_2\_100\_300\_10\_40 \textcolor{red}{\textcjheb{my+sqbm}} MBQSJM $|$(sind) suchen(d(e))\\
3.&192.&6383.&748.&24506.&11.&3&140&80\_50\_10 \textcolor{red}{\textcjheb{ynp}} PNJ $|$das Angesicht/das Antlitz\\
4.&193.&6384.&751.&24509.&14.&4&376&40\_6\_300\_30 \textcolor{red}{\textcjheb{l+swm}} MWSL $|$eines Herrschers/(des) Herrschenden\\
5.&194.&6385.&755.&24513.&18.&6&72&6\_40\_10\_5\_6\_5 \textcolor{red}{\textcjheb{hwhymw}} WMJHWH $|$doch von Jahwe/und von Jahwe\\
6.&195.&6386.&761.&24519.&24.&4&429&40\_300\_80\_9 \textcolor{red}{\textcjheb{.tp+sm}} MSPt $|$(kommt) das Recht\\
7.&196.&6387.&765.&24523.&28.&3&311&1\_10\_300 \textcolor{red}{\textcjheb{+sy'}} AJS $|$(des) Mannes\\
\end{tabular}\medskip \\
Ende des Verses 29.26\\
Verse: 850, Buchstaben: 30, 767, 24525, Totalwerte: 2072, 55448, 1728731\\
\\
Viele suchen das Angesicht eines Herrschers, doch von Jahwe kommt das Recht des Mannes.\\
\newpage 
{\bf -- 29.27}\\
\medskip \\
\begin{tabular}{rrrrrrrrp{120mm}}
WV&WK&WB&ABK&ABB&ABV&AnzB&TW&Zahlencode \textcolor{red}{$\boldsymbol{Grundtext}$} Umschrift $|$"Ubersetzung(en)\\
1.&197.&6388.&768.&24526.&1.&5&878&400\_6\_70\_2\_400 \textcolor{red}{\textcjheb{tb`wt}} TWaBT $|$(ein) Gr"auel\\
2.&198.&6389.&773.&24531.&6.&6&254&90\_4\_10\_100\_10\_40 \textcolor{red}{\textcjheb{myqyd.s}} "sDJQJM $|$(f"ur die) Gerechten\\
3.&199.&6390.&779.&24537.&12.&3&311&1\_10\_300 \textcolor{red}{\textcjheb{+sy'}} AJS $|$ist der Mann/(ist ein) Mann\\
4.&200.&6391.&782.&24540.&15.&3&106&70\_6\_30 \textcolor{red}{\textcjheb{lw`}} aWL $|$ungerechte/des Unrechts\\
5.&201.&6392.&785.&24543.&18.&6&884&6\_400\_6\_70\_2\_400 \textcolor{red}{\textcjheb{tb`wtw}} WTWaBT $|$und ein Gr"auel\\
6.&202.&6393.&791.&24549.&24.&3&570&200\_300\_70 \textcolor{red}{\textcjheb{`+sr}} RSa $|$f"ur den Gottlosen/(f"ur den) Frevler\\
7.&203.&6394.&794.&24552.&27.&3&510&10\_300\_200 \textcolor{red}{\textcjheb{r+sy}} JSR $|$wer wandelt geraden/(die) Geradheit\\
8.&204.&6395.&797.&24555.&30.&3&224&4\_200\_20 \textcolor{red}{\textcjheb{krd}} DRK $|$(des) Weges\\
\end{tabular}\medskip \\
Ende des Verses 29.27\\
Verse: 851, Buchstaben: 32, 799, 24557, Totalwerte: 3737, 59185, 1732468\\
\\
Der ungerechte Mann ist ein Greuel f"ur die Gerechten, und wer geraden Weges wandelt, ein Greuel f"ur den Gottlosen.\\
\\
{\bf Ende des Kapitels 29}\\
\newpage 
{\bf -- 30.1}\\
\medskip \\
\begin{tabular}{rrrrrrrrp{120mm}}
WV&WK&WB&ABK&ABB&ABV&AnzB&TW&Zahlencode \textcolor{red}{$\boldsymbol{Grundtext}$} Umschrift $|$"Ubersetzung(en)\\
1.&1.&6396.&1.&24558.&1.&4&216&4\_2\_200\_10 \textcolor{red}{\textcjheb{yrbd}} DBRJ $|$Worte\\
2.&2.&6397.&5.&24562.&5.&4&210&1\_3\_6\_200 \textcolor{red}{\textcjheb{rwg'}} AGWR $|$(von) Agur(s)///$<$Versammelter$>$\\
3.&3.&6398.&9.&24566.&9.&2&52&2\_50 \textcolor{red}{\textcjheb{nb}} BN $|$(des) Sohn(s)\\
4.&4.&6399.&11.&24568.&11.&3&115&10\_100\_5 \textcolor{red}{\textcjheb{hqy}} JQH $|$Jakes/Jakeh(s)//$<$H"orender$>$\\
5.&5.&6400.&14.&24571.&14.&4&346&5\_40\_300\_1 \textcolor{red}{\textcjheb{'+smh}} HMSA $|$der Ausspruch/aus Massa//$<$Last$>$\\
6.&6.&6401.&18.&24575.&18.&3&91&50\_1\_40 \textcolor{red}{\textcjheb{m'n}} NAM $|$es spricht/Spruch\\
7.&7.&6402.&21.&24578.&21.&4&210&5\_3\_2\_200 \textcolor{red}{\textcjheb{rbgh}} HGBR $|$der Mann/des Mannes\\
8.&8.&6403.&25.&24582.&25.&7&482&30\_1\_10\_400\_10\_1\_30 \textcolor{red}{\textcjheb{l'yty'l}} LAJTJAL $|$zu Ithiel/an Itiel//$<$mit mir ist Gott$>$\\
9.&9.&6404.&32.&24589.&32.&7&482&30\_1\_10\_400\_10\_1\_30 \textcolor{red}{\textcjheb{l'yty'l}} LAJTJAL $|$zu Ithiel/f"ur Itiel\\
10.&10.&6405.&39.&24596.&39.&4&57&6\_1\_20\_30 \textcolor{red}{\textcjheb{lk'w}} WAKL $|$und Ukal///$<$abgeh"armt$>$\\
\end{tabular}\medskip \\
Ende des Verses 30.1\\
Verse: 852, Buchstaben: 42, 42, 24599, Totalwerte: 2261, 2261, 1734729\\
\\
Worte Agurs, des Sohnes Jakes, der Ausspruch. Es spricht der Mann zu Ithiel, zu Ithiel und Ukal:\\
\newpage 
{\bf -- 30.2}\\
\medskip \\
\begin{tabular}{rrrrrrrrp{120mm}}
WV&WK&WB&ABK&ABB&ABV&AnzB&TW&Zahlencode \textcolor{red}{$\boldsymbol{Grundtext}$} Umschrift $|$"Ubersetzung(en)\\
1.&11.&6406.&43.&24600.&1.&2&30&20\_10 \textcolor{red}{\textcjheb{yk}} KJ $|$f"urwahr/wohl\\
2.&12.&6407.&45.&24602.&3.&3&272&2\_70\_200 \textcolor{red}{\textcjheb{r`b}} BaR $|$unvern"unftiger/dumm\\
3.&13.&6408.&48.&24605.&6.&4&81&1\_50\_20\_10 \textcolor{red}{\textcjheb{ykn'}} ANKJ $|$(bin) ich\\
4.&14.&6409.&52.&24609.&10.&4&351&40\_1\_10\_300 \textcolor{red}{\textcjheb{+sy'm}} MAJS $|$als irgend einer/(mehr als ein) Mann\\
5.&15.&6410.&56.&24613.&14.&3&37&6\_30\_1 \textcolor{red}{\textcjheb{'lw}} WLA $|$und nicht\\
6.&16.&6411.&59.&24616.&17.&4&462&2\_10\_50\_400 \textcolor{red}{\textcjheb{tnyb}} BJNT $|$Verstand/die Einsicht\\
7.&17.&6412.&63.&24620.&21.&3&45&1\_4\_40 \textcolor{red}{\textcjheb{md'}} ADM $|$(eines) Menschen\\
8.&18.&6413.&66.&24623.&24.&2&40&30\_10 \textcolor{red}{\textcjheb{yl}} LJ $|$habe ich/ich besitze\\
\end{tabular}\medskip \\
Ende des Verses 30.2\\
Verse: 853, Buchstaben: 25, 67, 24624, Totalwerte: 1318, 3579, 1736047\\
\\
F"urwahr, ich bin unvern"unftiger als irgend einer, und Menschenverstand habe ich nicht.\\
\newpage 
{\bf -- 30.3}\\
\medskip \\
\begin{tabular}{rrrrrrrrp{120mm}}
WV&WK&WB&ABK&ABB&ABV&AnzB&TW&Zahlencode \textcolor{red}{$\boldsymbol{Grundtext}$} Umschrift $|$"Ubersetzung(en)\\
1.&19.&6414.&68.&24625.&1.&3&37&6\_30\_1 \textcolor{red}{\textcjheb{'lw}} WLA $|$und nicht\\
2.&20.&6415.&71.&24628.&4.&5&484&30\_40\_4\_400\_10 \textcolor{red}{\textcjheb{ytdml}} LMDTJ $|$ich habe gelernt\\
3.&21.&6416.&76.&24633.&9.&4&73&8\_20\_40\_5 \textcolor{red}{\textcjheb{hmk.h}} CKMH $|$Weisheit\\
4.&22.&6417.&80.&24637.&13.&4&480&6\_4\_70\_400 \textcolor{red}{\textcjheb{t`dw}} WDaT $|$dass Erkenntnis/und Erkenntnis\\
5.&23.&6418.&84.&24641.&17.&5&454&100\_4\_300\_10\_40 \textcolor{red}{\textcjheb{my+sdq}} QDSJM $|$(des) Heiligen\\
6.&24.&6419.&89.&24646.&22.&3&75&1\_4\_70 \textcolor{red}{\textcjheb{`d'}} ADa $|$ich bes"a"se/ich wei"s\\
\end{tabular}\medskip \\
Ende des Verses 30.3\\
Verse: 854, Buchstaben: 24, 91, 24648, Totalwerte: 1603, 5182, 1737650\\
\\
Und Weisheit habe ich nicht gelernt, da"s ich Erkenntnis des Heiligen bes"a"se.\\
\newpage 
{\bf -- 30.4}\\
\medskip \\
\begin{tabular}{rrrrrrrrp{120mm}}
WV&WK&WB&ABK&ABB&ABV&AnzB&TW&Zahlencode \textcolor{red}{$\boldsymbol{Grundtext}$} Umschrift $|$"Ubersetzung(en)\\
1.&25.&6420.&92.&24649.&1.&2&50&40\_10 \textcolor{red}{\textcjheb{ym}} MJ $|$wer\\
2.&26.&6421.&94.&24651.&3.&3&105&70\_30\_5 \textcolor{red}{\textcjheb{hl`}} aLH $|$ist hinaufgestiegen gen/(er) stieg hinan\\
3.&27.&6422.&97.&24654.&6.&4&390&300\_40\_10\_40 \textcolor{red}{\textcjheb{mym+s}} SMJM $|$(die) Himmel\\
4.&28.&6423.&101.&24658.&10.&4&220&6\_10\_200\_4 \textcolor{red}{\textcjheb{dryw}} WJRD $|$und herniedergefahren/und (er) stieg hernieder\\
5.&29.&6424.&105.&24662.&14.&2&50&40\_10 \textcolor{red}{\textcjheb{ym}} MJ $|$wer\\
6.&30.&6425.&107.&24664.&16.&3&141&1\_60\_80 \textcolor{red}{\textcjheb{ps'}} AsP $|$hat gesammelt/(er) sammelte\\
7.&31.&6426.&110.&24667.&19.&3&214&200\_6\_8 \textcolor{red}{\textcjheb{.hwr}} RWC $|$(den) Wind\\
8.&32.&6427.&113.&24670.&22.&6&156&2\_8\_80\_50\_10\_6 \textcolor{red}{\textcjheb{wynp.hb}} BCPNJW $|$in seine F"auste/in seine Hohlh"ande\\
9.&33.&6428.&119.&24676.&28.&2&50&40\_10 \textcolor{red}{\textcjheb{ym}} MJ $|$wer\\
10.&34.&6429.&121.&24678.&30.&3&490&90\_200\_200 \textcolor{red}{\textcjheb{rr.s}} "sRR $|$gebunden/(er) band ein\\
11.&35.&6430.&124.&24681.&33.&3&90&40\_10\_40 \textcolor{red}{\textcjheb{mym}} MJM $|$(die) Wasser\\
12.&36.&6431.&127.&24684.&36.&5&377&2\_300\_40\_30\_5 \textcolor{red}{\textcjheb{hlm+sb}} BSMLH $|$in ein Tuch/in den Mantel\\
13.&37.&6432.&132.&24689.&41.&2&50&40\_10 \textcolor{red}{\textcjheb{ym}} MJ $|$wer\\
14.&38.&6433.&134.&24691.&43.&4&155&5\_100\_10\_40 \textcolor{red}{\textcjheb{myqh}} HQJM $|$hat aufgerichtet/(er) machte aufstehen\\
15.&39.&6434.&138.&24695.&47.&2&50&20\_30 \textcolor{red}{\textcjheb{lk}} KL $|$alle\\
16.&40.&6435.&140.&24697.&49.&4&151&1\_80\_60\_10 \textcolor{red}{\textcjheb{ysp'}} APsJ $|$Enden\\
17.&41.&6436.&144.&24701.&53.&3&291&1\_200\_90 \textcolor{red}{\textcjheb{.sr'}} AR"s $|$der Erde\\
18.&42.&6437.&147.&24704.&56.&2&45&40\_5 \textcolor{red}{\textcjheb{hm}} MH $|$was\\
19.&43.&6438.&149.&24706.&58.&3&346&300\_40\_6 \textcolor{red}{\textcjheb{wm+s}} SMW $|$(ist) sein Name\\
20.&44.&6439.&152.&24709.&61.&3&51&6\_40\_5 \textcolor{red}{\textcjheb{hmw}} WMH $|$und was\\
21.&45.&6440.&155.&24712.&64.&2&340&300\_40 \textcolor{red}{\textcjheb{m+s}} SM $|$der Name\\
22.&46.&6441.&157.&24714.&66.&3&58&2\_50\_6 \textcolor{red}{\textcjheb{wnb}} BNW $|$seines Sohnes\\
23.&47.&6442.&160.&24717.&69.&2&30&20\_10 \textcolor{red}{\textcjheb{yk}} KJ $|$wenn\\
24.&48.&6443.&162.&24719.&71.&3&474&400\_4\_70 \textcolor{red}{\textcjheb{`dt}} TDa $|$du (es) wei"st\\
\end{tabular}\medskip \\
Ende des Verses 30.4\\
Verse: 855, Buchstaben: 73, 164, 24721, Totalwerte: 4374, 9556, 1742024\\
\\
Wer ist hinaufgestiegen gen Himmel und herniedergefahren? Wer hat den Wind in seine F"auste gesammelt? Wer die Wasser in ein Tuch gebunden? Wer hat aufgerichtet alle Enden der Erde? Was ist sein Name, und was der Name seines Sohnes, wenn du es wei"st?\\
\newpage 
{\bf -- 30.5}\\
\medskip \\
\begin{tabular}{rrrrrrrrp{120mm}}
WV&WK&WB&ABK&ABB&ABV&AnzB&TW&Zahlencode \textcolor{red}{$\boldsymbol{Grundtext}$} Umschrift $|$"Ubersetzung(en)\\
1.&49.&6444.&165.&24722.&1.&2&50&20\_30 \textcolor{red}{\textcjheb{lk}} KL $|$alle/jegliches\\
2.&50.&6445.&167.&24724.&3.&4&641&1\_40\_200\_400 \textcolor{red}{\textcjheb{trm'}} AMRT $|$Rede/Wort\\
3.&51.&6446.&171.&24728.&7.&4&42&1\_30\_6\_5 \textcolor{red}{\textcjheb{hwl'}} ALWH $|$Gott(es)\\
4.&52.&6447.&175.&24732.&11.&5&381&90\_200\_6\_80\_5 \textcolor{red}{\textcjheb{hpwr.s}} "sRWPH $|$ist gel"autert/(ist) bew"ahrt\\
5.&53.&6448.&180.&24737.&16.&3&93&40\_3\_50 \textcolor{red}{\textcjheb{ngm}} MGN $|$(ein) Schild\\
6.&54.&6449.&183.&24740.&19.&3&12&5\_6\_1 \textcolor{red}{\textcjheb{'wh}} HWA $|$(ist) er\\
7.&55.&6450.&186.&24743.&22.&5&148&30\_8\_60\_10\_40 \textcolor{red}{\textcjheb{mys.hl}} LCsJM $|$denen die trauen/den sich Bergenden\\
8.&56.&6451.&191.&24748.&27.&2&8&2\_6 \textcolor{red}{\textcjheb{wb}} BW $|$auf ihn/bei ihm\\
\end{tabular}\medskip \\
Ende des Verses 30.5\\
Verse: 856, Buchstaben: 28, 192, 24749, Totalwerte: 1375, 10931, 1743399\\
\\
Alle Rede Gottes ist gel"autert; ein Schild ist er denen, die auf ihn trauen.\\
\newpage 
{\bf -- 30.6}\\
\medskip \\
\begin{tabular}{rrrrrrrrp{120mm}}
WV&WK&WB&ABK&ABB&ABV&AnzB&TW&Zahlencode \textcolor{red}{$\boldsymbol{Grundtext}$} Umschrift $|$"Ubersetzung(en)\\
1.&57.&6452.&193.&24750.&1.&2&31&1\_30 \textcolor{red}{\textcjheb{l'}} AL $|$nichts\\
2.&58.&6453.&195.&24752.&3.&4&546&400\_6\_60\_80 \textcolor{red}{\textcjheb{pswt}} TWsP $|$tue hinzu/du wirst hinzuf"ugen\\
3.&59.&6454.&199.&24756.&7.&2&100&70\_30 \textcolor{red}{\textcjheb{l`}} aL $|$zu/an\\
4.&60.&6455.&201.&24758.&9.&5&222&4\_2\_200\_10\_6 \textcolor{red}{\textcjheb{wyrbd}} DBRJW $|$seine(n) Worte(n)\\
5.&61.&6456.&206.&24763.&14.&2&130&80\_50 \textcolor{red}{\textcjheb{np}} PN $|$damit nicht/dass nicht\\
6.&62.&6457.&208.&24765.&16.&5&54&10\_6\_20\_10\_8 \textcolor{red}{\textcjheb{.hykwy}} JWKJC $|$er "uberf"uhre/er weist zurecht\\
7.&63.&6458.&213.&24770.&21.&2&22&2\_20 \textcolor{red}{\textcjheb{kb}} BK $|$dich\\
8.&64.&6459.&215.&24772.&23.&6&485&6\_50\_20\_7\_2\_400 \textcolor{red}{\textcjheb{tbzknw}} WNKZBT $|$und du als L"ugner erfunden werdest/und du erweist dich als l"ugnerisch\\
\end{tabular}\medskip \\
Ende des Verses 30.6\\
Verse: 857, Buchstaben: 28, 220, 24777, Totalwerte: 1590, 12521, 1744989\\
\\
Tue nichts zu seinen Worten hinzu, damit er dich nicht "uberf"uhre und du als L"ugner erfunden werdest.\\
\newpage 
{\bf -- 30.7}\\
\medskip \\
\begin{tabular}{rrrrrrrrp{120mm}}
WV&WK&WB&ABK&ABB&ABV&AnzB&TW&Zahlencode \textcolor{red}{$\boldsymbol{Grundtext}$} Umschrift $|$"Ubersetzung(en)\\
1.&65.&6460.&221.&24778.&1.&4&750&300\_400\_10\_40 \textcolor{red}{\textcjheb{myt+s}} STJM $|$zweierlei/zwei (Dinge)\\
2.&66.&6461.&225.&24782.&5.&5&741&300\_1\_30\_400\_10 \textcolor{red}{\textcjheb{ytl'+s}} SALTJ $|$erbitte ich/ich erbat\\
3.&67.&6462.&230.&24787.&10.&4&461&40\_1\_400\_20 \textcolor{red}{\textcjheb{kt'm}} MATK $|$von dir\\
4.&68.&6463.&234.&24791.&14.&2&31&1\_30 \textcolor{red}{\textcjheb{l'}} AL $|$nicht\\
5.&69.&6464.&236.&24793.&16.&4&560&400\_40\_50\_70 \textcolor{red}{\textcjheb{`nmt}} TMNa $|$verweigere es/versage es\\
6.&70.&6465.&240.&24797.&20.&4&140&40\_40\_50\_10 \textcolor{red}{\textcjheb{ynmm}} MMNJ $|$mir\\
7.&71.&6466.&244.&24801.&24.&4&251&2\_9\_200\_40 \textcolor{red}{\textcjheb{mr.tb}} BtRM $|$ehe/bevor\\
8.&72.&6467.&248.&24805.&28.&4&447&1\_40\_6\_400 \textcolor{red}{\textcjheb{twm'}} AMWT $|$ich sterbe\\
\end{tabular}\medskip \\
Ende des Verses 30.7\\
Verse: 858, Buchstaben: 31, 251, 24808, Totalwerte: 3381, 15902, 1748370\\
\\
Zweierlei erbitte ich von dir; verweigere es mir nicht, ehe ich sterbe:\\
\newpage 
{\bf -- 30.8}\\
\medskip \\
\begin{tabular}{rrrrrrrrp{120mm}}
WV&WK&WB&ABK&ABB&ABV&AnzB&TW&Zahlencode \textcolor{red}{$\boldsymbol{Grundtext}$} Umschrift $|$"Ubersetzung(en)\\
1.&73.&6468.&252.&24809.&1.&3&307&300\_6\_1 \textcolor{red}{\textcjheb{'w+s}} SWA $|$Eitles\\
2.&74.&6469.&255.&24812.&4.&4&212&6\_4\_2\_200 \textcolor{red}{\textcjheb{rbdw}} WDBR $|$und (ein) Wort\\
3.&75.&6470.&259.&24816.&8.&3&29&20\_7\_2 \textcolor{red}{\textcjheb{bzk}} KZB $|$(der) L"uge(n)\\
4.&76.&6471.&262.&24819.&11.&4&313&5\_200\_8\_100 \textcolor{red}{\textcjheb{q.hrh}} HRCQ $|$entferne/halte fern\\
5.&77.&6472.&266.&24823.&15.&4&140&40\_40\_50\_10 \textcolor{red}{\textcjheb{ynmm}} MMNJ $|$von mir\\
6.&78.&6473.&270.&24827.&19.&3&501&200\_1\_300 \textcolor{red}{\textcjheb{+s'r}} RAS $|$Armut\\
7.&79.&6474.&273.&24830.&22.&4&576&6\_70\_300\_200 \textcolor{red}{\textcjheb{r+s`w}} WaSR $|$und Reichtum\\
8.&80.&6475.&277.&24834.&26.&2&31&1\_30 \textcolor{red}{\textcjheb{l'}} AL $|$nicht\\
9.&81.&6476.&279.&24836.&28.&3&850&400\_400\_50 \textcolor{red}{\textcjheb{ntt}} TTN $|$gib/du m"ogest geben\\
10.&82.&6477.&282.&24839.&31.&2&40&30\_10 \textcolor{red}{\textcjheb{yl}} LJ $|$mir\\
11.&83.&6478.&284.&24841.&33.&7&364&5\_9\_200\_10\_80\_50\_10 \textcolor{red}{\textcjheb{ynpyr.th}} HtRJPNJ $|$speise mich mit dem/lass genie"sen mich\\
12.&84.&6479.&291.&24848.&40.&3&78&30\_8\_40 \textcolor{red}{\textcjheb{m.hl}} LCM $|$Brot\\
13.&85.&6480.&294.&24851.&43.&3&118&8\_100\_10 \textcolor{red}{\textcjheb{yq.h}} CQJ $|$mir beschiedenen/zugewiesen mir\\
\end{tabular}\medskip \\
Ende des Verses 30.8\\
Verse: 859, Buchstaben: 45, 296, 24853, Totalwerte: 3559, 19461, 1751929\\
\\
Eitles und L"ugenwort entferne von mir, Armut und Reichtum gib mir nicht, speise mich mit dem mir beschiedenen Brote;\\
\newpage 
{\bf -- 30.9}\\
\medskip \\
\begin{tabular}{rrrrrrrrp{120mm}}
WV&WK&WB&ABK&ABB&ABV&AnzB&TW&Zahlencode \textcolor{red}{$\boldsymbol{Grundtext}$} Umschrift $|$"Ubersetzung(en)\\
1.&86.&6481.&297.&24854.&1.&2&130&80\_50 \textcolor{red}{\textcjheb{np}} PN $|$damit nicht/dass nicht\\
2.&87.&6482.&299.&24856.&3.&4&373&1\_300\_2\_70 \textcolor{red}{\textcjheb{`b+s'}} ASBa $|$ich satt werde/ich satt bin\\
3.&88.&6483.&303.&24860.&7.&6&744&6\_20\_8\_300\_400\_10 \textcolor{red}{\textcjheb{yt+s.hkw}} WKCSTJ $|$und (ich) (ver)leugne\\
4.&89.&6484.&309.&24866.&13.&6&657&6\_1\_40\_200\_400\_10 \textcolor{red}{\textcjheb{ytrm'w}} WAMRTJ $|$und spreche/und ich sage\\
5.&90.&6485.&315.&24872.&19.&2&50&40\_10 \textcolor{red}{\textcjheb{ym}} MJ $|$wer (ist)\\
6.&91.&6486.&317.&24874.&21.&4&26&10\_5\_6\_5 \textcolor{red}{\textcjheb{hwhy}} JHWH $|$Jahwe\\
7.&92.&6487.&321.&24878.&25.&3&136&6\_80\_50 \textcolor{red}{\textcjheb{npw}} WPN $|$und damit nicht/und dass nicht\\
8.&93.&6488.&324.&24881.&28.&4&507&1\_6\_200\_300 \textcolor{red}{\textcjheb{+srw'}} AWRS $|$ich verarme\\
9.&94.&6489.&328.&24885.&32.&6&471&6\_3\_50\_2\_400\_10 \textcolor{red}{\textcjheb{ytbngw}} WGNBTJ $|$und (ich) stehle\\
10.&95.&6490.&334.&24891.&38.&6&1196&6\_400\_80\_300\_400\_10 \textcolor{red}{\textcjheb{yt+sptw}} WTPSTJ $|$und (ich vergreife) mich\\
11.&96.&6491.&340.&24897.&44.&2&340&300\_40 \textcolor{red}{\textcjheb{m+s}} SM $|$(am) Namen\\
12.&97.&6492.&342.&24899.&46.&4&46&1\_30\_5\_10 \textcolor{red}{\textcjheb{yhl'}} ALHJ $|$meines Gottes\\
\end{tabular}\medskip \\
Ende des Verses 30.9\\
Verse: 860, Buchstaben: 49, 345, 24902, Totalwerte: 4676, 24137, 1756605\\
\\
damit ich nicht satt werde und dich verleugne und spreche: Wer ist Jahwe? und damit ich nicht verarme und stehle, und mich vergreife an dem Namen meines Gottes.\\
\newpage 
{\bf -- 30.10}\\
\medskip \\
\begin{tabular}{rrrrrrrrp{120mm}}
WV&WK&WB&ABK&ABB&ABV&AnzB&TW&Zahlencode \textcolor{red}{$\boldsymbol{Grundtext}$} Umschrift $|$"Ubersetzung(en)\\
1.&98.&6493.&346.&24903.&1.&2&31&1\_30 \textcolor{red}{\textcjheb{l'}} AL $|$nicht\\
2.&99.&6494.&348.&24905.&3.&4&780&400\_30\_300\_50 \textcolor{red}{\textcjheb{n+slt}} TLSN $|$verleumde/du sollst verleumden\\
3.&100.&6495.&352.&24909.&7.&3&76&70\_2\_4 \textcolor{red}{\textcjheb{db`}} aBD $|$(einen) Knecht\\
4.&101.&6496.&355.&24912.&10.&2&31&1\_30 \textcolor{red}{\textcjheb{l'}} AL $|$bei\\
5.&102.&6497.&357.&24914.&12.&4&61&1\_4\_50\_6 \textcolor{red}{\textcjheb{wnd'}} ADNW $|$seinem Herrn\\
6.&103.&6498.&361.&24918.&16.&2&130&80\_50 \textcolor{red}{\textcjheb{np}} PN $|$damit nicht/dass nicht\\
7.&104.&6499.&363.&24920.&18.&5&190&10\_100\_30\_30\_20 \textcolor{red}{\textcjheb{kllqy}} JQLLK $|$er fluche dir/er verfluche dich\\
8.&105.&6500.&368.&24925.&23.&5&747&6\_1\_300\_40\_400 \textcolor{red}{\textcjheb{tm+s'w}} WASMT $|$und du m"usstest (es) b"u"sen\\
\end{tabular}\medskip \\
Ende des Verses 30.10\\
Verse: 861, Buchstaben: 27, 372, 24929, Totalwerte: 2046, 26183, 1758651\\
\\
Verleumde einen Knecht nicht bei seinem Herrn, damit er dir nicht fluche, und du es b"u"sen m"ussest.\\
\newpage 
{\bf -- 30.11}\\
\medskip \\
\begin{tabular}{rrrrrrrrp{120mm}}
WV&WK&WB&ABK&ABB&ABV&AnzB&TW&Zahlencode \textcolor{red}{$\boldsymbol{Grundtext}$} Umschrift $|$"Ubersetzung(en)\\
1.&106.&6501.&373.&24930.&1.&3&210&4\_6\_200 \textcolor{red}{\textcjheb{rwd}} DWR $|$(ein) Geschlecht\\
2.&107.&6502.&376.&24933.&4.&4&19&1\_2\_10\_6 \textcolor{red}{\textcjheb{wyb'}} ABJW $|$das seinem Vater/ seinen Vater\\
3.&108.&6503.&380.&24937.&8.&4&170&10\_100\_30\_30 \textcolor{red}{\textcjheb{llqy}} JQLL $|$(er (=es)) (ver)flucht\\
4.&109.&6504.&384.&24941.&12.&3&407&6\_1\_400 \textcolor{red}{\textcjheb{t'w}} WAT $|$und **\\
5.&110.&6505.&387.&24944.&15.&3&47&1\_40\_6 \textcolor{red}{\textcjheb{wm'}} AMW $|$seine Mutter\\
6.&111.&6506.&390.&24947.&18.&2&31&30\_1 \textcolor{red}{\textcjheb{'l}} LA $|$nicht\\
7.&112.&6507.&392.&24949.&20.&4&232&10\_2\_200\_20 \textcolor{red}{\textcjheb{krby}} JBRK $|$(er (=es)) segnet\\
\end{tabular}\medskip \\
Ende des Verses 30.11\\
Verse: 862, Buchstaben: 23, 395, 24952, Totalwerte: 1116, 27299, 1759767\\
\\
Ein Geschlecht, das seinem Vater flucht und seine Mutter nicht segnet;\\
\newpage 
{\bf -- 30.12}\\
\medskip \\
\begin{tabular}{rrrrrrrrp{120mm}}
WV&WK&WB&ABK&ABB&ABV&AnzB&TW&Zahlencode \textcolor{red}{$\boldsymbol{Grundtext}$} Umschrift $|$"Ubersetzung(en)\\
1.&113.&6508.&396.&24953.&1.&3&210&4\_6\_200 \textcolor{red}{\textcjheb{rwd}} DWR $|$(ein) Geschlecht\\
2.&114.&6509.&399.&24956.&4.&4&220&9\_5\_6\_200 \textcolor{red}{\textcjheb{rwh.t}} tHWR $|$das rein ist/(d"unkt sich) rein\\
3.&115.&6510.&403.&24960.&8.&6&148&2\_70\_10\_50\_10\_6 \textcolor{red}{\textcjheb{wyny`b}} BaJNJW $|$in seinen Augen\\
4.&116.&6511.&409.&24966.&14.&6&543&6\_40\_90\_1\_400\_6 \textcolor{red}{\textcjheb{wt'.smw}} WM"sATW $|$und von seinem Unflat\\
5.&117.&6512.&415.&24972.&20.&2&31&30\_1 \textcolor{red}{\textcjheb{'l}} LA $|$(doch) nicht\\
6.&118.&6513.&417.&24974.&22.&3&298&200\_8\_90 \textcolor{red}{\textcjheb{.s.hr}} RC"s $|$(er (=es) ist) gewaschen\\
\end{tabular}\medskip \\
Ende des Verses 30.12\\
Verse: 863, Buchstaben: 24, 419, 24976, Totalwerte: 1450, 28749, 1761217\\
\\
ein Geschlecht, das rein ist in seinen Augen und doch nicht gewaschen von seinem Unflat;\\
\newpage 
{\bf -- 30.13}\\
\medskip \\
\begin{tabular}{rrrrrrrrp{120mm}}
WV&WK&WB&ABK&ABB&ABV&AnzB&TW&Zahlencode \textcolor{red}{$\boldsymbol{Grundtext}$} Umschrift $|$"Ubersetzung(en)\\
1.&119.&6514.&420.&24977.&1.&3&210&4\_6\_200 \textcolor{red}{\textcjheb{rwd}} DWR $|$(ein) Geschlecht\\
2.&120.&6515.&423.&24980.&4.&2&45&40\_5 \textcolor{red}{\textcjheb{hm}} MH $|$wie\\
3.&121.&6516.&425.&24982.&6.&3&246&200\_40\_6 \textcolor{red}{\textcjheb{wmr}} RMW $|$stolz sind/sie (=es) erheben sich\\
4.&122.&6517.&428.&24985.&9.&5&146&70\_10\_50\_10\_6 \textcolor{red}{\textcjheb{wyny`}} aJNJW $|$seine Augen\\
5.&123.&6518.&433.&24990.&14.&7&322&6\_70\_80\_70\_80\_10\_6 \textcolor{red}{\textcjheb{wyp`p`w}} WaPaPJW $|$und seine Wimpern(reihen)\\
6.&124.&6519.&440.&24997.&21.&5&367&10\_50\_300\_1\_6 \textcolor{red}{\textcjheb{w'+sny}} JNSAW $|$(sie) erheben sich\\
\end{tabular}\medskip \\
Ende des Verses 30.13\\
Verse: 864, Buchstaben: 25, 444, 25001, Totalwerte: 1336, 30085, 1762553\\
\\
ein Geschlecht-wie stolz sind seine Augen, und seine Wimpern erheben sich! -\\
\newpage 
{\bf -- 30.14}\\
\medskip \\
\begin{tabular}{rrrrrrrrp{120mm}}
WV&WK&WB&ABK&ABB&ABV&AnzB&TW&Zahlencode \textcolor{red}{$\boldsymbol{Grundtext}$} Umschrift $|$"Ubersetzung(en)\\
1.&125.&6520.&445.&25002.&1.&3&210&4\_6\_200 \textcolor{red}{\textcjheb{rwd}} DWR $|$(ein) Geschlecht\\
2.&126.&6521.&448.&25005.&4.&5&616&8\_200\_2\_6\_400 \textcolor{red}{\textcjheb{twbr.h}} CRBWT $|$Schwerter\\
3.&127.&6522.&453.&25010.&9.&4&366&300\_50\_10\_6 \textcolor{red}{\textcjheb{wyn+s}} SNJW $|$sind dessen Z"ahne/(sind) seine Zahnreihen\\
4.&128.&6523.&457.&25014.&13.&7&503&6\_40\_1\_20\_30\_6\_400 \textcolor{red}{\textcjheb{twlk'mw}} WMAKLWT $|$und Messer\\
5.&129.&6524.&464.&25021.&20.&7&956&40\_400\_30\_70\_400\_10\_6 \textcolor{red}{\textcjheb{wyt`ltm}} MTLaTJW $|$sein(e) Gebiss(e)\\
6.&130.&6525.&471.&25028.&27.&4&81&30\_1\_20\_30 \textcolor{red}{\textcjheb{lk'l}} LAKL $|$(um) wegzufressen\\
7.&131.&6526.&475.&25032.&31.&5&180&70\_50\_10\_10\_40 \textcolor{red}{\textcjheb{myyn`}} aNJJM $|$(die) Elenden\\
8.&132.&6527.&480.&25037.&36.&4&331&40\_1\_200\_90 \textcolor{red}{\textcjheb{.sr'm}} MAR"s $|$von der Erde/aus dem Lande\\
9.&133.&6528.&484.&25041.&40.&8&125&6\_1\_2\_10\_6\_50\_10\_40 \textcolor{red}{\textcjheb{mynwyb'w}} WABJWNJM $|$und die D"urftigen/und Arme\\
10.&134.&6529.&492.&25049.&48.&4&85&40\_1\_4\_40 \textcolor{red}{\textcjheb{md'm}} MADM $|$aus der Menschen Mitte/aus der Menschheit\\
\end{tabular}\medskip \\
Ende des Verses 30.14\\
Verse: 865, Buchstaben: 51, 495, 25052, Totalwerte: 3453, 33538, 1766006\\
\\
ein Geschlecht, dessen Z"ahne Schwerter sind, und Messer sein Gebi"s, um wegzufressen die Elenden von der Erde und die D"urftigen aus der Menschen Mitte!\\
\newpage 
{\bf -- 30.15}\\
\medskip \\
\begin{tabular}{rrrrrrrrp{120mm}}
WV&WK&WB&ABK&ABB&ABV&AnzB&TW&Zahlencode \textcolor{red}{$\boldsymbol{Grundtext}$} Umschrift $|$"Ubersetzung(en)\\
1.&135.&6530.&496.&25053.&1.&6&241&30\_70\_30\_6\_100\_5 \textcolor{red}{\textcjheb{hqwl`l}} LaLWQH $|$der Blutegel hat/es hat der Blutsauger\\
2.&136.&6531.&502.&25059.&7.&3&710&300\_400\_10 \textcolor{red}{\textcjheb{yt+s}} STJ $|$zwei\\
3.&137.&6532.&505.&25062.&10.&4&458&2\_50\_6\_400 \textcolor{red}{\textcjheb{twnb}} BNWT $|$T"ochter\\
4.&138.&6533.&509.&25066.&14.&2&7&5\_2 \textcolor{red}{\textcjheb{bh}} HB $|$gib her\\
5.&139.&6534.&511.&25068.&16.&2&7&5\_2 \textcolor{red}{\textcjheb{bh}} HB $|$gib her\\
6.&140.&6535.&513.&25070.&18.&4&636&300\_30\_6\_300 \textcolor{red}{\textcjheb{+swl+s}} SLWS $|$drei\\
7.&141.&6536.&517.&25074.&22.&3&60&5\_50\_5 \textcolor{red}{\textcjheb{hnh}} HNH $|$sind es/sind diese\\
8.&142.&6537.&520.&25077.&25.&2&31&30\_1 \textcolor{red}{\textcjheb{'l}} LA $|$(die) nicht\\
9.&143.&6538.&522.&25079.&27.&6&827&400\_300\_2\_70\_50\_5 \textcolor{red}{\textcjheb{hn`b+st}} TSBaNH $|$(sie) werden satt\\
10.&144.&6539.&528.&25085.&33.&4&273&1\_200\_2\_70 \textcolor{red}{\textcjheb{`br'}} ARBa $|$vier\\
11.&145.&6540.&532.&25089.&37.&2&31&30\_1 \textcolor{red}{\textcjheb{'l}} LA $|$(die) nicht\\
12.&146.&6541.&534.&25091.&39.&4&247&1\_40\_200\_6 \textcolor{red}{\textcjheb{wrm'}} AMRW $|$sie sag(t)en\\
13.&147.&6542.&538.&25095.&43.&3&61&5\_6\_50 \textcolor{red}{\textcjheb{nwh}} HWN $|$genug\\
\end{tabular}\medskip \\
Ende des Verses 30.15\\
Verse: 866, Buchstaben: 45, 540, 25097, Totalwerte: 3589, 37127, 1769595\\
\\
Der Blutegel hat zwei T"ochter: gib her! gib her! Drei sind es, die nicht satt werden, vier, die nicht sagen: Genug!\\
\newpage 
{\bf -- 30.16}\\
\medskip \\
\begin{tabular}{rrrrrrrrp{120mm}}
WV&WK&WB&ABK&ABB&ABV&AnzB&TW&Zahlencode \textcolor{red}{$\boldsymbol{Grundtext}$} Umschrift $|$"Ubersetzung(en)\\
1.&148.&6543.&541.&25098.&1.&4&337&300\_1\_6\_30 \textcolor{red}{\textcjheb{lw'+s}} SAWL $|$der Scheol/(das) Totenreich\\
2.&149.&6544.&545.&25102.&5.&4&366&6\_70\_90\_200 \textcolor{red}{\textcjheb{r.s`w}} Wa"sR $|$und der verschlossene/und die Verschlossenheit\\
3.&150.&6545.&549.&25106.&9.&3&248&200\_8\_40 \textcolor{red}{\textcjheb{m.hr}} RCM $|$Mutterleib/(des) Mutterscho"ses\\
4.&151.&6546.&552.&25109.&12.&3&291&1\_200\_90 \textcolor{red}{\textcjheb{.sr'}} AR"s $|$die Erde\\
5.&152.&6547.&555.&25112.&15.&2&31&30\_1 \textcolor{red}{\textcjheb{'l}} LA $|$welche nicht/(die) nicht\\
6.&153.&6548.&557.&25114.&17.&4&377&300\_2\_70\_5 \textcolor{red}{\textcjheb{h`b+s}} SBaH $|$satt wird/(sie) ist satt\\
7.&154.&6549.&561.&25118.&21.&3&90&40\_10\_40 \textcolor{red}{\textcjheb{mym}} MJM $|$des Wassers\\
8.&155.&6550.&564.&25121.&24.&3&307&6\_1\_300 \textcolor{red}{\textcjheb{+s'w}} WAS $|$und das Feuer\\
9.&156.&6551.&567.&25124.&27.&2&31&30\_1 \textcolor{red}{\textcjheb{'l}} LA $|$(das) nicht\\
10.&157.&6552.&569.&25126.&29.&4&246&1\_40\_200\_5 \textcolor{red}{\textcjheb{hrm'}} AMRH $|$(sie (=es)) sagt(e)\\
11.&158.&6553.&573.&25130.&33.&3&61&5\_6\_50 \textcolor{red}{\textcjheb{nwh}} HWN $|$genug\\
\end{tabular}\medskip \\
Ende des Verses 30.16\\
Verse: 867, Buchstaben: 35, 575, 25132, Totalwerte: 2385, 39512, 1771980\\
\\
Der Scheol und der verschlossene Mutterleib, die Erde, welche des Wassers nicht satt wird, und das Feuer, das nicht sagt: Genug!\\
\newpage 
{\bf -- 30.17}\\
\medskip \\
\begin{tabular}{rrrrrrrrp{120mm}}
WV&WK&WB&ABK&ABB&ABV&AnzB&TW&Zahlencode \textcolor{red}{$\boldsymbol{Grundtext}$} Umschrift $|$"Ubersetzung(en)\\
1.&159.&6554.&576.&25133.&1.&3&130&70\_10\_50 \textcolor{red}{\textcjheb{ny`}} aJN $|$(ein) Auge\\
2.&160.&6555.&579.&25136.&4.&4&503&400\_30\_70\_3 \textcolor{red}{\textcjheb{g`lt}} TLaG $|$das verspottet\\
3.&161.&6556.&583.&25140.&8.&3&33&30\_1\_2 \textcolor{red}{\textcjheb{b'l}} LAB $|$den Vater\\
4.&162.&6557.&586.&25143.&11.&5&421&6\_400\_2\_6\_7 \textcolor{red}{\textcjheb{zwbtw}} WTBWZ $|$und verachtet/und sie (=es) missachtet\\
5.&163.&6558.&591.&25148.&16.&5&545&30\_10\_100\_5\_400 \textcolor{red}{\textcjheb{thqyl}} LJQHT $|$den Gehorsam\\
6.&164.&6559.&596.&25153.&21.&2&41&1\_40 \textcolor{red}{\textcjheb{m'}} AM $|$gegen die Mutter/(gegen"uber) der Mutter\\
7.&165.&6560.&598.&25155.&23.&5&321&10\_100\_200\_6\_5 \textcolor{red}{\textcjheb{hwrqy}} JQRWH $|$das werden aushacken/sie (=es) sollen aushacken sie (=es)\\
8.&166.&6561.&603.&25160.&28.&4&282&70\_200\_2\_10 \textcolor{red}{\textcjheb{ybr`}} aRBJ $|$die Raben\\
9.&167.&6562.&607.&25164.&32.&3&88&50\_8\_30 \textcolor{red}{\textcjheb{l.hn}} NCL $|$des Baches/am Bach\\
10.&168.&6563.&610.&25167.&35.&7&78&6\_10\_1\_20\_30\_6\_5 \textcolor{red}{\textcjheb{hwlk'yw}} WJAKLWH $|$und (sie (=es) sollen) fressen (sie (=es))\\
11.&169.&6564.&617.&25174.&42.&3&62&2\_50\_10 \textcolor{red}{\textcjheb{ynb}} BNJ $|$die Jungen/die S"ohne\\
12.&170.&6565.&620.&25177.&45.&3&550&50\_300\_200 \textcolor{red}{\textcjheb{r+sn}} NSR $|$des Adlers\\
\end{tabular}\medskip \\
Ende des Verses 30.17\\
Verse: 868, Buchstaben: 47, 622, 25179, Totalwerte: 3054, 42566, 1775034\\
\\
Ein Auge, das den Vater verspottet und den Gehorsam gegen die Mutter verachtet, das werden die Raben des Baches aushacken und die Jungen des Adlers fressen.\\
\newpage 
{\bf -- 30.18}\\
\medskip \\
\begin{tabular}{rrrrrrrrp{120mm}}
WV&WK&WB&ABK&ABB&ABV&AnzB&TW&Zahlencode \textcolor{red}{$\boldsymbol{Grundtext}$} Umschrift $|$"Ubersetzung(en)\\
1.&171.&6566.&623.&25180.&1.&4&635&300\_30\_300\_5 \textcolor{red}{\textcjheb{h+sl+s}} SLSH $|$drei (Dinge)\\
2.&172.&6567.&627.&25184.&5.&3&50&5\_40\_5 \textcolor{red}{\textcjheb{hmh}} HMH $|$sind es/sie (sind)\\
3.&173.&6568.&630.&25187.&8.&5&167&50\_80\_30\_1\_6 \textcolor{red}{\textcjheb{w'lpn}} NPLAW $|$(die sind) zu wunderbar\\
4.&174.&6569.&635.&25192.&13.&4&136&40\_40\_50\_6 \textcolor{red}{\textcjheb{wnmm}} MMNW $|$f"ur mich\\
5.&175.&6570.&639.&25196.&17.&5&279&6\_1\_200\_2\_70 \textcolor{red}{\textcjheb{`br'w}} WARBa $|$und vier\\
6.&176.&6571.&644.&25201.&22.&2&31&30\_1 \textcolor{red}{\textcjheb{'l}} LA $|$(die) nicht\\
7.&177.&6572.&646.&25203.&24.&6&534&10\_4\_70\_400\_10\_40 \textcolor{red}{\textcjheb{myt`dy}} JDaTJM $|$ich (er)kenne (sie)\\
\end{tabular}\medskip \\
Ende des Verses 30.18\\
Verse: 869, Buchstaben: 29, 651, 25208, Totalwerte: 1832, 44398, 1776866\\
\\
Drei sind es, die zu wunderbar f"ur mich sind, und vier, die ich nicht erkenne:\\
\newpage 
{\bf -- 30.19}\\
\medskip \\
\begin{tabular}{rrrrrrrrp{120mm}}
WV&WK&WB&ABK&ABB&ABV&AnzB&TW&Zahlencode \textcolor{red}{$\boldsymbol{Grundtext}$} Umschrift $|$"Ubersetzung(en)\\
1.&178.&6573.&652.&25209.&1.&3&224&4\_200\_20 \textcolor{red}{\textcjheb{krd}} DRK $|$der Weg/den Weg\\
2.&179.&6574.&655.&25212.&4.&4&555&5\_50\_300\_200 \textcolor{red}{\textcjheb{r+snh}} HNSR $|$des Adlers\\
3.&180.&6575.&659.&25216.&8.&5&392&2\_300\_40\_10\_40 \textcolor{red}{\textcjheb{mym+sb}} BSMJM $|$am Himmel/an den Himmeln\\
4.&181.&6576.&664.&25221.&13.&3&224&4\_200\_20 \textcolor{red}{\textcjheb{krd}} DRK $|$der Weg/den Weg\\
5.&182.&6577.&667.&25224.&16.&3&358&50\_8\_300 \textcolor{red}{\textcjheb{+s.hn}} NCS $|$einer Schlange/der Schlange\\
6.&183.&6578.&670.&25227.&19.&3&110&70\_30\_10 \textcolor{red}{\textcjheb{yl`}} aLJ $|$auf/"uber\\
7.&184.&6579.&673.&25230.&22.&3&296&90\_6\_200 \textcolor{red}{\textcjheb{rw.s}} "sWR $|$dem Felsen/den Felsen\\
8.&185.&6580.&676.&25233.&25.&3&224&4\_200\_20 \textcolor{red}{\textcjheb{krd}} DRK $|$der Weg/den Weg\\
9.&186.&6581.&679.&25236.&28.&4&66&1\_50\_10\_5 \textcolor{red}{\textcjheb{hyn'}} ANJH $|$eines Schiffes/des Schiffes\\
10.&187.&6582.&683.&25240.&32.&3&34&2\_30\_2 \textcolor{red}{\textcjheb{blb}} BLB $|$im Herzen\\
11.&188.&6583.&686.&25243.&35.&2&50&10\_40 \textcolor{red}{\textcjheb{my}} JM $|$des Meeres\\
12.&189.&6584.&688.&25245.&37.&4&230&6\_4\_200\_20 \textcolor{red}{\textcjheb{krdw}} WDRK $|$und der Weg/und den Weg\\
13.&190.&6585.&692.&25249.&41.&3&205&3\_2\_200 \textcolor{red}{\textcjheb{rbg}} GBR $|$eines Mannes/des Mannes\\
14.&191.&6586.&695.&25252.&44.&5&147&2\_70\_30\_40\_5 \textcolor{red}{\textcjheb{hml`b}} BaLMH $|$mit einer Jungfrau/beim M"adchen\\
\end{tabular}\medskip \\
Ende des Verses 30.19\\
Verse: 870, Buchstaben: 48, 699, 25256, Totalwerte: 3115, 47513, 1779981\\
\\
der Weg des Adlers am Himmel, der Weg einer Schlange auf dem Felsen, der Weg eines Schiffes im Herzen des Meeres, und der Weg eines Mannes mit einer Jungfrau. -\\
\newpage 
{\bf -- 30.20}\\
\medskip \\
\begin{tabular}{rrrrrrrrp{120mm}}
WV&WK&WB&ABK&ABB&ABV&AnzB&TW&Zahlencode \textcolor{red}{$\boldsymbol{Grundtext}$} Umschrift $|$"Ubersetzung(en)\\
1.&192.&6587.&700.&25257.&1.&2&70&20\_50 \textcolor{red}{\textcjheb{nk}} KN $|$(al)so (ist)\\
2.&193.&6588.&702.&25259.&3.&3&224&4\_200\_20 \textcolor{red}{\textcjheb{krd}} DRK $|$der Weg\\
3.&194.&6589.&705.&25262.&6.&3&306&1\_300\_5 \textcolor{red}{\textcjheb{h+s'}} ASH $|$(einer) Frau\\
4.&195.&6590.&708.&25265.&9.&5&571&40\_50\_1\_80\_400 \textcolor{red}{\textcjheb{tp'nm}} MNAPT $|$ehebrecherischen\\
5.&196.&6591.&713.&25270.&14.&4&56&1\_20\_30\_5 \textcolor{red}{\textcjheb{hlk'}} AKLH $|$sie isst\\
6.&197.&6592.&717.&25274.&18.&5&459&6\_40\_8\_400\_5 \textcolor{red}{\textcjheb{ht.hmw}} WMCTH $|$und (sie) wischt ab\\
7.&198.&6593.&722.&25279.&23.&3&95&80\_10\_5 \textcolor{red}{\textcjheb{hyp}} PJH $|$ihren Mund\\
8.&199.&6594.&725.&25282.&26.&5&252&6\_1\_40\_200\_5 \textcolor{red}{\textcjheb{hrm'w}} WAMRH $|$und (sie) spricht\\
9.&200.&6595.&730.&25287.&31.&2&31&30\_1 \textcolor{red}{\textcjheb{'l}} LA $|$nicht\\
10.&201.&6596.&732.&25289.&33.&5&590&80\_70\_30\_400\_10 \textcolor{red}{\textcjheb{ytl`p}} PaLTJ $|$habe ich begangen/ich habe getan\\
11.&202.&6597.&737.&25294.&38.&3&57&1\_6\_50 \textcolor{red}{\textcjheb{nw'}} AWN $|$ein Unrecht/S"unde\\
\end{tabular}\medskip \\
Ende des Verses 30.20\\
Verse: 871, Buchstaben: 40, 739, 25296, Totalwerte: 2711, 50224, 1782692\\
\\
Also ist der Weg eines ehebrecherischen Weibes: sie i"st, und wischt ihren Mund und spricht: Ich habe kein Unrecht begangen.\\
\newpage 
{\bf -- 30.21}\\
\medskip \\
\begin{tabular}{rrrrrrrrp{120mm}}
WV&WK&WB&ABK&ABB&ABV&AnzB&TW&Zahlencode \textcolor{red}{$\boldsymbol{Grundtext}$} Umschrift $|$"Ubersetzung(en)\\
1.&203.&6598.&740.&25297.&1.&3&808&400\_8\_400 \textcolor{red}{\textcjheb{t.ht}} TCT $|$unter\\
2.&204.&6599.&743.&25300.&4.&4&636&300\_30\_6\_300 \textcolor{red}{\textcjheb{+swl+s}} SLWS $|$Dreien\\
3.&205.&6600.&747.&25304.&8.&4&215&200\_3\_7\_5 \textcolor{red}{\textcjheb{hzgr}} RGZH $|$erzittert/sie (=es) erbebt(e)\\
4.&206.&6601.&751.&25308.&12.&3&291&1\_200\_90 \textcolor{red}{\textcjheb{.sr'}} AR"s $|$die Erde\\
5.&207.&6602.&754.&25311.&15.&4&814&6\_400\_8\_400 \textcolor{red}{\textcjheb{t.htw}} WTCT $|$und unter\\
6.&208.&6603.&758.&25315.&19.&4&273&1\_200\_2\_70 \textcolor{red}{\textcjheb{`br'}} ARBa $|$Vieren\\
7.&209.&6604.&762.&25319.&23.&2&31&30\_1 \textcolor{red}{\textcjheb{'l}} LA $|$nicht\\
8.&210.&6605.&764.&25321.&25.&4&456&400\_6\_20\_30 \textcolor{red}{\textcjheb{lkwt}} TWKL $|$sie kann\\
9.&211.&6606.&768.&25325.&29.&3&701&300\_1\_400 \textcolor{red}{\textcjheb{t'+s}} SAT $|$es aushalten/standhalten\\
\end{tabular}\medskip \\
Ende des Verses 30.21\\
Verse: 872, Buchstaben: 31, 770, 25327, Totalwerte: 4225, 54449, 1786917\\
\\
Unter dreien erzittert die Erde, und unter vieren kann sie es nicht aushalten:\\
\newpage 
{\bf -- 30.22}\\
\medskip \\
\begin{tabular}{rrrrrrrrp{120mm}}
WV&WK&WB&ABK&ABB&ABV&AnzB&TW&Zahlencode \textcolor{red}{$\boldsymbol{Grundtext}$} Umschrift $|$"Ubersetzung(en)\\
1.&212.&6607.&771.&25328.&1.&3&808&400\_8\_400 \textcolor{red}{\textcjheb{t.ht}} TCT $|$unter\\
2.&213.&6608.&774.&25331.&4.&3&76&70\_2\_4 \textcolor{red}{\textcjheb{db`}} aBD $|$(einem) Knecht\\
3.&214.&6609.&777.&25334.&7.&2&30&20\_10 \textcolor{red}{\textcjheb{yk}} KJ $|$wenn\\
4.&215.&6610.&779.&25336.&9.&5&106&10\_40\_30\_6\_20 \textcolor{red}{\textcjheb{kwlmy}} JMLWK $|$er wird K"onig\\
5.&216.&6611.&784.&25341.&14.&4&88&6\_50\_2\_30 \textcolor{red}{\textcjheb{lbnw}} WNBL $|$und einem gemeinen Menschen/und einem Toren\\
6.&217.&6612.&788.&25345.&18.&2&30&20\_10 \textcolor{red}{\textcjheb{yk}} KJ $|$wenn\\
7.&218.&6613.&790.&25347.&20.&4&382&10\_300\_2\_70 \textcolor{red}{\textcjheb{`b+sy}} JSBa $|$er satt ist/er ist satt\\
8.&219.&6614.&794.&25351.&24.&3&78&30\_8\_40 \textcolor{red}{\textcjheb{m.hl}} LCM $|$(an) Brot\\
\end{tabular}\medskip \\
Ende des Verses 30.22\\
Verse: 873, Buchstaben: 26, 796, 25353, Totalwerte: 1598, 56047, 1788515\\
\\
unter einem Knechte, wenn er K"onig wird, und einem gemeinen Menschen, wenn er satt Brot hat;\\
\newpage 
{\bf -- 30.23}\\
\medskip \\
\begin{tabular}{rrrrrrrrp{120mm}}
WV&WK&WB&ABK&ABB&ABV&AnzB&TW&Zahlencode \textcolor{red}{$\boldsymbol{Grundtext}$} Umschrift $|$"Ubersetzung(en)\\
1.&220.&6615.&797.&25354.&1.&3&808&400\_8\_400 \textcolor{red}{\textcjheb{t.ht}} TCT $|$unter\\
2.&221.&6616.&800.&25357.&4.&5&362&300\_50\_6\_1\_5 \textcolor{red}{\textcjheb{h'wn+s}} SNWAH $|$einer unleidlichen Frau/einer Verschm"ahten\\
3.&222.&6617.&805.&25362.&9.&2&30&20\_10 \textcolor{red}{\textcjheb{yk}} KJ $|$wenn\\
4.&223.&6618.&807.&25364.&11.&4&502&400\_2\_70\_30 \textcolor{red}{\textcjheb{l`bt}} TBaL $|$sie zur (Ehe)Frau genommen wird\\
5.&224.&6619.&811.&25368.&15.&5&399&6\_300\_80\_8\_5 \textcolor{red}{\textcjheb{h.hp+sw}} WSPCH $|$und einer Magd\\
6.&225.&6620.&816.&25373.&20.&2&30&20\_10 \textcolor{red}{\textcjheb{yk}} KJ $|$wenn\\
7.&226.&6621.&818.&25375.&22.&4&910&400\_10\_200\_300 \textcolor{red}{\textcjheb{+sryt}} TJRS $|$sie beerbt/sie vom Besitz vertreibt\\
8.&227.&6622.&822.&25379.&26.&5&610&3\_2\_200\_400\_5 \textcolor{red}{\textcjheb{htrbg}} GBRTH $|$ihre Herrin\\
\end{tabular}\medskip \\
Ende des Verses 30.23\\
Verse: 874, Buchstaben: 30, 826, 25383, Totalwerte: 3651, 59698, 1792166\\
\\
unter einem unleidlichen Weibe, wenn sie zur Frau genommen wird, und einer Magd, wenn sie ihre Herrin beerbt.\\
\newpage 
{\bf -- 30.24}\\
\medskip \\
\begin{tabular}{rrrrrrrrp{120mm}}
WV&WK&WB&ABK&ABB&ABV&AnzB&TW&Zahlencode \textcolor{red}{$\boldsymbol{Grundtext}$} Umschrift $|$"Ubersetzung(en)\\
1.&228.&6623.&827.&25384.&1.&5&278&1\_200\_2\_70\_5 \textcolor{red}{\textcjheb{h`br'}} ARBaH $|$vier\\
2.&229.&6624.&832.&25389.&6.&2&45&5\_40 \textcolor{red}{\textcjheb{mh}} HM $|$sind die/(sind) sie\\
3.&230.&6625.&834.&25391.&8.&4&169&100\_9\_50\_10 \textcolor{red}{\textcjheb{yn.tq}} QtNJ $|$Kleine(n)\\
4.&231.&6626.&838.&25395.&12.&3&291&1\_200\_90 \textcolor{red}{\textcjheb{.sr'}} AR"s $|$(auf) der Erde\\
5.&232.&6627.&841.&25398.&15.&4&56&6\_5\_40\_5 \textcolor{red}{\textcjheb{hmhw}} WHMH $|$und doch sind sie/und sie (sind)\\
6.&233.&6628.&845.&25402.&19.&5&118&8\_20\_40\_10\_40 \textcolor{red}{\textcjheb{mymk.h}} CKMJM $|$wohl/weise\\
7.&234.&6629.&850.&25407.&24.&6&158&40\_8\_20\_40\_10\_40 \textcolor{red}{\textcjheb{mymk.hm}} MCKMJM $|$mit Weiheit versehen/unter den Weisen\\
\end{tabular}\medskip \\
Ende des Verses 30.24\\
Verse: 875, Buchstaben: 29, 855, 25412, Totalwerte: 1115, 60813, 1793281\\
\\
Vier sind die Kleinen der Erde, und doch sind sie mit Weisheit wohl versehen:\\
\newpage 
{\bf -- 30.25}\\
\medskip \\
\begin{tabular}{rrrrrrrrp{120mm}}
WV&WK&WB&ABK&ABB&ABV&AnzB&TW&Zahlencode \textcolor{red}{$\boldsymbol{Grundtext}$} Umschrift $|$"Ubersetzung(en)\\
1.&235.&6630.&856.&25413.&1.&6&175&5\_50\_40\_30\_10\_40 \textcolor{red}{\textcjheb{mylmnh}} HNMLJM $|$die Ameisen\\
2.&236.&6631.&862.&25419.&7.&2&110&70\_40 \textcolor{red}{\textcjheb{m`}} aM $|$(ein) Volk\\
3.&237.&6632.&864.&25421.&9.&2&31&30\_1 \textcolor{red}{\textcjheb{'l}} LA $|$nicht/ohne\\
4.&238.&6633.&866.&25423.&11.&2&77&70\_7 \textcolor{red}{\textcjheb{z`}} aZ $|$starkes/Kraft\\
5.&239.&6634.&868.&25425.&13.&6&102&6\_10\_20\_10\_50\_6 \textcolor{red}{\textcjheb{wnykyw}} WJKJNW $|$und (doch) sie bereiten\\
6.&240.&6635.&874.&25431.&19.&4&202&2\_100\_10\_90 \textcolor{red}{\textcjheb{.syqb}} BQJ"s $|$im Sommer\\
7.&241.&6636.&878.&25435.&23.&4&118&30\_8\_40\_40 \textcolor{red}{\textcjheb{mm.hl}} LCMM $|$ihre Speise/ihre Nahrung\\
\end{tabular}\medskip \\
Ende des Verses 30.25\\
Verse: 876, Buchstaben: 26, 881, 25438, Totalwerte: 815, 61628, 1794096\\
\\
die Ameisen, ein nicht starkes Volk, und doch bereiten sie im Sommer ihre Speise;\\
\newpage 
{\bf -- 30.26}\\
\medskip \\
\begin{tabular}{rrrrrrrrp{120mm}}
WV&WK&WB&ABK&ABB&ABV&AnzB&TW&Zahlencode \textcolor{red}{$\boldsymbol{Grundtext}$} Umschrift $|$"Ubersetzung(en)\\
1.&242.&6637.&882.&25439.&1.&5&480&300\_80\_50\_10\_40 \textcolor{red}{\textcjheb{mynp+s}} SPNJM $|$die Klippend"achse/Klippdachse\\
2.&243.&6638.&887.&25444.&6.&2&110&70\_40 \textcolor{red}{\textcjheb{m`}} aM $|$(sind) (ein) Volk\\
3.&244.&6639.&889.&25446.&8.&2&31&30\_1 \textcolor{red}{\textcjheb{'l}} LA $|$nicht\\
4.&245.&6640.&891.&25448.&10.&4&206&70\_90\_6\_40 \textcolor{red}{\textcjheb{mw.s`}} a"sWM $|$kr"aftiges/stark(es)\\
5.&246.&6641.&895.&25452.&14.&6&372&6\_10\_300\_10\_40\_6 \textcolor{red}{\textcjheb{wmy+syw}} WJSJMW $|$und (doch) sie setz(t)en\\
6.&247.&6642.&901.&25458.&20.&4&162&2\_60\_30\_70 \textcolor{red}{\textcjheb{`lsb}} BsLa $|$auf den Felsen/in das Felsgestein\\
7.&248.&6643.&905.&25462.&24.&4&452&2\_10\_400\_40 \textcolor{red}{\textcjheb{mtyb}} BJTM $|$ihr Haus/ihre Behausung\\
\end{tabular}\medskip \\
Ende des Verses 30.26\\
Verse: 877, Buchstaben: 27, 908, 25465, Totalwerte: 1813, 63441, 1795909\\
\\
die Klippend"achse, ein nicht kr"aftiges Volk, und doch setzen sie ihr Haus auf den Felsen;\\
\newpage 
{\bf -- 30.27}\\
\medskip \\
\begin{tabular}{rrrrrrrrp{120mm}}
WV&WK&WB&ABK&ABB&ABV&AnzB&TW&Zahlencode \textcolor{red}{$\boldsymbol{Grundtext}$} Umschrift $|$"Ubersetzung(en)\\
1.&249.&6644.&909.&25466.&1.&3&90&40\_30\_20 \textcolor{red}{\textcjheb{klm}} MLK $|$(einen) K"onig\\
2.&250.&6645.&912.&25469.&4.&3&61&1\_10\_50 \textcolor{red}{\textcjheb{ny'}} AJN $|$nicht haben/nicht gibt es\\
3.&251.&6646.&915.&25472.&7.&5&238&30\_1\_200\_2\_5 \textcolor{red}{\textcjheb{hbr'l}} LARBH $|$die Heuschrecken/(f"ur den) Heuschreck (enschwarm)\\
4.&252.&6647.&920.&25477.&12.&4&107&6\_10\_90\_1 \textcolor{red}{\textcjheb{'.syw}} WJ"sA $|$und doch ziehen sie aus/und er (=es) zieht aus\\
5.&253.&6648.&924.&25481.&16.&3&188&8\_90\_90 \textcolor{red}{\textcjheb{.s.s.h}} C"s"s $|$in geordneten Scharen/wohlgeordnet\\
6.&254.&6649.&927.&25484.&19.&3&56&20\_30\_6 \textcolor{red}{\textcjheb{wlk}} KLW $|$allesamt/jeder von ihm\\
\end{tabular}\medskip \\
Ende des Verses 30.27\\
Verse: 878, Buchstaben: 21, 929, 25486, Totalwerte: 740, 64181, 1796649\\
\\
die Heuschrecken haben keinen K"onig, und doch ziehen sie allesamt aus in geordneten Scharen;\\
\newpage 
{\bf -- 30.28}\\
\medskip \\
\begin{tabular}{rrrrrrrrp{120mm}}
WV&WK&WB&ABK&ABB&ABV&AnzB&TW&Zahlencode \textcolor{red}{$\boldsymbol{Grundtext}$} Umschrift $|$"Ubersetzung(en)\\
1.&255.&6650.&930.&25487.&1.&5&790&300\_40\_40\_10\_400 \textcolor{red}{\textcjheb{tymm+s}} SMMJT $|$die Eidechse/(eine) Eidechse\\
2.&256.&6651.&935.&25492.&6.&5&66&2\_10\_4\_10\_40 \textcolor{red}{\textcjheb{mydyb}} BJDJM $|$mit (zwei) H"anden\\
3.&257.&6652.&940.&25497.&11.&4&1180&400\_400\_80\_300 \textcolor{red}{\textcjheb{+sptt}} TTPS $|$kannst du fangen/du kannst greifen\\
4.&258.&6653.&944.&25501.&15.&4&22&6\_5\_10\_1 \textcolor{red}{\textcjheb{'yhw}} WHJA $|$und (doch ist) sie\\
5.&259.&6654.&948.&25505.&19.&6&77&2\_5\_10\_20\_30\_10 \textcolor{red}{\textcjheb{ylkyhb}} BHJKLJ $|$in den Pal"asten\\
6.&260.&6655.&954.&25511.&25.&3&90&40\_30\_20 \textcolor{red}{\textcjheb{klm}} MLK $|$der K"onige/des K"onigs\\
\end{tabular}\medskip \\
Ende des Verses 30.28\\
Verse: 879, Buchstaben: 27, 956, 25513, Totalwerte: 2225, 66406, 1798874\\
\\
die Eidechse kannst du mit H"anden fangen, und doch ist sie in den Pal"asten der K"onige.\\
\newpage 
{\bf -- 30.29}\\
\medskip \\
\begin{tabular}{rrrrrrrrp{120mm}}
WV&WK&WB&ABK&ABB&ABV&AnzB&TW&Zahlencode \textcolor{red}{$\boldsymbol{Grundtext}$} Umschrift $|$"Ubersetzung(en)\\
1.&261.&6656.&957.&25514.&1.&4&635&300\_30\_300\_5 \textcolor{red}{\textcjheb{h+sl+s}} SLSH $|$drei\\
2.&262.&6657.&961.&25518.&5.&3&50&5\_40\_5 \textcolor{red}{\textcjheb{hmh}} HMH $|$haben/sie sind\\
3.&263.&6658.&964.&25521.&8.&6&81&40\_10\_9\_10\_2\_10 \textcolor{red}{\textcjheb{yby.tym}} MJtJBJ $|$einen stattlichen/ansehnlich\\
4.&264.&6659.&970.&25527.&14.&3&164&90\_70\_4 \textcolor{red}{\textcjheb{d`.s}} "saD $|$(mit) Schritt\\
5.&265.&6660.&973.&25530.&17.&6&284&6\_1\_200\_2\_70\_5 \textcolor{red}{\textcjheb{h`br'w}} WARBaH $|$und vier\\
6.&266.&6661.&979.&25536.&23.&5&71&40\_10\_9\_2\_10 \textcolor{red}{\textcjheb{yb.tym}} MJtBJ $|$einen stattlichen/gutmachend\\
7.&267.&6662.&984.&25541.&28.&3&450&30\_20\_400 \textcolor{red}{\textcjheb{tkl}} LKT $|$Gang/das Gehen\\
\end{tabular}\medskip \\
Ende des Verses 30.29\\
Verse: 880, Buchstaben: 30, 986, 25543, Totalwerte: 1735, 68141, 1800609\\
\\
Drei haben einen stattlichen Schritt, und vier einen stattlichen Gang:\\
\newpage 
{\bf -- 30.30}\\
\medskip \\
\begin{tabular}{rrrrrrrrp{120mm}}
WV&WK&WB&ABK&ABB&ABV&AnzB&TW&Zahlencode \textcolor{red}{$\boldsymbol{Grundtext}$} Umschrift $|$"Ubersetzung(en)\\
1.&268.&6663.&987.&25544.&1.&3&340&30\_10\_300 \textcolor{red}{\textcjheb{+syl}} LJS $|$(der) L"owe\\
2.&269.&6664.&990.&25547.&4.&4&211&3\_2\_6\_200 \textcolor{red}{\textcjheb{rwbg}} GBWR $|$der Held/(der) Starke\\
3.&270.&6665.&994.&25551.&8.&5&54&2\_2\_5\_40\_5 \textcolor{red}{\textcjheb{hmhbb}} BBHMH $|$unter den Tieren\\
4.&271.&6666.&999.&25556.&13.&3&37&6\_30\_1 \textcolor{red}{\textcjheb{'lw}} WLA $|$und (vor) nicht(s)\\
5.&272.&6667.&1002.&25559.&16.&4&318&10\_300\_6\_2 \textcolor{red}{\textcjheb{bw+sy}} JSWB $|$der zur"uckweicht/(er) kehrt um\\
6.&273.&6668.&1006.&25563.&20.&4&180&40\_80\_50\_10 \textcolor{red}{\textcjheb{ynpm}} MPNJ $|$/vor\\
7.&274.&6669.&1010.&25567.&24.&2&50&20\_30 \textcolor{red}{\textcjheb{lk}} KL $|$/irgendeinem\\
\end{tabular}\medskip \\
Ende des Verses 30.30\\
Verse: 881, Buchstaben: 25, 1011, 25568, Totalwerte: 1190, 69331, 1801799\\
\\
der L"owe, der Held unter den Tieren, und der vor nichts zur"uckweicht;\\
\newpage 
{\bf -- 30.31}\\
\medskip \\
\begin{tabular}{rrrrrrrrp{120mm}}
WV&WK&WB&ABK&ABB&ABV&AnzB&TW&Zahlencode \textcolor{red}{$\boldsymbol{Grundtext}$} Umschrift $|$"Ubersetzung(en)\\
1.&275.&6670.&1012.&25569.&1.&5&424&7\_200\_7\_10\_200 \textcolor{red}{\textcjheb{ryzrz}} ZRZJR $|$der Straffe/ein Geschn"urter\\
2.&276.&6671.&1017.&25574.&6.&5&540&40\_400\_50\_10\_40 \textcolor{red}{\textcjheb{myntm}} MTNJM $|$(an) Lenden\\
3.&277.&6672.&1022.&25579.&11.&2&7&1\_6 \textcolor{red}{\textcjheb{w'}} AW $|$oder\\
4.&278.&6673.&1024.&25581.&13.&3&710&400\_10\_300 \textcolor{red}{\textcjheb{+syt}} TJS $|$der Bock/ein Ziegenbock\\
5.&279.&6674.&1027.&25584.&16.&4&96&6\_40\_30\_20 \textcolor{red}{\textcjheb{klmw}} WMLK $|$und (ein) K"onig\\
6.&280.&6675.&1031.&25588.&20.&5&177&1\_30\_100\_6\_40 \textcolor{red}{\textcjheb{mwql'}} ALQWM $|$bei welchem ist Heerbann/kein Bestehen\\
7.&281.&6676.&1036.&25593.&25.&3&116&70\_40\_6 \textcolor{red}{\textcjheb{wm`}} aMW $|$/(gibt es) vor ihm\\
\end{tabular}\medskip \\
Ende des Verses 30.31\\
Verse: 882, Buchstaben: 27, 1038, 25595, Totalwerte: 2070, 71401, 1803869\\
\\
der Lendenstraffe, oder der Bock; und ein K"onig, bei welchem der Heerbann ist.\\
\newpage 
{\bf -- 30.32}\\
\medskip \\
\begin{tabular}{rrrrrrrrp{120mm}}
WV&WK&WB&ABK&ABB&ABV&AnzB&TW&Zahlencode \textcolor{red}{$\boldsymbol{Grundtext}$} Umschrift $|$"Ubersetzung(en)\\
1.&282.&6677.&1039.&25596.&1.&2&41&1\_40 \textcolor{red}{\textcjheb{m'}} AM $|$wenn\\
2.&283.&6678.&1041.&25598.&3.&4&482&50\_2\_30\_400 \textcolor{red}{\textcjheb{tlbn}} NBLT $|$du t"oricht gehandelt hast/du warst t"oricht\\
3.&284.&6679.&1045.&25602.&7.&6&758&2\_5\_400\_50\_300\_1 \textcolor{red}{\textcjheb{'+snthb}} BHTNSA $|$indem du dich erhobst/durch ein Sich "Uberheben\\
4.&285.&6680.&1051.&25608.&13.&3&47&6\_1\_40 \textcolor{red}{\textcjheb{m'w}} WAM $|$oder wenn/und wenn\\
5.&286.&6681.&1054.&25611.&16.&4&453&7\_40\_6\_400 \textcolor{red}{\textcjheb{twmz}} ZMWT $|$du B"oses ersonnen/du reiflich nachdenkst\\
6.&287.&6682.&1058.&25615.&20.&2&14&10\_4 \textcolor{red}{\textcjheb{dy}} JD $|$die Hand\\
7.&288.&6683.&1060.&25617.&22.&3&115&30\_80\_5 \textcolor{red}{\textcjheb{hpl}} LPH $|$auf den Mund\\
\end{tabular}\medskip \\
Ende des Verses 30.32\\
Verse: 883, Buchstaben: 24, 1062, 25619, Totalwerte: 1910, 73311, 1805779\\
\\
Wenn du t"oricht gehandelt hast, indem du dich erhobst, oder wenn du B"oses ersonnen: die Hand auf den Mund!\\
\newpage 
{\bf -- 30.33}\\
\medskip \\
\begin{tabular}{rrrrrrrrp{120mm}}
WV&WK&WB&ABK&ABB&ABV&AnzB&TW&Zahlencode \textcolor{red}{$\boldsymbol{Grundtext}$} Umschrift $|$"Ubersetzung(en)\\
1.&289.&6684.&1063.&25620.&1.&2&30&20\_10 \textcolor{red}{\textcjheb{yk}} KJ $|$denn\\
2.&290.&6685.&1065.&25622.&3.&3&140&40\_10\_90 \textcolor{red}{\textcjheb{.sym}} MJ"s $|$das Pressen/ein Pressen\\
3.&291.&6686.&1068.&25625.&6.&3&40&8\_30\_2 \textcolor{red}{\textcjheb{bl.h}} CLB $|$(der) Milch\\
4.&292.&6687.&1071.&25628.&9.&5&117&10\_6\_90\_10\_1 \textcolor{red}{\textcjheb{'y.swy}} JW"sJA $|$ergibt/er (=es) bringt hervor\\
5.&293.&6688.&1076.&25633.&14.&4&54&8\_40\_1\_5 \textcolor{red}{\textcjheb{h'm.h}} CMAH $|$Butter\\
6.&294.&6689.&1080.&25637.&18.&4&146&6\_40\_10\_90 \textcolor{red}{\textcjheb{.symw}} WMJ"s $|$und das Pressen/und ein Pressen\\
7.&295.&6690.&1084.&25641.&22.&2&81&1\_80 \textcolor{red}{\textcjheb{p'}} AP $|$(der) Nase\\
8.&296.&6691.&1086.&25643.&24.&5&117&10\_6\_90\_10\_1 \textcolor{red}{\textcjheb{'y.swy}} JW"sJA $|$ergibt/er (=es) bringt hervor\\
9.&297.&6692.&1091.&25648.&29.&2&44&4\_40 \textcolor{red}{\textcjheb{md}} DM $|$Blut\\
10.&298.&6693.&1093.&25650.&31.&4&146&6\_40\_10\_90 \textcolor{red}{\textcjheb{.symw}} WMJ"s $|$und das Pressen/und ein Pressen\\
11.&299.&6694.&1097.&25654.&35.&4&131&1\_80\_10\_40 \textcolor{red}{\textcjheb{myp'}} APJM $|$(des) Zornes\\
12.&300.&6695.&1101.&25658.&39.&5&117&10\_6\_90\_10\_1 \textcolor{red}{\textcjheb{'y.swy}} JW"sJA $|$ergibt/er (=es) bringt hervor\\
13.&301.&6696.&1106.&25663.&44.&3&212&200\_10\_2 \textcolor{red}{\textcjheb{byr}} RJB $|$Hader/Streit\\
\end{tabular}\medskip \\
Ende des Verses 30.33\\
Verse: 884, Buchstaben: 46, 1108, 25665, Totalwerte: 1375, 74686, 1807154\\
\\
Denn das Pressen der Milch ergibt Butter, und das Pressen der Nase ergibt Blut, und das Pressen des Zornes ergibt Hader.\\
\\
{\bf Ende des Kapitels 30}\\
\newpage 
{\bf -- 31.1}\\
\medskip \\
\begin{tabular}{rrrrrrrrp{120mm}}
WV&WK&WB&ABK&ABB&ABV&AnzB&TW&Zahlencode \textcolor{red}{$\boldsymbol{Grundtext}$} Umschrift $|$"Ubersetzung(en)\\
1.&1.&6697.&1.&25666.&1.&4&216&4\_2\_200\_10 \textcolor{red}{\textcjheb{yrbd}} DBRJ $|$Worte\\
2.&2.&6698.&5.&25670.&5.&5&107&30\_40\_6\_1\_30 \textcolor{red}{\textcjheb{l'wml}} LMWAL $|$(an) Lemuel(s)///$<$zu Gott hin$>$\\
3.&3.&6699.&10.&25675.&10.&3&90&40\_30\_20 \textcolor{red}{\textcjheb{klm}} MLK $|$des K"onigs\\
4.&4.&6700.&13.&25678.&13.&3&341&40\_300\_1 \textcolor{red}{\textcjheb{'+sm}} MSA $|$Ausspruch/(von) Massa\\
5.&5.&6701.&16.&25681.&16.&3&501&1\_300\_200 \textcolor{red}{\textcjheb{r+s'}} ASR $|$womit/(mit) welchen\\
6.&6.&6702.&19.&25684.&19.&5&676&10\_60\_200\_400\_6 \textcolor{red}{\textcjheb{wtrsy}} JsRTW $|$ihn unterwies/sie (=es) ermahnte ihn\\
7.&7.&6703.&24.&25689.&24.&3&47&1\_40\_6 \textcolor{red}{\textcjheb{wm'}} AMW $|$seine Mutter\\
\end{tabular}\medskip \\
Ende des Verses 31.1\\
Verse: 885, Buchstaben: 26, 26, 25691, Totalwerte: 1978, 1978, 1809132\\
\\
Worte Lemuels, des K"onigs; Ausspruch, womit seine Mutter ihn unterwies:\\
\newpage 
{\bf -- 31.2}\\
\medskip \\
\begin{tabular}{rrrrrrrrp{120mm}}
WV&WK&WB&ABK&ABB&ABV&AnzB&TW&Zahlencode \textcolor{red}{$\boldsymbol{Grundtext}$} Umschrift $|$"Ubersetzung(en)\\
1.&8.&6704.&27.&25692.&1.&2&45&40\_5 \textcolor{red}{\textcjheb{hm}} MH $|$was (raten dir)\\
2.&9.&6705.&29.&25694.&3.&3&212&2\_200\_10 \textcolor{red}{\textcjheb{yrb}} BRJ $|$mein Sohn\\
3.&10.&6706.&32.&25697.&6.&3&51&6\_40\_5 \textcolor{red}{\textcjheb{hmw}} WMH $|$und was\\
4.&11.&6707.&35.&25700.&9.&2&202&2\_200 \textcolor{red}{\textcjheb{rb}} BR $|$(dem) Sohn\\
5.&12.&6708.&37.&25702.&11.&4&71&2\_9\_50\_10 \textcolor{red}{\textcjheb{yn.tb}} BtNJ $|$meines Leibes\\
6.&13.&6709.&41.&25706.&15.&3&51&6\_40\_5 \textcolor{red}{\textcjheb{hmw}} WMH $|$und was\\
7.&14.&6710.&44.&25709.&18.&2&202&2\_200 \textcolor{red}{\textcjheb{rb}} BR $|$Sohn\\
8.&15.&6711.&46.&25711.&20.&4&264&50\_4\_200\_10 \textcolor{red}{\textcjheb{yrdn}} NDRJ $|$meiner Gel"ubde\\
\end{tabular}\medskip \\
Ende des Verses 31.2\\
Verse: 886, Buchstaben: 23, 49, 25714, Totalwerte: 1098, 3076, 1810230\\
\\
Was, mein Sohn, und was, Sohn meines Leibes, und was, Sohn meiner Gel"ubde?\\
\newpage 
{\bf -- 31.3}\\
\medskip \\
\begin{tabular}{rrrrrrrrp{120mm}}
WV&WK&WB&ABK&ABB&ABV&AnzB&TW&Zahlencode \textcolor{red}{$\boldsymbol{Grundtext}$} Umschrift $|$"Ubersetzung(en)\\
1.&16.&6712.&50.&25715.&1.&2&31&1\_30 \textcolor{red}{\textcjheb{l'}} AL $|$nicht\\
2.&17.&6713.&52.&25717.&3.&3&850&400\_400\_50 \textcolor{red}{\textcjheb{ntt}} TTN $|$gib/du sollst geben\\
3.&18.&6714.&55.&25720.&6.&5&430&30\_50\_300\_10\_40 \textcolor{red}{\textcjheb{my+snl}} LNSJM $|$den Frauen/an Frauen\\
4.&19.&6715.&60.&25725.&11.&4&68&8\_10\_30\_20 \textcolor{red}{\textcjheb{kly.h}} CJLK $|$deine Kraft\\
5.&20.&6716.&64.&25729.&15.&6&260&6\_4\_200\_20\_10\_20 \textcolor{red}{\textcjheb{kykrdw}} WDRKJK $|$noch deine Wege/und deine Wege\\
6.&21.&6717.&70.&25735.&21.&5&484&30\_40\_8\_6\_400 \textcolor{red}{\textcjheb{tw.hml}} LMCWT $|$den Verderberinnen/zum Ausmerzen\\
7.&22.&6718.&75.&25740.&26.&5&150&40\_30\_20\_10\_50 \textcolor{red}{\textcjheb{nyklm}} MLKJN $|$(der) K"onige\\
\end{tabular}\medskip \\
Ende des Verses 31.3\\
Verse: 887, Buchstaben: 30, 79, 25744, Totalwerte: 2273, 5349, 1812503\\
\\
Gib nicht den Weibern deine Kraft, noch deine Wege den Verderberinnen der K"onige.\\
\newpage 
{\bf -- 31.4}\\
\medskip \\
\begin{tabular}{rrrrrrrrp{120mm}}
WV&WK&WB&ABK&ABB&ABV&AnzB&TW&Zahlencode \textcolor{red}{$\boldsymbol{Grundtext}$} Umschrift $|$"Ubersetzung(en)\\
1.&23.&6719.&80.&25745.&1.&2&31&1\_30 \textcolor{red}{\textcjheb{l'}} AL $|$nicht\\
2.&24.&6720.&82.&25747.&3.&6&170&30\_40\_30\_20\_10\_40 \textcolor{red}{\textcjheb{myklml}} LMLKJM $|$(es ziemt sich) f"ur K"onige\\
3.&25.&6721.&88.&25753.&9.&5&107&30\_40\_6\_1\_30 \textcolor{red}{\textcjheb{l'wml}} LMWAL $|$Lemuel\\
4.&26.&6722.&93.&25758.&14.&2&31&1\_30 \textcolor{red}{\textcjheb{l'}} AL $|$nicht\\
5.&27.&6723.&95.&25760.&16.&6&170&30\_40\_30\_20\_10\_40 \textcolor{red}{\textcjheb{myklml}} LMLKJM $|$f"ur K"onige\\
6.&28.&6724.&101.&25766.&22.&3&706&300\_400\_6 \textcolor{red}{\textcjheb{wt+s}} STW $|$zu trinken\\
7.&29.&6725.&104.&25769.&25.&3&70&10\_10\_50 \textcolor{red}{\textcjheb{nyy}} JJN $|$Wein\\
8.&30.&6726.&107.&25772.&28.&8&349&6\_30\_200\_6\_7\_50\_10\_40 \textcolor{red}{\textcjheb{mynzwrlw}} WLRWZNJM $|$noch f"ur F"ursten/und den F"ursten\\
9.&31.&6727.&115.&25780.&36.&2&7&1\_6 \textcolor{red}{\textcjheb{w'}} AW $|$(zu fragen) wo ist/nicht\\
10.&32.&6728.&117.&25782.&38.&3&520&300\_20\_200 \textcolor{red}{\textcjheb{rk+s}} SKR $|$starkes Getr"ank/Rauschtrank\\
\end{tabular}\medskip \\
Ende des Verses 31.4\\
Verse: 888, Buchstaben: 40, 119, 25784, Totalwerte: 2161, 7510, 1814664\\
\\
Nicht f"ur K"onige ziemt es sich, Lemuel, nicht f"ur K"onige, Wein zu trinken, noch f"ur F"ursten, zu fragen: Wo ist starkes Getr"ank?\\
\newpage 
{\bf -- 31.5}\\
\medskip \\
\begin{tabular}{rrrrrrrrp{120mm}}
WV&WK&WB&ABK&ABB&ABV&AnzB&TW&Zahlencode \textcolor{red}{$\boldsymbol{Grundtext}$} Umschrift $|$"Ubersetzung(en)\\
1.&33.&6729.&120.&25785.&1.&2&130&80\_50 \textcolor{red}{\textcjheb{np}} PN $|$damit nicht/dass nicht\\
2.&34.&6730.&122.&25787.&3.&4&715&10\_300\_400\_5 \textcolor{red}{\textcjheb{ht+sy}} JSTH $|$er trinke\\
3.&35.&6731.&126.&25791.&7.&5&344&6\_10\_300\_20\_8 \textcolor{red}{\textcjheb{.hk+syw}} WJSKC $|$und vergesse\\
4.&36.&6732.&131.&25796.&12.&4&248&40\_8\_100\_100 \textcolor{red}{\textcjheb{qq.hm}} MCQQ $|$des Vorgeschriebenen/Festgesetztes\\
5.&37.&6733.&135.&25800.&16.&5&371&6\_10\_300\_50\_5 \textcolor{red}{\textcjheb{hn+syw}} WJSNH $|$und verdrehe\\
6.&38.&6734.&140.&25805.&21.&3&64&4\_10\_50 \textcolor{red}{\textcjheb{nyd}} DJN $|$die Rechtssache/(eine) Recht(ssache)\\
7.&39.&6735.&143.&25808.&24.&2&50&20\_30 \textcolor{red}{\textcjheb{lk}} KL $|$aller\\
8.&40.&6736.&145.&25810.&26.&3&62&2\_50\_10 \textcolor{red}{\textcjheb{ynb}} BNJ $|$Kinder/S"ohne\\
9.&41.&6737.&148.&25813.&29.&3&130&70\_50\_10 \textcolor{red}{\textcjheb{yn`}} aNJ $|$des Elends\\
\end{tabular}\medskip \\
Ende des Verses 31.5\\
Verse: 889, Buchstaben: 31, 150, 25815, Totalwerte: 2114, 9624, 1816778\\
\\
damit er nicht trinke und des Vorgeschriebenen vergesse, und verdrehe die Rechtssache aller Kinder des Elends. -\\
\newpage 
{\bf -- 31.6}\\
\medskip \\
\begin{tabular}{rrrrrrrrp{120mm}}
WV&WK&WB&ABK&ABB&ABV&AnzB&TW&Zahlencode \textcolor{red}{$\boldsymbol{Grundtext}$} Umschrift $|$"Ubersetzung(en)\\
1.&42.&6738.&151.&25816.&1.&3&456&400\_50\_6 \textcolor{red}{\textcjheb{wnt}} TNW $|$gebt\\
2.&43.&6739.&154.&25819.&4.&3&520&300\_20\_200 \textcolor{red}{\textcjheb{rk+s}} SKR $|$starkes Getr"ank/Rauschtrank\\
3.&44.&6740.&157.&25822.&7.&5&43&30\_1\_6\_2\_4 \textcolor{red}{\textcjheb{dbw'l}} LAWBD $|$dem Umkommenden/dem Dahinsiechenden\\
4.&45.&6741.&162.&25827.&12.&4&76&6\_10\_10\_50 \textcolor{red}{\textcjheb{nyyw}} WJJN $|$und Wein\\
5.&46.&6742.&166.&25831.&16.&4&280&30\_40\_200\_10 \textcolor{red}{\textcjheb{yrml}} LMRJ $|$denen die sind betr"ubter/den Bitteren\\
6.&47.&6743.&170.&25835.&20.&3&430&50\_80\_300 \textcolor{red}{\textcjheb{+spn}} NPS $|$(der) Seele\\
\end{tabular}\medskip \\
Ende des Verses 31.6\\
Verse: 890, Buchstaben: 22, 172, 25837, Totalwerte: 1805, 11429, 1818583\\
\\
Gebet starkes Getr"ank dem Umkommenden, und Wein denen, die betr"ubter Seele sind:\\
\newpage 
{\bf -- 31.7}\\
\medskip \\
\begin{tabular}{rrrrrrrrp{120mm}}
WV&WK&WB&ABK&ABB&ABV&AnzB&TW&Zahlencode \textcolor{red}{$\boldsymbol{Grundtext}$} Umschrift $|$"Ubersetzung(en)\\
1.&48.&6744.&173.&25838.&1.&4&715&10\_300\_400\_5 \textcolor{red}{\textcjheb{ht+sy}} JSTH $|$(d)er (m"oge) trinke(n)\\
2.&49.&6745.&177.&25842.&5.&5&344&6\_10\_300\_20\_8 \textcolor{red}{\textcjheb{.hk+syw}} WJSKC $|$und (er) (m"oge) vergesse(n)\\
3.&50.&6746.&182.&25847.&10.&4&516&200\_10\_300\_6 \textcolor{red}{\textcjheb{w+syr}} RJSW $|$seine Armut\\
4.&51.&6747.&186.&25851.&14.&5&152&6\_70\_40\_30\_6 \textcolor{red}{\textcjheb{wlm`w}} WaMLW $|$und seine(r) M"uhsal\\
5.&52.&6748.&191.&25856.&19.&2&31&30\_1 \textcolor{red}{\textcjheb{'l}} LA $|$nicht\\
6.&53.&6749.&193.&25858.&21.&4&237&10\_7\_20\_200 \textcolor{red}{\textcjheb{rkzy}} JZKR $|$(er m"oge) gedenke(n)\\
7.&54.&6750.&197.&25862.&25.&3&80&70\_6\_4 \textcolor{red}{\textcjheb{dw`}} aWD $|$mehr\\
\end{tabular}\medskip \\
Ende des Verses 31.7\\
Verse: 891, Buchstaben: 27, 199, 25864, Totalwerte: 2075, 13504, 1820658\\
\\
er trinke, und vergesse seine Armut und gedenke seiner M"uhsal nicht mehr.\\
\newpage 
{\bf -- 31.8}\\
\medskip \\
\begin{tabular}{rrrrrrrrp{120mm}}
WV&WK&WB&ABK&ABB&ABV&AnzB&TW&Zahlencode \textcolor{red}{$\boldsymbol{Grundtext}$} Umschrift $|$"Ubersetzung(en)\\
1.&55.&6751.&200.&25865.&1.&3&488&80\_400\_8 \textcolor{red}{\textcjheb{.htp}} PTC $|$tue auf/"offne\\
2.&56.&6752.&203.&25868.&4.&3&110&80\_10\_20 \textcolor{red}{\textcjheb{kyp}} PJK $|$deinen Mund\\
3.&57.&6753.&206.&25871.&7.&4&101&30\_1\_30\_40 \textcolor{red}{\textcjheb{ml'l}} LALM $|$f"ur den Stummen\\
4.&58.&6754.&210.&25875.&11.&2&31&1\_30 \textcolor{red}{\textcjheb{l'}} AL $|$f"ur\\
5.&59.&6755.&212.&25877.&13.&3&64&4\_10\_50 \textcolor{red}{\textcjheb{nyd}} DJN $|$die Rechtssache/(eine) Rechtssache\\
6.&60.&6756.&215.&25880.&16.&2&50&20\_30 \textcolor{red}{\textcjheb{lk}} KL $|$aller\\
7.&61.&6757.&217.&25882.&18.&3&62&2\_50\_10 \textcolor{red}{\textcjheb{ynb}} BNJ $|$/S"ohne\\
8.&62.&6758.&220.&25885.&21.&4&124&8\_30\_6\_80 \textcolor{red}{\textcjheb{pwl.h}} CLWP $|$Ungl"ucklichen/des Dahinschwindens\\
\end{tabular}\medskip \\
Ende des Verses 31.8\\
Verse: 892, Buchstaben: 24, 223, 25888, Totalwerte: 1030, 14534, 1821688\\
\\
Tue deinen Mund auf f"ur den Stummen, f"ur die Rechtssache aller Ungl"ucklichen.\\
\newpage 
{\bf -- 31.9}\\
\medskip \\
\begin{tabular}{rrrrrrrrp{120mm}}
WV&WK&WB&ABK&ABB&ABV&AnzB&TW&Zahlencode \textcolor{red}{$\boldsymbol{Grundtext}$} Umschrift $|$"Ubersetzung(en)\\
1.&63.&6759.&224.&25889.&1.&3&488&80\_400\_8 \textcolor{red}{\textcjheb{.htp}} PTC $|$tue auf/"offne\\
2.&64.&6760.&227.&25892.&4.&3&110&80\_10\_20 \textcolor{red}{\textcjheb{kyp}} PJK $|$deinen Mund\\
3.&65.&6761.&230.&25895.&7.&3&389&300\_80\_9 \textcolor{red}{\textcjheb{.tp+s}} SPt $|$richte\\
4.&66.&6762.&233.&25898.&10.&3&194&90\_4\_100 \textcolor{red}{\textcjheb{qd.s}} "sDQ $|$gerecht/(nach) Gerechtigkeit\\
5.&67.&6763.&236.&25901.&13.&4&70&6\_4\_10\_50 \textcolor{red}{\textcjheb{nydw}} WDJN $|$und schaffe Recht\\
6.&68.&6764.&240.&25905.&17.&3&130&70\_50\_10 \textcolor{red}{\textcjheb{yn`}} aNJ $|$dem Elenden/(dem) Bedr"uckten\\
7.&69.&6765.&243.&25908.&20.&6&75&6\_1\_2\_10\_6\_50 \textcolor{red}{\textcjheb{nwyb'w}} WABJWN $|$und dem D"urftigen/und Elenden\\
\end{tabular}\medskip \\
Ende des Verses 31.9\\
Verse: 893, Buchstaben: 25, 248, 25913, Totalwerte: 1456, 15990, 1823144\\
\\
Tue deinen Mund auf, richte gerecht, und schaffe Recht dem Elenden und dem D"urftigen.\\
\newpage 
{\bf -- 31.10}\\
\medskip \\
\begin{tabular}{rrrrrrrrp{120mm}}
WV&WK&WB&ABK&ABB&ABV&AnzB&TW&Zahlencode \textcolor{red}{$\boldsymbol{Grundtext}$} Umschrift $|$"Ubersetzung(en)\\
1.&70.&6766.&249.&25914.&1.&3&701&1\_300\_400 \textcolor{red}{\textcjheb{t+s'}} AST $|$(eine) Frau\\
2.&71.&6767.&252.&25917.&4.&3&48&8\_10\_30 \textcolor{red}{\textcjheb{ly.h}} CJL $|$wackere/t"uchtige\\
3.&72.&6768.&255.&25920.&7.&2&50&40\_10 \textcolor{red}{\textcjheb{ym}} MJ $|$wer\\
4.&73.&6769.&257.&25922.&9.&4&141&10\_40\_90\_1 \textcolor{red}{\textcjheb{'.smy}} JM"sA $|$wird sie finden/(er) findet (sie)\\
5.&74.&6770.&261.&25926.&13.&4&314&6\_200\_8\_100 \textcolor{red}{\textcjheb{q.hrw}} WRCQ $|$denn weit/und weit\\
6.&75.&6771.&265.&25930.&17.&7&280&40\_80\_50\_10\_50\_10\_40 \textcolor{red}{\textcjheb{mynynpm}} MPNJNJM $|$"uber Korallen/mehr als Korallen\\
7.&76.&6772.&272.&25937.&24.&4&265&40\_20\_200\_5 \textcolor{red}{\textcjheb{hrkm}} MKRH $|$steht ihr Wert/(ist) ihr Kaufpreis\\
\end{tabular}\medskip \\
Ende des Verses 31.10\\
Verse: 894, Buchstaben: 27, 275, 25940, Totalwerte: 1799, 17789, 1824943\\
\\
Ein wackeres Weib, wer wird es finden? Denn ihr Wert steht weit "uber Korallen.\\
\newpage 
{\bf -- 31.11}\\
\medskip \\
\begin{tabular}{rrrrrrrrp{120mm}}
WV&WK&WB&ABK&ABB&ABV&AnzB&TW&Zahlencode \textcolor{red}{$\boldsymbol{Grundtext}$} Umschrift $|$"Ubersetzung(en)\\
1.&77.&6773.&276.&25941.&1.&3&19&2\_9\_8 \textcolor{red}{\textcjheb{.h.tb}} BtC $|$(er (=es)) vertraut(e)\\
2.&78.&6774.&279.&25944.&4.&2&7&2\_5 \textcolor{red}{\textcjheb{hb}} BH $|$auf sie\\
3.&79.&6775.&281.&25946.&6.&2&32&30\_2 \textcolor{red}{\textcjheb{bl}} LB $|$das Herz\\
4.&80.&6776.&283.&25948.&8.&4&107&2\_70\_30\_5 \textcolor{red}{\textcjheb{hl`b}} BaLH $|$ihres Mannes/ihres Eheherrn\\
5.&81.&6777.&287.&25952.&12.&4&366&6\_300\_30\_30 \textcolor{red}{\textcjheb{ll+sw}} WSLL $|$und an Ausbeute/und des Gewinnes\\
6.&82.&6778.&291.&25956.&16.&2&31&30\_1 \textcolor{red}{\textcjheb{'l}} LA $|$nicht\\
7.&83.&6779.&293.&25958.&18.&4&278&10\_8\_60\_200 \textcolor{red}{\textcjheb{rs.hy}} JCsR $|$wird es ihm fehlen/er (er)mangelt\\
\end{tabular}\medskip \\
Ende des Verses 31.11\\
Verse: 895, Buchstaben: 21, 296, 25961, Totalwerte: 840, 18629, 1825783\\
\\
Das Herz ihres Mannes vertraut auf sie, und an Ausbeute wird es ihm nicht fehlen.\\
\newpage 
{\bf -- 31.12}\\
\medskip \\
\begin{tabular}{rrrrrrrrp{120mm}}
WV&WK&WB&ABK&ABB&ABV&AnzB&TW&Zahlencode \textcolor{red}{$\boldsymbol{Grundtext}$} Umschrift $|$"Ubersetzung(en)\\
1.&84.&6780.&297.&25962.&1.&6&484&3\_40\_30\_400\_5\_6 \textcolor{red}{\textcjheb{whtlmg}} GMLTHW $|$sie erweist ihm/sie behandelt(e) ihn\\
2.&85.&6781.&303.&25968.&7.&3&17&9\_6\_2 \textcolor{red}{\textcjheb{bw.t}} tWB $|$Gutes/(mit) Gutem\\
3.&86.&6782.&306.&25971.&10.&3&37&6\_30\_1 \textcolor{red}{\textcjheb{'lw}} WLA $|$und nicht(s)\\
4.&87.&6783.&309.&25974.&13.&2&270&200\_70 \textcolor{red}{\textcjheb{`r}} Ra $|$B"oses/(mit) B"osem\\
5.&88.&6784.&311.&25976.&15.&2&50&20\_30 \textcolor{red}{\textcjheb{lk}} KL $|$alle\\
6.&89.&6785.&313.&25978.&17.&3&60&10\_40\_10 \textcolor{red}{\textcjheb{ymy}} JMJ $|$Tage\\
7.&90.&6786.&316.&25981.&20.&4&33&8\_10\_10\_5 \textcolor{red}{\textcjheb{hyy.h}} CJJH $|$ihres Lebens\\
\end{tabular}\medskip \\
Ende des Verses 31.12\\
Verse: 896, Buchstaben: 23, 319, 25984, Totalwerte: 951, 19580, 1826734\\
\\
Sie erweist ihm Gutes und nichts B"oses alle Tage ihres Lebens.\\
\newpage 
{\bf -- 31.13}\\
\medskip \\
\begin{tabular}{rrrrrrrrp{120mm}}
WV&WK&WB&ABK&ABB&ABV&AnzB&TW&Zahlencode \textcolor{red}{$\boldsymbol{Grundtext}$} Umschrift $|$"Ubersetzung(en)\\
1.&91.&6787.&320.&25985.&1.&4&509&4\_200\_300\_5 \textcolor{red}{\textcjheb{h+srd}} DRSH $|$sie sucht(e) (um)\\
2.&92.&6788.&324.&25989.&5.&3&330&90\_40\_200 \textcolor{red}{\textcjheb{rm.s}} "sMR $|$Wolle\\
3.&93.&6789.&327.&25992.&8.&6&836&6\_80\_300\_400\_10\_40 \textcolor{red}{\textcjheb{myt+spw}} WPSTJM $|$und Flachs\\
4.&94.&6790.&333.&25998.&14.&4&776&6\_400\_70\_300 \textcolor{red}{\textcjheb{+s`tw}} WTaS $|$und arbeitet dann/und sie schafft\\
5.&95.&6791.&337.&26002.&18.&4&180&2\_8\_80\_90 \textcolor{red}{\textcjheb{.sp.hb}} BCP"s $|$mit Lust/gern\\
6.&96.&6792.&341.&26006.&22.&4&115&20\_80\_10\_5 \textcolor{red}{\textcjheb{hypk}} KPJH $|$ihrer H"ande/mit ihren H"anden\\
\end{tabular}\medskip \\
Ende des Verses 31.13\\
Verse: 897, Buchstaben: 25, 344, 26009, Totalwerte: 2746, 22326, 1829480\\
\\
Sie sucht Wolle und Flachs, und arbeitet dann mit Lust ihrer H"ande.\\
\newpage 
{\bf -- 31.14}\\
\medskip \\
\begin{tabular}{rrrrrrrrp{120mm}}
WV&WK&WB&ABK&ABB&ABV&AnzB&TW&Zahlencode \textcolor{red}{$\boldsymbol{Grundtext}$} Umschrift $|$"Ubersetzung(en)\\
1.&97.&6793.&345.&26010.&1.&4&420&5\_10\_400\_5 \textcolor{red}{\textcjheb{htyh}} HJTH $|$sie war (=ist)\\
2.&98.&6794.&349.&26014.&5.&6&487&20\_1\_50\_10\_6\_400 \textcolor{red}{\textcjheb{twyn'k}} KANJWT $|$gleich Schiffen/wie die Schiffe\\
3.&99.&6795.&355.&26020.&11.&4&274&60\_6\_8\_200 \textcolor{red}{\textcjheb{r.hws}} sWCR $|$(des) Kaufmanns/eines H"andlers\\
4.&100.&6796.&359.&26024.&15.&5&388&40\_40\_200\_8\_100 \textcolor{red}{\textcjheb{q.hrmm}} MMRCQ $|$von fernher/von ferne\\
5.&101.&6797.&364.&26029.&20.&4&413&400\_2\_10\_1 \textcolor{red}{\textcjheb{'ybt}} TBJA $|$bringt sie herbei/sie l"asst kommen\\
6.&102.&6798.&368.&26033.&24.&4&83&30\_8\_40\_5 \textcolor{red}{\textcjheb{hm.hl}} LCMH $|$ihr Brot/ihre Nahrung\\
\end{tabular}\medskip \\
Ende des Verses 31.14\\
Verse: 898, Buchstaben: 27, 371, 26036, Totalwerte: 2065, 24391, 1831545\\
\\
Sie ist Kaufmannsschiffen gleich, von fernher bringt sie ihr Brot herbei.\\
\newpage 
{\bf -- 31.15}\\
\medskip \\
\begin{tabular}{rrrrrrrrp{120mm}}
WV&WK&WB&ABK&ABB&ABV&AnzB&TW&Zahlencode \textcolor{red}{$\boldsymbol{Grundtext}$} Umschrift $|$"Ubersetzung(en)\\
1.&103.&6799.&372.&26037.&1.&4&546&6\_400\_100\_40 \textcolor{red}{\textcjheb{mqtw}} WTQM $|$und sie steht auf\\
2.&104.&6800.&376.&26041.&5.&4&82&2\_70\_6\_4 \textcolor{red}{\textcjheb{dw`b}} BaWD $|$wenn noch\\
3.&105.&6801.&380.&26045.&9.&4&75&30\_10\_30\_5 \textcolor{red}{\textcjheb{hlyl}} LJLH $|$(es) Nacht (ist)\\
4.&106.&6802.&384.&26049.&13.&4&856&6\_400\_400\_50 \textcolor{red}{\textcjheb{nttw}} WTTN $|$und bestimmt/und sie gibt\\
5.&107.&6803.&388.&26053.&17.&3&289&9\_200\_80 \textcolor{red}{\textcjheb{pr.t}} tRP $|$die Speise/Nahrung\\
6.&108.&6804.&391.&26056.&20.&5&447&30\_2\_10\_400\_5 \textcolor{red}{\textcjheb{htybl}} LBJTH $|$f"ur ihr Haus/ihrem Haus\\
7.&109.&6805.&396.&26061.&25.&3&114&6\_8\_100 \textcolor{red}{\textcjheb{q.hw}} WCQ $|$und das Tagewerk/und Geb"uhrendes\\
8.&110.&6806.&399.&26064.&28.&7&765&30\_50\_70\_200\_400\_10\_5 \textcolor{red}{\textcjheb{hytr`nl}} LNaRTJH $|$(f"ur) ihre(n) M"agde(n)\\
\end{tabular}\medskip \\
Ende des Verses 31.15\\
Verse: 899, Buchstaben: 34, 405, 26070, Totalwerte: 3174, 27565, 1834719\\
\\
Und sie steht auf, wenn es noch Nacht ist, und bestimmt die Speise f"ur ihr Haus und das Tagewerk f"ur ihre M"agde.\\
\newpage 
{\bf -- 31.16}\\
\medskip \\
\begin{tabular}{rrrrrrrrp{120mm}}
WV&WK&WB&ABK&ABB&ABV&AnzB&TW&Zahlencode \textcolor{red}{$\boldsymbol{Grundtext}$} Umschrift $|$"Ubersetzung(en)\\
1.&111.&6807.&406.&26071.&1.&4&92&7\_40\_40\_5 \textcolor{red}{\textcjheb{hmmz}} ZMMH $|$sie sinnt auf/sie trachtet (zu kaufen)\\
2.&112.&6808.&410.&26075.&5.&3&309&300\_4\_5 \textcolor{red}{\textcjheb{hd+s}} SDH $|$ein Feld\\
3.&113.&6809.&413.&26078.&8.&6&525&6\_400\_100\_8\_5\_6 \textcolor{red}{\textcjheb{wh.hqtw}} WTQCHW $|$und (sie) erwirbt ihn (=es)\\
4.&114.&6810.&419.&26084.&14.&4&330&40\_80\_200\_10 \textcolor{red}{\textcjheb{yrpm}} MPRJ $|$von der Frucht\\
5.&115.&6811.&423.&26088.&18.&4&115&20\_80\_10\_5 \textcolor{red}{\textcjheb{hypk}} KPJH $|$ihrer H"ande\\
6.&116.&6812.&427.&26092.&22.&3&129&50\_9\_70 \textcolor{red}{\textcjheb{`.tn}} Nta $|$sie pflanzt(e)\\
7.&117.&6813.&430.&26095.&25.&3&260&20\_200\_40 \textcolor{red}{\textcjheb{mrk}} KRM $|$(einen) Weinberg\\
\end{tabular}\medskip \\
Ende des Verses 31.16\\
Verse: 900, Buchstaben: 27, 432, 26097, Totalwerte: 1760, 29325, 1836479\\
\\
Sie sinnt auf ein Feld und erwirbt es; von der Frucht ihrer H"ande pflanzt sie einen Weinberg.\\
\newpage 
{\bf -- 31.17}\\
\medskip \\
\begin{tabular}{rrrrrrrrp{120mm}}
WV&WK&WB&ABK&ABB&ABV&AnzB&TW&Zahlencode \textcolor{red}{$\boldsymbol{Grundtext}$} Umschrift $|$"Ubersetzung(en)\\
1.&118.&6814.&433.&26098.&1.&4&216&8\_3\_200\_5 \textcolor{red}{\textcjheb{hrg.h}} CGRH $|$sie g"urtet(e)\\
2.&119.&6815.&437.&26102.&5.&4&85&2\_70\_6\_7 \textcolor{red}{\textcjheb{zw`b}} BaWZ $|$mit Kraft/in Kraft\\
3.&120.&6816.&441.&26106.&9.&5&505&40\_400\_50\_10\_5 \textcolor{red}{\textcjheb{hyntm}} MTNJH $|$ihre Lenden/ihre H"uften\\
4.&121.&6817.&446.&26111.&14.&5&537&6\_400\_1\_40\_90 \textcolor{red}{\textcjheb{.sm'tw}} WTAM"s $|$und st"arkt/und sie kr"aftigt\\
5.&122.&6818.&451.&26116.&19.&7&698&7\_200\_70\_6\_400\_10\_5 \textcolor{red}{\textcjheb{hytw`rz}} ZRaWTJH $|$ihre Arme\\
\end{tabular}\medskip \\
Ende des Verses 31.17\\
Verse: 901, Buchstaben: 25, 457, 26122, Totalwerte: 2041, 31366, 1838520\\
\\
Sie g"urtet ihre Lenden mit Kraft und st"arkt ihre Arme.\\
\newpage 
{\bf -- 31.18}\\
\medskip \\
\begin{tabular}{rrrrrrrrp{120mm}}
WV&WK&WB&ABK&ABB&ABV&AnzB&TW&Zahlencode \textcolor{red}{$\boldsymbol{Grundtext}$} Umschrift $|$"Ubersetzung(en)\\
1.&123.&6819.&458.&26123.&1.&4&124&9\_70\_40\_5 \textcolor{red}{\textcjheb{hm`.t}} taMH $|$sie erf"ahrt/sie f"uhlt\\
2.&124.&6820.&462.&26127.&5.&2&30&20\_10 \textcolor{red}{\textcjheb{yk}} KJ $|$dass\\
3.&125.&6821.&464.&26129.&7.&3&17&9\_6\_2 \textcolor{red}{\textcjheb{bw.t}} tWB $|$gut (ist)\\
4.&126.&6822.&467.&26132.&10.&4&273&60\_8\_200\_5 \textcolor{red}{\textcjheb{hr.hs}} sCRH $|$ihr Erwerb\\
5.&127.&6823.&471.&26136.&14.&2&31&30\_1 \textcolor{red}{\textcjheb{'l}} LA $|$nicht\\
6.&128.&6824.&473.&26138.&16.&4&37&10\_20\_2\_5 \textcolor{red}{\textcjheb{hbky}} JKBH $|$geht aus/er (=es) erlischt\\
7.&129.&6825.&477.&26142.&20.&4&72&2\_30\_10\_30 \textcolor{red}{\textcjheb{lylb}} BLJL $|$des Nachts/in der Nacht\\
8.&130.&6826.&481.&26146.&24.&3&255&50\_200\_5 \textcolor{red}{\textcjheb{hrn}} NRH $|$ihr Licht/ihre Leuchte\\
\end{tabular}\medskip \\
Ende des Verses 31.18\\
Verse: 902, Buchstaben: 26, 483, 26148, Totalwerte: 839, 32205, 1839359\\
\\
Sie erf"ahrt, da"s ihr Erwerb gut ist: des Nachts geht ihr Licht nicht aus;\\
\newpage 
{\bf -- 31.19}\\
\medskip \\
\begin{tabular}{rrrrrrrrp{120mm}}
WV&WK&WB&ABK&ABB&ABV&AnzB&TW&Zahlencode \textcolor{red}{$\boldsymbol{Grundtext}$} Umschrift $|$"Ubersetzung(en)\\
1.&131.&6827.&484.&26149.&1.&4&29&10\_4\_10\_5 \textcolor{red}{\textcjheb{hydy}} JDJH $|$ihre (beiden) H"ande\\
2.&132.&6828.&488.&26153.&5.&4&343&300\_30\_8\_5 \textcolor{red}{\textcjheb{h.hl+s}} SLCH $|$sie legt an/sie streckt aus\\
3.&133.&6829.&492.&26157.&9.&6&538&2\_20\_10\_300\_6\_200 \textcolor{red}{\textcjheb{rw+sykb}} BKJSWR $|$den Spinnrocken/nach dem Spinnrocken\\
4.&134.&6830.&498.&26163.&15.&5&121&6\_20\_80\_10\_5 \textcolor{red}{\textcjheb{hypkw}} WKPJH $|$und ihre Finger/und ihre H"ande\\
5.&135.&6831.&503.&26168.&20.&4&466&400\_40\_20\_6 \textcolor{red}{\textcjheb{wkmt}} TMKW $|$erfassen/sie halten\\
6.&136.&6832.&507.&26172.&24.&3&130&80\_30\_20 \textcolor{red}{\textcjheb{klp}} PLK $|$die Spindel\\
\end{tabular}\medskip \\
Ende des Verses 31.19\\
Verse: 903, Buchstaben: 26, 509, 26174, Totalwerte: 1627, 33832, 1840986\\
\\
sie legt ihre H"ande an den Spinnrocken, und ihre Finger erfassen die Spindel.\\
\newpage 
{\bf -- 31.20}\\
\medskip \\
\begin{tabular}{rrrrrrrrp{120mm}}
WV&WK&WB&ABK&ABB&ABV&AnzB&TW&Zahlencode \textcolor{red}{$\boldsymbol{Grundtext}$} Umschrift $|$"Ubersetzung(en)\\
1.&137.&6833.&510.&26175.&1.&3&105&20\_80\_5 \textcolor{red}{\textcjheb{hpk}} KPH $|$ihre (Hohl)Hand\\
2.&138.&6834.&513.&26178.&4.&4&585&80\_200\_300\_5 \textcolor{red}{\textcjheb{h+srp}} PRSH $|$sie breitet aus/sie "offnet\\
3.&139.&6835.&517.&26182.&8.&4&160&30\_70\_50\_10 \textcolor{red}{\textcjheb{yn`l}} LaNJ $|$(zu) dem Elenden\\
4.&140.&6836.&521.&26186.&12.&5&35&6\_10\_4\_10\_5 \textcolor{red}{\textcjheb{hydyw}} WJDJH $|$und ihre H"ande\\
5.&141.&6837.&526.&26191.&17.&4&343&300\_30\_8\_5 \textcolor{red}{\textcjheb{h.hl+s}} SLCH $|$streckt entgegen/sie streckt aus\\
6.&142.&6838.&530.&26195.&21.&6&99&30\_1\_2\_10\_6\_50 \textcolor{red}{\textcjheb{nwyb'l}} LABJWN $|$dem D"urftigen/zu dem Armen\\
\end{tabular}\medskip \\
Ende des Verses 31.20\\
Verse: 904, Buchstaben: 26, 535, 26200, Totalwerte: 1327, 35159, 1842313\\
\\
Sie breitet ihre Hand aus zu dem Elenden und streckt ihre H"ande dem D"urftigen entgegen.\\
\newpage 
{\bf -- 31.21}\\
\medskip \\
\begin{tabular}{rrrrrrrrp{120mm}}
WV&WK&WB&ABK&ABB&ABV&AnzB&TW&Zahlencode \textcolor{red}{$\boldsymbol{Grundtext}$} Umschrift $|$"Ubersetzung(en)\\
1.&143.&6839.&536.&26201.&1.&2&31&30\_1 \textcolor{red}{\textcjheb{'l}} LA $|$nicht\\
2.&144.&6840.&538.&26203.&3.&4&611&400\_10\_200\_1 \textcolor{red}{\textcjheb{'ryt}} TJRA $|$sie f"urchtet\\
3.&145.&6841.&542.&26207.&7.&5&447&30\_2\_10\_400\_5 \textcolor{red}{\textcjheb{htybl}} LBJTH $|$f"ur ihr Haus\\
4.&146.&6842.&547.&26212.&12.&4&373&40\_300\_30\_3 \textcolor{red}{\textcjheb{gl+sm}} MSLG $|$den Schnee/vor dem Schnee\\
5.&147.&6843.&551.&26216.&16.&2&30&20\_10 \textcolor{red}{\textcjheb{yk}} KJ $|$denn\\
6.&148.&6844.&553.&26218.&18.&2&50&20\_30 \textcolor{red}{\textcjheb{lk}} KL $|$ihr ganzes/all\\
7.&149.&6845.&555.&26220.&20.&4&417&2\_10\_400\_5 \textcolor{red}{\textcjheb{htyb}} BJTH $|$(ihr) Haus\\
8.&150.&6846.&559.&26224.&24.&3&332&30\_2\_300 \textcolor{red}{\textcjheb{+sbl}} LBS $|$(er (=es)) ist gekleidet\\
9.&151.&6847.&562.&26227.&27.&4&400&300\_50\_10\_40 \textcolor{red}{\textcjheb{myn+s}} SNJM $|$in Karmesin/mit Scharlach(wolle)\\
\end{tabular}\medskip \\
Ende des Verses 31.21\\
Verse: 905, Buchstaben: 30, 565, 26230, Totalwerte: 2691, 37850, 1845004\\
\\
Sie f"urchtet f"ur ihr Haus den Schnee nicht, denn ihr ganzes Haus ist in Karmesin gekleidet.\\
\newpage 
{\bf -- 31.22}\\
\medskip \\
\begin{tabular}{rrrrrrrrp{120mm}}
WV&WK&WB&ABK&ABB&ABV&AnzB&TW&Zahlencode \textcolor{red}{$\boldsymbol{Grundtext}$} Umschrift $|$"Ubersetzung(en)\\
1.&152.&6848.&566.&26231.&1.&6&296&40\_200\_2\_4\_10\_40 \textcolor{red}{\textcjheb{mydbrm}} MRBDJM $|$Teppiche/Decken\\
2.&153.&6849.&572.&26237.&7.&4&775&70\_300\_400\_5 \textcolor{red}{\textcjheb{ht+s`}} aSTH $|$sie verfertigt/sie fertigt\\
3.&154.&6850.&576.&26241.&11.&2&35&30\_5 \textcolor{red}{\textcjheb{hl}} LH $|$sich\\
4.&155.&6851.&578.&26243.&13.&2&600&300\_300 \textcolor{red}{\textcjheb{+s+s}} SS $|$Byssus\\
5.&156.&6852.&580.&26245.&15.&6&300&6\_1\_200\_3\_40\_50 \textcolor{red}{\textcjheb{nmgr'w}} WARGMN $|$und Purpur\\
6.&157.&6853.&586.&26251.&21.&5&343&30\_2\_6\_300\_5 \textcolor{red}{\textcjheb{h+swbl}} LBWSH $|$sind ihr Gewand/(ist) ihr Gewand\\
\end{tabular}\medskip \\
Ende des Verses 31.22\\
Verse: 906, Buchstaben: 25, 590, 26255, Totalwerte: 2349, 40199, 1847353\\
\\
Sie verfertigt sich Teppiche; Byssus und Purpur sind ihr Gewand.\\
\newpage 
{\bf -- 31.23}\\
\medskip \\
\begin{tabular}{rrrrrrrrp{120mm}}
WV&WK&WB&ABK&ABB&ABV&AnzB&TW&Zahlencode \textcolor{red}{$\boldsymbol{Grundtext}$} Umschrift $|$"Ubersetzung(en)\\
1.&158.&6854.&591.&26256.&1.&4&130&50\_6\_4\_70 \textcolor{red}{\textcjheb{`dwn}} NWDa $|$bekannt ist/er (=es) ist geachtet\\
2.&159.&6855.&595.&26260.&5.&6&622&2\_300\_70\_200\_10\_40 \textcolor{red}{\textcjheb{myr`+sb}} BSaRJM $|$in den Toren\\
3.&160.&6856.&601.&26266.&11.&4&107&2\_70\_30\_5 \textcolor{red}{\textcjheb{hl`b}} BaLH $|$ihr Mann/ihr Eheherr\\
4.&161.&6857.&605.&26270.&15.&5&710&2\_300\_2\_400\_6 \textcolor{red}{\textcjheb{wtb+sb}} BSBTW $|$indem er sitzt/wenn er sitzt\\
5.&162.&6858.&610.&26275.&20.&2&110&70\_40 \textcolor{red}{\textcjheb{m`}} aM $|$bei/mit\\
6.&163.&6859.&612.&26277.&22.&4&167&7\_100\_50\_10 \textcolor{red}{\textcjheb{ynqz}} ZQNJ $|$den "Altesten/den Alten\\
7.&164.&6860.&616.&26281.&26.&3&291&1\_200\_90 \textcolor{red}{\textcjheb{.sr'}} AR"s $|$des Landes \\
\end{tabular}\medskip \\
Ende des Verses 31.23\\
Verse: 907, Buchstaben: 28, 618, 26283, Totalwerte: 2137, 42336, 1849490\\
\\
Ihr Mann ist bekannt in den Toren, indem er sitzt bei den "Altesten des Landes.\\
\newpage 
{\bf -- 31.24}\\
\medskip \\
\begin{tabular}{rrrrrrrrp{120mm}}
WV&WK&WB&ABK&ABB&ABV&AnzB&TW&Zahlencode \textcolor{red}{$\boldsymbol{Grundtext}$} Umschrift $|$"Ubersetzung(en)\\
1.&165.&6861.&619.&26284.&1.&4&124&60\_4\_10\_50 \textcolor{red}{\textcjheb{nyds}} sDJN $|$Hemden/Untergewand\\
2.&166.&6862.&623.&26288.&5.&4&775&70\_300\_400\_5 \textcolor{red}{\textcjheb{ht+s`}} aSTH $|$sie (ver)fertigt\\
3.&167.&6863.&627.&26292.&9.&5&666&6\_400\_40\_20\_200 \textcolor{red}{\textcjheb{rkmtw}} WTMKR $|$und (sie) verkauft (sie)\\
4.&168.&6864.&632.&26297.&14.&5&223&6\_8\_3\_6\_200 \textcolor{red}{\textcjheb{rwg.hw}} WCGWR $|$und G"urtel\\
5.&169.&6865.&637.&26302.&19.&4&505&50\_400\_50\_5 \textcolor{red}{\textcjheb{hntn}} NTNH $|$liefert sie/sie gibt\\
6.&170.&6866.&641.&26306.&23.&6&230&30\_20\_50\_70\_50\_10 \textcolor{red}{\textcjheb{yn`nkl}} LKNaNJ $|$dem Kaufmann/an den H"andler\\
\end{tabular}\medskip \\
Ende des Verses 31.24\\
Verse: 908, Buchstaben: 28, 646, 26311, Totalwerte: 2523, 44859, 1852013\\
\\
Sie verfertigt Hemden und verkauft sie, und G"urtel liefert sie dem Kaufmann.\\
\newpage 
{\bf -- 31.25}\\
\medskip \\
\begin{tabular}{rrrrrrrrp{120mm}}
WV&WK&WB&ABK&ABB&ABV&AnzB&TW&Zahlencode \textcolor{red}{$\boldsymbol{Grundtext}$} Umschrift $|$"Ubersetzung(en)\\
1.&171.&6867.&647.&26312.&1.&2&77&70\_7 \textcolor{red}{\textcjheb{z`}} aZ $|$Macht/Kraft\\
2.&172.&6868.&649.&26314.&3.&4&215&6\_5\_4\_200 \textcolor{red}{\textcjheb{rdhw}} WHDR $|$und Hoheit/und Glanz\\
3.&173.&6869.&653.&26318.&7.&5&343&30\_2\_6\_300\_5 \textcolor{red}{\textcjheb{h+swbl}} LBWSH $|$(sind) ihr Gewand\\
4.&174.&6870.&658.&26323.&12.&5&814&6\_400\_300\_8\_100 \textcolor{red}{\textcjheb{q.h+stw}} WTSCQ $|$und (so) sie lacht\\
5.&175.&6871.&663.&26328.&17.&4&86&30\_10\_6\_40 \textcolor{red}{\textcjheb{mwyl}} LJWM $|$des Tags\\
6.&176.&6872.&667.&26332.&21.&5&265&1\_8\_200\_6\_50 \textcolor{red}{\textcjheb{nwr.h'}} ACRWN $|$k"unftigen\\
\end{tabular}\medskip \\
Ende des Verses 31.25\\
Verse: 909, Buchstaben: 25, 671, 26336, Totalwerte: 1800, 46659, 1853813\\
\\
Macht und Hoheit sind ihr Gewand, und so lacht sie des k"unftigen Tages.\\
\newpage 
{\bf -- 31.26}\\
\medskip \\
\begin{tabular}{rrrrrrrrp{120mm}}
WV&WK&WB&ABK&ABB&ABV&AnzB&TW&Zahlencode \textcolor{red}{$\boldsymbol{Grundtext}$} Umschrift $|$"Ubersetzung(en)\\
1.&177.&6873.&672.&26337.&1.&3&95&80\_10\_5 \textcolor{red}{\textcjheb{hyp}} PJH $|$ihren Mund\\
2.&178.&6874.&675.&26340.&4.&4&493&80\_400\_8\_5 \textcolor{red}{\textcjheb{h.htp}} PTCH $|$sie tut auf\\
3.&179.&6875.&679.&26344.&8.&5&75&2\_8\_20\_40\_5 \textcolor{red}{\textcjheb{hmk.hb}} BCKMH $|$mit Weisheit\\
4.&180.&6876.&684.&26349.&13.&5&1012&6\_400\_6\_200\_400 \textcolor{red}{\textcjheb{trwtw}} WTWRT $|$und Lehre/und Weisung\\
5.&181.&6877.&689.&26354.&18.&3&72&8\_60\_4 \textcolor{red}{\textcjheb{ds.h}} CsD $|$liebreiche/(der) G"ute\\
6.&182.&6878.&692.&26357.&21.&2&100&70\_30 \textcolor{red}{\textcjheb{l`}} aL $|$(ist) auf\\
7.&183.&6879.&694.&26359.&23.&5&391&30\_300\_6\_50\_5 \textcolor{red}{\textcjheb{hnw+sl}} LSWNH $|$ihrer Zunge\\
\end{tabular}\medskip \\
Ende des Verses 31.26\\
Verse: 910, Buchstaben: 27, 698, 26363, Totalwerte: 2238, 48897, 1856051\\
\\
Sie tut ihren Mund auf mit Weisheit, und liebreiche Lehre ist auf ihrer Zunge.\\
\newpage 
{\bf -- 31.27}\\
\medskip \\
\begin{tabular}{rrrrrrrrp{120mm}}
WV&WK&WB&ABK&ABB&ABV&AnzB&TW&Zahlencode \textcolor{red}{$\boldsymbol{Grundtext}$} Umschrift $|$"Ubersetzung(en)\\
1.&184.&6880.&699.&26364.&1.&5&191&90\_6\_80\_10\_5 \textcolor{red}{\textcjheb{hypw.s}} "sWPJH $|$sie "uberwacht/ausschauend (ist sie)\\
2.&185.&6881.&704.&26369.&6.&6&471&5\_30\_10\_20\_6\_400 \textcolor{red}{\textcjheb{twkylh}} HLJKWT $|$die Vorg"ange/nach den Wegen\\
3.&186.&6882.&710.&26375.&12.&4&417&2\_10\_400\_5 \textcolor{red}{\textcjheb{htyb}} BJTH $|$in ihrem Haus/ihres Hauses\\
4.&187.&6883.&714.&26379.&16.&4&84&6\_30\_8\_40 \textcolor{red}{\textcjheb{m.hlw}} WLCM $|$und (das) Brot\\
5.&188.&6884.&718.&26383.&20.&5&596&70\_90\_30\_6\_400 \textcolor{red}{\textcjheb{twl.s`}} a"sLWT $|$der Faulheit/der Tr"agheit\\
6.&189.&6885.&723.&26388.&25.&2&31&30\_1 \textcolor{red}{\textcjheb{'l}} LA $|$nicht\\
7.&190.&6886.&725.&26390.&27.&4&451&400\_1\_20\_30 \textcolor{red}{\textcjheb{lk't}} TAKL $|$sie isst\\
\end{tabular}\medskip \\
Ende des Verses 31.27\\
Verse: 911, Buchstaben: 30, 728, 26393, Totalwerte: 2241, 51138, 1858292\\
\\
Sie "uberwacht die Vorg"ange in ihrem Hause und i"st nicht das Brot der Faulheit.\\
\newpage 
{\bf -- 31.28}\\
\medskip \\
\begin{tabular}{rrrrrrrrp{120mm}}
WV&WK&WB&ABK&ABB&ABV&AnzB&TW&Zahlencode \textcolor{red}{$\boldsymbol{Grundtext}$} Umschrift $|$"Ubersetzung(en)\\
1.&191.&6887.&729.&26394.&1.&3&146&100\_40\_6 \textcolor{red}{\textcjheb{wmq}} QMW $|$(es) stehen auf/sie standen auf\\
2.&192.&6888.&732.&26397.&4.&4&67&2\_50\_10\_5 \textcolor{red}{\textcjheb{hynb}} BNJH $|$ihre S"ohne\\
3.&193.&6889.&736.&26401.&8.&7&528&6\_10\_1\_300\_200\_6\_5 \textcolor{red}{\textcjheb{hwr+s'yw}} WJASRWH $|$und (sie) preisen gl"ucklich sie\\
4.&194.&6890.&743.&26408.&15.&4&107&2\_70\_30\_5 \textcolor{red}{\textcjheb{hl`b}} BaLH $|$ihr Mann (steht auf)/ihr Eheherr\\
5.&195.&6891.&747.&26412.&19.&6&86&6\_10\_5\_30\_30\_5 \textcolor{red}{\textcjheb{hllhyw}} WJHLLH $|$und (er) r"uhmt sie\\
\end{tabular}\medskip \\
Ende des Verses 31.28\\
Verse: 912, Buchstaben: 24, 752, 26417, Totalwerte: 934, 52072, 1859226\\
\\
Ihre S"ohne stehen auf und preisen sie gl"ucklich, ihr Mann steht auf und r"uhmt sie:\\
\newpage 
{\bf -- 31.29}\\
\medskip \\
\begin{tabular}{rrrrrrrrp{120mm}}
WV&WK&WB&ABK&ABB&ABV&AnzB&TW&Zahlencode \textcolor{red}{$\boldsymbol{Grundtext}$} Umschrift $|$"Ubersetzung(en)\\
1.&196.&6892.&753.&26418.&1.&4&608&200\_2\_6\_400 \textcolor{red}{\textcjheb{twbr}} RBWT $|$viele\\
2.&197.&6893.&757.&26422.&5.&4&458&2\_50\_6\_400 \textcolor{red}{\textcjheb{twnb}} BNWT $|$T"ochter\\
3.&198.&6894.&761.&26426.&9.&3&376&70\_300\_6 \textcolor{red}{\textcjheb{w+s`}} aSW $|$haben gehandelt/(sie) erwarben\\
4.&199.&6895.&764.&26429.&12.&3&48&8\_10\_30 \textcolor{red}{\textcjheb{ly.h}} CJL $|$wacker/Verm"ogen\\
5.&200.&6896.&767.&26432.&15.&3&407&6\_1\_400 \textcolor{red}{\textcjheb{t'w}} WAT $|$du aber/und du\\
6.&201.&6897.&770.&26435.&18.&4&510&70\_30\_10\_400 \textcolor{red}{\textcjheb{tyl`}} aLJT $|$hast "ubertroffen/du stiegst hinauf\\
7.&202.&6898.&774.&26439.&22.&2&100&70\_30 \textcolor{red}{\textcjheb{l`}} aL $|$/"uber\\
8.&203.&6899.&776.&26441.&24.&4&105&20\_30\_50\_5 \textcolor{red}{\textcjheb{hnlk}} KLNH $|$sie alle\\
\end{tabular}\medskip \\
Ende des Verses 31.29\\
Verse: 913, Buchstaben: 27, 779, 26444, Totalwerte: 2612, 54684, 1861838\\
\\
"Viele T"ochter haben wacker gehandelt, du aber hast sie alle "ubertroffen!"\\
\newpage 
{\bf -- 31.30}\\
\medskip \\
\begin{tabular}{rrrrrrrrp{120mm}}
WV&WK&WB&ABK&ABB&ABV&AnzB&TW&Zahlencode \textcolor{red}{$\boldsymbol{Grundtext}$} Umschrift $|$"Ubersetzung(en)\\
1.&204.&6900.&780.&26445.&1.&3&600&300\_100\_200 \textcolor{red}{\textcjheb{rq+s}} SQR $|$Trug\\
2.&205.&6901.&783.&26448.&4.&3&63&5\_8\_50 \textcolor{red}{\textcjheb{n.hh}} HCN $|$(ist) (die) Anmut\\
3.&206.&6902.&786.&26451.&7.&4&43&6\_5\_2\_30 \textcolor{red}{\textcjheb{lbhw}} WHBL $|$und Eitelkeit/und ein Windhauch\\
4.&207.&6903.&790.&26455.&11.&4&105&5\_10\_80\_10 \textcolor{red}{\textcjheb{ypyh}} HJPJ $|$(ist) die Sch"onheit\\
5.&208.&6904.&794.&26459.&15.&3&306&1\_300\_5 \textcolor{red}{\textcjheb{h+s'}} ASH $|$(eine) Frau\\
6.&209.&6905.&797.&26462.&18.&4&611&10\_200\_1\_400 \textcolor{red}{\textcjheb{t'ry}} JRAT $|$die f"urchtet/(mit) (Ehr)Furcht\\
7.&210.&6906.&801.&26466.&22.&4&26&10\_5\_6\_5 \textcolor{red}{\textcjheb{hwhy}} JHWH $|$(vor) Jahwe\\
8.&211.&6907.&805.&26470.&26.&3&16&5\_10\_1 \textcolor{red}{\textcjheb{'yh}} HJA $|$sie\\
9.&212.&6908.&808.&26473.&29.&5&865&400\_400\_5\_30\_30 \textcolor{red}{\textcjheb{llhtt}} TTHLL $|$(sie) wird gepriesen (werden)\\
\end{tabular}\medskip \\
Ende des Verses 31.30\\
Verse: 914, Buchstaben: 33, 812, 26477, Totalwerte: 2635, 57319, 1864473\\
\\
Die Anmut ist Trug, und die Sch"onheit Eitelkeit; ein Weib, das Jahwe f"urchtet, sie wird gepriesen werden.\\
\newpage 
{\bf -- 31.31}\\
\medskip \\
\begin{tabular}{rrrrrrrrp{120mm}}
WV&WK&WB&ABK&ABB&ABV&AnzB&TW&Zahlencode \textcolor{red}{$\boldsymbol{Grundtext}$} Umschrift $|$"Ubersetzung(en)\\
1.&213.&6909.&813.&26478.&1.&3&456&400\_50\_6 \textcolor{red}{\textcjheb{wnt}} TNW $|$gebt\\
2.&214.&6910.&816.&26481.&4.&2&35&30\_5 \textcolor{red}{\textcjheb{hl}} LH $|$ihr/f"ur sie\\
3.&215.&6911.&818.&26483.&6.&4&330&40\_80\_200\_10 \textcolor{red}{\textcjheb{yrpm}} MPRJ $|$von der Frucht\\
4.&216.&6912.&822.&26487.&10.&4&29&10\_4\_10\_5 \textcolor{red}{\textcjheb{hydy}} JDJH $|$ihrer (zwei) H"ande\\
5.&217.&6913.&826.&26491.&14.&7&92&6\_10\_5\_30\_30\_6\_5 \textcolor{red}{\textcjheb{hwllhyw}} WJHLLWH $|$und m"ogen presien sie/und sie (=es) preisen sie\\
6.&218.&6914.&833.&26498.&21.&6&622&2\_300\_70\_200\_10\_40 \textcolor{red}{\textcjheb{myr`+sb}} BSaRJM $|$in den Toren\\
7.&219.&6915.&839.&26504.&27.&5&425&40\_70\_300\_10\_5 \textcolor{red}{\textcjheb{hy+s`m}} MaSJH $|$ihre Werke\\
\end{tabular}\medskip \\
Ende des Verses 31.31\\
Verse: 915, Buchstaben: 31, 843, 26508, Totalwerte: 1989, 59308, 1866462\\
\\
Gebet ihr von der Frucht ihrer H"ande; und in den Toren m"ogen ihre Werke sie preisen!\\
\\
{\bf Ende des Kapitels 31}\\

\bigskip				%%gro�er Abstand

\newpage
\hphantom{x}
\bigskip\bigskip\bigskip\bigskip\bigskip\bigskip
\begin{center}{ \huge {\bf Ende des Buches}}\end{center}


\end{document}



