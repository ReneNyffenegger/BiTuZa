\documentclass[a4paper,10pt,landscape]{article}
\usepackage[landscape]{geometry}	%%Querformat
\usepackage{fancyhdr}			%%Erweiterte Kopfzeilen
\usepackage{color}			%%Farben
\usepackage[greek,german]{babel}        %%Griechische und deutsche Namen
\usepackage{cjhebrew}                   %%Hebr�isches Packet
\usepackage{upgreek}                    %%nicht kursive griechische Buchstaben
\usepackage{amsbsy}                     %%fette griechische Buchstaben
\setlength{\parindent}{0pt}		%%Damit bei neuen Abs"atzen kein Einzug
\frenchspacing

\pagestyle{fancy}			%%Kopf-/Fu�zeilenstyle

\renewcommand{\headrulewidth}{0.5pt}	%%Strich in Kopfzeile
\renewcommand{\footrulewidth}{0pt}	%%kein Strich in Fu�zeile
\renewcommand{\sectionmark}[1]{\markright{{#1}}}
\lhead{\rightmark}
\chead{}
\rhead{Bibel in Text und Zahl}
\lfoot{pgz}
\cfoot{\thepage}			%%Seitenzahl
\rfoot{}

\renewcommand{\section}[3]{\begin{center}{ \huge {\bf \textsl{{#1}}\\ \textcolor{red}{\textsl{{#2}}}}}\end{center}
\sectionmark{{#3}}}

\renewcommand{\baselinestretch}{0.9}	%%Zeilenabstand

\begin{document}			%%Dokumentbeginn

\section{\bigskip\bigskip\bigskip\bigskip\bigskip\bigskip
\\Der zweite Brief des Johannes}
{}
{2. Johannes}	%%�berschrift (�bergibt {schwarzen
										%%Text}{roten Text}{Text in Kopfzeile}


\bigskip				%%gro�er Abstand

\newpage
\hphantom{x}
\bigskip\bigskip\bigskip\bigskip\bigskip\bigskip
\begin{center}{ \huge {\bf Erl"auterungen}}\end{center}

\medskip
In diesem Buch werden folgende Abk"urzungen verwendet:\\
WV = Nummer des Wortes im Vers\\
WK = Nummer des Wortes im Kapitel\\
WB = Nummer des Wortes im Buch\\
ABV = Nummer des Anfangsbuchstabens des Wortes im Vers\\
ABK = Nummer des Anfangsbuchstabens des Wortes im Kapitel\\
ABB = Nummer des Anfangsbuchstabens des Wortes im Buch\\
AnzB = Anzahl der Buchstaben des Wortes\\
TW = Totalwert des Wortes\\

\medskip
Am Ende eines Verses finden sich sieben Zahlen,\\
die folgende Bedeutung haben (von links nach rechts):\\
1. Nummer des Verses im Buch\\
2. Gesamtzahl der Buchstaben im Vers\\
3. Gesamtzahl der Buchstaben (bis einschlie"slich dieses Verses) im Kapitel\\
4. Gesamtzahl der Buchstaben (bis einschlie"slich dieses Verses) im Buch\\
5. Summe der Totalwerte des Verses\\
6. Summe der Totalwerte (bis einschlie"slich dieses Verses) im Kapitel\\
7. Summe der Totalwerte (bis einschlie"slich dieses Verses) im Buch\\



\newpage 
{\bf -- 1.1}\\
\medskip \\
\begin{tabular}{rrrrrrrrp{120mm}}
WV&WK&WB&ABK&ABB&ABV&AnzB&TW&Zahlencode \textcolor{red}{$\boldsymbol{Grundtext}$} Umschrift $|$"Ubersetzung(en)\\
1.&1.&1.&1.&1.&1.&1&70&70 \textcolor{red}{$\boldsymbol{\mathrm{o}}$} o $|$der\\
2.&2.&2.&2.&2.&2.&11&1462&80\_100\_5\_200\_2\_400\_300\_5\_100\_70\_200 \textcolor{red}{$\boldsymbol{\uppi\uprho\upepsilon\upsigma\upbeta\upsilon\uptau\upepsilon\uprho\mathrm{o}\upsigma}$} presb"uteros $|$"Alteste\\
3.&3.&3.&13.&13.&13.&7&388&5\_20\_30\_5\_20\_300\_8 \textcolor{red}{$\boldsymbol{\upepsilon\upkappa\uplambda\upepsilon\upkappa\uptau\upeta}$} eklekt"a $|$an (die) auserw"ahlte\\
4.&4.&4.&20.&20.&20.&5&531&20\_400\_100\_10\_1 \textcolor{red}{$\boldsymbol{\upkappa\upsilon\uprho\upiota\upalpha}$} k"urja $|$Frau/Herrin\\
5.&5.&5.&25.&25.&25.&3&31&20\_1\_10 \textcolor{red}{$\boldsymbol{\upkappa\upalpha\upiota}$} kaj $|$und\\
6.&6.&6.&28.&28.&28.&4&580&300\_70\_10\_200 \textcolor{red}{$\boldsymbol{\uptau\mathrm{o}\upiota\upsigma}$} tojs $|$(an) (die)\\
7.&7.&7.&32.&32.&32.&7&655&300\_5\_20\_50\_70\_10\_200 \textcolor{red}{$\boldsymbol{\uptau\upepsilon\upkappa\upnu\mathrm{o}\upiota\upsigma}$} teknojs $|$Kinder\\
8.&8.&8.&39.&39.&39.&5&909&1\_400\_300\_8\_200 \textcolor{red}{$\boldsymbol{\upalpha\upsilon\uptau\upeta\upsigma}$} a"ut"as $|$ihre\\
9.&9.&9.&44.&44.&44.&3&670&70\_400\_200 \textcolor{red}{$\boldsymbol{\mathrm{o}\upsilon\upsigma}$} o"us $|$die\\
10.&10.&10.&47.&47.&47.&3&808&5\_3\_800 \textcolor{red}{$\boldsymbol{\upepsilon\upgamma\upomega}$} egO $|$ich\\
11.&11.&11.&50.&50.&50.&5&885&1\_3\_1\_80\_800 \textcolor{red}{$\boldsymbol{\upalpha\upgamma\upalpha\uppi\upomega}$} agapO $|$liebe\\
12.&12.&12.&55.&55.&55.&2&55&5\_50 \textcolor{red}{$\boldsymbol{\upepsilon\upnu}$} en $|$in\\
13.&13.&13.&57.&57.&57.&7&64&1\_30\_8\_9\_5\_10\_1 \textcolor{red}{$\boldsymbol{\upalpha\uplambda\upeta\upvartheta\upepsilon\upiota\upalpha}$} al"aTeja $|$Wahrheit\\
14.&14.&14.&64.&64.&64.&3&31&20\_1\_10 \textcolor{red}{$\boldsymbol{\upkappa\upalpha\upiota}$} kaj $|$und\\
15.&15.&15.&67.&67.&67.&3&490&70\_400\_20 \textcolor{red}{$\boldsymbol{\mathrm{o}\upsilon\upkappa}$} o"uk $|$nicht\\
16.&16.&16.&70.&70.&70.&3&808&5\_3\_800 \textcolor{red}{$\boldsymbol{\upepsilon\upgamma\upomega}$} egO $|$ich\\
17.&17.&17.&73.&73.&73.&5&430&40\_70\_50\_70\_200 \textcolor{red}{$\boldsymbol{\upmu\mathrm{o}\upnu\mathrm{o}\upsigma}$} monos $|$allein\\
18.&18.&18.&78.&78.&78.&4&62&1\_30\_30\_1 \textcolor{red}{$\boldsymbol{\upalpha\uplambda\uplambda\upalpha}$} alla $|$sondern\\
19.&19.&19.&82.&82.&82.&3&31&20\_1\_10 \textcolor{red}{$\boldsymbol{\upkappa\upalpha\upiota}$} kaj $|$auch\\
20.&20.&20.&85.&85.&85.&6&636&80\_1\_50\_300\_5\_200 \textcolor{red}{$\boldsymbol{\uppi\upalpha\upnu\uptau\upepsilon\upsigma}$} pantes $|$alle\\
21.&21.&21.&91.&91.&91.&2&80&70\_10 \textcolor{red}{$\boldsymbol{\mathrm{o}\upiota}$} oj $|$welche/(die)\\
22.&22.&22.&93.&93.&93.&9&1453&5\_3\_50\_800\_20\_70\_300\_5\_200 \textcolor{red}{$\boldsymbol{\upepsilon\upgamma\upnu\upomega\upkappa\mathrm{o}\uptau\upepsilon\upsigma}$} egnOkotes $|$erkannt haben/erkannt Habenden\\
23.&23.&23.&102.&102.&102.&3&358&300\_8\_50 \textcolor{red}{$\boldsymbol{\uptau\upeta\upnu}$} t"an $|$die\\
24.&24.&24.&105.&105.&105.&8&114&1\_30\_8\_9\_5\_10\_1\_50 \textcolor{red}{$\boldsymbol{\upalpha\uplambda\upeta\upvartheta\upepsilon\upiota\upalpha\upnu}$} al"aTejan $|$Wahrheit\\
\end{tabular}\medskip \\
Ende des Verses 1.1\\
Verse: 1, Buchstaben: 112, 112, 112, Totalwerte: 11601, 11601, 11601\\
\\
Der "Alteste der auserw"ahlten Frau und ihren Kindern, die ich liebe in der Wahrheit; und nicht ich allein, sondern auch alle, welche die Wahrheit erkannt haben,\\
\newpage 
{\bf -- 1.2}\\
\medskip \\
\begin{tabular}{rrrrrrrrp{120mm}}
WV&WK&WB&ABK&ABB&ABV&AnzB&TW&Zahlencode \textcolor{red}{$\boldsymbol{Grundtext}$} Umschrift $|$"Ubersetzung(en)\\
1.&25.&25.&113.&113.&1.&3&15&4\_10\_1 \textcolor{red}{$\boldsymbol{\updelta\upiota\upalpha}$} dja $|$um willen/wegen\\
2.&26.&26.&116.&116.&4.&3&358&300\_8\_50 \textcolor{red}{$\boldsymbol{\uptau\upeta\upnu}$} t"an $|$der\\
3.&27.&27.&119.&119.&7.&8&114&1\_30\_8\_9\_5\_10\_1\_50 \textcolor{red}{$\boldsymbol{\upalpha\uplambda\upeta\upvartheta\upepsilon\upiota\upalpha\upnu}$} al"aTejan $|$Wahrheit\\
4.&28.&28.&127.&127.&15.&3&358&300\_8\_50 \textcolor{red}{$\boldsymbol{\uptau\upeta\upnu}$} t"an $|$die/(der)\\
5.&29.&29.&130.&130.&18.&8&816&40\_5\_50\_70\_400\_200\_1\_50 \textcolor{red}{$\boldsymbol{\upmu\upepsilon\upnu\mathrm{o}\upsilon\upsigma\upalpha\upnu}$} meno"usan $|$bleibt/bleibenden\\
6.&30.&30.&138.&138.&26.&2&55&5\_50 \textcolor{red}{$\boldsymbol{\upepsilon\upnu}$} en $|$in\\
7.&31.&31.&140.&140.&28.&4&108&8\_40\_10\_50 \textcolor{red}{$\boldsymbol{\upeta\upmu\upiota\upnu}$} "amjn $|$uns\\
8.&32.&32.&144.&144.&32.&3&31&20\_1\_10 \textcolor{red}{$\boldsymbol{\upkappa\upalpha\upiota}$} kaj $|$und\\
9.&33.&33.&147.&147.&35.&3&54&40\_5\_9 \textcolor{red}{$\boldsymbol{\upmu\upepsilon\upvartheta}$} meT $|$mit/bei\\
10.&34.&34.&150.&150.&38.&4&898&8\_40\_800\_50 \textcolor{red}{$\boldsymbol{\upeta\upmu\upomega\upnu}$} "amOn $|$uns\\
11.&35.&35.&154.&154.&42.&5&516&5\_200\_300\_1\_10 \textcolor{red}{$\boldsymbol{\upepsilon\upsigma\uptau\upalpha\upiota}$} estaj $|$sein wird\\
12.&36.&36.&159.&159.&47.&3&215&5\_10\_200 \textcolor{red}{$\boldsymbol{\upepsilon\upiota\upsigma}$} ejs $|$in\\
13.&37.&37.&162.&162.&50.&3&420&300\_70\_50 \textcolor{red}{$\boldsymbol{\uptau\mathrm{o}\upnu}$} ton $|$(die)\\
14.&38.&38.&165.&165.&53.&5&862&1\_10\_800\_50\_1 \textcolor{red}{$\boldsymbol{\upalpha\upiota\upomega\upnu\upalpha}$} ajOna $|$Ewigkeit\\
\end{tabular}\medskip \\
Ende des Verses 1.2\\
Verse: 2, Buchstaben: 57, 169, 169, Totalwerte: 4820, 16421, 16421\\
\\
um der Wahrheit willen, die in uns bleibt und mit uns sein wird in Ewigkeit.\\
\newpage 
{\bf -- 1.3}\\
\medskip \\
\begin{tabular}{rrrrrrrrp{120mm}}
WV&WK&WB&ABK&ABB&ABV&AnzB&TW&Zahlencode \textcolor{red}{$\boldsymbol{Grundtext}$} Umschrift $|$"Ubersetzung(en)\\
1.&39.&39.&170.&170.&1.&5&516&5\_200\_300\_1\_10 \textcolor{red}{$\boldsymbol{\upepsilon\upsigma\uptau\upalpha\upiota}$} estaj $|$(es) sei/sein wird\\
2.&40.&40.&175.&175.&6.&3&54&40\_5\_9 \textcolor{red}{$\boldsymbol{\upmu\upepsilon\upvartheta}$} meT $|$mit\\
3.&41.&41.&178.&178.&9.&4&898&8\_40\_800\_50 \textcolor{red}{$\boldsymbol{\upeta\upmu\upomega\upnu}$} "amOn $|$euch/uns\\
4.&42.&42.&182.&182.&13.&5&911&600\_1\_100\_10\_200 \textcolor{red}{$\boldsymbol{\upchi\upalpha\uprho\upiota\upsigma}$} carjs $|$Gnade\\
5.&43.&43.&187.&187.&18.&5&310&5\_30\_5\_70\_200 \textcolor{red}{$\boldsymbol{\upepsilon\uplambda\upepsilon\mathrm{o}\upsigma}$} eleos $|$Barmherzigkeit\\
6.&44.&44.&192.&192.&23.&6&181&5\_10\_100\_8\_50\_8 \textcolor{red}{$\boldsymbol{\upepsilon\upiota\uprho\upeta\upnu\upeta}$} ejr"an"a $|$Friede\\
7.&45.&45.&198.&198.&29.&4&182&80\_1\_100\_1 \textcolor{red}{$\boldsymbol{\uppi\upalpha\uprho\upalpha}$} para $|$von\\
8.&46.&46.&202.&202.&33.&4&484&9\_5\_70\_400 \textcolor{red}{$\boldsymbol{\upvartheta\upepsilon\mathrm{o}\upsilon}$} Teo"u $|$Gott\\
9.&47.&47.&206.&206.&37.&6&751&80\_1\_300\_100\_70\_200 \textcolor{red}{$\boldsymbol{\uppi\upalpha\uptau\uprho\mathrm{o}\upsigma}$} patros $|$(dem) Vater\\
10.&48.&48.&212.&212.&43.&3&31&20\_1\_10 \textcolor{red}{$\boldsymbol{\upkappa\upalpha\upiota}$} kaj $|$und\\
11.&49.&49.&215.&215.&46.&4&182&80\_1\_100\_1 \textcolor{red}{$\boldsymbol{\uppi\upalpha\uprho\upalpha}$} para $|$von\\
12.&50.&50.&219.&219.&50.&6&1000&20\_400\_100\_10\_70\_400 \textcolor{red}{$\boldsymbol{\upkappa\upsilon\uprho\upiota\mathrm{o}\upsilon}$} k"urjo"u $|$dem Herrn//\\
13.&51.&51.&225.&225.&56.&5&688&10\_8\_200\_70\_400 \textcolor{red}{$\boldsymbol{\upiota\upeta\upsigma\mathrm{o}\upsilon}$} j"aso"u $|$Jesus///$<$Jahwe ist Heil$>$\\
14.&52.&52.&230.&230.&61.&7&1680&600\_100\_10\_200\_300\_70\_400 \textcolor{red}{$\boldsymbol{\upchi\uprho\upiota\upsigma\uptau\mathrm{o}\upsilon}$} crjsto"u $|$Christus\\
15.&53.&53.&237.&237.&68.&3&770&300\_70\_400 \textcolor{red}{$\boldsymbol{\uptau\mathrm{o}\upsilon}$} to"u $|$dem\\
16.&54.&54.&240.&240.&71.&4&880&400\_10\_70\_400 \textcolor{red}{$\boldsymbol{\upsilon\upiota\mathrm{o}\upsilon}$} "ujo"u $|$Sohn\\
17.&55.&55.&244.&244.&75.&3&770&300\_70\_400 \textcolor{red}{$\boldsymbol{\uptau\mathrm{o}\upsilon}$} to"u $|$des\\
18.&56.&56.&247.&247.&78.&6&751&80\_1\_300\_100\_70\_200 \textcolor{red}{$\boldsymbol{\uppi\upalpha\uptau\uprho\mathrm{o}\upsigma}$} patros $|$Vaters\\
19.&57.&57.&253.&253.&84.&2&55&5\_50 \textcolor{red}{$\boldsymbol{\upepsilon\upnu}$} en $|$in\\
20.&58.&58.&255.&255.&86.&7&64&1\_30\_8\_9\_5\_10\_1 \textcolor{red}{$\boldsymbol{\upalpha\uplambda\upeta\upvartheta\upepsilon\upiota\upalpha}$} al"aTeja $|$Wahrheit\\
21.&59.&59.&262.&262.&93.&3&31&20\_1\_10 \textcolor{red}{$\boldsymbol{\upkappa\upalpha\upiota}$} kaj $|$und\\
22.&60.&60.&265.&265.&96.&5&93&1\_3\_1\_80\_8 \textcolor{red}{$\boldsymbol{\upalpha\upgamma\upalpha\uppi\upeta}$} agap"a $|$Liebe\\
\end{tabular}\medskip \\
Ende des Verses 1.3\\
Verse: 3, Buchstaben: 100, 269, 269, Totalwerte: 11282, 27703, 27703\\
\\
Es wird mit euch sein Gnade, Barmherzigkeit, Friede von Gott, dem Vater, und von dem Herrn Jesus Christus, dem Sohne des Vaters, in Wahrheit und Liebe.\\
\newpage 
{\bf -- 1.4}\\
\medskip \\
\begin{tabular}{rrrrrrrrp{120mm}}
WV&WK&WB&ABK&ABB&ABV&AnzB&TW&Zahlencode \textcolor{red}{$\boldsymbol{Grundtext}$} Umschrift $|$"Ubersetzung(en)\\
1.&61.&61.&270.&270.&1.&6&764&5\_600\_1\_100\_8\_50 \textcolor{red}{$\boldsymbol{\upepsilon\upchi\upalpha\uprho\upeta\upnu}$} ecar"an $|$es freut mich/ich habe mich gefreut\\
2.&62.&62.&276.&276.&7.&4&91&30\_10\_1\_50 \textcolor{red}{$\boldsymbol{\uplambda\upiota\upalpha\upnu}$} ljan $|$sehr\\
3.&63.&63.&280.&280.&11.&3&380&70\_300\_10 \textcolor{red}{$\boldsymbol{\mathrm{o}\uptau\upiota}$} otj $|$dass\\
4.&64.&64.&283.&283.&14.&6&534&5\_400\_100\_8\_20\_1 \textcolor{red}{$\boldsymbol{\upepsilon\upsilon\uprho\upeta\upkappa\upalpha}$} e"ur"aka $|$ich gefunden habe (solche)\\
5.&65.&65.&289.&289.&20.&2&25&5\_20 \textcolor{red}{$\boldsymbol{\upepsilon\upkappa}$} ek $|$unter\\
6.&66.&66.&291.&291.&22.&3&1150&300\_800\_50 \textcolor{red}{$\boldsymbol{\uptau\upomega\upnu}$} tOn $|$(den)\\
7.&67.&67.&294.&294.&25.&6&1225&300\_5\_20\_50\_800\_50 \textcolor{red}{$\boldsymbol{\uptau\upepsilon\upkappa\upnu\upomega\upnu}$} teknOn $|$Kindern\\
8.&68.&68.&300.&300.&31.&3&670&200\_70\_400 \textcolor{red}{$\boldsymbol{\upsigma\mathrm{o}\upsilon}$} so"u $|$deinen\\
9.&69.&69.&303.&303.&34.&13&1597&80\_5\_100\_10\_80\_1\_300\_70\_400\_50\_300\_1\_200 \textcolor{red}{$\boldsymbol{\uppi\upepsilon\uprho\upiota\uppi\upalpha\uptau\mathrm{o}\upsilon\upnu\uptau\upalpha\upsigma}$} perjpato"untas $|$die wandeln/Wandelnde\\
10.&70.&70.&316.&316.&47.&2&55&5\_50 \textcolor{red}{$\boldsymbol{\upepsilon\upnu}$} en $|$in\\
11.&71.&71.&318.&318.&49.&7&64&1\_30\_8\_9\_5\_10\_1 \textcolor{red}{$\boldsymbol{\upalpha\uplambda\upeta\upvartheta\upepsilon\upiota\upalpha}$} al"aTeja $|$(der) Wahrheit\\
12.&72.&72.&325.&325.&56.&5&1030&20\_1\_9\_800\_200 \textcolor{red}{$\boldsymbol{\upkappa\upalpha\upvartheta\upomega\upsigma}$} kaTOs $|$wie\\
13.&73.&73.&330.&330.&61.&7&513&5\_50\_300\_70\_30\_8\_50 \textcolor{red}{$\boldsymbol{\upepsilon\upnu\uptau\mathrm{o}\uplambda\upeta\upnu}$} entol"an $|$ein Gebot\\
14.&74.&74.&337.&337.&68.&8&203&5\_30\_1\_2\_70\_40\_5\_50 \textcolor{red}{$\boldsymbol{\upepsilon\uplambda\upalpha\upbeta\mathrm{o}\upmu\upepsilon\upnu}$} elabomen $|$wir empfangen haben\\
15.&75.&75.&345.&345.&76.&4&182&80\_1\_100\_1 \textcolor{red}{$\boldsymbol{\uppi\upalpha\uprho\upalpha}$} para $|$von\\
16.&76.&76.&349.&349.&80.&3&770&300\_70\_400 \textcolor{red}{$\boldsymbol{\uptau\mathrm{o}\upsilon}$} to"u $|$dem\\
17.&77.&77.&352.&352.&83.&6&751&80\_1\_300\_100\_70\_200 \textcolor{red}{$\boldsymbol{\uppi\upalpha\uptau\uprho\mathrm{o}\upsigma}$} patros $|$Vater\\
\end{tabular}\medskip \\
Ende des Verses 1.4\\
Verse: 4, Buchstaben: 88, 357, 357, Totalwerte: 10004, 37707, 37707\\
\\
Ich freute mich sehr, da"s ich einige von deinen Kindern in der Wahrheit wandelnd gefunden habe, wie wir von dem Vater ein Gebot empfangen haben.\\
\newpage 
{\bf -- 1.5}\\
\medskip \\
\begin{tabular}{rrrrrrrrp{120mm}}
WV&WK&WB&ABK&ABB&ABV&AnzB&TW&Zahlencode \textcolor{red}{$\boldsymbol{Grundtext}$} Umschrift $|$"Ubersetzung(en)\\
1.&78.&78.&358.&358.&1.&3&31&20\_1\_10 \textcolor{red}{$\boldsymbol{\upkappa\upalpha\upiota}$} kaj $|$und\\
2.&79.&79.&361.&361.&4.&3&500&50\_400\_50 \textcolor{red}{$\boldsymbol{\upnu\upsilon\upnu}$} n"un $|$nun/jetzt\\
3.&80.&80.&364.&364.&7.&5&2005&5\_100\_800\_300\_800 \textcolor{red}{$\boldsymbol{\upepsilon\uprho\upomega\uptau\upomega}$} erOtO $|$bitte ich\\
4.&81.&81.&369.&369.&12.&2&205&200\_5 \textcolor{red}{$\boldsymbol{\upsigma\upepsilon}$} se $|$dich\\
5.&82.&82.&371.&371.&14.&5&531&20\_400\_100\_10\_1 \textcolor{red}{$\boldsymbol{\upkappa\upsilon\uprho\upiota\upalpha}$} k"urja $|$Frau/Herrin\\
6.&83.&83.&376.&376.&19.&3&1070&70\_400\_600 \textcolor{red}{$\boldsymbol{\mathrm{o}\upsilon\upchi}$} o"uc $|$nicht\\
7.&84.&84.&379.&379.&22.&2&1000&800\_200 \textcolor{red}{$\boldsymbol{\upomega\upsigma}$} Os $|$als ob/wie\\
8.&85.&85.&381.&381.&24.&7&513&5\_50\_300\_70\_30\_8\_50 \textcolor{red}{$\boldsymbol{\upepsilon\upnu\uptau\mathrm{o}\uplambda\upeta\upnu}$} entol"an $|$ein Gebot\\
9.&86.&86.&388.&388.&31.&5&1404&3\_100\_1\_500\_800 \textcolor{red}{$\boldsymbol{\upgamma\uprho\upalpha\upvarphi\upomega}$} grafO $|$ich schreiben w"urde/schreibend\\
10.&87.&87.&393.&393.&36.&3&280&200\_70\_10 \textcolor{red}{$\boldsymbol{\upsigma\mathrm{o}\upiota}$} soj $|$dir\\
11.&88.&88.&396.&396.&39.&6&139&20\_1\_10\_50\_8\_50 \textcolor{red}{$\boldsymbol{\upkappa\upalpha\upiota\upnu\upeta\upnu}$} kajn"an $|$neues\\
12.&89.&89.&402.&402.&45.&4&62&1\_30\_30\_1 \textcolor{red}{$\boldsymbol{\upalpha\uplambda\uplambda\upalpha}$} alla $|$sondern\\
13.&90.&90.&406.&406.&49.&2&58&8\_50 \textcolor{red}{$\boldsymbol{\upeta\upnu}$} "an $|$dasjenige welches/das was\\
14.&91.&91.&408.&408.&51.&7&780&5\_10\_600\_70\_40\_5\_50 \textcolor{red}{$\boldsymbol{\upepsilon\upiota\upchi\mathrm{o}\upmu\upepsilon\upnu}$} ejcomen $|$wir gehabt haben/wir hatten\\
15.&92.&92.&415.&415.&58.&2&81&1\_80 \textcolor{red}{$\boldsymbol{\upalpha\uppi}$} ap $|$von\\
16.&93.&93.&417.&417.&60.&5&909&1\_100\_600\_8\_200 \textcolor{red}{$\boldsymbol{\upalpha\uprho\upchi\upeta\upsigma}$} arc"as $|$Anfang an\\
17.&94.&94.&422.&422.&65.&3&61&10\_50\_1 \textcolor{red}{$\boldsymbol{\upiota\upnu\upalpha}$} jna $|$dass\\
18.&95.&95.&425.&425.&68.&8&980&1\_3\_1\_80\_800\_40\_5\_50 \textcolor{red}{$\boldsymbol{\upalpha\upgamma\upalpha\uppi\upomega\upmu\upepsilon\upnu}$} agapOmen $|$wir lieben (sollen)\\
19.&96.&96.&433.&433.&76.&8&769&1\_30\_30\_8\_30\_70\_400\_200 \textcolor{red}{$\boldsymbol{\upalpha\uplambda\uplambda\upeta\uplambda\mathrm{o}\upsilon\upsigma}$} all"alo"us $|$einander\\
\end{tabular}\medskip \\
Ende des Verses 1.5\\
Verse: 5, Buchstaben: 83, 440, 440, Totalwerte: 11378, 49085, 49085\\
\\
Und nun bitte ich dich, Frau, nicht als ob ich ein neues Gebot dir schriebe, sondern das, welches wir von Anfang gehabt haben: da"s wir einander lieben sollen.\\
\newpage 
{\bf -- 1.6}\\
\medskip \\
\begin{tabular}{rrrrrrrrp{120mm}}
WV&WK&WB&ABK&ABB&ABV&AnzB&TW&Zahlencode \textcolor{red}{$\boldsymbol{Grundtext}$} Umschrift $|$"Ubersetzung(en)\\
1.&97.&97.&441.&441.&1.&3&31&20\_1\_10 \textcolor{red}{$\boldsymbol{\upkappa\upalpha\upiota}$} kaj $|$und\\
2.&98.&98.&444.&444.&4.&4&709&1\_400\_300\_8 \textcolor{red}{$\boldsymbol{\upalpha\upsilon\uptau\upeta}$} a"ut"a $|$darin/dies\\
3.&99.&99.&448.&448.&8.&5&565&5\_200\_300\_10\_50 \textcolor{red}{$\boldsymbol{\upepsilon\upsigma\uptau\upiota\upnu}$} estjn $|$besteht/ist\\
4.&100.&100.&453.&453.&13.&1&8&8 \textcolor{red}{$\boldsymbol{\upeta}$} "a $|$die\\
5.&101.&101.&454.&454.&14.&5&93&1\_3\_1\_80\_8 \textcolor{red}{$\boldsymbol{\upalpha\upgamma\upalpha\uppi\upeta}$} agap"a $|$Liebe\\
6.&102.&102.&459.&459.&19.&3&61&10\_50\_1 \textcolor{red}{$\boldsymbol{\upiota\upnu\upalpha}$} jna $|$dass\\
7.&103.&103.&462.&462.&22.&11&1471&80\_5\_100\_10\_80\_1\_300\_800\_40\_5\_50 \textcolor{red}{$\boldsymbol{\uppi\upepsilon\uprho\upiota\uppi\upalpha\uptau\upomega\upmu\upepsilon\upnu}$} perjpatOmen $|$wir wandeln\\
8.&104.&104.&473.&473.&33.&4&322&20\_1\_300\_1 \textcolor{red}{$\boldsymbol{\upkappa\upalpha\uptau\upalpha}$} kata $|$nach\\
9.&105.&105.&477.&477.&37.&3&501&300\_1\_200 \textcolor{red}{$\boldsymbol{\uptau\upalpha\upsigma}$} tas $|$(den)\\
10.&106.&106.&480.&480.&40.&7&656&5\_50\_300\_70\_30\_1\_200 \textcolor{red}{$\boldsymbol{\upepsilon\upnu\uptau\mathrm{o}\uplambda\upalpha\upsigma}$} entolas $|$Geboten\\
11.&107.&107.&487.&487.&47.&5&1171&1\_400\_300\_70\_400 \textcolor{red}{$\boldsymbol{\upalpha\upsilon\uptau\mathrm{o}\upsilon}$} a"uto"u $|$seinen\\
12.&108.&108.&492.&492.&52.&4&709&1\_400\_300\_8 \textcolor{red}{$\boldsymbol{\upalpha\upsilon\uptau\upeta}$} a"ut"a $|$dies\\
13.&109.&109.&496.&496.&56.&5&565&5\_200\_300\_10\_50 \textcolor{red}{$\boldsymbol{\upepsilon\upsigma\uptau\upiota\upnu}$} estjn $|$ist\\
14.&110.&110.&501.&501.&61.&1&8&8 \textcolor{red}{$\boldsymbol{\upeta}$} "a $|$das\\
15.&111.&111.&502.&502.&62.&6&463&5\_50\_300\_70\_30\_8 \textcolor{red}{$\boldsymbol{\upepsilon\upnu\uptau\mathrm{o}\uplambda\upeta}$} entol"a $|$Gebot\\
16.&112.&112.&508.&508.&68.&5&1030&20\_1\_9\_800\_200 \textcolor{red}{$\boldsymbol{\upkappa\upalpha\upvartheta\upomega\upsigma}$} kaTOs $|$wie\\
17.&113.&113.&513.&513.&73.&8&1004&8\_20\_70\_400\_200\_1\_300\_5 \textcolor{red}{$\boldsymbol{\upeta\upkappa\mathrm{o}\upsilon\upsigma\upalpha\uptau\upepsilon}$} "ako"usate $|$ihr (es) geh"ort habt\\
18.&114.&114.&521.&521.&81.&2&81&1\_80 \textcolor{red}{$\boldsymbol{\upalpha\uppi}$} ap $|$von\\
19.&115.&115.&523.&523.&83.&5&909&1\_100\_600\_8\_200 \textcolor{red}{$\boldsymbol{\upalpha\uprho\upchi\upeta\upsigma}$} arc"as $|$Anfang an\\
20.&116.&116.&528.&528.&88.&3&61&10\_50\_1 \textcolor{red}{$\boldsymbol{\upiota\upnu\upalpha}$} jna $|$dass\\
21.&117.&117.&531.&531.&91.&2&55&5\_50 \textcolor{red}{$\boldsymbol{\upepsilon\upnu}$} en $|$darin/in\\
22.&118.&118.&533.&533.&93.&4&709&1\_400\_300\_8 \textcolor{red}{$\boldsymbol{\upalpha\upsilon\uptau\upeta}$} a"ut"a $|$/ihr\\
23.&119.&119.&537.&537.&97.&10&889&80\_5\_100\_10\_80\_1\_300\_8\_300\_5 \textcolor{red}{$\boldsymbol{\uppi\upepsilon\uprho\upiota\uppi\upalpha\uptau\upeta\uptau\upepsilon}$} perjpat"ate $|$ihr wandeln sollt/ihr wandelt\\
\end{tabular}\medskip \\
Ende des Verses 1.6\\
Verse: 6, Buchstaben: 106, 546, 546, Totalwerte: 12071, 61156, 61156\\
\\
Und dies ist die Liebe, da"s wir nach seinen Geboten wandeln. Dies ist das Gebot, wie ihr von Anfang geh"ort habt, da"s ihr darin wandeln sollt.\\
\newpage 
{\bf -- 1.7}\\
\medskip \\
\begin{tabular}{rrrrrrrrp{120mm}}
WV&WK&WB&ABK&ABB&ABV&AnzB&TW&Zahlencode \textcolor{red}{$\boldsymbol{Grundtext}$} Umschrift $|$"Ubersetzung(en)\\
1.&120.&120.&547.&547.&1.&3&380&70\_300\_10 \textcolor{red}{$\boldsymbol{\mathrm{o}\uptau\upiota}$} otj $|$denn\\
2.&121.&121.&550.&550.&4.&6&290&80\_70\_30\_30\_70\_10 \textcolor{red}{$\boldsymbol{\uppi\mathrm{o}\uplambda\uplambda\mathrm{o}\upiota}$} polloj $|$viele\\
3.&122.&122.&556.&556.&10.&6&241&80\_30\_1\_50\_70\_10 \textcolor{red}{$\boldsymbol{\uppi\uplambda\upalpha\upnu\mathrm{o}\upiota}$} planoj $|$Verf"uhrer\\
4.&123.&123.&562.&562.&16.&8&382&5\_10\_200\_8\_30\_9\_70\_50 \textcolor{red}{$\boldsymbol{\upepsilon\upiota\upsigma\upeta\uplambda\upvartheta\mathrm{o}\upnu}$} ejs"alTon $|$sind hineingekommen/sind ausgegangen\\
5.&124.&124.&570.&570.&24.&3&215&5\_10\_200 \textcolor{red}{$\boldsymbol{\upepsilon\upiota\upsigma}$} ejs $|$in\\
6.&125.&125.&573.&573.&27.&3&420&300\_70\_50 \textcolor{red}{$\boldsymbol{\uptau\mathrm{o}\upnu}$} ton $|$die\\
7.&126.&126.&576.&576.&30.&6&450&20\_70\_200\_40\_70\_50 \textcolor{red}{$\boldsymbol{\upkappa\mathrm{o}\upsigma\upmu\mathrm{o}\upnu}$} kosmon $|$Welt\\
8.&127.&127.&582.&582.&36.&2&80&70\_10 \textcolor{red}{$\boldsymbol{\mathrm{o}\upiota}$} oj $|$die\\
9.&128.&128.&584.&584.&38.&2&48&40\_8 \textcolor{red}{$\boldsymbol{\upmu\upeta}$} m"a $|$nicht\\
10.&129.&129.&586.&586.&40.&12&1308&70\_40\_70\_30\_70\_3\_70\_400\_50\_300\_5\_200 \textcolor{red}{$\boldsymbol{\mathrm{o}\upmu\mathrm{o}\uplambda\mathrm{o}\upgamma\mathrm{o}\upsilon\upnu\uptau\upepsilon\upsigma}$} omologo"untes $|$bekennen/ Bekennenden\\
11.&130.&130.&598.&598.&52.&6&738&10\_8\_200\_70\_400\_50 \textcolor{red}{$\boldsymbol{\upiota\upeta\upsigma\mathrm{o}\upsilon\upnu}$} j"aso"un $|$(dass) Jesus\\
12.&131.&131.&604.&604.&58.&7&1330&600\_100\_10\_200\_300\_70\_50 \textcolor{red}{$\boldsymbol{\upchi\uprho\upiota\upsigma\uptau\mathrm{o}\upnu}$} crjston $|$Christus\\
13.&132.&132.&611.&611.&65.&9&990&5\_100\_600\_70\_40\_5\_50\_70\_50 \textcolor{red}{$\boldsymbol{\upepsilon\uprho\upchi\mathrm{o}\upmu\upepsilon\upnu\mathrm{o}\upnu}$} ercomenon $|$gekommen ist/als Kommenden\\
14.&133.&133.&620.&620.&74.&2&55&5\_50 \textcolor{red}{$\boldsymbol{\upepsilon\upnu}$} en $|$in\\
15.&134.&134.&622.&622.&76.&5&331&200\_1\_100\_20\_10 \textcolor{red}{$\boldsymbol{\upsigma\upalpha\uprho\upkappa\upiota}$} sarkj $|$(dem) Fleisch\\
16.&135.&135.&627.&627.&81.&5&1040&70\_400\_300\_70\_200 \textcolor{red}{$\boldsymbol{\mathrm{o}\upsilon\uptau\mathrm{o}\upsigma}$} o"utos $|$das/dies\\
17.&136.&136.&632.&632.&86.&5&565&5\_200\_300\_10\_50 \textcolor{red}{$\boldsymbol{\upepsilon\upsigma\uptau\upiota\upnu}$} estjn $|$ist\\
18.&137.&137.&637.&637.&91.&1&70&70 \textcolor{red}{$\boldsymbol{\mathrm{o}}$} o $|$der\\
19.&138.&138.&638.&638.&92.&6&431&80\_30\_1\_50\_70\_200 \textcolor{red}{$\boldsymbol{\uppi\uplambda\upalpha\upnu\mathrm{o}\upsigma}$} planos $|$Verf"uhrer/Betr"uger\\
20.&139.&139.&644.&644.&98.&3&31&20\_1\_10 \textcolor{red}{$\boldsymbol{\upkappa\upalpha\upiota}$} kaj $|$und\\
21.&140.&140.&647.&647.&101.&1&70&70 \textcolor{red}{$\boldsymbol{\mathrm{o}}$} o $|$der\\
22.&141.&141.&648.&648.&102.&11&1841&1\_50\_300\_10\_600\_100\_10\_200\_300\_70\_200 \textcolor{red}{$\boldsymbol{\upalpha\upnu\uptau\upiota\upchi\uprho\upiota\upsigma\uptau\mathrm{o}\upsigma}$} antjcrjstos $|$Antichrist\\
\end{tabular}\medskip \\
Ende des Verses 1.7\\
Verse: 7, Buchstaben: 112, 658, 658, Totalwerte: 11306, 72462, 72462\\
\\
Denn viele Verf"uhrer sind in die Welt ausgegangen, die nicht Jesum Christum im Fleische kommend bekennen; dies ist der Verf"uhrer und der Antichrist.\\
\newpage 
{\bf -- 1.8}\\
\medskip \\
\begin{tabular}{rrrrrrrrp{120mm}}
WV&WK&WB&ABK&ABB&ABV&AnzB&TW&Zahlencode \textcolor{red}{$\boldsymbol{Grundtext}$} Umschrift $|$"Ubersetzung(en)\\
1.&142.&142.&659.&659.&1.&7&427&2\_30\_5\_80\_5\_300\_5 \textcolor{red}{$\boldsymbol{\upbeta\uplambda\upepsilon\uppi\upepsilon\uptau\upepsilon}$} blepete $|$seht vor\\
2.&143.&143.&666.&666.&8.&7&1376&5\_1\_400\_300\_70\_400\_200 \textcolor{red}{$\boldsymbol{\upepsilon\upalpha\upsilon\uptau\mathrm{o}\upsilon\upsigma}$} ea"uto"us $|$euch\\
3.&144.&144.&673.&673.&15.&3&61&10\_50\_1 \textcolor{red}{$\boldsymbol{\upiota\upnu\upalpha}$} jna $|$dass\\
4.&145.&145.&676.&676.&18.&2&48&40\_8 \textcolor{red}{$\boldsymbol{\upmu\upeta}$} m"a $|$nicht\\
5.&146.&146.&678.&678.&20.&10&1281&1\_80\_70\_30\_5\_200\_800\_40\_5\_50 \textcolor{red}{$\boldsymbol{\upalpha\uppi\mathrm{o}\uplambda\upepsilon\upsigma\upomega\upmu\upepsilon\upnu}$} apolesOmen $|$wir verlieren/ihr verliert\\
6.&147.&147.&688.&688.&30.&1&1&1 \textcolor{red}{$\boldsymbol{\upalpha}$} a $|$was\\
7.&148.&148.&689.&689.&31.&11&375&5\_10\_100\_3\_1\_200\_1\_40\_5\_9\_1 \textcolor{red}{$\boldsymbol{\upepsilon\upiota\uprho\upgamma\upalpha\upsigma\upalpha\upmu\upepsilon\upvartheta\upalpha}$} ejrgasameTa $|$wir erarbeitet haben\\
8.&149.&149.&700.&700.&42.&4&62&1\_30\_30\_1 \textcolor{red}{$\boldsymbol{\upalpha\uplambda\uplambda\upalpha}$} alla $|$sondern\\
9.&150.&150.&704.&704.&46.&6&379&40\_10\_200\_9\_70\_50 \textcolor{red}{$\boldsymbol{\upmu\upiota\upsigma\upvartheta\mathrm{o}\upnu}$} mjsTon $|$(den) Lohn\\
10.&151.&151.&710.&710.&52.&5&226&80\_30\_8\_100\_8 \textcolor{red}{$\boldsymbol{\uppi\uplambda\upeta\uprho\upeta}$} pl"ar"a $|$vollen\\
11.&152.&152.&715.&715.&57.&10&1079&1\_80\_70\_30\_1\_2\_800\_40\_5\_50 \textcolor{red}{$\boldsymbol{\upalpha\uppi\mathrm{o}\uplambda\upalpha\upbeta\upomega\upmu\upepsilon\upnu}$} apolabOmen $|$empfangen/ empfangt\\
\end{tabular}\medskip \\
Ende des Verses 1.8\\
Verse: 8, Buchstaben: 66, 724, 724, Totalwerte: 5315, 77777, 77777\\
\\
Sehet auf euch selbst, auf da"s wir nicht verlieren, was wir erarbeitet haben, sondern vollen Lohn empfangen.\\
\newpage 
{\bf -- 1.9}\\
\medskip \\
\begin{tabular}{rrrrrrrrp{120mm}}
WV&WK&WB&ABK&ABB&ABV&AnzB&TW&Zahlencode \textcolor{red}{$\boldsymbol{Grundtext}$} Umschrift $|$"Ubersetzung(en)\\
1.&153.&153.&725.&725.&1.&3&281&80\_1\_200 \textcolor{red}{$\boldsymbol{\uppi\upalpha\upsigma}$} pas $|$jeder\\
2.&154.&154.&728.&728.&4.&1&70&70 \textcolor{red}{$\boldsymbol{\mathrm{o}}$} o $|$der/(der)\\
3.&155.&155.&729.&729.&5.&10&1095&80\_1\_100\_1\_2\_1\_10\_50\_800\_50 \textcolor{red}{$\boldsymbol{\uppi\upalpha\uprho\upalpha\upbeta\upalpha\upiota\upnu\upomega\upnu}$} parabajnOn $|$abweicht/ Weitergehende\\
4.&156.&156.&739.&739.&15.&3&31&20\_1\_10 \textcolor{red}{$\boldsymbol{\upkappa\upalpha\upiota}$} kaj $|$und\\
5.&157.&157.&742.&742.&18.&2&48&40\_8 \textcolor{red}{$\boldsymbol{\upmu\upeta}$} m"a $|$nicht\\
6.&158.&158.&744.&744.&20.&5&945&40\_5\_50\_800\_50 \textcolor{red}{$\boldsymbol{\upmu\upepsilon\upnu\upomega\upnu}$} menOn $|$bleibt/Bleibende\\
7.&159.&159.&749.&749.&25.&2&55&5\_50 \textcolor{red}{$\boldsymbol{\upepsilon\upnu}$} en $|$in\\
8.&160.&160.&751.&751.&27.&2&308&300\_8 \textcolor{red}{$\boldsymbol{\uptau\upeta}$} t"a $|$der\\
9.&161.&161.&753.&753.&29.&6&627&4\_10\_4\_1\_600\_8 \textcolor{red}{$\boldsymbol{\updelta\upiota\updelta\upalpha\upchi\upeta}$} djdac"a $|$Lehre\\
10.&162.&162.&759.&759.&35.&3&770&300\_70\_400 \textcolor{red}{$\boldsymbol{\uptau\mathrm{o}\upsilon}$} to"u $|$(des)\\
11.&163.&163.&762.&762.&38.&7&1680&600\_100\_10\_200\_300\_70\_400 \textcolor{red}{$\boldsymbol{\upchi\uprho\upiota\upsigma\uptau\mathrm{o}\upsilon}$} crjsto"u $|$Christus/Christi\\
12.&164.&164.&769.&769.&45.&4&134&9\_5\_70\_50 \textcolor{red}{$\boldsymbol{\upvartheta\upepsilon\mathrm{o}\upnu}$} Teon $|$Gott\\
13.&165.&165.&773.&773.&49.&3&490&70\_400\_20 \textcolor{red}{$\boldsymbol{\mathrm{o}\upsilon\upkappa}$} o"uk $|$nicht\\
14.&166.&166.&776.&776.&52.&4&620&5\_600\_5\_10 \textcolor{red}{$\boldsymbol{\upepsilon\upchi\upepsilon\upiota}$} ecej $|$(der) hat\\
15.&167.&167.&780.&780.&56.&1&70&70 \textcolor{red}{$\boldsymbol{\mathrm{o}}$} o $|$wer/der\\
16.&168.&168.&781.&781.&57.&5&945&40\_5\_50\_800\_50 \textcolor{red}{$\boldsymbol{\upmu\upepsilon\upnu\upomega\upnu}$} menOn $|$bleibt/Bleibende\\
17.&169.&169.&786.&786.&62.&2&55&5\_50 \textcolor{red}{$\boldsymbol{\upepsilon\upnu}$} en $|$in\\
18.&170.&170.&788.&788.&64.&2&308&300\_8 \textcolor{red}{$\boldsymbol{\uptau\upeta}$} t"a $|$der\\
19.&171.&171.&790.&790.&66.&6&627&4\_10\_4\_1\_600\_8 \textcolor{red}{$\boldsymbol{\updelta\upiota\updelta\upalpha\upchi\upeta}$} djdac"a $|$Lehre\\
20.&172.&172.&796.&796.&72.&3&770&300\_70\_400 \textcolor{red}{$\boldsymbol{\uptau\mathrm{o}\upsilon}$} to"u $|$des//\\
21.&173.&173.&799.&799.&75.&7&1680&600\_100\_10\_200\_300\_70\_400 \textcolor{red}{$\boldsymbol{\upchi\uprho\upiota\upsigma\uptau\mathrm{o}\upsilon}$} crjsto"u $|$Christus//\\
22.&174.&174.&806.&806.&82.&5&1040&70\_400\_300\_70\_200 \textcolor{red}{$\boldsymbol{\mathrm{o}\upsilon\uptau\mathrm{o}\upsigma}$} o"utos $|$der\\
23.&175.&175.&811.&811.&87.&3&31&20\_1\_10 \textcolor{red}{$\boldsymbol{\upkappa\upalpha\upiota}$} kaj $|$/sowohl\\
24.&176.&176.&814.&814.&90.&3&420&300\_70\_50 \textcolor{red}{$\boldsymbol{\uptau\mathrm{o}\upnu}$} ton $|$den\\
25.&177.&177.&817.&817.&93.&6&487&80\_1\_300\_5\_100\_1 \textcolor{red}{$\boldsymbol{\uppi\upalpha\uptau\upepsilon\uprho\upalpha}$} patera $|$Vater\\
\end{tabular}
\newpage
\begin{tabular}{rrrrrrrrp{120mm}}
WV&WK&WB&ABK&ABB&ABV&AnzB&TW&Zahlencode \textcolor{red}{$\boldsymbol{Grundtext}$} Umschrift $|$"Ubersetzung(en)\\
26.&178.&178.&823.&823.&99.&3&31&20\_1\_10 \textcolor{red}{$\boldsymbol{\upkappa\upalpha\upiota}$} kaj $|$und/als auch\\
27.&179.&179.&826.&826.&102.&3&420&300\_70\_50 \textcolor{red}{$\boldsymbol{\uptau\mathrm{o}\upnu}$} ton $|$den\\
28.&180.&180.&829.&829.&105.&4&530&400\_10\_70\_50 \textcolor{red}{$\boldsymbol{\upsilon\upiota\mathrm{o}\upnu}$} "ujon $|$Sohn\\
29.&181.&181.&833.&833.&109.&4&620&5\_600\_5\_10 \textcolor{red}{$\boldsymbol{\upepsilon\upchi\upepsilon\upiota}$} ecej $|$hat\\
\end{tabular}\medskip \\
Ende des Verses 1.9\\
Verse: 9, Buchstaben: 112, 836, 836, Totalwerte: 15188, 92965, 92965\\
\\
Jeder, der weitergeht und nicht bleibt in der Lehre des Christus, hat Gott nicht; wer in der Lehre bleibt, dieser hat sowohl den Vater als auch den Sohn.\\
\newpage 
{\bf -- 1.10}\\
\medskip \\
\begin{tabular}{rrrrrrrrp{120mm}}
WV&WK&WB&ABK&ABB&ABV&AnzB&TW&Zahlencode \textcolor{red}{$\boldsymbol{Grundtext}$} Umschrift $|$"Ubersetzung(en)\\
1.&182.&182.&837.&837.&1.&2&15&5\_10 \textcolor{red}{$\boldsymbol{\upepsilon\upiota}$} ej $|$wenn\\
2.&183.&183.&839.&839.&3.&3&510&300\_10\_200 \textcolor{red}{$\boldsymbol{\uptau\upiota\upsigma}$} tjs $|$jemand\\
3.&184.&184.&842.&842.&6.&7&1021&5\_100\_600\_5\_300\_1\_10 \textcolor{red}{$\boldsymbol{\upepsilon\uprho\upchi\upepsilon\uptau\upalpha\upiota}$} ercetaj $|$kommt\\
4.&185.&185.&849.&849.&13.&4&450&80\_100\_70\_200 \textcolor{red}{$\boldsymbol{\uppi\uprho\mathrm{o}\upsigma}$} pros $|$zu\\
5.&186.&186.&853.&853.&17.&4&641&400\_40\_1\_200 \textcolor{red}{$\boldsymbol{\upsilon\upmu\upalpha\upsigma}$} "umas $|$euch\\
6.&187.&187.&857.&857.&21.&3&31&20\_1\_10 \textcolor{red}{$\boldsymbol{\upkappa\upalpha\upiota}$} kaj $|$und\\
7.&188.&188.&860.&860.&24.&6&1059&300\_1\_400\_300\_8\_50 \textcolor{red}{$\boldsymbol{\uptau\upalpha\upsilon\uptau\upeta\upnu}$} ta"ut"an $|$diese\\
8.&189.&189.&866.&866.&30.&3&358&300\_8\_50 \textcolor{red}{$\boldsymbol{\uptau\upeta\upnu}$} t"an $|$(die)\\
9.&190.&190.&869.&869.&33.&7&677&4\_10\_4\_1\_600\_8\_50 \textcolor{red}{$\boldsymbol{\updelta\upiota\updelta\upalpha\upchi\upeta\upnu}$} djdac"an $|$Lehre\\
10.&191.&191.&876.&876.&40.&2&470&70\_400 \textcolor{red}{$\boldsymbol{\mathrm{o}\upsilon}$} o"u $|$nicht\\
11.&192.&192.&878.&878.&42.&5&620&500\_5\_100\_5\_10 \textcolor{red}{$\boldsymbol{\upvarphi\upepsilon\uprho\upepsilon\upiota}$} ferej $|$bringt\\
12.&193.&193.&883.&883.&47.&2&48&40\_8 \textcolor{red}{$\boldsymbol{\upmu\upeta}$} m"a $|$nicht\\
13.&194.&194.&885.&885.&49.&9&434&30\_1\_40\_2\_1\_50\_5\_300\_5 \textcolor{red}{$\boldsymbol{\uplambda\upalpha\upmu\upbeta\upalpha\upnu\upepsilon\uptau\upepsilon}$} lambanete $|$nehmt auf\\
14.&195.&195.&894.&894.&58.&5&821&1\_400\_300\_70\_50 \textcolor{red}{$\boldsymbol{\upalpha\upsilon\uptau\mathrm{o}\upnu}$} a"uton $|$den/ihn\\
15.&196.&196.&899.&899.&63.&3&215&5\_10\_200 \textcolor{red}{$\boldsymbol{\upepsilon\upiota\upsigma}$} ejs $|$in\\
16.&197.&197.&902.&902.&66.&6&161&70\_10\_20\_10\_1\_50 \textcolor{red}{$\boldsymbol{\mathrm{o}\upiota\upkappa\upiota\upalpha\upnu}$} ojkjan $|$das Haus\\
17.&198.&198.&908.&908.&72.&3&31&20\_1\_10 \textcolor{red}{$\boldsymbol{\upkappa\upalpha\upiota}$} kaj $|$und\\
18.&199.&199.&911.&911.&75.&7&776&600\_1\_10\_100\_5\_10\_50 \textcolor{red}{$\boldsymbol{\upchi\upalpha\upiota\uprho\upepsilon\upiota\upnu}$} cajrejn $|$gr"u"st/guten Gru"s\\
19.&200.&200.&918.&918.&82.&4&1501&1\_400\_300\_800 \textcolor{red}{$\boldsymbol{\upalpha\upsilon\uptau\upomega}$} a"utO $|$ihn/ihm\\
20.&201.&201.&922.&922.&86.&2&48&40\_8 \textcolor{red}{$\boldsymbol{\upmu\upeta}$} m"a $|$nicht\\
21.&202.&202.&924.&924.&88.&6&348&30\_5\_3\_5\_300\_5 \textcolor{red}{$\boldsymbol{\uplambda\upepsilon\upgamma\upepsilon\uptau\upepsilon}$} legete $|$/sagt\\
\end{tabular}\medskip \\
Ende des Verses 1.10\\
Verse: 10, Buchstaben: 93, 929, 929, Totalwerte: 10235, 103200, 103200\\
\\
Wenn jemand zu euch kommt und diese Lehre nicht bringt, so nehmet ihn nicht ins Haus auf und gr"u"set ihn nicht.\\
\newpage 
{\bf -- 1.11}\\
\medskip \\
\begin{tabular}{rrrrrrrrp{120mm}}
WV&WK&WB&ABK&ABB&ABV&AnzB&TW&Zahlencode \textcolor{red}{$\boldsymbol{Grundtext}$} Umschrift $|$"Ubersetzung(en)\\
1.&203.&203.&930.&930.&1.&1&70&70 \textcolor{red}{$\boldsymbol{\mathrm{o}}$} o $|$wer/der\\
2.&204.&204.&931.&931.&2.&3&104&3\_1\_100 \textcolor{red}{$\boldsymbol{\upgamma\upalpha\uprho}$} gar $|$denn\\
3.&205.&205.&934.&934.&5.&5&888&30\_5\_3\_800\_50 \textcolor{red}{$\boldsymbol{\uplambda\upepsilon\upgamma\upomega\upnu}$} legOn $|$/Sagende\\
4.&206.&206.&939.&939.&10.&4&1501&1\_400\_300\_800 \textcolor{red}{$\boldsymbol{\upalpha\upsilon\uptau\upomega}$} a"utO $|$ihn/ihm\\
5.&207.&207.&943.&943.&14.&7&776&600\_1\_10\_100\_5\_10\_50 \textcolor{red}{$\boldsymbol{\upchi\upalpha\upiota\uprho\upepsilon\upiota\upnu}$} cajrejn $|$gr"u"st/guten Gru"s\\
6.&208.&208.&950.&950.&21.&8&1015&20\_70\_10\_50\_800\_50\_5\_10 \textcolor{red}{$\boldsymbol{\upkappa\mathrm{o}\upiota\upnu\upomega\upnu\upepsilon\upiota}$} kojnOnej $|$macht sich teilhaftig/nimmt teil\\
7.&209.&209.&958.&958.&29.&4&580&300\_70\_10\_200 \textcolor{red}{$\boldsymbol{\uptau\mathrm{o}\upiota\upsigma}$} tojs $|$/an (den)\\
8.&210.&210.&962.&962.&33.&6&388&5\_100\_3\_70\_10\_200 \textcolor{red}{$\boldsymbol{\upepsilon\uprho\upgamma\mathrm{o}\upiota\upsigma}$} ergojs $|$Werke(n)\\
9.&211.&211.&968.&968.&39.&5&1171&1\_400\_300\_70\_400 \textcolor{red}{$\boldsymbol{\upalpha\upsilon\uptau\mathrm{o}\upsilon}$} a"uto"u $|$seiner/seinen\\
10.&212.&212.&973.&973.&44.&4&580&300\_70\_10\_200 \textcolor{red}{$\boldsymbol{\uptau\mathrm{o}\upiota\upsigma}$} tojs $|$(den)\\
11.&213.&213.&977.&977.&48.&8&588&80\_70\_50\_8\_100\_70\_10\_200 \textcolor{red}{$\boldsymbol{\uppi\mathrm{o}\upnu\upeta\uprho\mathrm{o}\upiota\upsigma}$} pon"arojs $|$b"osen\\
\end{tabular}\medskip \\
Ende des Verses 1.11\\
Verse: 11, Buchstaben: 55, 984, 984, Totalwerte: 7661, 110861, 110861\\
\\
Denn wer ihn gr"u"st, nimmt teil an seinen b"osen Werken.\\
\newpage 
{\bf -- 1.12}\\
\medskip \\
\begin{tabular}{rrrrrrrrp{120mm}}
WV&WK&WB&ABK&ABB&ABV&AnzB&TW&Zahlencode \textcolor{red}{$\boldsymbol{Grundtext}$} Umschrift $|$"Ubersetzung(en)\\
1.&214.&214.&985.&985.&1.&5&211&80\_70\_30\_30\_1 \textcolor{red}{$\boldsymbol{\uppi\mathrm{o}\uplambda\uplambda\upalpha}$} polla $|$viel/(noch) vieles\\
2.&215.&215.&990.&990.&6.&4&1455&5\_600\_800\_50 \textcolor{red}{$\boldsymbol{\upepsilon\upchi\upomega\upnu}$} ecOn $|$h"atte ich/habend\\
3.&216.&216.&994.&994.&10.&4&500&400\_40\_10\_50 \textcolor{red}{$\boldsymbol{\upsilon\upmu\upiota\upnu}$} "umjn $|$euch\\
4.&217.&217.&998.&998.&14.&7&669&3\_100\_1\_500\_5\_10\_50 \textcolor{red}{$\boldsymbol{\upgamma\uprho\upalpha\upvarphi\upepsilon\upiota\upnu}$} grafejn $|$zu schreiben\\
5.&218.&218.&1005.&1005.&21.&3&490&70\_400\_20 \textcolor{red}{$\boldsymbol{\mathrm{o}\upsilon\upkappa}$} o"uk $|$nicht\\
6.&219.&219.&1008.&1008.&24.&9&585&8\_2\_70\_400\_30\_8\_9\_8\_50 \textcolor{red}{$\boldsymbol{\upeta\upbeta\mathrm{o}\upsilon\uplambda\upeta\upvartheta\upeta\upnu}$} "abo"ul"aT"an $|$will es aber/wollte ich\\
7.&220.&220.&1017.&1017.&33.&3&15&4\_10\_1 \textcolor{red}{$\boldsymbol{\updelta\upiota\upalpha}$} dja $|$mit\\
8.&221.&221.&1020.&1020.&36.&6&1471&600\_1\_100\_300\_70\_400 \textcolor{red}{$\boldsymbol{\upchi\upalpha\uprho\uptau\mathrm{o}\upsilon}$} carto"u $|$Papier\\
9.&222.&222.&1026.&1026.&42.&3&31&20\_1\_10 \textcolor{red}{$\boldsymbol{\upkappa\upalpha\upiota}$} kaj $|$und\\
10.&223.&223.&1029.&1029.&45.&7&396&40\_5\_30\_1\_50\_70\_200 \textcolor{red}{$\boldsymbol{\upmu\upepsilon\uplambda\upalpha\upnu\mathrm{o}\upsigma}$} melanos $|$Tinte\\
11.&224.&224.&1036.&1036.&52.&4&62&1\_30\_30\_1 \textcolor{red}{$\boldsymbol{\upalpha\uplambda\uplambda\upalpha}$} alla $|$sondern\\
12.&225.&225.&1040.&1040.&56.&6&932&5\_30\_80\_10\_7\_800 \textcolor{red}{$\boldsymbol{\upepsilon\uplambda\uppi\upiota\upzeta\upomega}$} elpjzO $|$ich hoffe\\
13.&226.&226.&1046.&1046.&62.&6&109&5\_30\_9\_5\_10\_50 \textcolor{red}{$\boldsymbol{\upepsilon\uplambda\upvartheta\upepsilon\upiota\upnu}$} elTejn $|$zu kommen\\
14.&227.&227.&1052.&1052.&68.&4&450&80\_100\_70\_200 \textcolor{red}{$\boldsymbol{\uppi\uprho\mathrm{o}\upsigma}$} pros $|$zu\\
15.&228.&228.&1056.&1056.&72.&4&641&400\_40\_1\_200 \textcolor{red}{$\boldsymbol{\upsilon\upmu\upalpha\upsigma}$} "umas $|$euch\\
16.&229.&229.&1060.&1060.&76.&3&31&20\_1\_10 \textcolor{red}{$\boldsymbol{\upkappa\upalpha\upiota}$} kaj $|$und\\
17.&230.&230.&1063.&1063.&79.&5&611&200\_300\_70\_40\_1 \textcolor{red}{$\boldsymbol{\upsigma\uptau\mathrm{o}\upmu\upalpha}$} stoma $|$m"undlich/(von) Mund\\
18.&231.&231.&1068.&1068.&84.&4&450&80\_100\_70\_200 \textcolor{red}{$\boldsymbol{\uppi\uprho\mathrm{o}\upsigma}$} pros $|$/zu\\
19.&232.&232.&1072.&1072.&88.&5&611&200\_300\_70\_40\_1 \textcolor{red}{$\boldsymbol{\upsigma\uptau\mathrm{o}\upmu\upalpha}$} stoma $|$(mit euch)/Mund\\
20.&233.&233.&1077.&1077.&93.&7&280&30\_1\_30\_8\_200\_1\_10 \textcolor{red}{$\boldsymbol{\uplambda\upalpha\uplambda\upeta\upsigma\upalpha\upiota}$} lal"asaj $|$zu reden\\
21.&234.&234.&1084.&1084.&100.&3&61&10\_50\_1 \textcolor{red}{$\boldsymbol{\upiota\upnu\upalpha}$} jna $|$damit\\
22.&235.&235.&1087.&1087.&103.&1&8&8 \textcolor{red}{$\boldsymbol{\upeta}$} "a $|$(die)\\
23.&236.&236.&1088.&1088.&104.&4&702&600\_1\_100\_1 \textcolor{red}{$\boldsymbol{\upchi\upalpha\uprho\upalpha}$} cara $|$Freude\\
24.&237.&237.&1092.&1092.&108.&4&898&8\_40\_800\_50 \textcolor{red}{$\boldsymbol{\upeta\upmu\upomega\upnu}$} "amOn $|$unsere\\
25.&238.&238.&1096.&1096.&112.&1&8&8 \textcolor{red}{$\boldsymbol{\upeta}$} "a $|$sei/ist\\
\end{tabular}
\newpage
\begin{tabular}{rrrrrrrrp{120mm}}
WV&WK&WB&ABK&ABB&ABV&AnzB&TW&Zahlencode \textcolor{red}{$\boldsymbol{Grundtext}$} Umschrift $|$"Ubersetzung(en)\\
26.&239.&239.&1097.&1097.&113.&11&1206&80\_5\_80\_30\_8\_100\_800\_40\_5\_50\_8 \textcolor{red}{$\boldsymbol{\uppi\upepsilon\uppi\uplambda\upeta\uprho\upomega\upmu\upepsilon\upnu\upeta}$} pepl"arOmen"a $|$v"ollig/ vollkommen\\
\end{tabular}\medskip \\
Ende des Verses 1.12\\
Verse: 12, Buchstaben: 123, 1107, 1107, Totalwerte: 12883, 123744, 123744\\
\\
Da ich euch vieles zu schreiben habe, wollte ich es nicht mit Papier und Tinte tun, sondern ich hoffe, zu euch zu kommen und m"undlich mit euch zu reden, auf da"s unsere Freude v"ollig sei.\\
\newpage 
{\bf -- 1.13}\\
\medskip \\
\begin{tabular}{rrrrrrrrp{120mm}}
WV&WK&WB&ABK&ABB&ABV&AnzB&TW&Zahlencode \textcolor{red}{$\boldsymbol{Grundtext}$} Umschrift $|$"Ubersetzung(en)\\
1.&240.&240.&1108.&1108.&1.&9&605&1\_200\_80\_1\_7\_5\_300\_1\_10 \textcolor{red}{$\boldsymbol{\upalpha\upsigma\uppi\upalpha\upzeta\upepsilon\uptau\upalpha\upiota}$} aspazetaj $|$es gr"u"sen/(es) lassen gr"u"sen\\
2.&241.&241.&1117.&1117.&10.&2&205&200\_5 \textcolor{red}{$\boldsymbol{\upsigma\upepsilon}$} se $|$dich\\
3.&242.&242.&1119.&1119.&12.&2&301&300\_1 \textcolor{red}{$\boldsymbol{\uptau\upalpha}$} ta $|$die\\
4.&243.&243.&1121.&1121.&14.&5&376&300\_5\_20\_50\_1 \textcolor{red}{$\boldsymbol{\uptau\upepsilon\upkappa\upnu\upalpha}$} tekna $|$Kinder\\
5.&244.&244.&1126.&1126.&19.&3&508&300\_8\_200 \textcolor{red}{$\boldsymbol{\uptau\upeta\upsigma}$} t"as $|$(der)\\
6.&245.&245.&1129.&1129.&22.&7&748&1\_4\_5\_30\_500\_8\_200 \textcolor{red}{$\boldsymbol{\upalpha\updelta\upepsilon\uplambda\upvarphi\upeta\upsigma}$} adelf"as $|$Schwester\\
7.&246.&246.&1136.&1136.&29.&3&670&200\_70\_400 \textcolor{red}{$\boldsymbol{\upsigma\mathrm{o}\upsilon}$} so"u $|$deiner\\
8.&247.&247.&1139.&1139.&32.&3&508&300\_8\_200 \textcolor{red}{$\boldsymbol{\uptau\upeta\upsigma}$} t"as $|$(der)\\
9.&248.&248.&1142.&1142.&35.&8&588&5\_20\_30\_5\_20\_300\_8\_200 \textcolor{red}{$\boldsymbol{\upepsilon\upkappa\uplambda\upepsilon\upkappa\uptau\upeta\upsigma}$} eklekt"as $|$auserw"ahlten\\
10.&249.&249.&1150.&1150.&43.&4&99&1\_40\_8\_50 \textcolor{red}{$\boldsymbol{\upalpha\upmu\upeta\upnu}$} am"an $|$amen//\\
\end{tabular}\medskip \\
Ende des Verses 1.13\\
Verse: 13, Buchstaben: 46, 1153, 1153, Totalwerte: 4608, 128352, 128352\\
\\
Es gr"u"sen dich die Kinder deiner auserw"ahlten Schwester.\\
\\
{\bf Ende des Kapitels 1}\\

\bigskip				%%gro�er Abstand

\newpage
\hphantom{x}
\bigskip\bigskip\bigskip\bigskip\bigskip\bigskip
\begin{center}{ \huge {\bf Ende des Buches}}\end{center}


\end{document}



