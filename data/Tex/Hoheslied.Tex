\documentclass[a4paper,10pt,landscape]{article}
\usepackage[landscape]{geometry}	%%Querformat
\usepackage{fancyhdr}			%%Erweiterte Kopfzeilen
\usepackage{color}			%%Farben
\usepackage[german]{babel}		%%Deutsche Namen
\usepackage{cjhebrew} %%Hebr�isches Packet
\usepackage{upgreek}                    %%nicht kursive griechische Buchstaben
\usepackage{amsbsy}                     %%fette griechische Buchstaben
\setlength{\parindent}{0pt}		%%Damit bei neuen Abs"atzen kein Einzug
\frenchspacing

\pagestyle{fancy}			%%Kopf-/Fu�zeilenstyle

\renewcommand{\headrulewidth}{0.5pt}	%%Strich in Kopfzeile
\renewcommand{\footrulewidth}{0pt}	%%kein Strich in Fu�zeile
\renewcommand{\sectionmark}[1]{\markright{{#1}}}
\lhead{\rightmark}
\chead{}
\rhead{Bibel in Text und Zahl}
\lfoot{pgz}
\cfoot{\thepage}			%%Seitenzahl
\rfoot{}

\renewcommand{\section}[3]{\begin{center}{ \huge {\bf \textsl{{#1}}\\ \textcolor{red}{\textsl{{#2}}}}}\end{center}
\sectionmark{{#3}}}

\renewcommand{\baselinestretch}{0.9}	%%Zeilenabstand

\begin{document}			%%Dokumentbeginn

\section{\bigskip\bigskip\bigskip\bigskip\bigskip\bigskip
\\Das Lied der Lieder\\ (Das Hohelied)}
{}
{Hohelied}	%%�berschrift (�bergibt {schwarzen
										%%Text}{roten Text}{Text in Kopfzeile}


\bigskip				%%gro�er Abstand

\newpage
\hphantom{x}
\bigskip\bigskip\bigskip\bigskip\bigskip\bigskip
\begin{center}{ \huge {\bf Erl"auterungen}}\end{center}

\medskip
In diesem Buch werden folgende Abk"urzungen verwendet:\\
WV = Nummer des Wortes im Vers\\
WK = Nummer des Wortes im Kapitel\\
WB = Nummer des Wortes im Buch\\
ABV = Nummer des Anfangsbuchstabens des Wortes im Vers\\
ABK = Nummer des Anfangsbuchstabens des Wortes im Kapitel\\
ABB = Nummer des Anfangsbuchstabens des Wortes im Buch\\
AnzB = Anzahl der Buchstaben des Wortes\\
TW = Totalwert des Wortes\\

\medskip
Am Ende eines Verses finden sich sieben Zahlen,\\
die folgende Bedeutung haben (von links nach rechts):\\
1. Nummer des Verses im Buch\\
2. Gesamtzahl der Buchstaben im Vers\\
3. Gesamtzahl der Buchstaben (bis einschlie"slich dieses Verses) im Kapitel\\
4. Gesamtzahl der Buchstaben (bis einschlie"slich dieses Verses) im Buch\\
5. Summe der Totalwerte des Verses\\
6. Summe der Totalwerte (bis einschlie"slich dieses Verses) im Kapitel\\
7. Summe der Totalwerte (bis einschlie"slich dieses Verses) im Buch\\



\newpage 
{\bf -- 1.1}\\
\medskip \\
\begin{tabular}{rrrrrrrrp{120mm}}
WV&WK&WB&ABK&ABB&ABV&AnzB&TW&Zahlencode \textcolor{red}{$\boldsymbol{Grundtext}$} Umschrift $|$"Ubersetzung(en)\\
1.&1.&1.&1.&1.&1.&3&510&300\_10\_200 \textcolor{red}{\textcjheb{ry+s}} SJR $|$(das) Lied\\
2.&2.&2.&4.&4.&4.&6&565&5\_300\_10\_200\_10\_40 \textcolor{red}{\textcjheb{myry+sh}} HSJRJM $|$die (=der) Lieder\\
3.&3.&3.&10.&10.&10.&3&501&1\_300\_200 \textcolor{red}{\textcjheb{r+s'}} ASR $|$/welches\\
4.&4.&4.&13.&13.&13.&5&405&30\_300\_30\_40\_5 \textcolor{red}{\textcjheb{hml+sl}} LSLMH $|$von Salomo (Schlomo)///$<$friedlich$>$\\
\end{tabular}\medskip \\
Ende des Verses 1.1\\
Verse: 1, Buchstaben: 17, 17, 17, Totalwerte: 1981, 1981, 1981\\
\\
Das Lied der Lieder, von Salomo.\\
\newpage 
{\bf -- 1.2}\\
\medskip \\
\begin{tabular}{rrrrrrrrp{120mm}}
WV&WK&WB&ABK&ABB&ABV&AnzB&TW&Zahlencode \textcolor{red}{$\boldsymbol{Grundtext}$} Umschrift $|$"Ubersetzung(en)\\
1.&5.&5.&18.&18.&1.&5&470&10\_300\_100\_50\_10 \textcolor{red}{\textcjheb{ynq+sy}} JSQNJ $|$er k"usse mich\\
2.&6.&6.&23.&23.&6.&7&906&40\_50\_300\_10\_100\_6\_400 \textcolor{red}{\textcjheb{twqy+snm}} MNSJQWT $|$mit (den) K"ussen\\
3.&7.&7.&30.&30.&13.&4&101&80\_10\_5\_6 \textcolor{red}{\textcjheb{whyp}} PJHW $|$seines Mundes\\
4.&8.&8.&34.&34.&17.&2&30&20\_10 \textcolor{red}{\textcjheb{yk}} KJ $|$denn\\
5.&9.&9.&36.&36.&19.&5&67&9\_6\_2\_10\_40 \textcolor{red}{\textcjheb{mybw.t}} tWBJM $|$besser ist/gut (sind)\\
6.&10.&10.&41.&41.&24.&4&38&4\_4\_10\_20 \textcolor{red}{\textcjheb{kydd}} DDJK $|$deine Liebe/deine Liebkosungen\\
7.&11.&11.&45.&45.&28.&4&110&40\_10\_10\_50 \textcolor{red}{\textcjheb{nyym}} MJJN $|$(mehr) als Wein\\
\end{tabular}\medskip \\
Ende des Verses 1.2\\
Verse: 2, Buchstaben: 31, 48, 48, Totalwerte: 1722, 3703, 3703\\
\\
Er k"usse mich mit den K"ussen seines Mundes, denn deine Liebe ist besser als Wein.\\
\newpage 
{\bf -- 1.3}\\
\medskip \\
\begin{tabular}{rrrrrrrrp{120mm}}
WV&WK&WB&ABK&ABB&ABV&AnzB&TW&Zahlencode \textcolor{red}{$\boldsymbol{Grundtext}$} Umschrift $|$"Ubersetzung(en)\\
1.&12.&12.&49.&49.&1.&4&248&30\_200\_10\_8 \textcolor{red}{\textcjheb{.hyrl}} LRJC $|$an (Wohl)geruch\\
2.&13.&13.&53.&53.&5.&5&420&300\_40\_50\_10\_20 \textcolor{red}{\textcjheb{kynm+s}} SMNJK $|$deine Salben/deine (Salb)"Ole\\
3.&14.&14.&58.&58.&10.&5&67&9\_6\_2\_10\_40 \textcolor{red}{\textcjheb{mybw.t}} tWBJM $|$sind lieblich/(sind) gut\\
4.&15.&15.&63.&63.&15.&3&390&300\_40\_50 \textcolor{red}{\textcjheb{nm+s}} SMN $|$ein Salb"ol/(wie) (Duft)"Ol\\
5.&16.&16.&66.&66.&18.&4&706&400\_6\_200\_100 \textcolor{red}{\textcjheb{qrwt}} TWRQ $|$ausgegossen(es)\\
6.&17.&17.&70.&70.&22.&3&360&300\_40\_20 \textcolor{red}{\textcjheb{km+s}} SMK $|$(ist) dein Name\\
7.&18.&18.&73.&73.&25.&2&100&70\_30 \textcolor{red}{\textcjheb{l`}} aL $|$wegen\\
8.&19.&19.&75.&75.&27.&2&70&20\_50 \textcolor{red}{\textcjheb{nk}} KN $|$diesem\\
9.&20.&20.&77.&77.&29.&5&546&70\_30\_40\_6\_400 \textcolor{red}{\textcjheb{twml`}} aLMWT $|$die Jungfrauen/(die) M"adchen\\
10.&21.&21.&82.&82.&34.&5&34&1\_5\_2\_6\_20 \textcolor{red}{\textcjheb{kwbh'}} AHBWK $|$(sie) lieben dich\\
\end{tabular}\medskip \\
Ende des Verses 1.3\\
Verse: 3, Buchstaben: 38, 86, 86, Totalwerte: 2941, 6644, 6644\\
\\
Lieblich an Geruch sind deine Salben, ein ausgegossenes Salb"ol ist dein Name; darum lieben dich die Jungfrauen.\\
\newpage 
{\bf -- 1.4}\\
\medskip \\
\begin{tabular}{rrrrrrrrp{120mm}}
WV&WK&WB&ABK&ABB&ABV&AnzB&TW&Zahlencode \textcolor{red}{$\boldsymbol{Grundtext}$} Umschrift $|$"Ubersetzung(en)\\
1.&22.&22.&87.&87.&1.&5&420&40\_300\_20\_50\_10 \textcolor{red}{\textcjheb{ynk+sm}} MSKNJ $|$ziehe mich\\
2.&23.&23.&92.&92.&6.&5&239&1\_8\_200\_10\_20 \textcolor{red}{\textcjheb{kyr.h'}} ACRJK $|$dir nach\\
3.&24.&24.&97.&97.&11.&5&351&50\_200\_6\_90\_5 \textcolor{red}{\textcjheb{h.swrn}} NRW"sH $|$werden wir laufen/lass uns (ent)eilen\\
4.&25.&25.&102.&102.&16.&6&78&5\_2\_10\_1\_50\_10 \textcolor{red}{\textcjheb{yn'ybh}} HBJANJ $|$es hat mich gef"uhrt/er (=es) lies mich kommen\\
5.&26.&26.&108.&108.&22.&4&95&5\_40\_30\_20 \textcolor{red}{\textcjheb{klmh}} HMLK $|$der K"onig\\
6.&27.&27.&112.&112.&26.&5&228&8\_4\_200\_10\_6 \textcolor{red}{\textcjheb{wyrd.h}} CDRJW $|$in seine Gem"acher\\
7.&28.&28.&117.&117.&31.&5&98&50\_3\_10\_30\_5 \textcolor{red}{\textcjheb{hlygn}} NGJLH $|$wir wollen frohlocken/wir wollen jubeln\\
8.&29.&29.&122.&122.&36.&6&409&6\_50\_300\_40\_8\_5 \textcolor{red}{\textcjheb{h.hm+snw}} WNSMCH $|$und (wir wollen) uns freuen\\
9.&30.&30.&128.&128.&42.&2&22&2\_20 \textcolor{red}{\textcjheb{kb}} BK $|$deiner/an dir\\
10.&31.&31.&130.&130.&44.&6&292&50\_7\_20\_10\_200\_5 \textcolor{red}{\textcjheb{hrykzn}} NZKJRH $|$wir wollen preisen/wir wollen r"uhmen\\
11.&32.&32.&136.&136.&50.&4&38&4\_4\_10\_20 \textcolor{red}{\textcjheb{kydd}} DDJK $|$deine Liebe/deine Liebkosungen\\
12.&33.&33.&140.&140.&54.&4&110&40\_10\_10\_50 \textcolor{red}{\textcjheb{nyym}} MJJN $|$(mehr) als Wein\\
13.&34.&34.&144.&144.&58.&6&600&40\_10\_300\_200\_10\_40 \textcolor{red}{\textcjheb{myr+sym}} MJSRJM $|$in Aufrichtigkeit/Aufrichtigkeiten\\
14.&35.&35.&150.&150.&64.&5&34&1\_5\_2\_6\_20 \textcolor{red}{\textcjheb{kwbh'}} AHBWK $|$sie lieben dich\\
\end{tabular}\medskip \\
Ende des Verses 1.4\\
Verse: 4, Buchstaben: 68, 154, 154, Totalwerte: 3014, 9658, 9658\\
\\
Ziehe mich: wir werden dir nachlaufen. Der K"onig hat mich in seine Gem"acher gef"uhrt: wir wollen frohlocken und deiner uns freuen, wollen deine Liebe preisen mehr als Wein! Sie lieben dich in Aufrichtigkeit.\\
\newpage 
{\bf -- 1.5}\\
\medskip \\
\begin{tabular}{rrrrrrrrp{120mm}}
WV&WK&WB&ABK&ABB&ABV&AnzB&TW&Zahlencode \textcolor{red}{$\boldsymbol{Grundtext}$} Umschrift $|$"Ubersetzung(en)\\
1.&36.&36.&155.&155.&1.&5&519&300\_8\_6\_200\_5 \textcolor{red}{\textcjheb{hrw.h+s}} SCWRH $|$schwarz(e)\\
2.&37.&37.&160.&160.&6.&3&61&1\_50\_10 \textcolor{red}{\textcjheb{yn'}} ANJ $|$(bin) ich\\
3.&38.&38.&163.&163.&9.&5&68&6\_50\_1\_6\_5 \textcolor{red}{\textcjheb{hw'nw}} WNAWH $|$aber anmutig/doch anmutig(e)\\
4.&39.&39.&168.&168.&14.&4&458&2\_50\_6\_400 \textcolor{red}{\textcjheb{twnb}} BNWT $|$(ihr) T"ochter\\
5.&40.&40.&172.&172.&18.&6&586&10\_200\_6\_300\_30\_40 \textcolor{red}{\textcjheb{ml+swry}} JRWSLM $|$Jerusalem(s)///$<$Erbteil des Friedens$>$\\
6.&41.&41.&178.&178.&24.&5&66&20\_1\_5\_30\_10 \textcolor{red}{\textcjheb{ylh'k}} KAHLJ $|$wie die Zelte/wie Zeltbeh"ange\\
7.&42.&42.&183.&183.&29.&3&304&100\_4\_200 \textcolor{red}{\textcjheb{rdq}} QDR $|$Kedar(s)///$<$dunkelh"autig$>$\\
8.&43.&43.&186.&186.&32.&7&716&20\_10\_200\_10\_70\_6\_400 \textcolor{red}{\textcjheb{tw`yryk}} KJRJaWT $|$wie die Zeltbeh"ange/wie Zeltdecken\\
9.&44.&44.&193.&193.&39.&4&375&300\_30\_40\_5 \textcolor{red}{\textcjheb{hml+s}} SLMH $|$Salomo(s)\\
\end{tabular}\medskip \\
Ende des Verses 1.5\\
Verse: 5, Buchstaben: 42, 196, 196, Totalwerte: 3153, 12811, 12811\\
\\
Ich bin schwarz, aber anmutig, T"ochter Jerusalems, wie die Zelte Kedars, wie die Zeltbeh"ange Salomos.\\
\newpage 
{\bf -- 1.6}\\
\medskip \\
\begin{tabular}{rrrrrrrrp{120mm}}
WV&WK&WB&ABK&ABB&ABV&AnzB&TW&Zahlencode \textcolor{red}{$\boldsymbol{Grundtext}$} Umschrift $|$"Ubersetzung(en)\\
1.&45.&45.&197.&197.&1.&2&31&1\_30 \textcolor{red}{\textcjheb{l'}} AL $|$nicht\\
2.&46.&46.&199.&199.&3.&6&667&400\_200\_1\_6\_50\_10 \textcolor{red}{\textcjheb{ynw'rt}} TRAWNJ $|$seht an mich\\
3.&47.&47.&205.&205.&9.&4&361&300\_1\_50\_10 \textcolor{red}{\textcjheb{yn'+s}} SANJ $|$weil ich bin/welche ich (bin)\\
4.&48.&48.&209.&209.&13.&6&1116&300\_8\_200\_8\_200\_400 \textcolor{red}{\textcjheb{tr.hr.h+s}} SCRCRT $|$schw"arzlich(e)\\
5.&49.&49.&215.&215.&19.&7&1147&300\_300\_7\_80\_400\_50\_10 \textcolor{red}{\textcjheb{yntpz+s+s}} SSZPTNJ $|$weil mich verbrannt hat/welche, sie (=es) br"aunte mich\\
6.&50.&50.&222.&222.&26.&4&645&5\_300\_40\_300 \textcolor{red}{\textcjheb{+sm+sh}} HSMS $|$die Sonne\\
7.&51.&51.&226.&226.&30.&3&62&2\_50\_10 \textcolor{red}{\textcjheb{ynb}} BNJ $|$die S"ohne\\
8.&52.&52.&229.&229.&33.&3&51&1\_40\_10 \textcolor{red}{\textcjheb{ym'}} AMJ $|$meine(r) Mutter\\
9.&53.&53.&232.&232.&36.&4&264&50\_8\_200\_6 \textcolor{red}{\textcjheb{wr.hn}} NCRW $|$z"urnten/sie schnaubten\\
10.&54.&54.&236.&236.&40.&2&12&2\_10 \textcolor{red}{\textcjheb{yb}} BJ $|$mir/gegen mich\\
11.&55.&55.&238.&238.&42.&4&400&300\_40\_50\_10 \textcolor{red}{\textcjheb{ynm+s}} SMNJ $|$bestellten mich/sie setzten (ein) mich\\
12.&56.&56.&242.&242.&46.&4&264&50\_9\_200\_5 \textcolor{red}{\textcjheb{hr.tn}} NtRH $|$zur H"uterin\\
13.&57.&57.&246.&246.&50.&2&401&1\_400 \textcolor{red}{\textcjheb{t'}} AT $|$**\\
14.&58.&58.&248.&248.&52.&6&315&5\_20\_200\_40\_10\_40 \textcolor{red}{\textcjheb{mymrkh}} HKRMJM $|$der Weinberge\\
15.&59.&59.&254.&254.&58.&4&270&20\_200\_40\_10 \textcolor{red}{\textcjheb{ymrk}} KRMJ $|$meinen Weinberg\\
16.&60.&60.&258.&258.&62.&3&340&300\_30\_10 \textcolor{red}{\textcjheb{yl+s}} SLJ $|$eigenen/welcher zu mir\\
17.&61.&61.&261.&261.&65.&2&31&30\_1 \textcolor{red}{\textcjheb{'l}} LA $|$nicht\\
18.&62.&62.&263.&263.&67.&5&669&50\_9\_200\_400\_10 \textcolor{red}{\textcjheb{ytr.tn}} NtRTJ $|$ich habe geh"utet\\
\end{tabular}\medskip \\
Ende des Verses 1.6\\
Verse: 6, Buchstaben: 71, 267, 267, Totalwerte: 7046, 19857, 19857\\
\\
Sehet mich nicht an, weil ich schw"arzlich bin, weil die Sonne mich verbrannt hat; meiner Mutter S"ohne z"urnten mir, bestellten mich zur H"uterin der Weinberge; meinen eigenen Weinberg habe ich nicht geh"utet. -\\
\newpage 
{\bf -- 1.7}\\
\medskip \\
\begin{tabular}{rrrrrrrrp{120mm}}
WV&WK&WB&ABK&ABB&ABV&AnzB&TW&Zahlencode \textcolor{red}{$\boldsymbol{Grundtext}$} Umschrift $|$"Ubersetzung(en)\\
1.&63.&63.&268.&268.&1.&5&27&5\_3\_10\_4\_5 \textcolor{red}{\textcjheb{hdygh}} HGJDH $|$sage an/erz"ahle\\
2.&64.&64.&273.&273.&6.&2&40&30\_10 \textcolor{red}{\textcjheb{yl}} LJ $|$mir\\
3.&65.&65.&275.&275.&8.&5&313&300\_1\_5\_2\_5 \textcolor{red}{\textcjheb{hbh'+s}} SAHBH $|$du den liebt/welchen sie liebte\\
4.&66.&66.&280.&280.&13.&4&440&50\_80\_300\_10 \textcolor{red}{\textcjheb{y+spn}} NPSJ $|$meine Seele\\
5.&67.&67.&284.&284.&17.&4&36&1\_10\_20\_5 \textcolor{red}{\textcjheb{hky'}} AJKH $|$wo\\
6.&68.&68.&288.&288.&21.&4&675&400\_200\_70\_5 \textcolor{red}{\textcjheb{h`rt}} TRaH $|$du weidest\\
7.&69.&69.&292.&292.&25.&4&36&1\_10\_20\_5 \textcolor{red}{\textcjheb{hky'}} AJKH $|$wo\\
8.&70.&70.&296.&296.&29.&5&702&400\_200\_2\_10\_90 \textcolor{red}{\textcjheb{.sybrt}} TRBJ"s $|$du l"asst lagern\\
9.&71.&71.&301.&301.&34.&6&347&2\_90\_5\_200\_10\_40 \textcolor{red}{\textcjheb{myrh.sb}} B"sHRJM $|$am Mittag/in den Mittag(sstunden)\\
10.&72.&72.&307.&307.&40.&4&375&300\_30\_40\_5 \textcolor{red}{\textcjheb{hml+s}} SLMH $|$denn warum/welche zu was (=wozu)\\
11.&73.&73.&311.&311.&44.&4&21&1\_5\_10\_5 \textcolor{red}{\textcjheb{hyh'}} AHJH $|$ich sollte sein\\
12.&74.&74.&315.&315.&48.&5&114&20\_70\_9\_10\_5 \textcolor{red}{\textcjheb{hy.t`k}} KatJH $|$wie eine Verschleierte/wie eine Umherirrende\\
13.&75.&75.&320.&320.&53.&2&100&70\_30 \textcolor{red}{\textcjheb{l`}} aL $|$bei\\
14.&76.&76.&322.&322.&55.&4&284&70\_4\_200\_10 \textcolor{red}{\textcjheb{yrd`}} aDRJ $|$(den) Herden\\
15.&77.&77.&326.&326.&59.&5&240&8\_2\_200\_10\_20 \textcolor{red}{\textcjheb{kyrb.h}} CBRJK $|$deine(r) Genossen\\
\end{tabular}\medskip \\
Ende des Verses 1.7\\
Verse: 7, Buchstaben: 63, 330, 330, Totalwerte: 3750, 23607, 23607\\
\\
Sage mir an, du, den meine Seele liebt, wo weidest du, wo l"assest du lagern am Mittag? Denn warum sollte ich wie eine Verschleierte sein bei den Herden deiner Genossen? -\\
\newpage 
{\bf -- 1.8}\\
\medskip \\
\begin{tabular}{rrrrrrrrp{120mm}}
WV&WK&WB&ABK&ABB&ABV&AnzB&TW&Zahlencode \textcolor{red}{$\boldsymbol{Grundtext}$} Umschrift $|$"Ubersetzung(en)\\
1.&78.&78.&331.&331.&1.&2&41&1\_40 \textcolor{red}{\textcjheb{m'}} AM $|$wenn\\
2.&79.&79.&333.&333.&3.&2&31&30\_1 \textcolor{red}{\textcjheb{'l}} LA $|$nicht\\
3.&80.&80.&335.&335.&5.&4&484&400\_4\_70\_10 \textcolor{red}{\textcjheb{y`dt}} TDaJ $|$du (es) wei"st (das)\\
4.&81.&81.&339.&339.&9.&2&50&30\_20 \textcolor{red}{\textcjheb{kl}} LK $|$du/zu dir\\
5.&82.&82.&341.&341.&11.&4&100&5\_10\_80\_5 \textcolor{red}{\textcjheb{hpyh}} HJPH $|$Sch"onste/die (=du) sch"one\\
6.&83.&83.&345.&345.&15.&5&402&2\_50\_300\_10\_40 \textcolor{red}{\textcjheb{my+snb}} BNSJM $|$unter den Frauen\\
7.&84.&84.&350.&350.&20.&3&101&90\_1\_10 \textcolor{red}{\textcjheb{y'.s}} "sAJ $|$so geh hinaus/ziehe hinaus\\
8.&85.&85.&353.&353.&23.&2&50&30\_20 \textcolor{red}{\textcjheb{kl}} LK $|$/f"ur dich\\
9.&86.&86.&355.&355.&25.&5&184&2\_70\_100\_2\_10 \textcolor{red}{\textcjheb{ybq`b}} BaQBJ $|$den Spuren nach/auf den Fersen\\
10.&87.&87.&360.&360.&30.&4&146&5\_90\_1\_50 \textcolor{red}{\textcjheb{n'.sh}} H"sAN $|$der Herde\\
11.&88.&88.&364.&364.&34.&4&286&6\_200\_70\_10 \textcolor{red}{\textcjheb{y`rw}} WRaJ $|$und weide\\
12.&89.&89.&368.&368.&38.&2&401&1\_400 \textcolor{red}{\textcjheb{t'}} AT $|$**\\
13.&90.&90.&370.&370.&40.&6&447&3\_4\_10\_400\_10\_20 \textcolor{red}{\textcjheb{kytydg}} GDJTJK $|$deine Zicklein/deine B"ocklein\\
14.&91.&91.&376.&376.&46.&2&100&70\_30 \textcolor{red}{\textcjheb{l`}} aL $|$bei/an\\
15.&92.&92.&378.&378.&48.&6&816&40\_300\_20\_50\_6\_400 \textcolor{red}{\textcjheb{twnk+sm}} MSKNWT $|$den Wohnungen\\
16.&93.&93.&384.&384.&54.&5&325&5\_200\_70\_10\_40 \textcolor{red}{\textcjheb{my`rh}} HRaJM $|$der Hirten\\
\end{tabular}\medskip \\
Ende des Verses 1.8\\
Verse: 8, Buchstaben: 58, 388, 388, Totalwerte: 3964, 27571, 27571\\
\\
Wenn du es nicht wei"st, du Sch"onste unter den Frauen, so geh hinaus, den Spuren der Herde nach und weide deine Zicklein bei den Wohnungen der Hirten.\\
\newpage 
{\bf -- 1.9}\\
\medskip \\
\begin{tabular}{rrrrrrrrp{120mm}}
WV&WK&WB&ABK&ABB&ABV&AnzB&TW&Zahlencode \textcolor{red}{$\boldsymbol{Grundtext}$} Umschrift $|$"Ubersetzung(en)\\
1.&94.&94.&389.&389.&1.&5&560&30\_60\_60\_400\_10 \textcolor{red}{\textcjheb{ytssl}} LssTJ $|$einem Rosse/mit meiner Stute\\
2.&95.&95.&394.&394.&6.&5&234&2\_200\_20\_2\_10 \textcolor{red}{\textcjheb{ybkrb}} BRKBJ $|$an dem Prachtwagen/an den Wagen(z"ugen)\\
3.&96.&96.&399.&399.&11.&4&355&80\_200\_70\_5 \textcolor{red}{\textcjheb{h`rp}} PRaH $|$des Pharao/Pharao(s)\\
4.&97.&97.&403.&403.&15.&6&484&4\_40\_10\_400\_10\_20 \textcolor{red}{\textcjheb{kytymd}} DMJTJK $|$ich vergleiche dich\\
5.&98.&98.&409.&409.&21.&5&690&200\_70\_10\_400\_10 \textcolor{red}{\textcjheb{yty`r}} RaJTJ $|$meine Freundin\\
\end{tabular}\medskip \\
Ende des Verses 1.9\\
Verse: 9, Buchstaben: 25, 413, 413, Totalwerte: 2323, 29894, 29894\\
\\
Einem Rosse an des Pharao Prachtwagen vergleiche ich dich, meine Freundin.\\
\newpage 
{\bf -- 1.10}\\
\medskip \\
\begin{tabular}{rrrrrrrrp{120mm}}
WV&WK&WB&ABK&ABB&ABV&AnzB&TW&Zahlencode \textcolor{red}{$\boldsymbol{Grundtext}$} Umschrift $|$"Ubersetzung(en)\\
1.&99.&99.&414.&414.&1.&4&63&50\_1\_6\_6 \textcolor{red}{\textcjheb{ww'n}} NAWW $|$anmutig sind/lieblich sind\\
2.&100.&100.&418.&418.&5.&5&78&30\_8\_10\_10\_20 \textcolor{red}{\textcjheb{kyy.hl}} LCJJK $|$deine Wangen\\
3.&101.&101.&423.&423.&10.&5&652&2\_400\_200\_10\_40 \textcolor{red}{\textcjheb{myrtb}} BTRJM $|$in den Kettchen/mit (Perlen)reihen\\
4.&102.&102.&428.&428.&15.&5&317&90\_6\_1\_200\_20 \textcolor{red}{\textcjheb{kr'w.s}} "sWARK $|$dein Hals\\
5.&103.&103.&433.&433.&20.&7&273&2\_8\_200\_6\_7\_10\_40 \textcolor{red}{\textcjheb{myzwr.hb}} BCRWZJM $|$in den Schn"uren/mit den Schn"uren\\
\end{tabular}\medskip \\
Ende des Verses 1.10\\
Verse: 10, Buchstaben: 26, 439, 439, Totalwerte: 1383, 31277, 31277\\
\\
Anmutig sind deine Wangen in den Kettchen, dein Hals in den Schn"uren.\\
\newpage 
{\bf -- 1.11}\\
\medskip \\
\begin{tabular}{rrrrrrrrp{120mm}}
WV&WK&WB&ABK&ABB&ABV&AnzB&TW&Zahlencode \textcolor{red}{$\boldsymbol{Grundtext}$} Umschrift $|$"Ubersetzung(en)\\
1.&104.&104.&440.&440.&1.&4&616&400\_6\_200\_10 \textcolor{red}{\textcjheb{yrwt}} TWRJ $|$Kettchen/(Perlen)reihen\\
2.&105.&105.&444.&444.&5.&3&14&7\_5\_2 \textcolor{red}{\textcjheb{bhz}} ZHB $|$goldene/(von) Gold\\
3.&106.&106.&447.&447.&8.&4&425&50\_70\_300\_5 \textcolor{red}{\textcjheb{h+s`n}} NaSH $|$wir wollen machen\\
4.&107.&107.&451.&451.&12.&2&50&30\_20 \textcolor{red}{\textcjheb{kl}} LK $|$dir/f"ur dich\\
5.&108.&108.&453.&453.&14.&2&110&70\_40 \textcolor{red}{\textcjheb{m`}} aM $|$mit\\
6.&109.&109.&455.&455.&16.&5&560&50\_100\_4\_6\_400 \textcolor{red}{\textcjheb{twdqn}} NQDWT $|$Punkten/K"uglein\\
7.&110.&110.&460.&460.&21.&4&165&5\_20\_60\_80 \textcolor{red}{\textcjheb{pskh}} HKsP $|$von Silber/aus Silber\\
\end{tabular}\medskip \\
Ende des Verses 1.11\\
Verse: 11, Buchstaben: 24, 463, 463, Totalwerte: 1940, 33217, 33217\\
\\
Wir wollen dir goldene Kettchen machen mit Punkten von Silber. -\\
\newpage 
{\bf -- 1.12}\\
\medskip \\
\begin{tabular}{rrrrrrrrp{120mm}}
WV&WK&WB&ABK&ABB&ABV&AnzB&TW&Zahlencode \textcolor{red}{$\boldsymbol{Grundtext}$} Umschrift $|$"Ubersetzung(en)\\
1.&111.&111.&464.&464.&1.&2&74&70\_4 \textcolor{red}{\textcjheb{d`}} aD $|$w"ahrend/bis\\
2.&112.&112.&466.&466.&3.&5&395&300\_5\_40\_30\_20 \textcolor{red}{\textcjheb{klmh+s}} SHMLK $|$(dass) der K"onig\\
3.&113.&113.&471.&471.&8.&5&110&2\_40\_60\_2\_6 \textcolor{red}{\textcjheb{wbsmb}} BMsBW $|$war an seiner Tafel/weilt bei seiner (Tafel)runde\\
4.&114.&114.&476.&476.&13.&4&264&50\_200\_4\_10 \textcolor{red}{\textcjheb{ydrn}} NRDJ $|$meine Narde\\
5.&115.&115.&480.&480.&17.&3&500&50\_400\_50 \textcolor{red}{\textcjheb{ntn}} NTN $|$gab/er (=sie) gab\\
6.&116.&116.&483.&483.&20.&4&224&200\_10\_8\_6 \textcolor{red}{\textcjheb{w.hyr}} RJCW $|$ihren Duft\\
\end{tabular}\medskip \\
Ende des Verses 1.12\\
Verse: 12, Buchstaben: 23, 486, 486, Totalwerte: 1567, 34784, 34784\\
\\
W"ahrend der K"onig an seiner Tafel war, gab meine Narde ihren Duft.\\
\newpage 
{\bf -- 1.13}\\
\medskip \\
\begin{tabular}{rrrrrrrrp{120mm}}
WV&WK&WB&ABK&ABB&ABV&AnzB&TW&Zahlencode \textcolor{red}{$\boldsymbol{Grundtext}$} Umschrift $|$"Ubersetzung(en)\\
1.&117.&117.&487.&487.&1.&4&496&90\_200\_6\_200 \textcolor{red}{\textcjheb{rwr.s}} "sRWR $|$(ein) B"undel\\
2.&118.&118.&491.&491.&5.&3&245&5\_40\_200 \textcolor{red}{\textcjheb{rmh}} HMR $|$(der) Myrrhe\\
3.&119.&119.&494.&494.&8.&4&24&4\_6\_4\_10 \textcolor{red}{\textcjheb{ydwd}} DWDJ $|$mein Geliebter/mein Freund\\
4.&120.&120.&498.&498.&12.&2&40&30\_10 \textcolor{red}{\textcjheb{yl}} LJ $|$(ist) mir\\
5.&121.&121.&500.&500.&14.&3&62&2\_10\_50 \textcolor{red}{\textcjheb{nyb}} BJN $|$(das) zwischen\\
6.&122.&122.&503.&503.&17.&3&314&300\_4\_10 \textcolor{red}{\textcjheb{yd+s}} SDJ $|$meinen Br"usten\\
7.&123.&123.&506.&506.&20.&4&100&10\_30\_10\_50 \textcolor{red}{\textcjheb{nyly}} JLJN $|$(er) ruht\\
\end{tabular}\medskip \\
Ende des Verses 1.13\\
Verse: 13, Buchstaben: 23, 509, 509, Totalwerte: 1281, 36065, 36065\\
\\
Mein Geliebter ist mir ein B"undel Myrrhe, das zwischen meinen Br"usten ruht.\\
\newpage 
{\bf -- 1.14}\\
\medskip \\
\begin{tabular}{rrrrrrrrp{120mm}}
WV&WK&WB&ABK&ABB&ABV&AnzB&TW&Zahlencode \textcolor{red}{$\boldsymbol{Grundtext}$} Umschrift $|$"Ubersetzung(en)\\
1.&124.&124.&510.&510.&1.&4&351&1\_300\_20\_30 \textcolor{red}{\textcjheb{lk+s'}} ASKL $|$(eine) Traube\\
2.&125.&125.&514.&514.&5.&4&305&5\_20\_80\_200 \textcolor{red}{\textcjheb{rpkh}} HKPR $|$(von) Zyper/des Zyprus\\
3.&126.&126.&518.&518.&9.&4&24&4\_6\_4\_10 \textcolor{red}{\textcjheb{ydwd}} DWDJ $|$mein Geliebter/mein Freund\\
4.&127.&127.&522.&522.&13.&2&40&30\_10 \textcolor{red}{\textcjheb{yl}} LJ $|$(ist) mir\\
5.&128.&128.&524.&524.&15.&5&272&2\_20\_200\_40\_10 \textcolor{red}{\textcjheb{ymrkb}} BKRMJ $|$in den Weinbergen/in den Weing"arten\\
6.&129.&129.&529.&529.&20.&3&130&70\_10\_50 \textcolor{red}{\textcjheb{ny`}} aJN $|$von En///$<$Brunnen$>$\\
7.&130.&130.&532.&532.&23.&3&17&3\_4\_10 \textcolor{red}{\textcjheb{ydg}} GDJ $|$Gedi///$<$B"ockchen$>$\\
\end{tabular}\medskip \\
Ende des Verses 1.14\\
Verse: 14, Buchstaben: 25, 534, 534, Totalwerte: 1139, 37204, 37204\\
\\
Eine Zypertraube ist mir mein Geliebter, in den Weinbergen von Engedi. -\\
\newpage 
{\bf -- 1.15}\\
\medskip \\
\begin{tabular}{rrrrrrrrp{120mm}}
WV&WK&WB&ABK&ABB&ABV&AnzB&TW&Zahlencode \textcolor{red}{$\boldsymbol{Grundtext}$} Umschrift $|$"Ubersetzung(en)\\
1.&131.&131.&535.&535.&1.&3&75&5\_50\_20 \textcolor{red}{\textcjheb{knh}} HNK $|$siehe du\\
2.&132.&132.&538.&538.&4.&3&95&10\_80\_5 \textcolor{red}{\textcjheb{hpy}} JPH $|$(bist) sch"on(e)\\
3.&133.&133.&541.&541.&7.&5&690&200\_70\_10\_400\_10 \textcolor{red}{\textcjheb{yty`r}} RaJTJ $|$meine Freundin\\
4.&134.&134.&546.&546.&12.&3&75&5\_50\_20 \textcolor{red}{\textcjheb{knh}} HNK $|$siehe du\\
5.&135.&135.&549.&549.&15.&3&95&10\_80\_5 \textcolor{red}{\textcjheb{hpy}} JPH $|$(bist) sch"on(e)\\
6.&136.&136.&552.&552.&18.&5&160&70\_10\_50\_10\_20 \textcolor{red}{\textcjheb{kyny`}} aJNJK $|$deine Augen\\
7.&137.&137.&557.&557.&23.&5&116&10\_6\_50\_10\_40 \textcolor{red}{\textcjheb{mynwy}} JWNJM $|$(sind) (wie) Tauben\\
\end{tabular}\medskip \\
Ende des Verses 1.15\\
Verse: 15, Buchstaben: 27, 561, 561, Totalwerte: 1306, 38510, 38510\\
\\
Siehe, du bist sch"on, meine Freundin, siehe, du bist sch"on, deine Augen sind Tauben. -\\
\newpage 
{\bf -- 1.16}\\
\medskip \\
\begin{tabular}{rrrrrrrrp{120mm}}
WV&WK&WB&ABK&ABB&ABV&AnzB&TW&Zahlencode \textcolor{red}{$\boldsymbol{Grundtext}$} Umschrift $|$"Ubersetzung(en)\\
1.&138.&138.&562.&562.&1.&3&75&5\_50\_20 \textcolor{red}{\textcjheb{knh}} HNK $|$siehe du\\
2.&139.&139.&565.&565.&4.&3&95&10\_80\_5 \textcolor{red}{\textcjheb{hpy}} JPH $|$(bist) sch"on\\
3.&140.&140.&568.&568.&7.&4&24&4\_6\_4\_10 \textcolor{red}{\textcjheb{ydwd}} DWDJ $|$mein Geliebter\\
4.&141.&141.&572.&572.&11.&2&81&1\_80 \textcolor{red}{\textcjheb{p'}} AP $|$ja/auch\\
5.&142.&142.&574.&574.&13.&4&170&50\_70\_10\_40 \textcolor{red}{\textcjheb{my`n}} NaJM $|$holdselig/lieblich\\
6.&143.&143.&578.&578.&17.&2&81&1\_80 \textcolor{red}{\textcjheb{p'}} AP $|$ja/auch\\
7.&144.&144.&580.&580.&19.&5&626&70\_200\_300\_50\_6 \textcolor{red}{\textcjheb{wn+sr`}} aRSNW $|$unser(e) Lager(st"atte)\\
8.&145.&145.&585.&585.&24.&5&375&200\_70\_50\_50\_5 \textcolor{red}{\textcjheb{hnn`r}} RaNNH $|$ist frisches Gr"un/(ist) laubreich(e)\\
\end{tabular}\medskip \\
Ende des Verses 1.16\\
Verse: 16, Buchstaben: 28, 589, 589, Totalwerte: 1527, 40037, 40037\\
\\
Siehe, du bist sch"on, mein Geliebter, ja, holdselig; ja, unser Lager ist frisches Gr"un.\\
\newpage 
{\bf -- 1.17}\\
\medskip \\
\begin{tabular}{rrrrrrrrp{120mm}}
WV&WK&WB&ABK&ABB&ABV&AnzB&TW&Zahlencode \textcolor{red}{$\boldsymbol{Grundtext}$} Umschrift $|$"Ubersetzung(en)\\
1.&146.&146.&590.&590.&1.&4&706&100\_200\_6\_400 \textcolor{red}{\textcjheb{twrq}} QRWT $|$die Balken\\
2.&147.&147.&594.&594.&5.&5&468&2\_400\_10\_50\_6 \textcolor{red}{\textcjheb{wnytb}} BTJNW $|$unserer Behausung/unserer H"auser\\
3.&148.&148.&599.&599.&10.&5&258&1\_200\_7\_10\_40 \textcolor{red}{\textcjheb{myzr'}} ARZJM $|$(sind) Zedern\\
4.&149.&149.&604.&604.&15.&6&283&200\_8\_10\_9\_50\_6 \textcolor{red}{\textcjheb{wn.ty.hr}} RCJtNW $|$unser Get"afel/unsere T"afelung\\
5.&150.&150.&610.&610.&21.&6&658&2\_200\_6\_400\_10\_40 \textcolor{red}{\textcjheb{mytwrb}} BRWTJM $|$Zypressen\\
\end{tabular}\medskip \\
Ende des Verses 1.17\\
Verse: 17, Buchstaben: 26, 615, 615, Totalwerte: 2373, 42410, 42410\\
\\
Die Balken unserer Behausung sind Zedern, unser Get"afel Zypressen.\\
\\
{\bf Ende des Kapitels 1}\\
\newpage 
{\bf -- 2.1}\\
\medskip \\
\begin{tabular}{rrrrrrrrp{120mm}}
WV&WK&WB&ABK&ABB&ABV&AnzB&TW&Zahlencode \textcolor{red}{$\boldsymbol{Grundtext}$} Umschrift $|$"Ubersetzung(en)\\
1.&1.&151.&1.&616.&1.&3&61&1\_50\_10 \textcolor{red}{\textcjheb{yn'}} ANJ $|$ich (bin)\\
2.&2.&152.&4.&619.&4.&5&530&8\_2\_90\_30\_400 \textcolor{red}{\textcjheb{tl.sb.h}} CB"sLT $|$eine Narzisse/(die wei"se) Lilie\\
3.&3.&153.&9.&624.&9.&5&561&5\_300\_200\_6\_50 \textcolor{red}{\textcjheb{nwr+sh}} HSRWN $|$Sarons/(des) Scharon//$<$Ebene$>$\\
4.&4.&154.&14.&629.&14.&5&1056&300\_6\_300\_50\_400 \textcolor{red}{\textcjheb{tn+sw+s}} SWSNT $|$eine Lilie/(die wei"se) Lilie\\
5.&5.&155.&19.&634.&19.&6&265&5\_70\_40\_100\_10\_40 \textcolor{red}{\textcjheb{myqm`h}} HaMQJM $|$(der) T"aler\\
\end{tabular}\medskip \\
Ende des Verses 2.1\\
Verse: 18, Buchstaben: 24, 24, 639, Totalwerte: 2473, 2473, 44883\\
\\
Ich bin eine Narzisse Sarons, eine Lilie der T"aler. -\\
\newpage 
{\bf -- 2.2}\\
\medskip \\
\begin{tabular}{rrrrrrrrp{120mm}}
WV&WK&WB&ABK&ABB&ABV&AnzB&TW&Zahlencode \textcolor{red}{$\boldsymbol{Grundtext}$} Umschrift $|$"Ubersetzung(en)\\
1.&6.&156.&25.&640.&1.&6&681&20\_300\_6\_300\_50\_5 \textcolor{red}{\textcjheb{hn+sw+sk}} KSWSNH $|$wie (eine) (wei"se) Lilie\\
2.&7.&157.&31.&646.&7.&3&62&2\_10\_50 \textcolor{red}{\textcjheb{nyb}} BJN $|$inmitten/zwischen\\
3.&8.&158.&34.&649.&10.&6&77&5\_8\_6\_8\_10\_40 \textcolor{red}{\textcjheb{my.hw.hh}} HCWCJM $|$der Dornen/den Dornen\\
4.&9.&159.&40.&655.&16.&2&70&20\_50 \textcolor{red}{\textcjheb{nk}} KN $|$so (ist)\\
5.&10.&160.&42.&657.&18.&5&690&200\_70\_10\_400\_10 \textcolor{red}{\textcjheb{yty`r}} RaJTJ $|$meine Freundin\\
6.&11.&161.&47.&662.&23.&3&62&2\_10\_50 \textcolor{red}{\textcjheb{nyb}} BJN $|$inmitten/zwischen\\
7.&12.&162.&50.&665.&26.&5&463&5\_2\_50\_6\_400 \textcolor{red}{\textcjheb{twnbh}} HBNWT $|$der T"ochter/den T"ochtern\\
\end{tabular}\medskip \\
Ende des Verses 2.2\\
Verse: 19, Buchstaben: 30, 54, 669, Totalwerte: 2105, 4578, 46988\\
\\
Wie eine Lilie inmitten der Dornen, so ist meine Freundin inmitten der T"ochter. -\\
\newpage 
{\bf -- 2.3}\\
\medskip \\
\begin{tabular}{rrrrrrrrp{120mm}}
WV&WK&WB&ABK&ABB&ABV&AnzB&TW&Zahlencode \textcolor{red}{$\boldsymbol{Grundtext}$} Umschrift $|$"Ubersetzung(en)\\
1.&13.&163.&55.&670.&1.&5&514&20\_400\_80\_6\_8 \textcolor{red}{\textcjheb{.hwptk}} KTPWC $|$wie ein Apfelbaum\\
2.&14.&164.&60.&675.&6.&4&172&2\_70\_90\_10 \textcolor{red}{\textcjheb{y.s`b}} Ba"sJ $|$unter den B"aumen\\
3.&15.&165.&64.&679.&10.&4&285&5\_10\_70\_200 \textcolor{red}{\textcjheb{r`yh}} HJaR $|$des Waldes\\
4.&16.&166.&68.&683.&14.&2&70&20\_50 \textcolor{red}{\textcjheb{nk}} KN $|$so (ist)\\
5.&17.&167.&70.&685.&16.&4&24&4\_6\_4\_10 \textcolor{red}{\textcjheb{ydwd}} DWDJ $|$mein Geliebter/mein Freund\\
6.&18.&168.&74.&689.&20.&3&62&2\_10\_50 \textcolor{red}{\textcjheb{nyb}} BJN $|$inmitten/zwischen\\
7.&19.&169.&77.&692.&23.&5&107&5\_2\_50\_10\_40 \textcolor{red}{\textcjheb{mynbh}} HBNJM $|$der S"ohne/den J"unglingen\\
8.&20.&170.&82.&697.&28.&4&128&2\_90\_30\_6 \textcolor{red}{\textcjheb{wl.sb}} B"sLW $|$in seinen Schatten/in seinem Schatten\\
9.&21.&171.&86.&701.&32.&5&462&8\_40\_4\_400\_10 \textcolor{red}{\textcjheb{ytdm.h}} CMDTJ $|$habe ich mich mit Wonne/ich begehrte\\
10.&22.&172.&91.&706.&37.&6&728&6\_10\_300\_2\_400\_10 \textcolor{red}{\textcjheb{ytb+syw}} WJSBTJ $|$gesetzt/und ich sa"s\\
11.&23.&173.&97.&712.&43.&5&302&6\_80\_200\_10\_6 \textcolor{red}{\textcjheb{wyrpw}} WPRJW $|$und seine Frucht\\
12.&24.&174.&102.&717.&48.&4&546&40\_400\_6\_100 \textcolor{red}{\textcjheb{qwtm}} MTWQ $|$(ist) s"u"s\\
13.&25.&175.&106.&721.&52.&4&68&30\_8\_20\_10 \textcolor{red}{\textcjheb{yk.hl}} LCKJ $|$meinem Gaumen/f"ur meinen Gaumen\\
\end{tabular}\medskip \\
Ende des Verses 2.3\\
Verse: 20, Buchstaben: 55, 109, 724, Totalwerte: 3468, 8046, 50456\\
\\
Wie ein Apfelbaum unter den B"aumen des Waldes, so ist mein Geliebter inmitten der S"ohne; ich habe mich mit Wonne in seinen Schatten gesetzt, und seine Frucht ist meinem Gaumen s"u"s.\\
\newpage 
{\bf -- 2.4}\\
\medskip \\
\begin{tabular}{rrrrrrrrp{120mm}}
WV&WK&WB&ABK&ABB&ABV&AnzB&TW&Zahlencode \textcolor{red}{$\boldsymbol{Grundtext}$} Umschrift $|$"Ubersetzung(en)\\
1.&26.&176.&110.&725.&1.&6&78&5\_2\_10\_1\_50\_10 \textcolor{red}{\textcjheb{yn'ybh}} HBJANJ $|$er hat mich gef"uhrt/er machte kommen mich\\
2.&27.&177.&116.&731.&7.&2&31&1\_30 \textcolor{red}{\textcjheb{l'}} AL $|$in/zu\\
3.&28.&178.&118.&733.&9.&3&412&2\_10\_400 \textcolor{red}{\textcjheb{tyb}} BJT $|$das Haus/dem Haus\\
4.&29.&179.&121.&736.&12.&4&75&5\_10\_10\_50 \textcolor{red}{\textcjheb{nyyh}} HJJN $|$des Weines\\
5.&30.&180.&125.&740.&16.&5&49&6\_4\_3\_30\_6 \textcolor{red}{\textcjheb{wlgdw}} WDGLW $|$und sein Panier\\
6.&31.&181.&130.&745.&21.&3&110&70\_30\_10 \textcolor{red}{\textcjheb{yl`}} aLJ $|$"uber mir\\
7.&32.&182.&133.&748.&24.&4&13&1\_5\_2\_5 \textcolor{red}{\textcjheb{hbh'}} AHBH $|$(ist) die Liebe\\
\end{tabular}\medskip \\
Ende des Verses 2.4\\
Verse: 21, Buchstaben: 27, 136, 751, Totalwerte: 768, 8814, 51224\\
\\
Er hat mich in das Haus des Weines gef"uhrt, und sein Panier "uber mir ist die Liebe.\\
\newpage 
{\bf -- 2.5}\\
\medskip \\
\begin{tabular}{rrrrrrrrp{120mm}}
WV&WK&WB&ABK&ABB&ABV&AnzB&TW&Zahlencode \textcolor{red}{$\boldsymbol{Grundtext}$} Umschrift $|$"Ubersetzung(en)\\
1.&33.&183.&137.&752.&1.&6&186&60\_40\_20\_6\_50\_10 \textcolor{red}{\textcjheb{ynwkms}} sMKWNJ $|$st"arket mich/erquicket mich\\
2.&34.&184.&143.&758.&7.&7&1019&2\_1\_300\_10\_300\_6\_400 \textcolor{red}{\textcjheb{tw+sy+s'b}} BASJSWT $|$mit Traubenkuchen\\
3.&35.&185.&150.&765.&14.&6&350&200\_80\_4\_6\_50\_10 \textcolor{red}{\textcjheb{ynwdpr}} RPDWNJ $|$erquicket mich/labet mich\\
4.&36.&186.&156.&771.&20.&7&546&2\_400\_80\_6\_8\_10\_40 \textcolor{red}{\textcjheb{my.hwptb}} BTPWCJM $|$mit "Apfeln\\
5.&37.&187.&163.&778.&27.&2&30&20\_10 \textcolor{red}{\textcjheb{yk}} KJ $|$denn\\
6.&38.&188.&165.&780.&29.&4&444&8\_6\_30\_400 \textcolor{red}{\textcjheb{tlw.h}} CWLT $|$krank\\
7.&39.&189.&169.&784.&33.&4&13&1\_5\_2\_5 \textcolor{red}{\textcjheb{hbh'}} AHBH $|$(vor) Liebe\\
8.&40.&190.&173.&788.&37.&3&61&1\_50\_10 \textcolor{red}{\textcjheb{yn'}} ANJ $|$ich (bin)\\
\end{tabular}\medskip \\
Ende des Verses 2.5\\
Verse: 22, Buchstaben: 39, 175, 790, Totalwerte: 2649, 11463, 53873\\
\\
St"arket mich mit Traubenkuchen, erquicket mich mit "Apfeln, denn ich bin krank vor Liebe! -\\
\newpage 
{\bf -- 2.6}\\
\medskip \\
\begin{tabular}{rrrrrrrrp{120mm}}
WV&WK&WB&ABK&ABB&ABV&AnzB&TW&Zahlencode \textcolor{red}{$\boldsymbol{Grundtext}$} Umschrift $|$"Ubersetzung(en)\\
1.&41.&191.&176.&791.&1.&5&377&300\_40\_1\_30\_6 \textcolor{red}{\textcjheb{wl'm+s}} SMALW $|$seine Linke\\
2.&42.&192.&181.&796.&6.&3&808&400\_8\_400 \textcolor{red}{\textcjheb{t.ht}} TCT $|$ist unter/(ruht) unter\\
3.&43.&193.&184.&799.&9.&5&541&30\_200\_1\_300\_10 \textcolor{red}{\textcjheb{y+s'rl}} LRASJ $|$meinem Haupt\\
4.&44.&194.&189.&804.&14.&6&122&6\_10\_40\_10\_50\_6 \textcolor{red}{\textcjheb{wnymyw}} WJMJNW $|$und seine Rechte\\
5.&45.&195.&195.&810.&20.&6&570&400\_8\_2\_100\_50\_10 \textcolor{red}{\textcjheb{ynqb.ht}} TCBQNJ $|$umfasst mich/(sie) umf"angt mich\\
\end{tabular}\medskip \\
Ende des Verses 2.6\\
Verse: 23, Buchstaben: 25, 200, 815, Totalwerte: 2418, 13881, 56291\\
\\
Seine Linke ist unter meinem Haupte, und seine Rechte umfa"st mich.\\
\newpage 
{\bf -- 2.7}\\
\medskip \\
\begin{tabular}{rrrrrrrrp{120mm}}
WV&WK&WB&ABK&ABB&ABV&AnzB&TW&Zahlencode \textcolor{red}{$\boldsymbol{Grundtext}$} Umschrift $|$"Ubersetzung(en)\\
1.&46.&196.&201.&816.&1.&6&787&5\_300\_2\_70\_400\_10 \textcolor{red}{\textcjheb{yt`b+sh}} HSBaTJ $|$ich beschw"ore\\
2.&47.&197.&207.&822.&7.&4&461&1\_400\_20\_40 \textcolor{red}{\textcjheb{mkt'}} ATKM $|$euch\\
3.&48.&198.&211.&826.&11.&4&458&2\_50\_6\_400 \textcolor{red}{\textcjheb{twnb}} BNWT $|$(ihr) T"ochter\\
4.&49.&199.&215.&830.&15.&6&586&10\_200\_6\_300\_30\_40 \textcolor{red}{\textcjheb{ml+swry}} JRWSLM $|$Jerusalem(s)\\
5.&50.&200.&221.&836.&21.&6&501&2\_90\_2\_1\_6\_400 \textcolor{red}{\textcjheb{tw'b.sb}} B"sBAWT $|$bei den Gazellen\\
6.&51.&201.&227.&842.&27.&2&7&1\_6 \textcolor{red}{\textcjheb{w'}} AW $|$oder\\
7.&52.&202.&229.&844.&29.&6&449&2\_1\_10\_30\_6\_400 \textcolor{red}{\textcjheb{twly'b}} BAJLWT $|$bei den Hindinnen\\
8.&53.&203.&235.&850.&35.&4&314&5\_300\_4\_5 \textcolor{red}{\textcjheb{hd+sh}} HSDH $|$des Feldes\\
9.&54.&204.&239.&854.&39.&2&41&1\_40 \textcolor{red}{\textcjheb{m'}} AM $|$dass nicht/wenn\\
10.&55.&205.&241.&856.&41.&5&686&400\_70\_10\_200\_6 \textcolor{red}{\textcjheb{wry`t}} TaJRW $|$ihr weckt\\
11.&56.&206.&246.&861.&46.&3&47&6\_1\_40 \textcolor{red}{\textcjheb{m'w}} WAM $|$noch/und nicht\\
12.&57.&207.&249.&864.&49.&6&882&400\_70\_6\_200\_200\_6 \textcolor{red}{\textcjheb{wrrw`t}} TaWRRW $|$aufweckt/ihr aufst"ort\\
13.&58.&208.&255.&870.&55.&2&401&1\_400 \textcolor{red}{\textcjheb{t'}} AT $|$**\\
14.&59.&209.&257.&872.&57.&5&18&5\_1\_5\_2\_5 \textcolor{red}{\textcjheb{hbh'h}} HAHBH $|$die Liebe\\
15.&60.&210.&262.&877.&62.&2&74&70\_4 \textcolor{red}{\textcjheb{d`}} aD $|$bis\\
16.&61.&211.&264.&879.&64.&5&878&300\_400\_8\_80\_90 \textcolor{red}{\textcjheb{.sp.ht+s}} STCP"s $|$(dass) es ihr gef"allt\\
\end{tabular}\medskip \\
Ende des Verses 2.7\\
Verse: 24, Buchstaben: 68, 268, 883, Totalwerte: 6590, 20471, 62881\\
\\
Ich beschw"ore euch, T"ochter Jerusalems, bei den Gazellen oder bei den Hindinnen des Feldes, da"s ihr nicht wecket noch aufwecket die Liebe, bis es ihr gef"allt!\\
\newpage 
{\bf -- 2.8}\\
\medskip \\
\begin{tabular}{rrrrrrrrp{120mm}}
WV&WK&WB&ABK&ABB&ABV&AnzB&TW&Zahlencode \textcolor{red}{$\boldsymbol{Grundtext}$} Umschrift $|$"Ubersetzung(en)\\
1.&62.&212.&269.&884.&1.&3&136&100\_6\_30 \textcolor{red}{\textcjheb{lwq}} QWL $|$horch\\
2.&63.&213.&272.&887.&4.&4&24&4\_6\_4\_10 \textcolor{red}{\textcjheb{ydwd}} DWDJ $|$mein Geliebter\\
3.&64.&214.&276.&891.&8.&3&60&5\_50\_5 \textcolor{red}{\textcjheb{hnh}} HNH $|$siehe\\
4.&65.&215.&279.&894.&11.&2&12&7\_5 \textcolor{red}{\textcjheb{hz}} ZH $|$da er/dieser\\
5.&66.&216.&281.&896.&13.&2&3&2\_1 \textcolor{red}{\textcjheb{'b}} BA $|$kommt/kommend(er)\\
6.&67.&217.&283.&898.&15.&4&77&40\_4\_30\_3 \textcolor{red}{\textcjheb{gldm}} MDLG $|$springend(er)\\
7.&68.&218.&287.&902.&19.&2&100&70\_30 \textcolor{red}{\textcjheb{l`}} aL $|$"uber\\
8.&69.&219.&289.&904.&21.&5&260&5\_5\_200\_10\_40 \textcolor{red}{\textcjheb{myrhh}} HHRJM $|$die Berge\\
9.&70.&220.&294.&909.&26.&4&310&40\_100\_80\_90 \textcolor{red}{\textcjheb{.spqm}} MQP"s $|$h"upfend(er)\\
10.&71.&221.&298.&913.&30.&2&100&70\_30 \textcolor{red}{\textcjheb{l`}} aL $|$"uber\\
11.&72.&222.&300.&915.&32.&6&486&5\_3\_2\_70\_6\_400 \textcolor{red}{\textcjheb{tw`bgh}} HGBaWT $|$die H"ugel\\
\end{tabular}\medskip \\
Ende des Verses 2.8\\
Verse: 25, Buchstaben: 37, 305, 920, Totalwerte: 1568, 22039, 64449\\
\\
Horch! mein Geliebter! Siehe, da kommt er, springend "uber die Berge, h"upfend "uber die H"ugel.\\
\newpage 
{\bf -- 2.9}\\
\medskip \\
\begin{tabular}{rrrrrrrrp{120mm}}
WV&WK&WB&ABK&ABB&ABV&AnzB&TW&Zahlencode \textcolor{red}{$\boldsymbol{Grundtext}$} Umschrift $|$"Ubersetzung(en)\\
1.&73.&223.&306.&921.&1.&4&55&4\_6\_40\_5 \textcolor{red}{\textcjheb{hmwd}} DWMH $|$(es) gleicht/gleichend(er) (ist)\\
2.&74.&224.&310.&925.&5.&4&24&4\_6\_4\_10 \textcolor{red}{\textcjheb{ydwd}} DWDJ $|$mein Geliebter/mein Freund\\
3.&75.&225.&314.&929.&9.&4&132&30\_90\_2\_10 \textcolor{red}{\textcjheb{yb.sl}} L"sBJ $|$einer Gazelle/der Gazelle\\
4.&76.&226.&318.&933.&13.&2&7&1\_6 \textcolor{red}{\textcjheb{w'}} AW $|$oder\\
5.&77.&227.&320.&935.&15.&4&380&30\_70\_80\_200 \textcolor{red}{\textcjheb{rp`l}} LaPR $|$einem Jungen/dem Kitz\\
6.&78.&228.&324.&939.&19.&6&96&5\_1\_10\_30\_10\_40 \textcolor{red}{\textcjheb{myly'h}} HAJLJM $|$der Hirsche\\
7.&79.&229.&330.&945.&25.&3&60&5\_50\_5 \textcolor{red}{\textcjheb{hnh}} HNH $|$siehe\\
8.&80.&230.&333.&948.&28.&2&12&7\_5 \textcolor{red}{\textcjheb{hz}} ZH $|$da er/dieser\\
9.&81.&231.&335.&950.&30.&4&120&70\_6\_40\_4 \textcolor{red}{\textcjheb{dmw`}} aWMD $|$steht/(ist) stehend(er)\\
10.&82.&232.&339.&954.&34.&3&209&1\_8\_200 \textcolor{red}{\textcjheb{r.h'}} ACR $|$hinter\\
11.&83.&233.&342.&957.&37.&5&506&20\_400\_30\_50\_6 \textcolor{red}{\textcjheb{wnltk}} KTLNW $|$unserer Mauer/unserer (Haus)Wand\\
12.&84.&234.&347.&962.&42.&5&361&40\_300\_3\_10\_8 \textcolor{red}{\textcjheb{.hyg+sm}} MSGJC $|$schaut/(ist) schauend(er)\\
13.&85.&235.&352.&967.&47.&2&90&40\_50 \textcolor{red}{\textcjheb{nm}} MN $|$durch\\
14.&86.&236.&354.&969.&49.&6&499&5\_8\_30\_50\_6\_400 \textcolor{red}{\textcjheb{twnl.hh}} HCLNWT $|$die Fenster\\
15.&87.&237.&360.&975.&55.&4&230&40\_90\_10\_90 \textcolor{red}{\textcjheb{.sy.sm}} M"sJ"s $|$blickt/sp"ahend(er)\\
16.&88.&238.&364.&979.&59.&2&90&40\_50 \textcolor{red}{\textcjheb{nm}} MN $|$durch\\
17.&89.&239.&366.&981.&61.&6&283&5\_8\_200\_20\_10\_40 \textcolor{red}{\textcjheb{mykr.hh}} HCRKJM $|$die Gitter\\
\end{tabular}\medskip \\
Ende des Verses 2.9\\
Verse: 26, Buchstaben: 66, 371, 986, Totalwerte: 3154, 25193, 67603\\
\\
Mein Geliebter gleicht einer Gazelle, oder einem Jungen der Hirsche. Siehe, da steht er hinter unserer Mauer, schaut durch die Fenster, blickt durch die Gitter.\\
\newpage 
{\bf -- 2.10}\\
\medskip \\
\begin{tabular}{rrrrrrrrp{120mm}}
WV&WK&WB&ABK&ABB&ABV&AnzB&TW&Zahlencode \textcolor{red}{$\boldsymbol{Grundtext}$} Umschrift $|$"Ubersetzung(en)\\
1.&90.&240.&372.&987.&1.&3&125&70\_50\_5 \textcolor{red}{\textcjheb{hn`}} aNH $|$(es) hob an/er (=es) hebt an\\
2.&91.&241.&375.&990.&4.&4&24&4\_6\_4\_10 \textcolor{red}{\textcjheb{ydwd}} DWDJ $|$mein Geliebter/mein Freund\\
3.&92.&242.&379.&994.&8.&4&247&6\_1\_40\_200 \textcolor{red}{\textcjheb{rm'w}} WAMR $|$und sprach/und er sagt(e)\\
4.&93.&243.&383.&998.&12.&2&40&30\_10 \textcolor{red}{\textcjheb{yl}} LJ $|$zu mir\\
5.&94.&244.&385.&1000.&14.&4&156&100\_6\_40\_10 \textcolor{red}{\textcjheb{ymwq}} QWMJ $|$mache auf/stehe auf\\
6.&95.&245.&389.&1004.&18.&2&50&30\_20 \textcolor{red}{\textcjheb{kl}} LK $|$dich/zu dir\\
7.&96.&246.&391.&1006.&20.&5&690&200\_70\_10\_400\_10 \textcolor{red}{\textcjheb{yty`r}} RaJTJ $|$meine Freundin\\
8.&97.&247.&396.&1011.&25.&4&500&10\_80\_400\_10 \textcolor{red}{\textcjheb{ytpy}} JPTJ $|$meine Sch"one\\
9.&98.&248.&400.&1015.&29.&4&66&6\_30\_20\_10 \textcolor{red}{\textcjheb{yklw}} WLKJ $|$und komm\\
10.&99.&249.&404.&1019.&33.&2&50&30\_20 \textcolor{red}{\textcjheb{kl}} LK $|$/doch\\
\end{tabular}\medskip \\
Ende des Verses 2.10\\
Verse: 27, Buchstaben: 34, 405, 1020, Totalwerte: 1948, 27141, 69551\\
\\
Mein Geliebter hob an und sprach zu mir: Mache dich auf, meine Freundin, meine Sch"one, und komm!\\
\newpage 
{\bf -- 2.11}\\
\medskip \\
\begin{tabular}{rrrrrrrrp{120mm}}
WV&WK&WB&ABK&ABB&ABV&AnzB&TW&Zahlencode \textcolor{red}{$\boldsymbol{Grundtext}$} Umschrift $|$"Ubersetzung(en)\\
1.&100.&250.&406.&1021.&1.&2&30&20\_10 \textcolor{red}{\textcjheb{yk}} KJ $|$denn\\
2.&101.&251.&408.&1023.&3.&3&60&5\_50\_5 \textcolor{red}{\textcjheb{hnh}} HNH $|$siehe\\
3.&102.&252.&411.&1026.&6.&4&471&5\_60\_400\_6 \textcolor{red}{\textcjheb{wtsh}} HsTW $|$der Winter\\
4.&103.&253.&415.&1030.&10.&3&272&70\_2\_200 \textcolor{red}{\textcjheb{rb`}} aBR $|$ist vorbei/er ging vor"uber\\
5.&104.&254.&418.&1033.&13.&4&348&5\_3\_300\_40 \textcolor{red}{\textcjheb{m+sgh}} HGSM $|$der Regen\\
6.&105.&255.&422.&1037.&17.&3&118&8\_30\_80 \textcolor{red}{\textcjheb{pl.h}} CLP $|$ist vor"uber/(er) zog vorbei\\
7.&106.&256.&425.&1040.&20.&3&55&5\_30\_20 \textcolor{red}{\textcjheb{klh}} HLK $|$er ist/er ging\\
8.&107.&257.&428.&1043.&23.&2&36&30\_6 \textcolor{red}{\textcjheb{wl}} LW $|$dahin\\
\end{tabular}\medskip \\
Ende des Verses 2.11\\
Verse: 28, Buchstaben: 24, 429, 1044, Totalwerte: 1390, 28531, 70941\\
\\
Denn siehe, der Winter ist vorbei, der Regen ist vor"uber, er ist dahin.\\
\newpage 
{\bf -- 2.12}\\
\medskip \\
\begin{tabular}{rrrrrrrrp{120mm}}
WV&WK&WB&ABK&ABB&ABV&AnzB&TW&Zahlencode \textcolor{red}{$\boldsymbol{Grundtext}$} Umschrift $|$"Ubersetzung(en)\\
1.&108.&258.&430.&1045.&1.&6&245&5\_50\_90\_50\_10\_40 \textcolor{red}{\textcjheb{myn.snh}} HN"sNJM $|$die Blumen/die Bl"uten\\
2.&109.&259.&436.&1051.&7.&4&257&50\_200\_1\_6 \textcolor{red}{\textcjheb{w'rn}} NRAW $|$erscheinen/(sie) wurden gesehen\\
3.&110.&260.&440.&1055.&11.&4&293&2\_1\_200\_90 \textcolor{red}{\textcjheb{.sr'b}} BAR"s $|$im Lande\\
4.&111.&261.&444.&1059.&15.&2&470&70\_400 \textcolor{red}{\textcjheb{t`}} aT $|$die Zeit\\
5.&112.&262.&446.&1061.&17.&5&262&5\_7\_40\_10\_200 \textcolor{red}{\textcjheb{rymzh}} HZMJR $|$des Gesangs\\
6.&113.&263.&451.&1066.&22.&4&88&5\_3\_10\_70 \textcolor{red}{\textcjheb{`ygh}} HGJa $|$ist gekommen/er (=sie) ist eingetroffen\\
7.&114.&264.&455.&1070.&26.&4&142&6\_100\_6\_30 \textcolor{red}{\textcjheb{lwqw}} WQWL $|$und die Stimme\\
8.&115.&265.&459.&1074.&30.&4&611&5\_400\_6\_200 \textcolor{red}{\textcjheb{rwth}} HTWR $|$der Turteltaube\\
9.&116.&266.&463.&1078.&34.&4&460&50\_300\_40\_70 \textcolor{red}{\textcjheb{`m+sn}} NSMa $|$l"asst sich h"oren/er (=sie) wurde geh"ort\\
10.&117.&267.&467.&1082.&38.&6&349&2\_1\_200\_90\_50\_6 \textcolor{red}{\textcjheb{wn.sr'b}} BAR"sNW $|$in unserem Lande\\
\end{tabular}\medskip \\
Ende des Verses 2.12\\
Verse: 29, Buchstaben: 43, 472, 1087, Totalwerte: 3177, 31708, 74118\\
\\
Die Blumen erscheinen im Lande, die Zeit des Gesanges ist gekommen, und die Stimme der Turteltaube l"a"st sich h"oren in unserem Lande.\\
\newpage 
{\bf -- 2.13}\\
\medskip \\
\begin{tabular}{rrrrrrrrp{120mm}}
WV&WK&WB&ABK&ABB&ABV&AnzB&TW&Zahlencode \textcolor{red}{$\boldsymbol{Grundtext}$} Umschrift $|$"Ubersetzung(en)\\
1.&118.&268.&473.&1088.&1.&5&461&5\_400\_1\_50\_5 \textcolor{red}{\textcjheb{hn'th}} HTANH $|$der Feigenbaum\\
2.&119.&269.&478.&1093.&6.&4&72&8\_50\_9\_5 \textcolor{red}{\textcjheb{h.tn.h}} CNtH $|$r"otet/sie (=er) treibt\\
3.&120.&270.&482.&1097.&10.&4&98&80\_3\_10\_5 \textcolor{red}{\textcjheb{hygp}} PGJH $|$seine Feigen/seine Fruchtkeime\\
4.&121.&271.&486.&1101.&14.&7&194&6\_5\_3\_80\_50\_10\_40 \textcolor{red}{\textcjheb{mynpghw}} WHGPNJM $|$und die Weinst"ocke sind/und die Reben\\
5.&122.&272.&493.&1108.&21.&4&304&60\_40\_4\_200 \textcolor{red}{\textcjheb{rdms}} sMDR $|$in der Bl"ute\\
6.&123.&273.&497.&1112.&25.&4&506&50\_400\_50\_6 \textcolor{red}{\textcjheb{wntn}} NTNW $|$geben/sie gaben\\
7.&124.&274.&501.&1116.&29.&3&218&200\_10\_8 \textcolor{red}{\textcjheb{.hyr}} RJC $|$Duft/(Wohl)Geruch\\
8.&125.&275.&504.&1119.&32.&4&156&100\_6\_40\_10 \textcolor{red}{\textcjheb{ymwq}} QWMJ $|$auf/steh auf\\
9.&126.&276.&508.&1123.&36.&3&60&30\_20\_10 \textcolor{red}{\textcjheb{ykl}} LKJ $|$mache dich/komm\\
10.&127.&277.&511.&1126.&39.&5&690&200\_70\_10\_400\_10 \textcolor{red}{\textcjheb{yty`r}} RaJTJ $|$meine Freundin\\
11.&128.&278.&516.&1131.&44.&4&500&10\_80\_400\_10 \textcolor{red}{\textcjheb{ytpy}} JPTJ $|$meine Sch"one\\
12.&129.&279.&520.&1135.&48.&4&66&6\_30\_20\_10 \textcolor{red}{\textcjheb{yklw}} WLKJ $|$und komm\\
13.&130.&280.&524.&1139.&52.&2&50&30\_20 \textcolor{red}{\textcjheb{kl}} LK $|$/doch\\
\end{tabular}\medskip \\
Ende des Verses 2.13\\
Verse: 30, Buchstaben: 53, 525, 1140, Totalwerte: 3375, 35083, 77493\\
\\
Der Feigenbaum r"otet seine Feigen, und die Weinst"ocke sind in der Bl"ute, geben Duft. Mache dich auf, meine Freundin, meine Sch"one, und komm!\\
\newpage 
{\bf -- 2.14}\\
\medskip \\
\begin{tabular}{rrrrrrrrp{120mm}}
WV&WK&WB&ABK&ABB&ABV&AnzB&TW&Zahlencode \textcolor{red}{$\boldsymbol{Grundtext}$} Umschrift $|$"Ubersetzung(en)\\
1.&131.&281.&526.&1141.&1.&5&476&10\_6\_50\_400\_10 \textcolor{red}{\textcjheb{ytnwy}} JWNTJ $|$meine Taube\\
2.&132.&282.&531.&1146.&6.&5&29&2\_8\_3\_6\_10 \textcolor{red}{\textcjheb{ywg.hb}} BCGWJ $|$im Gekl"uft/in den K"uften\\
3.&133.&283.&536.&1151.&11.&4&165&5\_60\_30\_70 \textcolor{red}{\textcjheb{`lsh}} HsLa $|$der Felsen/des Felsens\\
4.&134.&284.&540.&1155.&15.&4&662&2\_60\_400\_200 \textcolor{red}{\textcjheb{rtsb}} BsTR $|$im Versteck\\
5.&135.&285.&544.&1159.&19.&6&257&5\_40\_4\_200\_3\_5 \textcolor{red}{\textcjheb{hgrdmh}} HMDRGH $|$der Felsw"ande/des Felsensteiges\\
6.&136.&286.&550.&1165.&25.&6&276&5\_200\_1\_10\_50\_10 \textcolor{red}{\textcjheb{yny'rh}} HRAJNJ $|$lass mich sehen/lass mich schauen\\
7.&137.&287.&556.&1171.&31.&2&401&1\_400 \textcolor{red}{\textcjheb{t'}} AT $|$**\\
8.&138.&288.&558.&1173.&33.&5&271&40\_200\_1\_10\_20 \textcolor{red}{\textcjheb{ky'rm}} MRAJK $|$deine Gestalt/dein Aussehen\\
9.&139.&289.&563.&1178.&38.&8&495&5\_300\_40\_10\_70\_10\_50\_10 \textcolor{red}{\textcjheb{yny`ym+sh}} HSMJaJNJ $|$lass mich h"oren\\
10.&140.&290.&571.&1186.&46.&2&401&1\_400 \textcolor{red}{\textcjheb{t'}} AT $|$**\\
11.&141.&291.&573.&1188.&48.&4&156&100\_6\_30\_20 \textcolor{red}{\textcjheb{klwq}} QWLK $|$deine Stimme\\
12.&142.&292.&577.&1192.&52.&2&30&20\_10 \textcolor{red}{\textcjheb{yk}} KJ $|$denn\\
13.&143.&293.&579.&1194.&54.&4&156&100\_6\_30\_20 \textcolor{red}{\textcjheb{klwq}} QWLK $|$deine Stimme\\
14.&144.&294.&583.&1198.&58.&3&272&70\_200\_2 \textcolor{red}{\textcjheb{br`}} aRB $|$ist s"u"s/(ist) angenehm\\
15.&145.&295.&586.&1201.&61.&6&277&6\_40\_200\_1\_10\_20 \textcolor{red}{\textcjheb{ky'rmw}} WMRAJK $|$und deine Gestalt/und dein Aussehen\\
16.&146.&296.&592.&1207.&67.&4&62&50\_1\_6\_5 \textcolor{red}{\textcjheb{hw'n}} NAWH $|$anmutig/(ist) lieblich\\
\end{tabular}\medskip \\
Ende des Verses 2.14\\
Verse: 31, Buchstaben: 70, 595, 1210, Totalwerte: 4386, 39469, 81879\\
\\
Meine Taube im Gekl"uft der Felsen, im Versteck der Felsw"ande, la"s mich deine Gestalt sehen, la"s mich deine Stimme h"oren; denn deine Stimme ist s"u"s und deine Gestalt anmutig. -\\
\newpage 
{\bf -- 2.15}\\
\medskip \\
\begin{tabular}{rrrrrrrrp{120mm}}
WV&WK&WB&ABK&ABB&ABV&AnzB&TW&Zahlencode \textcolor{red}{$\boldsymbol{Grundtext}$} Umschrift $|$"Ubersetzung(en)\\
1.&147.&297.&596.&1211.&1.&4&22&1\_8\_7\_6 \textcolor{red}{\textcjheb{wz.h'}} ACZW $|$fanget\\
2.&148.&298.&600.&1215.&5.&3&86&30\_50\_6 \textcolor{red}{\textcjheb{wnl}} LNW $|$(f"ur) uns\\
3.&149.&299.&603.&1218.&8.&6&456&300\_6\_70\_30\_10\_40 \textcolor{red}{\textcjheb{myl`w+s}} SWaLJM $|$(die) F"uchse\\
4.&150.&300.&609.&1224.&14.&6&456&300\_6\_70\_30\_10\_40 \textcolor{red}{\textcjheb{myl`w+s}} SWaLJM $|$(die) F"uchse\\
5.&151.&301.&615.&1230.&20.&5&209&100\_9\_50\_10\_40 \textcolor{red}{\textcjheb{myn.tq}} QtNJM $|$kleine(n)\\
6.&152.&302.&620.&1235.&25.&6&130&40\_8\_2\_30\_10\_40 \textcolor{red}{\textcjheb{mylb.hm}} MCBLJM $|$welche verderben/die zugrunde richtend(e) (sind)\\
7.&153.&303.&626.&1241.&31.&5&310&20\_200\_40\_10\_40 \textcolor{red}{\textcjheb{mymrk}} KRMJM $|$die Weinberge/Weing"arten\\
8.&154.&304.&631.&1246.&36.&7&332&6\_20\_200\_40\_10\_50\_6 \textcolor{red}{\textcjheb{wnymrkw}} WKRMJNW $|$denn unsere Weinberge/und unsere Weing"arten\\
9.&155.&305.&638.&1253.&43.&4&304&60\_40\_4\_200 \textcolor{red}{\textcjheb{rdms}} sMDR $|$in (der) Bl"ute (sind)\\
\end{tabular}\medskip \\
Ende des Verses 2.15\\
Verse: 32, Buchstaben: 46, 641, 1256, Totalwerte: 2305, 41774, 84184\\
\\
Fanget uns die F"uchse, die kleinen F"uchse, welche die Weinberge verderben; denn unsere Weinberge sind in der Bl"ute!\\
\newpage 
{\bf -- 2.16}\\
\medskip \\
\begin{tabular}{rrrrrrrrp{120mm}}
WV&WK&WB&ABK&ABB&ABV&AnzB&TW&Zahlencode \textcolor{red}{$\boldsymbol{Grundtext}$} Umschrift $|$"Ubersetzung(en)\\
1.&156.&306.&642.&1257.&1.&4&24&4\_6\_4\_10 \textcolor{red}{\textcjheb{ydwd}} DWDJ $|$mein Geliebter/mein Freund\\
2.&157.&307.&646.&1261.&5.&2&40&30\_10 \textcolor{red}{\textcjheb{yl}} LJ $|$mein (ist)\\
3.&158.&308.&648.&1263.&7.&4&67&6\_1\_50\_10 \textcolor{red}{\textcjheb{yn'w}} WANJ $|$und ich\\
4.&159.&309.&652.&1267.&11.&2&36&30\_6 \textcolor{red}{\textcjheb{wl}} LW $|$bin sein\\
5.&160.&310.&654.&1269.&13.&4&280&5\_200\_70\_5 \textcolor{red}{\textcjheb{h`rh}} HRaH $|$der weidet/der weidende(r) (ist)\\
6.&161.&311.&658.&1273.&17.&7&708&2\_300\_6\_300\_50\_10\_40 \textcolor{red}{\textcjheb{myn+sw+sb}} BSWSNJM $|$unter den (wei"sen) Lilien\\
\end{tabular}\medskip \\
Ende des Verses 2.16\\
Verse: 33, Buchstaben: 23, 664, 1279, Totalwerte: 1155, 42929, 85339\\
\\
Mein Geliebter ist mein, und ich bin sein, der unter den Lilien weidet. -\\
\newpage 
{\bf -- 2.17}\\
\medskip \\
\begin{tabular}{rrrrrrrrp{120mm}}
WV&WK&WB&ABK&ABB&ABV&AnzB&TW&Zahlencode \textcolor{red}{$\boldsymbol{Grundtext}$} Umschrift $|$"Ubersetzung(en)\\
1.&162.&312.&665.&1280.&1.&2&74&70\_4 \textcolor{red}{\textcjheb{d`}} aD $|$bis\\
2.&163.&313.&667.&1282.&3.&5&404&300\_10\_80\_6\_8 \textcolor{red}{\textcjheb{.hwpy+s}} SJPWC $|$sich k"uhlt/dass er (=es) weht\\
3.&164.&314.&672.&1287.&8.&4&61&5\_10\_6\_40 \textcolor{red}{\textcjheb{mwyh}} HJWM $|$der Tag\\
4.&165.&315.&676.&1291.&12.&4&122&6\_50\_60\_6 \textcolor{red}{\textcjheb{wsnw}} WNsW $|$und (sie (=es)) fliehen\\
5.&166.&316.&680.&1295.&16.&6&205&5\_90\_30\_30\_10\_40 \textcolor{red}{\textcjheb{myll.sh}} H"sLLJM $|$die Schatten\\
6.&167.&317.&686.&1301.&22.&2&62&60\_2 \textcolor{red}{\textcjheb{bs}} sB $|$wende dich/wandle\\
7.&168.&318.&688.&1303.&24.&3&49&4\_40\_5 \textcolor{red}{\textcjheb{hmd}} DMH $|$sei gleich/mach "ahnlich\\
8.&169.&319.&691.&1306.&27.&2&50&30\_20 \textcolor{red}{\textcjheb{kl}} LK $|$/dich\\
9.&170.&320.&693.&1308.&29.&4&24&4\_6\_4\_10 \textcolor{red}{\textcjheb{ydwd}} DWDJ $|$mein Geliebter/mein Freund\\
10.&171.&321.&697.&1312.&33.&4&132&30\_90\_2\_10 \textcolor{red}{\textcjheb{yb.sl}} L"sBJ $|$einer Gazelle\\
11.&172.&322.&701.&1316.&37.&2&7&1\_6 \textcolor{red}{\textcjheb{w'}} AW $|$oder\\
12.&173.&323.&703.&1318.&39.&4&380&30\_70\_80\_200 \textcolor{red}{\textcjheb{rp`l}} LaPR $|$einem Jungen/einem Kitz\\
13.&174.&324.&707.&1322.&43.&6&96&5\_1\_10\_30\_10\_40 \textcolor{red}{\textcjheb{myly'h}} HAJLJM $|$der Hirsche\\
14.&175.&325.&713.&1328.&49.&2&100&70\_30 \textcolor{red}{\textcjheb{l`}} aL $|$auf\\
15.&176.&326.&715.&1330.&51.&3&215&5\_200\_10 \textcolor{red}{\textcjheb{yrh}} HRJ $|$den Bergen\\
16.&177.&327.&718.&1333.&54.&3&602&2\_400\_200 \textcolor{red}{\textcjheb{rtb}} BTR $|$zerkl"ufteten/(im) Felsengekl"uft\\
\end{tabular}\medskip \\
Ende des Verses 2.17\\
Verse: 34, Buchstaben: 56, 720, 1335, Totalwerte: 2583, 45512, 87922\\
\\
Bis der Tag sich k"uhlt und die Schatten fliehen, wende dich, sei, mein Geliebter, gleich einer Gazelle oder einem Jungen der Hirsche auf den zerkl"ufteten Bergen!\\
\\
{\bf Ende des Kapitels 2}\\
\newpage 
{\bf -- 3.1}\\
\medskip \\
\begin{tabular}{rrrrrrrrp{120mm}}
WV&WK&WB&ABK&ABB&ABV&AnzB&TW&Zahlencode \textcolor{red}{$\boldsymbol{Grundtext}$} Umschrift $|$"Ubersetzung(en)\\
1.&1.&328.&1.&1336.&1.&2&100&70\_30 \textcolor{red}{\textcjheb{l`}} aL $|$auf\\
2.&2.&329.&3.&1338.&3.&5&372&40\_300\_20\_2\_10 \textcolor{red}{\textcjheb{ybk+sm}} MSKBJ $|$meinem Lager\\
3.&3.&330.&8.&1343.&8.&6&478&2\_30\_10\_30\_6\_400 \textcolor{red}{\textcjheb{twlylb}} BLJLWT $|$in den N"achten\\
4.&4.&331.&14.&1349.&14.&5&812&2\_100\_300\_400\_10 \textcolor{red}{\textcjheb{yt+sqb}} BQSTJ $|$ich suchte\\
5.&5.&332.&19.&1354.&19.&2&401&1\_400 \textcolor{red}{\textcjheb{t'}} AT $|$**\\
6.&6.&333.&21.&1356.&21.&5&313&300\_1\_5\_2\_5 \textcolor{red}{\textcjheb{hbh'+s}} SAHBH $|$den liebt/welchen sie (=es) liebte\\
7.&7.&334.&26.&1361.&26.&4&440&50\_80\_300\_10 \textcolor{red}{\textcjheb{y+spn}} NPSJ $|$meine Seele\\
8.&8.&335.&30.&1365.&30.&6&818&2\_100\_300\_400\_10\_6 \textcolor{red}{\textcjheb{wyt+sqb}} BQSTJW $|$ich suchte ihn\\
9.&9.&336.&36.&1371.&36.&3&37&6\_30\_1 \textcolor{red}{\textcjheb{'lw}} WLA $|$und nicht\\
10.&10.&337.&39.&1374.&39.&6&547&40\_90\_1\_400\_10\_6 \textcolor{red}{\textcjheb{wyt'.sm}} M"sATJW $|$ich fand ihn\\
\end{tabular}\medskip \\
Ende des Verses 3.1\\
Verse: 35, Buchstaben: 44, 44, 1379, Totalwerte: 4318, 4318, 92240\\
\\
Auf meinem Lager in den N"achten suchte ich, den meine Seele liebt: ich suchte ihn und fand ihn nicht.\\
\newpage 
{\bf -- 3.2}\\
\medskip \\
\begin{tabular}{rrrrrrrrp{120mm}}
WV&WK&WB&ABK&ABB&ABV&AnzB&TW&Zahlencode \textcolor{red}{$\boldsymbol{Grundtext}$} Umschrift $|$"Ubersetzung(en)\\
1.&11.&338.&45.&1380.&1.&5&152&1\_100\_6\_40\_5 \textcolor{red}{\textcjheb{hmwq'}} AQWMH $|$ich will aufstehen/ich will mich aufmachen\\
2.&12.&339.&50.&1385.&6.&2&51&50\_1 \textcolor{red}{\textcjheb{'n}} NA $|$doch\\
3.&13.&340.&52.&1387.&8.&7&82&6\_1\_60\_6\_2\_2\_5 \textcolor{red}{\textcjheb{hbbws'w}} WAsWBBH $|$und umhergehen/und ich will herumgehen\\
4.&14.&341.&59.&1394.&15.&4&282&2\_70\_10\_200 \textcolor{red}{\textcjheb{ry`b}} BaJR $|$in der Stadt\\
5.&15.&342.&63.&1398.&19.&6&458&2\_300\_6\_100\_10\_40 \textcolor{red}{\textcjheb{myqw+sb}} BSWQJM $|$auf den Stra"sen\\
6.&16.&343.&69.&1404.&25.&7&624&6\_2\_200\_8\_2\_6\_400 \textcolor{red}{\textcjheb{twb.hrbw}} WBRCBWT $|$und auf den Pl"atzen\\
7.&17.&344.&76.&1411.&32.&5&408&1\_2\_100\_300\_5 \textcolor{red}{\textcjheb{h+sqb'}} ABQSH $|$(ich) will suchen\\
8.&18.&345.&81.&1416.&37.&2&401&1\_400 \textcolor{red}{\textcjheb{t'}} AT $|$**\\
9.&19.&346.&83.&1418.&39.&5&313&300\_1\_5\_2\_5 \textcolor{red}{\textcjheb{hbh'+s}} SAHBH $|$den liebt/welchen sie (=es) liebt(e)\\
10.&20.&347.&88.&1423.&44.&4&440&50\_80\_300\_10 \textcolor{red}{\textcjheb{y+spn}} NPSJ $|$meine Seele\\
11.&21.&348.&92.&1427.&48.&6&818&2\_100\_300\_400\_10\_6 \textcolor{red}{\textcjheb{wyt+sqb}} BQSTJW $|$ich suchte ihn\\
12.&22.&349.&98.&1433.&54.&3&37&6\_30\_1 \textcolor{red}{\textcjheb{'lw}} WLA $|$und nicht\\
13.&23.&350.&101.&1436.&57.&6&547&40\_90\_1\_400\_10\_6 \textcolor{red}{\textcjheb{wyt'.sm}} M"sATJW $|$(ich) fand ihn\\
\end{tabular}\medskip \\
Ende des Verses 3.2\\
Verse: 36, Buchstaben: 62, 106, 1441, Totalwerte: 4613, 8931, 96853\\
\\
Ich will doch aufstehen und in der Stadt umhergehen, auf den Stra"sen und auf den Pl"atzen, will suchen, den meine Seele liebt. Ich suchte ihn und fand ihn nicht.\\
\newpage 
{\bf -- 3.3}\\
\medskip \\
\begin{tabular}{rrrrrrrrp{120mm}}
WV&WK&WB&ABK&ABB&ABV&AnzB&TW&Zahlencode \textcolor{red}{$\boldsymbol{Grundtext}$} Umschrift $|$"Ubersetzung(en)\\
1.&24.&351.&107.&1442.&1.&6&197&40\_90\_1\_6\_50\_10 \textcolor{red}{\textcjheb{ynw'.sm}} M"sAWNJ $|$sie (=es) fanden mich\\
2.&25.&352.&113.&1448.&7.&6&595&5\_300\_40\_200\_10\_40 \textcolor{red}{\textcjheb{myrm+sh}} HSMRJM $|$die W"achter\\
3.&26.&353.&119.&1454.&13.&6&119&5\_60\_2\_2\_10\_40 \textcolor{red}{\textcjheb{mybbsh}} HsBBJM $|$die umhergehen(de(n))\\
4.&27.&354.&125.&1460.&19.&4&282&2\_70\_10\_200 \textcolor{red}{\textcjheb{ry`b}} BaJR $|$in der Stadt\\
5.&28.&355.&129.&1464.&23.&2&401&1\_400 \textcolor{red}{\textcjheb{t'}} AT $|$**\\
6.&29.&356.&131.&1466.&25.&5&313&300\_1\_5\_2\_5 \textcolor{red}{\textcjheb{hbh'+s}} SAHBH $|$den liebt/welchen sie (=es) liebt(e)\\
7.&30.&357.&136.&1471.&30.&4&440&50\_80\_300\_10 \textcolor{red}{\textcjheb{y+spn}} NPSJ $|$meine Seele\\
8.&31.&358.&140.&1475.&34.&5&651&200\_1\_10\_400\_40 \textcolor{red}{\textcjheb{mty'r}} RAJTM $|$ihr habt (den) gesehen\\
\end{tabular}\medskip \\
Ende des Verses 3.3\\
Verse: 37, Buchstaben: 38, 144, 1479, Totalwerte: 2998, 11929, 99851\\
\\
Es fanden mich die W"achter, die in der Stadt umhergehen: Habt ihr den gesehen, den meine Seele liebt?\\
\newpage 
{\bf -- 3.4}\\
\medskip \\
\begin{tabular}{rrrrrrrrp{120mm}}
WV&WK&WB&ABK&ABB&ABV&AnzB&TW&Zahlencode \textcolor{red}{$\boldsymbol{Grundtext}$} Umschrift $|$"Ubersetzung(en)\\
1.&32.&359.&145.&1480.&1.&4&139&20\_40\_70\_9 \textcolor{red}{\textcjheb{.t`mk}} KMat $|$kaum war ich/ein wenig nur war es\\
2.&33.&360.&149.&1484.&5.&6&982&300\_70\_2\_200\_400\_10 \textcolor{red}{\textcjheb{ytrb`+s}} SaBRTJ $|$vor"uber/dass ich war vorbeigegangen\\
3.&34.&361.&155.&1490.&11.&3&85&40\_5\_40 \textcolor{red}{\textcjheb{mhm}} MHM $|$an ihnen\\
4.&35.&362.&158.&1493.&14.&2&74&70\_4 \textcolor{red}{\textcjheb{d`}} aD $|$da/bis\\
5.&36.&363.&160.&1495.&16.&6&841&300\_40\_90\_1\_400\_10 \textcolor{red}{\textcjheb{yt'.sm+s}} SM"sATJ $|$(dass) ich fand\\
6.&37.&364.&166.&1501.&22.&2&401&1\_400 \textcolor{red}{\textcjheb{t'}} AT $|$**\\
7.&38.&365.&168.&1503.&24.&5&313&300\_1\_5\_2\_5 \textcolor{red}{\textcjheb{hbh'+s}} SAHBH $|$den liebt/welchen sie (=es) liebt(e)\\
8.&39.&366.&173.&1508.&29.&4&440&50\_80\_300\_10 \textcolor{red}{\textcjheb{y+spn}} NPSJ $|$meine Seele\\
9.&40.&367.&177.&1512.&33.&6&432&1\_8\_7\_400\_10\_6 \textcolor{red}{\textcjheb{wytz.h'}} ACZTJW $|$ich ergriff ihn/ich hielt fest ihn\\
10.&41.&368.&183.&1518.&39.&3&37&6\_30\_1 \textcolor{red}{\textcjheb{'lw}} WLA $|$und nicht\\
11.&42.&369.&186.&1521.&42.&5&337&1\_200\_80\_50\_6 \textcolor{red}{\textcjheb{wnpr'}} ARPNW $|$lie"s ihn/ich lasse ihn\\
12.&43.&370.&191.&1526.&47.&2&74&70\_4 \textcolor{red}{\textcjheb{d`}} aD $|$bis\\
13.&44.&371.&193.&1528.&49.&8&734&300\_5\_2\_10\_1\_400\_10\_6 \textcolor{red}{\textcjheb{wyt'ybh+s}} SHBJATJW $|$ich gebracht hatte ihn/dass ich mach(t)e kommen ihn\\
14.&45.&372.&201.&1536.&57.&2&31&1\_30 \textcolor{red}{\textcjheb{l'}} AL $|$in\\
15.&46.&373.&203.&1538.&59.&3&412&2\_10\_400 \textcolor{red}{\textcjheb{tyb}} BJT $|$das Haus\\
16.&47.&374.&206.&1541.&62.&3&51&1\_40\_10 \textcolor{red}{\textcjheb{ym'}} AMJ $|$meiner Mutter\\
17.&48.&375.&209.&1544.&65.&3&37&6\_1\_30 \textcolor{red}{\textcjheb{l'w}} WAL $|$und in das/und zum\\
18.&49.&376.&212.&1547.&68.&3&212&8\_4\_200 \textcolor{red}{\textcjheb{rd.h}} CDR $|$Gemach\\
19.&50.&377.&215.&1550.&71.&5&621&5\_6\_200\_400\_10 \textcolor{red}{\textcjheb{ytrwh}} HWRTJ $|$meiner Geb"arerin\\
\end{tabular}\medskip \\
Ende des Verses 3.4\\
Verse: 38, Buchstaben: 75, 219, 1554, Totalwerte: 6253, 18182, 106104\\
\\
Kaum war ich an ihnen vor"uber, da fand ich, den meine Seele liebt. Ich ergriff ihn und lie"s ihn nicht, bis ich ihn gebracht hatte in das Haus meiner Mutter und in das Gemach meiner Geb"arerin.\\
\newpage 
{\bf -- 3.5}\\
\medskip \\
\begin{tabular}{rrrrrrrrp{120mm}}
WV&WK&WB&ABK&ABB&ABV&AnzB&TW&Zahlencode \textcolor{red}{$\boldsymbol{Grundtext}$} Umschrift $|$"Ubersetzung(en)\\
1.&51.&378.&220.&1555.&1.&6&787&5\_300\_2\_70\_400\_10 \textcolor{red}{\textcjheb{yt`b+sh}} HSBaTJ $|$ich beschw"ore\\
2.&52.&379.&226.&1561.&7.&4&461&1\_400\_20\_40 \textcolor{red}{\textcjheb{mkt'}} ATKM $|$euch\\
3.&53.&380.&230.&1565.&11.&4&458&2\_50\_6\_400 \textcolor{red}{\textcjheb{twnb}} BNWT $|$(ihr) T"ochter\\
4.&54.&381.&234.&1569.&15.&6&586&10\_200\_6\_300\_30\_40 \textcolor{red}{\textcjheb{ml+swry}} JRWSLM $|$Jerusalem(s)\\
5.&55.&382.&240.&1575.&21.&6&501&2\_90\_2\_1\_6\_400 \textcolor{red}{\textcjheb{tw'b.sb}} B"sBAWT $|$bei den Gazellen\\
6.&56.&383.&246.&1581.&27.&2&7&1\_6 \textcolor{red}{\textcjheb{w'}} AW $|$oder\\
7.&57.&384.&248.&1583.&29.&6&449&2\_1\_10\_30\_6\_400 \textcolor{red}{\textcjheb{twly'b}} BAJLWT $|$bei den Hindinnen\\
8.&58.&385.&254.&1589.&35.&4&314&5\_300\_4\_5 \textcolor{red}{\textcjheb{hd+sh}} HSDH $|$des Feldes\\
9.&59.&386.&258.&1593.&39.&2&41&1\_40 \textcolor{red}{\textcjheb{m'}} AM $|$dass nicht\\
10.&60.&387.&260.&1595.&41.&5&686&400\_70\_10\_200\_6 \textcolor{red}{\textcjheb{wry`t}} TaJRW $|$ihr wecket\\
11.&61.&388.&265.&1600.&46.&3&47&6\_1\_40 \textcolor{red}{\textcjheb{m'w}} WAM $|$noch/und nicht\\
12.&62.&389.&268.&1603.&49.&6&882&400\_70\_6\_200\_200\_6 \textcolor{red}{\textcjheb{wrrw`t}} TaWRRW $|$aufwecket/ihr aufst"ort\\
13.&63.&390.&274.&1609.&55.&2&401&1\_400 \textcolor{red}{\textcjheb{t'}} AT $|$**\\
14.&64.&391.&276.&1611.&57.&5&18&5\_1\_5\_2\_5 \textcolor{red}{\textcjheb{hbh'h}} HAHBH $|$die Liebe\\
15.&65.&392.&281.&1616.&62.&2&74&70\_4 \textcolor{red}{\textcjheb{d`}} aD $|$bis\\
16.&66.&393.&283.&1618.&64.&5&878&300\_400\_8\_80\_90 \textcolor{red}{\textcjheb{.sp.ht+s}} STCP"s $|$(dass) es ihr gef"allt\\
\end{tabular}\medskip \\
Ende des Verses 3.5\\
Verse: 39, Buchstaben: 68, 287, 1622, Totalwerte: 6590, 24772, 112694\\
\\
Ich beschw"ore euch, T"ochter Jerusalems, bei den Gazellen oder bei den Hindinnen des Feldes, da"s ihr nicht wecket noch aufwecket die Liebe, bis es ihr gef"allt!\\
\newpage 
{\bf -- 3.6}\\
\medskip \\
\begin{tabular}{rrrrrrrrp{120mm}}
WV&WK&WB&ABK&ABB&ABV&AnzB&TW&Zahlencode \textcolor{red}{$\boldsymbol{Grundtext}$} Umschrift $|$"Ubersetzung(en)\\
1.&67.&394.&288.&1623.&1.&2&50&40\_10 \textcolor{red}{\textcjheb{ym}} MJ $|$wer (ist)\\
2.&68.&395.&290.&1625.&3.&3&408&7\_1\_400 \textcolor{red}{\textcjheb{t'z}} ZAT $|$die(se)\\
3.&69.&396.&293.&1628.&6.&3&105&70\_30\_5 \textcolor{red}{\textcjheb{hl`}} aLH $|$die da heraufkommt/Heraufkommende\\
4.&70.&397.&296.&1631.&9.&2&90&40\_50 \textcolor{red}{\textcjheb{nm}} MN $|$von/aus\\
5.&71.&398.&298.&1633.&11.&5&251&5\_40\_4\_2\_200 \textcolor{red}{\textcjheb{rbdmh}} HMDBR $|$der W"uste her/der Steppe\\
6.&72.&399.&303.&1638.&16.&7&1076&20\_400\_10\_40\_200\_6\_400 \textcolor{red}{\textcjheb{twrmytk}} KTJMRWT $|$wie S"aulen\\
7.&73.&400.&310.&1645.&23.&3&420&70\_300\_50 \textcolor{red}{\textcjheb{n+s`}} aSN $|$(von) Rauch\\
8.&74.&401.&313.&1648.&26.&5&749&40\_100\_9\_200\_400 \textcolor{red}{\textcjheb{tr.tqm}} MQtRT $|$durchduftet/durchw"urzt\\
9.&75.&402.&318.&1653.&31.&3&246&40\_6\_200 \textcolor{red}{\textcjheb{rwm}} MWR $|$von Myrrhe/(mit) Myrrhe\\
10.&76.&403.&321.&1656.&34.&6&99&6\_30\_2\_6\_50\_5 \textcolor{red}{\textcjheb{hnwblw}} WLBWNH $|$und Weihrauch\\
11.&77.&404.&327.&1662.&40.&3&90&40\_20\_30 \textcolor{red}{\textcjheb{lkm}} MKL $|$von allerlei/mit allerart\\
12.&78.&405.&330.&1665.&43.&4&503&1\_2\_100\_400 \textcolor{red}{\textcjheb{tqb'}} ABQT $|$(Gew"urz)Pulver\\
13.&79.&406.&334.&1669.&47.&4&256&200\_6\_20\_30 \textcolor{red}{\textcjheb{lkwr}} RWKL $|$des Kr"amers/(eines) Kr"amers\\
\end{tabular}\medskip \\
Ende des Verses 3.6\\
Verse: 40, Buchstaben: 50, 337, 1672, Totalwerte: 4343, 29115, 117037\\
\\
Wer ist die, die da heraufkommt von der W"uste her wie Rauchs"aulen, durchduftet von Myrrhe und Weihrauch, von allerlei Gew"urzpulver des Kr"amers?\\
\newpage 
{\bf -- 3.7}\\
\medskip \\
\begin{tabular}{rrrrrrrrp{120mm}}
WV&WK&WB&ABK&ABB&ABV&AnzB&TW&Zahlencode \textcolor{red}{$\boldsymbol{Grundtext}$} Umschrift $|$"Ubersetzung(en)\\
1.&80.&407.&338.&1673.&1.&3&60&5\_50\_5 \textcolor{red}{\textcjheb{hnh}} HNH $|$siehe (da)\\
2.&81.&408.&341.&1676.&4.&4&455&40\_9\_400\_6 \textcolor{red}{\textcjheb{wt.tm}} MtTW $|$(das) Tragbett/sein Bett\\
3.&82.&409.&345.&1680.&8.&6&705&300\_30\_300\_30\_40\_5 \textcolor{red}{\textcjheb{hml+sl+s}} SLSLMH $|$Salomos/welche ist das Salomos\\
4.&83.&410.&351.&1686.&14.&4&650&300\_300\_10\_40 \textcolor{red}{\textcjheb{my+s+s}} SSJM $|$sechzig\\
5.&84.&411.&355.&1690.&18.&5&255&3\_2\_200\_10\_40 \textcolor{red}{\textcjheb{myrbg}} GBRJM $|$Helden/(tapfere) Krieger\\
6.&85.&412.&360.&1695.&23.&4&74&60\_2\_10\_2 \textcolor{red}{\textcjheb{bybs}} sBJB $|$(sind) rings um\\
7.&86.&413.&364.&1699.&27.&2&35&30\_5 \textcolor{red}{\textcjheb{hl}} LH $|$dasselbe her/um sie\\
8.&87.&414.&366.&1701.&29.&5&255&40\_3\_2\_200\_10 \textcolor{red}{\textcjheb{yrbgm}} MGBRJ $|$von den Helden\\
9.&88.&415.&371.&1706.&34.&5&541&10\_300\_200\_1\_30 \textcolor{red}{\textcjheb{l'r+sy}} JSRAL $|$Israel(s)///$<$F"urst Gottes$>$\\
\end{tabular}\medskip \\
Ende des Verses 3.7\\
Verse: 41, Buchstaben: 38, 375, 1710, Totalwerte: 3030, 32145, 120067\\
\\
Siehe da, Salomos Tragbett: Sechzig Helden rings um dasselbe her von den Helden Israels.\\
\newpage 
{\bf -- 3.8}\\
\medskip \\
\begin{tabular}{rrrrrrrrp{120mm}}
WV&WK&WB&ABK&ABB&ABV&AnzB&TW&Zahlencode \textcolor{red}{$\boldsymbol{Grundtext}$} Umschrift $|$"Ubersetzung(en)\\
1.&89.&416.&376.&1711.&1.&3&90&20\_30\_40 \textcolor{red}{\textcjheb{mlk}} KLM $|$alle sie\\
2.&90.&417.&379.&1714.&4.&4&26&1\_8\_7\_10 \textcolor{red}{\textcjheb{yz.h'}} ACZJ $|$f"uhren das/bewaffnet mit\\
3.&91.&418.&383.&1718.&8.&3&210&8\_200\_2 \textcolor{red}{\textcjheb{br.h}} CRB $|$Schwert\\
4.&92.&419.&386.&1721.&11.&5&124&40\_30\_40\_4\_10 \textcolor{red}{\textcjheb{ydmlm}} MLMDJ $|$(sind) ge"ubt\\
5.&93.&420.&391.&1726.&16.&5&123&40\_30\_8\_40\_5 \textcolor{red}{\textcjheb{hm.hlm}} MLCMH $|$im Kriege/(f"ur den) Kampf\\
6.&94.&421.&396.&1731.&21.&3&311&1\_10\_300 \textcolor{red}{\textcjheb{+sy'}} AJS $|$(ein) jeder\\
7.&95.&422.&399.&1734.&24.&4&216&8\_200\_2\_6 \textcolor{red}{\textcjheb{wbr.h}} CRBW $|$hat sein Schwert/(tr"agt) sein Schwert\\
8.&96.&423.&403.&1738.&28.&2&100&70\_30 \textcolor{red}{\textcjheb{l`}} aL $|$an\\
9.&97.&424.&405.&1740.&30.&4&236&10\_200\_20\_6 \textcolor{red}{\textcjheb{wkry}} JRKW $|$seiner H"ufte\\
10.&98.&425.&409.&1744.&34.&4&132&40\_80\_8\_4 \textcolor{red}{\textcjheb{d.hpm}} MPCD $|$zum Schutz vor dem Schrecken/gegen Schrecken\\
11.&99.&426.&413.&1748.&38.&6&478&2\_30\_10\_30\_6\_400 \textcolor{red}{\textcjheb{twlylb}} BLJLWT $|$in den N"achten\\
\end{tabular}\medskip \\
Ende des Verses 3.8\\
Verse: 42, Buchstaben: 43, 418, 1753, Totalwerte: 2046, 34191, 122113\\
\\
Sie alle f"uhren das Schwert, sind ge"ubt im Kriege; ein jeder hat sein Schwert an seiner H"ufte, zum Schutz vor dem Schrecken in den N"achten. -\\
\newpage 
{\bf -- 3.9}\\
\medskip \\
\begin{tabular}{rrrrrrrrp{120mm}}
WV&WK&WB&ABK&ABB&ABV&AnzB&TW&Zahlencode \textcolor{red}{$\boldsymbol{Grundtext}$} Umschrift $|$"Ubersetzung(en)\\
1.&100.&427.&419.&1754.&1.&6&347&1\_80\_200\_10\_6\_50 \textcolor{red}{\textcjheb{nwyrp'}} APRJWN $|$ein Prachtbett/(einen) Tragsessel\\
2.&101.&428.&425.&1760.&7.&3&375&70\_300\_5 \textcolor{red}{\textcjheb{h+s`}} aSH $|$hat gemacht/er (=es) lie"s machen\\
3.&102.&429.&428.&1763.&10.&2&36&30\_6 \textcolor{red}{\textcjheb{wl}} LW $|$sich\\
4.&103.&430.&430.&1765.&12.&4&95&5\_40\_30\_20 \textcolor{red}{\textcjheb{klmh}} HMLK $|$der K"onig\\
5.&104.&431.&434.&1769.&16.&4&375&300\_30\_40\_5 \textcolor{red}{\textcjheb{hml+s}} SLMH $|$Salomo/Schelomo\\
6.&105.&432.&438.&1773.&20.&4&210&40\_70\_90\_10 \textcolor{red}{\textcjheb{y.s`m}} Ma"sJ $|$von dem Holz/aus H"olzern\\
7.&106.&433.&442.&1777.&24.&6&143&5\_30\_2\_50\_6\_50 \textcolor{red}{\textcjheb{nwnblh}} HLBNWN $|$des Libanon///$<$der Wei"se$>$\\
\end{tabular}\medskip \\
Ende des Verses 3.9\\
Verse: 43, Buchstaben: 29, 447, 1782, Totalwerte: 1581, 35772, 123694\\
\\
Der K"onig Salomo hat sich ein Prachtbett gemacht von dem Holze des Libanon.\\
\newpage 
{\bf -- 3.10}\\
\medskip \\
\begin{tabular}{rrrrrrrrp{120mm}}
WV&WK&WB&ABK&ABB&ABV&AnzB&TW&Zahlencode \textcolor{red}{$\boldsymbol{Grundtext}$} Umschrift $|$"Ubersetzung(en)\\
1.&107.&434.&448.&1783.&1.&6&136&70\_40\_6\_4\_10\_6 \textcolor{red}{\textcjheb{wydwm`}} aMWDJW $|$seine S"aulen\\
2.&108.&435.&454.&1789.&7.&3&375&70\_300\_5 \textcolor{red}{\textcjheb{h+s`}} aSH $|$hat er gemacht/er lie"s machen\\
3.&109.&436.&457.&1792.&10.&3&160&20\_60\_80 \textcolor{red}{\textcjheb{psk}} KsP $|$von Silber/(aus) Silber\\
4.&110.&437.&460.&1795.&13.&6&700&200\_80\_10\_4\_400\_6 \textcolor{red}{\textcjheb{wtdypr}} RPJDTW $|$seine Lehne\\
5.&111.&438.&466.&1801.&19.&3&14&7\_5\_2 \textcolor{red}{\textcjheb{bhz}} ZHB $|$von Gold/(aus) Gold\\
6.&112.&439.&469.&1804.&22.&5&268&40\_200\_20\_2\_6 \textcolor{red}{\textcjheb{wbkrm}} MRKBW $|$seinen (Kissen)Sitz\\
7.&113.&440.&474.&1809.&27.&5&294&1\_200\_3\_40\_50 \textcolor{red}{\textcjheb{nmgr'}} ARGMN $|$von Purpur\\
8.&114.&441.&479.&1814.&32.&4&432&400\_6\_20\_6 \textcolor{red}{\textcjheb{wkwt}} TWKW $|$das Innere/sein Inneres\\
9.&115.&442.&483.&1818.&36.&4&376&200\_90\_6\_80 \textcolor{red}{\textcjheb{pw.sr}} R"sWP $|$ist kunstvoll gestickt/gepolstert\\
10.&116.&443.&487.&1822.&40.&4&13&1\_5\_2\_5 \textcolor{red}{\textcjheb{hbh'}} AHBH $|$aus Liebe/(mit) Liebe\\
11.&117.&444.&491.&1826.&44.&5&498&40\_2\_50\_6\_400 \textcolor{red}{\textcjheb{twnbm}} MBNWT $|$von den T"ochtern\\
12.&118.&445.&496.&1831.&49.&6&586&10\_200\_6\_300\_30\_40 \textcolor{red}{\textcjheb{ml+swry}} JRWSLM $|$Jerusalem(s)\\
\end{tabular}\medskip \\
Ende des Verses 3.10\\
Verse: 44, Buchstaben: 54, 501, 1836, Totalwerte: 3852, 39624, 127546\\
\\
Seine S"aulen hat er von Silber gemacht, seine Lehne von Gold, seinen Sitz von Purpur; das Innere ist kunstvoll gestickt, aus Liebe, von den T"ochtern Jerusalems.\\
\newpage 
{\bf -- 3.11}\\
\medskip \\
\begin{tabular}{rrrrrrrrp{120mm}}
WV&WK&WB&ABK&ABB&ABV&AnzB&TW&Zahlencode \textcolor{red}{$\boldsymbol{Grundtext}$} Umschrift $|$"Ubersetzung(en)\\
1.&119.&446.&502.&1837.&1.&5&156&90\_1\_10\_50\_5 \textcolor{red}{\textcjheb{hny'.s}} "sAJNH $|$kommet heraus/zieht hinaus\\
2.&120.&447.&507.&1842.&6.&6&272&6\_200\_1\_10\_50\_5 \textcolor{red}{\textcjheb{hny'rw}} WRAJNH $|$und betrachtet/und schaut\\
3.&121.&448.&513.&1848.&12.&4&458&2\_50\_6\_400 \textcolor{red}{\textcjheb{twnb}} BNWT $|$(ihr) T"ochter\\
4.&122.&449.&517.&1852.&16.&4&156&90\_10\_6\_50 \textcolor{red}{\textcjheb{nwy.s}} "sJWN $|$Zion(s)///$<$Burg$>$\\
5.&123.&450.&521.&1856.&20.&4&92&2\_40\_30\_20 \textcolor{red}{\textcjheb{klmb}} BMLK $|$den K"onig\\
6.&124.&451.&525.&1860.&24.&4&375&300\_30\_40\_5 \textcolor{red}{\textcjheb{hml+s}} SLMH $|$Salomo/Schlomo\\
7.&125.&452.&529.&1864.&28.&5&286&2\_70\_9\_200\_5 \textcolor{red}{\textcjheb{hr.t`b}} BatRH $|$in der Krone/mit Krone\\
8.&126.&453.&534.&1869.&33.&5&584&300\_70\_9\_200\_5 \textcolor{red}{\textcjheb{hr.t`+s}} SatRH $|$mit welcher gekr"ont hat/mit welcher sie (=es) bekr"anzte\\
9.&127.&454.&539.&1874.&38.&2&36&30\_6 \textcolor{red}{\textcjheb{wl}} LW $|$ihn\\
10.&128.&455.&541.&1876.&40.&3&47&1\_40\_6 \textcolor{red}{\textcjheb{wm'}} AMW $|$seine Mutter\\
11.&129.&456.&544.&1879.&43.&4&58&2\_10\_6\_40 \textcolor{red}{\textcjheb{mwyb}} BJWM $|$am Tag\\
12.&130.&457.&548.&1883.&47.&5&864&8\_400\_50\_400\_6 \textcolor{red}{\textcjheb{wtnt.h}} CTNTW $|$seiner Verm"ahlung\\
13.&131.&458.&553.&1888.&52.&5&64&6\_2\_10\_6\_40 \textcolor{red}{\textcjheb{mwybw}} WBJWM $|$und am Tag\\
14.&132.&459.&558.&1893.&57.&4&748&300\_40\_8\_400 \textcolor{red}{\textcjheb{t.hm+s}} SMCT $|$(der) Freude\\
15.&133.&460.&562.&1897.&61.&3&38&30\_2\_6 \textcolor{red}{\textcjheb{wbl}} LBW $|$seines Herzens\\
\end{tabular}\medskip \\
Ende des Verses 3.11\\
Verse: 45, Buchstaben: 63, 564, 1899, Totalwerte: 4234, 43858, 131780\\
\\
Kommet heraus, T"ochter Zions, und betrachtet den K"onig Salomo in der Krone, mit welcher seine Mutter ihn gekr"ont hat am Tage seiner Verm"ahlung und am Tage der Freude seines Herzens!\\
\\
{\bf Ende des Kapitels 3}\\
\newpage 
{\bf -- 4.1}\\
\medskip \\
\begin{tabular}{rrrrrrrrp{120mm}}
WV&WK&WB&ABK&ABB&ABV&AnzB&TW&Zahlencode \textcolor{red}{$\boldsymbol{Grundtext}$} Umschrift $|$"Ubersetzung(en)\\
1.&1.&461.&1.&1900.&1.&3&75&5\_50\_20 \textcolor{red}{\textcjheb{knh}} HNK $|$siehe du (bist)\\
2.&2.&462.&4.&1903.&4.&3&95&10\_80\_5 \textcolor{red}{\textcjheb{hpy}} JPH $|$sch"on(e)\\
3.&3.&463.&7.&1906.&7.&5&690&200\_70\_10\_400\_10 \textcolor{red}{\textcjheb{yty`r}} RaJTJ $|$meine Freundin\\
4.&4.&464.&12.&1911.&12.&3&75&5\_50\_20 \textcolor{red}{\textcjheb{knh}} HNK $|$siehe du (bist)\\
5.&5.&465.&15.&1914.&15.&3&95&10\_80\_5 \textcolor{red}{\textcjheb{hpy}} JPH $|$sch"on(e)\\
6.&6.&466.&18.&1917.&18.&5&160&70\_10\_50\_10\_20 \textcolor{red}{\textcjheb{kyny`}} aJNJK $|$deine Augen\\
7.&7.&467.&23.&1922.&23.&5&116&10\_6\_50\_10\_40 \textcolor{red}{\textcjheb{mynwy}} JWNJM $|$sind Tauben/gleichen Tauben\\
8.&8.&468.&28.&1927.&28.&4&116&40\_2\_70\_4 \textcolor{red}{\textcjheb{d`bm}} MBaD $|$hinter/durch\\
9.&9.&469.&32.&1931.&32.&5&580&30\_90\_40\_400\_20 \textcolor{red}{\textcjheb{ktm.sl}} L"sMTK $|$deinem Schleier/deinen Schleier\\
10.&10.&470.&37.&1936.&37.&4&590&300\_70\_200\_20 \textcolor{red}{\textcjheb{kr`+s}} SaRK $|$dein Haar\\
11.&11.&471.&41.&1940.&41.&4&294&20\_70\_4\_200 \textcolor{red}{\textcjheb{rd`k}} KaDR $|$(ist) wie eine Herde\\
12.&12.&472.&45.&1944.&45.&5&132&5\_70\_7\_10\_40 \textcolor{red}{\textcjheb{myz`h}} HaZJM $|$(der) Ziegen\\
13.&13.&473.&50.&1949.&50.&5&639&300\_3\_30\_300\_6 \textcolor{red}{\textcjheb{w+slg+s}} SGLSW $|$die lagern an den Abh"angen/welche (sie) wallten herab\\
14.&14.&474.&55.&1954.&55.&3&245&40\_5\_200 \textcolor{red}{\textcjheb{rhm}} MHR $|$des Gebirges/vom Berg\\
15.&15.&475.&58.&1957.&58.&4&107&3\_30\_70\_4 \textcolor{red}{\textcjheb{d`lg}} GLaD $|$Gilead///$<$H"ugel$>$\\
\end{tabular}\medskip \\
Ende des Verses 4.1\\
Verse: 46, Buchstaben: 61, 61, 1960, Totalwerte: 4009, 4009, 135789\\
\\
Siehe, du bist sch"on, meine Freundin, siehe, du bist sch"on: Deine Augen sind Tauben hinter deinem Schleier. Dein Haar ist wie eine Herde Ziegen, die an den Abh"angen des Gebirges Gilead lagern.\\
\newpage 
{\bf -- 4.2}\\
\medskip \\
\begin{tabular}{rrrrrrrrp{120mm}}
WV&WK&WB&ABK&ABB&ABV&AnzB&TW&Zahlencode \textcolor{red}{$\boldsymbol{Grundtext}$} Umschrift $|$"Ubersetzung(en)\\
1.&16.&476.&62.&1961.&1.&4&380&300\_50\_10\_20 \textcolor{red}{\textcjheb{kyn+s}} SNJK $|$deine Z"ahne\\
2.&17.&477.&66.&1965.&5.&4&294&20\_70\_4\_200 \textcolor{red}{\textcjheb{rd`k}} KaDR $|$(sind) wie eine Herde\\
3.&18.&478.&70.&1969.&9.&7&609&5\_100\_90\_6\_2\_6\_400 \textcolor{red}{\textcjheb{twbw.sqh}} HQ"sWBWT $|$geschorener Schafe /der (frisch) Geschorenen\\
4.&19.&479.&77.&1976.&16.&4&406&300\_70\_30\_6 \textcolor{red}{\textcjheb{wl`+s}} SaLW $|$die heraufkommen/welche (sie) stiegen herauf\\
5.&20.&480.&81.&1980.&20.&2&90&40\_50 \textcolor{red}{\textcjheb{nm}} MN $|$aus\\
6.&21.&481.&83.&1982.&22.&5&308&5\_200\_8\_90\_5 \textcolor{red}{\textcjheb{h.s.hrh}} HRC"sH $|$der Schwemme\\
7.&22.&482.&88.&1987.&27.&4&390&300\_20\_30\_40 \textcolor{red}{\textcjheb{mlk+s}} SKLM $|$welche allzumal/welche all sie\\
8.&23.&483.&92.&1991.&31.&7&897&40\_400\_1\_10\_40\_6\_400 \textcolor{red}{\textcjheb{twmy'tm}} MTAJMWT $|$Zwillinge geb"aren/sind zwillingstr"achtig\\
9.&24.&484.&99.&1998.&38.&5&361&6\_300\_20\_30\_5 \textcolor{red}{\textcjheb{hlk+sw}} WSKLH $|$und unfruchtbar/und eine ohne Junge\\
10.&25.&485.&104.&2003.&43.&3&61&1\_10\_50 \textcolor{red}{\textcjheb{ny'}} AJN $|$nicht ist (eines)\\
11.&26.&486.&107.&2006.&46.&3&47&2\_5\_40 \textcolor{red}{\textcjheb{mhb}} BHM $|$unter ihnen\\
\end{tabular}\medskip \\
Ende des Verses 4.2\\
Verse: 47, Buchstaben: 48, 109, 2008, Totalwerte: 3843, 7852, 139632\\
\\
Deine Z"ahne sind wie eine Herde geschorener Schafe, die aus der Schwemme heraufkommen, welche allzumal Zwillinge geb"aren, und keines unter ihnen ist unfruchtbar.\\
\newpage 
{\bf -- 4.3}\\
\medskip \\
\begin{tabular}{rrrrrrrrp{120mm}}
WV&WK&WB&ABK&ABB&ABV&AnzB&TW&Zahlencode \textcolor{red}{$\boldsymbol{Grundtext}$} Umschrift $|$"Ubersetzung(en)\\
1.&27.&487.&110.&2009.&1.&4&43&20\_8\_6\_9 \textcolor{red}{\textcjheb{.tw.hk}} KCWt $|$wie eine Schur/wie (ein) Faden\\
2.&28.&488.&114.&2013.&5.&4&365&5\_300\_50\_10 \textcolor{red}{\textcjheb{yn+sh}} HSNJ $|$(von) Karmesin\\
3.&29.&489.&118.&2017.&9.&6&1210&300\_80\_400\_400\_10\_20 \textcolor{red}{\textcjheb{kyttp+s}} SPTTJK $|$(sind) deine Lippen\\
4.&30.&490.&124.&2023.&15.&7&282&6\_40\_4\_2\_200\_10\_20 \textcolor{red}{\textcjheb{kyrbdmw}} WMDBRJK $|$und dein Mund/und dein M"undchen\\
5.&31.&491.&131.&2030.&22.&4&62&50\_1\_6\_5 \textcolor{red}{\textcjheb{hw'n}} NAWH $|$ist zierlich/(ist) lieblich\\
6.&32.&492.&135.&2034.&26.&4&138&20\_80\_30\_8 \textcolor{red}{\textcjheb{.hlpk}} KPLC $|$wie ein Schnittst"uck/wie eine Scheibe\\
7.&33.&493.&139.&2038.&30.&5&301&5\_200\_40\_6\_50 \textcolor{red}{\textcjheb{nwmrh}} HRMWN $|$einer Granate/des Granatapfels\\
8.&34.&494.&144.&2043.&35.&4&720&200\_100\_400\_20 \textcolor{red}{\textcjheb{ktqr}} RQTK $|$ist deine Schl"afe/(sieht aus) deine Schl"afe\\
9.&35.&495.&148.&2047.&39.&4&116&40\_2\_70\_4 \textcolor{red}{\textcjheb{d`bm}} MBaD $|$hinter/durch\\
10.&36.&496.&152.&2051.&43.&5&580&30\_90\_40\_400\_20 \textcolor{red}{\textcjheb{ktm.sl}} L"sMTK $|$deinem Schleier/deinen Schleier\\
\end{tabular}\medskip \\
Ende des Verses 4.3\\
Verse: 48, Buchstaben: 47, 156, 2055, Totalwerte: 3817, 11669, 143449\\
\\
Deine Lippen sind wie eine Karmesinschnur, und dein Mund ist zierlich. Wie ein Schnittst"uck einer Granate ist deine Schl"afe hinter deinem Schleier.\\
\newpage 
{\bf -- 4.4}\\
\medskip \\
\begin{tabular}{rrrrrrrrp{120mm}}
WV&WK&WB&ABK&ABB&ABV&AnzB&TW&Zahlencode \textcolor{red}{$\boldsymbol{Grundtext}$} Umschrift $|$"Ubersetzung(en)\\
1.&37.&497.&157.&2056.&1.&5&97&20\_40\_3\_4\_30 \textcolor{red}{\textcjheb{ldgmk}} KMGDL $|$wie der Turm\\
2.&38.&498.&162.&2061.&6.&4&24&4\_6\_10\_4 \textcolor{red}{\textcjheb{dywd}} DWJD $|$David(s)///$<$Geliebter$>$\\
3.&39.&499.&166.&2065.&10.&5&317&90\_6\_1\_200\_20 \textcolor{red}{\textcjheb{kr'w.s}} "sWARK $|$(ist) dein Hals\\
4.&40.&500.&171.&2070.&15.&4&68&2\_50\_6\_10 \textcolor{red}{\textcjheb{ywnb}} BNWJ $|$(der) gebaut(er) (ist)\\
5.&41.&501.&175.&2074.&19.&7&956&30\_400\_30\_80\_10\_6\_400 \textcolor{red}{\textcjheb{twypltl}} LTLPJWT $|$in Terassen/als Zeughaus\\
6.&42.&502.&182.&2081.&26.&3&111&1\_30\_80 \textcolor{red}{\textcjheb{pl'}} ALP $|$tausend/eine Tausend(zahl)\\
7.&43.&503.&185.&2084.&29.&4&98&5\_40\_3\_50 \textcolor{red}{\textcjheb{ngmh}} HMGN $|$(der) Schilde\\
8.&44.&504.&189.&2088.&33.&4&446&400\_30\_6\_10 \textcolor{red}{\textcjheb{ywlt}} TLWJ $|$h"angen/(ist) aufgeh"angt(er)\\
9.&45.&505.&193.&2092.&37.&4&116&70\_30\_10\_6 \textcolor{red}{\textcjheb{wyl`}} aLJW $|$daran/an ihm\\
10.&46.&506.&197.&2096.&41.&2&50&20\_30 \textcolor{red}{\textcjheb{lk}} KL $|$alle\\
11.&47.&507.&199.&2098.&43.&4&349&300\_30\_9\_10 \textcolor{red}{\textcjheb{y.tl+s}} SLtJ $|$Schilde/K"ocher\\
12.&48.&508.&203.&2102.&47.&7&266&5\_3\_2\_6\_200\_10\_40 \textcolor{red}{\textcjheb{myrwbgh}} HGBWRJM $|$der Helden\\
\end{tabular}\medskip \\
Ende des Verses 4.4\\
Verse: 49, Buchstaben: 53, 209, 2108, Totalwerte: 2898, 14567, 146347\\
\\
Dein Hals ist wie der Turm Davids, der in Terrassen gebaut ist: tausend Schilde h"angen daran, alle Schilde der Helden.\\
\newpage 
{\bf -- 4.5}\\
\medskip \\
\begin{tabular}{rrrrrrrrp{120mm}}
WV&WK&WB&ABK&ABB&ABV&AnzB&TW&Zahlencode \textcolor{red}{$\boldsymbol{Grundtext}$} Umschrift $|$"Ubersetzung(en)\\
1.&49.&509.&210.&2109.&1.&3&360&300\_50\_10 \textcolor{red}{\textcjheb{yn+s}} SNJ $|$beide(n)\\
2.&50.&510.&213.&2112.&4.&4&334&300\_4\_10\_20 \textcolor{red}{\textcjheb{kyd+s}} SDJK $|$deine Br"uste\\
3.&51.&511.&217.&2116.&8.&4&380&20\_300\_50\_10 \textcolor{red}{\textcjheb{yn+sk}} KSNJ $|$(sind) wie zwei\\
4.&52.&512.&221.&2120.&12.&5&400&70\_80\_200\_10\_40 \textcolor{red}{\textcjheb{myrp`}} aPRJM $|$Rehe\\
5.&53.&513.&226.&2125.&17.&5&457&400\_1\_6\_40\_10 \textcolor{red}{\textcjheb{ymw't}} TAWMJ $|$wie ein Zwillingspaar/wie Zwillinge\\
6.&54.&514.&231.&2130.&22.&4&107&90\_2\_10\_5 \textcolor{red}{\textcjheb{hyb.s}} "sBJH $|$junger Gazellen/der Hindin\\
7.&55.&515.&235.&2134.&26.&6&331&5\_200\_6\_70\_10\_40 \textcolor{red}{\textcjheb{my`wrh}} HRWaJM $|$die weiden(d(e) (sind))\\
8.&56.&516.&241.&2140.&32.&7&708&2\_300\_6\_300\_50\_10\_40 \textcolor{red}{\textcjheb{myn+sw+sb}} BSWSNJM $|$unter den Lilien\\
\end{tabular}\medskip \\
Ende des Verses 4.5\\
Verse: 50, Buchstaben: 38, 247, 2146, Totalwerte: 3077, 17644, 149424\\
\\
Deine beiden Br"uste sind wie ein Zwillingspaar junger Gazellen, die unter den Lilien weiden. -\\
\newpage 
{\bf -- 4.6}\\
\medskip \\
\begin{tabular}{rrrrrrrrp{120mm}}
WV&WK&WB&ABK&ABB&ABV&AnzB&TW&Zahlencode \textcolor{red}{$\boldsymbol{Grundtext}$} Umschrift $|$"Ubersetzung(en)\\
1.&57.&517.&248.&2147.&1.&2&74&70\_4 \textcolor{red}{\textcjheb{d`}} aD $|$bis\\
2.&58.&518.&250.&2149.&3.&5&404&300\_10\_80\_6\_8 \textcolor{red}{\textcjheb{.hwpy+s}} SJPWC $|$sich k"uhlt/dass (k"uhl) weht\\
3.&59.&519.&255.&2154.&8.&4&61&5\_10\_6\_40 \textcolor{red}{\textcjheb{mwyh}} HJWM $|$der Tag/der Tag(eswind)\\
4.&60.&520.&259.&2158.&12.&4&122&6\_50\_60\_6 \textcolor{red}{\textcjheb{wsnw}} WNsW $|$und (es) fliehen/und sie flohen\\
5.&61.&521.&263.&2162.&16.&6&205&5\_90\_30\_30\_10\_40 \textcolor{red}{\textcjheb{myll.sh}} H"sLLJM $|$die Schatten\\
6.&62.&522.&269.&2168.&22.&3&51&1\_30\_20 \textcolor{red}{\textcjheb{kl'}} ALK $|$ich will (hin)gehen\\
7.&63.&523.&272.&2171.&25.&2&40&30\_10 \textcolor{red}{\textcjheb{yl}} LJ $|$/f"ur mich\\
8.&64.&524.&274.&2173.&27.&2&31&1\_30 \textcolor{red}{\textcjheb{l'}} AL $|$zum\\
9.&65.&525.&276.&2175.&29.&2&205&5\_200 \textcolor{red}{\textcjheb{rh}} HR $|$Berg\\
10.&66.&526.&278.&2177.&31.&4&251&5\_40\_6\_200 \textcolor{red}{\textcjheb{rwmh}} HMWR $|$(der) Myrrhe(n)\\
11.&67.&527.&282.&2181.&35.&3&37&6\_1\_30 \textcolor{red}{\textcjheb{l'w}} WAL $|$und zum\\
12.&68.&528.&285.&2184.&38.&4&475&3\_2\_70\_400 \textcolor{red}{\textcjheb{t`bg}} GBaT $|$H"ugel\\
13.&69.&529.&289.&2188.&42.&6&98&5\_30\_2\_6\_50\_5 \textcolor{red}{\textcjheb{hnwblh}} HLBWNH $|$(des) Weihrauch(s)\\
\end{tabular}\medskip \\
Ende des Verses 4.6\\
Verse: 51, Buchstaben: 47, 294, 2193, Totalwerte: 2054, 19698, 151478\\
\\
Bis der Tag sich k"uhlt und die Schatten fliehen, will ich zum Myrrhenberge hingehen und zum Weihrauchh"ugel. -\\
\newpage 
{\bf -- 4.7}\\
\medskip \\
\begin{tabular}{rrrrrrrrp{120mm}}
WV&WK&WB&ABK&ABB&ABV&AnzB&TW&Zahlencode \textcolor{red}{$\boldsymbol{Grundtext}$} Umschrift $|$"Ubersetzung(en)\\
1.&70.&530.&295.&2194.&1.&3&70&20\_30\_20 \textcolor{red}{\textcjheb{klk}} KLK $|$ganz bist du\\
2.&71.&531.&298.&2197.&4.&3&95&10\_80\_5 \textcolor{red}{\textcjheb{hpy}} JPH $|$sch"on(e)\\
3.&72.&532.&301.&2200.&7.&5&690&200\_70\_10\_400\_10 \textcolor{red}{\textcjheb{yty`r}} RaJTJ $|$meine Freundin\\
4.&73.&533.&306.&2205.&12.&4&92&6\_40\_6\_40 \textcolor{red}{\textcjheb{mwmw}} WMWM $|$und ein Makel\\
5.&74.&534.&310.&2209.&16.&3&61&1\_10\_50 \textcolor{red}{\textcjheb{ny'}} AJN $|$nicht ist\\
6.&75.&535.&313.&2212.&19.&2&22&2\_20 \textcolor{red}{\textcjheb{kb}} BK $|$an dir\\
\end{tabular}\medskip \\
Ende des Verses 4.7\\
Verse: 52, Buchstaben: 20, 314, 2213, Totalwerte: 1030, 20728, 152508\\
\\
Ganz sch"on bist du, meine Freundin, und kein Makel ist an dir.\\
\newpage 
{\bf -- 4.8}\\
\medskip \\
\begin{tabular}{rrrrrrrrp{120mm}}
WV&WK&WB&ABK&ABB&ABV&AnzB&TW&Zahlencode \textcolor{red}{$\boldsymbol{Grundtext}$} Umschrift $|$"Ubersetzung(en)\\
1.&76.&536.&315.&2214.&1.&3&411&1\_400\_10 \textcolor{red}{\textcjheb{yt'}} ATJ $|$mit mir (herab von)\\
2.&77.&537.&318.&2217.&4.&6&178&40\_30\_2\_50\_6\_50 \textcolor{red}{\textcjheb{nwnblm}} MLBNWN $|$dem Libanon\\
3.&78.&538.&324.&2223.&10.&3&55&20\_30\_5 \textcolor{red}{\textcjheb{hlk}} KLH $|$meine Braut/(o) Braut\\
4.&79.&539.&327.&2226.&13.&3&411&1\_400\_10 \textcolor{red}{\textcjheb{yt'}} ATJ $|$mit mir (von)\\
5.&80.&540.&330.&2229.&16.&6&178&40\_30\_2\_50\_6\_50 \textcolor{red}{\textcjheb{nwnblm}} MLBNWN $|$dem Libanon\\
6.&81.&541.&336.&2235.&22.&5&419&400\_2\_6\_1\_10 \textcolor{red}{\textcjheb{y'wbt}} TBWAJ $|$sollst du kommen/komm\\
7.&82.&542.&341.&2240.&27.&5&916&400\_300\_6\_200\_10 \textcolor{red}{\textcjheb{yrw+st}} TSWRJ $|$sollst du schauen herab/schaue aus\\
8.&83.&543.&346.&2245.&32.&4&541&40\_200\_1\_300 \textcolor{red}{\textcjheb{+s'rm}} MRAS $|$vom Gipfel/vom Kopf\\
9.&84.&544.&350.&2249.&36.&4&96&1\_40\_50\_5 \textcolor{red}{\textcjheb{hnm'}} AMNH $|$des Amana///$<$Treue$>$\\
10.&85.&545.&354.&2253.&40.&4&541&40\_200\_1\_300 \textcolor{red}{\textcjheb{+s'rm}} MRAS $|$vom Gipfel/vom Kopf\\
11.&86.&546.&358.&2257.&44.&4&560&300\_50\_10\_200 \textcolor{red}{\textcjheb{ryn+s}} SNJR $|$(des) Senir///$<$Lichtberg$>$\\
12.&87.&547.&362.&2261.&48.&6&310&6\_8\_200\_40\_6\_50 \textcolor{red}{\textcjheb{nwmr.hw}} WCRMWN $|$und (des) Hermon///$<$unzug"anglich$>$\\
13.&88.&548.&368.&2267.&54.&6&606&40\_40\_70\_50\_6\_400 \textcolor{red}{\textcjheb{twn`mm}} MMaNWT $|$von den Lagerst"atten/von den Aufenthaltsorten\\
14.&89.&549.&374.&2273.&60.&5&617&1\_200\_10\_6\_400 \textcolor{red}{\textcjheb{twyr'}} ARJWT $|$der L"owen\\
15.&90.&550.&379.&2278.&65.&5&455&40\_5\_200\_200\_10 \textcolor{red}{\textcjheb{yrrhm}} MHRRJ $|$von den Bergen\\
16.&91.&551.&384.&2283.&70.&5&340&50\_40\_200\_10\_40 \textcolor{red}{\textcjheb{myrmn}} NMRJM $|$der Panther/der Leoparden\\
\end{tabular}\medskip \\
Ende des Verses 4.8\\
Verse: 53, Buchstaben: 74, 388, 2287, Totalwerte: 6634, 27362, 159142\\
\\
Mit mir vom Libanon herab, meine Braut, mit mir vom Libanon sollst du kommen; vom Gipfel des Amana herab sollst du schauen, vom Gipfel des Senir und Hermon, von den Lagerst"atten der L"owen, von den Bergen der Panther.\\
\newpage 
{\bf -- 4.9}\\
\medskip \\
\begin{tabular}{rrrrrrrrp{120mm}}
WV&WK&WB&ABK&ABB&ABV&AnzB&TW&Zahlencode \textcolor{red}{$\boldsymbol{Grundtext}$} Umschrift $|$"Ubersetzung(en)\\
1.&92.&552.&389.&2288.&1.&6&494&30\_2\_2\_400\_50\_10 \textcolor{red}{\textcjheb{yntbbl}} LBBTNJ $|$du hast mir das Herz geraubt/du beraubst mich des Verstandes\\
2.&93.&553.&395.&2294.&7.&4&419&1\_8\_400\_10 \textcolor{red}{\textcjheb{yt.h'}} ACTJ $|$meine Schwester\\
3.&94.&554.&399.&2298.&11.&3&55&20\_30\_5 \textcolor{red}{\textcjheb{hlk}} KLH $|$meine Braut/(o) Braut\\
4.&95.&555.&402.&2301.&14.&7&504&30\_2\_2\_400\_10\_50\_10 \textcolor{red}{\textcjheb{ynytbbl}} LBBTJNJ $|$du hast mir das Herz geraubt/du beraubst mich des Verstandes\\
5.&96.&556.&409.&2308.&21.&4&15&2\_1\_8\_4 \textcolor{red}{\textcjheb{d.h'b}} BACD $|$mit einem (Blick)\\
6.&97.&557.&413.&2312.&25.&6&200&40\_70\_10\_50\_10\_20 \textcolor{red}{\textcjheb{kyny`m}} MaJNJK $|$deiner Blicke/deiner Augen\\
7.&98.&558.&419.&2318.&31.&4&15&2\_1\_8\_4 \textcolor{red}{\textcjheb{d.h'b}} BACD $|$mit einer/mit einem\\
8.&99.&559.&423.&2322.&35.&3&220&70\_50\_100 \textcolor{red}{\textcjheb{qn`}} aNQ $|$Kette/(Hals)Geschmeide\\
9.&100.&560.&426.&2325.&38.&7&416&40\_90\_6\_200\_50\_10\_20 \textcolor{red}{\textcjheb{kynrw.sm}} M"sWRNJK $|$von deinem Halsschmuck/an deinen Halsketten\\
\end{tabular}\medskip \\
Ende des Verses 4.9\\
Verse: 54, Buchstaben: 44, 432, 2331, Totalwerte: 2338, 29700, 161480\\
\\
Du hast mir das Herz geraubt, meine Schwester, meine Braut; du hast mir das Herz geraubt mit einem deiner Blicke, mit einer Kette von deinem Halsschmuck.\\
\newpage 
{\bf -- 4.10}\\
\medskip \\
\begin{tabular}{rrrrrrrrp{120mm}}
WV&WK&WB&ABK&ABB&ABV&AnzB&TW&Zahlencode \textcolor{red}{$\boldsymbol{Grundtext}$} Umschrift $|$"Ubersetzung(en)\\
1.&101.&561.&433.&2332.&1.&2&45&40\_5 \textcolor{red}{\textcjheb{hm}} MH $|$wie\\
2.&102.&562.&435.&2334.&3.&3&96&10\_80\_6 \textcolor{red}{\textcjheb{wpy}} JPW $|$sch"on ist/(sie) sind sch"on\\
3.&103.&563.&438.&2337.&6.&4&38&4\_4\_10\_20 \textcolor{red}{\textcjheb{kydd}} DDJK $|$deine Liebe/deine Liebkosungen\\
4.&104.&564.&442.&2341.&10.&4&419&1\_8\_400\_10 \textcolor{red}{\textcjheb{yt.h'}} ACTJ $|$meine Schwester\\
5.&105.&565.&446.&2345.&14.&3&55&20\_30\_5 \textcolor{red}{\textcjheb{hlk}} KLH $|$meine Braut/(o) Braut\\
6.&106.&566.&449.&2348.&17.&2&45&40\_5 \textcolor{red}{\textcjheb{hm}} MH $|$wie\\
7.&107.&567.&451.&2350.&19.&3&17&9\_2\_6 \textcolor{red}{\textcjheb{wb.t}} tBW $|$viel besser ist/(sie) sind gut (=s"u"s)\\
8.&108.&568.&454.&2353.&22.&4&38&4\_4\_10\_20 \textcolor{red}{\textcjheb{kydd}} DDJK $|$deine Liebe/deine Liebkosungen\\
9.&109.&569.&458.&2357.&26.&4&110&40\_10\_10\_50 \textcolor{red}{\textcjheb{nyym}} MJJN $|$(mehr) als Wein\\
10.&110.&570.&462.&2361.&30.&4&224&6\_200\_10\_8 \textcolor{red}{\textcjheb{.hyrw}} WRJC $|$und der Duft\\
11.&111.&571.&466.&2365.&34.&5&420&300\_40\_50\_10\_20 \textcolor{red}{\textcjheb{kynm+s}} SMNJK $|$deiner Salben/deiner (Salb)"ole\\
12.&112.&572.&471.&2370.&39.&3&90&40\_20\_30 \textcolor{red}{\textcjheb{lkm}} MKL $|$(mehr) als alle\\
13.&113.&573.&474.&2373.&42.&5&392&2\_300\_40\_10\_40 \textcolor{red}{\textcjheb{mym+sb}} BSMJM $|$Gew"urze/Balsam(d"uft)e\\
\end{tabular}\medskip \\
Ende des Verses 4.10\\
Verse: 55, Buchstaben: 46, 478, 2377, Totalwerte: 1989, 31689, 163469\\
\\
Wie sch"on ist deine Liebe, meine Schwester, meine Braut; wieviel besser ist deine Liebe als Wein, und der Duft deiner Salben als alle Gew"urze! Honigseim tr"aufeln deine Lippen, meine Braut;\\
\newpage 
{\bf -- 4.11}\\
\medskip \\
\begin{tabular}{rrrrrrrrp{120mm}}
WV&WK&WB&ABK&ABB&ABV&AnzB&TW&Zahlencode \textcolor{red}{$\boldsymbol{Grundtext}$} Umschrift $|$"Ubersetzung(en)\\
1.&114.&574.&479.&2378.&1.&3&530&50\_80\_400 \textcolor{red}{\textcjheb{tpn}} NPT $|$Honig(seim)\\
2.&115.&575.&482.&2381.&4.&5&544&400\_9\_80\_50\_5 \textcolor{red}{\textcjheb{hnp.tt}} TtPNH $|$(sie) tr"aufeln\\
3.&116.&576.&487.&2386.&9.&7&1216&300\_80\_400\_6\_400\_10\_20 \textcolor{red}{\textcjheb{kytwtp+s}} SPTWTJK $|$deine Lippen\\
4.&117.&577.&494.&2393.&16.&3&55&20\_30\_5 \textcolor{red}{\textcjheb{hlk}} KLH $|$meine Braut/(o) Braut\\
5.&118.&578.&497.&2396.&19.&3&306&4\_2\_300 \textcolor{red}{\textcjheb{+sbd}} DBS $|$Honig\\
6.&119.&579.&500.&2399.&22.&4&46&6\_8\_30\_2 \textcolor{red}{\textcjheb{bl.hw}} WCLB $|$und Milch\\
7.&120.&580.&504.&2403.&26.&3&808&400\_8\_400 \textcolor{red}{\textcjheb{t.ht}} TCT $|$ist unter/(sind) unter\\
8.&121.&581.&507.&2406.&29.&5&406&30\_300\_6\_50\_20 \textcolor{red}{\textcjheb{knw+sl}} LSWNK $|$deiner Zunge\\
9.&122.&582.&512.&2411.&34.&4&224&6\_200\_10\_8 \textcolor{red}{\textcjheb{.hyrw}} WRJC $|$und der Duft\\
10.&123.&583.&516.&2415.&38.&6&800&300\_30\_40\_400\_10\_20 \textcolor{red}{\textcjheb{kytml+s}} SLMTJK $|$deiner Gew"ander/deiner Kleider\\
11.&124.&584.&522.&2421.&44.&4&238&20\_200\_10\_8 \textcolor{red}{\textcjheb{.hyrk}} KRJC $|$wie der Duft/(ist) wie der Geruch\\
12.&125.&585.&526.&2425.&48.&5&138&30\_2\_50\_6\_50 \textcolor{red}{\textcjheb{nwnbl}} LBNWN $|$(des) Libanon\\
\end{tabular}\medskip \\
Ende des Verses 4.11\\
Verse: 56, Buchstaben: 52, 530, 2429, Totalwerte: 5311, 37000, 168780\\
\\
Honig und Milch ist unter deiner Zunge, und der Duft deiner Gew"ander wie der Duft des Libanon.\\
\newpage 
{\bf -- 4.12}\\
\medskip \\
\begin{tabular}{rrrrrrrrp{120mm}}
WV&WK&WB&ABK&ABB&ABV&AnzB&TW&Zahlencode \textcolor{red}{$\boldsymbol{Grundtext}$} Umschrift $|$"Ubersetzung(en)\\
1.&126.&586.&531.&2430.&1.&2&53&3\_50 \textcolor{red}{\textcjheb{ng}} GN $|$(ein) Garten\\
2.&127.&587.&533.&2432.&3.&4&156&50\_70\_6\_30 \textcolor{red}{\textcjheb{lw`n}} NaWL $|$verschlossener/verriegelt(er)\\
3.&128.&588.&537.&2436.&7.&4&419&1\_8\_400\_10 \textcolor{red}{\textcjheb{yt.h'}} ACTJ $|$(ist) meine Schwester\\
4.&129.&589.&541.&2440.&11.&3&55&20\_30\_5 \textcolor{red}{\textcjheb{hlk}} KLH $|$(meine) Braut\\
5.&130.&590.&544.&2443.&14.&2&33&3\_30 \textcolor{red}{\textcjheb{lg}} GL $|$ein Born/(ein) Brunnen(schacht)\\
6.&131.&591.&546.&2445.&16.&4&156&50\_70\_6\_30 \textcolor{red}{\textcjheb{lw`n}} NaWL $|$verschlossener/verriegelt(er)\\
7.&132.&592.&550.&2449.&20.&4&170&40\_70\_10\_50 \textcolor{red}{\textcjheb{ny`m}} MaJN $|$eine Quelle/(ein) Quellort\\
8.&133.&593.&554.&2453.&24.&4&454&8\_400\_6\_40 \textcolor{red}{\textcjheb{mwt.h}} CTWM $|$versiegelt(e)(r)\\
\end{tabular}\medskip \\
Ende des Verses 4.12\\
Verse: 57, Buchstaben: 27, 557, 2456, Totalwerte: 1496, 38496, 170276\\
\\
Ein verschlossener Garten ist meine Schwester, meine Braut, ein verschlossener Born, eine versiegelte Quelle.\\
\newpage 
{\bf -- 4.13}\\
\medskip \\
\begin{tabular}{rrrrrrrrp{120mm}}
WV&WK&WB&ABK&ABB&ABV&AnzB&TW&Zahlencode \textcolor{red}{$\boldsymbol{Grundtext}$} Umschrift $|$"Ubersetzung(en)\\
1.&134.&594.&558.&2457.&1.&5&368&300\_30\_8\_10\_20 \textcolor{red}{\textcjheb{ky.hl+s}} SLCJK $|$was dir entsprosst/deine Spr"osslinge\\
2.&135.&595.&563.&2462.&6.&4&344&80\_200\_4\_60 \textcolor{red}{\textcjheb{sdrp}} PRDs $|$ist ein Lustgarten/(bilden einen) Baumgarten\\
3.&136.&596.&567.&2466.&10.&6&346&200\_40\_6\_50\_10\_40 \textcolor{red}{\textcjheb{mynwmr}} RMWNJM $|$von Granaten/(von) Granat"apfeln\\
4.&137.&597.&573.&2472.&16.&2&110&70\_40 \textcolor{red}{\textcjheb{m`}} aM $|$nebst/mit\\
5.&138.&598.&575.&2474.&18.&3&290&80\_200\_10 \textcolor{red}{\textcjheb{yrp}} PRJ $|$Fr"uchten/Frucht\\
6.&139.&599.&578.&2477.&21.&5&97&40\_3\_4\_10\_40 \textcolor{red}{\textcjheb{mydgm}} MGDJM $|$edlen/(von) k"ostlichen\\
7.&140.&600.&583.&2482.&26.&5&350&20\_80\_200\_10\_40 \textcolor{red}{\textcjheb{myrpk}} KPRJM $|$Zyperblumen/Hennastr"auchern\\
8.&141.&601.&588.&2487.&31.&2&110&70\_40 \textcolor{red}{\textcjheb{m`}} aM $|$nebst/mit\\
9.&142.&602.&590.&2489.&33.&5&304&50\_200\_4\_10\_40 \textcolor{red}{\textcjheb{mydrn}} NRDJM $|$Narden\\
\end{tabular}\medskip \\
Ende des Verses 4.13\\
Verse: 58, Buchstaben: 37, 594, 2493, Totalwerte: 2319, 40815, 172595\\
\\
Was dir entspro"st, ist ein Lustgarten von Granaten nebst edlen Fr"uchten, Zyperblumen nebst Narden; Narde und Safran.\\
\newpage 
{\bf -- 4.14}\\
\medskip \\
\begin{tabular}{rrrrrrrrp{120mm}}
WV&WK&WB&ABK&ABB&ABV&AnzB&TW&Zahlencode \textcolor{red}{$\boldsymbol{Grundtext}$} Umschrift $|$"Ubersetzung(en)\\
1.&143.&603.&595.&2494.&1.&3&254&50\_200\_4 \textcolor{red}{\textcjheb{drn}} NRD $|$Narde\\
2.&144.&604.&598.&2497.&4.&5&286&6\_20\_200\_20\_40 \textcolor{red}{\textcjheb{mkrkw}} WKRKM $|$und Safran/und Kurkuma\\
3.&145.&605.&603.&2502.&9.&3&155&100\_50\_5 \textcolor{red}{\textcjheb{hnq}} QNH $|$W"urzrohr/W"urzgras\\
4.&146.&606.&606.&2505.&12.&6&252&6\_100\_50\_40\_6\_50 \textcolor{red}{\textcjheb{nwmnqw}} WQNMWN $|$und Zimt(baum)\\
5.&147.&607.&612.&2511.&18.&2&110&70\_40 \textcolor{red}{\textcjheb{m`}} aM $|$nebst/samt\\
6.&148.&608.&614.&2513.&20.&2&50&20\_30 \textcolor{red}{\textcjheb{lk}} KL $|$all(erlei)\\
7.&149.&609.&616.&2515.&22.&3&170&70\_90\_10 \textcolor{red}{\textcjheb{y.s`}} a"sJ $|$Geh"olz/H"olzern\\
8.&150.&610.&619.&2518.&25.&5&93&30\_2\_6\_50\_5 \textcolor{red}{\textcjheb{hnwbl}} LBWNH $|$(des) Weihrauch(s)\\
9.&151.&611.&624.&2523.&30.&2&240&40\_200 \textcolor{red}{\textcjheb{rm}} MR $|$(der) Myrrhe\\
10.&152.&612.&626.&2525.&32.&6&448&6\_1\_5\_30\_6\_400 \textcolor{red}{\textcjheb{twlh'w}} WAHLWT $|$und Aloe/und Adlerholzb"aumen\\
11.&153.&613.&632.&2531.&38.&2&110&70\_40 \textcolor{red}{\textcjheb{m`}} aM $|$nebst/mit\\
12.&154.&614.&634.&2533.&40.&2&50&20\_30 \textcolor{red}{\textcjheb{lk}} KL $|$allen/all(erlei)\\
13.&155.&615.&636.&2535.&42.&4&511&200\_1\_300\_10 \textcolor{red}{\textcjheb{y+s'r}} RASJ $|$vortrefflichsten/H"auptern (=Erlesenen)\\
14.&156.&616.&640.&2539.&46.&5&392&2\_300\_40\_10\_40 \textcolor{red}{\textcjheb{mym+sb}} BSMJM $|$Gew"urzen/(der) Balsam(b"aum)e\\
\end{tabular}\medskip \\
Ende des Verses 4.14\\
Verse: 59, Buchstaben: 50, 644, 2543, Totalwerte: 3121, 43936, 175716\\
\\
W"urzrohr und Zimt, nebst allerlei Weihrauchgeh"olz, Myrrhe und Aloe nebst allen vortrefflichsten Gew"urzen;\\
\newpage 
{\bf -- 4.15}\\
\medskip \\
\begin{tabular}{rrrrrrrrp{120mm}}
WV&WK&WB&ABK&ABB&ABV&AnzB&TW&Zahlencode \textcolor{red}{$\boldsymbol{Grundtext}$} Umschrift $|$"Ubersetzung(en)\\
1.&157.&617.&645.&2544.&1.&4&170&40\_70\_10\_50 \textcolor{red}{\textcjheb{ny`m}} MaJN $|$eine Quelle/(der) Quellort\\
2.&158.&618.&649.&2548.&5.&4&103&3\_50\_10\_40 \textcolor{red}{\textcjheb{myng}} GNJM $|$(des) Garten(s)/in den G"arten\\
3.&159.&619.&653.&2552.&9.&3&203&2\_1\_200 \textcolor{red}{\textcjheb{r'b}} BAR $|$(ist) (ein) Brunnen\\
4.&160.&620.&656.&2555.&12.&3&90&40\_10\_40 \textcolor{red}{\textcjheb{mym}} MJM $|$Wassers/(von) Wassern\\
5.&161.&621.&659.&2558.&15.&4&68&8\_10\_10\_40 \textcolor{red}{\textcjheb{myy.h}} CJJM $|$lebendige(n)\\
6.&162.&622.&663.&2562.&19.&6&143&6\_50\_7\_30\_10\_40 \textcolor{red}{\textcjheb{mylznw}} WNZLJM $|$und (es) flie"sen B"ache/und str"omend\\
7.&163.&623.&669.&2568.&25.&2&90&40\_50 \textcolor{red}{\textcjheb{nm}} MN $|$(die) von dem\\
8.&164.&624.&671.&2570.&27.&5&138&30\_2\_50\_6\_50 \textcolor{red}{\textcjheb{nwnbl}} LBNWN $|$Libanon\\
\end{tabular}\medskip \\
Ende des Verses 4.15\\
Verse: 60, Buchstaben: 31, 675, 2574, Totalwerte: 1005, 44941, 176721\\
\\
eine Gartenquelle, ein Brunnen lebendigen Wassers, und B"ache, die vom Libanon flie"sen. -\\
\newpage 
{\bf -- 4.16}\\
\medskip \\
\begin{tabular}{rrrrrrrrp{120mm}}
WV&WK&WB&ABK&ABB&ABV&AnzB&TW&Zahlencode \textcolor{red}{$\boldsymbol{Grundtext}$} Umschrift $|$"Ubersetzung(en)\\
1.&165.&625.&676.&2575.&1.&4&286&70\_6\_200\_10 \textcolor{red}{\textcjheb{yrw`}} aWRJ $|$wache auf/erwache\\
2.&166.&626.&680.&2579.&5.&4&226&90\_80\_6\_50 \textcolor{red}{\textcjheb{nwp.s}} "sPWN $|$Nordwind\\
3.&167.&627.&684.&2583.&9.&5&25&6\_2\_6\_1\_10 \textcolor{red}{\textcjheb{y'wbw}} WBWAJ $|$und komm\\
4.&168.&628.&689.&2588.&14.&4&500&400\_10\_40\_50 \textcolor{red}{\textcjheb{nmyt}} TJMN $|$S"udwind\\
5.&169.&629.&693.&2592.&18.&5&113&5\_80\_10\_8\_10 \textcolor{red}{\textcjheb{y.hyph}} HPJCJ $|$durchwehe\\
6.&170.&630.&698.&2597.&23.&3&63&3\_50\_10 \textcolor{red}{\textcjheb{yng}} GNJ $|$meinen Garten\\
7.&171.&631.&701.&2600.&26.&4&53&10\_7\_30\_6 \textcolor{red}{\textcjheb{wlzy}} JZLW $|$lass tr"aufeln/sie (=es) sollen str"omen\\
8.&172.&632.&705.&2604.&30.&5&358&2\_300\_40\_10\_6 \textcolor{red}{\textcjheb{wym+sb}} BSMJW $|$seine Wohlger"uche/seine Balsam(d"ufte)\\
9.&173.&633.&710.&2609.&35.&3&13&10\_2\_1 \textcolor{red}{\textcjheb{'by}} JBA $|$komme/er (=es) m"oge kommen\\
10.&174.&634.&713.&2612.&38.&4&24&4\_6\_4\_10 \textcolor{red}{\textcjheb{ydwd}} DWDJ $|$mein Geliebter\\
11.&175.&635.&717.&2616.&42.&4&89&30\_3\_50\_6 \textcolor{red}{\textcjheb{wngl}} LGNW $|$in seinen Garten\\
12.&176.&636.&721.&2620.&46.&5&67&6\_10\_1\_20\_30 \textcolor{red}{\textcjheb{lk'yw}} WJAKL $|$und esse/und er m"oge genie"sen\\
13.&177.&637.&726.&2625.&51.&3&290&80\_200\_10 \textcolor{red}{\textcjheb{yrp}} PRJ $|$die Frucht\\
14.&178.&638.&729.&2628.&54.&5&63&40\_3\_4\_10\_6 \textcolor{red}{\textcjheb{wydgm}} MGDJW $|$die ihm k"ostliche/seiner K"ostlichkeiten\\
\end{tabular}\medskip \\
Ende des Verses 4.16\\
Verse: 61, Buchstaben: 58, 733, 2632, Totalwerte: 2170, 47111, 178891\\
\\
Wache auf, Nordwind, und komm, S"udwind: durchwehe meinen Garten, la"s tr"aufeln seine Wohlger"uche! Mein Geliebter komme in seinen Garten und esse die ihm k"ostliche Frucht. -\\
\\
{\bf Ende des Kapitels 4}\\
\newpage 
{\bf -- 5.1}\\
\medskip \\
\begin{tabular}{rrrrrrrrp{120mm}}
WV&WK&WB&ABK&ABB&ABV&AnzB&TW&Zahlencode \textcolor{red}{$\boldsymbol{Grundtext}$} Umschrift $|$"Ubersetzung(en)\\
1.&1.&639.&1.&2633.&1.&4&413&2\_1\_400\_10 \textcolor{red}{\textcjheb{yt'b}} BATJ $|$ich bin gekommen/ich kam\\
2.&2.&640.&5.&2637.&5.&4&93&30\_3\_50\_10 \textcolor{red}{\textcjheb{yngl}} LGNJ $|$in meinen Garten\\
3.&3.&641.&9.&2641.&9.&4&419&1\_8\_400\_10 \textcolor{red}{\textcjheb{yt.h'}} ACTJ $|$meine Schwester\\
4.&4.&642.&13.&2645.&13.&3&55&20\_30\_5 \textcolor{red}{\textcjheb{hlk}} KLH $|$meine Braut/(o) Braut\\
5.&5.&643.&16.&2648.&16.&5&621&1\_200\_10\_400\_10 \textcolor{red}{\textcjheb{ytyr'}} ARJTJ $|$habe gepfl"uckt/ich pfl"uckte\\
6.&6.&644.&21.&2653.&21.&4&256&40\_6\_200\_10 \textcolor{red}{\textcjheb{yrwm}} MWRJ $|$meine Myrrhe\\
7.&7.&645.&25.&2657.&25.&2&110&70\_40 \textcolor{red}{\textcjheb{m`}} aM $|$samt/mit\\
8.&8.&646.&27.&2659.&27.&4&352&2\_300\_40\_10 \textcolor{red}{\textcjheb{ym+sb}} BSMJ $|$meinem Balsam\\
9.&9.&647.&31.&2663.&31.&5&461&1\_20\_30\_400\_10 \textcolor{red}{\textcjheb{ytlk'}} AKLTJ $|$habe gegessen/ich esse\\
10.&10.&648.&36.&2668.&36.&4&290&10\_70\_200\_10 \textcolor{red}{\textcjheb{yr`y}} JaRJ $|$meine Wabe\\
11.&11.&649.&40.&2672.&40.&2&110&70\_40 \textcolor{red}{\textcjheb{m`}} aM $|$samt/mit\\
12.&12.&650.&42.&2674.&42.&4&316&4\_2\_300\_10 \textcolor{red}{\textcjheb{y+sbd}} DBSJ $|$meinem Honig\\
13.&13.&651.&46.&2678.&46.&5&1120&300\_400\_10\_400\_10 \textcolor{red}{\textcjheb{ytyt+s}} STJTJ $|$getrunken/ich trinke\\
14.&14.&652.&51.&2683.&51.&4&80&10\_10\_50\_10 \textcolor{red}{\textcjheb{ynyy}} JJNJ $|$meinen Wein\\
15.&15.&653.&55.&2687.&55.&2&110&70\_40 \textcolor{red}{\textcjheb{m`}} aM $|$samt/mit\\
16.&16.&654.&57.&2689.&57.&4&50&8\_30\_2\_10 \textcolor{red}{\textcjheb{ybl.h}} CLBJ $|$meiner Milch\\
17.&17.&655.&61.&2693.&61.&4&57&1\_20\_30\_6 \textcolor{red}{\textcjheb{wlk'}} AKLW $|$esst\\
18.&18.&656.&65.&2697.&65.&4&320&200\_70\_10\_40 \textcolor{red}{\textcjheb{my`r}} RaJM $|$Freunde/Gef"ahrten\\
19.&19.&657.&69.&2701.&69.&3&706&300\_400\_6 \textcolor{red}{\textcjheb{wt+s}} STW $|$trinkt\\
20.&20.&658.&72.&2704.&72.&5&532&6\_300\_20\_200\_6 \textcolor{red}{\textcjheb{wrk+sw}} WSKRW $|$und berauscht euch\\
21.&21.&659.&77.&2709.&77.&5&64&4\_6\_4\_10\_40 \textcolor{red}{\textcjheb{mydwd}} DWDJM $|$Geliebte/an Liebe(sgel"usten)\\
\end{tabular}\medskip \\
Ende des Verses 5.1\\
Verse: 62, Buchstaben: 81, 81, 2713, Totalwerte: 6535, 6535, 185426\\
\\
Ich bin in meinen Garten gekommen, meine Schwester, meine Braut, habe meine Myrrhe gepfl"uckt samt meinem Balsam, habe meine Wabe gegessen samt meinem Honig, meinen Wein getrunken samt meiner Milch. Esset, Freunde; trinket und berauschet euch, Geliebte!\\
\newpage 
{\bf -- 5.2}\\
\medskip \\
\begin{tabular}{rrrrrrrrp{120mm}}
WV&WK&WB&ABK&ABB&ABV&AnzB&TW&Zahlencode \textcolor{red}{$\boldsymbol{Grundtext}$} Umschrift $|$"Ubersetzung(en)\\
1.&22.&660.&82.&2714.&1.&3&61&1\_50\_10 \textcolor{red}{\textcjheb{yn'}} ANJ $|$ich\\
2.&23.&661.&85.&2717.&4.&4&365&10\_300\_50\_5 \textcolor{red}{\textcjheb{hn+sy}} JSNH $|$schlief/(war) schlafend(e)\\
3.&24.&662.&89.&2721.&8.&4&48&6\_30\_2\_10 \textcolor{red}{\textcjheb{yblw}} WLBJ $|$aber mein Herz/und mein Herz\\
4.&25.&663.&93.&2725.&12.&2&270&70\_200 \textcolor{red}{\textcjheb{r`}} aR $|$wachte/er (=es) war wachend\\
5.&26.&664.&95.&2727.&14.&3&136&100\_6\_30 \textcolor{red}{\textcjheb{lwq}} QWL $|$horch\\
6.&27.&665.&98.&2730.&17.&4&24&4\_6\_4\_10 \textcolor{red}{\textcjheb{ydwd}} DWDJ $|$mein Geliebter\\
7.&28.&666.&102.&2734.&21.&4&190&4\_6\_80\_100 \textcolor{red}{\textcjheb{qpwd}} DWPQ $|$er klopft/(ist) anklopfend(er)\\
8.&29.&667.&106.&2738.&25.&4&498&80\_400\_8\_10 \textcolor{red}{\textcjheb{y.htp}} PTCJ $|$tu auf\\
9.&30.&668.&110.&2742.&29.&2&40&30\_10 \textcolor{red}{\textcjheb{yl}} LJ $|$mir\\
10.&31.&669.&112.&2744.&31.&4&419&1\_8\_400\_10 \textcolor{red}{\textcjheb{yt.h'}} ACTJ $|$meine Schwester\\
11.&32.&670.&116.&2748.&35.&5&690&200\_70\_10\_400\_10 \textcolor{red}{\textcjheb{yty`r}} RaJTJ $|$meine Freundin\\
12.&33.&671.&121.&2753.&40.&5&476&10\_6\_50\_400\_10 \textcolor{red}{\textcjheb{ytnwy}} JWNTJ $|$meine Taube\\
13.&34.&672.&126.&2758.&45.&4&850&400\_40\_400\_10 \textcolor{red}{\textcjheb{ytmt}} TMTJ $|$meine Vollkommene/meine makellose\\
14.&35.&673.&130.&2762.&49.&5&811&300\_200\_1\_300\_10 \textcolor{red}{\textcjheb{y+s'r+s}} SRASJ $|$denn mein Haupt/weil mein Haupt\\
15.&36.&674.&135.&2767.&54.&4&121&50\_40\_30\_1 \textcolor{red}{\textcjheb{'lmn}} NMLA $|$(er (=es)) ist voll\\
16.&37.&675.&139.&2771.&58.&2&39&9\_30 \textcolor{red}{\textcjheb{l.t}} tL $|$(von) Tau\\
17.&38.&676.&141.&2773.&60.&6&612&100\_6\_90\_6\_400\_10 \textcolor{red}{\textcjheb{ytw.swq}} QW"sWTJ $|$meine Locken\\
18.&39.&677.&147.&2779.&66.&5&340&200\_60\_10\_60\_10 \textcolor{red}{\textcjheb{ysysr}} RsJsJ $|$voll Tropfen/(von) Tropfen\\
19.&40.&678.&152.&2784.&71.&4&75&30\_10\_30\_5 \textcolor{red}{\textcjheb{hlyl}} LJLH $|$der Nacht\\
\end{tabular}\medskip \\
Ende des Verses 5.2\\
Verse: 63, Buchstaben: 74, 155, 2787, Totalwerte: 6065, 12600, 191491\\
\\
Ich schlief, aber mein Herz wachte. Horch! Mein Geliebter! Er klopft: Tue mir auf, meine Schwester, meine Freundin, meine Taube, meine Vollkommene! Denn mein Haupt ist voll Tau, meine Locken voll Tropfen der Nacht. -\\
\newpage 
{\bf -- 5.3}\\
\medskip \\
\begin{tabular}{rrrrrrrrp{120mm}}
WV&WK&WB&ABK&ABB&ABV&AnzB&TW&Zahlencode \textcolor{red}{$\boldsymbol{Grundtext}$} Umschrift $|$"Ubersetzung(en)\\
1.&41.&679.&156.&2788.&1.&5&799&80\_300\_9\_400\_10 \textcolor{red}{\textcjheb{yt.t+sp}} PStTJ $|$ich habe ausgezogen\\
2.&42.&680.&161.&2793.&6.&2&401&1\_400 \textcolor{red}{\textcjheb{t'}} AT $|$**\\
3.&43.&681.&163.&2795.&8.&5&880&20\_400\_50\_400\_10 \textcolor{red}{\textcjheb{ytntk}} KTNTJ $|$mein Kleid\\
4.&44.&682.&168.&2800.&13.&5&56&1\_10\_20\_20\_5 \textcolor{red}{\textcjheb{hkky'}} AJKKH $|$wie\\
5.&45.&683.&173.&2805.&18.&6&388&1\_30\_2\_300\_50\_5 \textcolor{red}{\textcjheb{hn+sbl'}} ALBSNH $|$ich sollte anziehen es (wieder)\\
6.&46.&684.&179.&2811.&24.&5&708&200\_8\_90\_400\_10 \textcolor{red}{\textcjheb{yt.s.hr}} RC"sTJ $|$ich habe gewaschen\\
7.&47.&685.&184.&2816.&29.&2&401&1\_400 \textcolor{red}{\textcjheb{t'}} AT $|$**\\
8.&48.&686.&186.&2818.&31.&4&243&200\_3\_30\_10 \textcolor{red}{\textcjheb{ylgr}} RGLJ $|$meine F"u"se\\
9.&49.&687.&190.&2822.&35.&5&56&1\_10\_20\_20\_5 \textcolor{red}{\textcjheb{hkky'}} AJKKH $|$wie\\
10.&50.&688.&195.&2827.&40.&5&180&1\_9\_50\_80\_40 \textcolor{red}{\textcjheb{mpn.t'}} AtNPM $|$ich sollte sie (wieder) beschmutzen\\
\end{tabular}\medskip \\
Ende des Verses 5.3\\
Verse: 64, Buchstaben: 44, 199, 2831, Totalwerte: 4112, 16712, 195603\\
\\
Ich habe mein Kleid ausgezogen, wie sollte ich es wieder anziehen? Ich habe meine F"u"se gewaschen, wie sollte ich sie wieder beschmutzen? -\\
\newpage 
{\bf -- 5.4}\\
\medskip \\
\begin{tabular}{rrrrrrrrp{120mm}}
WV&WK&WB&ABK&ABB&ABV&AnzB&TW&Zahlencode \textcolor{red}{$\boldsymbol{Grundtext}$} Umschrift $|$"Ubersetzung(en)\\
1.&51.&689.&200.&2832.&1.&4&24&4\_6\_4\_10 \textcolor{red}{\textcjheb{ydwd}} DWDJ $|$mein Geliebter/mein Freund\\
2.&52.&690.&204.&2836.&5.&3&338&300\_30\_8 \textcolor{red}{\textcjheb{.hl+s}} SLC $|$(er) streckte\\
3.&53.&691.&207.&2839.&8.&3&20&10\_4\_6 \textcolor{red}{\textcjheb{wdy}} JDW $|$seine Hand\\
4.&54.&692.&210.&2842.&11.&2&90&40\_50 \textcolor{red}{\textcjheb{nm}} MN $|$durch\\
5.&55.&693.&212.&2844.&13.&3&213&5\_8\_200 \textcolor{red}{\textcjheb{r.hh}} HCR $|$die "Offnung\\
6.&56.&694.&215.&2847.&16.&4&126&6\_40\_70\_10 \textcolor{red}{\textcjheb{y`mw}} WMaJ $|$und mein Inneres/und meine Eingeweide\\
7.&57.&695.&219.&2851.&20.&3&51&5\_40\_6 \textcolor{red}{\textcjheb{wmh}} HMW $|$ward erregt/(sie) waren erregt\\
8.&58.&696.&222.&2854.&23.&4&116&70\_30\_10\_6 \textcolor{red}{\textcjheb{wyl`}} aLJW $|$seinetwegen\\
\end{tabular}\medskip \\
Ende des Verses 5.4\\
Verse: 65, Buchstaben: 26, 225, 2857, Totalwerte: 978, 17690, 196581\\
\\
Mein Geliebter streckte seine Hand durch die "Offnung, und mein Inneres ward seinetwegen erregt.\\
\newpage 
{\bf -- 5.5}\\
\medskip \\
\begin{tabular}{rrrrrrrrp{120mm}}
WV&WK&WB&ABK&ABB&ABV&AnzB&TW&Zahlencode \textcolor{red}{$\boldsymbol{Grundtext}$} Umschrift $|$"Ubersetzung(en)\\
1.&59.&697.&226.&2858.&1.&4&550&100\_40\_400\_10 \textcolor{red}{\textcjheb{ytmq}} QMTJ $|$ich stand auf\\
2.&60.&698.&230.&2862.&5.&3&61&1\_50\_10 \textcolor{red}{\textcjheb{yn'}} ANJ $|$/ich (selbst)\\
3.&61.&699.&233.&2865.&8.&4&518&30\_80\_400\_8 \textcolor{red}{\textcjheb{.htpl}} LPTC $|$um zu "offnen\\
4.&62.&700.&237.&2869.&12.&5&54&30\_4\_6\_4\_10 \textcolor{red}{\textcjheb{ydwdl}} LDWDJ $|$meinem Geliebten/f"ur meinen Freund\\
5.&63.&701.&242.&2874.&17.&4&30&6\_10\_4\_10 \textcolor{red}{\textcjheb{ydyw}} WJDJ $|$und meine H"ande\\
6.&64.&702.&246.&2878.&21.&4&145&50\_9\_80\_6 \textcolor{red}{\textcjheb{wp.tn}} NtPW $|$(sie) troffen\\
7.&65.&703.&250.&2882.&25.&3&246&40\_6\_200 \textcolor{red}{\textcjheb{rwm}} MWR $|$(von) Myrrhe\\
8.&66.&704.&253.&2885.&28.&7&579&6\_1\_90\_2\_70\_400\_10 \textcolor{red}{\textcjheb{yt`b.s'w}} WA"sBaTJ $|$und meine Finger\\
9.&67.&705.&260.&2892.&35.&3&246&40\_6\_200 \textcolor{red}{\textcjheb{rwm}} MWR $|$(von) Myrrhe\\
10.&68.&706.&263.&2895.&38.&3&272&70\_2\_200 \textcolor{red}{\textcjheb{rb`}} aBR $|$flie"sender/fl"ussiger\\
11.&69.&707.&266.&2898.&41.&2&100&70\_30 \textcolor{red}{\textcjheb{l`}} aL $|$an/auf\\
12.&70.&708.&268.&2900.&43.&4&506&20\_80\_6\_400 \textcolor{red}{\textcjheb{twpk}} KPWT $|$dem Griffe/die Griffe\\
13.&71.&709.&272.&2904.&47.&6&201&5\_40\_50\_70\_6\_30 \textcolor{red}{\textcjheb{lw`nmh}} HMNaWL $|$des Riegels\\
\end{tabular}\medskip \\
Ende des Verses 5.5\\
Verse: 66, Buchstaben: 52, 277, 2909, Totalwerte: 3508, 21198, 200089\\
\\
Ich stand auf, um meinem Geliebten zu "offnen, und meine H"ande troffen von Myrrhe und meine Finger von flie"sender Myrrhe an dem Griffe des Riegels.\\
\newpage 
{\bf -- 5.6}\\
\medskip \\
\begin{tabular}{rrrrrrrrp{120mm}}
WV&WK&WB&ABK&ABB&ABV&AnzB&TW&Zahlencode \textcolor{red}{$\boldsymbol{Grundtext}$} Umschrift $|$"Ubersetzung(en)\\
1.&72.&710.&278.&2910.&1.&5&898&80\_400\_8\_400\_10 \textcolor{red}{\textcjheb{yt.htp}} PTCTJ $|$ich "offnete\\
2.&73.&711.&283.&2915.&6.&3&61&1\_50\_10 \textcolor{red}{\textcjheb{yn'}} ANJ $|$/ich (selbst)\\
3.&74.&712.&286.&2918.&9.&5&54&30\_4\_6\_4\_10 \textcolor{red}{\textcjheb{ydwdl}} LDWDJ $|$meinem Geliebten/f"ur meinen Freund\\
4.&75.&713.&291.&2923.&14.&5&30&6\_4\_6\_4\_10 \textcolor{red}{\textcjheb{ydwdw}} WDWDJ $|$aber mein Geliebter/und mein Freund\\
5.&76.&714.&296.&2928.&19.&3&148&8\_40\_100 \textcolor{red}{\textcjheb{qm.h}} CMQ $|$hatte sich umgewandt/(er) war weggegangen\\
6.&77.&715.&299.&2931.&22.&3&272&70\_2\_200 \textcolor{red}{\textcjheb{rb`}} aBR $|$(er) (war) weiter gegangen\\
7.&78.&716.&302.&2934.&25.&4&440&50\_80\_300\_10 \textcolor{red}{\textcjheb{y+spn}} NPSJ $|$ich war/meine Seele\\
8.&79.&717.&306.&2938.&29.&4&106&10\_90\_1\_5 \textcolor{red}{\textcjheb{h'.sy}} J"sAH $|$au"ser mir/sie schwand dahin\\
9.&80.&718.&310.&2942.&33.&5&214&2\_4\_2\_200\_6 \textcolor{red}{\textcjheb{wrbdb}} BDBRW $|$w"ahrend er redete/bei seinem Sprechen\\
10.&81.&719.&315.&2947.&38.&7&823&2\_100\_300\_400\_10\_5\_6 \textcolor{red}{\textcjheb{whyt+sqb}} BQSTJHW $|$ich suchte ihn\\
11.&82.&720.&322.&2954.&45.&3&37&6\_30\_1 \textcolor{red}{\textcjheb{'lw}} WLA $|$und nicht/aber nicht\\
12.&83.&721.&325.&2957.&48.&7&552&40\_90\_1\_400\_10\_5\_6 \textcolor{red}{\textcjheb{whyt'.sm}} M"sATJHW $|$(ich) fand ihn\\
13.&84.&722.&332.&2964.&55.&6&717&100\_200\_1\_400\_10\_6 \textcolor{red}{\textcjheb{wyt'rq}} QRATJW $|$ich rief ihn\\
14.&85.&723.&338.&2970.&61.&3&37&6\_30\_1 \textcolor{red}{\textcjheb{'lw}} WLA $|$und nicht/aber nicht\\
15.&86.&724.&341.&2973.&64.&4&180&70\_50\_50\_10 \textcolor{red}{\textcjheb{ynn`}} aNNJ $|$er antwortete mir\\
\end{tabular}\medskip \\
Ende des Verses 5.6\\
Verse: 67, Buchstaben: 67, 344, 2976, Totalwerte: 4569, 25767, 204658\\
\\
Ich "offnete meinem Geliebten; aber mein Geliebter hatte sich umgewandt, war weitergegangen. Ich war au"ser mir, w"ahrend er redete. Ich suchte ihn und fand ihn nicht; ich rief ihn und er antwortete mir nicht.\\
\newpage 
{\bf -- 5.7}\\
\medskip \\
\begin{tabular}{rrrrrrrrp{120mm}}
WV&WK&WB&ABK&ABB&ABV&AnzB&TW&Zahlencode \textcolor{red}{$\boldsymbol{Grundtext}$} Umschrift $|$"Ubersetzung(en)\\
1.&87.&725.&345.&2977.&1.&5&191&40\_90\_1\_50\_10 \textcolor{red}{\textcjheb{yn'.sm}} M"sANJ $|$es fanden mich/sie (=es) trafen mich\\
2.&88.&726.&350.&2982.&6.&6&595&5\_300\_40\_200\_10\_40 \textcolor{red}{\textcjheb{myrm+sh}} HSMRJM $|$die W"achter\\
3.&89.&727.&356.&2988.&12.&6&119&5\_60\_2\_2\_10\_40 \textcolor{red}{\textcjheb{mybbsh}} HsBBJM $|$die umhergehen(den)\\
4.&90.&728.&362.&2994.&18.&4&282&2\_70\_10\_200 \textcolor{red}{\textcjheb{ry`b}} BaJR $|$in der Stadt/um die Stadt\\
5.&91.&729.&366.&2998.&22.&5&91&5\_20\_6\_50\_10 \textcolor{red}{\textcjheb{ynwkh}} HKWNJ $|$sie schlugen mich\\
6.&92.&730.&371.&3003.&27.&6&306&80\_90\_70\_6\_50\_10 \textcolor{red}{\textcjheb{ynw`.sp}} P"saWNJ $|$(sie) verwundeten mich\\
7.&93.&731.&377.&3009.&33.&4&357&50\_300\_1\_6 \textcolor{red}{\textcjheb{w'+sn}} NSAW $|$nahmen weg/sie (=es) nahmen ab\\
8.&94.&732.&381.&3013.&37.&2&401&1\_400 \textcolor{red}{\textcjheb{t'}} AT $|$**\\
9.&95.&733.&383.&3015.&39.&5&228&200\_4\_10\_4\_10 \textcolor{red}{\textcjheb{ydydr}} RDJDJ $|$meinen Schleier/mein Oberkleid\\
10.&96.&734.&388.&3020.&44.&4&150&40\_70\_30\_10 \textcolor{red}{\textcjheb{yl`m}} MaLJ $|$(von) mir\\
11.&97.&735.&392.&3024.&48.&4&550&300\_40\_200\_10 \textcolor{red}{\textcjheb{yrm+s}} SMRJ $|$die W"achter/die Bewachenden\\
12.&98.&736.&396.&3028.&52.&5&459&5\_8\_40\_6\_400 \textcolor{red}{\textcjheb{twm.hh}} HCMWT $|$der Mauern\\
\end{tabular}\medskip \\
Ende des Verses 5.7\\
Verse: 68, Buchstaben: 56, 400, 3032, Totalwerte: 3729, 29496, 208387\\
\\
Es fanden mich die W"achter, die in der Stadt umhergehen: sie schlugen mich, verwundeten mich; die W"achter der Mauern nahmen mir meinen Schleier weg.\\
\newpage 
{\bf -- 5.8}\\
\medskip \\
\begin{tabular}{rrrrrrrrp{120mm}}
WV&WK&WB&ABK&ABB&ABV&AnzB&TW&Zahlencode \textcolor{red}{$\boldsymbol{Grundtext}$} Umschrift $|$"Ubersetzung(en)\\
1.&99.&737.&401.&3033.&1.&6&787&5\_300\_2\_70\_400\_10 \textcolor{red}{\textcjheb{yt`b+sh}} HSBaTJ $|$ich beschw"ore\\
2.&100.&738.&407.&3039.&7.&4&461&1\_400\_20\_40 \textcolor{red}{\textcjheb{mkt'}} ATKM $|$euch\\
3.&101.&739.&411.&3043.&11.&4&458&2\_50\_6\_400 \textcolor{red}{\textcjheb{twnb}} BNWT $|$(ihr) T"ochter\\
4.&102.&740.&415.&3047.&15.&6&586&10\_200\_6\_300\_30\_40 \textcolor{red}{\textcjheb{ml+swry}} JRWSLM $|$Jerusalem(s)\\
5.&103.&741.&421.&3053.&21.&2&41&1\_40 \textcolor{red}{\textcjheb{m'}} AM $|$wenn\\
6.&104.&742.&423.&3055.&23.&5&537&400\_40\_90\_1\_6 \textcolor{red}{\textcjheb{w'.smt}} TM"sAW $|$ihr findet\\
7.&105.&743.&428.&3060.&28.&2&401&1\_400 \textcolor{red}{\textcjheb{t'}} AT $|$**\\
8.&106.&744.&430.&3062.&30.&4&24&4\_6\_4\_10 \textcolor{red}{\textcjheb{ydwd}} DWDJ $|$meinen Geliebten/meinen Freund\\
9.&107.&745.&434.&3066.&34.&2&45&40\_5 \textcolor{red}{\textcjheb{hm}} MH $|$was\\
10.&108.&746.&436.&3068.&36.&5&423&400\_3\_10\_4\_6 \textcolor{red}{\textcjheb{wdygt}} TGJDW $|$sollt ihr berichten/ihr sollt sagen\\
11.&109.&747.&441.&3073.&41.&2&36&30\_6 \textcolor{red}{\textcjheb{wl}} LW $|$(zu) ihm\\
12.&110.&748.&443.&3075.&43.&5&744&300\_8\_6\_30\_400 \textcolor{red}{\textcjheb{tlw.h+s}} SCWLT $|$dass krank (seiend)\\
13.&111.&749.&448.&3080.&48.&4&13&1\_5\_2\_5 \textcolor{red}{\textcjheb{hbh'}} AHBH $|$(vor) Liebe\\
14.&112.&750.&452.&3084.&52.&3&61&1\_50\_10 \textcolor{red}{\textcjheb{yn'}} ANJ $|$ich (bin)\\
\end{tabular}\medskip \\
Ende des Verses 5.8\\
Verse: 69, Buchstaben: 54, 454, 3086, Totalwerte: 4617, 34113, 213004\\
\\
Ich beschw"ore euch, T"ochter Jerusalems, wenn ihr meinen Geliebten findet, was sollt ihr ihm berichten? Da"s ich krank bin vor Liebe. -\\
\newpage 
{\bf -- 5.9}\\
\medskip \\
\begin{tabular}{rrrrrrrrp{120mm}}
WV&WK&WB&ABK&ABB&ABV&AnzB&TW&Zahlencode \textcolor{red}{$\boldsymbol{Grundtext}$} Umschrift $|$"Ubersetzung(en)\\
1.&113.&751.&455.&3087.&1.&2&45&40\_5 \textcolor{red}{\textcjheb{hm}} MH $|$was ist/was (hat)\\
2.&114.&752.&457.&3089.&3.&4&34&4\_6\_4\_20 \textcolor{red}{\textcjheb{kdwd}} DWDK $|$dein Geliebter/dein Freund\\
3.&115.&753.&461.&3093.&7.&4&54&40\_4\_6\_4 \textcolor{red}{\textcjheb{dwdm}} MDWD $|$vor einem anderen Geliebten/mehr als ein anderer Freund\\
4.&116.&754.&465.&3097.&11.&4&100&5\_10\_80\_5 \textcolor{red}{\textcjheb{hpyh}} HJPH $|$du Sch"onste/die (=du) sch"on(st)e\\
5.&117.&755.&469.&3101.&15.&5&402&2\_50\_300\_10\_40 \textcolor{red}{\textcjheb{my+snb}} BNSJM $|$unter den Frauen\\
6.&118.&756.&474.&3106.&20.&2&45&40\_5 \textcolor{red}{\textcjheb{hm}} MH $|$was ist/was (hat)\\
7.&119.&757.&476.&3108.&22.&4&34&4\_6\_4\_20 \textcolor{red}{\textcjheb{kdwd}} DWDK $|$dein Geliebter/dein Freund\\
8.&120.&758.&480.&3112.&26.&4&54&40\_4\_6\_4 \textcolor{red}{\textcjheb{dwdm}} MDWD $|$vor einem anderen Geliebten/mehr als ein anderer Freund\\
9.&121.&759.&484.&3116.&30.&4&345&300\_20\_20\_5 \textcolor{red}{\textcjheb{hkk+s}} SKKH $|$dass also\\
10.&122.&760.&488.&3120.&34.&7&833&5\_300\_2\_70\_400\_50\_6 \textcolor{red}{\textcjheb{wnt`b+sh}} HSBaTNW $|$du uns beschw"orst/du hast schw"oren machen uns\\
\end{tabular}\medskip \\
Ende des Verses 5.9\\
Verse: 70, Buchstaben: 40, 494, 3126, Totalwerte: 1946, 36059, 214950\\
\\
Was ist dein Geliebter vor einem anderen Geliebten, du Sch"onste unter den Frauen? Was ist dein Geliebter vor einem anderen Geliebten, da"s du uns also beschw"orst? -\\
\newpage 
{\bf -- 5.10}\\
\medskip \\
\begin{tabular}{rrrrrrrrp{120mm}}
WV&WK&WB&ABK&ABB&ABV&AnzB&TW&Zahlencode \textcolor{red}{$\boldsymbol{Grundtext}$} Umschrift $|$"Ubersetzung(en)\\
1.&123.&761.&495.&3127.&1.&4&24&4\_6\_4\_10 \textcolor{red}{\textcjheb{ydwd}} DWDJ $|$mein Geliebter/mein Freund\\
2.&124.&762.&499.&3131.&5.&2&98&90\_8 \textcolor{red}{\textcjheb{.h.s}} "sC $|$ist wei"s/(ist) gl"anzend(er)\\
3.&125.&763.&501.&3133.&7.&5&57&6\_1\_4\_6\_40 \textcolor{red}{\textcjheb{mwd'w}} WADWM $|$und rot(braun)\\
4.&126.&764.&506.&3138.&12.&4&43&4\_3\_6\_30 \textcolor{red}{\textcjheb{lwgd}} DGWL $|$ausgezeichnet(er)\\
5.&127.&765.&510.&3142.&16.&5&249&40\_200\_2\_2\_5 \textcolor{red}{\textcjheb{hbbrm}} MRBBH $|$vor Zehntausenden/aus Zehntausenden\\
\end{tabular}\medskip \\
Ende des Verses 5.10\\
Verse: 71, Buchstaben: 20, 514, 3146, Totalwerte: 471, 36530, 215421\\
\\
Mein Geliebter ist wei"s und rot, ausgezeichnet vor Zehntausenden.\\
\newpage 
{\bf -- 5.11}\\
\medskip \\
\begin{tabular}{rrrrrrrrp{120mm}}
WV&WK&WB&ABK&ABB&ABV&AnzB&TW&Zahlencode \textcolor{red}{$\boldsymbol{Grundtext}$} Umschrift $|$"Ubersetzung(en)\\
1.&128.&766.&515.&3147.&1.&4&507&200\_1\_300\_6 \textcolor{red}{\textcjheb{w+s'r}} RASW $|$sein Haupt\\
2.&129.&767.&519.&3151.&5.&3&460&20\_400\_40 \textcolor{red}{\textcjheb{mtk}} KTM $|$(ist) Gold\\
3.&130.&768.&522.&3154.&8.&2&87&80\_7 \textcolor{red}{\textcjheb{zp}} PZ $|$gediegenes (feines)\\
4.&131.&769.&524.&3156.&10.&7&618&100\_6\_90\_6\_400\_10\_6 \textcolor{red}{\textcjheb{wytw.swq}} QW"sWTJW $|$seine Locken\\
5.&132.&770.&531.&3163.&17.&6&910&400\_30\_400\_30\_10\_40 \textcolor{red}{\textcjheb{myltlt}} TLTLJM $|$sind herabwallend/wie Dattelrispen\\
6.&133.&771.&537.&3169.&23.&5&914&300\_8\_200\_6\_400 \textcolor{red}{\textcjheb{twr.h+s}} SCRWT $|$schwarz(e)\\
7.&134.&772.&542.&3174.&28.&5&298&20\_70\_6\_200\_2 \textcolor{red}{\textcjheb{brw`k}} KaWRB $|$wie der Rabe\\
\end{tabular}\medskip \\
Ende des Verses 5.11\\
Verse: 72, Buchstaben: 32, 546, 3178, Totalwerte: 3794, 40324, 219215\\
\\
Sein Haupt ist gediegenes, feines Gold, seine Locken sind herabwallend, schwarz wie der Rabe;\\
\newpage 
{\bf -- 5.12}\\
\medskip \\
\begin{tabular}{rrrrrrrrp{120mm}}
WV&WK&WB&ABK&ABB&ABV&AnzB&TW&Zahlencode \textcolor{red}{$\boldsymbol{Grundtext}$} Umschrift $|$"Ubersetzung(en)\\
1.&135.&773.&547.&3179.&1.&5&146&70\_10\_50\_10\_6 \textcolor{red}{\textcjheb{wyny`}} aJNJW $|$seine Augen\\
2.&136.&774.&552.&3184.&6.&6&136&20\_10\_6\_50\_10\_40 \textcolor{red}{\textcjheb{mynwyk}} KJWNJM $|$wie Tauben\\
3.&137.&775.&558.&3190.&12.&2&100&70\_30 \textcolor{red}{\textcjheb{l`}} aL $|$an\\
4.&138.&776.&560.&3192.&14.&5&201&1\_80\_10\_100\_10 \textcolor{red}{\textcjheb{yqyp'}} APJQJ $|$B"achen/Flussbetten\\
5.&139.&777.&565.&3197.&19.&3&90&40\_10\_40 \textcolor{red}{\textcjheb{mym}} MJM $|$(mit) Wasser\\
6.&140.&778.&568.&3200.&22.&5&704&200\_8\_90\_6\_400 \textcolor{red}{\textcjheb{tw.s.hr}} RC"sWT $|$badend/gebadet\\
7.&141.&779.&573.&3205.&27.&4&42&2\_8\_30\_2 \textcolor{red}{\textcjheb{bl.hb}} BCLB $|$in (der) Milch\\
8.&142.&780.&577.&3209.&31.&5&718&10\_300\_2\_6\_400 \textcolor{red}{\textcjheb{twb+sy}} JSBWT $|$eingefasste/sitzend(e)\\
9.&143.&781.&582.&3214.&36.&2&100&70\_30 \textcolor{red}{\textcjheb{l`}} aL $|$/an\\
10.&144.&782.&584.&3216.&38.&4&471&40\_30\_1\_400 \textcolor{red}{\textcjheb{t'lm}} MLAT $|$Steine/(Wasser)f"ulle\\
\end{tabular}\medskip \\
Ende des Verses 5.12\\
Verse: 73, Buchstaben: 41, 587, 3219, Totalwerte: 2708, 43032, 221923\\
\\
seine Augen wie Tauben an Wasserb"achen, badend in Milch, eingefa"ste Steine;\\
\newpage 
{\bf -- 5.13}\\
\medskip \\
\begin{tabular}{rrrrrrrrp{120mm}}
WV&WK&WB&ABK&ABB&ABV&AnzB&TW&Zahlencode \textcolor{red}{$\boldsymbol{Grundtext}$} Umschrift $|$"Ubersetzung(en)\\
1.&145.&783.&588.&3220.&1.&4&54&30\_8\_10\_6 \textcolor{red}{\textcjheb{wy.hl}} LCJW $|$seine Wangen/seine Kinnbacken\\
2.&146.&784.&592.&3224.&5.&6&699&20\_70\_200\_6\_3\_400 \textcolor{red}{\textcjheb{tgwr`k}} KaRWGT $|$wie Beete/wie ein Beet\\
3.&147.&785.&598.&3230.&11.&4&347&5\_2\_300\_40 \textcolor{red}{\textcjheb{m+sbh}} HBSM $|$von W"urzkraut/des Balsam(strauches)\\
4.&148.&786.&602.&3234.&15.&6&483&40\_3\_4\_30\_6\_400 \textcolor{red}{\textcjheb{twldgm}} MGDLWT $|$Anh"ohen/wie T"urme\\
5.&149.&787.&608.&3240.&21.&6&398&40\_200\_100\_8\_10\_40 \textcolor{red}{\textcjheb{my.hqrm}} MRQCJM $|$von duftenden Pflanzen/(von) Gew"urzkr"autern\\
6.&150.&788.&614.&3246.&27.&7&1202&300\_80\_400\_6\_400\_10\_6 \textcolor{red}{\textcjheb{wytwtp+s}} SPTWTJW $|$seine Lippen\\
7.&151.&789.&621.&3253.&34.&6&706&300\_6\_300\_50\_10\_40 \textcolor{red}{\textcjheb{myn+sw+s}} SWSNJM $|$(wie) (wei"se) Lilien\\
8.&152.&790.&627.&3259.&40.&5&545&50\_9\_80\_6\_400 \textcolor{red}{\textcjheb{twp.tn}} NtPWT $|$tr"aufelnd\\
9.&153.&791.&632.&3264.&45.&3&246&40\_6\_200 \textcolor{red}{\textcjheb{rwm}} MWR $|$(von) Myrrhe\\
10.&154.&792.&635.&3267.&48.&3&272&70\_2\_200 \textcolor{red}{\textcjheb{rb`}} aBR $|$flie"sender/fl"ussiger\\
\end{tabular}\medskip \\
Ende des Verses 5.13\\
Verse: 74, Buchstaben: 50, 637, 3269, Totalwerte: 4952, 47984, 226875\\
\\
seine Wangen wie Beete von W"urzkraut, Anh"ohen von duftenden Pflanzen; seine Lippen Lilien, tr"aufelnd von flie"sender Myrrhe;\\
\newpage 
{\bf -- 5.14}\\
\medskip \\
\begin{tabular}{rrrrrrrrp{120mm}}
WV&WK&WB&ABK&ABB&ABV&AnzB&TW&Zahlencode \textcolor{red}{$\boldsymbol{Grundtext}$} Umschrift $|$"Ubersetzung(en)\\
1.&155.&793.&638.&3270.&1.&4&30&10\_4\_10\_6 \textcolor{red}{\textcjheb{wydy}} JDJW $|$seine H"ande\\
2.&156.&794.&642.&3274.&5.&5&83&3\_30\_10\_30\_10 \textcolor{red}{\textcjheb{ylylg}} GLJLJ $|$Rollen/(sind) Rundstangen\\
3.&157.&795.&647.&3279.&10.&3&14&7\_5\_2 \textcolor{red}{\textcjheb{bhz}} ZHB $|$goldene/(von) Gold\\
4.&158.&796.&650.&3282.&13.&6&161&40\_40\_30\_1\_10\_40 \textcolor{red}{\textcjheb{my'lmm}} MMLAJM $|$besetzt\\
5.&159.&797.&656.&3288.&19.&6&1212&2\_400\_200\_300\_10\_300 \textcolor{red}{\textcjheb{+sy+srtb}} BTRSJS $|$mit Topasen/mit Tarschisch\\
6.&160.&798.&662.&3294.&25.&4&126&40\_70\_10\_6 \textcolor{red}{\textcjheb{wy`m}} MaJW $|$sein (Unter)Leib\\
7.&161.&799.&666.&3298.&29.&3&770&70\_300\_400 \textcolor{red}{\textcjheb{t+s`}} aST $|$ein Kunstwerk\\
8.&162.&800.&669.&3301.&32.&2&350&300\_50 \textcolor{red}{\textcjheb{n+s}} SN $|$von Elfenbein/(aus) Elfenbein\\
9.&163.&801.&671.&3303.&34.&5&620&40\_70\_30\_80\_400 \textcolor{red}{\textcjheb{tpl`m}} MaLPT $|$bedeckt(e)\\
10.&164.&802.&676.&3308.&39.&6&400&60\_80\_10\_200\_10\_40 \textcolor{red}{\textcjheb{myryps}} sPJRJM $|$mit Saphiren\\
\end{tabular}\medskip \\
Ende des Verses 5.14\\
Verse: 75, Buchstaben: 44, 681, 3313, Totalwerte: 3766, 51750, 230641\\
\\
seine H"ande goldene Rollen, mit Topasen besetzt; sein Leib ein Kunstwerk von Elfenbein, bedeckt mit Saphiren;\\
\newpage 
{\bf -- 5.15}\\
\medskip \\
\begin{tabular}{rrrrrrrrp{120mm}}
WV&WK&WB&ABK&ABB&ABV&AnzB&TW&Zahlencode \textcolor{red}{$\boldsymbol{Grundtext}$} Umschrift $|$"Ubersetzung(en)\\
1.&165.&803.&682.&3314.&1.&5&422&300\_6\_100\_10\_6 \textcolor{red}{\textcjheb{wyqw+s}} SWQJW $|$seine Schenkel\\
2.&166.&804.&687.&3319.&6.&5&130&70\_40\_6\_4\_10 \textcolor{red}{\textcjheb{ydwm`}} aMWDJ $|$(sind) S"aulen\\
3.&167.&805.&692.&3324.&11.&2&600&300\_300 \textcolor{red}{\textcjheb{+s+s}} SS $|$(von) (wei"sem) Marmor\\
4.&168.&806.&694.&3326.&13.&6&164&40\_10\_60\_4\_10\_40 \textcolor{red}{\textcjheb{mydsym}} MJsDJM $|$gegr"undet(e)\\
5.&169.&807.&700.&3332.&19.&2&100&70\_30 \textcolor{red}{\textcjheb{l`}} aL $|$auf\\
6.&170.&808.&702.&3334.&21.&4&65&1\_4\_50\_10 \textcolor{red}{\textcjheb{ynd'}} ADNJ $|$Unters"atze/Fu"sgestellen\\
7.&171.&809.&706.&3338.&25.&2&87&80\_7 \textcolor{red}{\textcjheb{zp}} PZ $|$von feinem Gold/(von) gediegenem Gold\\
8.&172.&810.&708.&3340.&27.&5&252&40\_200\_1\_5\_6 \textcolor{red}{\textcjheb{wh'rm}} MRAHW $|$seine Gestalt/sein Aussehen\\
9.&173.&811.&713.&3345.&32.&6&158&20\_30\_2\_50\_6\_50 \textcolor{red}{\textcjheb{nwnblk}} KLBNWN $|$wie der Libanon\\
10.&174.&812.&719.&3351.&38.&4&216&2\_8\_6\_200 \textcolor{red}{\textcjheb{rw.hb}} BCWR $|$auserlesen/(ist) erlesen(er)\\
11.&175.&813.&723.&3355.&42.&6&278&20\_1\_200\_7\_10\_40 \textcolor{red}{\textcjheb{myzr'k}} KARZJM $|$wie die Zedern\\
\end{tabular}\medskip \\
Ende des Verses 5.15\\
Verse: 76, Buchstaben: 47, 728, 3360, Totalwerte: 2472, 54222, 233113\\
\\
seine Schenkel S"aulen von wei"sem Marmor, gegr"undet auf Unters"atze von feinem Golde; seine Gestalt wie der Libanon, auserlesen wie die Zedern;\\
\newpage 
{\bf -- 5.16}\\
\medskip \\
\begin{tabular}{rrrrrrrrp{120mm}}
WV&WK&WB&ABK&ABB&ABV&AnzB&TW&Zahlencode \textcolor{red}{$\boldsymbol{Grundtext}$} Umschrift $|$"Ubersetzung(en)\\
1.&176.&814.&729.&3361.&1.&3&34&8\_20\_6 \textcolor{red}{\textcjheb{wk.h}} CKW $|$sein Gaumen\\
2.&177.&815.&732.&3364.&4.&6&630&40\_40\_400\_100\_10\_40 \textcolor{red}{\textcjheb{myqtmm}} MMTQJM $|$(ist) (lauter) S"u"sigkeit(en)\\
3.&178.&816.&738.&3370.&10.&4&62&6\_20\_30\_6 \textcolor{red}{\textcjheb{wlkw}} WKLW $|$und alles an ihm\\
4.&179.&817.&742.&3374.&14.&6&142&40\_8\_40\_4\_10\_40 \textcolor{red}{\textcjheb{mydm.hm}} MCMDJM $|$ist lieblich/Lieblichkeit(en)\\
5.&180.&818.&748.&3380.&20.&2&12&7\_5 \textcolor{red}{\textcjheb{hz}} ZH $|$das ist/dieser\\
6.&181.&819.&750.&3382.&22.&4&24&4\_6\_4\_10 \textcolor{red}{\textcjheb{ydwd}} DWDJ $|$mein Geliebter/mein Freund\\
7.&182.&820.&754.&3386.&26.&3&18&6\_7\_5 \textcolor{red}{\textcjheb{hzw}} WZH $|$und das/und dieser\\
8.&183.&821.&757.&3389.&29.&3&280&200\_70\_10 \textcolor{red}{\textcjheb{y`r}} RaJ $|$mein Freund/mein Gef"ahrte\\
9.&184.&822.&760.&3392.&32.&4&458&2\_50\_6\_400 \textcolor{red}{\textcjheb{twnb}} BNWT $|$(ihr) T"ochter\\
10.&185.&823.&764.&3396.&36.&6&586&10\_200\_6\_300\_30\_40 \textcolor{red}{\textcjheb{ml+swry}} JRWSLM $|$Jerusalem(s)\\
\end{tabular}\medskip \\
Ende des Verses 5.16\\
Verse: 77, Buchstaben: 41, 769, 3401, Totalwerte: 2246, 56468, 235359\\
\\
sein Gaumen ist lauter S"u"sigkeit, und alles an ihm ist lieblich. Das ist mein Geliebter, und das mein Freund, ihr T"ochter Jerusalems! -\\
\\
{\bf Ende des Kapitels 5}\\
\newpage 
{\bf -- 6.1}\\
\medskip \\
\begin{tabular}{rrrrrrrrp{120mm}}
WV&WK&WB&ABK&ABB&ABV&AnzB&TW&Zahlencode \textcolor{red}{$\boldsymbol{Grundtext}$} Umschrift $|$"Ubersetzung(en)\\
1.&1.&824.&1.&3402.&1.&3&56&1\_50\_5 \textcolor{red}{\textcjheb{hn'}} ANH $|$wohin\\
2.&2.&825.&4.&3405.&4.&3&55&5\_30\_20 \textcolor{red}{\textcjheb{klh}} HLK $|$ist gegangen/(er) ging\\
3.&3.&826.&7.&3408.&7.&4&34&4\_6\_4\_20 \textcolor{red}{\textcjheb{kdwd}} DWDK $|$dein Geliebter/dein Freund\\
4.&4.&827.&11.&3412.&11.&4&100&5\_10\_80\_5 \textcolor{red}{\textcjheb{hpyh}} HJPH $|$du Sch"onste/die (=du) sch"on(st)e\\
5.&5.&828.&15.&3416.&15.&5&402&2\_50\_300\_10\_40 \textcolor{red}{\textcjheb{my+snb}} BNSJM $|$unter den Frauen\\
6.&6.&829.&20.&3421.&20.&3&56&1\_50\_5 \textcolor{red}{\textcjheb{hn'}} ANH $|$wohin\\
7.&7.&830.&23.&3424.&23.&3&135&80\_50\_5 \textcolor{red}{\textcjheb{hnp}} PNH $|$hat sich gewendet/(er) hat sich gewandt\\
8.&8.&831.&26.&3427.&26.&4&34&4\_6\_4\_20 \textcolor{red}{\textcjheb{kdwd}} DWDK $|$dein Geliebter/dein Freund\\
9.&9.&832.&30.&3431.&30.&7&514&6\_50\_2\_100\_300\_50\_6 \textcolor{red}{\textcjheb{wn+sqbnw}} WNBQSNW $|$und wir (wollen) suchen ihn\\
10.&10.&833.&37.&3438.&37.&3&130&70\_40\_20 \textcolor{red}{\textcjheb{km`}} aMK $|$mit dir\\
\end{tabular}\medskip \\
Ende des Verses 6.1\\
Verse: 78, Buchstaben: 39, 39, 3440, Totalwerte: 1516, 1516, 236875\\
\\
Wohin ist dein Geliebter gegangen, du Sch"onste unter den Frauen? wohin hat dein Geliebter sich gewendet? Und wir wollen ihn mit dir suchen. -\\
\newpage 
{\bf -- 6.2}\\
\medskip \\
\begin{tabular}{rrrrrrrrp{120mm}}
WV&WK&WB&ABK&ABB&ABV&AnzB&TW&Zahlencode \textcolor{red}{$\boldsymbol{Grundtext}$} Umschrift $|$"Ubersetzung(en)\\
1.&11.&834.&40.&3441.&1.&4&24&4\_6\_4\_10 \textcolor{red}{\textcjheb{ydwd}} DWDJ $|$mein Geliebter/mein Freund\\
2.&12.&835.&44.&3445.&5.&3&214&10\_200\_4 \textcolor{red}{\textcjheb{dry}} JRD $|$ist hinabgegangen/(er) ging hinab\\
3.&13.&836.&47.&3448.&8.&4&89&30\_3\_50\_6 \textcolor{red}{\textcjheb{wngl}} LGNW $|$in seinen Garten\\
4.&14.&837.&51.&3452.&12.&7&715&30\_70\_200\_6\_3\_6\_400 \textcolor{red}{\textcjheb{twgwr`l}} LaRWGWT $|$zu (den) Beeten\\
5.&15.&838.&58.&3459.&19.&4&347&5\_2\_300\_40 \textcolor{red}{\textcjheb{m+sbh}} HBSM $|$(des) W"urzkraut(s)/des Balsam(strauches)\\
6.&16.&839.&62.&3463.&23.&5&706&30\_200\_70\_6\_400 \textcolor{red}{\textcjheb{tw`rl}} LRaWT $|$(um) zu weiden\\
7.&17.&840.&67.&3468.&28.&5&105&2\_3\_50\_10\_40 \textcolor{red}{\textcjheb{myngb}} BGNJM $|$in den G"arten\\
8.&18.&841.&72.&3473.&33.&5&175&6\_30\_30\_100\_9 \textcolor{red}{\textcjheb{.tqllw}} WLLQt $|$und zu pl"ucken\\
9.&19.&842.&77.&3478.&38.&6&706&300\_6\_300\_50\_10\_40 \textcolor{red}{\textcjheb{myn+sw+s}} SWSNJM $|$(wei"se) Lilien\\
\end{tabular}\medskip \\
Ende des Verses 6.2\\
Verse: 79, Buchstaben: 43, 82, 3483, Totalwerte: 3081, 4597, 239956\\
\\
Mein Geliebter ist in seinen Garten hinabgegangen, zu den W"urzkrautbeeten, um in den G"arten zu weiden und Lilien zu pfl"ucken.\\
\newpage 
{\bf -- 6.3}\\
\medskip \\
\begin{tabular}{rrrrrrrrp{120mm}}
WV&WK&WB&ABK&ABB&ABV&AnzB&TW&Zahlencode \textcolor{red}{$\boldsymbol{Grundtext}$} Umschrift $|$"Ubersetzung(en)\\
1.&20.&843.&83.&3484.&1.&3&61&1\_50\_10 \textcolor{red}{\textcjheb{yn'}} ANJ $|$ich (bin)\\
2.&21.&844.&86.&3487.&4.&5&54&30\_4\_6\_4\_10 \textcolor{red}{\textcjheb{ydwdl}} LDWDJ $|$meines Geliebten/zu meinem Freund\\
3.&22.&845.&91.&3492.&9.&5&30&6\_4\_6\_4\_10 \textcolor{red}{\textcjheb{ydwdw}} WDWDJ $|$und mein Geliebter/und mein Freund\\
4.&23.&846.&96.&3497.&14.&2&40&30\_10 \textcolor{red}{\textcjheb{yl}} LJ $|$ist mein/zu mir\\
5.&24.&847.&98.&3499.&16.&4&280&5\_200\_70\_5 \textcolor{red}{\textcjheb{h`rh}} HRaH $|$der weidet/der weidend(er)\\
6.&25.&848.&102.&3503.&20.&7&708&2\_300\_6\_300\_50\_10\_40 \textcolor{red}{\textcjheb{myn+sw+sb}} BSWSNJM $|$unter den (wei"sen) Lilien\\
\end{tabular}\medskip \\
Ende des Verses 6.3\\
Verse: 80, Buchstaben: 26, 108, 3509, Totalwerte: 1173, 5770, 241129\\
\\
Ich bin meines Geliebten; und mein Geliebter ist mein, der unter den Lilien weidet.\\
\newpage 
{\bf -- 6.4}\\
\medskip \\
\begin{tabular}{rrrrrrrrp{120mm}}
WV&WK&WB&ABK&ABB&ABV&AnzB&TW&Zahlencode \textcolor{red}{$\boldsymbol{Grundtext}$} Umschrift $|$"Ubersetzung(en)\\
1.&26.&849.&109.&3510.&1.&3&95&10\_80\_5 \textcolor{red}{\textcjheb{hpy}} JPH $|$sch"on(e)\\
2.&27.&850.&112.&3513.&4.&2&401&1\_400 \textcolor{red}{\textcjheb{t'}} AT $|$du (bist)\\
3.&28.&851.&114.&3515.&6.&5&690&200\_70\_10\_400\_10 \textcolor{red}{\textcjheb{yty`r}} RaJTJ $|$meine Freundin\\
4.&29.&852.&119.&3520.&11.&5&715&20\_400\_200\_90\_5 \textcolor{red}{\textcjheb{h.srtk}} KTR"sH $|$wie Tirza///$<$Anmut$>$\\
5.&30.&853.&124.&3525.&16.&4&62&50\_1\_6\_5 \textcolor{red}{\textcjheb{hw'n}} NAWH $|$lieblich/anmutig(e)\\
6.&31.&854.&128.&3529.&20.&7&606&20\_10\_200\_6\_300\_30\_40 \textcolor{red}{\textcjheb{ml+swryk}} KJRWSLM $|$wie Jerusalem\\
7.&32.&855.&135.&3536.&27.&4&56&1\_10\_40\_5 \textcolor{red}{\textcjheb{hmy'}} AJMH $|$furchtbar(e)\\
8.&33.&856.&139.&3540.&31.&7&513&20\_50\_4\_3\_30\_6\_400 \textcolor{red}{\textcjheb{twlgdnk}} KNDGLWT $|$wie Kriegsscharen/wie Gescharte\\
\end{tabular}\medskip \\
Ende des Verses 6.4\\
Verse: 81, Buchstaben: 37, 145, 3546, Totalwerte: 3138, 8908, 244267\\
\\
Du bist sch"on, meine Freundin, wie Tirza, lieblich wie Jerusalem, furchtbar wie Kriegsscharen.\\
\newpage 
{\bf -- 6.5}\\
\medskip \\
\begin{tabular}{rrrrrrrrp{120mm}}
WV&WK&WB&ABK&ABB&ABV&AnzB&TW&Zahlencode \textcolor{red}{$\boldsymbol{Grundtext}$} Umschrift $|$"Ubersetzung(en)\\
1.&34.&857.&146.&3547.&1.&4&77&5\_60\_2\_10 \textcolor{red}{\textcjheb{ybsh}} HsBJ $|$wende\\
2.&35.&858.&150.&3551.&5.&5&160&70\_10\_50\_10\_20 \textcolor{red}{\textcjheb{kyny`}} aJNJK $|$deine Augen\\
3.&36.&859.&155.&3556.&10.&5&107&40\_50\_3\_4\_10 \textcolor{red}{\textcjheb{ydgnm}} MNGDJ $|$von mir ab\\
4.&37.&860.&160.&3561.&15.&3&345&300\_5\_40 \textcolor{red}{\textcjheb{mh+s}} SHM $|$denn sie\\
5.&38.&861.&163.&3564.&18.&7&282&5\_200\_5\_10\_2\_50\_10 \textcolor{red}{\textcjheb{ynbyhrh}} HRHJBNJ $|$"uberw"altigen mich/sie verwirr(t)en mich\\
6.&39.&862.&170.&3571.&25.&4&590&300\_70\_200\_20 \textcolor{red}{\textcjheb{kr`+s}} SaRK $|$dein Haar\\
7.&40.&863.&174.&3575.&29.&4&294&20\_70\_4\_200 \textcolor{red}{\textcjheb{rd`k}} KaDR $|$(ist) wie eine Herde\\
8.&41.&864.&178.&3579.&33.&5&132&5\_70\_7\_10\_40 \textcolor{red}{\textcjheb{myz`h}} HaZJM $|$(der) Ziegen\\
9.&42.&865.&183.&3584.&38.&5&639&300\_3\_30\_300\_6 \textcolor{red}{\textcjheb{w+slg+s}} SGLSW $|$die lagern/welche (sie) wall(t)en herab\\
10.&43.&866.&188.&3589.&43.&2&90&40\_50 \textcolor{red}{\textcjheb{nm}} MN $|$an den Abh"angen/von\\
11.&44.&867.&190.&3591.&45.&5&112&5\_3\_30\_70\_4 \textcolor{red}{\textcjheb{d`lgh}} HGLaD $|$des Gilead/(dem) Gilead\\
\end{tabular}\medskip \\
Ende des Verses 6.5\\
Verse: 82, Buchstaben: 49, 194, 3595, Totalwerte: 2828, 11736, 247095\\
\\
Wende deine Augen von mir ab, denn sie "uberw"altigen mich. Dein Haar ist wie eine Herde Ziegen, die an den Abh"angen des Gilead lagern;\\
\newpage 
{\bf -- 6.6}\\
\medskip \\
\begin{tabular}{rrrrrrrrp{120mm}}
WV&WK&WB&ABK&ABB&ABV&AnzB&TW&Zahlencode \textcolor{red}{$\boldsymbol{Grundtext}$} Umschrift $|$"Ubersetzung(en)\\
1.&45.&868.&195.&3596.&1.&4&380&300\_50\_10\_20 \textcolor{red}{\textcjheb{kyn+s}} SNJK $|$deine Z"ahne\\
2.&46.&869.&199.&3600.&5.&4&294&20\_70\_4\_200 \textcolor{red}{\textcjheb{rd`k}} KaDR $|$(sind) wie eine Herde\\
3.&47.&870.&203.&3604.&9.&6&293&5\_200\_8\_30\_10\_40 \textcolor{red}{\textcjheb{myl.hrh}} HRCLJM $|$Mutterschafe/von Schafen\\
4.&48.&871.&209.&3610.&15.&4&406&300\_70\_30\_6 \textcolor{red}{\textcjheb{wl`+s}} SaLW $|$die heraufkommen/welche (sie) stiegen herauf\\
5.&49.&872.&213.&3614.&19.&2&90&40\_50 \textcolor{red}{\textcjheb{nm}} MN $|$aus\\
6.&50.&873.&215.&3616.&21.&5&308&5\_200\_8\_90\_5 \textcolor{red}{\textcjheb{h.s.hrh}} HRC"sH $|$der Schwemme\\
7.&51.&874.&220.&3621.&26.&4&390&300\_20\_30\_40 \textcolor{red}{\textcjheb{mlk+s}} SKLM $|$welche allzumal/welche allesamt\\
8.&52.&875.&224.&3625.&30.&7&897&40\_400\_1\_10\_40\_6\_400 \textcolor{red}{\textcjheb{twmy'tm}} MTAJMWT $|$Zwillinge geb"aren/sind zwillingstr"achtig\\
9.&53.&876.&231.&3632.&37.&5&361&6\_300\_20\_30\_5 \textcolor{red}{\textcjheb{hlk+sw}} WSKLH $|$und unfruchtbar/und eine ohne Junge\\
10.&54.&877.&236.&3637.&42.&3&61&1\_10\_50 \textcolor{red}{\textcjheb{ny'}} AJN $|$nicht ist\\
11.&55.&878.&239.&3640.&45.&3&47&2\_5\_40 \textcolor{red}{\textcjheb{mhb}} BHM $|$unter ihnen (eines)\\
\end{tabular}\medskip \\
Ende des Verses 6.6\\
Verse: 83, Buchstaben: 47, 241, 3642, Totalwerte: 3527, 15263, 250622\\
\\
deine Z"ahne sind wie eine Herde Mutterschafe, die aus der Schwemme heraufkommen, welche allzumal Zwillinge geb"aren, und keines unter ihnen ist unfruchtbar;\\
\newpage 
{\bf -- 6.7}\\
\medskip \\
\begin{tabular}{rrrrrrrrp{120mm}}
WV&WK&WB&ABK&ABB&ABV&AnzB&TW&Zahlencode \textcolor{red}{$\boldsymbol{Grundtext}$} Umschrift $|$"Ubersetzung(en)\\
1.&56.&879.&242.&3643.&1.&4&138&20\_80\_30\_8 \textcolor{red}{\textcjheb{.hlpk}} KPLC $|$wie ein Schnittst"uck/wie eine Scheibe\\
2.&57.&880.&246.&3647.&5.&5&301&5\_200\_40\_6\_50 \textcolor{red}{\textcjheb{nwmrh}} HRMWN $|$einer Granate/des Granatapfels\\
3.&58.&881.&251.&3652.&10.&4&720&200\_100\_400\_20 \textcolor{red}{\textcjheb{ktqr}} RQTK $|$ist deine Schl"afe/(sieht aus) deine Schl"afe\\
4.&59.&882.&255.&3656.&14.&4&116&40\_2\_70\_4 \textcolor{red}{\textcjheb{d`bm}} MBaD $|$hinter/durch\\
5.&60.&883.&259.&3660.&18.&5&580&30\_90\_40\_400\_20 \textcolor{red}{\textcjheb{ktm.sl}} L"sMTK $|$deinem Schleier/deinen Schleier\\
\end{tabular}\medskip \\
Ende des Verses 6.7\\
Verse: 84, Buchstaben: 22, 263, 3664, Totalwerte: 1855, 17118, 252477\\
\\
wie ein Schnittst"uck einer Granate ist deine Schl"afe hinter deinem Schleier.\\
\newpage 
{\bf -- 6.8}\\
\medskip \\
\begin{tabular}{rrrrrrrrp{120mm}}
WV&WK&WB&ABK&ABB&ABV&AnzB&TW&Zahlencode \textcolor{red}{$\boldsymbol{Grundtext}$} Umschrift $|$"Ubersetzung(en)\\
1.&61.&884.&264.&3665.&1.&4&650&300\_300\_10\_40 \textcolor{red}{\textcjheb{my+s+s}} SSJM $|$sechzig\\
2.&62.&885.&268.&3669.&5.&3&50&5\_40\_5 \textcolor{red}{\textcjheb{hmh}} HMH $|$sind der/(sind) sie (=es)\\
3.&63.&886.&271.&3672.&8.&5&496&40\_30\_20\_6\_400 \textcolor{red}{\textcjheb{twklm}} MLKWT $|$K"oniginnen\\
4.&64.&887.&276.&3677.&13.&6&446&6\_300\_40\_50\_10\_40 \textcolor{red}{\textcjheb{mynm+sw}} WSMNJM $|$und achtzig\\
5.&65.&888.&282.&3683.&19.&7&473&80\_10\_30\_3\_300\_10\_40 \textcolor{red}{\textcjheb{my+sglyp}} PJLGSJM $|$der Kebsweiber/(sind) Nebenfrauen\\
6.&66.&889.&289.&3690.&26.&6&552&6\_70\_30\_40\_6\_400 \textcolor{red}{\textcjheb{twml`w}} WaLMWT $|$und Jungfrauen/und (der) M"adchen\\
7.&67.&890.&295.&3696.&32.&3&61&1\_10\_50 \textcolor{red}{\textcjheb{ny'}} AJN $|$ohne/gibt es nicht\\
8.&68.&891.&298.&3699.&35.&4&380&40\_60\_80\_200 \textcolor{red}{\textcjheb{rpsm}} MsPR $|$(eine) Zahl\\
\end{tabular}\medskip \\
Ende des Verses 6.8\\
Verse: 85, Buchstaben: 38, 301, 3702, Totalwerte: 3108, 20226, 255585\\
\\
Sechzig sind der K"oniginnen und achtzig der Kebsweiber, und Jungfrauen ohne Zahl.\\
\newpage 
{\bf -- 6.9}\\
\medskip \\
\begin{tabular}{rrrrrrrrp{120mm}}
WV&WK&WB&ABK&ABB&ABV&AnzB&TW&Zahlencode \textcolor{red}{$\boldsymbol{Grundtext}$} Umschrift $|$"Ubersetzung(en)\\
1.&69.&892.&302.&3703.&1.&3&409&1\_8\_400 \textcolor{red}{\textcjheb{t.h'}} ACT $|$eine/einzig\\
2.&70.&893.&305.&3706.&4.&3&16&5\_10\_1 \textcolor{red}{\textcjheb{'yh}} HJA $|$ist/sie (ist)\\
3.&71.&894.&308.&3709.&7.&5&476&10\_6\_50\_400\_10 \textcolor{red}{\textcjheb{ytnwy}} JWNTJ $|$meine Taube\\
4.&72.&895.&313.&3714.&12.&4&850&400\_40\_400\_10 \textcolor{red}{\textcjheb{ytmt}} TMTJ $|$meine Vollkommene/meine Vollendete\\
5.&73.&896.&317.&3718.&16.&3&409&1\_8\_400 \textcolor{red}{\textcjheb{t.h'}} ACT $|$einzig(e)\\
6.&74.&897.&320.&3721.&19.&3&16&5\_10\_1 \textcolor{red}{\textcjheb{'yh}} HJA $|$sie (ist)\\
7.&75.&898.&323.&3724.&22.&4&76&30\_1\_40\_5 \textcolor{red}{\textcjheb{hm'l}} LAMH $|$ihrer Mutter/f"ur ihre Mutter\\
8.&76.&899.&327.&3728.&26.&3&207&2\_200\_5 \textcolor{red}{\textcjheb{hrb}} BRH $|$die Auserkorene/untadelig\\
9.&77.&900.&330.&3731.&29.&3&16&5\_10\_1 \textcolor{red}{\textcjheb{'yh}} HJA $|$sie (ist)\\
10.&78.&901.&333.&3734.&32.&7&485&30\_10\_6\_30\_4\_400\_5 \textcolor{red}{\textcjheb{htdlwyl}} LJWLDTH $|$ihrer Geb"arerin/f"ur ihre Geb"arerin\\
11.&79.&902.&340.&3741.&39.&4&212&200\_1\_6\_5 \textcolor{red}{\textcjheb{hw'r}} RAWH $|$(es) sahen sie/sie (=es) sahen sie\\
12.&80.&903.&344.&3745.&43.&4&458&2\_50\_6\_400 \textcolor{red}{\textcjheb{twnb}} BNWT $|$(die) T"ochter\\
13.&81.&904.&348.&3749.&47.&7&528&6\_10\_1\_300\_200\_6\_5 \textcolor{red}{\textcjheb{hwr+s'yw}} WJASRWH $|$und (sie (=es)) priesen gl"ucklich sie\\
14.&82.&905.&355.&3756.&54.&5&496&40\_30\_20\_6\_400 \textcolor{red}{\textcjheb{twklm}} MLKWT $|$K"oniginnen\\
15.&83.&906.&360.&3761.&59.&8&479&6\_80\_10\_30\_3\_300\_10\_40 \textcolor{red}{\textcjheb{my+sglypw}} WPJLGSJM $|$und Kebsweiber/und Nebenfrauen\\
16.&84.&907.&368.&3769.&67.&7&92&6\_10\_5\_30\_30\_6\_5 \textcolor{red}{\textcjheb{hwllhyw}} WJHLLWH $|$und sie r"uhmten sie/und sie bewunderten sie\\
\end{tabular}\medskip \\
Ende des Verses 6.9\\
Verse: 86, Buchstaben: 73, 374, 3775, Totalwerte: 5225, 25451, 260810\\
\\
Eine ist meine Taube, meine Vollkommene; sie ist die einzige ihrer Mutter, sie ist die Auserkorene ihrer Geb"arerin. T"ochter sahen sie und priesen sie gl"ucklich, K"oniginnen und Kebsweiber, und sie r"uhmten sie.\\
\newpage 
{\bf -- 6.10}\\
\medskip \\
\begin{tabular}{rrrrrrrrp{120mm}}
WV&WK&WB&ABK&ABB&ABV&AnzB&TW&Zahlencode \textcolor{red}{$\boldsymbol{Grundtext}$} Umschrift $|$"Ubersetzung(en)\\
1.&85.&908.&375.&3776.&1.&2&50&40\_10 \textcolor{red}{\textcjheb{ym}} MJ $|$wer (ist)\\
2.&86.&909.&377.&3778.&3.&3&408&7\_1\_400 \textcolor{red}{\textcjheb{t'z}} ZAT $|$sie/diese\\
3.&87.&910.&380.&3781.&6.&6&540&5\_50\_300\_100\_80\_5 \textcolor{red}{\textcjheb{hpq+snh}} HNSQPH $|$die da hervorgl"anzt/die Herabblickende\\
4.&88.&911.&386.&3787.&12.&3&66&20\_40\_6 \textcolor{red}{\textcjheb{wmk}} KMW $|$wie\\
5.&89.&912.&389.&3790.&15.&3&508&300\_8\_200 \textcolor{red}{\textcjheb{r.h+s}} SCR $|$die Morgenr"ote\\
6.&90.&913.&392.&3793.&18.&3&95&10\_80\_5 \textcolor{red}{\textcjheb{hpy}} JPH $|$sch"on(e)\\
7.&91.&914.&395.&3796.&21.&5&107&20\_30\_2\_50\_5 \textcolor{red}{\textcjheb{hnblk}} KLBNH $|$wie der Mond/wie die Wei"se\\
8.&92.&915.&400.&3801.&26.&3&207&2\_200\_5 \textcolor{red}{\textcjheb{hrb}} BRH $|$rein(e)\\
9.&93.&916.&403.&3804.&29.&4&73&20\_8\_40\_5 \textcolor{red}{\textcjheb{hm.hk}} KCMH $|$wie die Sonne/wie die (Sonnen)glut\\
10.&94.&917.&407.&3808.&33.&4&56&1\_10\_40\_5 \textcolor{red}{\textcjheb{hmy'}} AJMH $|$furchtbar(e)\\
11.&95.&918.&411.&3812.&37.&7&513&20\_50\_4\_3\_30\_6\_400 \textcolor{red}{\textcjheb{twlgdnk}} KNDGLWT $|$wie Kriegsscharen/wie Gescharte\\
\end{tabular}\medskip \\
Ende des Verses 6.10\\
Verse: 87, Buchstaben: 43, 417, 3818, Totalwerte: 2623, 28074, 263433\\
\\
Wer ist sie, die da hervorgl"anzt wie die Morgenr"ote, sch"on wie der Mond, rein wie die Sonne, furchtbar wie Kriegsscharen? -\\
\newpage 
{\bf -- 6.11}\\
\medskip \\
\begin{tabular}{rrrrrrrrp{120mm}}
WV&WK&WB&ABK&ABB&ABV&AnzB&TW&Zahlencode \textcolor{red}{$\boldsymbol{Grundtext}$} Umschrift $|$"Ubersetzung(en)\\
1.&96.&919.&418.&3819.&1.&2&31&1\_30 \textcolor{red}{\textcjheb{l'}} AL $|$in den/zum\\
2.&97.&920.&420.&3821.&3.&3&453&3\_50\_400 \textcolor{red}{\textcjheb{tng}} GNT $|$Garten\\
3.&98.&921.&423.&3824.&6.&4&17&1\_3\_6\_7 \textcolor{red}{\textcjheb{zwg'}} AGWZ $|$(der) Nuss/der N"usse\\
4.&99.&922.&427.&3828.&10.&5&624&10\_200\_4\_400\_10 \textcolor{red}{\textcjheb{ytdry}} JRDTJ $|$ich ging hinab\\
5.&100.&923.&432.&3833.&15.&5&637&30\_200\_1\_6\_400 \textcolor{red}{\textcjheb{tw'rl}} LRAWT $|$um zu (be)sehen\\
6.&101.&924.&437.&3838.&20.&4&15&2\_1\_2\_10 \textcolor{red}{\textcjheb{yb'b}} BABJ $|$die jungen Triebe/nach den Trieben\\
7.&102.&925.&441.&3842.&24.&4&93&5\_50\_8\_30 \textcolor{red}{\textcjheb{l.hnh}} HNCL $|$des Tales/in dem Tal\\
8.&103.&926.&445.&3846.&28.&5&637&30\_200\_1\_6\_400 \textcolor{red}{\textcjheb{tw'rl}} LRAWT $|$um zu sehen\\
9.&104.&927.&450.&3851.&33.&5&298&5\_80\_200\_8\_5 \textcolor{red}{\textcjheb{h.hrph}} HPRCH $|$ob ausgeschlagen w"are/ob sie Knospen trieb\\
10.&105.&928.&455.&3856.&38.&4&138&5\_3\_80\_50 \textcolor{red}{\textcjheb{npgh}} HGPN $|$der Weinstock/die Weinrebe\\
11.&106.&929.&459.&3860.&42.&4&151&5\_50\_90\_6 \textcolor{red}{\textcjheb{w.snh}} HN"sW $|$ob bl"uhten/sie (=es) bl"uhten\\
12.&107.&930.&463.&3864.&46.&6&345&5\_200\_40\_50\_10\_40 \textcolor{red}{\textcjheb{mynmrh}} HRMNJM $|$die Granaten/die Granat"apfel(b"aume)\\
\end{tabular}\medskip \\
Ende des Verses 6.11\\
Verse: 88, Buchstaben: 51, 468, 3869, Totalwerte: 3439, 31513, 266872\\
\\
In den Nu"sgarten ging ich hinab, um die jungen Triebe des Tales zu besehen, um zu sehen, ob der Weinstock ausgeschlagen w"are, ob die Granaten bl"uhten.\\
\newpage 
{\bf -- 6.12}\\
\medskip \\
\begin{tabular}{rrrrrrrrp{120mm}}
WV&WK&WB&ABK&ABB&ABV&AnzB&TW&Zahlencode \textcolor{red}{$\boldsymbol{Grundtext}$} Umschrift $|$"Ubersetzung(en)\\
1.&108.&931.&469.&3870.&1.&2&31&30\_1 \textcolor{red}{\textcjheb{'l}} LA $|$nicht/nein\\
2.&109.&932.&471.&3872.&3.&5&494&10\_4\_70\_400\_10 \textcolor{red}{\textcjheb{yt`dy}} JDaTJ $|$bewusst/ich kannte\\
3.&110.&933.&476.&3877.&8.&4&440&50\_80\_300\_10 \textcolor{red}{\textcjheb{y+spn}} NPSJ $|$meine Seele\\
4.&111.&934.&480.&3881.&12.&5&800&300\_40\_400\_50\_10 \textcolor{red}{\textcjheb{yntm+s}} SMTNJ $|$(sie) (ver)setzte mich\\
5.&112.&935.&485.&3886.&17.&6&668&40\_200\_20\_2\_6\_400 \textcolor{red}{\textcjheb{twbkrm}} MRKBWT $|$auf den Prachtwagen/(zu den) Wagen\\
6.&113.&936.&491.&3892.&23.&3&120&70\_40\_10 \textcolor{red}{\textcjheb{ym`}} aMJ $|$meines Volkes\\
7.&114.&937.&494.&3895.&26.&4&66&50\_4\_10\_2 \textcolor{red}{\textcjheb{bydn}} NDJB $|$(des) willigen/des edlen\\
\end{tabular}\medskip \\
Ende des Verses 6.12\\
Verse: 89, Buchstaben: 29, 497, 3898, Totalwerte: 2619, 34132, 269491\\
\\
Unbewu"st setzte mich meine Seele auf den Prachtwagen meines willigen Volkes. -\\
\newpage 
{\bf -- 6.13 (7.1)}\\
\medskip \\
\begin{tabular}{rrrrrrrrp{120mm}}
WV&WK&WB&ABK&ABB&ABV&AnzB&TW&Zahlencode \textcolor{red}{$\boldsymbol{Grundtext}$} Umschrift $|$"Ubersetzung(en)\\
1.&115.&938.&498.&3899.&1.&4&318&300\_6\_2\_10 \textcolor{red}{\textcjheb{ybw+s}} SWBJ $|$kehre um\\
2.&116.&939.&502.&3903.&5.&4&318&300\_6\_2\_10 \textcolor{red}{\textcjheb{ybw+s}} SWBJ $|$kehre um\\
3.&117.&940.&506.&3907.&9.&7&791&5\_300\_6\_30\_40\_10\_400 \textcolor{red}{\textcjheb{tymlw+sh}} HSWLMJT $|$Sulamith/du Schulamitin //$<$Friedliche$>$\\
4.&118.&941.&513.&3914.&16.&4&318&300\_6\_2\_10 \textcolor{red}{\textcjheb{ybw+s}} SWBJ $|$kehre um\\
5.&119.&942.&517.&3918.&20.&4&318&300\_6\_2\_10 \textcolor{red}{\textcjheb{ybw+s}} SWBJ $|$kehre um\\
6.&120.&943.&521.&3922.&24.&5&76&6\_50\_8\_7\_5 \textcolor{red}{\textcjheb{hz.hnw}} WNCZH $|$dass wir anschauen/und wir bewundern\\
7.&121.&944.&526.&3927.&29.&2&22&2\_20 \textcolor{red}{\textcjheb{kb}} BK $|$dich\\
8.&122.&945.&528.&3929.&31.&2&45&40\_5 \textcolor{red}{\textcjheb{hm}} MH $|$was\\
9.&123.&946.&530.&3931.&33.&4&421&400\_8\_7\_6 \textcolor{red}{\textcjheb{wz.ht}} TCZW $|$m"oget ihr schauen/ihr wollt bewundern\\
10.&124.&947.&534.&3935.&37.&7&788&2\_300\_6\_30\_40\_10\_400 \textcolor{red}{\textcjheb{tymlw+sb}} BSWLMJT $|$an der Sulamith/an der Schulamitin\\
11.&125.&948.&541.&3942.&44.&5&498&20\_40\_8\_30\_400 \textcolor{red}{\textcjheb{tl.hmk}} KMCLT $|$wie den Reigen/beim Reigentanz\\
12.&126.&949.&546.&3947.&49.&6&153&5\_40\_8\_50\_10\_40 \textcolor{red}{\textcjheb{myn.hmh}} HMCNJM $|$von Machanaim/der Heerlager\\
\end{tabular}\medskip \\
Ende des Verses 6.13\\
Verse: 90, Buchstaben: 54, 551, 3952, Totalwerte: 4066, 38198, 273557\\
\\
Kehre um, kehre um, Sulamith; kehre um, kehre um, da"s wir dich anschauen! -Was m"oget ihr an der Sulamith schauen? -Wie den Reigen von Machanaim.\\
\\
{\bf Ende des Kapitels 6}\\
\newpage 
{\bf -- 7.1 (7.2)}\\
\medskip \\
\begin{tabular}{rrrrrrrrp{120mm}}
WV&WK&WB&ABK&ABB&ABV&AnzB&TW&Zahlencode \textcolor{red}{$\boldsymbol{Grundtext}$} Umschrift $|$"Ubersetzung(en)\\
1.&1.&950.&1.&3953.&1.&2&45&40\_5 \textcolor{red}{\textcjheb{hm}} MH $|$wie\\
2.&2.&951.&3.&3955.&3.&3&96&10\_80\_6 \textcolor{red}{\textcjheb{wpy}} JPW $|$(sie (=es)) sind sch"on\\
3.&3.&952.&6.&3958.&6.&5&220&80\_70\_40\_10\_20 \textcolor{red}{\textcjheb{kym`p}} PaMJK $|$deine Tritte/deine Schritte\\
4.&4.&953.&11.&3963.&11.&6&202&2\_50\_70\_30\_10\_40 \textcolor{red}{\textcjheb{myl`nb}} BNaLJM $|$in den Schuhen/in den Sandalen\\
5.&5.&954.&17.&3969.&17.&2&402&2\_400 \textcolor{red}{\textcjheb{tb}} BT $|$(du) Tochter\\
6.&6.&955.&19.&3971.&19.&4&66&50\_4\_10\_2 \textcolor{red}{\textcjheb{bydn}} NDJB $|$(eines) F"ursten\\
7.&7.&956.&23.&3975.&23.&5&164&8\_40\_6\_100\_10 \textcolor{red}{\textcjheb{yqwm.h}} CMWQJ $|$die Biegungen/die Rundungen\\
8.&8.&957.&28.&3980.&28.&5&260&10\_200\_20\_10\_20 \textcolor{red}{\textcjheb{kykry}} JRKJK $|$deiner H"uften/(von) deinen H"uften\\
9.&9.&958.&33.&3985.&33.&3&66&20\_40\_6 \textcolor{red}{\textcjheb{wmk}} KMW $|$(sind) wie\\
10.&10.&959.&36.&3988.&36.&5&89&8\_30\_1\_10\_40 \textcolor{red}{\textcjheb{my'l.h}} CLAJM $|$(ein) (Hals)geschmeide\\
11.&11.&960.&41.&3993.&41.&4&415&40\_70\_300\_5 \textcolor{red}{\textcjheb{h+s`m}} MaSH $|$(ein) (Mach)Werk\\
12.&12.&961.&45.&3997.&45.&3&24&10\_4\_10 \textcolor{red}{\textcjheb{ydy}} JDJ $|$von Hand/von H"anden\\
13.&13.&962.&48.&4000.&48.&3&91&1\_40\_50 \textcolor{red}{\textcjheb{nm'}} AMN $|$(eines) K"unstler(s)\\
\end{tabular}\medskip \\
Ende des Verses 7.1\\
Verse: 91, Buchstaben: 50, 50, 4002, Totalwerte: 2140, 2140, 275697\\
\\
Wie sch"on sind deine Tritte in den Schuhen, F"urstentochter! Die Biegungen deiner H"uften sind wie ein Halsgeschmeide, ein Werk von K"unstlerhand.\\
\newpage 
{\bf -- 7.2 (7.3)}\\
\medskip \\
\begin{tabular}{rrrrrrrrp{120mm}}
WV&WK&WB&ABK&ABB&ABV&AnzB&TW&Zahlencode \textcolor{red}{$\boldsymbol{Grundtext}$} Umschrift $|$"Ubersetzung(en)\\
1.&14.&963.&51.&4003.&1.&4&720&300\_200\_200\_20 \textcolor{red}{\textcjheb{krr+s}} SRRK $|$dein Nabel\\
2.&15.&964.&55.&4007.&5.&3&54&1\_3\_50 \textcolor{red}{\textcjheb{ng'}} AGN $|$(ist) eine Schale\\
3.&16.&965.&58.&4010.&8.&4&270&5\_60\_5\_200 \textcolor{red}{\textcjheb{rhsh}} HsHR $|$runde/der Rundung\\
4.&17.&966.&62.&4014.&12.&2&31&1\_30 \textcolor{red}{\textcjheb{l'}} AL $|$nicht\\
5.&18.&967.&64.&4016.&14.&4&278&10\_8\_60\_200 \textcolor{red}{\textcjheb{rs.hy}} JCsR $|$in welcher mangelt/er (=es) wird mangeln\\
6.&19.&968.&68.&4020.&18.&4&55&5\_40\_7\_3 \textcolor{red}{\textcjheb{gzmh}} HMZG $|$der Mischwein\\
7.&20.&969.&72.&4024.&22.&4&81&2\_9\_50\_20 \textcolor{red}{\textcjheb{kn.tb}} BtNK $|$dein Leib\\
8.&21.&970.&76.&4028.&26.&4&710&70\_200\_40\_400 \textcolor{red}{\textcjheb{tmr`}} aRMT $|$ein Haufen/gleich einem Haufen\\
9.&22.&971.&80.&4032.&30.&4&67&8\_9\_10\_40 \textcolor{red}{\textcjheb{my.t.h}} CtJM $|$(von) Weizen\\
10.&23.&972.&84.&4036.&34.&4&74&60\_6\_3\_5 \textcolor{red}{\textcjheb{hgws}} sWGH $|$umz"aunt/umhegt(e)\\
11.&24.&973.&88.&4040.&38.&7&708&2\_300\_6\_300\_50\_10\_40 \textcolor{red}{\textcjheb{myn+sw+sb}} BSWSNJM $|$mit (wei"sen) Lilien\\
\end{tabular}\medskip \\
Ende des Verses 7.2\\
Verse: 92, Buchstaben: 44, 94, 4046, Totalwerte: 3048, 5188, 278745\\
\\
Dein Nabel ist eine runde Schale, in welcher der Mischwein nicht mangelt; dein Leib ein Weizenhaufen, umz"aunt mit Lilien.\\
\newpage 
{\bf -- 7.3 (7.4)}\\
\medskip \\
\begin{tabular}{rrrrrrrrp{120mm}}
WV&WK&WB&ABK&ABB&ABV&AnzB&TW&Zahlencode \textcolor{red}{$\boldsymbol{Grundtext}$} Umschrift $|$"Ubersetzung(en)\\
1.&25.&974.&95.&4047.&1.&3&360&300\_50\_10 \textcolor{red}{\textcjheb{yn+s}} SNJ $|$beide(n)\\
2.&26.&975.&98.&4050.&4.&4&334&300\_4\_10\_20 \textcolor{red}{\textcjheb{kyd+s}} SDJK $|$deine Br"uste\\
3.&27.&976.&102.&4054.&8.&4&380&20\_300\_50\_10 \textcolor{red}{\textcjheb{yn+sk}} KSNJ $|$(sind) wie (zwei)\\
4.&28.&977.&106.&4058.&12.&5&400&70\_80\_200\_10\_40 \textcolor{red}{\textcjheb{myrp`}} aPRJM $|$junge(r)/Jungrehe\\
5.&29.&978.&111.&4063.&17.&4&451&400\_1\_40\_10 \textcolor{red}{\textcjheb{ym't}} TAMJ $|$ein Zwillingspaar/Zwillinge\\
6.&30.&979.&115.&4067.&21.&4&107&90\_2\_10\_5 \textcolor{red}{\textcjheb{hyb.s}} "sBJH $|$Gazellen/der Gazelle\\
\end{tabular}\medskip \\
Ende des Verses 7.3\\
Verse: 93, Buchstaben: 24, 118, 4070, Totalwerte: 2032, 7220, 280777\\
\\
Deine beiden Br"uste sind wie ein Zwillingspaar junger Gazellen.\\
\newpage 
{\bf -- 7.4 (7.5)}\\
\medskip \\
\begin{tabular}{rrrrrrrrp{120mm}}
WV&WK&WB&ABK&ABB&ABV&AnzB&TW&Zahlencode \textcolor{red}{$\boldsymbol{Grundtext}$} Umschrift $|$"Ubersetzung(en)\\
1.&31.&980.&119.&4071.&1.&5&317&90\_6\_1\_200\_20 \textcolor{red}{\textcjheb{kr'w.s}} "sWARK $|$dein Hals\\
2.&32.&981.&124.&4076.&6.&5&97&20\_40\_3\_4\_30 \textcolor{red}{\textcjheb{ldgmk}} KMGDL $|$(ist) wie ein Turm\\
3.&33.&982.&129.&4081.&11.&3&355&5\_300\_50 \textcolor{red}{\textcjheb{n+sh}} HSN $|$(von) Elfenbein\\
4.&34.&983.&132.&4084.&14.&5&160&70\_10\_50\_10\_20 \textcolor{red}{\textcjheb{kyny`}} aJNJK $|$deine Augen\\
5.&35.&984.&137.&4089.&19.&5&628&2\_200\_20\_6\_400 \textcolor{red}{\textcjheb{twkrb}} BRKWT $|$wie die Teiche/gleichen den Teichen\\
6.&36.&985.&142.&4094.&24.&6&368&2\_8\_300\_2\_6\_50 \textcolor{red}{\textcjheb{nwb+s.hb}} BCSBWN $|$zu Hesbon/in Heschbon//$<$Klugheit$>$\\
7.&37.&986.&148.&4100.&30.&2&100&70\_30 \textcolor{red}{\textcjheb{l`}} aL $|$am\\
8.&38.&987.&150.&4102.&32.&3&570&300\_70\_200 \textcolor{red}{\textcjheb{r`+s}} SaR $|$Tor\\
9.&39.&988.&153.&4105.&35.&2&402&2\_400 \textcolor{red}{\textcjheb{tb}} BT $|$der Stadt/(von) Bat//$<$Tochter$>$\\
10.&40.&989.&155.&4107.&37.&4&252&200\_2\_10\_40 \textcolor{red}{\textcjheb{mybr}} RBJM $|$volkreichen/Rabbim///$<$der Vielen$>$\\
11.&41.&990.&159.&4111.&41.&3&101&1\_80\_20 \textcolor{red}{\textcjheb{kp'}} APK $|$deine Nase\\
12.&42.&991.&162.&4114.&44.&5&97&20\_40\_3\_4\_30 \textcolor{red}{\textcjheb{ldgmk}} KMGDL $|$(ist) wie der Turm\\
13.&43.&992.&167.&4119.&49.&6&143&5\_30\_2\_50\_6\_50 \textcolor{red}{\textcjheb{nwnblh}} HLBNWN $|$(des) Libanon\\
14.&44.&993.&173.&4125.&55.&4&181&90\_6\_80\_5 \textcolor{red}{\textcjheb{hpw.s}} "sWPH $|$der hinschaut/schauend(er)\\
15.&45.&994.&177.&4129.&59.&3&140&80\_50\_10 \textcolor{red}{\textcjheb{ynp}} PNJ $|$nach/gegen\\
16.&46.&995.&180.&4132.&62.&4&444&4\_40\_300\_100 \textcolor{red}{\textcjheb{q+smd}} DMSQ $|$Damaskus///$<$Betriebsamkeit$>$\\
\end{tabular}\medskip \\
Ende des Verses 7.4\\
Verse: 94, Buchstaben: 65, 183, 4135, Totalwerte: 4355, 11575, 285132\\
\\
Dein Hals ist wie ein Turm von Elfenbein; deine Augen wie die Teiche zu Hesbon am Tore der volkreichen Stadt; deine Nase wie der Libanon-Turm, der nach Damaskus hinschaut.\\
\newpage 
{\bf -- 7.5 (7.6)}\\
\medskip \\
\begin{tabular}{rrrrrrrrp{120mm}}
WV&WK&WB&ABK&ABB&ABV&AnzB&TW&Zahlencode \textcolor{red}{$\boldsymbol{Grundtext}$} Umschrift $|$"Ubersetzung(en)\\
1.&47.&996.&184.&4136.&1.&4&521&200\_1\_300\_20 \textcolor{red}{\textcjheb{k+s'r}} RASK $|$dein Haupt\\
2.&48.&997.&188.&4140.&5.&4&130&70\_30\_10\_20 \textcolor{red}{\textcjheb{kyl`}} aLJK $|$auf dir\\
3.&49.&998.&192.&4144.&9.&5&310&20\_20\_200\_40\_30 \textcolor{red}{\textcjheb{lmrkk}} KKRML $|$(ist) wie der Karmel///$<$Fruchtgarten$>$\\
4.&50.&999.&197.&4149.&14.&4&440&6\_4\_30\_400 \textcolor{red}{\textcjheb{tldw}} WDLT $|$und das herabwallende Haar/und das Geflecht\\
5.&51.&1000.&201.&4153.&18.&4&521&200\_1\_300\_20 \textcolor{red}{\textcjheb{k+s'r}} RASK $|$deines Haupts\\
6.&52.&1001.&205.&4157.&22.&6&314&20\_1\_200\_3\_40\_50 \textcolor{red}{\textcjheb{nmgr'k}} KARGMN $|$wie Purpur(wolle)\\
7.&53.&1002.&211.&4163.&28.&3&90&40\_30\_20 \textcolor{red}{\textcjheb{klm}} MLK $|$(ein) K"onig\\
8.&54.&1003.&214.&4166.&31.&4&267&1\_60\_6\_200 \textcolor{red}{\textcjheb{rws'}} AsWR $|$ist gefesselt/(liegt) gefesselt\\
9.&55.&1004.&218.&4170.&35.&6&266&2\_200\_5\_9\_10\_40 \textcolor{red}{\textcjheb{my.thrb}} BRHtJM $|$durch deine Locken/in den Schlingen\\
\end{tabular}\medskip \\
Ende des Verses 7.5\\
Verse: 95, Buchstaben: 40, 223, 4175, Totalwerte: 2859, 14434, 287991\\
\\
Dein Haupt auf dir ist wie der Karmel, und das herabwallende Haar deines Hauptes wie Purpur: ein K"onig ist gefesselt durch deine Locken!\\
\newpage 
{\bf -- 7.6 (7.7)}\\
\medskip \\
\begin{tabular}{rrrrrrrrp{120mm}}
WV&WK&WB&ABK&ABB&ABV&AnzB&TW&Zahlencode \textcolor{red}{$\boldsymbol{Grundtext}$} Umschrift $|$"Ubersetzung(en)\\
1.&56.&1005.&224.&4176.&1.&2&45&40\_5 \textcolor{red}{\textcjheb{hm}} MH $|$wie\\
2.&57.&1006.&226.&4178.&3.&4&500&10\_80\_10\_400 \textcolor{red}{\textcjheb{typy}} JPJT $|$du bist sch"on\\
3.&58.&1007.&230.&4182.&7.&3&51&6\_40\_5 \textcolor{red}{\textcjheb{hmw}} WMH $|$und wie\\
4.&59.&1008.&233.&4185.&10.&4&560&50\_70\_40\_400 \textcolor{red}{\textcjheb{tm`n}} NaMT $|$du bist lieblich\\
5.&60.&1009.&237.&4189.&14.&4&13&1\_5\_2\_5 \textcolor{red}{\textcjheb{hbh'}} AHBH $|$(o) Liebe\\
6.&61.&1010.&241.&4193.&18.&8&581&2\_400\_70\_50\_6\_3\_10\_40 \textcolor{red}{\textcjheb{mygwn`tb}} BTaNWGJM $|$unter den Wonnen/in den Wonnen\\
\end{tabular}\medskip \\
Ende des Verses 7.6\\
Verse: 96, Buchstaben: 25, 248, 4200, Totalwerte: 1750, 16184, 289741\\
\\
Wie sch"on bist du, und wie lieblich bist du, o Liebe, unter den Wonnen!\\
\newpage 
{\bf -- 7.7 (7.8)}\\
\medskip \\
\begin{tabular}{rrrrrrrrp{120mm}}
WV&WK&WB&ABK&ABB&ABV&AnzB&TW&Zahlencode \textcolor{red}{$\boldsymbol{Grundtext}$} Umschrift $|$"Ubersetzung(en)\\
1.&62.&1011.&249.&4201.&1.&3&408&7\_1\_400 \textcolor{red}{\textcjheb{t'z}} ZAT $|$dieser\\
2.&63.&1012.&252.&4204.&4.&5&566&100\_6\_40\_400\_20 \textcolor{red}{\textcjheb{ktmwq}} QWMTK $|$dein Wuchs\\
3.&64.&1013.&257.&4209.&9.&4&449&4\_40\_400\_5 \textcolor{red}{\textcjheb{htmd}} DMTH $|$(sie (=er)) gleicht\\
4.&65.&1014.&261.&4213.&13.&4&670&30\_400\_40\_200 \textcolor{red}{\textcjheb{rmtl}} LTMR $|$der Palme/einer Palme\\
5.&66.&1015.&265.&4217.&17.&5&340&6\_300\_4\_10\_20 \textcolor{red}{\textcjheb{kyd+sw}} WSDJK $|$und deine Br"uste\\
6.&67.&1016.&270.&4222.&22.&7&787&30\_1\_300\_20\_30\_6\_400 \textcolor{red}{\textcjheb{twlk+s'l}} LASKLWT $|$den Trauben\\
\end{tabular}\medskip \\
Ende des Verses 7.7\\
Verse: 97, Buchstaben: 28, 276, 4228, Totalwerte: 3220, 19404, 292961\\
\\
Dieser dein Wuchs gleicht der Palme, und deine Br"uste den Trauben.\\
\newpage 
{\bf -- 7.8 (7.9)}\\
\medskip \\
\begin{tabular}{rrrrrrrrp{120mm}}
WV&WK&WB&ABK&ABB&ABV&AnzB&TW&Zahlencode \textcolor{red}{$\boldsymbol{Grundtext}$} Umschrift $|$"Ubersetzung(en)\\
1.&68.&1017.&277.&4229.&1.&5&651&1\_40\_200\_400\_10 \textcolor{red}{\textcjheb{ytrm'}} AMRTJ $|$ich sprach\\
2.&69.&1018.&282.&4234.&6.&4&106&1\_70\_30\_5 \textcolor{red}{\textcjheb{hl`'}} AaLH $|$ich will ersteigen/ich will hinaufsteigen\\
3.&70.&1019.&286.&4238.&10.&4&642&2\_400\_40\_200 \textcolor{red}{\textcjheb{rmtb}} BTMR $|$(auf) die Palme\\
4.&71.&1020.&290.&4242.&14.&4&21&1\_8\_7\_5 \textcolor{red}{\textcjheb{hz.h'}} ACZH $|$will erfassen/ich will greifen\\
5.&72.&1021.&294.&4246.&18.&7&238&2\_60\_50\_60\_50\_10\_6 \textcolor{red}{\textcjheb{wynsnsb}} BsNsNJW $|$ihre Zweige/nach ihren Rispen\\
6.&73.&1022.&301.&4253.&25.&5&37&6\_10\_5\_10\_6 \textcolor{red}{\textcjheb{wyhyw}} WJHJW $|$und (es) sollen sein mir/und sie (=es) sind\\
7.&74.&1023.&306.&4258.&30.&2&51&50\_1 \textcolor{red}{\textcjheb{'n}} NA $|$/doch\\
8.&75.&1024.&308.&4260.&32.&4&334&300\_4\_10\_20 \textcolor{red}{\textcjheb{kyd+s}} SDJK $|$deine Br"uste\\
9.&76.&1025.&312.&4264.&36.&7&777&20\_1\_300\_20\_30\_6\_400 \textcolor{red}{\textcjheb{twlk+s'k}} KASKLWT $|$wie Trauben\\
10.&77.&1026.&319.&4271.&43.&4&138&5\_3\_80\_50 \textcolor{red}{\textcjheb{npgh}} HGPN $|$des Weinstocks\\
11.&78.&1027.&323.&4275.&47.&4&224&6\_200\_10\_8 \textcolor{red}{\textcjheb{.hyrw}} WRJC $|$und der Duft\\
12.&79.&1028.&327.&4279.&51.&3&101&1\_80\_20 \textcolor{red}{\textcjheb{kp'}} APK $|$deiner Nase\\
13.&80.&1029.&330.&4282.&54.&7&564&20\_400\_80\_6\_8\_10\_40 \textcolor{red}{\textcjheb{my.hwptk}} KTPWCJM $|$wie "Apfel/wie (Duft) von "Apfeln\\
\end{tabular}\medskip \\
Ende des Verses 7.8\\
Verse: 98, Buchstaben: 60, 336, 4288, Totalwerte: 3884, 23288, 296845\\
\\
Ich sprach: Ich will die Palme ersteigen, will ihre Zweige erfassen; und deine Br"uste sollen mir sein wie Trauben des Weinstocks, und der Duft deiner Nase wie "Apfel,\\
\newpage 
{\bf -- 7.9 (7.10)}\\
\medskip \\
\begin{tabular}{rrrrrrrrp{120mm}}
WV&WK&WB&ABK&ABB&ABV&AnzB&TW&Zahlencode \textcolor{red}{$\boldsymbol{Grundtext}$} Umschrift $|$"Ubersetzung(en)\\
1.&81.&1030.&337.&4289.&1.&4&54&6\_8\_20\_20 \textcolor{red}{\textcjheb{kk.hw}} WCKK $|$und dein Gaumen\\
2.&82.&1031.&341.&4293.&5.&4&90&20\_10\_10\_50 \textcolor{red}{\textcjheb{nyyk}} KJJN $|$wie Wein\\
3.&83.&1032.&345.&4297.&9.&4&22&5\_9\_6\_2 \textcolor{red}{\textcjheb{bw.th}} HtWB $|$der beste/dem guten\\
4.&84.&1033.&349.&4301.&13.&4&61&5\_6\_30\_20 \textcolor{red}{\textcjheb{klwh}} HWLK $|$der sanft hinuntergleitet/hingleitend\\
5.&85.&1034.&353.&4305.&17.&5&54&30\_4\_6\_4\_10 \textcolor{red}{\textcjheb{ydwdl}} LDWDJ $|$meinem Geliebten/zu meinem Freund\\
6.&86.&1035.&358.&4310.&22.&7&630&30\_40\_10\_300\_200\_10\_40 \textcolor{red}{\textcjheb{myr+syml}} LMJSRJM $|$/zu Geradheiten\\
7.&87.&1036.&365.&4317.&29.&4&14&4\_6\_2\_2 \textcolor{red}{\textcjheb{bbwd}} DWBB $|$der schleicht/flie"send\\
8.&88.&1037.&369.&4321.&33.&4&790&300\_80\_400\_10 \textcolor{red}{\textcjheb{ytp+s}} SPTJ $|$("uber) (die) Lippen\\
9.&89.&1038.&373.&4325.&37.&5&410&10\_300\_50\_10\_40 \textcolor{red}{\textcjheb{myn+sy}} JSNJM $|$der Schlummernden/(von) Schlafende(n)\\
\end{tabular}\medskip \\
Ende des Verses 7.9\\
Verse: 99, Buchstaben: 41, 377, 4329, Totalwerte: 2125, 25413, 298970\\
\\
und dein Gaumen wie der beste Wein, ...der meinem Geliebten sanft hinuntergleitet, der "uber die Lippen der Schlummernden schleicht.\\
\newpage 
{\bf -- 7.10 (7.11)}\\
\medskip \\
\begin{tabular}{rrrrrrrrp{120mm}}
WV&WK&WB&ABK&ABB&ABV&AnzB&TW&Zahlencode \textcolor{red}{$\boldsymbol{Grundtext}$} Umschrift $|$"Ubersetzung(en)\\
1.&90.&1039.&378.&4330.&1.&3&61&1\_50\_10 \textcolor{red}{\textcjheb{yn'}} ANJ $|$ich (bin)\\
2.&91.&1040.&381.&4333.&4.&5&54&30\_4\_6\_4\_10 \textcolor{red}{\textcjheb{ydwdl}} LDWDJ $|$meines Geliebten/geh"ore meinem Freund\\
3.&92.&1041.&386.&4338.&9.&4&116&6\_70\_30\_10 \textcolor{red}{\textcjheb{yl`w}} WaLJ $|$und nach mir\\
4.&93.&1042.&390.&4342.&13.&6&1212&400\_300\_6\_100\_400\_6 \textcolor{red}{\textcjheb{wtqw+st}} TSWQTW $|$ist sein Verlangen/(steht) sein Verlangen\\
\end{tabular}\medskip \\
Ende des Verses 7.10\\
Verse: 100, Buchstaben: 18, 395, 4347, Totalwerte: 1443, 26856, 300413\\
\\
Ich bin meines Geliebten und nach mir ist sein Verlangen.\\
\newpage 
{\bf -- 7.11 (7.12)}\\
\medskip \\
\begin{tabular}{rrrrrrrrp{120mm}}
WV&WK&WB&ABK&ABB&ABV&AnzB&TW&Zahlencode \textcolor{red}{$\boldsymbol{Grundtext}$} Umschrift $|$"Ubersetzung(en)\\
1.&94.&1043.&396.&4348.&1.&3&55&30\_20\_5 \textcolor{red}{\textcjheb{hkl}} LKH $|$komm\\
2.&95.&1044.&399.&4351.&4.&4&24&4\_6\_4\_10 \textcolor{red}{\textcjheb{ydwd}} DWDJ $|$mein Geliebter/mein Freund\\
3.&96.&1045.&403.&4355.&8.&3&141&50\_90\_1 \textcolor{red}{\textcjheb{'.sn}} N"sA $|$lass uns hinausgehen/wir gehen hinaus\\
4.&97.&1046.&406.&4358.&11.&4&314&5\_300\_4\_5 \textcolor{red}{\textcjheb{hd+sh}} HSDH $|$(auf) das Feld\\
5.&98.&1047.&410.&4362.&15.&5&145&50\_30\_10\_50\_5 \textcolor{red}{\textcjheb{hnyln}} NLJNH $|$"ubernachten/wir wollen n"achtigen\\
6.&99.&1048.&415.&4367.&20.&6&352&2\_20\_80\_200\_10\_40 \textcolor{red}{\textcjheb{myrpkb}} BKPRJM $|$in den D"orfern/unter den Hennab"aumen\\
\end{tabular}\medskip \\
Ende des Verses 7.11\\
Verse: 101, Buchstaben: 25, 420, 4372, Totalwerte: 1031, 27887, 301444\\
\\
Komm, mein Geliebter, la"s uns aufs Feld hinausgehen, in den D"orfern "ubernachten.\\
\newpage 
{\bf -- 7.12 (7.13)}\\
\medskip \\
\begin{tabular}{rrrrrrrrp{120mm}}
WV&WK&WB&ABK&ABB&ABV&AnzB&TW&Zahlencode \textcolor{red}{$\boldsymbol{Grundtext}$} Umschrift $|$"Ubersetzung(en)\\
1.&100.&1049.&421.&4373.&1.&6&425&50\_300\_20\_10\_40\_5 \textcolor{red}{\textcjheb{hmyk+sn}} NSKJMH $|$wir wollen uns fr"uh aufmachen/wir wollen hinausgehen\\
2.&101.&1050.&427.&4379.&7.&6&340&30\_20\_200\_40\_10\_40 \textcolor{red}{\textcjheb{mymrkl}} LKRMJM $|$nach den Weinbergen/in die Weing"arten\\
3.&102.&1051.&433.&4385.&13.&4&256&50\_200\_1\_5 \textcolor{red}{\textcjheb{h'rn}} NRAH $|$wollen sehen/wir wollen schauen\\
4.&103.&1052.&437.&4389.&17.&2&41&1\_40 \textcolor{red}{\textcjheb{m'}} AM $|$ob\\
5.&104.&1053.&439.&4391.&19.&4&293&80\_200\_8\_5 \textcolor{red}{\textcjheb{h.hrp}} PRCH $|$ausgeschlagen ist/(sie) gesprosst\\
6.&105.&1054.&443.&4395.&23.&4&138&5\_3\_80\_50 \textcolor{red}{\textcjheb{npgh}} HGPN $|$der Weinstock/die Weinrebe\\
7.&106.&1055.&447.&4399.&27.&3&488&80\_400\_8 \textcolor{red}{\textcjheb{.htp}} PTC $|$sich ge"offnet hat/er (=es) "offnete (sich)\\
8.&107.&1056.&450.&4402.&30.&5&309&5\_60\_40\_4\_200 \textcolor{red}{\textcjheb{rdmsh}} HsMDR $|$die Weinbl"ute/die Knospenh"ulle\\
9.&108.&1057.&455.&4407.&35.&4&151&5\_50\_90\_6 \textcolor{red}{\textcjheb{w.snh}} HN"sW $|$ob bl"uhen/(sie) Bl"uten machten\\
10.&109.&1058.&459.&4411.&39.&7&351&5\_200\_40\_6\_50\_10\_40 \textcolor{red}{\textcjheb{mynwmrh}} HRMWNJM $|$die Granaten/die Granat"apfel(b"aume)\\
11.&110.&1059.&466.&4418.&46.&2&340&300\_40 \textcolor{red}{\textcjheb{m+s}} SM $|$dort\\
12.&111.&1060.&468.&4420.&48.&3&451&1\_400\_50 \textcolor{red}{\textcjheb{nt'}} ATN $|$ich will geben\\
13.&112.&1061.&471.&4423.&51.&2&401&1\_400 \textcolor{red}{\textcjheb{t'}} AT $|$**\\
14.&113.&1062.&473.&4425.&53.&3&18&4\_4\_10 \textcolor{red}{\textcjheb{ydd}} DDJ $|$meine Liebe/meine Liebkosungen\\
15.&114.&1063.&476.&4428.&56.&2&50&30\_20 \textcolor{red}{\textcjheb{kl}} LK $|$(zu) dir\\
\end{tabular}\medskip \\
Ende des Verses 7.12\\
Verse: 102, Buchstaben: 57, 477, 4429, Totalwerte: 4052, 31939, 305496\\
\\
Wir wollen uns fr"uh aufmachen nach den Weinbergen, wollen sehen, ob der Weinstock ausgeschlagen ist, die Weinbl"ute sich ge"offnet hat, ob die Granaten bl"uhen; dort will ich dir meine Liebe geben.\\
\newpage 
{\bf -- 7.13 (7.14)}\\
\medskip \\
\begin{tabular}{rrrrrrrrp{120mm}}
WV&WK&WB&ABK&ABB&ABV&AnzB&TW&Zahlencode \textcolor{red}{$\boldsymbol{Grundtext}$} Umschrift $|$"Ubersetzung(en)\\
1.&115.&1064.&478.&4430.&1.&7&70&5\_4\_6\_4\_1\_10\_40 \textcolor{red}{\textcjheb{my'dwdh}} HDWDAJM $|$die Liebes"apfel\\
2.&116.&1065.&485.&4437.&8.&4&506&50\_400\_50\_6 \textcolor{red}{\textcjheb{wntn}} NTNW $|$/sie (gaben)\\
3.&117.&1066.&489.&4441.&12.&3&218&200\_10\_8 \textcolor{red}{\textcjheb{.hyr}} RJC $|$duften/(Wohl)Geruch\\
4.&118.&1067.&492.&4444.&15.&3&106&6\_70\_30 \textcolor{red}{\textcjheb{l`w}} WaL $|$und "uber/und vor\\
5.&119.&1068.&495.&4447.&18.&6&554&80\_400\_8\_10\_50\_6 \textcolor{red}{\textcjheb{wny.htp}} PTCJNW $|$unseren T"uren\\
6.&120.&1069.&501.&4453.&24.&2&50&20\_30 \textcolor{red}{\textcjheb{lk}} KL $|$(sind) allerlei\\
7.&121.&1070.&503.&4455.&26.&5&97&40\_3\_4\_10\_40 \textcolor{red}{\textcjheb{mydgm}} MGDJM $|$edle Fr"uchte/K"ostlichkeiten\\
8.&122.&1071.&508.&4460.&31.&5&362&8\_4\_300\_10\_40 \textcolor{red}{\textcjheb{my+sd.h}} CDSJM $|$neue\\
9.&123.&1072.&513.&4465.&36.&2&43&3\_40 \textcolor{red}{\textcjheb{mg}} GM $|$und/(als) auch\\
10.&124.&1073.&515.&4467.&38.&5&410&10\_300\_50\_10\_40 \textcolor{red}{\textcjheb{myn+sy}} JSNJM $|$alte/(vor)j"ahrige\\
11.&125.&1074.&520.&4472.&43.&4&24&4\_6\_4\_10 \textcolor{red}{\textcjheb{ydwd}} DWDJ $|$mein Geliebter/mein Freund\\
12.&126.&1075.&524.&4476.&47.&5&630&90\_80\_50\_400\_10 \textcolor{red}{\textcjheb{ytnp.s}} "sPNTJ $|$die ich aufbewahrt habe/ich bewahrte (auf)\\
13.&127.&1076.&529.&4481.&52.&2&50&30\_20 \textcolor{red}{\textcjheb{kl}} LK $|$dir/f"ur dich\\
\end{tabular}\medskip \\
Ende des Verses 7.13\\
Verse: 103, Buchstaben: 53, 530, 4482, Totalwerte: 3120, 35059, 308616\\
\\
Die Liebes"apfel duften, und "uber unseren T"uren sind allerlei edle Fr"uchte, neue und alte, die ich, mein Geliebter, dir aufbewahrt habe.\\
\\
{\bf Ende des Kapitels 7}\\
\newpage 
{\bf -- 8.1}\\
\medskip \\
\begin{tabular}{rrrrrrrrp{120mm}}
WV&WK&WB&ABK&ABB&ABV&AnzB&TW&Zahlencode \textcolor{red}{$\boldsymbol{Grundtext}$} Umschrift $|$"Ubersetzung(en)\\
1.&1.&1077.&1.&4483.&1.&2&50&40\_10 \textcolor{red}{\textcjheb{ym}} MJ $|$/wer\\
2.&2.&1078.&3.&4485.&3.&4&480&10\_400\_50\_20 \textcolor{red}{\textcjheb{knty}} JTNK $|$(o) w"arest du (doch)\\
3.&3.&1079.&7.&4489.&7.&3&29&20\_1\_8 \textcolor{red}{\textcjheb{.h'k}} KAC $|$gleich einem Bruder/wie ein Bruder\\
4.&4.&1080.&10.&4492.&10.&2&40&30\_10 \textcolor{red}{\textcjheb{yl}} LJ $|$(zu) mir\\
5.&5.&1081.&12.&4494.&12.&4&166&10\_6\_50\_100 \textcolor{red}{\textcjheb{qnwy}} JWNQ $|$der gesogen/saugend(er)\\
6.&6.&1082.&16.&4498.&16.&3&314&300\_4\_10 \textcolor{red}{\textcjheb{yd+s}} SDJ $|$die Br"uste\\
7.&7.&1083.&19.&4501.&19.&3&51&1\_40\_10 \textcolor{red}{\textcjheb{ym'}} AMJ $|$meiner Mutter\\
8.&8.&1084.&22.&4504.&22.&5&152&1\_40\_90\_1\_20 \textcolor{red}{\textcjheb{k'.sm'}} AM"sAK $|$f"ande ich dich/ich werde finden dich\\
9.&9.&1085.&27.&4509.&27.&4&106&2\_8\_6\_90 \textcolor{red}{\textcjheb{.sw.hb}} BCW"s $|$drau"sen/im Drau"sen\\
10.&10.&1086.&31.&4513.&31.&4&421&1\_300\_100\_20 \textcolor{red}{\textcjheb{kq+s'}} ASQK $|$ich wollte dich k"ussen/ich w"urde dich k"ussen\\
11.&11.&1087.&35.&4517.&35.&2&43&3\_40 \textcolor{red}{\textcjheb{mg}} GM $|$und/auch\\
12.&12.&1088.&37.&4519.&37.&2&31&30\_1 \textcolor{red}{\textcjheb{'l}} LA $|$nicht\\
13.&13.&1089.&39.&4521.&39.&5&31&10\_2\_6\_7\_6 \textcolor{red}{\textcjheb{wzwby}} JBWZW $|$man w"urde verachten/sie w"urden verachten\\
14.&14.&1090.&44.&4526.&44.&2&40&30\_10 \textcolor{red}{\textcjheb{yl}} LJ $|$mich\\
\end{tabular}\medskip \\
Ende des Verses 8.1\\
Verse: 104, Buchstaben: 45, 45, 4527, Totalwerte: 1954, 1954, 310570\\
\\
O w"arest du mir gleich einem Bruder, der die Br"uste meiner Mutter gesogen! F"ande ich dich drau"sen, ich wollte dich k"ussen; und man w"urde mich nicht verachten.\\
\newpage 
{\bf -- 8.2}\\
\medskip \\
\begin{tabular}{rrrrrrrrp{120mm}}
WV&WK&WB&ABK&ABB&ABV&AnzB&TW&Zahlencode \textcolor{red}{$\boldsymbol{Grundtext}$} Umschrift $|$"Ubersetzung(en)\\
1.&15.&1091.&46.&4528.&1.&5&79&1\_50\_5\_3\_20 \textcolor{red}{\textcjheb{kghn'}} ANHGK $|$ich w"urde f"uhren dich\\
2.&16.&1092.&51.&4533.&6.&5&34&1\_2\_10\_1\_20 \textcolor{red}{\textcjheb{k'yb'}} ABJAK $|$dich hineinbringen/ich br"achte dich\\
3.&17.&1093.&56.&4538.&11.&2&31&1\_30 \textcolor{red}{\textcjheb{l'}} AL $|$in\\
4.&18.&1094.&58.&4540.&13.&3&412&2\_10\_400 \textcolor{red}{\textcjheb{tyb}} BJT $|$das Haus\\
5.&19.&1095.&61.&4543.&16.&3&51&1\_40\_10 \textcolor{red}{\textcjheb{ym'}} AMJ $|$meiner Mutter\\
6.&20.&1096.&64.&4546.&19.&6&534&400\_30\_40\_4\_50\_10 \textcolor{red}{\textcjheb{yndmlt}} TLMDNJ $|$du w"urdest belehren mich/du solltest belehren mich\\
7.&21.&1097.&70.&4552.&25.&4&421&1\_300\_100\_20 \textcolor{red}{\textcjheb{kq+s'}} ASQK $|$ich w"urde dich tr"anken/ich tr"ankte dich\\
8.&22.&1098.&74.&4556.&29.&4&110&40\_10\_10\_50 \textcolor{red}{\textcjheb{nyym}} MJJN $|$mit Wein\\
9.&23.&1099.&78.&4560.&33.&4&313&5\_200\_100\_8 \textcolor{red}{\textcjheb{.hqrh}} HRQC $|$(der) W"urz(e)\\
10.&24.&1100.&82.&4564.&37.&5&240&40\_70\_60\_10\_60 \textcolor{red}{\textcjheb{sys`m}} MasJs $|$mit dem Moste/von dem Most\\
11.&25.&1101.&87.&4569.&42.&4&300&200\_40\_50\_10 \textcolor{red}{\textcjheb{ynmr}} RMNJ $|$meiner Granaten/meines Granatapfel(baums)\\
\end{tabular}\medskip \\
Ende des Verses 8.2\\
Verse: 105, Buchstaben: 45, 90, 4572, Totalwerte: 2525, 4479, 313095\\
\\
Ich w"urde dich f"uhren, dich hineinbringen in meiner Mutter Haus, du w"urdest mich belehren; ich w"urde dich tr"anken mit W"urzwein, mit dem Moste meiner Granaten. -\\
\newpage 
{\bf -- 8.3}\\
\medskip \\
\begin{tabular}{rrrrrrrrp{120mm}}
WV&WK&WB&ABK&ABB&ABV&AnzB&TW&Zahlencode \textcolor{red}{$\boldsymbol{Grundtext}$} Umschrift $|$"Ubersetzung(en)\\
1.&26.&1102.&91.&4573.&1.&5&377&300\_40\_1\_30\_6 \textcolor{red}{\textcjheb{wl'm+s}} SMALW $|$seine Linke (l"age)\\
2.&27.&1103.&96.&4578.&6.&3&808&400\_8\_400 \textcolor{red}{\textcjheb{t.ht}} TCT $|$(sei) unter\\
3.&28.&1104.&99.&4581.&9.&4&511&200\_1\_300\_10 \textcolor{red}{\textcjheb{y+s'r}} RASJ $|$meinem Haupt\\
4.&29.&1105.&103.&4585.&13.&6&122&6\_10\_40\_10\_50\_6 \textcolor{red}{\textcjheb{wnymyw}} WJMJNW $|$und seine Rechte\\
5.&30.&1106.&109.&4591.&19.&6&570&400\_8\_2\_100\_50\_10 \textcolor{red}{\textcjheb{ynqb.ht}} TCBQNJ $|$umfasse mich/(sie) umfinge mich\\
\end{tabular}\medskip \\
Ende des Verses 8.3\\
Verse: 106, Buchstaben: 24, 114, 4596, Totalwerte: 2388, 6867, 315483\\
\\
Seine Linke sei unter meinem Haupte, und seine Rechte umfasse mich.\\
\newpage 
{\bf -- 8.4}\\
\medskip \\
\begin{tabular}{rrrrrrrrp{120mm}}
WV&WK&WB&ABK&ABB&ABV&AnzB&TW&Zahlencode \textcolor{red}{$\boldsymbol{Grundtext}$} Umschrift $|$"Ubersetzung(en)\\
1.&31.&1107.&115.&4597.&1.&6&787&5\_300\_2\_70\_400\_10 \textcolor{red}{\textcjheb{yt`b+sh}} HSBaTJ $|$ich beschw"ore\\
2.&32.&1108.&121.&4603.&7.&4&461&1\_400\_20\_40 \textcolor{red}{\textcjheb{mkt'}} ATKM $|$euch\\
3.&33.&1109.&125.&4607.&11.&4&458&2\_50\_6\_400 \textcolor{red}{\textcjheb{twnb}} BNWT $|$(ihr) T"ochter\\
4.&34.&1110.&129.&4611.&15.&6&586&10\_200\_6\_300\_30\_40 \textcolor{red}{\textcjheb{ml+swry}} JRWSLM $|$Jerusalem(s)\\
5.&35.&1111.&135.&4617.&21.&2&45&40\_5 \textcolor{red}{\textcjheb{hm}} MH $|$dass nicht/was\\
6.&36.&1112.&137.&4619.&23.&5&686&400\_70\_10\_200\_6 \textcolor{red}{\textcjheb{wry`t}} TaJRW $|$ihr wecket/ihr wollt wecken\\
7.&37.&1113.&142.&4624.&28.&3&51&6\_40\_5 \textcolor{red}{\textcjheb{hmw}} WMH $|$noch/und was\\
8.&38.&1114.&145.&4627.&31.&5&876&400\_70\_200\_200\_6 \textcolor{red}{\textcjheb{wrr`t}} TaRRW $|$aufwecket/ich wollt aufst"oren\\
9.&39.&1115.&150.&4632.&36.&2&401&1\_400 \textcolor{red}{\textcjheb{t'}} AT $|$**\\
10.&40.&1116.&152.&4634.&38.&5&18&5\_1\_5\_2\_5 \textcolor{red}{\textcjheb{hbh'h}} HAHBH $|$die Liebe\\
11.&41.&1117.&157.&4639.&43.&2&74&70\_4 \textcolor{red}{\textcjheb{d`}} aD $|$bis\\
12.&42.&1118.&159.&4641.&45.&5&878&300\_400\_8\_80\_90 \textcolor{red}{\textcjheb{.sp.ht+s}} STCP"s $|$(dass) es ihr gef"allt\\
\end{tabular}\medskip \\
Ende des Verses 8.4\\
Verse: 107, Buchstaben: 49, 163, 4645, Totalwerte: 5321, 12188, 320804\\
\\
Ich beschw"ore euch, T"ochter Jerusalems, da"s ihr nicht wecket noch aufwecket die Liebe, bis es ihr gef"allt!\\
\newpage 
{\bf -- 8.5}\\
\medskip \\
\begin{tabular}{rrrrrrrrp{120mm}}
WV&WK&WB&ABK&ABB&ABV&AnzB&TW&Zahlencode \textcolor{red}{$\boldsymbol{Grundtext}$} Umschrift $|$"Ubersetzung(en)\\
1.&43.&1119.&164.&4646.&1.&2&50&40\_10 \textcolor{red}{\textcjheb{ym}} MJ $|$wer\\
2.&44.&1120.&166.&4648.&3.&3&408&7\_1\_400 \textcolor{red}{\textcjheb{t'z}} ZAT $|$ist sie/die(se)\\
3.&45.&1121.&169.&4651.&6.&3&105&70\_30\_5 \textcolor{red}{\textcjheb{hl`}} aLH $|$die heraufkommt/Heraufziehende\\
4.&46.&1122.&172.&4654.&9.&2&90&40\_50 \textcolor{red}{\textcjheb{nm}} MN $|$von/aus\\
5.&47.&1123.&174.&4656.&11.&5&251&5\_40\_4\_2\_200 \textcolor{red}{\textcjheb{rbdmh}} HMDBR $|$der W"uste her/der Steppe\\
6.&48.&1124.&179.&4661.&16.&6&1220&40\_400\_200\_80\_100\_400 \textcolor{red}{\textcjheb{tqprtm}} MTRPQT $|$sich lehnend/sich anschmiegend\\
7.&49.&1125.&185.&4667.&22.&2&100&70\_30 \textcolor{red}{\textcjheb{l`}} aL $|$an\\
8.&50.&1126.&187.&4669.&24.&4&19&4\_6\_4\_5 \textcolor{red}{\textcjheb{hdwd}} DWDH $|$ihren Geliebten/ihren Freund\\
9.&51.&1127.&191.&4673.&28.&3&808&400\_8\_400 \textcolor{red}{\textcjheb{t.ht}} TCT $|$unter\\
10.&52.&1128.&194.&4676.&31.&5&499&5\_400\_80\_6\_8 \textcolor{red}{\textcjheb{.hwpth}} HTPWC $|$den Apfelbaum\\
11.&53.&1129.&199.&4681.&36.&7&906&70\_6\_200\_200\_400\_10\_20 \textcolor{red}{\textcjheb{kytrrw`}} aWRRTJK $|$habe ich dich geweckt/ich weckte dich\\
12.&54.&1130.&206.&4688.&43.&3&345&300\_40\_5 \textcolor{red}{\textcjheb{hm+s}} SMH $|$dort(hin)\\
13.&55.&1131.&209.&4691.&46.&5&460&8\_2\_30\_400\_20 \textcolor{red}{\textcjheb{ktlb.h}} CBLTK $|$hat Wehen gehabt mit dir/sie (=es) empfing dich\\
14.&56.&1132.&214.&4696.&51.&3&61&1\_40\_20 \textcolor{red}{\textcjheb{km'}} AMK $|$deine Mutter\\
15.&57.&1133.&217.&4699.&54.&3&345&300\_40\_5 \textcolor{red}{\textcjheb{hm+s}} SMH $|$dort(hin)\\
16.&58.&1134.&220.&4702.&57.&4&45&8\_2\_30\_5 \textcolor{red}{\textcjheb{hlb.h}} CBLH $|$hat Wehen gehabt/sie (=es) empfing\\
17.&59.&1135.&224.&4706.&61.&5&464&10\_30\_4\_400\_20 \textcolor{red}{\textcjheb{ktdly}} JLDTK $|$die dich geboren/deine Geb"arerin\\
\end{tabular}\medskip \\
Ende des Verses 8.5\\
Verse: 108, Buchstaben: 65, 228, 4710, Totalwerte: 6176, 18364, 326980\\
\\
Wer ist sie, die da heraufkommt von der W"uste her, sich lehnend auf ihren Geliebten? Unter dem Apfelbaume habe ich dich geweckt. Dort hat mit dir Wehen gehabt deine Mutter, dort hat Wehen gehabt, die dich geboren. -\\
\newpage 
{\bf -- 8.6}\\
\medskip \\
\begin{tabular}{rrrrrrrrp{120mm}}
WV&WK&WB&ABK&ABB&ABV&AnzB&TW&Zahlencode \textcolor{red}{$\boldsymbol{Grundtext}$} Umschrift $|$"Ubersetzung(en)\\
1.&60.&1136.&229.&4711.&1.&5&410&300\_10\_40\_50\_10 \textcolor{red}{\textcjheb{ynmy+s}} SJMNJ $|$lege mich\\
2.&61.&1137.&234.&4716.&6.&5&474&20\_8\_6\_400\_40 \textcolor{red}{\textcjheb{mtw.hk}} KCWTM $|$wie einen Siegelring/wie den Siegelring\\
3.&62.&1138.&239.&4721.&11.&2&100&70\_30 \textcolor{red}{\textcjheb{l`}} aL $|$an\\
4.&63.&1139.&241.&4723.&13.&3&52&30\_2\_20 \textcolor{red}{\textcjheb{kbl}} LBK $|$dein Herz\\
5.&64.&1140.&244.&4726.&16.&5&474&20\_8\_6\_400\_40 \textcolor{red}{\textcjheb{mtw.hk}} KCWTM $|$wie einen Siegelring/wie den Siegelring\\
6.&65.&1141.&249.&4731.&21.&2&100&70\_30 \textcolor{red}{\textcjheb{l`}} aL $|$an\\
7.&66.&1142.&251.&4733.&23.&5&303&7\_200\_6\_70\_20 \textcolor{red}{\textcjheb{k`wrz}} ZRWaK $|$deinen Arm\\
8.&67.&1143.&256.&4738.&28.&2&30&20\_10 \textcolor{red}{\textcjheb{yk}} KJ $|$denn\\
9.&68.&1144.&258.&4740.&30.&3&82&70\_7\_5 \textcolor{red}{\textcjheb{hz`}} aZH $|$gewaltsam ist/er (=es) ist stark\\
10.&69.&1145.&261.&4743.&33.&4&466&20\_40\_6\_400 \textcolor{red}{\textcjheb{twmk}} KMWT $|$wie der Tod\\
11.&70.&1146.&265.&4747.&37.&4&13&1\_5\_2\_5 \textcolor{red}{\textcjheb{hbh'}} AHBH $|$die Liebe\\
12.&71.&1147.&269.&4751.&41.&3&405&100\_300\_5 \textcolor{red}{\textcjheb{h+sq}} QSH $|$hart(e)\\
13.&72.&1148.&272.&4754.&44.&5&357&20\_300\_1\_6\_30 \textcolor{red}{\textcjheb{lw'+sk}} KSAWL $|$wie (der) Scheol\\
14.&73.&1149.&277.&4759.&49.&4&156&100\_50\_1\_5 \textcolor{red}{\textcjheb{h'nq}} QNAH $|$ihr Eifer/(die) Leidenschaft\\
15.&74.&1150.&281.&4763.&53.&5&595&200\_300\_80\_10\_5 \textcolor{red}{\textcjheb{hyp+sr}} RSPJH $|$ihre Gluten/ihre Br"ande\\
16.&75.&1151.&286.&4768.&58.&4&590&200\_300\_80\_10 \textcolor{red}{\textcjheb{yp+sr}} RSPJ $|$(sind) Gluten/(sind) Br"ande\\
17.&76.&1152.&290.&4772.&62.&2&301&1\_300 \textcolor{red}{\textcjheb{+s'}} AS $|$(des) Feuer(s)\\
18.&77.&1153.&292.&4774.&64.&7&752&300\_30\_5\_2\_400\_10\_5 \textcolor{red}{\textcjheb{hytbhl+s}} SLHBTJH $|$eine Flamme Jahs/welch(e) (sind) Flamme Jahwe(s)\\
\end{tabular}\medskip \\
Ende des Verses 8.6\\
Verse: 109, Buchstaben: 70, 298, 4780, Totalwerte: 5660, 24024, 332640\\
\\
Lege mich wie einen Siegelring an dein Herz, wie einen Siegelring an deinen Arm! Denn die Liebe ist gewaltsam wie der Tod, hart wie der Scheol ihr Eifer; ihre Gluten sind Feuergluten, eine Flamme Jahs.\\
\newpage 
{\bf -- 8.7}\\
\medskip \\
\begin{tabular}{rrrrrrrrp{120mm}}
WV&WK&WB&ABK&ABB&ABV&AnzB&TW&Zahlencode \textcolor{red}{$\boldsymbol{Grundtext}$} Umschrift $|$"Ubersetzung(en)\\
1.&78.&1154.&299.&4781.&1.&3&90&40\_10\_40 \textcolor{red}{\textcjheb{mym}} MJM $|$Wasser\\
2.&79.&1155.&302.&4784.&4.&4&252&200\_2\_10\_40 \textcolor{red}{\textcjheb{mybr}} RBJM $|$gro"se/viele\\
3.&80.&1156.&306.&4788.&8.&2&31&30\_1 \textcolor{red}{\textcjheb{'l}} LA $|$nicht\\
4.&81.&1157.&308.&4790.&10.&5&72&10\_6\_20\_30\_6 \textcolor{red}{\textcjheb{wlkwy}} JWKLW $|$(sie) verm"ogen\\
5.&82.&1158.&313.&4795.&15.&5&458&30\_20\_2\_6\_400 \textcolor{red}{\textcjheb{twbkl}} LKBWT $|$(aus) zu l"oschen\\
6.&83.&1159.&318.&4800.&20.&2&401&1\_400 \textcolor{red}{\textcjheb{t'}} AT $|$**\\
7.&84.&1160.&320.&4802.&22.&5&18&5\_1\_5\_2\_5 \textcolor{red}{\textcjheb{hbh'h}} HAHBH $|$die Liebe\\
8.&85.&1161.&325.&4807.&27.&6&667&6\_50\_5\_200\_6\_400 \textcolor{red}{\textcjheb{twrhnw}} WNHRWT $|$und Str"ome\\
9.&86.&1162.&331.&4813.&33.&2&31&30\_1 \textcolor{red}{\textcjheb{'l}} LA $|$nicht\\
10.&87.&1163.&333.&4815.&35.&6&410&10\_300\_9\_80\_6\_5 \textcolor{red}{\textcjheb{hwp.t+sy}} JStPWH $|$"uberfluten sie/sie k"onnen fortschwemmen sie\\
11.&88.&1164.&339.&4821.&41.&2&41&1\_40 \textcolor{red}{\textcjheb{m'}} AM $|$wenn\\
12.&89.&1165.&341.&4823.&43.&3&460&10\_400\_50 \textcolor{red}{\textcjheb{nty}} JTN $|$geben wollte/er (=es) g"abe\\
13.&90.&1166.&344.&4826.&46.&3&311&1\_10\_300 \textcolor{red}{\textcjheb{+sy'}} AJS $|$ein Mann/jemand\\
14.&91.&1167.&347.&4829.&49.&2&401&1\_400 \textcolor{red}{\textcjheb{t'}} AT $|$**\\
15.&92.&1168.&349.&4831.&51.&2&50&20\_30 \textcolor{red}{\textcjheb{lk}} KL $|$allen/das ganze\\
16.&93.&1169.&351.&4833.&53.&3&61&5\_6\_50 \textcolor{red}{\textcjheb{nwh}} HWN $|$Reichtum/Gut\\
17.&94.&1170.&354.&4836.&56.&4&418&2\_10\_400\_6 \textcolor{red}{\textcjheb{wtyb}} BJTW $|$seines Hauses\\
18.&95.&1171.&358.&4840.&60.&5&15&2\_1\_5\_2\_5 \textcolor{red}{\textcjheb{hbh'b}} BAHBH $|$um die Liebe/f"ur die Liebe\\
19.&96.&1172.&363.&4845.&65.&3&15&2\_6\_7 \textcolor{red}{\textcjheb{zwb}} BWZ $|$verachten/(ein) Verachten\\
20.&97.&1173.&366.&4848.&68.&5&31&10\_2\_6\_7\_6 \textcolor{red}{\textcjheb{wzwby}} JBWZW $|$nur man w"urde (verachten)/sie w"urden verachten\\
21.&98.&1174.&371.&4853.&73.&2&36&30\_6 \textcolor{red}{\textcjheb{wl}} LW $|$ihn\\
\end{tabular}\medskip \\
Ende des Verses 8.7\\
Verse: 110, Buchstaben: 74, 372, 4854, Totalwerte: 4269, 28293, 336909\\
\\
Gro"se Wasser verm"ogen nicht die Liebe auszul"oschen, und Str"ome "uberfluten sie nicht. Wenn ein Mann allen Reichtum seines Hauses um die Liebe geben wollte, man w"urde ihn nur verachten.\\
\newpage 
{\bf -- 8.8}\\
\medskip \\
\begin{tabular}{rrrrrrrrp{120mm}}
WV&WK&WB&ABK&ABB&ABV&AnzB&TW&Zahlencode \textcolor{red}{$\boldsymbol{Grundtext}$} Umschrift $|$"Ubersetzung(en)\\
1.&99.&1175.&373.&4855.&1.&4&415&1\_8\_6\_400 \textcolor{red}{\textcjheb{tw.h'}} ACWT $|$(eine) Schwester\\
2.&100.&1176.&377.&4859.&5.&3&86&30\_50\_6 \textcolor{red}{\textcjheb{wnl}} LNW $|$wir haben/zu uns\\
3.&101.&1177.&380.&4862.&8.&4&164&100\_9\_50\_5 \textcolor{red}{\textcjheb{hn.tq}} QtNH $|$(eine) kleine\\
4.&102.&1178.&384.&4866.&12.&5&360&6\_300\_4\_10\_40 \textcolor{red}{\textcjheb{myd+sw}} WSDJM $|$noch Br"uste/und Br"uste\\
5.&103.&1179.&389.&4871.&17.&3&61&1\_10\_50 \textcolor{red}{\textcjheb{ny'}} AJN $|$nicht\\
6.&104.&1180.&392.&4874.&20.&2&35&30\_5 \textcolor{red}{\textcjheb{hl}} LH $|$die hat/sie (hat)\\
7.&105.&1181.&394.&4876.&22.&2&45&40\_5 \textcolor{red}{\textcjheb{hm}} MH $|$was\\
8.&106.&1182.&396.&4878.&24.&4&425&50\_70\_300\_5 \textcolor{red}{\textcjheb{h+s`n}} NaSH $|$sollen wir tun/wir sollen machen\\
9.&107.&1183.&400.&4882.&28.&6&495&30\_1\_8\_400\_50\_6 \textcolor{red}{\textcjheb{wnt.h'l}} LACTNW $|$mit unserer Schwester\\
10.&108.&1184.&406.&4888.&34.&4&58&2\_10\_6\_40 \textcolor{red}{\textcjheb{mwyb}} BJWM $|$am Tag\\
11.&109.&1185.&410.&4892.&38.&5&516&300\_10\_4\_2\_200 \textcolor{red}{\textcjheb{rbdy+s}} SJDBR $|$da man werben wird/er (=es) wird geworben\\
12.&110.&1186.&415.&4897.&43.&2&7&2\_5 \textcolor{red}{\textcjheb{hb}} BH $|$um sie\\
\end{tabular}\medskip \\
Ende des Verses 8.8\\
Verse: 111, Buchstaben: 44, 416, 4898, Totalwerte: 2667, 30960, 339576\\
\\
Wir haben eine Schwester, eine kleine, die noch keine Br"uste hat; was sollen wir mit unserer Schwester tun an dem Tage, da man um sie werben wird?\\
\newpage 
{\bf -- 8.9}\\
\medskip \\
\begin{tabular}{rrrrrrrrp{120mm}}
WV&WK&WB&ABK&ABB&ABV&AnzB&TW&Zahlencode \textcolor{red}{$\boldsymbol{Grundtext}$} Umschrift $|$"Ubersetzung(en)\\
1.&111.&1187.&417.&4899.&1.&2&41&1\_40 \textcolor{red}{\textcjheb{m'}} AM $|$wenn\\
2.&112.&1188.&419.&4901.&3.&4&59&8\_6\_40\_5 \textcolor{red}{\textcjheb{hmw.h}} CWMH $|$(eine) Mauer\\
3.&113.&1189.&423.&4905.&7.&3&16&5\_10\_1 \textcolor{red}{\textcjheb{'yh}} HJA $|$sie (ist)\\
4.&114.&1190.&426.&4908.&10.&4&107&50\_2\_50\_5 \textcolor{red}{\textcjheb{hnbn}} NBNH $|$wir (wollen) bauen\\
5.&115.&1191.&430.&4912.&14.&4&115&70\_30\_10\_5 \textcolor{red}{\textcjheb{hyl`}} aLJH $|$darauf/auf sie\\
6.&116.&1192.&434.&4916.&18.&4&619&9\_10\_200\_400 \textcolor{red}{\textcjheb{try.t}} tJRT $|$eine Zinne/eine (Mauer)Krone\\
7.&117.&1193.&438.&4920.&22.&3&160&20\_60\_80 \textcolor{red}{\textcjheb{psk}} KsP $|$von Silber/(aus) Silber\\
8.&118.&1194.&441.&4923.&25.&3&47&6\_1\_40 \textcolor{red}{\textcjheb{m'w}} WAM $|$und wenn\\
9.&119.&1195.&444.&4926.&28.&3&434&4\_30\_400 \textcolor{red}{\textcjheb{tld}} DLT $|$eine T"ure\\
10.&120.&1196.&447.&4929.&31.&3&16&5\_10\_1 \textcolor{red}{\textcjheb{'yh}} HJA $|$sie (ist)\\
11.&121.&1197.&450.&4932.&34.&4&346&50\_90\_6\_200 \textcolor{red}{\textcjheb{rw.sn}} N"sWR $|$so wollen wir verschlie"sen/wir befestigen\\
12.&122.&1198.&454.&4936.&38.&4&115&70\_30\_10\_5 \textcolor{red}{\textcjheb{hyl`}} aLJH $|$(um) sie\\
13.&123.&1199.&458.&4940.&42.&3&44&30\_6\_8 \textcolor{red}{\textcjheb{.hwl}} LWC $|$mit (einem) Brett/Planken\\
14.&124.&1200.&461.&4943.&45.&3&208&1\_200\_7 \textcolor{red}{\textcjheb{zr'}} ARZ $|$(aus) Zedern\\
\end{tabular}\medskip \\
Ende des Verses 8.9\\
Verse: 112, Buchstaben: 47, 463, 4945, Totalwerte: 2327, 33287, 341903\\
\\
Wenn sie eine Mauer ist, so wollen wir eine Zinne von Silber darauf bauen; und wenn sie eine T"ur ist, so wollen wir sie mit einem Zedernbrett verschlie"sen.\\
\newpage 
{\bf -- 8.10}\\
\medskip \\
\begin{tabular}{rrrrrrrrp{120mm}}
WV&WK&WB&ABK&ABB&ABV&AnzB&TW&Zahlencode \textcolor{red}{$\boldsymbol{Grundtext}$} Umschrift $|$"Ubersetzung(en)\\
1.&125.&1201.&464.&4946.&1.&3&61&1\_50\_10 \textcolor{red}{\textcjheb{yn'}} ANJ $|$ich (bin)\\
2.&126.&1202.&467.&4949.&4.&4&59&8\_6\_40\_5 \textcolor{red}{\textcjheb{hmw.h}} CWMH $|$(eine) Mauer\\
3.&127.&1203.&471.&4953.&8.&4&320&6\_300\_4\_10 \textcolor{red}{\textcjheb{yd+sw}} WSDJ $|$und meine Br"uste\\
4.&128.&1204.&475.&4957.&12.&7&503&20\_40\_3\_4\_30\_6\_400 \textcolor{red}{\textcjheb{twldgmk}} KMGDLWT $|$wie die T"urme\\
5.&129.&1205.&482.&4964.&19.&2&8&1\_7 \textcolor{red}{\textcjheb{z'}} AZ $|$da\\
6.&130.&1206.&484.&4966.&21.&5&435&5\_10\_10\_400\_10 \textcolor{red}{\textcjheb{ytyyh}} HJJTJ $|$ich wurde/ich war\\
7.&131.&1207.&489.&4971.&26.&6&148&2\_70\_10\_50\_10\_6 \textcolor{red}{\textcjheb{wyny`b}} BaJNJW $|$in seinen Augen\\
8.&132.&1208.&495.&4977.&32.&6&557&20\_40\_6\_90\_1\_400 \textcolor{red}{\textcjheb{t'.swmk}} KMW"sAT $|$wie eine die findet/wie eine Findende\\
9.&133.&1209.&501.&4983.&38.&4&376&300\_30\_6\_40 \textcolor{red}{\textcjheb{mwl+s}} SLWM $|$Frieden\\
\end{tabular}\medskip \\
Ende des Verses 8.10\\
Verse: 113, Buchstaben: 41, 504, 4986, Totalwerte: 2467, 35754, 344370\\
\\
Ich bin eine Mauer, und meine Br"uste sind wie T"urme; da wurde ich in seinen Augen wie eine, die Frieden findet.\\
\newpage 
{\bf -- 8.11}\\
\medskip \\
\begin{tabular}{rrrrrrrrp{120mm}}
WV&WK&WB&ABK&ABB&ABV&AnzB&TW&Zahlencode \textcolor{red}{$\boldsymbol{Grundtext}$} Umschrift $|$"Ubersetzung(en)\\
1.&134.&1210.&505.&4987.&1.&3&260&20\_200\_40 \textcolor{red}{\textcjheb{mrk}} KRM $|$(einen) Weinberg\\
2.&135.&1211.&508.&4990.&4.&3&20&5\_10\_5 \textcolor{red}{\textcjheb{hyh}} HJH $|$hatte/er (=es) war\\
3.&136.&1212.&511.&4993.&7.&5&405&30\_300\_30\_40\_5 \textcolor{red}{\textcjheb{hml+sl}} LSLMH $|$Salomo/zu Schlomo\\
4.&137.&1213.&516.&4998.&12.&4&104&2\_2\_70\_30 \textcolor{red}{\textcjheb{l`bb}} BBaL $|$zu Baal/in Baal//$<$Herr$>$\\
5.&138.&1214.&520.&5002.&16.&4&101&5\_40\_6\_50 \textcolor{red}{\textcjheb{nwmh}} HMWN $|$Hamon///$<$der Volksmenge$>$\\
6.&139.&1215.&524.&5006.&20.&3&500&50\_400\_50 \textcolor{red}{\textcjheb{ntn}} NTN $|$er ("uber)gab\\
7.&140.&1216.&527.&5009.&23.&2&401&1\_400 \textcolor{red}{\textcjheb{t'}} AT $|$**\\
8.&141.&1217.&529.&5011.&25.&4&265&5\_20\_200\_40 \textcolor{red}{\textcjheb{mrkh}} HKRM $|$den Weinberg\\
9.&142.&1218.&533.&5015.&29.&6&339&30\_50\_9\_200\_10\_40 \textcolor{red}{\textcjheb{myr.tnl}} LNtRJM $|$den H"utern/an den H"utenden\\
10.&143.&1219.&539.&5021.&35.&3&311&1\_10\_300 \textcolor{red}{\textcjheb{+sy'}} AJS $|$ein jeder/(ein) Mann\\
11.&144.&1220.&542.&5024.&38.&3&13&10\_2\_1 \textcolor{red}{\textcjheb{'by}} JBA $|$(er) sollte (auf)bringen\\
12.&145.&1221.&545.&5027.&41.&5&298&2\_80\_200\_10\_6 \textcolor{red}{\textcjheb{wyrpb}} BPRJW $|$f"ur seine Frucht\\
13.&146.&1222.&550.&5032.&46.&3&111&1\_30\_80 \textcolor{red}{\textcjheb{pl'}} ALP $|$tausend\\
14.&147.&1223.&553.&5035.&49.&3&160&20\_60\_80 \textcolor{red}{\textcjheb{psk}} KsP $|$Silbersekel/Silber(St"ucke)\\
\end{tabular}\medskip \\
Ende des Verses 8.11\\
Verse: 114, Buchstaben: 51, 555, 5037, Totalwerte: 3288, 39042, 347658\\
\\
Salomo hatte einen Weinberg zu Baal-Hamon; er "ubergab den Weinberg den H"utern: ein jeder sollte f"ur seine Frucht tausend Silbersekel bringen.\\
\newpage 
{\bf -- 8.12}\\
\medskip \\
\begin{tabular}{rrrrrrrrp{120mm}}
WV&WK&WB&ABK&ABB&ABV&AnzB&TW&Zahlencode \textcolor{red}{$\boldsymbol{Grundtext}$} Umschrift $|$"Ubersetzung(en)\\
1.&148.&1224.&556.&5038.&1.&4&270&20\_200\_40\_10 \textcolor{red}{\textcjheb{ymrk}} KRMJ $|$mein Weinberg\\
2.&149.&1225.&560.&5042.&5.&3&340&300\_30\_10 \textcolor{red}{\textcjheb{yl+s}} SLJ $|$eigener/welcher zu mir\\
3.&150.&1226.&563.&5045.&8.&4&170&30\_80\_50\_10 \textcolor{red}{\textcjheb{ynpl}} LPNJ $|$ist vor mir\\
4.&151.&1227.&567.&5049.&12.&4&116&5\_1\_30\_80 \textcolor{red}{\textcjheb{pl'h}} HALP $|$die tausend\\
5.&152.&1228.&571.&5053.&16.&2&50&30\_20 \textcolor{red}{\textcjheb{kl}} LK $|$sind dein/f"ur dich\\
6.&153.&1229.&573.&5055.&18.&4&375&300\_30\_40\_5 \textcolor{red}{\textcjheb{hml+s}} SLMH $|$Salomo/Schelomo\\
7.&154.&1230.&577.&5059.&22.&6&497&6\_40\_1\_400\_10\_40 \textcolor{red}{\textcjheb{myt'mw}} WMATJM $|$und zweihundert(e)\\
8.&155.&1231.&583.&5065.&28.&6&339&30\_50\_9\_200\_10\_40 \textcolor{red}{\textcjheb{myr.tnl}} LNtRJM $|$seien den H"utern/zu den H"utenden\\
9.&156.&1232.&589.&5071.&34.&2&401&1\_400 \textcolor{red}{\textcjheb{t'}} AT $|$**\\
10.&157.&1233.&591.&5073.&36.&4&296&80\_200\_10\_6 \textcolor{red}{\textcjheb{wyrp}} PRJW $|$seine(r) Frucht\\
\end{tabular}\medskip \\
Ende des Verses 8.12\\
Verse: 115, Buchstaben: 39, 594, 5076, Totalwerte: 2854, 41896, 350512\\
\\
Mein eigener Weinberg ist vor mir; die tausend sind dein, Salomo, und zweihundert seien den H"utern seiner Frucht.\\
\newpage 
{\bf -- 8.13}\\
\medskip \\
\begin{tabular}{rrrrrrrrp{120mm}}
WV&WK&WB&ABK&ABB&ABV&AnzB&TW&Zahlencode \textcolor{red}{$\boldsymbol{Grundtext}$} Umschrift $|$"Ubersetzung(en)\\
1.&158.&1234.&595.&5077.&1.&6&723&5\_10\_6\_300\_2\_400 \textcolor{red}{\textcjheb{tb+swyh}} HJWSBT $|$Bewohnerin/du Wohnende\\
2.&159.&1235.&601.&5083.&7.&5&105&2\_3\_50\_10\_40 \textcolor{red}{\textcjheb{myngb}} BGNJM $|$der G"arten/in den G"arten\\
3.&160.&1236.&606.&5088.&12.&5&260&8\_2\_200\_10\_40 \textcolor{red}{\textcjheb{myrb.h}} CBRJM $|$die Genossen/Freunde\\
4.&161.&1237.&611.&5093.&17.&7&502&40\_100\_300\_10\_2\_10\_40 \textcolor{red}{\textcjheb{myby+sqm}} MQSJBJM $|$horchen/(sind) hinh"orend(e)\\
5.&162.&1238.&618.&5100.&24.&5&186&30\_100\_6\_30\_20 \textcolor{red}{\textcjheb{klwql}} LQWLK $|$auf deine Stimme\\
6.&163.&1239.&623.&5105.&29.&8&495&5\_300\_40\_10\_70\_10\_50\_10 \textcolor{red}{\textcjheb{yny`ym+sh}} HSMJaJNJ $|$lass sie (mich) h"oren\\
\end{tabular}\medskip \\
Ende des Verses 8.13\\
Verse: 116, Buchstaben: 36, 630, 5112, Totalwerte: 2271, 44167, 352783\\
\\
Bewohnerin der G"arten, die Genossen horchen auf deine Stimme; la"s sie mich h"oren!\\
\newpage 
{\bf -- 8.14}\\
\medskip \\
\begin{tabular}{rrrrrrrrp{120mm}}
WV&WK&WB&ABK&ABB&ABV&AnzB&TW&Zahlencode \textcolor{red}{$\boldsymbol{Grundtext}$} Umschrift $|$"Ubersetzung(en)\\
1.&164.&1240.&631.&5113.&1.&3&210&2\_200\_8 \textcolor{red}{\textcjheb{.hrb}} BRC $|$enteile\\
2.&165.&1241.&634.&5116.&4.&4&24&4\_6\_4\_10 \textcolor{red}{\textcjheb{ydwd}} DWDJ $|$mein Geliebter/mein Freund\\
3.&166.&1242.&638.&5120.&8.&4&55&6\_4\_40\_5 \textcolor{red}{\textcjheb{hmdw}} WDMH $|$und sei gleich/und tue es gleich\\
4.&167.&1243.&642.&5124.&12.&2&50&30\_20 \textcolor{red}{\textcjheb{kl}} LK $|$/f"ur dich\\
5.&168.&1244.&644.&5126.&14.&4&132&30\_90\_2\_10 \textcolor{red}{\textcjheb{yb.sl}} L"sBJ $|$einer Gazelle/der Gazelle\\
6.&169.&1245.&648.&5130.&18.&2&7&1\_6 \textcolor{red}{\textcjheb{w'}} AW $|$oder\\
7.&170.&1246.&650.&5132.&20.&4&380&30\_70\_80\_200 \textcolor{red}{\textcjheb{rp`l}} LaPR $|$einem Jungen/dem B"ocklein\\
8.&171.&1247.&654.&5136.&24.&6&96&5\_1\_10\_30\_10\_40 \textcolor{red}{\textcjheb{myly'h}} HAJLJM $|$der Hirsche\\
9.&172.&1248.&660.&5142.&30.&2&100&70\_30 \textcolor{red}{\textcjheb{l`}} aL $|$auf\\
10.&173.&1249.&662.&5144.&32.&3&215&5\_200\_10 \textcolor{red}{\textcjheb{yrh}} HRJ $|$den Bergen/dem Berge\\
11.&174.&1250.&665.&5147.&35.&5&392&2\_300\_40\_10\_40 \textcolor{red}{\textcjheb{mym+sb}} BSMJM $|$duftenden/(der) Balsam(b"aum)e\\
\end{tabular}\medskip \\
Ende des Verses 8.14\\
Verse: 117, Buchstaben: 39, 669, 5151, Totalwerte: 1661, 45828, 354444\\
\\
Enteile, mein Geliebter, und sei gleich einer Gazelle oder einem Jungen der Hirsche auf den duftenden Bergen!\\
\\
{\bf Ende des Kapitels 8}\\

\bigskip				%%gro�er Abstand

\newpage
\hphantom{x}
\bigskip\bigskip\bigskip\bigskip\bigskip\bigskip
\begin{center}{ \huge {\bf Ende des Buches}}\end{center}


\end{document}



