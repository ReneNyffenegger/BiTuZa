\documentclass[a4paper,10pt,landscape]{article}
\usepackage[landscape]{geometry}	%%Querformat
\usepackage{fancyhdr}			%%Erweiterte Kopfzeilen
\usepackage{color}			%%Farben
\usepackage[greek,german]{babel}        %%Griechische und deutsche Namen
\usepackage{cjhebrew}                   %%Hebr�isches Packet
\usepackage{upgreek}                    %%nicht kursive griechische Buchstaben
\usepackage{amsbsy}                     %%fette griechische Buchstaben
\setlength{\parindent}{0pt}		%%Damit bei neuen Abs"atzen kein Einzug
\frenchspacing

\pagestyle{fancy}			%%Kopf-/Fu�zeilenstyle

\renewcommand{\headrulewidth}{0.5pt}	%%Strich in Kopfzeile
\renewcommand{\footrulewidth}{0pt}	%%kein Strich in Fu�zeile
\renewcommand{\sectionmark}[1]{\markright{{#1}}}
\lhead{\rightmark}
\chead{}
\rhead{Bibel in Text und Zahl}
\lfoot{pgz}
\cfoot{\thepage}			%%Seitenzahl
\rfoot{}

\renewcommand{\section}[3]{\begin{center}{ \huge {\bf \textsl{{#1}}\\ \textcolor{red}{\textsl{{#2}}}}}\end{center}
\sectionmark{{#3}}}

\renewcommand{\baselinestretch}{0.9}	%%Zeilenabstand

\begin{document}			%%Dokumentbeginn

\section{\bigskip\bigskip\bigskip\bigskip\bigskip\bigskip
\\Der dritte Brief des Johannes}
{}
{3. Johannes}	%%�berschrift (�bergibt {schwarzen
										%%Text}{roten Text}{Text in Kopfzeile}


\bigskip				%%gro�er Abstand

\newpage
\hphantom{x}
\bigskip\bigskip\bigskip\bigskip\bigskip\bigskip
\begin{center}{ \huge {\bf Erl"auterungen}}\end{center}

\medskip
In diesem Buch werden folgende Abk"urzungen verwendet:\\
WV = Nummer des Wortes im Vers\\
WK = Nummer des Wortes im Kapitel\\
WB = Nummer des Wortes im Buch\\
ABV = Nummer des Anfangsbuchstabens des Wortes im Vers\\
ABK = Nummer des Anfangsbuchstabens des Wortes im Kapitel\\
ABB = Nummer des Anfangsbuchstabens des Wortes im Buch\\
AnzB = Anzahl der Buchstaben des Wortes\\
TW = Totalwert des Wortes\\

\medskip
Am Ende eines Verses finden sich sieben Zahlen,\\
die folgende Bedeutung haben (von links nach rechts):\\
1. Nummer des Verses im Buch\\
2. Gesamtzahl der Buchstaben im Vers\\
3. Gesamtzahl der Buchstaben (bis einschlie"slich dieses Verses) im Kapitel\\
4. Gesamtzahl der Buchstaben (bis einschlie"slich dieses Verses) im Buch\\
5. Summe der Totalwerte des Verses\\
6. Summe der Totalwerte (bis einschlie"slich dieses Verses) im Kapitel\\
7. Summe der Totalwerte (bis einschlie"slich dieses Verses) im Buch\\



\newpage 
{\bf -- 1.1}\\
\medskip \\
\begin{tabular}{rrrrrrrrp{120mm}}
WV&WK&WB&ABK&ABB&ABV&AnzB&TW&Zahlencode \textcolor{red}{$\boldsymbol{Grundtext}$} Umschrift $|$"Ubersetzung(en)\\
1.&1.&1.&1.&1.&1.&1&70&70 \textcolor{red}{$\boldsymbol{\mathrm{o}}$} o $|$der\\
2.&2.&2.&2.&2.&2.&11&1462&80\_100\_5\_200\_2\_400\_300\_5\_100\_70\_200 \textcolor{red}{$\boldsymbol{\uppi\uprho\upepsilon\upsigma\upbeta\upsilon\uptau\upepsilon\uprho\mathrm{o}\upsigma}$} presb"uteros $|$"Alteste\\
3.&3.&3.&13.&13.&13.&4&814&3\_1\_10\_800 \textcolor{red}{$\boldsymbol{\upgamma\upalpha\upiota\upomega}$} gajO $|$an Gajus/an Gaius//$<$Erdmann$>$\\
4.&4.&4.&17.&17.&17.&2&1100&300\_800 \textcolor{red}{$\boldsymbol{\uptau\upomega}$} tO $|$den\\
5.&5.&5.&19.&19.&19.&7&1193&1\_3\_1\_80\_8\_300\_800 \textcolor{red}{$\boldsymbol{\upalpha\upgamma\upalpha\uppi\upeta\uptau\upomega}$} agap"atO $|$geliebten\\
6.&6.&6.&26.&26.&26.&2&120&70\_50 \textcolor{red}{$\boldsymbol{\mathrm{o}\upnu}$} on $|$den\\
7.&7.&7.&28.&28.&28.&3&808&5\_3\_800 \textcolor{red}{$\boldsymbol{\upepsilon\upgamma\upomega}$} egO $|$ich\\
8.&8.&8.&31.&31.&31.&5&885&1\_3\_1\_80\_800 \textcolor{red}{$\boldsymbol{\upalpha\upgamma\upalpha\uppi\upomega}$} agapO $|$liebe\\
9.&9.&9.&36.&36.&36.&2&55&5\_50 \textcolor{red}{$\boldsymbol{\upepsilon\upnu}$} en $|$in\\
10.&10.&10.&38.&38.&38.&7&64&1\_30\_8\_9\_5\_10\_1 \textcolor{red}{$\boldsymbol{\upalpha\uplambda\upeta\upvartheta\upepsilon\upiota\upalpha}$} al"aTeja $|$Wahrheit\\
\end{tabular}\medskip \\
Ende des Verses 1.1\\
Verse: 1, Buchstaben: 44, 44, 44, Totalwerte: 6571, 6571, 6571\\
\\
Der "Alteste dem geliebten Gajus, den ich liebe in der Wahrheit.\\
\newpage 
{\bf -- 1.2}\\
\medskip \\
\begin{tabular}{rrrrrrrrp{120mm}}
WV&WK&WB&ABK&ABB&ABV&AnzB&TW&Zahlencode \textcolor{red}{$\boldsymbol{Grundtext}$} Umschrift $|$"Ubersetzung(en)\\
1.&11.&11.&45.&45.&1.&7&398&1\_3\_1\_80\_8\_300\_5 \textcolor{red}{$\boldsymbol{\upalpha\upgamma\upalpha\uppi\upeta\uptau\upepsilon}$} agap"ate $|$mein Lieber/Geliebter\\
2.&12.&12.&52.&52.&8.&4&195&80\_5\_100\_10 \textcolor{red}{$\boldsymbol{\uppi\upepsilon\uprho\upiota}$} perj $|$in\\
3.&13.&13.&56.&56.&12.&6&1281&80\_1\_50\_300\_800\_50 \textcolor{red}{$\boldsymbol{\uppi\upalpha\upnu\uptau\upomega\upnu}$} pantOn $|$allen (Dingen)\\
4.&14.&14.&62.&62.&18.&7&1126&5\_400\_600\_70\_40\_1\_10 \textcolor{red}{$\boldsymbol{\upepsilon\upsilon\upchi\mathrm{o}\upmu\upalpha\upiota}$} e"ucomaj $|$w"unsche ich\\
5.&15.&15.&69.&69.&25.&2&205&200\_5 \textcolor{red}{$\boldsymbol{\upsigma\upepsilon}$} se $|$dir/(dass) du\\
6.&16.&16.&71.&71.&27.&10&1169&5\_400\_70\_4\_70\_400\_200\_9\_1\_10 \textcolor{red}{$\boldsymbol{\upepsilon\upsilon\mathrm{o}\updelta\mathrm{o}\upsilon\upsigma\upvartheta\upalpha\upiota}$} e"uodo"usTaj $|$Wohlergehen/einen guten Weg gef"uhrst wirst\\
7.&17.&17.&81.&81.&37.&3&31&20\_1\_10 \textcolor{red}{$\boldsymbol{\upkappa\upalpha\upiota}$} kaj $|$und\\
8.&18.&18.&84.&84.&40.&9&539&400\_3\_10\_1\_10\_50\_5\_10\_50 \textcolor{red}{$\boldsymbol{\upsilon\upgamma\upiota\upalpha\upiota\upnu\upepsilon\upiota\upnu}$} "ugjajnejn $|$Gesundheit/gesund bist\\
9.&19.&19.&93.&93.&49.&5&1030&20\_1\_9\_800\_200 \textcolor{red}{$\boldsymbol{\upkappa\upalpha\upvartheta\upomega\upsigma}$} kaTOs $|$(so) wie\\
10.&20.&20.&98.&98.&54.&9&1260&5\_400\_70\_4\_70\_400\_300\_1\_10 \textcolor{red}{$\boldsymbol{\upepsilon\upsilon\mathrm{o}\updelta\mathrm{o}\upsilon\uptau\upalpha\upiota}$} e"uodo"utaj $|$es wohlgeht/einen guten Weg gef"uhrt wird\\
11.&21.&21.&107.&107.&63.&3&670&200\_70\_400 \textcolor{red}{$\boldsymbol{\upsigma\mathrm{o}\upsilon}$} so"u $|$deine(r)\\
12.&22.&22.&110.&110.&66.&1&8&8 \textcolor{red}{$\boldsymbol{\upeta}$} "a $|$(die)\\
13.&23.&23.&111.&111.&67.&4&1708&700\_400\_600\_8 \textcolor{red}{$\boldsymbol{\uppsi\upsilon\upchi\upeta}$} P"uc"a $|$Seele\\
\end{tabular}\medskip \\
Ende des Verses 1.2\\
Verse: 2, Buchstaben: 70, 114, 114, Totalwerte: 9620, 16191, 16191\\
\\
Geliebter, ich w"unsche, da"s es dir in allem wohlgehe und du gesund seiest, gleichwie es deiner Seele wohlgeht.\\
\newpage 
{\bf -- 1.3}\\
\medskip \\
\begin{tabular}{rrrrrrrrp{120mm}}
WV&WK&WB&ABK&ABB&ABV&AnzB&TW&Zahlencode \textcolor{red}{$\boldsymbol{Grundtext}$} Umschrift $|$"Ubersetzung(en)\\
1.&24.&24.&115.&115.&1.&6&764&5\_600\_1\_100\_8\_50 \textcolor{red}{$\boldsymbol{\upepsilon\upchi\upalpha\uprho\upeta\upnu}$} ecar"an $|$ich freute mich/ich habe mich gefreut\\
2.&25.&25.&121.&121.&7.&3&104&3\_1\_100 \textcolor{red}{$\boldsymbol{\upgamma\upalpha\uprho}$} gar $|$denn\\
3.&26.&26.&124.&124.&10.&4&91&30\_10\_1\_50 \textcolor{red}{$\boldsymbol{\uplambda\upiota\upalpha\upnu}$} ljan $|$sehr\\
4.&27.&27.&128.&128.&14.&9&1720&5\_100\_600\_70\_40\_5\_50\_800\_50 \textcolor{red}{$\boldsymbol{\upepsilon\uprho\upchi\mathrm{o}\upmu\upepsilon\upnu\upomega\upnu}$} ercomenOn $|$(als) kamen\\
5.&28.&28.&137.&137.&23.&7&1390&1\_4\_5\_30\_500\_800\_50 \textcolor{red}{$\boldsymbol{\upalpha\updelta\upepsilon\uplambda\upvarphi\upomega\upnu}$} adelfOn $|$Br"uder\\
6.&29.&29.&144.&144.&30.&3&31&20\_1\_10 \textcolor{red}{$\boldsymbol{\upkappa\upalpha\upiota}$} kaj $|$und\\
7.&30.&30.&147.&147.&33.&12&2611&40\_1\_100\_300\_400\_100\_70\_400\_50\_300\_800\_50 \textcolor{red}{$\boldsymbol{\upmu\upalpha\uprho\uptau\upsilon\uprho\mathrm{o}\upsilon\upnu\uptau\upomega\upnu}$} mart"uro"untOn $|$Zeugnis ablegten\\
8.&31.&31.&159.&159.&45.&3&670&200\_70\_400 \textcolor{red}{$\boldsymbol{\upsigma\mathrm{o}\upsilon}$} so"u $|$von deiner/f"ur deine\\
9.&32.&32.&162.&162.&48.&2&308&300\_8 \textcolor{red}{$\boldsymbol{\uptau\upeta}$} t"a $|$(die)\\
10.&33.&33.&164.&164.&50.&7&64&1\_30\_8\_9\_5\_10\_1 \textcolor{red}{$\boldsymbol{\upalpha\uplambda\upeta\upvartheta\upepsilon\upiota\upalpha}$} al"aTeja $|$(in) Wahrheit\\
11.&34.&34.&171.&171.&57.&5&1030&20\_1\_9\_800\_200 \textcolor{red}{$\boldsymbol{\upkappa\upalpha\upvartheta\upomega\upsigma}$} kaTOs $|$wie\\
12.&35.&35.&176.&176.&62.&2&600&200\_400 \textcolor{red}{$\boldsymbol{\upsigma\upsilon}$} s"u $|$du\\
13.&36.&36.&178.&178.&64.&2&55&5\_50 \textcolor{red}{$\boldsymbol{\upepsilon\upnu}$} en $|$in\\
14.&37.&37.&180.&180.&66.&7&64&1\_30\_8\_9\_5\_10\_1 \textcolor{red}{$\boldsymbol{\upalpha\uplambda\upeta\upvartheta\upepsilon\upiota\upalpha}$} al"aTeja $|$(der) Wahrheit\\
15.&38.&38.&187.&187.&73.&10&791&80\_5\_100\_10\_80\_1\_300\_5\_10\_200 \textcolor{red}{$\boldsymbol{\uppi\upepsilon\uprho\upiota\uppi\upalpha\uptau\upepsilon\upiota\upsigma}$} perjpatejs $|$wandelst\\
\end{tabular}\medskip \\
Ende des Verses 1.3\\
Verse: 3, Buchstaben: 82, 196, 196, Totalwerte: 10293, 26484, 26484\\
\\
Denn ich freute mich sehr, als Br"uder kamen und Zeugnis gaben von deinem Festhalten an der Wahrheit, gleichwie du in der Wahrheit wandelst.\\
\newpage 
{\bf -- 1.4}\\
\medskip \\
\begin{tabular}{rrrrrrrrp{120mm}}
WV&WK&WB&ABK&ABB&ABV&AnzB&TW&Zahlencode \textcolor{red}{$\boldsymbol{Grundtext}$} Umschrift $|$"Ubersetzung(en)\\
1.&39.&39.&197.&197.&1.&10&588&40\_5\_10\_7\_70\_300\_5\_100\_1\_50 \textcolor{red}{$\boldsymbol{\upmu\upepsilon\upiota\upzeta\mathrm{o}\uptau\upepsilon\uprho\upalpha\upnu}$} mejzoteran $|$gr"o"sere\\
2.&40.&40.&207.&207.&11.&6&1920&300\_70\_400\_300\_800\_50 \textcolor{red}{$\boldsymbol{\uptau\mathrm{o}\upsilon\uptau\upomega\upnu}$} to"utOn $|$als die/als diese (Berichte)\\
3.&41.&41.&213.&213.&17.&3&490&70\_400\_20 \textcolor{red}{$\boldsymbol{\mathrm{o}\upsilon\upkappa}$} o"uk $|$nicht\\
4.&42.&42.&216.&216.&20.&3&1405&5\_600\_800 \textcolor{red}{$\boldsymbol{\upepsilon\upchi\upomega}$} ecO $|$habe ich\\
5.&43.&43.&219.&219.&23.&5&752&600\_1\_100\_1\_50 \textcolor{red}{$\boldsymbol{\upchi\upalpha\uprho\upalpha\upnu}$} caran $|$Freude\\
6.&44.&44.&224.&224.&28.&3&61&10\_50\_1 \textcolor{red}{$\boldsymbol{\upiota\upnu\upalpha}$} jna $|$/dass\\
7.&45.&45.&227.&227.&31.&5&1291&1\_20\_70\_400\_800 \textcolor{red}{$\boldsymbol{\upalpha\upkappa\mathrm{o}\upsilon\upomega}$} ako"uO $|$zu h"oren/h"ore ich\\
8.&46.&46.&232.&232.&36.&2&301&300\_1 \textcolor{red}{$\boldsymbol{\uptau\upalpha}$} ta $|$(die)\\
9.&47.&47.&234.&234.&38.&3&46&5\_40\_1 \textcolor{red}{$\boldsymbol{\upepsilon\upmu\upalpha}$} ema $|$meine\\
10.&48.&48.&237.&237.&41.&5&376&300\_5\_20\_50\_1 \textcolor{red}{$\boldsymbol{\uptau\upepsilon\upkappa\upnu\upalpha}$} tekna $|$Kinder\\
11.&49.&49.&242.&242.&46.&2&55&5\_50 \textcolor{red}{$\boldsymbol{\upepsilon\upnu}$} en $|$in\\
12.&50.&50.&244.&244.&48.&7&64&1\_30\_8\_9\_5\_10\_1 \textcolor{red}{$\boldsymbol{\upalpha\uplambda\upeta\upvartheta\upepsilon\upiota\upalpha}$} al"aTeja $|$(der) Wahrheit\\
13.&51.&51.&251.&251.&55.&12&1397&80\_5\_100\_10\_80\_1\_300\_70\_400\_50\_300\_1 \textcolor{red}{$\boldsymbol{\uppi\upepsilon\uprho\upiota\uppi\upalpha\uptau\mathrm{o}\upsilon\upnu\uptau\upalpha}$} perjpato"unta $|$wandeln(d)\\
\end{tabular}\medskip \\
Ende des Verses 1.4\\
Verse: 4, Buchstaben: 66, 262, 262, Totalwerte: 8746, 35230, 35230\\
\\
Ich habe keine gr"o"sere Freude als dies, da"s ich h"ore, da"s meine Kinder in der Wahrheit wandeln.\\
\newpage 
{\bf -- 1.5}\\
\medskip \\
\begin{tabular}{rrrrrrrrp{120mm}}
WV&WK&WB&ABK&ABB&ABV&AnzB&TW&Zahlencode \textcolor{red}{$\boldsymbol{Grundtext}$} Umschrift $|$"Ubersetzung(en)\\
1.&52.&52.&263.&263.&1.&7&398&1\_3\_1\_80\_8\_300\_5 \textcolor{red}{$\boldsymbol{\upalpha\upgamma\upalpha\uppi\upeta\uptau\upepsilon}$} agap"ate $|$mein Lieber/Geliebter\\
2.&53.&53.&270.&270.&8.&6&710&80\_10\_200\_300\_70\_50 \textcolor{red}{$\boldsymbol{\uppi\upiota\upsigma\uptau\mathrm{o}\upnu}$} pjston $|$treu\\
3.&54.&54.&276.&276.&14.&6&375&80\_70\_10\_5\_10\_200 \textcolor{red}{$\boldsymbol{\uppi\mathrm{o}\upiota\upepsilon\upiota\upsigma}$} pojejs $|$handelst du/tust du\\
4.&55.&55.&282.&282.&20.&1&70&70 \textcolor{red}{$\boldsymbol{\mathrm{o}}$} o $|$indem/(der)\\
5.&56.&56.&283.&283.&21.&3&56&5\_1\_50 \textcolor{red}{$\boldsymbol{\upepsilon\upalpha\upnu}$} ean $|$was\\
6.&57.&57.&286.&286.&24.&6&317&5\_100\_3\_1\_200\_8 \textcolor{red}{$\boldsymbol{\upepsilon\uprho\upgamma\upalpha\upsigma\upeta}$} ergas"a $|$du tust/du wirkst\\
7.&58.&58.&292.&292.&30.&3&215&5\_10\_200 \textcolor{red}{$\boldsymbol{\upepsilon\upiota\upsigma}$} ejs $|$an\\
8.&59.&59.&295.&295.&33.&4&970&300\_70\_400\_200 \textcolor{red}{$\boldsymbol{\uptau\mathrm{o}\upsilon\upsigma}$} to"us $|$den\\
9.&60.&60.&299.&299.&37.&8&1210&1\_4\_5\_30\_500\_70\_400\_200 \textcolor{red}{$\boldsymbol{\upalpha\updelta\upepsilon\uplambda\upvarphi\mathrm{o}\upsilon\upsigma}$} adelfo"us $|$Br"udern\\
10.&61.&61.&307.&307.&45.&3&31&20\_1\_10 \textcolor{red}{$\boldsymbol{\upkappa\upalpha\upiota}$} kaj $|$auch/und\\
11.&62.&62.&310.&310.&48.&3&215&5\_10\_200 \textcolor{red}{$\boldsymbol{\upepsilon\upiota\upsigma}$} ejs $|$an\\
12.&63.&63.&313.&313.&51.&4&970&300\_70\_400\_200 \textcolor{red}{$\boldsymbol{\uptau\mathrm{o}\upsilon\upsigma}$} to"us $|$den/dies\\
13.&64.&64.&317.&317.&55.&6&785&60\_5\_50\_70\_400\_200 \textcolor{red}{$\boldsymbol{\upxi\upepsilon\upnu\mathrm{o}\upsilon\upsigma}$} xeno"us $|$unbekannten/fremden\\
\end{tabular}\medskip \\
Ende des Verses 1.5\\
Verse: 5, Buchstaben: 60, 322, 322, Totalwerte: 6322, 41552, 41552\\
\\
Geliebter, treulich tust du, was irgend du an den Br"udern, und zwar an Fremden, getan haben magst,\\
\newpage 
{\bf -- 1.6}\\
\medskip \\
\begin{tabular}{rrrrrrrrp{120mm}}
WV&WK&WB&ABK&ABB&ABV&AnzB&TW&Zahlencode \textcolor{red}{$\boldsymbol{Grundtext}$} Umschrift $|$"Ubersetzung(en)\\
1.&65.&65.&323.&323.&1.&2&80&70\_10 \textcolor{red}{$\boldsymbol{\mathrm{o}\upiota}$} oj $|$die\\
2.&66.&66.&325.&325.&3.&11&1205&5\_40\_1\_100\_300\_400\_100\_8\_200\_1\_50 \textcolor{red}{$\boldsymbol{\upepsilon\upmu\upalpha\uprho\uptau\upsilon\uprho\upeta\upsigma\upalpha\upnu}$} emart"ur"asan $|$Zeugnis abgelegt haben\\
3.&67.&67.&336.&336.&14.&3&670&200\_70\_400 \textcolor{red}{$\boldsymbol{\upsigma\mathrm{o}\upsilon}$} so"u $|$von deiner/f"ur deine\\
4.&68.&68.&339.&339.&17.&2&308&300\_8 \textcolor{red}{$\boldsymbol{\uptau\upeta}$} t"a $|$(die)\\
5.&69.&69.&341.&341.&19.&5&93&1\_3\_1\_80\_8 \textcolor{red}{$\boldsymbol{\upalpha\upgamma\upalpha\uppi\upeta}$} agap"a $|$Liebe\\
6.&70.&70.&346.&346.&24.&7&1065&5\_50\_800\_80\_10\_70\_50 \textcolor{red}{$\boldsymbol{\upepsilon\upnu\upomega\uppi\upiota\mathrm{o}\upnu}$} enOpjon $|$vor\\
7.&71.&71.&353.&353.&31.&9&494&5\_20\_20\_30\_8\_200\_10\_1\_200 \textcolor{red}{$\boldsymbol{\upepsilon\upkappa\upkappa\uplambda\upeta\upsigma\upiota\upalpha\upsigma}$} ekkl"asjas $|$(der) Gemeinde\\
8.&72.&72.&362.&362.&40.&3&670&70\_400\_200 \textcolor{red}{$\boldsymbol{\mathrm{o}\upsilon\upsigma}$} o"us $|$/die\\
9.&73.&73.&365.&365.&43.&5&1051&20\_1\_30\_800\_200 \textcolor{red}{$\boldsymbol{\upkappa\upalpha\uplambda\upomega\upsigma}$} kalOs $|$wohl/gut\\
10.&74.&74.&370.&370.&48.&8&583&80\_70\_10\_8\_200\_5\_10\_200 \textcolor{red}{$\boldsymbol{\uppi\mathrm{o}\upiota\upeta\upsigma\upepsilon\upiota\upsigma}$} poj"asejs $|$wirst du tun\\
11.&75.&75.&378.&378.&56.&9&1276&80\_100\_70\_80\_5\_40\_700\_1\_200 \textcolor{red}{$\boldsymbol{\uppi\uprho\mathrm{o}\uppi\upepsilon\upmu\uppsi\upalpha\upsigma}$} propemPas $|$wenn du Geleit gibst/geleitet habend\\
12.&76.&76.&387.&387.&65.&5&1071&1\_60\_10\_800\_200 \textcolor{red}{$\boldsymbol{\upalpha\upxi\upiota\upomega\upsigma}$} axjOs $|$(wie es) w"urdig (ist)\\
13.&77.&77.&392.&392.&70.&3&770&300\_70\_400 \textcolor{red}{$\boldsymbol{\uptau\mathrm{o}\upsilon}$} to"u $|$(des)\\
14.&78.&78.&395.&395.&73.&4&484&9\_5\_70\_400 \textcolor{red}{$\boldsymbol{\upvartheta\upepsilon\mathrm{o}\upsilon}$} Teo"u $|$Gottes\\
\end{tabular}\medskip \\
Ende des Verses 1.6\\
Verse: 6, Buchstaben: 76, 398, 398, Totalwerte: 9820, 51372, 51372\\
\\
(die von deiner Liebe Zeugnis gegeben haben vor der Versammlung) und du wirst wohltun, wenn du sie auf eine gottesw"urdige Weise geleitest.\\
\newpage 
{\bf -- 1.7}\\
\medskip \\
\begin{tabular}{rrrrrrrrp{120mm}}
WV&WK&WB&ABK&ABB&ABV&AnzB&TW&Zahlencode \textcolor{red}{$\boldsymbol{Grundtext}$} Umschrift $|$"Ubersetzung(en)\\
1.&79.&79.&399.&399.&1.&4&585&400\_80\_5\_100 \textcolor{red}{$\boldsymbol{\upsilon\uppi\upepsilon\uprho}$} "uper $|$um willen/f"ur\\
2.&80.&80.&403.&403.&5.&3&104&3\_1\_100 \textcolor{red}{$\boldsymbol{\upgamma\upalpha\uprho}$} gar $|$denn\\
3.&81.&81.&406.&406.&8.&3&770&300\_70\_400 \textcolor{red}{$\boldsymbol{\uptau\mathrm{o}\upsilon}$} to"u $|$seines/den\\
4.&82.&82.&409.&409.&11.&8&801&70\_50\_70\_40\_1\_300\_70\_200 \textcolor{red}{$\boldsymbol{\mathrm{o}\upnu\mathrm{o}\upmu\upalpha\uptau\mathrm{o}\upsigma}$} onomatos $|$Namen(s)\\
5.&83.&83.&417.&417.&19.&7&232&5\_60\_8\_30\_9\_70\_50 \textcolor{red}{$\boldsymbol{\upepsilon\upxi\upeta\uplambda\upvartheta\mathrm{o}\upnu}$} ex"alTon $|$sind sie ausgezogen\\
6.&84.&84.&424.&424.&26.&5&107&40\_8\_4\_5\_50 \textcolor{red}{$\boldsymbol{\upmu\upeta\updelta\upepsilon\upnu}$} m"aden $|$ohne/nichts\\
7.&85.&85.&429.&429.&31.&11&749&30\_1\_40\_2\_1\_50\_70\_50\_300\_5\_200 \textcolor{red}{$\boldsymbol{\uplambda\upalpha\upmu\upbeta\upalpha\upnu\mathrm{o}\upnu\uptau\upepsilon\upsigma}$} lambanontes $|$etwas anzunehmen/annehmend\\
8.&86.&86.&440.&440.&42.&3&151&1\_80\_70 \textcolor{red}{$\boldsymbol{\upalpha\uppi\mathrm{o}}$} apo $|$von\\
9.&87.&87.&443.&443.&45.&3&1150&300\_800\_50 \textcolor{red}{$\boldsymbol{\uptau\upomega\upnu}$} tOn $|$den\\
10.&88.&88.&446.&446.&48.&5&914&5\_9\_50\_800\_50 \textcolor{red}{$\boldsymbol{\upepsilon\upvartheta\upnu\upomega\upnu}$} eTnOn $|$Heiden\\
\end{tabular}\medskip \\
Ende des Verses 1.7\\
Verse: 7, Buchstaben: 52, 450, 450, Totalwerte: 5563, 56935, 56935\\
\\
Denn f"ur den Namen sind sie ausgegangen und nehmen nichts von denen aus den Nationen.\\
\newpage 
{\bf -- 1.8}\\
\medskip \\
\begin{tabular}{rrrrrrrrp{120mm}}
WV&WK&WB&ABK&ABB&ABV&AnzB&TW&Zahlencode \textcolor{red}{$\boldsymbol{Grundtext}$} Umschrift $|$"Ubersetzung(en)\\
1.&89.&89.&451.&451.&1.&5&263&8\_40\_5\_10\_200 \textcolor{red}{$\boldsymbol{\upeta\upmu\upepsilon\upiota\upsigma}$} "amejs $|$wir\\
2.&90.&90.&456.&456.&6.&3&520&70\_400\_50 \textcolor{red}{$\boldsymbol{\mathrm{o}\upsilon\upnu}$} o"un $|$(al)so\\
3.&91.&91.&459.&459.&9.&9&780&70\_500\_5\_10\_30\_70\_40\_5\_50 \textcolor{red}{$\boldsymbol{\mathrm{o}\upvarphi\upepsilon\upiota\uplambda\mathrm{o}\upmu\upepsilon\upnu}$} ofejlomen $|$sind nun verpflichtet/sind schuldig\\
4.&92.&92.&468.&468.&18.&12&340&1\_80\_70\_30\_1\_40\_2\_1\_50\_5\_10\_50 \textcolor{red}{$\boldsymbol{\upalpha\uppi\mathrm{o}\uplambda\upalpha\upmu\upbeta\upalpha\upnu\upepsilon\upiota\upnu}$} apolambanejn $|$aufzunehmen\\
5.&93.&93.&480.&480.&30.&4&970&300\_70\_400\_200 \textcolor{red}{$\boldsymbol{\uptau\mathrm{o}\upsilon\upsigma}$} to"us $|$/die\\
6.&94.&94.&484.&484.&34.&9&1820&300\_70\_10\_70\_400\_300\_70\_400\_200 \textcolor{red}{$\boldsymbol{\uptau\mathrm{o}\upiota\mathrm{o}\upsilon\uptau\mathrm{o}\upsilon\upsigma}$} tojo"uto"us $|$solche/so Beschaffenen\\
7.&95.&95.&493.&493.&43.&3&61&10\_50\_1 \textcolor{red}{$\boldsymbol{\upiota\upnu\upalpha}$} jna $|$damit\\
8.&96.&96.&496.&496.&46.&8&838&200\_400\_50\_5\_100\_3\_70\_10 \textcolor{red}{$\boldsymbol{\upsigma\upsilon\upnu\upepsilon\uprho\upgamma\mathrm{o}\upiota}$} s"unergoj $|$Mitarbeiter\\
9.&97.&97.&504.&504.&54.&8&918&3\_10\_50\_800\_40\_5\_9\_1 \textcolor{red}{$\boldsymbol{\upgamma\upiota\upnu\upomega\upmu\upepsilon\upvartheta\upalpha}$} gjnOmeTa $|$wir werden\\
10.&98.&98.&512.&512.&62.&2&308&300\_8 \textcolor{red}{$\boldsymbol{\uptau\upeta}$} t"a $|$der/f"ur die\\
11.&99.&99.&514.&514.&64.&7&64&1\_30\_8\_9\_5\_10\_1 \textcolor{red}{$\boldsymbol{\upalpha\uplambda\upeta\upvartheta\upepsilon\upiota\upalpha}$} al"aTeja $|$Wahrheit\\
\end{tabular}\medskip \\
Ende des Verses 1.8\\
Verse: 8, Buchstaben: 70, 520, 520, Totalwerte: 6882, 63817, 63817\\
\\
Wir nun sind schuldig, solche aufzunehmen, auf da"s wir Mitarbeiter der Wahrheit werden.\\
\newpage 
{\bf -- 1.9}\\
\medskip \\
\begin{tabular}{rrrrrrrrp{120mm}}
WV&WK&WB&ABK&ABB&ABV&AnzB&TW&Zahlencode \textcolor{red}{$\boldsymbol{Grundtext}$} Umschrift $|$"Ubersetzung(en)\\
1.&100.&100.&521.&521.&1.&6&810&5\_3\_100\_1\_700\_1 \textcolor{red}{$\boldsymbol{\upepsilon\upgamma\uprho\upalpha\uppsi\upalpha}$} egraPa $|$ich habe geschrieben (etwas)\\
2.&101.&101.&527.&527.&7.&2&308&300\_8 \textcolor{red}{$\boldsymbol{\uptau\upeta}$} t"a $|$der\\
3.&102.&102.&529.&529.&9.&8&294&5\_20\_20\_30\_8\_200\_10\_1 \textcolor{red}{$\boldsymbol{\upepsilon\upkappa\upkappa\uplambda\upeta\upsigma\upiota\upalpha}$} ekkl"asja $|$Gemeinde//Versammlung\\
4.&103.&103.&537.&537.&17.&3&61&1\_30\_30 \textcolor{red}{$\boldsymbol{\upalpha\uplambda\uplambda}$} all $|$aber\\
5.&104.&104.&540.&540.&20.&1&70&70 \textcolor{red}{$\boldsymbol{\mathrm{o}}$} o $|$der\\
6.&105.&105.&541.&541.&21.&12&3145&500\_10\_30\_70\_80\_100\_800\_300\_5\_400\_800\_50 \textcolor{red}{$\boldsymbol{\upvarphi\upiota\uplambda\mathrm{o}\uppi\uprho\upomega\uptau\upepsilon\upsilon\upomega\upnu}$} fjloprOte"uOn $|$der Erste sein m"ochte/der Erste sein Wollende\\
7.&106.&106.&553.&553.&33.&5&1551&1\_400\_300\_800\_50 \textcolor{red}{$\boldsymbol{\upalpha\upsilon\uptau\upomega\upnu}$} a"utOn $|$bei ihnen/unter ihnen\\
8.&107.&107.&558.&558.&38.&9&1197&4\_10\_70\_300\_100\_5\_500\_8\_200 \textcolor{red}{$\boldsymbol{\updelta\upiota\mathrm{o}\uptau\uprho\upepsilon\upvarphi\upeta\upsigma}$} djotref"as $|$Diotrephes///$<$von Zeus ern"ahrt$>$\\
9.&108.&108.&567.&567.&47.&3&490&70\_400\_20 \textcolor{red}{$\boldsymbol{\mathrm{o}\upsilon\upkappa}$} o"uk $|$nicht\\
10.&109.&109.&570.&570.&50.&10&1020&5\_80\_10\_4\_5\_600\_5\_300\_1\_10 \textcolor{red}{$\boldsymbol{\upepsilon\uppi\upiota\updelta\upepsilon\upchi\upepsilon\uptau\upalpha\upiota}$} epjdecetaj $|$nimmt an\\
11.&110.&110.&580.&580.&60.&4&249&8\_40\_1\_200 \textcolor{red}{$\boldsymbol{\upeta\upmu\upalpha\upsigma}$} "amas $|$uns\\
\end{tabular}\medskip \\
Ende des Verses 1.9\\
Verse: 9, Buchstaben: 63, 583, 583, Totalwerte: 9195, 73012, 73012\\
\\
Ich schrieb etwas an die Versammlung, aber Diotrephes, der gern unter ihnen der erste sein will, nimmt uns nicht an.\\
\newpage 
{\bf -- 1.10}\\
\medskip \\
\begin{tabular}{rrrrrrrrp{120mm}}
WV&WK&WB&ABK&ABB&ABV&AnzB&TW&Zahlencode \textcolor{red}{$\boldsymbol{Grundtext}$} Umschrift $|$"Ubersetzung(en)\\
1.&111.&111.&584.&584.&1.&3&15&4\_10\_1 \textcolor{red}{$\boldsymbol{\updelta\upiota\upalpha}$} dja $|$darum/des-//wegen\\
2.&112.&112.&587.&587.&4.&5&1140&300\_70\_400\_300\_70 \textcolor{red}{$\boldsymbol{\uptau\mathrm{o}\upsilon\uptau\mathrm{o}}$} to"uto $|$/wegen//diesem\\
3.&113.&113.&592.&592.&9.&3&56&5\_1\_50 \textcolor{red}{$\boldsymbol{\upepsilon\upalpha\upnu}$} ean $|$wenn\\
4.&114.&114.&595.&595.&12.&4&844&5\_30\_9\_800 \textcolor{red}{$\boldsymbol{\upepsilon\uplambda\upvartheta\upomega}$} elTO $|$ich komme\\
5.&115.&115.&599.&599.&16.&8&1648&400\_80\_70\_40\_50\_8\_200\_800 \textcolor{red}{$\boldsymbol{\upsilon\uppi\mathrm{o}\upmu\upnu\upeta\upsigma\upomega}$} "upomn"asO $|$will ich ihm vorhalten/werde ich erinnern\\
6.&116.&116.&607.&607.&24.&5&1171&1\_400\_300\_70\_400 \textcolor{red}{$\boldsymbol{\upalpha\upsilon\uptau\mathrm{o}\upsilon}$} a"uto"u $|$(an) seine\\
7.&117.&117.&612.&612.&29.&2&301&300\_1 \textcolor{red}{$\boldsymbol{\uptau\upalpha}$} ta $|$(die)\\
8.&118.&118.&614.&614.&31.&4&109&5\_100\_3\_1 \textcolor{red}{$\boldsymbol{\upepsilon\uprho\upgamma\upalpha}$} erga $|$Werke\\
9.&119.&119.&618.&618.&35.&1&1&1 \textcolor{red}{$\boldsymbol{\upalpha}$} a $|$die\\
10.&120.&120.&619.&619.&36.&5&175&80\_70\_10\_5\_10 \textcolor{red}{$\boldsymbol{\uppi\mathrm{o}\upiota\upepsilon\upiota}$} pojej $|$er tut\\
11.&121.&121.&624.&624.&41.&6&383&30\_70\_3\_70\_10\_200 \textcolor{red}{$\boldsymbol{\uplambda\mathrm{o}\upgamma\mathrm{o}\upiota\upsigma}$} logojs $|$(indem) mit Worten\\
12.&122.&122.&630.&630.&47.&8&588&80\_70\_50\_8\_100\_70\_10\_200 \textcolor{red}{$\boldsymbol{\uppi\mathrm{o}\upnu\upeta\uprho\mathrm{o}\upiota\upsigma}$} pon"arojs $|$b"osen\\
13.&123.&123.&638.&638.&55.&7&1881&500\_30\_400\_1\_100\_800\_50 \textcolor{red}{$\boldsymbol{\upvarphi\uplambda\upsilon\upalpha\uprho\upomega\upnu}$} fl"uarOn $|$er verleumdet/schw"atzend-anklagend\\
14.&124.&124.&645.&645.&62.&4&249&8\_40\_1\_200 \textcolor{red}{$\boldsymbol{\upeta\upmu\upalpha\upsigma}$} "amas $|$uns\\
15.&125.&125.&649.&649.&66.&3&31&20\_1\_10 \textcolor{red}{$\boldsymbol{\upkappa\upalpha\upiota}$} kaj $|$und\\
16.&126.&126.&652.&652.&69.&2&48&40\_8 \textcolor{red}{$\boldsymbol{\upmu\upeta}$} m"a $|$nicht\\
17.&127.&127.&654.&654.&71.&10&956&1\_100\_20\_70\_400\_40\_5\_50\_70\_200 \textcolor{red}{$\boldsymbol{\upalpha\uprho\upkappa\mathrm{o}\upsilon\upmu\upepsilon\upnu\mathrm{o}\upsigma}$} arko"umenos $|$genug da-/sich begn"ugend\\
18.&128.&128.&664.&664.&81.&3&95&5\_80\_10 \textcolor{red}{$\boldsymbol{\upepsilon\uppi\upiota}$} epj $|$mit\\
19.&129.&129.&667.&667.&84.&7&1350&300\_70\_400\_300\_70\_10\_200 \textcolor{red}{$\boldsymbol{\uptau\mathrm{o}\upsilon\uptau\mathrm{o}\upiota\upsigma}$} to"utojs $|$/diesem\\
20.&130.&130.&674.&674.&91.&4&775&70\_400\_300\_5 \textcolor{red}{$\boldsymbol{\mathrm{o}\upsilon\uptau\upepsilon}$} o"ute $|$(auf der einen Seite) nicht\\
21.&131.&131.&678.&678.&95.&5&971&1\_400\_300\_70\_200 \textcolor{red}{$\boldsymbol{\upalpha\upsilon\uptau\mathrm{o}\upsigma}$} a"utos $|$er\\
22.&132.&132.&683.&683.&100.&10&1020&5\_80\_10\_4\_5\_600\_5\_300\_1\_10 \textcolor{red}{$\boldsymbol{\upepsilon\uppi\upiota\updelta\upepsilon\upchi\upepsilon\uptau\upalpha\upiota}$} epjdecetaj $|$selbst nimmt auf/nimmt an\\
23.&133.&133.&693.&693.&110.&4&970&300\_70\_400\_200 \textcolor{red}{$\boldsymbol{\uptau\mathrm{o}\upsilon\upsigma}$} to"us $|$die\\
24.&134.&134.&697.&697.&114.&8&1210&1\_4\_5\_30\_500\_70\_400\_200 \textcolor{red}{$\boldsymbol{\upalpha\updelta\upepsilon\uplambda\upvarphi\mathrm{o}\upsilon\upsigma}$} adelfo"us $|$Br"uder\\
25.&135.&135.&705.&705.&122.&3&31&20\_1\_10 \textcolor{red}{$\boldsymbol{\upkappa\upalpha\upiota}$} kaj $|$und auch/auf der anderen Seite\\
\end{tabular}
\newpage
\begin{tabular}{rrrrrrrrp{120mm}}
WV&WK&WB&ABK&ABB&ABV&AnzB&TW&Zahlencode \textcolor{red}{$\boldsymbol{Grundtext}$} Umschrift $|$"Ubersetzung(en)\\
26.&136.&136.&708.&708.&125.&4&970&300\_70\_400\_200 \textcolor{red}{$\boldsymbol{\uptau\mathrm{o}\upsilon\upsigma}$} to"us $|$denen die/die\\
27.&137.&137.&712.&712.&129.&11&1337&2\_70\_400\_30\_70\_40\_5\_50\_70\_400\_200 \textcolor{red}{$\boldsymbol{\upbeta\mathrm{o}\upsilon\uplambda\mathrm{o}\upmu\upepsilon\upnu\mathrm{o}\upsilon\upsigma}$} bo"ulomeno"us $|$es tun wollen/Wollenden\\
28.&138.&138.&723.&723.&140.&6&1265&20\_800\_30\_400\_5\_10 \textcolor{red}{$\boldsymbol{\upkappa\upomega\uplambda\upsilon\upepsilon\upiota}$} kOl"uej $|$verwehrt er es/hindert er\\
29.&139.&139.&729.&729.&146.&3&31&20\_1\_10 \textcolor{red}{$\boldsymbol{\upkappa\upalpha\upiota}$} kaj $|$und\\
30.&140.&140.&732.&732.&149.&2&25&5\_20 \textcolor{red}{$\boldsymbol{\upepsilon\upkappa}$} ek $|$aus\\
31.&141.&141.&734.&734.&151.&3&508&300\_8\_200 \textcolor{red}{$\boldsymbol{\uptau\upeta\upsigma}$} t"as $|$der\\
32.&142.&142.&737.&737.&154.&9&494&5\_20\_20\_30\_8\_200\_10\_1\_200 \textcolor{red}{$\boldsymbol{\upepsilon\upkappa\upkappa\uplambda\upeta\upsigma\upiota\upalpha\upsigma}$} ekkl"asjas $|$Gemeinde\\
33.&143.&143.&746.&746.&163.&8&103&5\_20\_2\_1\_30\_30\_5\_10 \textcolor{red}{$\boldsymbol{\upepsilon\upkappa\upbeta\upalpha\uplambda\uplambda\upepsilon\upiota}$} ekballej $|$st"o"st (er) (sie) (hin)aus\\
\end{tabular}\medskip \\
Ende des Verses 1.10\\
Verse: 10, Buchstaben: 170, 753, 753, Totalwerte: 20751, 93763, 93763\\
\\
Deshalb, wenn ich komme, will ich seiner Werke gedenken, die er tut, indem er mit b"osen Worten wider uns schwatzt; und sich hiermit nicht begn"ugend, nimmt er selbst die Br"uder nicht an und wehrt auch denen, die es wollen, und st"o"st sie aus der Versammlung.\\
\newpage 
{\bf -- 1.11}\\
\medskip \\
\begin{tabular}{rrrrrrrrp{120mm}}
WV&WK&WB&ABK&ABB&ABV&AnzB&TW&Zahlencode \textcolor{red}{$\boldsymbol{Grundtext}$} Umschrift $|$"Ubersetzung(en)\\
1.&144.&144.&754.&754.&1.&7&398&1\_3\_1\_80\_8\_300\_5 \textcolor{red}{$\boldsymbol{\upalpha\upgamma\upalpha\uppi\upeta\uptau\upepsilon}$} agap"ate $|$mein Lieber/Geliebter\\
2.&145.&145.&761.&761.&8.&2&48&40\_8 \textcolor{red}{$\boldsymbol{\upmu\upeta}$} m"a $|$nicht\\
3.&146.&146.&763.&763.&10.&5&560&40\_10\_40\_70\_400 \textcolor{red}{$\boldsymbol{\upmu\upiota\upmu\mathrm{o}\upsilon}$} mjmo"u $|$ahme nach\\
4.&147.&147.&768.&768.&15.&2&370&300\_70 \textcolor{red}{$\boldsymbol{\uptau\mathrm{o}}$} to $|$das\\
5.&148.&148.&770.&770.&17.&5&161&20\_1\_20\_70\_50 \textcolor{red}{$\boldsymbol{\upkappa\upalpha\upkappa\mathrm{o}\upnu}$} kakon $|$B"ose\\
6.&149.&149.&775.&775.&22.&4&62&1\_30\_30\_1 \textcolor{red}{$\boldsymbol{\upalpha\uplambda\uplambda\upalpha}$} alla $|$sondern\\
7.&150.&150.&779.&779.&26.&2&370&300\_70 \textcolor{red}{$\boldsymbol{\uptau\mathrm{o}}$} to $|$das\\
8.&151.&151.&781.&781.&28.&6&134&1\_3\_1\_9\_70\_50 \textcolor{red}{$\boldsymbol{\upalpha\upgamma\upalpha\upvartheta\mathrm{o}\upnu}$} agaTon $|$Gute\\
9.&152.&152.&787.&787.&34.&1&70&70 \textcolor{red}{$\boldsymbol{\mathrm{o}}$} o $|$wer/der\\
10.&153.&153.&788.&788.&35.&10&1094&1\_3\_1\_9\_70\_80\_70\_10\_800\_50 \textcolor{red}{$\boldsymbol{\upalpha\upgamma\upalpha\upvartheta\mathrm{o}\uppi\mathrm{o}\upiota\upomega\upnu}$} agaTopojOn $|$Gutes tut/Gutes Tuende\\
11.&154.&154.&798.&798.&45.&2&25&5\_20 \textcolor{red}{$\boldsymbol{\upepsilon\upkappa}$} ek $|$aus\\
12.&155.&155.&800.&800.&47.&3&770&300\_70\_400 \textcolor{red}{$\boldsymbol{\uptau\mathrm{o}\upsilon}$} to"u $|$der/(dem)\\
13.&156.&156.&803.&803.&50.&4&484&9\_5\_70\_400 \textcolor{red}{$\boldsymbol{\upvartheta\upepsilon\mathrm{o}\upsilon}$} Teo"u $|$Gott\\
14.&157.&157.&807.&807.&54.&5&565&5\_200\_300\_10\_50 \textcolor{red}{$\boldsymbol{\upepsilon\upsigma\uptau\upiota\upnu}$} estjn $|$ist\\
15.&158.&158.&812.&812.&59.&1&70&70 \textcolor{red}{$\boldsymbol{\mathrm{o}}$} o $|$wer/(der)\\
16.&159.&159.&813.&813.&60.&2&9&4\_5 \textcolor{red}{$\boldsymbol{\updelta\upepsilon}$} de $|$aber//\\
17.&160.&160.&815.&815.&62.&9&1121&20\_1\_20\_70\_80\_70\_10\_800\_50 \textcolor{red}{$\boldsymbol{\upkappa\upalpha\upkappa\mathrm{o}\uppi\mathrm{o}\upiota\upomega\upnu}$} kakopojOn $|$B"oses tut/B"oses Tuende\\
18.&161.&161.&824.&824.&71.&3&1070&70\_400\_600 \textcolor{red}{$\boldsymbol{\mathrm{o}\upsilon\upchi}$} o"uc $|$nicht\\
19.&162.&162.&827.&827.&74.&7&981&5\_800\_100\_1\_20\_5\_50 \textcolor{red}{$\boldsymbol{\upepsilon\upomega\uprho\upalpha\upkappa\upepsilon\upnu}$} eOraken $|$hat gesehen\\
20.&163.&163.&834.&834.&81.&3&420&300\_70\_50 \textcolor{red}{$\boldsymbol{\uptau\mathrm{o}\upnu}$} ton $|$(den)\\
21.&164.&164.&837.&837.&84.&4&134&9\_5\_70\_50 \textcolor{red}{$\boldsymbol{\upvartheta\upepsilon\mathrm{o}\upnu}$} Teon $|$Gott\\
\end{tabular}\medskip \\
Ende des Verses 1.11\\
Verse: 11, Buchstaben: 87, 840, 840, Totalwerte: 8916, 102679, 102679\\
\\
Geliebter, ahme nicht das B"ose nach, sondern das Gute. Wer Gutes tut, ist aus Gott; wer B"oses tut, hat Gott nicht gesehen.\\
\newpage 
{\bf -- 1.12}\\
\medskip \\
\begin{tabular}{rrrrrrrrp{120mm}}
WV&WK&WB&ABK&ABB&ABV&AnzB&TW&Zahlencode \textcolor{red}{$\boldsymbol{Grundtext}$} Umschrift $|$"Ubersetzung(en)\\
1.&165.&165.&841.&841.&1.&8&1270&4\_8\_40\_8\_300\_100\_10\_800 \textcolor{red}{$\boldsymbol{\updelta\upeta\upmu\upeta\uptau\uprho\upiota\upomega}$} d"am"atrjO $|$(dem) Demetrius///$<$der G"ottin Demeter (Diana) geweiht$>$\\
2.&166.&166.&849.&849.&9.&12&1305&40\_5\_40\_1\_100\_300\_400\_100\_8\_300\_1\_10 \textcolor{red}{$\boldsymbol{\upmu\upepsilon\upmu\upalpha\uprho\uptau\upsilon\uprho\upeta\uptau\upalpha\upiota}$} memart"ur"ataj $|$wird ein gutes Zeugnis ausgestellt/ist ein gutes Zeugnis ausgestellt worden\\
3.&167.&167.&861.&861.&21.&3&550&400\_80\_70 \textcolor{red}{$\boldsymbol{\upsilon\uppi\mathrm{o}}$} "upo $|$von\\
4.&168.&168.&864.&864.&24.&6&1281&80\_1\_50\_300\_800\_50 \textcolor{red}{$\boldsymbol{\uppi\upalpha\upnu\uptau\upomega\upnu}$} pantOn $|$allen\\
5.&169.&169.&870.&870.&30.&3&31&20\_1\_10 \textcolor{red}{$\boldsymbol{\upkappa\upalpha\upiota}$} kaj $|$und\\
6.&170.&170.&873.&873.&33.&2&480&400\_80 \textcolor{red}{$\boldsymbol{\upsilon\uppi}$} "up $|$von\\
7.&171.&171.&875.&875.&35.&5&909&1\_400\_300\_8\_200 \textcolor{red}{$\boldsymbol{\upalpha\upsilon\uptau\upeta\upsigma}$} a"ut"as $|$selbst\\
8.&172.&172.&880.&880.&40.&3&508&300\_8\_200 \textcolor{red}{$\boldsymbol{\uptau\upeta\upsigma}$} t"as $|$der\\
9.&173.&173.&883.&883.&43.&8&264&1\_30\_8\_9\_5\_10\_1\_200 \textcolor{red}{$\boldsymbol{\upalpha\uplambda\upeta\upvartheta\upepsilon\upiota\upalpha\upsigma}$} al"aTejas $|$Wahrheit\\
10.&174.&174.&891.&891.&51.&3&31&20\_1\_10 \textcolor{red}{$\boldsymbol{\upkappa\upalpha\upiota}$} kaj $|$/und\\
11.&175.&175.&894.&894.&54.&5&263&8\_40\_5\_10\_200 \textcolor{red}{$\boldsymbol{\upeta\upmu\upepsilon\upiota\upsigma}$} "amejs $|$wir\\
12.&176.&176.&899.&899.&59.&2&9&4\_5 \textcolor{red}{$\boldsymbol{\updelta\upepsilon}$} de $|$auch\\
13.&177.&177.&901.&901.&61.&11&1506&40\_1\_100\_300\_400\_100\_70\_400\_40\_5\_50 \textcolor{red}{$\boldsymbol{\upmu\upalpha\uprho\uptau\upsilon\uprho\mathrm{o}\upsilon\upmu\upepsilon\upnu}$} mart"uro"umen $|$geben Zeugnis daf"ur/legen Zeugnis ab (f"ur ihn)\\
14.&178.&178.&912.&912.&72.&3&31&20\_1\_10 \textcolor{red}{$\boldsymbol{\upkappa\upalpha\upiota}$} kaj $|$und\\
15.&179.&179.&915.&915.&75.&6&390&70\_10\_4\_1\_300\_5 \textcolor{red}{$\boldsymbol{\mathrm{o}\upiota\updelta\upalpha\uptau\upepsilon}$} ojdate $|$ihr wisst/du wei"st\\
16.&180.&180.&921.&921.&81.&3&380&70\_300\_10 \textcolor{red}{$\boldsymbol{\mathrm{o}\uptau\upiota}$} otj $|$dass\\
17.&181.&181.&924.&924.&84.&1&8&8 \textcolor{red}{$\boldsymbol{\upeta}$} "a $|$(das)\\
18.&182.&182.&925.&925.&85.&8&952&40\_1\_100\_300\_400\_100\_10\_1 \textcolor{red}{$\boldsymbol{\upmu\upalpha\uprho\uptau\upsilon\uprho\upiota\upalpha}$} mart"urja $|$Zeugnis\\
19.&183.&183.&933.&933.&93.&4&898&8\_40\_800\_50 \textcolor{red}{$\boldsymbol{\upeta\upmu\upomega\upnu}$} "amOn $|$unser\\
20.&184.&184.&937.&937.&97.&6&256&1\_30\_8\_9\_8\_200 \textcolor{red}{$\boldsymbol{\upalpha\uplambda\upeta\upvartheta\upeta\upsigma}$} al"aT"as $|$wahr\\
21.&185.&185.&943.&943.&103.&5&565&5\_200\_300\_10\_50 \textcolor{red}{$\boldsymbol{\upepsilon\upsigma\uptau\upiota\upnu}$} estjn $|$ist\\
\end{tabular}\medskip \\
Ende des Verses 1.12\\
Verse: 12, Buchstaben: 107, 947, 947, Totalwerte: 11887, 114566, 114566\\
\\
Dem Demetrius wird Zeugnis gegeben von allen und von der Wahrheit selbst; aber auch wir geben Zeugnis, und du wei"st, da"s unser Zeugnis wahr ist.\\
\newpage 
{\bf -- 1.13}\\
\medskip \\
\begin{tabular}{rrrrrrrrp{120mm}}
WV&WK&WB&ABK&ABB&ABV&AnzB&TW&Zahlencode \textcolor{red}{$\boldsymbol{Grundtext}$} Umschrift $|$"Ubersetzung(en)\\
1.&186.&186.&948.&948.&1.&5&211&80\_70\_30\_30\_1 \textcolor{red}{$\boldsymbol{\uppi\mathrm{o}\uplambda\uplambda\upalpha}$} polla $|$(noch) vieles\\
2.&187.&187.&953.&953.&6.&5&735&5\_10\_600\_70\_50 \textcolor{red}{$\boldsymbol{\upepsilon\upiota\upchi\mathrm{o}\upnu}$} ejcon $|$h"atte ich\\
3.&188.&188.&958.&958.&11.&7&669&3\_100\_1\_500\_5\_10\_50 \textcolor{red}{$\boldsymbol{\upgamma\uprho\upalpha\upvarphi\upepsilon\upiota\upnu}$} grafejn $|$zu schreiben\\
4.&189.&189.&965.&965.&18.&3&61&1\_30\_30 \textcolor{red}{$\boldsymbol{\upalpha\uplambda\uplambda}$} all $|$aber\\
5.&190.&190.&968.&968.&21.&2&470&70\_400 \textcolor{red}{$\boldsymbol{\mathrm{o}\upsilon}$} o"u $|$nicht\\
6.&191.&191.&970.&970.&23.&4&844&9\_5\_30\_800 \textcolor{red}{$\boldsymbol{\upvartheta\upepsilon\uplambda\upomega}$} TelO $|$will ich\\
7.&192.&192.&974.&974.&27.&3&15&4\_10\_1 \textcolor{red}{$\boldsymbol{\updelta\upiota\upalpha}$} dja $|$mit\\
8.&193.&193.&977.&977.&30.&7&396&40\_5\_30\_1\_50\_70\_200 \textcolor{red}{$\boldsymbol{\upmu\upepsilon\uplambda\upalpha\upnu\mathrm{o}\upsigma}$} melanos $|$Tinte\\
9.&194.&194.&984.&984.&37.&3&31&20\_1\_10 \textcolor{red}{$\boldsymbol{\upkappa\upalpha\upiota}$} kaj $|$und\\
10.&195.&195.&987.&987.&40.&7&562&20\_1\_30\_1\_40\_70\_400 \textcolor{red}{$\boldsymbol{\upkappa\upalpha\uplambda\upalpha\upmu\mathrm{o}\upsilon}$} kalamo"u $|$Feder/Schreibrohr\\
11.&196.&196.&994.&994.&47.&3&280&200\_70\_10 \textcolor{red}{$\boldsymbol{\upsigma\mathrm{o}\upiota}$} soj $|$dir\\
12.&197.&197.&997.&997.&50.&6&815&3\_100\_1\_700\_1\_10 \textcolor{red}{$\boldsymbol{\upgamma\uprho\upalpha\uppsi\upalpha\upiota}$} graPaj $|$schreiben\\
\end{tabular}\medskip \\
Ende des Verses 1.13\\
Verse: 13, Buchstaben: 55, 1002, 1002, Totalwerte: 5089, 119655, 119655\\
\\
Ich h"atte dir vieles zu schreiben, aber ich will dir nicht mit Tinte und Feder schreiben,\\
\newpage 
{\bf -- 1.14}\\
\medskip \\
\begin{tabular}{rrrrrrrrp{120mm}}
WV&WK&WB&ABK&ABB&ABV&AnzB&TW&Zahlencode \textcolor{red}{$\boldsymbol{Grundtext}$} Umschrift $|$"Ubersetzung(en)\\
1.&198.&198.&1003.&1003.&1.&6&932&5\_30\_80\_10\_7\_800 \textcolor{red}{$\boldsymbol{\upepsilon\uplambda\uppi\upiota\upzeta\upomega}$} elpjzO $|$ich hoffe\\
2.&199.&199.&1009.&1009.&7.&2&9&4\_5 \textcolor{red}{$\boldsymbol{\updelta\upepsilon}$} de $|$aber\\
3.&200.&200.&1011.&1011.&9.&6&1419&5\_400\_9\_5\_800\_200 \textcolor{red}{$\boldsymbol{\upepsilon\upsilon\upvartheta\upepsilon\upomega\upsigma}$} e"uTeOs $|$(als)bald\\
4.&201.&201.&1017.&1017.&15.&5&79&10\_4\_5\_10\_50 \textcolor{red}{$\boldsymbol{\upiota\updelta\upepsilon\upiota\upnu}$} jdejn $|$zu sehen\\
5.&202.&202.&1022.&1022.&20.&2&205&200\_5 \textcolor{red}{$\boldsymbol{\upsigma\upepsilon}$} se $|$dich\\
6.&203.&203.&1024.&1024.&22.&3&31&20\_1\_10 \textcolor{red}{$\boldsymbol{\upkappa\upalpha\upiota}$} kaj $|$und (dann)\\
7.&204.&204.&1027.&1027.&25.&5&611&200\_300\_70\_40\_1 \textcolor{red}{$\boldsymbol{\upsigma\uptau\mathrm{o}\upmu\upalpha}$} stoma $|$m"undlich/(von) Mund\\
8.&205.&205.&1032.&1032.&30.&4&450&80\_100\_70\_200 \textcolor{red}{$\boldsymbol{\uppi\uprho\mathrm{o}\upsigma}$} pros $|$mit-/zu\\
9.&206.&206.&1036.&1036.&34.&5&611&200\_300\_70\_40\_1 \textcolor{red}{$\boldsymbol{\upsigma\uptau\mathrm{o}\upmu\upalpha}$} stoma $|$einander/Mund\\
10.&207.&207.&1041.&1041.&39.&9&434&30\_1\_30\_8\_200\_70\_40\_5\_50 \textcolor{red}{$\boldsymbol{\uplambda\upalpha\uplambda\upeta\upsigma\mathrm{o}\upmu\upepsilon\upnu}$} lal"asomen $|$wollen wir reden/werden wir reden\\
11.&208.&208.&1050.&1050.&48.&6&181&5\_10\_100\_8\_50\_8 \textcolor{red}{$\boldsymbol{\upepsilon\upiota\uprho\upeta\upnu\upeta}$} ejr"an"a $|$Friede (sei)///--- 1.15\\
12.&209.&209.&1056.&1056.&54.&3&280&200\_70\_10 \textcolor{red}{$\boldsymbol{\upsigma\mathrm{o}\upiota}$} soj $|$(mit) dir\\
13.&210.&210.&1059.&1059.&57.&10&720&1\_200\_80\_1\_7\_70\_50\_300\_1\_10 \textcolor{red}{$\boldsymbol{\upalpha\upsigma\uppi\upalpha\upzeta\mathrm{o}\upnu\uptau\upalpha\upiota}$} aspazontaj $|$(es) (lassen) gr"u"sen\\
14.&211.&211.&1069.&1069.&67.&2&205&200\_5 \textcolor{red}{$\boldsymbol{\upsigma\upepsilon}$} se $|$dich\\
15.&212.&212.&1071.&1071.&69.&2&80&70\_10 \textcolor{red}{$\boldsymbol{\mathrm{o}\upiota}$} oj $|$die\\
16.&213.&213.&1073.&1073.&71.&5&620&500\_10\_30\_70\_10 \textcolor{red}{$\boldsymbol{\upvarphi\upiota\uplambda\mathrm{o}\upiota}$} fjloj $|$Freunde\\
17.&214.&214.&1078.&1078.&76.&7&759&1\_200\_80\_1\_7\_70\_400 \textcolor{red}{$\boldsymbol{\upalpha\upsigma\uppi\upalpha\upzeta\mathrm{o}\upsilon}$} aspazo"u $|$gr"u"se\\
18.&215.&215.&1085.&1085.&83.&4&970&300\_70\_400\_200 \textcolor{red}{$\boldsymbol{\uptau\mathrm{o}\upsilon\upsigma}$} to"us $|$die\\
19.&216.&216.&1089.&1089.&87.&6&1210&500\_10\_30\_70\_400\_200 \textcolor{red}{$\boldsymbol{\upvarphi\upiota\uplambda\mathrm{o}\upsilon\upsigma}$} fjlo"us $|$Freunde\\
20.&217.&217.&1095.&1095.&93.&3&321&20\_1\_300 \textcolor{red}{$\boldsymbol{\upkappa\upalpha\uptau}$} kat $|$mit\\
21.&218.&218.&1098.&1098.&96.&5&231&70\_50\_70\_40\_1 \textcolor{red}{$\boldsymbol{\mathrm{o}\upnu\mathrm{o}\upmu\upalpha}$} onoma $|$(einzeln) Namen\\
\end{tabular}\medskip \\
Ende des Verses 1.14\\
Verse: 14, Buchstaben: 100, 1102, 1102, Totalwerte: 10358, 130013, 130013\\
\\
sondern ich hoffe, dich bald zu sehen, und wir wollen m"undlich miteinander reden. (1:15) Friede dir! Es gr"u"sen dich die Freunde. Gr"u"se die Freunde mit Namen.\\
\\
{\bf Ende des Kapitels 1}\\

\bigskip				%%gro�er Abstand

\newpage
\hphantom{x}
\bigskip\bigskip\bigskip\bigskip\bigskip\bigskip
\begin{center}{ \huge {\bf Ende des Buches}}\end{center}


\end{document}



